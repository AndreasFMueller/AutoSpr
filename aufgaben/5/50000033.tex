Betrachten Sie die Turing-Maschine mit dem folgenden Zustandsdiagramm.
Alle fehlenden Übergänge sind als Übergänge zu einem Zustand
$q_{\text{reject}}$ zu interpretieren.
\begin{center}
\begin{tikzpicture}[>=latex,thick]

\pgfmathparse{1.5*sqrt(3)}
\xdef\ds{\pgfmathresult}

\coordinate (q0) at (-3,1.5);
\coordinate (q1) at (0,1.5);
\coordinate (q2) at (0,-1.5);
\coordinate (q3) at (3,-1.5);
\coordinate (reject) at (-3,-4.5);
\coordinate (q4) at (3,1.5);
\coordinate (q5) at (6,1.5);
\coordinate (accept) at (6,-1.5);

\node at (q0) {$q_0$};
\node at (q1) {$q_1$};
\node at (q2) {$q_2$};
\node at (q3) {$q_3$};
\node at (q4) {$q_4$};
\node at (q5) {$q_5$};
\node at (accept) {$q_{\text{accept}}$};

\draw (q0) circle[radius=0.4];
\draw (q1) circle[radius=0.4];
\draw (q2) circle[radius=0.4];
\draw (q3) circle[radius=0.4];
\draw (q4) circle[radius=0.4];
\draw (q5) circle[radius=0.4];
%\draw (reject) ellipse (0.8 and 0.4);
%\draw (reject) ellipse (0.7 and 0.3);
\draw (accept) ellipse (0.8 and 0.4);
\draw (accept) ellipse (0.7 and 0.3);

\draw[->,shorten >= 0.4cm] (-5,1.5) -- (q0);

\draw[->,shorten >= 0.4cm,shorten <= 0.4cm] (q0) -- (q1);
\draw[->,shorten >= 0.4cm,shorten <= 0.4cm] (q1) -- (q4);
\draw[->,shorten >= 0.4cm,shorten <= 0.4cm] (q4) -- (q5);
\draw[->,shorten >= 0.4cm,shorten <= 0.4cm] (q4) -- (q5);
\draw[->,shorten >= 0.4cm,shorten <= 0.4cm] (q2) -- (q3);
\draw[->,shorten >= 0.4cm,shorten <= 0.4cm] (q4) -- (accept);
\draw[->,shorten >= 0.4cm,shorten <= 0.4cm] (q5) -- (accept);


\draw[->,shorten >= 0.4cm,shorten <= 0.4cm] (q1) to[out=-60,in=60] (q2);
\draw[->,shorten >= 0.4cm,shorten <= 0.4cm] (q2) to[out=120,in=-120] (q1);

\draw[->,shorten >= 0.4cm,shorten <= 0.4cm]
	(q0) to[out=60,in=120,distance=2cm] (q0);
\draw[->,shorten >= 0.4cm,shorten <= 0.4cm]
	(q1) to[out=60,in=120,distance=2cm] (q1);
\draw[->,shorten >= 0.4cm,shorten <= 0.4cm]
	(q4) to[out=60,in=120,distance=2cm] (q4);
\draw[->,shorten >= 0.4cm,shorten <= 0.4cm]
	(q5) to[out=60,in=120,distance=2cm] (q5);
\draw[->,shorten >= 0.4cm,shorten <= 0.4cm]
	(q2) to[out=-60,in=-120,distance=2cm] (q2);
\draw[->,shorten >= 0.4cm,shorten <= 0.4cm]
	(q3) to[out=-60,in=-120,distance=2cm] (q3);
\draw[->,shorten >= 0.5cm,shorten <= 0.4cm] (q3) -- (q4);

\node at ($0.5*(q0)+0.5*(q1)$) [above] {$\texttt{1}\to\texttt{x},L$};
\node at ($0.5*(q1)+0.5*(q4)$) [above] {$\blank\to\blank,R$};
\node at ($0.5*(q4)+0.5*(q5)$) [above] {$\texttt{1}\to\texttt{1},R$};
\node at ($0.5*(q2)+0.5*(q3)$) [above] {$\blank\to\blank,L$};

\node at ($(accept)+(0,1.0)$) [above] {$\blank\to\blank,R$};

\node at ($(q0)+(0,1.6)$) {$\texttt{0}\to\texttt{0},R$};
\node at ($(q1)+(0,1.6)$) {$\texttt{x}\to\texttt{x},L$};
\node at ($(q4)+(0,1.6)$) {$\texttt{x}\to\texttt{x},R$};
\node at ($(q5)+(0,1.6)$) {$\texttt{1}\to\texttt{1},R$};
\node at ($(q2)+(0,-1.6)$) {$\texttt{x}\to\texttt{x},R$};
\node at ($(q3)+(0,-1.6)$) {$\texttt{x}\to\texttt{x},L$};

\node at ($0.5*(q1)+0.5*(q2)+(-0.5,0)$) [left] {$\texttt{1}\to\texttt{x},L$};
\node at ($0.5*(q1)+0.5*(q2)+(0.5,0)$) [right] {$\texttt{0}\to\texttt{x},R$};

\node at ($0.5*(q3)+0.5*(q4)$) [below right] {$\blank\to\blank,R$};

\end{tikzpicture}
\end{center}
\begin{teilaufgaben}
\item Wie wird das Wort \texttt{0011} verarbeitet?
\item Welche der folgenden Wörter werden akzeptiert:
$\texttt{01}$,
$\texttt{011}$ und
$\texttt{001}$?
\end{teilaufgaben}

\thema{Turing-Maschine}
\thema{Zustandsdiagramm}

\bgroup
\definecolor{darkred}{rgb}{0.8,0,0}
\def\rot#1{{\color{darkred}\texttt{#1}}}
\def\schwarz#1{\texttt{#1}}
\def\rotb{{\color{darkred}\blank}}


\begin{loesung}
\begin{teilaufgaben}
\item Die folgende Tabelle zeigt die Verarbeitung des Wortes \texttt{0011}
\begin{center}
\def\b{\phantom{\texttt{0}}}
\def\r#1{\bgroup\color{red}\texttt{#1}\egroup}
\def\s#1{\texttt{#1}}
\begin{tabular}{>{$}c<{$}|cccccc|>{$}l<{$}}
\text{Zustand}&\multicolumn{6}{l|}{Band}&\text{Berechnungsgeschichte}\\
\hline
q_0& \blank & \rot{0} & \schwarz{0} & \schwarz{1} & \schwarz{1} & \blank & \blank q_0\r{0}\s{011}\blank\blank\\
q_0& \blank & \schwarz{0} & \rot{0} & \schwarz{1} & \schwarz{1} & \blank & \blank \s{0}q_0\r{0}\s{11}\blank\blank\\
q_0& \blank & \schwarz{0} & \schwarz{0} & \rot{1} & \schwarz{1} & \blank & \blank \s{00}q_0\r{1}\s{1}\blank\blank\\
q_1& \blank & \schwarz{0} & \rot{0} & \schwarz{x} & \schwarz{1} & \blank & \blank \s{0}q_1\r{0}\s{x1}\blank\blank\\
q_2& \blank & \schwarz{0} & \schwarz{x} & \rot{x} & \schwarz{1} & \blank & \blank \s{0x}q_2\r{x}\s{1}\blank\blank\\
q_2& \blank & \schwarz{0} & \schwarz{x} & \schwarz{x} & \rot{1} & \blank & \blank \s{0xx}q_2\r{1}\blank\blank\\
q_1& \blank & \schwarz{0} & \schwarz{x} & \rot{x} & \schwarz{x} & \blank & \blank \s{0x}q_1\r{x}\s{x}\blank\blank\\
q_1& \blank & \schwarz{0} & \rot{x} & \schwarz{x} & \schwarz{x} & \blank & \blank \s{0}q_1\r{x}\s{xx}\blank\blank\\
q_1& \blank & \rot{0} & \schwarz{x} & \schwarz{x} & \schwarz{x} & \blank & \blank q_1\r{0}\s{xxx}\blank\blank\\
q_2& \blank & \schwarz{x} & \rot{x} & \schwarz{x} & \schwarz{x} & \blank & \blank \s{x}q_2\r{x}\s{xx}\blank\blank\\
q_2& \blank & \schwarz{x} & \schwarz{x} & \rot{x} & \schwarz{x} & \blank & \blank \s{xx}q_2\r{x}\s{x}\blank\blank\\
q_2& \blank & \schwarz{x} & \schwarz{x} & \schwarz{x} & \rot{x} & \blank & \blank \s{xxx}q_2\r{x}\blank\blank\\
q_2& \blank & \schwarz{x} & \schwarz{x} & \schwarz{x} & \schwarz{x} & \rotb & \blank \s{xxxx}q_2\r{\blank}\blank\\
q_3& \blank & \schwarz{x} & \schwarz{x} & \schwarz{x} & \rot{x} & \blank & \blank \s{xxx}q_3\r{x}\blank\blank\\
q_3& \blank & \schwarz{x} & \schwarz{x} & \rot{x} & \schwarz{x} & \blank & \blank \s{xx}q_3\r{x}\s{x}\blank\blank\\
q_3& \blank & \schwarz{x} & \rot{x} & \schwarz{x} & \schwarz{x} & \blank & \blank \s{x}q_3\r{x}\s{xx}\blank\blank\\
q_3& \blank & \rot{x} & \schwarz{x} & \schwarz{x} & \schwarz{x} & \blank & \blank q_3\r{x}\s{xxx}\blank\blank\\
q_3& \rotb & \schwarz{x} & \schwarz{x} & \schwarz{x} & \schwarz{x} & \blank & q_3\r{\blank}\s{xxxx}\blank\blank\\
q_4& \blank & \rot{x} & \schwarz{x} & \schwarz{x} & \schwarz{x} & \blank & \blank q_4\r{x}\s{xxx}\blank\blank\\
q_4& \blank & \schwarz{x} & \rot{x} & \schwarz{x} & \schwarz{x} & \blank & \blank \s{x}q_4\r{x}\s{xx}\blank\blank\\
q_4& \blank & \schwarz{x} & \schwarz{x} & \rot{x} & \schwarz{x} & \blank & \blank \s{xx}q_4\r{x}\s{x}\blank\blank\\
q_4& \blank & \schwarz{x} & \schwarz{x} & \schwarz{x} & \rot{x} & \blank & \blank \s{xxx}q_4\r{x}\blank\blank\\
q_4& \blank & \schwarz{x} & \schwarz{x} & \schwarz{x} & \schwarz{x} & \rotb & \blank \s{xxxx}q_4\r{\blank}\blank\\
q_{\text{accept}}&
 \blank & \schwarz{x} & \schwarz{x} & \schwarz{x} & \schwarz{x} & \blank & \blank \s{xxxx}\blank q_a\r{\blank}\\
\hline
\end{tabular}
\end{center}
\item
Die folgenden Berechnungsgeschichten zeigen, dass \texttt{01}
und \texttt{011} akzeptiert werden, dass aber \texttt{001}
verworfen wird:
\begin{center}
\def\b{\phantom{\texttt{0}}}
\def\r#1{\bgroup\color{red}\texttt{#1}\egroup}
\def\s#1{\texttt{#1}}
\begin{tabular}{>{$}c<{$}|cccc|>{$}l<{$}}
\text{Zustand}&\multicolumn{4}{l|}{Band}&\text{Berechnungsgeschichte}\\
\hline
q_0& \blank & \rot{0} & \schwarz{1} & \blank & \blank q_0\r{0}\s{1}\blank\blank\\
q_0& \blank & \schwarz{0} & \rot{1} & \blank & \blank \s{0}q_0\r{1}\blank\blank\\
q_1& \blank & \rot{0} & \schwarz{x} & \blank & \blank q_1\r{0}\s{x}\blank\blank\\
q_1& \blank & \schwarz{x} & \rot{x} & \blank & \blank \s{x}q_1\r{x}\blank\blank\\
q_2& \blank & \schwarz{x} & \schwarz{x} & \rotb & \blank \s{x}{x}q_2\r{\blank}\blank\\
q_3& \blank & \schwarz{x} & \rot{x} & \blank & \blank \s{x}q_3\r{x}\blank\blank\\
q_3& \blank & \rot{x} & \schwarz{x} & \blank & \blank q_3\r{x}\s{x}\blank\blank\\
q_3& \rotb & \schwarz{x} & \schwarz{x} & \blank & q_3\r{\blank}\s{xx}\blank\blank\\
q_4& \blank & \rot{x} & \schwarz{x} & \blank & \blank q_4\r{x}\s{x}\blank\blank\\
q_4& \blank & \schwarz{x} & \rot{x} & \blank & \blank\s{x}q_4\r{x}\blank\blank\\
q_4& \blank & \schwarz{x} & \schwarz{x} & \rotb & \blank\s{xx}q_4\r{\blank}\blank\\
q_{\text{accept}}& \blank & \schwarz{x} & \schwarz{x} & \blank & \blank\s{xx}\blank q_a\r{\blank}\\
\hline
\end{tabular}
\\[10pt]
\begin{tabular}{>{$}c<{$}|ccccc|>{$}l<{$}}
\text{Zustand}&\multicolumn{5}{l|}{Band}&\text{Berechnungsgeschichte}\\
\hline
q_0& \blank & \rot{0} & \schwarz{1} & \schwarz{1} & \blank & \blank\blank q_0\r{0}\s{01}\blank\blank\\
q_0& \blank & \schwarz{0} & \rot{1} & \schwarz{1} & \blank & \blank\blank \s{0}q_0\r{1}\s{1}\blank\blank\\
q_1& \blank & \rot{0} & \schwarz{x} & \schwarz{1} & \blank & \blank\blank  q_1\r{0}\s{x1}\blank\blank \\
q_2& \blank & \schwarz{x} & \rot{x} & \schwarz{1} & \blank & \blank\blank \s{x}q_2\r{x}\s{1}\blank\blank\\
q_2& \blank & \schwarz{x} & \schwarz{x} & \rot{1} & \blank & \blank\blank \s{xx}q_2\r{1}\blank\blank\\
q_1& \blank & \schwarz{x} & \rot{x} & \schwarz{x} & \blank & \blank\blank \s{x}q_1\r{x}\s{x}\blank\blank\\
q_1& \blank & \rot{x} & \schwarz{x} & \schwarz{x} & \blank & \blank\blank q_1\r{x}\s{xx}\blank\blank\\
q_1& \rotb & \schwarz{x} & \schwarz{x} & \schwarz{x} & \blank & \blank q_1\r{\blank}\s{xxx}\blank\blank\\
q_4& \blank & \rot{x} & \schwarz{x} & \schwarz{x} & \blank & \blank\blank q_4\r{x}\s{xx}\blank\blank\\
q_4& \blank & \schwarz{x} & \rot{x} & \schwarz{x} & \blank & \blank\blank \s{x}q_4\r{x}\s{x}\blank\blank\\
q_4& \blank & \schwarz{x} & \schwarz{x} & \rot{x} & \blank & \blank\blank \s{xx}q_4\r{x}\blank\blank\\
q_4& \blank & \schwarz{x} & \schwarz{x} & \schwarz{x} & \rotb & \blank\blank\s{xxx}q_4\r{\blank}\blank\\
q_{\text{accept}}& \blank & \schwarz{x} & \schwarz{x} & \schwarz{x} & \blank & \blank\blank\s{xxx}\blank q_a\r{\blank}\\
\hline
\end{tabular}
\\[10pt]
\begin{tabular}{>{$}c<{$}|ccccc|>{$}l<{$}}
\text{Zustand}&\multicolumn{5}{l|}{Band}&\text{Berechnungsgeschichte}\\
\hline
q_0& \blank & \rot{0} & \schwarz{0} & \schwarz{1} & \blank & \blank q_0\r{0}\s{01}\blank\\
q_0& \blank & \schwarz{0} & \rot{0} & \schwarz{1} & \blank & \blank \s{0}q_0\r{0}\s{1}\blank\\
q_0& \blank & \schwarz{0} & \schwarz{0} & \rot{1} & \blank & \blank \s{00}q_0\r{1}\blank\\
q_1& \blank & \schwarz{0} & \rot{0} & \schwarz{x} & \blank & \blank \s{0}q_1\r{0}\s{x}\blank\\
q_2& \blank & \schwarz{0} & \schwarz{x} & \rot{x} & \blank & \blank \s{0x}q_2\r{x}\blank\\
q_2& \blank & \schwarz{0} & \schwarz{x} & \schwarz{x} & \rotb & \blank\s{0xx}q_2\r{\blank}\\
q_3& \blank & \schwarz{0} & \schwarz{x} & \rot{x} & \blank & \blank\s{0x}q_3\r{x}\blank\\
q_3& \blank & \schwarz{0} & \rot{x} & \schwarz{x} & \blank & \blank\s{0}q_3\r{x}\s{x}\blank\\
q_3& \blank & \rot{0} & \schwarz{x} & \schwarz{x} & \blank & \blank q_3\r{0}\s{xx}\blank\\
q_{\text{reject}}& \blank & \schwarz{0} & \schwarz{x} & \schwarz{x} &  & q_r\r{\blank}\s{0xx}\blank\\
\hline
\end{tabular}
\end{center}
\end{teilaufgaben}
\end{loesung}
\egroup

\begin{bewertung}
\begin{teilaufgaben}
\item 
Akzeptiert ({\bf A}) 1 Punkt,
Berechnungsgeschichte ({\bf G}) 1 Punkt,
\item
Für jedes Wort Punkt ({\bf B}) total 3 Punkte.
\end{teilaufgaben}
\end{bewertung}
