Beschreiben Sie eine Turing-Maschine, die einen gegebenen Stack-Automaten simuliert.

\thema{Stackautomat}
\thema{Turing-Maschine}

\begin{loesung}
Wir verwenden eine nicht deterministische Turing-Maschine mit zwei Bändern.
Das erste Band
enthält den Input, das zweite Band dient als Stack. Eine Turing-Maschine
verändert in jedem Schritt die Kopf-Position, ein Stack-Automat kann
jedoch in einem "Ubergang den Inhalt des Stack unverändert lassen,
oder keinen Input lesen. Durch Kombination von zwei "Ubergängen
kann dieses Verhalten jedoch auch auf einer Turing-Maschine simuliert
werden.

Um den Stackautomaten zu simulieren, müssen wir die Operationen des
Stackautomaten abbilden:
\begin{itemize}
\item Lesen eines Input-Zeichen: Kopf auf Band 1 um ein Feld nach
rechts bewegen. Damit das Ende des Input erkannt werden kann, muss ein
zusätzliches Bandalphabetzeichen erfunden werden, welches das Ende
des Input signalisiert.
\item Inhalt auf dem Stack ersetzen: Inhalt des Feldes unter dem Kopf
auf Band 2 ersetzen.
\item Push: Kopf auf Band 2 ein Feld nach rechts bewegen und Feld überschreiben.
\item Pop: Kopf auf Band 2 ein Feld nach links bewegen.
\end{itemize}
Wenn der Kopf auf Band 1 auf das Zeichen für das Ende des Input zeigt,
und der Stackautomatenzustand akzeptiert, hält akzeptiert die Maschine.
Andernfalls verwirft sie.
\end{loesung}
