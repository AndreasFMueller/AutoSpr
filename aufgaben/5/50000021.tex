Gegeben ist die Turingmaschine mit dem Zustandsdiagramm
\begin{center}
\def\l{2.5}
\def\zustand#1#2{
	\draw #1 circle[radius=0.35];
	\node at #1 {$#2\mathstrut$};
}
\begin{tikzpicture}[>=latex,thick]
\coordinate (q0) at (-\l,0);
\coordinate (q3) at (0,0);
\coordinate (q1) at (0,\l);
\coordinate (q2) at (0,-\l);
\coordinate (qreject) at ({1.5*\l},0);
\coordinate (qaccept) at ({-1.7*\l},{-0.8*\l});
\draw (q0) circle[radius=0.35];
\zustand{(q0)}{q_0}
\zustand{(q1)}{q_1}
\zustand{(q2)}{q_2}
\zustand{(q3)}{q_3}
\node[ellipse,draw,double] (qa) at (qaccept) {$q_\text{accept}$};
\node[ellipse,draw,double] (qr) at (qreject) {$q_\text{reject}$};
\draw[->,shorten <= 0.35cm] (q0) -- (qa.north east);
\draw[->,shorten <= 0.35cm] (q1) -- (qr.north west);
\draw[->,shorten <= 0.35cm] (q2) -- (qr.south west);
\draw[<-,shorten <= 0.35cm] (q0) -- +(-1.5,0);
\draw[->,shorten <= 0.35cm,shorten >= 0.35cm] (q0) -- (q1);
\draw[->,shorten <= 0.35cm,shorten >= 0.35cm] (q1) -- (q3);
\draw[->,shorten <= 0.35cm,shorten >= 0.35cm] (q0) -- (q2);
\draw[->,shorten <= 0.35cm,shorten >= 0.35cm] (q2) -- (q3);
\draw[->,shorten <= 0.35cm,shorten >= 0.35cm] (q3) -- (q0);
\draw[->,shorten <= 0.35cm,shorten >= 0.35cm]
	(q1) to[out=60,in=120,distance=1.2cm] (q1);
\draw[->,shorten <= 0.35cm,shorten >= 0.35cm]
	(q2) to[out=-60,in=-120,distance=1.2cm] (q2);
\draw[->,shorten <= 0.35cm,shorten >= 0.35cm]
	(q3) to[out=30,in=-30,distance=1.2cm] (q3);
\draw[->,shorten <= 0.35cm,shorten >= 0.35cm]
	(q0) to[out=90,in=135,distance=1.2cm] (q0);
\node at ($0.5*(q0)+0.5*(q3)$) [above] {$\scriptstyle\blank\to\blank,R$};
\node at ($0.5*(q0)+0.5*(q1)$) [above left]
	{$\scriptstyle\texttt{1}\to\texttt{x},R$};
\node at ($0.5*(q0)+0.5*(q2)$) [below left]
	{$\scriptstyle\texttt{0}\to\texttt{x},R$};
\node at ($0.5*(q1)+0.5*(qr.north west)$) [above right]
	{$\scriptstyle\blank\to\blank,R$};
\node at ($0.5*(q2)+0.5*(qr.south west)$) [below right]
	{$\scriptstyle\blank\to\blank,R$};
\node at ($0.5*(q1)+0.5*(q3)$)
	[right] {$\scriptstyle\texttt{0}\to\texttt{x},L$};
\node at ($0.5*(q2)+0.5*(q3)$)
	[right] {$\scriptstyle\texttt{1}\to\texttt{x},L$};
\node at ($(q0)+(0,0.8)$)
	[above left] {$\scriptstyle\texttt{x}\to\texttt{x},R$};
\node at ($(q3)+(0.8,0)$) [right] 
	{$\substack{
	{\texttt{0}\to\texttt{0},L}\\
	{\texttt{1}\to\texttt{1},L}\\
	{\texttt{x}\to\texttt{x},L}}$};
\node at ($(q1)+(0,0.8)$) [above]
	{$\substack{
	{\texttt{1}\to\texttt{1},R}\\
	{\texttt{x}\to\texttt{x},R}}$};
\node at ($(q2)+(0,-0.8)$) [below]
	{$\substack{
	{\texttt{0}\to\texttt{0},R}\\
	{\texttt{x}\to\texttt{x},R}}$};
\node at ($0.3*(q0)+0.7*(qa.north east)$)
	[above left] {$\scriptstyle\blank\to\blank,R$};
\end{tikzpicture}
\end{center}
%\[
%%\xymatrixrowsep{1in}
%%\xymatrixcolsep{1in}
%\entrymodifiers={++[o][F]}
%\xymatrix{
%*+\txt{}
%	&*+\txt{}
%		&*+\txt{}
%			&{q_1} \ar[ddrrr]^{\blank\to\blank,R}
%			       \ar[dd]^{\texttt{0}\to\texttt{x},L}
%			       \ar@(ur,ul)_{\substack{\texttt{1}\to\texttt{1},R
%						\\
%						\texttt{x}\to\texttt{x},R}}
%\\
%*+\txt{}
%\\
%*+\txt{} \ar[r]
%	&{q_0} \ar[dl]_{\blank\to\blank,R}
%	       \ar[uurr]^{\texttt{1}\to\texttt{x},R}
%	       \ar[ddrr]_{\texttt{0}\to\texttt{x},R}
%	       \ar@(u,ul)_{\texttt{x}\to\texttt{x},R}
%		&*+\txt{}
%			&{q_3} \ar[ll]_{\blank\to\blank,R}
%			       \ar@(ur,dr)^{\substack{
%{\texttt{0}\to\texttt{0},L}\\
%{\texttt{1}\to\texttt{1},L}\\
%{\texttt{x}\to\texttt{x},L}}}
%				&*+\txt{}
%					&*+\txt{}
%						&*++[o][F=]{q_{\text{reject}}}
%\\
%*++[o][F=]{q_{\text{accept}}}
%\\
%*+\txt{}
%	&*+\txt{}
%		&*+\txt{}
%			&{q_2} \ar[uurrr]_{\blank\to\blank,R}
%			       \ar[uu]_{\texttt{1}\to\texttt{x},L}
%			       \ar@(dr,dl)^{\substack{\texttt{0}\to\texttt{0},R
%					\\
%					\texttt{x}\to\texttt{x},R}}
%}
%\]
über dem Alphabet
$\Sigma = \{\texttt{0},\texttt{1}\}$
und dem Bandalphabet
$\Gamma = \{\texttt{0},\texttt{1},\texttt{x},\blank\}$.

\begin{teilaufgaben}
\item Wird das Wort \texttt{101} akzeptiert?
\item Wird das Wort \texttt{0110} akzeptiert?
\item Falls das Band zwei durch $\blank$ getrennte Wörter enthält,
kann das zweite Wort einen Einfluss darauf haben, ob der Input akzeptiert
wird?
\item
Es wird behauptet, dass die Maschine alle Wörter
$w$ mit $|w|_{\texttt{0}}=|w|_{\texttt{1}}$
akzeptiert.
Ist dies korrekt?
\end{teilaufgaben}

\thema{Turing-Maschine}
\thema{Zustandsdiagramm}

\begin{loesung}
\begin{teilaufgaben}
\item
Die Verarbeitung des Wortes \texttt{101} ist (Zeichen unter dem
Schreib-/Lesekopf jeweils rot)
\begin{center}
\def\r{\color{red}}
\begin{tabular}{>{$}r<{$}|>{\tt}c>{\tt}c>{\tt}c>{\tt}c>{\tt}c>{\tt}c}
q_0              &    \blank & \r 1 &    0 &    1 &    \blank &    \blank \\
q_1              &    \blank &    x & \r 0 &    1 &    \blank &    \blank \\
q_3              &    \blank & \r x &    x &    1 &    \blank &    \blank \\
q_3              & \r \blank &    x &    x &    1 &    \blank &    \blank \\
q_0              &    \blank & \r x &    x &    1 &    \blank &    \blank \\
q_0              &    \blank &    x & \r x &    1 &    \blank &    \blank \\
q_0              &    \blank &    x &    x & \r 1 &    \blank &    \blank \\
q_1              &    \blank &    x &    x &    x & \r \blank &    \blank \\
q_{\text{reject}}&     \blank&    x &    x &    x &    \blank & \r \blank
\end{tabular}
\end{center}
Das Wort \texttt{101} wird nicht akzeptiert.
\item
Die Verarbeitung des Wortes \texttt{0110} ist (Zeichen unter dem
Schreib-/Lesekopf jeweils rot)
\begin{center}
\def\r{\color{red}}
\begin{tabular}{>{$}r<{$}|>{\tt}c>{\tt}c>{\tt}c>{\tt}c>{\tt}c>{\tt}c>{\tt}c}
q_0 &  \blank&\r 0&   1&   1&   0&  \blank&  \blank\\
q_2 &  \blank&   x&\r 1&   1&   0&  \blank&  \blank\\
q_3 &  \blank&\r x&   x&   1&   0&  \blank&  \blank\\
q_3 &\r\blank&   x&   x&   1&   0&  \blank&  \blank\\
q_0 &  \blank&\r x&   x&   1&   0&  \blank&  \blank\\
q_0 &  \blank&   x&\r x&   1&   0&  \blank&  \blank\\
q_0 &  \blank&   x&   x&\r 1&   0&  \blank&  \blank\\
q_1 &  \blank&   x&   x&   x&\r 0&  \blank&  \blank\\
q_3 &  \blank&   x&   x&\r x&   x&  \blank&  \blank\\
q_3 &  \blank&   x&\r x&   x&   x&  \blank&  \blank\\
q_3 &  \blank&\r x&   x&   x&   x&  \blank&  \blank\\
q_3 &\r\blank&   x&   x&   x&   x&  \blank&  \blank\\
q_0 &  \blank&\r x&   x&   x&   x&  \blank&  \blank\\
q_0 &  \blank&   x&\r x&   x&   x&  \blank&  \blank\\
q_0 &  \blank&   x&   x&\r x&   x&  \blank&  \blank\\
q_0 &  \blank&   x&   x&   x&\r x&  \blank&  \blank\\
q_0 &  \blank&   x&   x&   x&   x&\r\blank&  \blank\\
q_{\text{accept}} &  \blank&   x&   x&   x&   x&  \blank&\r\blank\\
\end{tabular}
\end{center}
Das Wort \texttt{0110} wird also akzeptiert.
\item
Zunächst ist festzuhalten, dass $\Sigma$ das Zeichen $\blank$ nicht enthält,
dass diese hypothetische Situation also normalerweise nicht auftritt.
Leerzeichen zeigen also immer an, dass man auf einen Teil des Bandes
gestossen ist, der nicht vom Input-Wort initialisiert worden ist.
Es geht also um die hypothetische Situation, dass man im Input $\blank$-Zeichen
zulässt, und man möchte wissen, wie die Maschine damit umgeht.

Das erste $\blank$-Zeichen rechts vom Wortanfang kann nur durch die Übergänge 
von $q_1$ nach $q_{\text{reject}}$,
von $q_2$ nach $q_{\text{reject}}$
und
von $q_0$ nach $q_{\text{accept}}$
überschritten werden.
Der Übergang von $q_3$ nach $q_0$ folgt immer auf einen Übergang
mit Kopfbewegung nach links, das Leerzeichen muss also ein Leerzeichen
links vom ersten Wort sein.
Da die Maschine nach allen genannten Übergängen anhält, wird der
nachfolgende Teil des Bandes nie gelesen, und kann daher auch den
Ausgang nicht beeinflussen.
\item
Die Turingmaschine überschreibt \texttt{0} und \texttt{1} paarweise mit
\texttt{x} und akzeptiert, wenn am Schluss nur \texttt{x} dastehen.
Die akzeptierte Sprache ist daher
\[
L=\{ w\in\Sigma^* \;|\; |w|_{\texttt{0}}=|w|_{\texttt{1}}\}.
\qedhere
\]
\end{teilaufgaben}
\end{loesung}

\begin{bewertung}
Teilaufgabe a) ({\bf A}) 1 Punkt,
Teilaufgabe b) ({\bf B}) 1 Punkt,
Teilaufgabe c): korrekte Antwort ($\textbf{C}_a$) 1 Punkt,
Begründung ($\textbf{C}_b$) 1 Punkt,
Teilaufgabe d): korrekte Antwort ($\textbf{D}_a$) 1 Punkt,
Begründung ($\textbf{D}_b$) 1 Punkt.
\end{bewertung}

