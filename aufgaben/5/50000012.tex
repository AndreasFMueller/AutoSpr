Hilberts zehntes Problem verlangt, man solle einen Algorithmus
konstruieren, welcher entscheidet, ob ein Polynom $p(x_1,\dots,x_n)$
mit ganzzahligen
Koeffizienten eine ganzzahlige Nullstelle hat.
Warum ist der folgende Algorithmus keine Lösung des Problems?
\begin{enumerate}
\item Erzeuge alle in Frage kommenden Tupel $x_1,\dots,x_n$
\item Berechne $p(x_1,\dots,x_n)$ für diese Werte der Argumente
\item Falls einer der Polynomwerte $0$ ist, akzeptiere, andernfalls
verwerfe.
\end{enumerate}
Wie könnte man den Algorithmus verbessern, damit er erkennen kann,
ob ein Polynom eine ganzzahlige Lösung hat?

\thema{Entscheidbarkeit}

\begin{loesung}
Die Beschreibung verlangt, zuerst alle $n$-Tupel zu erzeugen,
und möglicherweise irgendwo auf ein Band zu schreiben. Dieser
Schritt wird nie terminieren, so dass der Algorithmus keine
Wörter erkennen kann.

Korrigiert werden könnte dieses Defizit dadurch, dass man eine
Aufzählung der $n$-Tupel $(x_1,\dots,x_n)$ konstruiert, und ein
$n$-Tupel nach dem anderen in das Polynom $p$
einsetzt. Falls das Polynom den Wert $0$ annimmt, akzeptiert man
das Polynom. Insbesondere kann man niemals verwerfen, falls
ein Polynom keine Nullstelle hat, terminiert die Maschine nicht.
Eine Aufzählung der $n$-Tupel kann zum Beispiel dadurch konstruiert
werden, dass zunächst alle Tupel mit
$\max_{1\le i\le n}|x_i|=0$
aufgezählt werden (es gibt genau ein solches $n$-Tupel),
dann alle $n$-Tupel mit
$\max_{1\le i\le n}|x_i|=1$,
gefolgt von den $n$-Tupeln mit
$\max_{1\le i\le n}|x_i|=2$, und so weiter.

Yuri Matjassevitch hat aufbauend auf Arbeiten von Julia Robinson bewiesen,
dass es
keinen Entscheider für das zehnte Hilbertsche Problem gibt,
ein Erkenner wie eben konstruiert ist also das beste, was
man bekommen kann.
\end{loesung}
