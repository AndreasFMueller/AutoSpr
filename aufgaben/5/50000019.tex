Gegeben ist eine Turing-Maschine $M$ über dem Alphabet
$\Sigma=\{\texttt{0}, \texttt{1}\}$,
Bandalphabet
$\Gamma=\{\texttt{0}, \texttt{1}, \blank\}$
und mit drei Zuständen
(inklusive $q_{\text{accept}}$ und $q_{\text{reject}}$).
Man weiss folgendes über $M$:
\begin{enumerate}
\item\label{50000019:liestalles} Jedes Zeichen des Inputwortes wird
von $M$ irgendwann gelesen.
\item\label{50000019:haeltaufblank} Sie hält im Zustand $q_{\text{accept}}$
nur, wenn sie auf ein Blank $\blank$ auf dem Band gestossen ist.
\end{enumerate}
\begin{teilaufgaben}
\item
Wieviele verschiedene Turing-Maschinen mit diesen Eigenschaften gibt es?
\item
Welche Sprache akzeptieren sie?
\end{teilaufgaben}

\thema{Turing-Maschine}

\begin{loesung}
\begin{teilaufgaben}
\item
Da die Turing-Maschine nach Bedingung~1 ganz bestimmt etwas tut, kann
weder $q_{\text{accept}}$ noch $q_{\text{reject}}$ der Startzustand
sein.

Das Zustandsdiagramm einer Turing-Maschine mit drei Zuständen, die
Bedingung~\ref{50000019:haeltaufblank} erfüllt, sieht so aus:
\[
\entrymodifiers={++[o][F]}
\xymatrix @+5mm {
*+\txt{} \ar[r]
        &{q_0} \ar[r]^{\blank\to ?,?}
		&*++[o][F=]{q_{\text{accept}}}
\\
*+\txt{}
	&*++[o][F=]{q_{\text{reject}}}
}
\]
Natürlich fehlen hier noch einige Pfeile. Damit alle Zeichen gelesen werden
können, müssen im Zustand $q_0$ Übergänge sowohl für \texttt{0}
wie auch für \texttt{1} möglich sein. Damit das ganze Wort gelesen werden
kann, muss für beide Zeichen die Kopfbewegung nach rechts erfolgen.
Würde nämlich nach einem Zeichen, z.~B.~\texttt{0}, eine Kopfbewegung
nach links erfolgen, dann könnten im Wort \texttt{00} das zweite
Zeichen nicht gelesen werden, obwohl Bedingung~1 dies verlangt.

Das Zustandsdiagramm muss also wie folgt aussehen:
\[
\entrymodifiers={++[o][F]}
\xymatrix @+5mm {
*+\txt{} \ar[r]
        &{q_0} \ar[r]^{\blank\to {\color{red}?},{\color{red}?}}
		\ar@(ur,u)_{\texttt{0}\to {\color{red}?},\text{R}}
		\ar@(ul,u)^{\texttt{1}\to {\color{red}?},\text{R}}
		&*++[o][F=]{q_{\text{accept}}}
\\
*+\txt{}
	&*++[o][F=]{q_{\text{reject}}}
}
\]
Nur noch die roten Fragezeichen {\color{red}?} müssen festgelegt
werden.
Da hierfür jeweils drei mögliche Werte aus $\Gamma$ in Frage kommen
und zwei Richtungen,
gibt es $3^3\cdot 2 =54$ verschiedene Turing-Maschinen mit den verlangten
Eigenschaften.

Man kann argumentieren, dass die Bewegungsrichtung beim Übergang zu 
$q_{\text{accept}}$ gegenstandslos ist, da unabhängig von der
Bewegungsrichtung das berechnete Resultat auf dem Band das gleiche sein wird.
Wenn man dies akzeptiert, gibt es nur noch 27 verschiedene Turing-Maschinen
mit den verlangten Eigenschaften.

Man kann auch ablesen, dass der Zustand $q_{\text{reject}}$ nicht erreicht
werden kann.
Damit bleiben nur noch die zwei Möglichkeiten, dass die Maschine im
Zustand $q_{\text{accept}}$ hält, oder dass sie unendlich lange läuft.

\item
Offenbar kommen Kopfbewegungen nach links gar nicht vor, der Kopf bewegt
sich nach jedem Schritt nach rechts, bis er auf ein $\blank$ trifft.
Die Zeichen, die geschrieben werden, werden also nicht wieder gelesen, und
haben daher keinen Einfluss auf die Sprache.
Die Machine liest also in jedem Fall einfach nur von links nach rechts
und hält beim ersten $\blank$ im Zustad $q_{\text{accept}}$.
Die akzeptierte Sprache ist also unabhängig von der Wahl der roten
{\color{red}?} Fragezeichen immer $L=\Sigma^*$.
\qedhere
\end{teilaufgaben}
\end{loesung}

\begin{bewertung}
\begin{teilaufgaben}
\item
Startzustand ({\bf S}) 1 Punkt,
Übergang für $\blank$ ({\bf B}) 1 Punkt,
Übergänge für \texttt{0} und \texttt{1} haben Kopfbewegung nach rechts
({\bf R}) 1 Punkt,
Wahlmöglichkeiten nur für die geschriebenen Zeichen ({\bf W}) 1 Punkt,
Anzahl Wahlmöglichkeiten ({\bf A}) 1 Punkt.
\item
Sprache $\Sigma^*$ ({\bf *}) 1 Punkt.
\end{teilaufgaben}
\end{bewertung}


