Konstruieren Sie eine Turing-Maschine, die eine Binärzahl auf dem Band
hochzählt ohne je damit aufzuhören.

\thema{Turing-Maschine}
\thema{Zustandsdiagramm}

\begin{loesung}
In Aufgabe~\ref{50000001} wurde gezeigt, wie man eine Binärzahl auf dem Band
um eins erhöht.
Die dort konstruiert Maschine
\[
\entrymodifiers={++[o][F]}
\xymatrix @+5mm {
*+\txt{}\ar[r]
	&q_0\ar@(ur,ul)_{\scriptstyle{\tt 0}\to{\tt 0},R\atop \scriptstyle{\tt 1}\to{\tt 1},R}
	    \ar[r]^{\blank\to\blank,L}
		&q_1\ar[r]^{{\scriptstyle{\tt 0}\to{\tt 1},R}
		    \atop {\scriptstyle\blank\to{\tt 1},R}}
		    \ar@(ur,ul)_{{\tt 1}\to{\tt 0},L}
			&*++[o][F=]{q_{\text{accept}}}
}
\]
muss man nun so erweitern, dass sie dies immer wieder tut.
\[
\entrymodifiers={++[o][F]}
\xymatrix @+5mm {
*+\txt{}\ar[r]
	&{q_0}	
	    \ar@(ur,ul)_{\scriptstyle{\tt 0}\to{\tt 0},R\atop \scriptstyle{\tt 1}\to{\tt 1},R}
	    \ar@/^/[r]^{\blank\to\blank,L}
		&{q_1}
		    \ar@(ur,dr)^{{\tt 1}\to{\tt 0},L}
		    \ar@/^/[l]^{{\scriptstyle{\tt 0}\to{\tt 1},R}
			\atop {\scriptstyle\blank\to{\tt 1},R}}
%			&*++[o][F=]{q_{\text{accept}}}
}
\qedhere
\]
\end{loesung}

\begin{diskussion}
Eine solche Turing-Maschine hat Google als Doodle am 23.~Juni 2012
aus Anlass von Alan Turings hundertstem Geburtstag publiziert.
URL: \url{https://www.google.com/doodles/alan-turings-100th-birthday}
\end{diskussion}

