Sind die folgenden Mengen abzählbar oder überabzählbar unendlich?
\begin{teilaufgaben}
\item
Die Menge $G_n$ aller linearen Gleichungssysteme mit $n$ Gleichungen
und $n$ Unbekannten und ganzzahligen Koeffizienten und rechten Seiten.
\item
Die Menge $G$ aller linearen Gleichungssysteme mit gleich vielen Gleichungen
wie Unbekannten und ganzzahligen Koeffizienten.
\item
Die Menge $\mathbb Z[X]_n$ aller Polynome
\[
a_{\mathstrut n}X^n+a_{n-1}X^{n-1}+\dots a_2X^2+a_1X+a_0
\]
vom Grad $n$ mit ganzzahligen Koeffizienten.
\item
Die Menge $\mathbb Z[X]$ aller Polynome
beliebigen Grades $n$ mit ganzzahligen Koeffizienten.
\item
Eine Zahl $x\in \mathbb R$ heisst algebraisch, wenn sie Nullstelle
eines Polynoms mit ganzzahligen Koeffizienten ist.
Die Zahl $\sqrt{2}$ ist Nullstelle von $X^2-2$ und ist daher algebraisch.
Die Zahlen $\pi$ und $e$ sind dagegen nicht algebraisch, dies ist sehr
aufwendig zu beweisen.
Ist die Menge $\mathbb{A}$ aller algebraischen Zahlen abzählbar oder
überabzählbar unendlich?
\end{teilaufgaben}

\themaL{abzahlbar}{abzählbar}

\begin{loesung}
\begin{teilaufgaben}
\item
Ein $n\times n$-Gleichungssystem mit ganzzahligen 
Koeffizienten ist beschrieben durch $n^2+n$ ganze Zahlen, nämlich
die $n^2$ Koeffizienten und die $n$ rechten Seiten.
Die Menge der $n\times n$-Gleichungssysteme ist daher gleich
mächtig wie $\mathbb Z^{n(n+1)}$, eine abzählbar unendliche Menge.
\item
Die Menge $G$ aller Gleichungssysteme ist daher 
\[
G = \bigcup_{n=1}^\infty G_n \simeq \bigcup_{n=1}^\infty \mathbb Z^{n(n+1)}.
\]
Die rechte Seite ist als abzählbare Vereinigung von abzählbaren Mengen
wieder abzählbar.
\item
Die Menge der Polynome vom Grad $n$ mit ganzzahligen Koeffizienten
ist gleich mächtig wie die Menge $\mathbb Z^{n+1}$.
Die Abbildung
\[
a_{\mathstrut n}X^n+a_{n-1}X^{n-1}+\dots a_2X^2+a_1X+a_0
\mapsto
(a_0,a_1,\dots,a_{n+1},a_{\mathstrut n}) \in \mathbb{Z}^{n+1}
\]
vermittelt die bijektive Abbildung zwischen $G$ und $\mathbb Z^{n+1}$.
\item
Die Menge $\mathbb Z[X]$ der ganzzahligen Polynome ist gleichmächtig
\[
\mathbb Z[X]
\simeq
\bigcup_{n=1}^\infty \mathbb Z[X]_n
\simeq
\bigcup_{n=1}^\infty \mathbb Z^{n+1}.
\]
Da $\mathbb Z[X]_n$ abzählbar unendlich ist, ist auch $\mathbb Z[X]$
als abzählbare Vereinigung abzählbarer Mengen abzählbar.
\item
Zu jeder algebraischen Zahl gibt es ein ganzzahliges Polynom, welches
die algebraische Zahl als Nullstelle hat.
Es gibt nur abzählbar unendlich viele verschiedene ganzzahlige Polynome
also kann es höchstens abzählbar unendlich viele algebraische Zahlen geben.
\qedhere
\end{teilaufgaben}
\end{loesung}


