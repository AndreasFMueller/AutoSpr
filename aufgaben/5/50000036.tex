Zeichnen Sie Zustandsdiagramme für die Turing-Maschinen
mit dem Alphabet $\Sigma=\{\texttt{0},\texttt{1}\}$, die die folgenden
Sprachen akzeptieren.
\begin{teilaufgaben}
\item
\(
L_1
=
\{ w\in\Sigma^*
\mid 
|w| > 1
\wedge
\operatorname{first}(w) = \operatorname{last}(w)
\}
\),
wobei $\operatorname{first}(w)$ das erste Zeichen von $w$
ist und $\operatorname{last}(w)$ das letzte.
In $L_1$ sind genau die Wörter mit mindestens zwei Zeichen, deren erstes und
letztes Zeichen übereinstimmt.
\item
\(
L_2
=
\{ w\in\Sigma^*
\mid 
|w| > 1
\wedge
|w|_{\texttt{0}} \equiv |w|_{\texttt{1}} \mod 2
\}
\).
In $L_2$ sind genau die Wörter mit mindestens zwei Zeichen, 
die gleich viele \texttt{0} wie \texttt{1} modulo $2$ enthalten.
\item
\(
L_3
=
\{ w\in\Sigma^*
\mid 
|w| > 1
\wedge
\operatorname{first}(w) = \operatorname{last}(w)
\wedge
|w|_{\texttt{0}} \equiv |w|_{\texttt{1}} \mod 2
\}
=
L_1\cap L_2
\)
\end{teilaufgaben}

\thema{Turing-Maschine}

\begin{loesung}
In Teilaufgabe c) müssen beide Bedingungen von Teilaufgaben a) und b)
geprüft werden.
Dies kann man erreichen, indem man die Maschinen nacheinander laufen
lässt, dazu muss die erste Maschine aber den ursprünglichen Zustand
wieder herstellen, d.~h.~Schreib-/Lesekopf auf dem ersten Zeichen.
Dem wird in der Konstruktion von a) Rechnung getragen.
\begin{teilaufgaben}
\item
Das folgende Zustandsdiagramm akzeptiert genau die Wörter aus $L_1$:
\begin{center}
\begin{tikzpicture}[>=latex,thick]

\coordinate (S) at (-2.25,0);

\coordinate (A0) at (-1.0,2.16);
\coordinate (A1) at (1.5,2.16);
\coordinate (A2) at (4.0,2.16);
\coordinate (B0) at (-1.0,-2.16);
\coordinate (B1) at (1.5,-2.16);
\coordinate (B2) at (4.0,-2.16);

\coordinate (C) at (5.25,0);

\coordinate (R) at (2.75,0);
\coordinate (A) at (8.05,0);

\node at (S) {$q_0$};
\node at (C) {$q_1$};
\node at (A0) {$r_0$};
\node at (A1) {$r_1$};
\node at (A2) {$r_2$};
\node at (B0) {$s_0$};
\node at (B1) {$s_1$};
\node at (B2) {$s_2$};

\draw (S) circle[radius=0.3];

\draw (A0) circle[radius=0.3];
\draw (A1) circle[radius=0.3];
\draw (A2) circle[radius=0.3];

\draw (B0) circle[radius=0.3];
\draw (B1) circle[radius=0.3];
\draw (B2) circle[radius=0.3];

\draw (C) circle[radius=0.3];

\draw (R) ellipse (0.6cm and 0.3cm);
\draw (R) ellipse (0.65cm and 0.35cm);
\node at (R) {$q_{\text{reject}}$};

\draw (A) ellipse (0.6cm and 0.3cm);
\draw (A) ellipse (0.65cm and 0.35cm);
\node at (A) {$q_{\text{accept}}$};

\draw[->,shorten >= 0.3cm,shorten <= 0.3cm] (S) -- (A0);
\draw[->,shorten >= 0.3cm,shorten <= 0.3cm] (A0) -- (A1);
\draw[->,shorten >= 0.3cm,shorten <= 0.3cm] (A1) -- (A2);
\draw[->,shorten >= 0.3cm,shorten <= 0.3cm] (S) -- (B0);
\draw[->,shorten >= 0.3cm,shorten <= 0.3cm] (B0) -- (B1);
\draw[->,shorten >= 0.3cm,shorten <= 0.3cm] (B1) -- (B2);

\draw[->,shorten >= 0.3cm,shorten <= 0.3cm] (A2) -- (C);
\draw[->,shorten >= 0.3cm,shorten <= 0.3cm] (B2) -- (C);

\draw[->,shorten >= 0.3cm,shorten <= 0.3cm]
	(A1) to[out=45,in=135,distance=1cm] (A1);
\draw[->,shorten >= 0.3cm,shorten <= 0.3cm]
	(B1) to[out=-45,in=-135,distance=1cm] (B1);

\draw[->,shorten >= 0.6cm,shorten <= 0.3cm]
	(C) -- (A);

%\draw[->,shorten >= 0.4cm,shorten <= 0.3cm] (A1) -- (R);
\draw[->,shorten >= 0.4cm,shorten <= 0.3cm] (A2) -- (R);
%\draw[->,shorten >= 0.4cm,shorten <= 0.3cm] (B1) -- (R);
\draw[->,shorten >= 0.4cm,shorten <= 0.3cm] (B2) -- (R);

\draw[->,shorten >= 0.3cm,shorten <= 0.3cm]
	(C) to[out=-20,in=-70,distance=1cm] (C);
\node at ($(C)+(0.4,-0.7)$) [right] {$\texttt{.}\to\texttt{.},L$};

\node at ($0.5*(A0)+0.5*(A1)$) [above] {$\texttt{.}\to\texttt{.},R$};
\node at ($0.5*(B0)+0.5*(B1)$) [below] {$\texttt{.}\to\texttt{.},R$};

\node at ($0.5*(A0)+0.5*(S)$) [above left] {$\texttt{0}\to\texttt{0},R$};
\node at ($0.5*(B0)+0.5*(S)$) [below left] {$\texttt{1}\to\texttt{1},R$};

\node at ($0.75*(A2)+0.25*(C)+(0,-0.2)$) [above right] {$\texttt{0}\to\texttt{0},L$};
\node at ($0.75*(B2)+0.25*(C)+(0,+0.2)$) [below right] {$\texttt{1}\to\texttt{1},L$};

\node at ($0.5*(A2)+0.5*(R)$) [left] {$\blank\to\blank,R$};
\node at ($0.5*(B2)+0.5*(R)$) [left] {$\blank\to\blank,R$};

\node at ($0.5*(A1)+0.5*(A2)$) [above] {$\blank\to\blank,L$};
\node at ($0.5*(B1)+0.5*(B2)$) [below] {$\blank\to\blank,L$};

\node at ($(A1)+(0,0.8)$) {$\texttt{.}\to\texttt{.},R$};
\node at ($(B1)-(0,0.8)$) {$\texttt{.}\to\texttt{.},R$};

\node at ($0.5*(C)+0.5*(A)-(0.15,0)$) [above] {$\blank\to\blank,R$};

\draw[->,shorten >= 0.3cm] (-3.0,0) -- (S);

\end{tikzpicture}
\end{center}
Die Zustände $r_1$ und $r_2$ werden benutzt, um zu einem Anfangszeichen
\texttt{0}
eine zugehöriges Endzeichen
\texttt{0}
zu finden, und entsprechend werden $s_1$ und $s_2$ für das Zeichen
\texttt{1} benutzt.
Pfeile, die mit $\texttt{.}\to\texttt{.},R$  beschriftet sind beschreiben
Übergänge, die für jedes von \blank{} verschiedene Zeichen genommen werden.
Die Turing-Maschine bewegt im Zustand $q_1$ den Schreib-/Lesekopf wieder
zurück zum Anfang des Wortes, dies wird für Teilaufgabe c) nützlich werden.
\item
Ein Wort hat genau dann gleich viele \texttt{0} wie \texttt{1} modulo 2,
wenn die Anzahl der Zeichen gerade ist.
Das folgende Zustandsdiagramm beschreibt eine Turing-Maschine, die genau
die Wörter der Sprache $L_2$ akzeptiert.
\begin{center}
\begin{tikzpicture}[>=latex,thick]

\coordinate (S) at (0,0);
\coordinate (A) at (3,0);
\coordinate (B) at (6,0);
\coordinate (a) at (9,0);
\coordinate (r) at (1.5,-2);

\draw[->,shorten >= 0.3cm] (-1.5,0) -- (S);

\node at (S) {$q_0$};
\node at (A) {$q_1$};
\node at (B) {$q_2$};
\node at (a) {$q_{\text{accept}}$};
\node at (r) {$q_{\text{reject}}$};

\draw (S) circle[radius=0.3];
\draw (A) circle[radius=0.3];
\draw (B) circle[radius=0.3];
\draw (a) ellipse (0.6cm and 0.3cm);
\draw (a) ellipse (0.65cm and 0.35cm);
\draw (r) ellipse (0.6cm and 0.3cm);
\draw (r) ellipse (0.65cm and 0.35cm);

\draw[->,shorten >= 0.3cm, shorten <= 0.3cm] (S) -- (A);

\draw[->,shorten >= 0.3cm, shorten <= 0.3cm] (A) to[out=30,in=150] (B);
\draw[->,shorten >= 0.3cm, shorten <= 0.3cm] (B) to[out=-150,in=-30] (A);

\draw[->,shorten >= 0.40cm, shorten <= 0.3cm] (A) -- (r);
\draw[->,shorten >= 0.40cm, shorten <= 0.3cm] (S) -- (r);
\draw[->,shorten >= 0.65cm, shorten <= 0.3cm] (B) -- (a);


\node at ($0.5*(S)+0.5*(A)$) [above] {$.\to.,R$};
\node at ($0.5*(A)+0.5*(B)+(0,0.35)$) [above] {$.\to.,R$};
\node at ($0.5*(A)+0.5*(B)-(0,0.35)$) [below] {$.\to.,R$};

\node at ($0.5*(S)+0.5*(r)$) [below left] {$\blank\to\blank,R$};
\node at ($0.5*(A)+0.5*(r)$) [below right] {$\blank\to\blank,R$};
\node at ($0.5*(B)+0.5*(a)$) [above] {$\blank\to\blank,R$};

\end{tikzpicture}
\end{center}
\item
Die Turing-Maschine von Teilaufgabe ist so aufgebaut, dass im Falle,
dass das Wort akzeptiert wird, der Schreib-/Lesekopf wieder auf dem
ersten Zeichen steht.
Man kann daher ein Turing-Maschine für die Sprache $L_3$ bekommen,
indem man man die beiden Zustandsdiagramme der Teilaufgaben a) und b)
verkettet, indem man den $q_{\text{accept}}$ von a) zum Startzustand
von b) macht.
\qedhere
\end{teilaufgaben}
\end{loesung}

\begin{bewertung}
\begin{teilaufgaben}
\item Unterschiedliche Pfade für die beiden möglichen Anfangszeichen 
({\bf A}) 1 Punkt,
Überspringen aller Zwischenzeichen ({\bf U}) 1 Punkt,
Zurückgehen am Ende ({\bf Z}) 1 Punkt (evtl in Teilaufgabe c).
\item Vier Zustände für die mögliche Paritäten ({\bf P}) 1 Punkt,
Akzeptieren nur bei gleicher Parität ({\bf G}) 1 Punkt.
\item Verkettung der Lösungen von a) und b) ({\bf V}) 1 Punkt.
\end{teilaufgaben}
\end{bewertung}
