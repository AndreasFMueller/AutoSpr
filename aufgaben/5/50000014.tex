Sei $\Sigma=\{{\tt 1}\}$. Zeigen Sie: es gibt eine nicht entscheidbare
Sprache $L$ über $\Sigma$.
D.~h.~es gibt eine Teilmenge $L\subset \Sigma^*$, die nicht
von einer Turing-Maschine erkannt werden kann\footnote{Da in $L$
nur die Länge eines Wortes wesentlich ist, sagt dieses Resultat
aus, dass es eine Zahlenmenge gibt, die nicht von einer Turingmaschine
berechnet werden kann.}.

\begin{hinweis}
Hilbert Hotel.
\end{hinweis}

\begin{loesung}
Die Menge der entscheidbaren Sprachen ist abzählbar, wie bereits in
der Vorlesung gezeigt wurde. Sie stehen ja
in eins-zu-eins Beziehung zu den Turing-Maschinen. Die Beschreibungen
der Turing-Maschinen kann man ja der Grösse nach und bei gleicher Grösse
lexikographisch anordnen, und so eine Auf\-zählung aller Turing-Maschinen
erhalten.

Jeder Teilmenge von $\Sigma^*$ entspricht eindeutig eine Teilmenge
von $\mathbb N$, zu einer Teilmenge $L\subset\Sigma^*$
bildet man einfach die Menge der in $L$ vorkommenden Wortlängen.

Die Menge der Teilmengen von $\mathbb N$, die Potenzmenge $P(\mathbb N)$,
ist aber überabzählbar.
Es gibt also wesentlich mehr Sprachen als Turing-Maschinen, es
muss also auch Sprachen geben, die nicht von einer Turing-Maschine
erkannt werden können.
\end{loesung}
