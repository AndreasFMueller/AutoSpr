In der Turing-Maschine 
\[
M = (Q, \Sigma, \Gamma, \delta, q_0, q_{\text{accept}}, q_{\text{reject}})
\]
mit
\begin{align*}
Q &= \{q_0, q_1,q_{\text{accept}}, a_{\text{reject}}\}
\\
\Sigma &= \{\texttt{1}\}
\\
\Gamma &= \{\texttt{1},\texttt{0}\}
\end{align*}
hat die Übergangsfunktion $\delta(q,a) = (q',a',d)$ folgende Wertetabelle:
\begin{center}
\begin{tabular}{|>{$}c<{$}>{$}c<{$}|>{$}c<{$}>{$}c<{$}>{$}c<{$}|}
\hline
q&a&q'&a'&d\\
\hline
q_0&\texttt{0}&q_1              &\texttt{1}&\text{R}\\
q_0&\texttt{1}&q_1              &\texttt{1}&\text{L}\\
q_1&\texttt{0}&q_0              &\texttt{1}&\text{L}\\
q_1&\texttt{1}&q_{\text{accept}}&\texttt{1}&\text{R}\\
\hline
\end{tabular}
\end{center}
Man beachte, dass das Zeichen \texttt{0} die Funktion des Leerzeichens
auf dem Band "ubernimmt.
Das Band ist daher zu Beginn mit \texttt{0} initialisiert.

\begin{teilaufgaben}
\item Zeichnen Sie ein Zustandsdiagramm dieser Turing-Maschine.
\item Hält diese Turing-Maschine auf leerem Band und wenn ja, nach
wievielen Schritten und mit welchem Bandinhalt?
\end{teilaufgaben}

Dieser Automat heisst ein {\em Fleissiger Biber} mit zwei Zuständen.

\thema{Turing-Maschine}
\thema{Zustandsdiagramm}

\begin{loesung}
\begin{teilaufgaben}
\item
Die Turingmaschine wird durch das Zustandsdiagramm
\[
\entrymodifiers={++[o][F]}
\xymatrix @+5mm {
*+\txt{} \ar[r]
	&{q_0} \ar@/^/[r]^{\scriptstyle\texttt{0}\to\texttt{1},R \atop \scriptstyle\texttt{1}\to\texttt{1},L}
		&{q_1}	\ar[r]^{\texttt{1}\to\texttt{1},R}
			\ar@/^/[l]^{\texttt{0}\to\texttt{1},L}
			&*++[o][F=]{q_{\text{accept}}}
}
\]
beschrieben.
\item
Die Berechnung der Turingmaschine $M$ auf leerem Band verläuft wie folgt
(aktuelle Kopfposition {\text{red}rot} angedeutet):
\begin{center}
\begin{tabular}{l>{$}r<{$} >{$}c<{$} >{$}c<{$} >{$}c<{$} >{$}c<{$} >{$}c<{$} >{$}c<{$} >{$}c<{$} >{$}c<{$} >{$}c<{$} >{$}c<{$} }
% 0
&
q_0 &
\dots &
\texttt{0} &
\texttt{0} &
\texttt{0} &
\texttt{0} &
\color{red}\texttt{0} &
\texttt{0} &
\texttt{0} &
\texttt{0} &
\dots
\\
% 1
1:&
q_1 &
\dots &
\texttt{0} &
\texttt{0} &
\texttt{0} &
\texttt{0} &
\texttt{1} &
\color{red}\texttt{0} &
\texttt{0} &
\texttt{0} &
\dots
\\
% 2
2:&
q_0 &
\dots &
\texttt{0} &
\texttt{0} &
\texttt{0} &
\texttt{0} &
\color{red}\texttt{1} &
\texttt{1} &
\texttt{0} &
\texttt{0} &
\dots
\\
% 3
3:&
q_1 &
\dots &
\texttt{0} &
\texttt{0} &
\texttt{0} &
\color{red}\texttt{0} &
\texttt{1} &
\texttt{1} &
\texttt{0} &
\texttt{0} &
\dots
\\
% 4
4:&
q_0 &
\dots &
\texttt{0} &
\texttt{0} &
\color{red}\texttt{0} &
\texttt{1} &
\texttt{1} &
\texttt{1} &
\texttt{0} &
\texttt{0} &
\dots
\\
% 5
5:&
q_1 &
\dots &
\texttt{0} &
\texttt{0} &
\texttt{1} &
\color{red}\texttt{1} &
\texttt{1} &
\texttt{1} &
\texttt{0} &
\texttt{0} &
\dots
\\
% 6
6:&
q_{\text{accept}} &
\dots &
\texttt{0} &
\texttt{0} &
\texttt{1} &
\texttt{1} &
\color{red}\texttt{1} &
\texttt{1} &
\texttt{0} &
\texttt{0} &
\dots
\\
\end{tabular}
\end{center}
Die Maschine hält nach sechs Schritten im Zustand $q_{\text{accept}}$
und hinterlässt die Zeichenkette
\texttt{1111} auf dem Band.
\qedhere
\end{teilaufgaben}
\end{loesung}

\begin{diskussion}
Unter einem {\em Fleissigen Biber} mit $n$ Zuständen versteht
man eine Turingmaschine mit $n$ Zuständen zusätzlich zu $a_{\text{accept}}$
und $q_{\text{reject}}$ und dem Bandalphabet $\Sigma=\{\texttt{1}\}$,
die auf einem leeren Band die mit dieser Anzahl von Zuständen maximal
mögliche Anzahl von Einsen schreibt und dann anhält.
Die Anzahl $k_n$ der geschriebenen Einsen definiert die Rad\'o-Funktion.
Die Turingmaschine in der Aufgabe ist ein Fleissiger Biber mit zwei Zuständen,
also ist $k_2=4$.
Für $n>4$ ist $k_n$ nicht exakt bekannt, aber es sind Abschätzungen bekannt.
Für $n\ge 7$ sind nicht einmal Abschätzungen bekannt.
\end{diskussion}

\begin{bewertung}
Zustandsdiagramm ({\bf Z}) 3 Punkte,
Berechnungsgeschichte ({\bf B}) 2 Punkte,
Resultat auf dem Band ({\bf R}) 1 Punkt.
\end{bewertung}
