Beschreiben Sie eine Turing-Maschine, welche entscheidet, ob die
Anzahl der $1$ in einem Wort $w\in\Sigma^*$ ($\Sigma=\{0,1\}$)
mindestens so gross ist wie die Anzahl der $0$.
Sie soll also im Zustand $q_{\text{accept}}$
genau dann anhalten, wenn $|w|_1\ge |w|_0$.

\thema{Turing-Maschine}

\begin{loesung}
Der Turing-Maschinen-Algorithmus muss folgendes implementieren
\begin{enumerate}
\item Fahre nach rechts bis zur ersten {\tt 0}, falls keine  {\tt 0}
gefunden $\to q_{\text{accept}}$
\item Ersetze die {\tt 0} durch ein {\tt x}
\item Fahre nach Rechts ans Ende des Bandes
\item Fahre nach Links bis zur ersten {\tt 1}
\item Ersetze die {\tt 1} durch {\tt x}, falls keine {\tt 1}
gefunden: $\to q_{\text{reject}}$.
\item Fahre nach links bis zum Anfang des Bandes
\item Weiter bei 1.
\qedhere
\end{enumerate}
\end{loesung}
