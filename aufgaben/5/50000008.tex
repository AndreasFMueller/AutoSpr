In Aufgabe~\ref{40000024} wurden Tupeladditionen untersucht.
Nehmen Sie jetzt an,
dass die $n$-Tupel nur einstellige Zahlen, also Ziffern, enthalten
dürfen. Beschreiben Sie eine Turing-Maschine, die Tupel-Additionen
überprüft, also genau solche Tupel-Additionsausdrücke akzeptiert,
die korrekt sind.

\begin{loesung}
Wir beschreiben eine Turing-Maschine mit zwei Bändern, die das
Problem löst. Als Alphabet verwenden wir die Zeichen '{\tt (}',
'{\tt ,}', '{\tt )}', '{\tt +}', '{\tt =}', die Ziffern '{\tt 0}' bis '{\tt 9}'
und ein Zeichen '{\tt x}' welches immer dann verwendet wird, wenn eine
Addition eine Resultat $\ge 10$ ergibt.

Zu Beginn steht das Additionsproblem auf dem einen Band der Turing Maschine.
Die Maschine liest jetzt auf dem ersten Band von links nach rechts,
und kopiert alle Ziffern auf das zweite Band, bis sie auf das '{\tt +}'
trifft. Dann fährt sie auf dem zweiten Band zurück zum Anfang.
Anschliessend liest sie auf dem ersten Band weiter. Jedesmal, wenn unter
dem Lesekopf des ersten Bandes eine Ziffer auftaucht, wird die Summe mit der
Ziffer unter dem Lesekopf des zweiten Bandes gebildet, und damit die Zelle
des zeiten Bandes überschrieben. Danach bewegen sich beide Köpfe nach
rechts. Dies wird wiederholt bis der Lesekopf des ersten Bandes auf das
'{\tt =}' trifft, in diesem Moment wird das Lesekopf des zweiten Bandes
wieder zum Anfang zurück gefahren. Jetzt werden die Ziffern des ersten
Bandes mit den Ziffern des zweiten Bandes verglichen. Akzeptiert wird
der Input, wenn alle die Ziffernfolgen immer gleich viele Element haben
und der Vergleich der Ziffernfolgen im letzten Teil keinen Unterschied ergibt.
In allen anderen Fällen wird der Input verworfen.
\end{loesung}
