Beschreiben Sie die Vorgehensweise einer Turing-Maschine, die Wörter
über dem Alphabet $\Sigma=\{\texttt{0},\texttt{1}\}$ wie folgt erkennt:
\begin{teilaufgaben}
\item Das Wort besteht nur aus Nullen
\item Das Wort enthält keine Nullen
\item Das Wort hat die Form $\texttt{0}^n\texttt{1}^m$, $n,m>0$
\end{teilaufgaben}

\begin{loesung}
\definecolor{darkred}{rgb}{0.8,0,0}
\definecolor{darkgreen}{rgb}{0,0.6,0}
\begin{teilaufgaben}
\item 
Der folgende Pseudocode arbeitet bis zum Ende des Wortes und verwirft
das Wort, sobald ein \texttt{1} gefunden wird:
\begin{enumerate}[1.]
\item Falls \texttt{1}: {\color{darkred}reject}
\item Solange \texttt{0}: ein Feld nach rechts
\item Falls \blank: {\color{darkgreen}accept}
\item Alle anderen Zeichen: {\color{darkred}reject}
\end{enumerate}
\item 
Dies ist die gleiche Aufgabe wir a) mit vertauschten Rollen von
\texttt{0} und \texttt{1}:
\begin{enumerate}[1.]
\item Falls \texttt{0}: {\color{darkred}reject}
\item Solange \texttt{1}: ein Feld nach rechts
\item Falls \blank: {\color{darkgreen}accept}
\item Alle anderen Zeichen: {\color{darkred}reject}
\end{enumerate}
\item
Zunächst muss geprüft werden, dass das Wort mit \texttt{0} beginnt,
dann können alle Nullen übersprungen werden.
Anschliessend muss mindestens ein \texttt{1} kommen, dann können alle
Einsen übersprüngen werden.
Dann muss das Wort zu Ende sein.
Diese Prüfungen führt der folgende Code durch:
\begin{enumerate}[1.]
\item Falls \texttt{1}: {\color{darkred}reject}
\item Solange \texttt{0}: ein Feld nach rechts
\item Falls \blank: {\color{darkred}reject}
\item Solange \texttt{1}: ein Feld nach rechts
\item Falls \blank: {\color{darkgreen}accept}
\item Alle anderen Zeichen: {\color{darkred}reject}
\end{enumerate}
\end{teilaufgaben}
\end{loesung}
