Betrachten Sie die Turing-Maschine mit Bandalphabet
$\Gamma=\{\texttt{0},\texttt{1},\blank\}$
mit dem Zustandsdiagramm
\[
\entrymodifiers={++[o][F]}
\xymatrix @+5mm {
*+\txt{}
	&*+\txt{}
		&{q_1} \ar[r]^{.\to.,L}
			&{q_2} \ar[d]^{.\to \texttt{0},R}
\\
*+\txt{} \ar[r]
	&{q_0} \ar[d]_{\blank\to\blank,L}
		\ar[ur]^{\texttt{0}\to\texttt{0},L}
		\ar[dr]^{\texttt{1}\to\texttt{0},L}
		&{q_6} \ar[l]_{.\to .,R}
			&{q_5} \ar[l]_{.\to .,R}
\\
*+\txt{}
	&{q_7} \ar[d]_{.\to\blank,L}
		&{q_3} \ar[r]_{.\to.,L}
			&{q_4} \ar[u]_{.\to \texttt{1},R}
\\
*+\txt{}
	&{q_8} \ar[r]^{.\to\blank,L}
		&*++[o][F=]{q_{\text{accept}}}
			&*++[o][F=]{q_{\text{reject}}}
}
\]
Punkte in den Übergangsregeln bedeuten, dass der Übergang unabhängig
vom aktuellen Bandzeichen erfolgt (Punkt links vom Pfeil $\to$) 
und das aktuell Bandzeichen nicht überschrieben wird (Punkt rechts vom
Pfeil).
\begin{teilaufgaben}
\item
Wie verarbeitet die Maschine das leere Wort $\varepsilon$?
\item
Wie verarbeitet die Maschine das Inputwort \texttt{01}?
\item
Was tut die Maschine für ein beliebiges Wort?
\item
Wo steht der Schreib-/Lesekopf am Ende der Berechnung?
\item
Wieviele Schritte braucht die Maschine, bis sie anhält?
\end{teilaufgaben}

\thema{Turing-Maschine}
\thema{Zustandsdiagramm}

\begin{loesung}
\begin{teilaufgaben}
\item
Für das leere Wort ist der erste Übergang $q_0\to q_7$. 
Da aber links von der Ausgangsposition nur Lehrzeichen auf dem Band
stehen, bewirken die Übergänge nichts ausser dass am Ende der Kopf
drei Zeichen weiter links stehen.
\item
Die Verarbeitung des Wortes \texttt{01} ändert den Bandinhalt wie folgt.
Die aktuelle Kopfposition ist durch rote Zeichen angedeutet.
\begin{center}
\def\b{\phantom{q_0}}
\def\r#1{\bgroup\color{red}\texttt{#1}\egroup}
\def\s#1{\texttt{#1}}
\def\R{\color{red}}
\begin{tabular}{>{$}c<{$}|
>{$}c<{$}
>{$}c<{$}
>{$}c<{$}
>{$}c<{$}
>{$}c<{$}
>{$}c<{$}|>{$}l<{$}}
\text{Zustand}&\multicolumn{6}{l|}{Band}&\text{Berechnungsgeschichte}\\
\hline
q_0              &\blank&\blank    &\blank    &\R\texttt{0}&\texttt{1}&\blank&\blank\blank\blank q_0\r{0}\s{1}\blank\\
q_1              &\blank&\blank    &\R\blank    &\texttt{0}&\texttt{1}&\blank&\blank\blank q_1\r{\blank}\s{01}\blank\\
q_2              &\blank&\R\blank    &\blank    &\texttt{0}&\texttt{1}&\blank&\blank q_2\r{\blank}\blank\s{01}\blank\\
q_5              &\blank&\texttt{0}&\R\blank    &\texttt{0}&\texttt{1}&\blank&\blank\s{0}q_5\r{\blank}\s{01}\blank\\
q_6              &\blank&\texttt{0}&\blank    &\R\texttt{0}&\texttt{1}&\blank&\blank\s{0}\blank q_0\r{0}\s{1}\blank\\
q_0              &\blank&\texttt{0}&\blank    &\texttt{0}&\R\texttt{1}&\blank&\blank\s{0}\blank\s{0}q_0\r{1}\blank\\
q_3              &\blank&\texttt{0}&\blank    &\R\texttt{0}&\texttt{1}&\blank&\blank\s{0}\blank q_3\r{0}\s{1}\blank\\
q_4              &\blank&\texttt{0}&\R\blank    &\texttt{0}&\texttt{1}&\blank&\blank\s{0}q_4\r{\blank}\s{01}\blank\\
q_5              &\blank&\texttt{0}&\texttt{1}&\R\texttt{0}&\texttt{1}&\blank&\blank\s{01}q_5\r{0}\s{1}\blank\\
q_6              &\blank&\texttt{0}&\texttt{1}&\texttt{0}&\R\texttt{1}&\blank&\blank\s{010}q_6\r{1}\blank\\
q_0              &\blank&\texttt{0}&\texttt{1}&\texttt{0}&\texttt{1}&\R\blank&\blank\s{0101}q_0\r{\blank}\\
q_7              &\blank&\texttt{0}&\texttt{1}&\texttt{0}&\R\texttt{1}&\blank&\blank\s{010}q_7\r{1}\blank\\
q_8              &\blank&\texttt{0}&\texttt{1}&\R\texttt{0}&\blank    &\blank&\blank\s{01}q_8\r{0}\blank\blank\\
q_{\text{accept}}&\blank&\texttt{0}&\R\texttt{1}&\blank    &\blank    &\blank&\blank\s{0}q_a\r{1}\blank\blank\blank\\
\hline
\end{tabular}
\end{center}
\item
Die Maschine kopiert das Inputwort um zwei Zellen nach links.
\item
Am Ende der Berechnung steht der Schreibe-/Lese-Kopf auf dem letzten Zeichen
des Wortes.
\item
Für jedes Zeichen durchläuft die Maschine entweder die Zustände
$a_0\to q_1\to q_2\to q_5\to q_6\to q_0$, falls das Zeichen \texttt{0} ist,
oder die Zustände
$a_0\to q_3\to q_4\to q_5\to q_6\to q_0$, falls das Zeichen \texttt{1} ist.
In jedem Fall werden genau 5 Zustandsübergänge ausgeführt.
Am Ende des Wortes werden die Übergänge 
$q_0\to q_7\to q_8\to q_{\text{reject}}$ ausgeführt, dies sind 3 Übergänge.
Insgesamt werden daher genau $5\cdot|w|+3$ Schritte durchgeführt werden.
\qedhere
\end{teilaufgaben}
\end{loesung}

\begin{bewertung}
\begin{teilaufgaben}
\item 1 Punkt ({\bf A})
\item 1 Punkt ({\bf B})
\item 1 Punkt ({\bf C})
\item 1 Punkt ({\bf D})
\item 2 Punkte ({\bf E})
\end{teilaufgaben}
\end{bewertung}

