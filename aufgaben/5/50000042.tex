Betrachten Sie die Turing-Maschine mit dem Bandalphabet
$\Gamma=\{\blank,\texttt{0},\texttt{1}\}$
und dem Zustandsdiagramm
\begin{center}
\def\l{3}
\begin{tikzpicture}[>=latex,thick]
\definecolor{darkred}{rgb}{0.8,0,0}
\coordinate (q0) at (0,0);
\coordinate (q1) at ({\l},0);
\coordinate (q2) at ({2*\l},0);
\coordinate (q3) at ({3*\l},0);
\coordinate (q4) at ({0.5*\l},{-sqrt(3)*\l/2});
\coordinate (qr) at ({1.5*\l},{-sqrt(3)*\l/2});
\coordinate (qa) at ({2.5*\l},{-sqrt(3)*\l/2});
\node at (q0) {$q_0$};
\node at (q1) {$q_1$};
\node at (q2) {$q_2$};
\node at (q3) {$q_3$};
\node at (qa) {$q_{\text{accept}}$};
\node at (qr) {$q_{\text{reject}}$};
\node at (q4) {$q_4$};
\draw (q0) circle[radius=0.35];
\draw (q1) circle[radius=0.35];
\draw (q2) circle[radius=0.35];
\draw (q3) circle[radius=0.35];
\draw (q4) circle[radius=0.35];
\draw (qa) ellipse(0.6cm and 0.3cm);
\draw (qa) ellipse(0.65cm and 0.35cm);
\draw (qr) ellipse(0.6cm and 0.3cm);
\draw (qr) ellipse(0.65cm and 0.35cm);
\draw[->,shorten >= 0.35cm] ($(q0)+(-1.5,0)$) -- (q0);
\draw[->,shorten <= 0.35cm,shorten >= 0.35cm] (q0) -- (q1);
\node at ($0.5*(q0)+0.5*(q1)$) [above] {$
\texttt{0}\to\texttt{0},R
\atop
\blank\to\texttt{0},R
$};
\draw[->,shorten <= 0.35cm,shorten >= 0.35cm] (q0) -- (q4);
\node at ($0.5*(q0)+0.5*(q4)$) [left] {$\scriptstyle\texttt{1}\to\texttt{1},L$};
\draw[->,shorten <= 0.35cm,shorten >= 0.35cm] (q4) -- (q1);
\node at ($0.5*(q1)+0.5*(q4)$) [right] {$\scriptstyle\blank\to\texttt{0},R$};
\draw[->,shorten <= 0.35cm,shorten >= 0.65cm] (q4) -- (qr);
\node at ($0.5*(q4)+0.5*(qr)$) [below] {$\texttt{0}\to\texttt{0},R
\atop
\ifthenelse{\boolean{loesungen}}{\color{darkred}
\texttt{1}\to\texttt{1},R
}{}
$};
\draw[->,shorten <= 0.35cm,shorten >= 0.35cm] (q1) -- (q2);
\draw[->,shorten <= 0.35cm,shorten >= 0.35cm] (q2) -- (q3);
\ifthenelse{\boolean{loesungen}}{
	\draw[->,shorten <= 0.35cm,shorten >= 0.35cm,color=darkred] (q2) -- (qr);
	\node[color=darkred] at ($0.5*(q2)+0.5*(qr)$)
		[right] {$\scriptstyle\blank\to\blank,R$};
}{}
\draw[->,shorten <= 0.35cm,shorten >= 0.35cm]
	(q1) to[out=60,in=120,distance=1cm] (q1);
\node at ($(q1)+(0,1.1)$) {$
	\texttt{0}\to\texttt{0},R
	\atop
	\texttt{1}\to\texttt{1},R$};
\node at ($0.5*(q1)+0.5*(q2)$) [above] {$\scriptstyle\blank\to\blank,L$};
\draw[->,shorten <= 0.35cm,shorten >= 0.35cm]
	(q2) to[out=60,in=120,distance=1cm] (q2);
\node at ($(q2)+(0,1)$) {$\scriptstyle\texttt{1}\to\texttt{1},L$};
\node at ($0.5*(q2)+0.5*(q3)$) [above] {$\scriptstyle\texttt{0}\to\texttt{1},R$};
\draw[->,shorten <= 0.35cm,shorten >= 0.35cm]
	(q3) to[out=60,in=120,distance=1cm] (q3);
\node at ($(q3)+(0,1)$) {$\scriptstyle\texttt{1}\to\texttt{0},R$};
\draw[->,shorten <= 0.35cm,shorten >= 0.35cm] (q3) -- (qa);
\node at ($0.5*(q3)+0.5*(qa)$) [right] {$\texttt{0}\to\texttt{0},R
\atop
\blank\to\blank,R$};
\end{tikzpicture}
\end{center}
\begin{teilaufgaben}
\item
Im Zustandsdiagramm fehlen einige Übergänge.
Ergänzen Sie wo nötig Übergänge, die den Bandinhalt nicht verändern und
in $q_{\text{reject}}$ enden.
\item
Wie lautet die Berechnungsgeschichte für die Verarbeitung des Inputworts
\texttt{11011}?
\item
Wird das Wort \texttt{11011} akzeptiert?
\item
Was steht nach der Verarbeitung des Wortes \texttt{11011} auf dem Band
und wo steht der Schreib-/Lesekopf?
\item
Interpretiert man das Wort auf dem Band als Binärzahl, implementiert
die Turing-Maschine eine arithmetische Operation.
Welche?
\end{teilaufgaben}

\begin{loesung}
\definecolor{darkred}{rgb}{0.8,0,0}
\begin{teilaufgaben}
\item Es fehlen die Übergänge
\begin{center}
\begin{tikzpicture}[>=latex,thick]
\coordinate (q4) at (-3,0);
\coordinate (qr) at (0,0);
\coordinate (q2) at (3,0);
\draw (q2) circle[radius=0.35cm];
\draw (q4) circle[radius=0.35cm];
\draw (qr) ellipse(0.6cm and 0.3cm);
\draw (qr) ellipse(0.65cm and 0.35cm);
\node at (q2) {$q_2$};
\node at (q4) {$q_4$};
\node at (qr) {$q_{\text{reject}}$};
\draw[->,shorten <= 0.35cm,shorten >= 0.65cm] (q2) -- (qr);
\node at ($0.6*(q4)+0.4*(qr)$) [above]
	{$\scriptstyle \texttt{1}\to\texttt{1},R$};
\draw[->,shorten <= 0.35cm,shorten >= 0.65cm] (q4) -- (qr);
\node at ($0.6*(q2)+0.4*(qr)$) [above]
	{$\scriptstyle \blank\to\blank,R$};
\end{tikzpicture}
\end{center}
die im Zustandsdiagramm in der Aufgabenstellung {\color{darkred}rot} ergänzt wurden.
\item
Die Berechnungsgeschichte lautet
\bgroup
\def\z#1{\hbox to0.4cm{\hfill\clap{$#1$}\hfill}}
\def\zb{\hbox to0.4cm{\hfill\clap{\blank}\hfill}}
\def\zt#1{\hbox to0.4cm{\hfill\clap{\texttt{#1}}\hfill}}
\begin{align*}
&\z{\blank} \zb     \z{q_0} \zt{1}  \zt{1}  \zt{0}  \zt{1}  \zt{1}  \zb  \zb\\
&\z{\blank} \z{q_4} \zb     \zt{1}  \zt{1}  \zt{0}  \zt{1}  \zt{1}  \zb  \zb\\
&\z{\blank} \zt{0}  \z{q_1} \zt{1}  \zt{1}  \zt{0}  \zt{1}  \zt{1}  \zb  \zb\\
&\z{\blank} \zt{0}  \zt{1}  \z{q_1} \zt{1}  \zt{0}  \zt{1}  \zt{1}  \zb  \zb\\
&\z{\blank} \zt{0}  \zt{1}  \zt{1}  \z{q_1} \zt{0}  \zt{1}  \zt{1}  \zb  \zb\\
&\z{\blank} \zt{0}  \zt{1}  \zt{1}  \zt{0}  \z{q_1} \zt{1}  \zt{1}  \zb  \zb\\
&\z{\blank} \zt{0}  \zt{1}  \zt{1}  \zt{0}  \zt{1}  \z{q_1} \zt{1}  \zb  \zb\\
&\z{\blank} \zt{0}  \zt{1}  \zt{1}  \zt{0}  \zt{1}  \zt{1}  \z{q_1} \zb  \zb\\
&\z{\blank} \zt{0}  \zt{1}  \zt{1}  \zt{0}  \zt{1}  \z{q_2} \zt{1}  \zb  \zb\\
&\z{\blank} \zt{0}  \zt{1}  \zt{1}  \zt{0}  \z{q_2} \zt{1}  \zt{1}  \zb  \zb\\
&\z{\blank} \zt{0}  \zt{1}  \zt{1}  \z{q_2} \zt{0}  \zt{1}  \zt{1}  \zb  \zb\\
&\z{\blank} \zt{0}  \zt{1}  \zt{1}  \zt{1}  \z{q_3} \zt{1}  \zt{1}  \zb  \zb\\
&\z{\blank} \zt{0}  \zt{1}  \zt{1}  \zt{1}  \zt{0}  \z{q_3} \zt{1}  \zb  \zb\\
&\z{\blank} \zt{0}  \zt{1}  \zt{1}  \zt{1}  \zt{0}  \zt{0}  \z{q_3} \zb  \zb\\
&\z{\blank} \zt{0}  \zt{1}  \zt{1}  \zt{1}  \zt{0}  \zt{0}  \zb  \z{q_a} \zb\\
\end{align*}
\egroup
\item
Da die Berechnungsgeschichte im Zustand $q_{\text{accept}}$ endet,
wird das Wort akzeptiert.
\item
Das Wort auf dem Band nach der Berechnung ist \texttt{011100}.
\item
Die Maschine sucht das letzte Zeichen \texttt{0}, ändert es in \texttt{1}
und setzt alle nachfolgenden Einsen auf \texttt{0}.
Als arithmetische Operation ist dies das Inkrement.
\qedhere
\end{teilaufgaben}
\end{loesung}

\begin{bewertung}
Fehlende Übergänge ({\bf U}) 1 Punkt,
Format der Berechnungsgeschichte  ({\bf F}) 1 Punkt,
Korrektheit ({\bf K}) 1 Punkt,
Wort wird akzeptiert ({\bf A}) 1 Punkt,
Resultat der Berechnung ({\bf R}) 1 Punkt,
Inkrement-Operation ({\bf I}) 1 Punkt.
\end{bewertung}
