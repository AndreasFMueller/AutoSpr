Betrachten Sie die Turing-Machine mit dem Bandalphabet
$\Gamma=\{\texttt{0},\texttt{1},\blank\}$
mit dem Zustandsdiagramm
\[
\entrymodifiers={++[o][F]}
\xymatrix @+15mm {
*+\txt{} \ar[r]
	&{q_0}	\ar@(ur,ul)_{{\scriptstyle\texttt{0}\to\texttt{0},R}
			\atop{\scriptstyle\texttt{1}\to\texttt{1},R}}
		\ar[r]^{\blank\to\blank,L}
		&{q_1}	\ar[r]^{{\scriptstyle\texttt{0}\to\texttt{0},L}
			\atop {\scriptstyle\texttt{1}\to\texttt{1},L}}
			\ar[r]_{\blank\to\texttt{0},L}
			&{q_2} \ar@(r,u)_{\texttt{1}\to\texttt{0},L}
				\ar[d]^{{\scriptstyle\blank\to\texttt{1},L}
				    \atop{\scriptstyle\texttt{0}\to\texttt{1},L}}
\\
*+\txt{}
	&*++[o][F=]{q_{\text{reject}}}
		&*+\txt{}
			&*++[o][F=]{q_{\text{accept}}}
}
\]
\begin{teilaufgaben}
\item
Wie verarbeitet die Machine die Inputwörter $\varepsilon$,
\texttt{001} und \texttt{110}?
\item
Wenn die Inputwörter als Binärzahlen interpretiert werden, wie lässt sich
dann die Wirkung der Turing-Maschine als arithmetische Operation beschreiben?
\item
Hält diese Turing-Maschine immer an? Falls ja, wieviele Schritte
führt die Maschine höchstens aus, bis sie anhält?
\end{teilaufgaben}

\thema{Turing-Maschine}
\thema{Zustandsdiagramm}

\begin{loesung}
\begin{teilaufgaben}
\item
Die Resultate sind \texttt{10}, \texttt{011} und \texttt{1000} kann man
aus der Berechnungsgeschichte ablesen:
\begin{center}
\begin{tabular}{>{$}c<{$}|>{$}c<{$}>{$}c<{$}>{$}c<{$}>{$}c<{$}}
q_0&\blank&\blank&\blank&\color{red}\blank\\
q_1&\blank&\blank&\color{red}\blank&\blank\\
q_2&\blank&\color{red}\blank&\texttt{0}&\blank\\
q_{\text{accept}}&\color{red}\blank&\texttt{1}&\texttt{0}&\blank\\
\strut&&&\\
\strut&&&\\
\strut&&&\\
\strut&&&\\
\strut&&&\\
\end{tabular}
\qquad
\begin{tabular}{>{$}c<{$}|
>{$}c<{$}
>{$}c<{$}
>{$}c<{$}
>{$}c<{$}
>{$}c<{$}}
q_0&\blank&\color{red}\texttt{0}&\texttt{0}&\texttt{1}&\blank\\
q_0&\blank&\texttt{0}&\color{red}\texttt{0}&\texttt{1}&\blank\\
q_0&\blank&\texttt{0}&\texttt{0}&\color{red}\texttt{1}&\blank\\
q_0&\blank&\texttt{0}&\texttt{0}&\texttt{1}&\color{red}\blank\\
q_1&\blank&\texttt{0}&\texttt{0}&\color{red}\texttt{1}&\blank\\
q_2&\blank&\texttt{0}&\color{red}\texttt{0}&\texttt{1}&\blank\\
q_{\text{accept}}&\blank&\color{red}\texttt{0}&\texttt{1}&\texttt{1}&\blank\\
\strut&&&&&\\
\strut&&&&&\\
\end{tabular}
\qquad
\begin{tabular}{>{$}c<{$}|
>{$}c<{$}
>{$}c<{$}
>{$}c<{$}
>{$}c<{$}
>{$}c<{$}
>{$}c<{$}}
q_0&\blank&\blank&\color{red}\texttt{1}&\texttt{1}&\texttt{0}&\blank\\
q_0&\blank&\blank&\texttt{1}&\color{red}\texttt{1}&\texttt{0}&\blank\\
q_0&\blank&\blank&\texttt{1}&\texttt{1}&\color{red}\texttt{0}&\blank\\
q_0&\blank&\blank&\texttt{1}&\texttt{1}&\texttt{0}&\color{red}\blank\\
q_1&\blank&\blank&\texttt{1}&\texttt{1}&\color{red}\texttt{0}&\blank\\
q_2&\blank&\blank&\texttt{1}&\color{red}\texttt{1}&\texttt{0}&\blank\\
q_2&\blank&\blank&\color{red}\texttt{1}&\texttt{0}&\texttt{0}&\blank\\
q_2&\blank&\color{red}\blank&\texttt{0}&\texttt{0}&\texttt{0}&\blank\\
q_{\text{accept}}&\color{red}\blank&\texttt{1}&\texttt{0}&\texttt{0}&\texttt{0}&\blank\\
\end{tabular}
\end{center}
\item
Die Maschine funktioniert ganz ähnlich wie die Turing-Maschine aus den
Übungen, die eine Binärzahl inkrementiert hat.
Allerdings wird beim Übergang von $q_1$ zu $q_2$ die letzte Stelle
übersprungen bevor mit dem inkrementieren begonnen wird.
Dies bedeutet, dass die Turing-Maschine die Addition der Konstanten $2$ 
implementiert.
\item 
Im Zustand $q_0$ bewegt sich der Kopf immer nach rechts, bis er auf ein
Leerzeichen trifft.
Danach bewegt er sich immer nach links, im schlimmsten Fall wieder bis
er auf ein Leerzeichen trifft.
Nach maximal $2|w|+2$ Schritten hält die Maschine an.
\qedhere
\end{teilaufgaben}
\end{loesung}

\begin{bewertung}
\begin{teilaufgaben}
\item 1 Punkt für die Verarbeitung jedes Wortes,
\item Addition von 2 ({\bf A}) 1 Punkt,
\item TM hält ({\bf H}) 1 Punkt, Laufzeit ({\bf L}) 1 Punkt.
\end{teilaufgaben}
\end{bewertung}
