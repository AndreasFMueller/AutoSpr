Sei $\Sigma=\{{\tt 0},{\tt 1},{\tt 2}\}$.
\[
L=\{
w\in\Sigma^*
\,|\,
|w|_{\tt 0} \le
|w|_{\tt 1} \le
|w|_{\tt 2}
\}
\]
Beschreiben Sie einen Turing-Maschinen-Algorithmus, mit dem man W"orter der
Sprache $L$ erkennen kann.

\ifthenelse{\boolean{loesungen}}{
\begin{loesung}
Der Algorithmus k"onnte so vorgehen:
\begin{enumerate}
\item Gehe von links nach rechts durch das Band bis Du eine
{\tt 0} findest, ersetze diese durch {\tt x} und gehe zur"uck
zum Anfang. Falls keine {\tt 0} gefunden wurde, fahre weiter bei
\ref{einer}
\item
Gehe von links nach rechts durch das Band bis Du eine
{\tt 1} findest, ersetze diese durch {\tt y}. Falls keine {\tt 1}
gefunden, verwerfe das Wort. Fahre zur"uck zum Anfang des Bandes.
\item Fahre von links nach rechts durch das Band bis du eine
{\tt 2} findest, ersetze diese durch {\tt z}. Falls es keine {\tt 2}
gibt, verwerfe das Wort. Sonst fahre weiter bei 1.
\item\label{einer}
Gehe von links nach rechts durch das Band bis Du eine
{\tt 1} findest, ersetze diese durch {\tt y}. Falls keine {\tt 1}
gefunden, akzeptiere das Wort. Fahre zur"uck zum Anfang des Bandes.
\item Fahre von links nach rechts durch das Band bis du eine
{\tt 2} findest, ersetze diese durch {\tt z}. Falls es keine {\tt 2}
gibt, verwerfe das Wort. Sonst fahre weiter bei \ref{einer}.
\end{enumerate}
Schritte 1--3 stellen sicher, dass es zu jeder {\tt 0} auch mindestens
eine {\tt 1} und ein {\tt 2} gibt. Dies w"urde aber auch W"orter wie
{\tt 112} akzeptabel machen, daher kontrollieren die Schritte 4 und 5,
ob zu jeder verbleibenden {\tt 1} auch mindestens eine {\tt 2}
vorhanden ist.

Ein alternativer Algorithmus verwendet drei Hilfsb"ander.  Die {\tt 0}
werden auf Hilfsband $0$ kopiert, die {\tt 1} auf Hilfsband $1$ und
die {\tt 2} auf Hilfsband 2. Sobald das Wort gelesen ist, werden
die K"opfe auf den Hilfsb"andern jeweils gemeinsam um eine Stelle nach
links gefahren, bis einer der K"opfe in der Anfangsposition ist.
Falls der Kopf von Hilfsband $0$ nicht in der Anfangsposition ist,
wird das Wort verworfen. Dann werden die K"opfe der Hilfsb"ander $1$
und $2$ gemeinsam nach links gefahren, bis einer der beiden auf dem
Wortanfang steht. Steht der Kopf von Hilfsband $1$ nicht in der
Anfangsposition, wird das Wort verworfen. In allen anderen F"allen
wird das Wort akzeptiert.
\end{loesung}
}{ }

