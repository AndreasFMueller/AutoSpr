Ersetzt man in einem XML-Dokument alle öffnenden Elemente durch {\tt <},
alle schliessenden Elemente durch {\tt >} und alle leeren Elemente
durch {\tt /}, und lässt alles andere weg, bleibt ein Wort über
dem Alphabet $\Sigma =\{{\tt <},{\tt >},{\tt /}\}$ übrig.
Diese Sprache, im Folgenden mit $L$ bezeichnet, wurde in Aufgabe~\ref{40000013}
als nicht regulär erkannt.
\begin{teilaufgaben}
\item
Beschreiben Sie einen Turing-Maschinen-Algorithmus, welcher genau die
Wörter der Sprache $L$ erkennt.
\item
Welche Komplexität hat ihr Algorithmus auf einer Standard-Turing-Maschine?
\end{teilaufgaben}

\begin{hinweis}
Es darf in a) auch eine erweiterte Turing-Maschine
mit mehreren Bändern oder mehreren Spuren verwendet werden.
\end{hinweis}

\thema{Turing-Maschine}
\thema{Komplexität}

\begin{loesung}
\begin{teilaufgaben}
\item
Die Sprache besteht aus Wörtern über dem Alphabet 
$\Sigma =\{{\tt <},{\tt >},{\tt /}\}$, die so aufgebaut sind,
dass die `spitzen' Klammern `{\tt <}' und `{\tt >}' korrekt
geschachtelt sind. Diese Sprache ist kontextfrei (es wäre leicht,
dafür eine Grammatik anzugeben, wir brauchen das aber nicht), 
kann also sogar mit einem Stackautomaten analysiert werden.
Das Zustandsdiagramm
\[
\entrymodifiers={++[o][F]}
\xymatrix{
*+\txt{}\ar[r]
        &{}\ar[r]^{\varepsilon,\varepsilon\to {\tt \$}}
                &{}\ar@(ul,u)^{{\tt <},\varepsilon\to{\tt <}}
                    \ar@(ur,u)_{{\tt >},{\tt <}\to\varepsilon}
                    \ar@(dl,dr)_{{\tt /},\varepsilon\to\varepsilon}
                    \ar[r]^{\varepsilon,{\tt\$}\to\varepsilon}
                        &*++[o][F=]{}
}
\]
beschreibt diesen Stackautomaten.
Wir bauen daher eine Turing-Maschine, die einen Stackautomaten
simuliert. Dazu verwenden wir eine TM mit zwei Bändern, wobei das zweite
Band die Funktion des Stack übernimmt.

Zu Beginn steht also das zu untersuchende Wort auf Band~1,
der Schreib-/Lesekopf steht auf dem ersten Zeichen.
Das zweite Band ist leer.
{
\renewcommand{\theenumii}{\arabic{enumii}}
\renewcommand{\labelenumii}{\theenumii.}
\begin{enumerate}
\item Falls das aktuelle Zeichen auf Band 1 ein `{\tt <}' ist,
schreibe es auf Band zwei, und bewege beide Schreib-/Leseköpfe ein
Feld nach rechts. Fahre weiter bei 1.
\item Falls das aktuelle Zeichen auf Band 1 ein `{\tt >}' ist,
und das aktuelle Zeichen auf Band 2 ein `{\tt <}', überschreibe
das Zeichen auf Band 2 mit \textvisiblespace und bewege den Kopf
nach links. Der Schreib-/Lesekopf von Band 1 wird nach rechts bewegt.
Fahre weietr bei 1.
\item Falls das aktuelle Zeichen auf Band 1 ein `{\tt /}' ist, bewege
den Kopf nach rechts.
Fahre weiter bei 1.
\item Falls das aktuelle Zeichen auf Band 1 ein \textvisiblespace{} ist,
akzeptiere.
\item In allen anderen Fällen verwerfe.
\end{enumerate}
}
\item
Der oben beschriebene Algorithmus hat Komplexität $O(n)$ auf einer 
Turing-Maschine mit zwei Bändern. Nach einem Satz aus der Vorlesung
kann dieser Algorithmus in einen für eine Standard-Turing-Maschine
umgewandelt werden, wobei die Laufzeit quadriert wird. Der für 
Standardturingmaschinen konvertierte Algorithmus hätte also die
Laufzeit $O(n^2)$.
\qedhere
\end{teilaufgaben}
\end{loesung}

