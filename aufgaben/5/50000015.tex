Die Sprache $L$ "uber dem Alphabet $\Sigma = \{{\tt 0},\;{\tt 1}\}$
besteht aus denjenigen W"ortern, in denen jede Folge von {\tt 0}
von einer mindestens so langen Folge von {\tt 1} gefolgt wird.
Beschreiben Sie eine Standard-Turing-Maschinen, welche $L$
entscheidet.

\begin{loesung}
Eine solche Turing-Maschine muss wiederholt Sequenzen der
Form ${\tt 0}^n{\tt 1}^m$ analysieren und kontrollieren, ob $m \ge n$ ist,
bis sie auf ein Zeichen \blank\ trifft, welches das Ende des
Wortes anzeigt. Da wir das Ende solcher Sequenzen markieren 
k"onnen m"ussen, verwenden wir ein zus"atzliches Bandzeichen
{\tt x}, das Bandalphabet ist also $\Gamma
=\{{\tt 0},\,{\tt 1},\,{\tt x}\}$.

Die Idee des folgenden Algorithmus ist, in jeder Sequenz jeweils
die {\tt 1} und {\tt 0} paarweise zu markieren, bis klar ist,
ob die Sequenz die Bedingung $m\ge n$ erf"ullt.

Der folgende Algorithmus tut dies:
\begin{enumerate}
\item
\label{50000015:l0}
Falls das Zeichen unter dem Schreib-Lesekopf ein \blank\ ist,
terminiere mit $q_{\text{accept}}$.
\item
\label{50000015:l1}
Fahre nach rechts, "uberspringe alle {\tt 0} oder {\tt 1}.
Nach dem letzten Zeichen bewege den Kopf nach links.
\item Falls das aktuelle Zeichen ein {\tt 0} ist, terminiere mit
$q_{\text{reject}}$.
\item Falls das aktuelle Zeichen ein {\tt 1} ist, "uberschreibe es
mit einem {\tt x} und bewege den Kopf nach links.
\item Fahre nach links, "uberspringe alle {\tt 1}.
\item Fahre nach links, "uberspringe alle {\tt 0},
dann bewege den Kopf nach rechts.
\item Falls das aktuelle Zeichen ein {\tt 0} ist, "uberschreibe es mit
{\tt x}, bewege den Kopf nach rechts und fahre weiter bei \ref{50000015:l1}.
\item Falls das aktuelle Zeichen ein {\tt 1} ist, gibt es offenbar in
diesem Teil des Wortes keine {\tt 0} mehr, dieser Teil des Wortes
erf"ullt also die gestellten Bedingungen. Fahre nach rechts, "uberspringe
alle {\tt 1} und alle {\tt x} und fahre weiter bei \ref{50000015:l0}.
\end{enumerate}
\end{loesung}

\begin{bewertung}
Verwendung eines Hilfszeichens zur Unterteilung in Sequenzen
und zum Z"ahlen ({\bf H}) 1 Punkt,
Analyse jeder einzelnen Sequenz ({\bf S}) 1 Punkt,
Paarweises abz"ahlen der {\tt 0} und {\tt 1} ({\bf P}) 1 Punkt,
Abbruchbedingung f"ur $q_{\text{accept}}$ ({\bf A}) 1 Punkt,
Abbruchbedingung f"ur $q_{\text{reject}}$ ({\bf R}) 1 Punkt,
Weiterfahr-Bedingungung f"ur n"achste Sequenz ({\bf W}) 1 Punkt.
\end{bewertung}


