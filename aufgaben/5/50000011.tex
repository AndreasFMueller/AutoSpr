Sei $\Sigma=\{{\tt 0},{\tt 1}\}$. In der Vorlesung wurde gezeigt, dass 
die Menge aller Wörter $\Sigma^*$ abzählbar ist, die Menge $P(\Sigma^*)$
aller Sprachen aber nicht. Zeigen Sie, dass die Menge aller endlichen
Sprachen abzählbar ist.

\themaL{abzahlbar}{abzählbar}

\begin{loesung}
Man kann eine Aufzählung der endlichen Teilmengen von $\Sigma^*$ wie
folgt konstruieren.
\begin{compactenum}
\item Die Menge der Sprachen mit genau einem Wort ist abzählbar, da die
Menge $\Sigma^*$ der Wörter abzählbar ist.
\item Die  Menge der Sprachen mit genau zwei Wörtern ist abzählbar,
sie ist eine Menge von Paaren von Wörtern, aber $\Sigma^*\times \Sigma^*$
ist abzählbar.
\item Die Menge der Sprachen mit genau drei Wörtern ist abzählbar,
sie besteht aus Paaren bestehend aus einer Sprache aus dem ersten
Schritt und einer Sprache aus dem zweiten Schritt, es gibt also 
abzählbar viele davon.
\item Auf die gleiche Art sieht man, dass die Menge der Sprachen mit
genau $n$ Wörtern abzählbar ist.
\end{compactenum}
Die Sprachen mit $n$ Wörtern bilden also jeweils eine abzählbare Menge,
sie lassen sich also mit zwei Zahlen eindeutig identifizieren: der Anzahl
Wörter $n$ und der Nummer $m$ innerhalb Sprachen mit $n$ Wörtern.
Die Menge aller Paare ist aber wieder abzählbar.
\end{loesung}

