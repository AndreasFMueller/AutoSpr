Betrachten Sie die Turing-Maschine mit Input-Alphabet
$\{\texttt{0},\texttt{1}\}$ und mit dem folgenden Zustandsdiagramm
\[
\entrymodifiers={++[o][F]}
\xymatrix @+5mm {
*+\txt{} 
	&*+\txt{} 
		&{q_1}	\ar[dr]_{\blank\to\blank,L}
			\ar[drrr]^{\texttt{1}\to\texttt{1},R}
			\ar@(ur,ul)_{\texttt{0}\to\texttt{0},R}
\\
*+\txt{} \ar[r]
	&{q_0}	\ar[dl]_{\blank\to\blank,R}
		\ar[ur]^{\texttt{0}\to\texttt{0},R}
		\ar[dr]_{\texttt{1}\to\texttt{1},R}
		&*+\txt{}
			&{q_3}	\ar@(ur,dr)^{\texttt{0}\to\texttt{1},L
					\atop \texttt{1}\to\texttt{0},L}
				\ar[ll]^{\blank\to\blank,R}
				&*+\txt{}
					&*++[o][F=]{q_{\text{accept}}}
\\
*++[o][F=]{q_{\text{reject}}}
	&*+\txt{}
		&{q_2}	\ar[ur]^{\blank\to\blank,L}
			\ar[urrr]_{\texttt{0}\to\texttt{0},R}
			\ar@(dr,dl)^{\texttt{1}\to\texttt{1},R}
}
\]
\begin{teilaufgaben}
\item
Welche W"orter werden von dieser Maschine verworfen?
\item
Finden sie die zwei k"urzesten W"orter, die von dieser Maschine
akzeptiert werden.
\item
Gibt es W"orter, auf denen die Maschine nicht anh"alt?
\item
Ist die von dieser Maschine akzeptierte Sprache regul"ar?
\end{teilaufgaben}

\begin{loesung}
\begin{teilaufgaben}
\item
Das leere Wort f"uhrt die Maschine unmittelbar in den Zustand
$q_{\text{reject}}$.
\item
Der k"urzeste Weg zum Zustand $q_{\text{accept}}$ f"uhrt auf
direktem Weg von $q_0$ "uber $q_1$ oder $q_2$ nach $q_{\text{accept}}$.
Er wird genommen, wenn auf dem Band die W"orter \texttt{01} bzw.~\texttt{10}
stehen.
Insbesondere folgt auch, dass ein Wort nur akzeptiert werden kann, wenn es
verschiedene Zeichen enth"alt.
\item
Die Maschine kann nur dann nicht anhalten, wenn sie den Weg "uber den
Zustand $q_3$ zur"uck zum Zustand $q_0$ nimmt.
Dazu darf sie von $q_1$ aus niemals ein Zeichen \texttt{1} treffen
oder von $q_2$ aus niemals ein Zeichen \texttt{0}, in beiden
F"allen besteht das Wort also nur aus einer Art Zeichen.
Im Zustand $q_3$ werden diese Zeichen durch das entgegengesetzte Zeichen
ersetzt, so dass bei der Ankunft im Zustand $q_0$ ein urspr"unglich aus
Einsen bestehendes Wort durch ein gleichlanges Wort aus Nullen ersetzt 
worden ist und umgekehrt.
Auf W"orter aus nur einer Art von Zeichen terminiert die Maschine also nicht.
\item
Die Sprache akzeptiert genau die W"orter, die verschiedene Zeichen enthalten.
Nicht akzeptiert werden die W"orter die auf den regul"aren Ausdruck
\texttt{0*|1*} passen.
Da das Komplement einer regul"aren Sprache wieder regul"ar ist, ist
die akzeptierte Sprache ebenfalss regul"ar.
\qedhere
\end{teilaufgaben}
\end{loesung}


\begin{bewertung}
Leeres Wort ({\bf E}) 1 Punkt,
K"urzeste W"orter ({\bf K}) 1 Punkt,
Begr"undung f"ur W"orter, auf denen die Maschine nicht anh"alt ({\bf B})
1 Punkt,
Beschreibung der W"orter, auf denen die Sprache nicht h"alt ({\bf H}) 1 Punkt,
Sprache der Maschine ({\bf S}) 1 Punkt,
Regularit"at ({\bf R}) 1 Punkt.
\end{bewertung}

