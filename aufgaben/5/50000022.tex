Betrachten Sie die Turing-Maschine mit Input-Alphabet
$\{\texttt{0},\texttt{1}\}$ und mit dem folgenden Zustandsdiagramm
\[
\entrymodifiers={++[o][F]}
\xymatrix @+5mm {
*+\txt{} 
	&*+\txt{} 
		&{q_1}	\ar[dr]_{\blank\to\blank,L}
			\ar[drrr]^{\texttt{1}\to\texttt{1},R}
			\ar@(ur,ul)_{\texttt{0}\to\texttt{0},R}
\\
*+\txt{} \ar[r]
	&{q_0}	\ar[dl]_{\blank\to\blank,R}
		\ar[ur]^{\texttt{0}\to\texttt{0},R}
		\ar[dr]_{\texttt{1}\to\texttt{1},R}
		&*+\txt{}
			&{q_3}	\ar@(ur,dr)^{\texttt{0}\to\texttt{1},L
					\atop \texttt{1}\to\texttt{0},L}
				\ar[ll]^{\blank\to\blank,R}
				&*+\txt{}
					&*++[o][F=]{q_{\text{accept}}}
\\
*++[o][F=]{q_{\text{reject}}}
	&*+\txt{}
		&{q_2}	\ar[ur]^{\blank\to\blank,L}
			\ar[urrr]_{\texttt{0}\to\texttt{0},R}
			\ar@(dr,dl)^{\texttt{1}\to\texttt{1},R}
}
\]
\begin{teilaufgaben}
\item
Welche Wörter werden von dieser Maschine verworfen?
\item
Finden sie die zwei kürzesten Wörter, die von dieser Maschine
akzeptiert werden.
\item
Gibt es Wörter, auf denen die Maschine nicht anhält?
\item
Ist die von dieser Maschine akzeptierte Sprache regulär?
\end{teilaufgaben}

\begin{loesung}
\begin{teilaufgaben}
\item
Das leere Wort führt die Maschine unmittelbar in den Zustand
$q_{\text{reject}}$.
\item
Der kürzeste Weg zum Zustand $q_{\text{accept}}$ führt auf
direktem Weg von $q_0$ über $q_1$ oder $q_2$ nach $q_{\text{accept}}$.
Er wird genommen, wenn auf dem Band die Wörter \texttt{01} bzw.~\texttt{10}
stehen.
Insbesondere folgt auch, dass ein Wort nur akzeptiert werden kann, wenn es
verschiedene Zeichen enthält.
\item
Die Maschine kann nur dann nicht anhalten, wenn sie den Weg über den
Zustand $q_3$ zurück zum Zustand $q_0$ nimmt.
Dazu darf sie von $q_1$ aus niemals ein Zeichen \texttt{1} treffen
oder von $q_2$ aus niemals ein Zeichen \texttt{0}, in beiden
Fällen besteht das Wort also nur aus einer Art Zeichen.
Im Zustand $q_3$ werden diese Zeichen durch das entgegengesetzte Zeichen
ersetzt, so dass bei der Ankunft im Zustand $q_0$ ein ursprünglich aus
Einsen bestehendes Wort durch ein gleichlanges Wort aus Nullen ersetzt 
worden ist und umgekehrt.
Auf Wörter aus nur einer Art von Zeichen terminiert die Maschine also nicht.
\item
Die Sprache akzeptiert genau die Wörter, die verschiedene Zeichen enthalten.
Nicht akzeptiert werden die Wörter die auf den regulären Ausdruck
\texttt{0*|1*} passen.
Da das Komplement einer regulären Sprache wieder regulär ist, ist
die akzeptierte Sprache ebenfalss regulär.
\qedhere
\end{teilaufgaben}
\end{loesung}


\begin{bewertung}
Leeres Wort ({\bf E}) 1 Punkt,
Kürzeste Wörter ({\bf K}) 1 Punkt,
Begründung für Wörter, auf denen die Maschine nicht anhält ({\bf B})
1 Punkt,
Beschreibung der Wörter, auf denen die Sprache nicht hält ({\bf H}) 1 Punkt,
Sprache der Maschine ({\bf S}) 1 Punkt,
Regularität ({\bf R}) 1 Punkt.
\end{bewertung}

