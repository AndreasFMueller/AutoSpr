Betrachten Sie die Turing-Maschine mit Bandalphabet
$\Gamma=\{\texttt{0},\texttt{1},\blank\}$
und Zustandsdiagramm
\[
\entrymodifiers={++[o][F]}
\xymatrix @+5mm {
*+\txt{}\ar[r]
	&{q_0}	\ar@/_/[r]_{\texttt{0}\to \texttt{0},R}
		\ar@(ur,ul)_{\texttt{1}\to\texttt{1},R}
		\ar[d]_{\blank\to\blank,L}
		&{q_1}	\ar@/_/[l]_{\texttt{0}\to\texttt{0},R}
			\ar@(ur,ul)_{\texttt{1}\to\texttt{1},R}
			\ar[d]^{\blank\to\blank,L}
\\
*++[o][F=]{q_{\text{accept}}}
	&{q_2}	\ar@/_/[r]_{\texttt{1}\to \texttt{1},L}
		\ar@(dr,dl)^{\texttt{0}\to\texttt{0},L}
		\ar[l]_{\blank\to\blank,L}
		&{q_3}	\ar@/_/[l]_{\texttt{1}\to \texttt{1},L}
			\ar@(dr,dl)^{\texttt{0}\to\texttt{0},L}
			\ar[r]^{\blank\to\blank,L}
			&*++[o][F=]{q_{\text{reject}}}
}
\]
\begin{teilaufgaben}
\item
Wie verarbeitet die Maschine das Wort \texttt{010}?
\item
Welche der Wörter
\texttt{0110},
\texttt{10100},
\texttt{100100},
\texttt{101111}
und
\texttt{1101000}
werden akzeptiert?
\item
Welche Sprache wird von der Turing-Maschine akzeptiert?
\end{teilaufgaben}

\thema{Turing-Maschine}
\thema{Zustandsdiagramm}

\begin{loesung}
\begin{teilaufgaben}
\item
Die folgende Tabelle beschreibt die Berechnung
\begin{center}
\begin{tabular}{>{$}r<{$}|>{$}l<{$}>{$}l<{$}>{$}l<{$}>{$}l<{$}>{$}l<{$}}
q_0 & \blank & \color{red}\texttt{0} & \texttt{1} & \texttt{0} & \blank \\
q_1 & \blank & \texttt{0} & \color{red}\texttt{1} & \texttt{0} & \blank \\
q_1 & \blank & \texttt{0} & \texttt{1} & \color{red}\texttt{0} & \blank \\
q_0 & \blank & \texttt{0} & \texttt{1} & \texttt{0} & \color{red}\blank \\
q_2 & \blank & \texttt{0} & \texttt{1} & \color{red}\texttt{0} & \blank \\
q_2 & \blank & \texttt{0} & \color{red}\texttt{1} & \texttt{0} & \blank \\
q_3 & \blank & \color{red}\texttt{0} & \texttt{1} & \texttt{0} & \blank \\
q_3 & \color{red}\blank & \texttt{0} & \texttt{1} & \texttt{0} & \blank \\
q_{\text{reject}} & &&&&
\end{tabular}
\end{center}
Das Wort \texttt{010} wird nicht akzeptiert.
\item
Wie in c) gezeigt wird, werden genau die Wörter akzeptiert, welche für
die die Anzahl Nullen und Einsen beide gerade oder beide ungerade sind:
\begin{align*}
\text{akzeptiert:}&
\text{
\texttt{0110},
\texttt{100100},
\texttt{101111},
}
\\
\text{nicht akzeptiert:}&
\text{
\texttt{10100},
\texttt{1101000}
}
\end{align*}
\item
Die oberen zwei Zustände durchlaufen das Wort von links nach rechts,
wobei der Zustand mit jeder gefundenen \texttt{0} wechselt.
Der Zustand $q_0$ steht also für eine gerade Anzahl gefundener Nullen,
$q_1$ für eine ungerade Anzahl.
Am Ende des Wortes wechselt die Maschine zu den unteren zwei Zuständen
und tut dort dasselbe für die Einsen.
Der Zustand $q_2$ steht dafür, dass noch eine gerade Anzahl von Einsen
erwartet wird, $q_3$ steht für eine ungerade Anzahl.
Wörter werden akzeptiert, wenn die Anzahl der Nullen und Einsen die 
gleiche Parität hat:
\[
L = \{ w\in\{\texttt{0},\texttt{1}\}^*\;|\;
|w|_{\texttt{0}}
\equiv
|w|_{\texttt{1}}
\mod 2\}.
\]
Wenn für ein Wort $|w|_{\texttt{0}}\equiv |w|_{\texttt{1}}\mod 2$ gilt, dann gilt
auch 
\[
|w|
=
|w|_{\texttt{0}} + |w|_{\texttt{1}} \equiv 0\mod 2,
\]
d.~h.~es werden genau die Wörter gerader Länge akzeptiert.
\qedhere
\end{teilaufgaben}
\end{loesung}

\begin{bewertung}
\begin{teilaufgaben}
\item Dokumentation der Berechnung ({\bf D}) 1 Punkt,
\item ({\bf B}) 3 Punkte,
\item Teillösung 1 Punkt, vollständige Lösung 2 Punkte.
\end{teilaufgaben}
\end{bewertung}


