Beschreiben Sie ein Turing-Maschinen-Programm, welches herausfindet,
ob eine Binärzahl durch drei teilbar ist.

\thema{Turing-Maschine}

\begin{loesung}
Für das Problem der Teilbarkeit durch drei wurde in der Vorlesung
ein endlicher Automat gezeigt:
\[
\entrymodifiers={++[o][F]}
\xymatrix @-1mm {
*+\txt{}\ar[r]
	&*++[o][F=]{0}	\ar@(ur,ul)_{\tt 0}
			\ar@/^/[r]^{\tt 1}
		&{1}	\ar@/^/[r]^{\tt 0}
			\ar@/^/[l]^{\tt 1}
			&{2}	\ar@/^/[l]^{\tt 0}
				\ar@(ur,ul)_{\tt 1}
}
\]
Die gesuchte Turing-Maschine muss also genau diesen endlichen Automaten
implementieren, was mit dem folgenden Zustandsdiagramm möglich ist:
\[
\entrymodifiers={++[o][F]}
\xymatrix @-1mm {
*+\txt{}\ar[r]
	&{0}	\ar@(ur,ul)_{{\tt 0}\to{\tt 0},\text{R}}
		\ar@/^/[r]^{{\tt 1}\to{\tt 1},\text{R}}
		\ar[d]_{\blank\to\blank,\text{R}}
		&{1}	\ar@/^/[r]^{{\tt 0}\to{\tt 0},\text{R}}
			\ar@/^/[l]^{{\tt 1}\to{\tt 1},\text{R}}
			\ar[dr]_{\blank\to\blank,\text{R}}
			&{2}	\ar@/^/[l]^{{\tt 0}\to{\tt 0},\text{R}}
				\ar@(ur,ul)_{{\tt 1}\to{\tt 1},\text{R}}
				\ar[d]^{\blank\to\blank,\text{R}}
\\
*+\txt{}
	&*++[o][F=]{q_{\text{accept}}}
		&*+\txt{}
			&*++[o][F=]{q_{\text{reject}}}
}
\]

Eine andere Lösung könnte man dadurch erhalten, dass man die TM
auf dem Band von der Zahl 3 (binär 11) subtrahieren lässt. Dies
kann nach dem Muster der binären Addition (``schriftliche'' Subtraktion,
Video aus der Vorlesung)
durch dreimaliges binäres Rückwärtszählen (Binärzähler aus den
"Ubungen) erfolgen.
Wenn
das aufgeht, also auf dem Band nur Nullen stehen, dann ist die
Zahl durch drei teilbar.

Noch ein weiterer Lösungsweg besteht darin, die Zahl zuerst auf einem zweiten
Band in unäre Darstellung umzuwandeln. Dazu braucht man ein drittes Band,
auf dem man die Zweierpotenzen in unärer Darstellung berechnet, indem
man den Inhalt des Bandes nochmals an dessen Ende kopiert. Die unäre
Darstellung des Inputs bekommt man, indem man für jede Stelle 1
des Inputs die zugehörige unäre Darstellung des zugehörigen Stellenwertes
ans Ende von Band~2 kopiert. Teilbarkeit durch drei kann man in der
unären Darstellung dadurch prüfen, dass man die Zeichen von Band~2
in Dreiergruppen löscht. Wenn das aufgeht, ist die Zahl durch drei teilbar
und kann akzeptiert werden.
\end{loesung}

\begin{bewertung}
Endlicher Automat für 3er-Rest ({\bf D}) 2 Punkte,
"Ubersetzung des Automaten in eine TM-Maschine: 4 Punkte.
Falls in einem TM-Programm eine nicht weiter spezifizierte Operation
``Division durch 3'' oder ``Modulo 3'' verwendet wurde, wurden
nur 3 Punkte vergeben.
\end{bewertung}
