\bgroup
\definecolor{wordlegreen}{rgb}{0.32,0.55,0.30}
\definecolor{wordleolive}{rgb}{0.71,0.62,0.23}
\def\schwarz#1{{\texttt{#1}}}
\def\gruen#1{{\color{wordlegreen}\texttt{#1}}}
\def\oliv#1{{\color{wordleolive}\texttt{#1}}}

Gegeben ist das Alphabet $\Sigma$ bestehend aus den Buchstaben
\texttt{A}--\texttt{Z}
in den Farben schwarz, grün und oliv, so wie sie im Spiel {\em Wordle}
verwendet werden (siehe auch Aufgabe~\ref{30000076}).
\begin{teilaufgaben}
\item
Zeichnen Sie das Zustandsdiagramm einer Turing-Maschine, die 
ein auf dem Band eingegebenes, fünf Zeichen langes Wort von links
nach rechts liest,
ähnlich wie im Spiel mit dem Wort \texttt{OCHSE}\footnote{Das Rätselwort im
offiziellen Wördl von \url{https://wordle.at} vom 31.~Mai 2022.}
vergleicht und dabei die Zeichen genau wie folgt modifiziert:
\begin{itemize}
\item
Hat ein Zeichen bereits eine der Farben grün oder oliv, wird der Input
verworfen.
\item
Ist das Zeichen richtig, soll es grün werden.
\item
Ist das Zeichen an dieser Stelle falsch, kommt aber an einer anderen Stelle im 
Wort vor, wird es oliv.
\end{itemize}
Falls das Wort auf dem Band nicht exakt fünf Zeichen hat, wird es verworfen.
\item
Schreiben Sie die Berechnungsgeschichte für das Eingabewort \texttt{WOCHE}
auf.
\item
Der in a) beschriebene Algorithmus weicht von dem tatsächlich verwendeten
Algorithmus ab (Aufgabe~\ref{30000076}).
Zeigen Sie den Unterschied mit Hilfe des Wortes \texttt{ECHSE}.
\end{teilaufgaben}

\begin{loesung}
Das Alphabet besteht aus den Zeichen
\[
\Sigma
=
\{
\texttt{A},\texttt{B},\dots,\texttt{Z},
{\color{wordlegreen}\texttt{A}},{\color{wordlegreen}\texttt{B}},
	\dots,{\color{wordlegreen}\texttt{Z}},
{\color{wordleolive}\texttt{A}},{\color{wordleolive}\texttt{B}},
	\dots,{\color{wordleolive}\texttt{Z}},
\blank
\}.
\]
\begin{teilaufgaben}
\item Das folgende Zustandsdiagramm löst die Aufgabe:
\begin{center}
\def\s{1.4}

\begin{tikzpicture}[>=latex,thick]

\coordinate (A1) at (0,0);
\coordinate (A2) at (2.2,0);
\coordinate (A3) at (4.4,0);
\coordinate (A4) at (6.6,0);
\coordinate (A5) at (8.8,0);
\coordinate (A6) at (11,0);
\coordinate (A) at (13.6,0);
\coordinate (R) at (4.4,-2.5);

\draw (A1) circle[radius=0.3]; \node at (A1) {$q_0$};
\draw (A2) circle[radius=0.3]; \node at (A2) {$q_1$};
\draw (A3) circle[radius=0.3]; \node at (A3) {$q_2$};
\draw (A4) circle[radius=0.3]; \node at (A4) {$q_3$};
\draw (A5) circle[radius=0.3]; \node at (A5) {$q_4$};
\draw (A6) circle[radius=0.3]; \node at (A6) {$q_5$};

\draw (A) ellipse(0.55 and 0.3);
\draw (A) ellipse(0.6 and 0.35);
\node at (A) {$q_\text{accept}$};

\draw (R) ellipse(0.5 and 0.3);
\draw (R) ellipse(0.55 and 0.35);
\node at (R) {$q_\text{reject}$};

\draw[->,shorten >= 0.3cm] (-1.5,0) -- (A1);
\draw[->,shorten >= 0.3cm,shorten <= 0.3cm] (A1) -- (A2);
\draw[->,shorten >= 0.3cm,shorten <= 0.3cm] (A2) -- (A3);
\draw[->,shorten >= 0.3cm,shorten <= 0.3cm] (A3) -- (A4);
\draw[->,shorten >= 0.3cm,shorten <= 0.3cm] (A4) -- (A5);
\draw[->,shorten >= 0.3cm,shorten <= 0.3cm] (A5) -- (A6);
\draw[->,shorten >= 0.6cm,shorten <= 0.3cm] (A6) -- (A);

\node at ($0.5*(A1)+0.5*(A2)+(0,0.3)$) {$\otimes$};
\node at ($0.5*(A1)+0.5*(A2)+(0,0.7)$)
	{$\texttt{E}\rightarrow{\color{wordleolive}\texttt{E}},R$};
\node at ($0.5*(A1)+0.5*(A2)+(0,1.1)$)
	{$\texttt{S}\rightarrow{\color{wordleolive}\texttt{S}},R$};
\node at ($0.5*(A1)+0.5*(A2)+(0,1.5)$)
	{$\texttt{H}\rightarrow{\color{wordleolive}\texttt{H}},R$};
\node at ($0.5*(A1)+0.5*(A2)+(0,1.9)$)
	{$\texttt{C}\rightarrow{\color{wordleolive}\texttt{C}},R$};
\node at ($0.5*(A1)+0.5*(A2)+(0,2.3)$)
	{$\texttt{O}\rightarrow{\color{wordlegreen}\texttt{O}},R$};

\node at ($0.5*(A2)+0.5*(A3)+(0,0.3)$) {$\otimes$};
\node at ($0.5*(A2)+0.5*(A3)+(0,0.7)$)
	{$\texttt{E}\rightarrow{\color{wordleolive}\texttt{E}},R$};
\node at ($0.5*(A2)+0.5*(A3)+(0,1.1)$)
	{$\texttt{S}\rightarrow{\color{wordleolive}\texttt{S}},R$};
\node at ($0.5*(A2)+0.5*(A3)+(0,1.5)$)
	{$\texttt{H}\rightarrow{\color{wordleolive}\texttt{H}},R$};
\node at ($0.5*(A2)+0.5*(A3)+(0,1.9)$)
	{$\texttt{C}\rightarrow{\color{wordlegreen}\texttt{C}},R$};
\node at ($0.5*(A2)+0.5*(A3)+(0,2.3)$)
	{$\texttt{O}\rightarrow{\color{wordleolive}\texttt{O}},R$};

\node at ($0.5*(A3)+0.5*(A4)+(0,0.3)$) {$\otimes$};
\node at ($0.5*(A3)+0.5*(A4)+(0,0.7)$)
	{$\texttt{E}\rightarrow{\color{wordleolive}\texttt{E}},R$};
\node at ($0.5*(A3)+0.5*(A4)+(0,1.1)$)
	{$\texttt{S}\rightarrow{\color{wordleolive}\texttt{S}},R$};
\node at ($0.5*(A3)+0.5*(A4)+(0,1.5)$)
	{$\texttt{H}\rightarrow{\color{wordlegreen}\texttt{H}},R$};
\node at ($0.5*(A3)+0.5*(A4)+(0,1.9)$)
	{$\texttt{C}\rightarrow{\color{wordleolive}\texttt{C}},R$};
\node at ($0.5*(A3)+0.5*(A4)+(0,2.3)$)
	{$\texttt{O}\rightarrow{\color{wordleolive}\texttt{O}},R$};

\node at ($0.5*(A4)+0.5*(A5)+(0,0.3)$) {$\otimes$};
\node at ($0.5*(A4)+0.5*(A5)+(0,0.7)$)
	{$\texttt{E}\rightarrow{\color{wordleolive}\texttt{E}},R$};
\node at ($0.5*(A4)+0.5*(A5)+(0,1.1)$)
	{$\texttt{S}\rightarrow{\color{wordlegreen}\texttt{S}},R$};
\node at ($0.5*(A4)+0.5*(A5)+(0,1.5)$)
	{$\texttt{H}\rightarrow{\color{wordleolive}\texttt{H}},R$};
\node at ($0.5*(A4)+0.5*(A5)+(0,1.9)$)
	{$\texttt{C}\rightarrow{\color{wordleolive}\texttt{C}},R$};
\node at ($0.5*(A4)+0.5*(A5)+(0,2.3)$)
	{$\texttt{O}\rightarrow{\color{wordleolive}\texttt{O}},R$};

\node at ($0.5*(A5)+0.5*(A6)+(0,0.3)$) {$\otimes$};
\node at ($0.5*(A5)+0.5*(A6)+(0,0.7)$)
	{$\texttt{E}\rightarrow{\color{wordlegreen}\texttt{E}},R$};
\node at ($0.5*(A5)+0.5*(A6)+(0,1.1)$)
	{$\texttt{S}\rightarrow{\color{wordleolive}\texttt{S}},R$};
\node at ($0.5*(A5)+0.5*(A6)+(0,1.5)$)
	{$\texttt{H}\rightarrow{\color{wordleolive}\texttt{H}},R$};
\node at ($0.5*(A5)+0.5*(A6)+(0,1.9)$)
	{$\texttt{C}\rightarrow{\color{wordleolive}\texttt{C}},R$};
\node at ($0.5*(A5)+0.5*(A6)+(0,2.3)$)
	{$\texttt{O}\rightarrow{\color{wordleolive}\texttt{O}},R$};

\node at ($0.5*(A6)+0.5*(A)+(-0.2,0.3)$) {$\blank\to\blank,R$};

\draw[->,shorten >= 0.45cm,shorten <= 0.3cm] (A1) -- (R);
\draw[->,shorten >= 0.4cm,shorten <= 0.3cm] (A2) -- (R);
\draw[->,shorten >= 0.3cm,shorten <= 0.3cm] (A3) -- (R);
\draw[->,shorten >= 0.4cm,shorten <= 0.3cm] (A4) -- (R);
\draw[->,shorten >= 0.45cm,shorten <= 0.3cm] (A5) -- (R);

\fill[color=gray!20] ($(A3)+(-4,-1.4)$) rectangle ($(A3)+(4,-1.0)$);
\node at ($(A3)+(0,-1.2)$) {$\blank \to \blank,R
\qquad
{\color{wordlegreen}x}\to{\color{wordlegreen}x},R
\qquad
{\color{wordleolive}x}\to{\color{wordleolive}x},R
$};

\draw[->,shorten <= 0.3cm,shorten >= 0.6cm] (A6) to[out=-90,in=0] (R);
\node at ($(A6)+(-0.3,-1.5)$) [right] {$\texttt{.}\to \texttt{.},R$};

\end{tikzpicture}
\end{center}
Dabei bedeutet das Zeichen $\otimes$ den Übergang $x\to x,R$ für jedes
schwarze Zeichen $x$, welches nicht in 
$\{
\texttt{O},
\texttt{C},
\texttt{H},
\texttt{S},
\texttt{E}
\}$ ist.
Im grauen Balken sind ausserdem mit ${\color{wordlegreen}x}$ und 
${\color{wordleolive}x}$ alle farbigen Zeichen gemeint.
Im Übergang vom Zustand $q_5$ in den Zustand $q_\text{reject}$
bedeutet der Punkt \texttt{.} alle Zeichen ausser dem \blank.
\item
Die Berechnungsgeschichte des Wortes \texttt{WOCHE} ist
\[
\begin{array}{ccccccc}
q_0        &\schwarz{W}&\schwarz{O}&\schwarz{C}&\schwarz{H}&\schwarz{E}&\blank\\
\schwarz{W}&q_1        &\schwarz{O}&\schwarz{C}&\schwarz{H}&\schwarz{E}&\blank\\
\schwarz{W}&\oliv{O}&q_2        &\schwarz{C}&\schwarz{H}&\schwarz{E}&\blank\\
\schwarz{W}&\oliv{O}&\oliv{C}&q_3        &\schwarz{H}&\schwarz{E}&\blank\\
\schwarz{W}&\oliv{O}&\oliv{C}&\oliv{H}&q_4        &\schwarz{E}&\blank\\
\schwarz{W}&\oliv{O}&\oliv{C}&\oliv{H}&\gruen{E}&q_5        &\blank\\
\schwarz{W}&\oliv{O}&\oliv{C}&\oliv{H}&\gruen{E}&\blank     &q_{\text{accept}}\\
\end{array}
\]
\item
Die Berechnungsgeschichte zur Verarbeitung des Wortes
\texttt{ECHSE} 
ist
\[
\begin{array}{ccccccc}
q_0        &\schwarz{E}&\schwarz{C}&\schwarz{H}&\schwarz{S}&\schwarz{E}&\blank\\
\oliv{E}&q_1        &\schwarz{C}&\schwarz{H}&\schwarz{S}&\schwarz{E}&\blank\\
\oliv{E}&\gruen{C}&q_2        &\schwarz{H}&\schwarz{S}&\schwarz{E}&\blank\\
\oliv{E}&\gruen{C}&\gruen{H}&q_3        &\schwarz{S}&\schwarz{E}&\blank\\
\oliv{E}&\gruen{C}&\gruen{H}&\gruen{S}&q_4        &\schwarz{E}&\blank\\
\oliv{E}&\gruen{C}&\gruen{H}&\gruen{S}&\gruen{E}&q_5        &\blank\\
\oliv{E}&\gruen{C}&\gruen{H}&\gruen{S}&\gruen{E}&\blank     &q_{\text{accept}}\\
\end{array}
\]
Dieses Resultat deutet im richtigen Wordle an, dass das Wort
zwei Zeichen \texttt{E} enthält, was aber hier nicht der Fall ist.
\qedhere
\end{teilaufgaben}
\end{loesung}

\begin{bewertung}
\begin{teilaufgaben}
\item
Zustandsdiagramm mit Farbübergängen ({\bf F}) 1 Punkt,
korrekt erkannter Länge ({\bf L}) 1 Punkt,
farbige Zeichen werden verworfen ({\bf V}) 1 Punkt,
nicht passende Zeichen bleiben schwarz ({\bf S}) 1 Punkt.
\item
Berechnungsgeschichte ({\bf B}) 1 Punkt.
\item
Abweichung ({\bf A}) 1 Punkt.
\end{teilaufgaben}
\end{bewertung}
\egroup
