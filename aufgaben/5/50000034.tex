Betrachten Sie die Turing-Maschine $M$ mit dem Zustandsdiagramm
\begin{center}
\begin{tikzpicture}[>=latex,thick]

\def\d{4}
\pgfmathparse{\d*sqrt(3)/2}
\xdef\ds{\pgfmathresult}

\coordinate (q0) at (0,0);
\coordinate (q1) at ({\d/2},{-\ds});
\coordinate (q2) at ({-\d/2},{-\ds});
\coordinate (accept) at (\d,0);
\coordinate (reject) at (0,{-2*\ds});

\node at (q0) {$q_0$};
\node at (q1) {$q_1$};
\node at (q2) {$q_2$};
\node at (accept) {$q_{\text{accept}}$};
\node at (reject) {$q_{\text{reject}}$};

\draw (q0) circle[radius=0.4];
\draw (q1) circle[radius=0.4];
\draw (q2) circle[radius=0.4];

\draw (accept) ellipse (0.7 and 0.4);
\draw (accept) ellipse (0.65 and 0.35);
\draw (reject) ellipse (0.7 and 0.4);
\draw (reject) ellipse (0.65 and 0.35);

\draw[->,shorten >= 0.4cm,shorten <= 0.4cm] (q0) to[out=-30,in=90] (q1);
\draw[->,shorten >= 0.4cm,shorten <= 0.4cm] (q1) to[out=-150,in=-30] (q2);
\draw[->,shorten >= 0.4cm,shorten <= 0.4cm] (q2) to[out=90,in=-150] (q0);

\draw[->,shorten >= 0.4cm,shorten <= 0.4cm] (q0) -- (q2);
\draw[->,shorten >= 0.4cm,shorten <= 0.4cm] (q2) -- (q1);
\draw[->,shorten >= 0.4cm,shorten <= 0.4cm] (q1) -- (q0);

\draw[->,shorten >= 0.7cm,shorten <= 0.4cm] (q0) -- (accept);
\draw[->,shorten >= 0.4cm,shorten <= 0.4cm] (q1) -- (reject);
\draw[->,shorten >= 0.4cm,shorten <= 0.4cm] (q2) -- (reject);

\draw[->,shorten >= 0.4cm] ($(q0)+(-2,0)$) -- (q0);

\node at ($0.5*(q0)+0.5*(accept)$) [above] {$\blank\to\blank,R$};

\node at ($0.5*(q1)+0.5*(q2)$) [above] {$\texttt{1}\to \texttt{1},R$};
\node at ($0.5*(q0)+0.5*(q1)$) [below,rotate=-60]
	{$\texttt{1}\to \texttt{1},R$};
\node at ($0.5*(q0)+0.5*(q2)$) [below,rotate=60]
	{$\texttt{1}\to \texttt{1},R$};

\node at ($0.5*(q0)+0.5*(q1)+(0.5,0.3)$)
	[right] {$\texttt{0}\to\texttt{0},R$};
\node at ($0.5*(q0)+0.5*(q2)+(-0.5,0.3)$)
	[left] {$\texttt{0}\to\texttt{0},R$};
\node at ($0.5*(q1)+0.5*(q2)+(0,-0.5)$)
	[below] {$\texttt{0}\to\texttt{0},R$};

\node at ($0.5*(reject)+0.5*(q2)$) [below,rotate=-60]
	{$\blank\to \blank,R$};
\node at ($0.5*(reject)+0.5*(q1)$) [below,rotate=60]
	{$\blank\to \blank,R$};

\end{tikzpicture}
\end{center}
\begin{teilaufgaben}
\item Wie wird das Wort \texttt{001011} verarbeitet?
\item Welche der Wörter 
\begin{center}
\texttt{01},
\texttt{01000},
\texttt{01000000},
\texttt{111},
\texttt{0111},
\texttt{000111},
\texttt{000111111},
\texttt{00010101111}
\end{center}
werden akzeptiert?
\item Welche Sprache wird von $M$ akzeptiert?
\end{teilaufgaben}

\thema{Turing-Maschine}
\thema{Zustandsdiagramm}

\begin{loesung}
\def\s#1{\texttt{#1}}
\def\r#1{{\color{red}\texttt{#1}}}
\def\brank{{\color{red}\blank}}
\begin{teilaufgaben}
\item  Die Berechnungsgeschichte für das Wort $w = \texttt{001011}$
ist
\begin{center}
\begin{tabular}{>{$}c<{$}|cccccccc}
\text{Zustand}&&&&&&&&\\
\hline
q_0& \blank & \r{0} & \s{0} & \s{1} & \s{0} & \s{1} & \s{1} & \blank \\
q_1& \blank & \s{0} & \r{0} & \s{1} & \s{0} & \s{1} & \s{1} & \blank \\
q_2& \blank & \s{0} & \s{0} & \r{1} & \s{0} & \s{1} & \s{1} & \blank \\
q_1& \blank & \s{0} & \s{0} & \s{1} & \r{0} & \s{1} & \s{1} & \blank \\
q_2& \blank & \s{0} & \s{0} & \s{1} & \s{0} & \r{1} & \s{1} & \blank \\
q_1& \blank & \s{0} & \s{0} & \s{1} & \s{0} & \s{1} & \r{1} & \blank \\
q_0& \blank & \s{0} & \s{0} & \s{1} & \s{0} & \s{1} & \s{1} & \brank \\
q_{\text{accept}} 
   & \blank & \s{0} & \s{0} & \s{1} & \s{0} & \s{1} & \s{1} & \blank \\
\hline
\end{tabular}
\end{center}
\item
Da nur Kopfverschiebungen nach rechts vorkommen, wird das Wort genau
einmal durchgelesen, die Maschine realisiert also eigentlich nur einen
endlichen Automaten.
Mit jedem \texttt{0} auf dem Band ändert sich der Zustand im oberen
Dreieck des Zustandsdiagramm im Urzeigersinn,
mit jedem \texttt{1} auf dem Band im Gegenuhrzeigersinn.
Akzeptiert wird, wenn der Zustand am Ende des Wortes $q_0$ ist.
Damit findet man schnell, dass
\begin{center}
\texttt{01},
\texttt{01000},
\texttt{01000000},
\texttt{111},
\texttt{000111}
\texttt{000111111}
\end{center}
akzeptiert und
\begin{center}
\texttt{0111},
\texttt{00010101111}
\end{center}
verworfen werden.
\item
Akzeptiert werden kann nur, wenn die Differenz der Anzahl der Nullen und
Einsen die Maschine am Ende des Wortes im Zustand $q_0$ lässt.
Die akzeptierte Sprache besteht daher aus den Wörtern, deren Anzahlen
von Nullen und Einsen gleichen Rest bei Teilung durch 3 haben,
also
\[
L = \{
w\in\Sigma^*
\;|\;
|w|_{\texttt{0}} \equiv |w|_{\texttt{1}} \mod 3
\}.
\qedhere
\]
\end{teilaufgaben}
\end{loesung}

\begin{bewertung}
\begin{teilaufgaben}
\item 
Berechnungsgeschichte ({\bf G}) 1 Punkt,
akzeptiert ({\bf A}) 1 Punkt.
\item
Drei Punkte ({\bf B}) 3 Punkte.
\item
Dreierrestbedingung ({\bf R}) 1 Punkt.
\end{teilaufgaben}
\end{bewertung}
