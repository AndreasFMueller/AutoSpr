Programmiersprachen der C-Familie verwenden Anführungszeichen
({\tt\textquotedblright}), um Zeichenketten zu markieren.
Wenn aber ein Anführungszeichen Teil der
Zeichenkette sein soll, wird es durch einen vorangestelltes {\tt\textbackslash}
gekennzeichnet.
Ebenso muss ein {\tt\textbackslash} durch einen vorangestellten {\tt\textbackslash}
markiert werden.
Natürlich muss zum öffnenden Anführungszeichen
auch immer ein schliessendes gehören. Beispiel:
\begin{center}
\tt
\textquotedblright dies ist ein String mit \textbackslash\textquotedblright eingebetteten\textbackslash\textquotedblright\ Anfuehrungszeichen\textquotedblright
\end{center}
\begin{teilaufgaben}
\item
Beschreiben Sie eine Turing-Maschinen, welche korrekt formattierte
Zeichenketten akzeptiert.
Visualisieren Sie die Turing-Maschine mit einem Zustandsdiagramm.
\item
Ist die Validierung von solchen Zeichenketten eventuell auch mit einer
einfacheren Maschine möglich?
\end{teilaufgaben}

\thema{Turing-Maschine}

\begin{hinweis}
Gehen Sie davon aus, dass das Zeichen \blank\ verschieden ist
vom Leerzeichen ({\tt ' '}, Hex {\tt 0x20}).
\end{hinweis}

\begin{loesung}
\begin{teilaufgaben}
\item
Die Turing-Maschine verwendet als Bandalphabet das ASCII-Alphabet 
sowie ein zusätzliches Zeichen \blank. Die Turing-Maschine implementiert
folgenden Algorithmus:
\begin{enumerate}
\renewcommand{\theenumii}{\arabic{enumii}}
\item Prüfe, ob das Zeichen ein {\tt \textquotedblright} ist.
Falls nicht: $q_{\text{reject}}$.
Falls ja, Kopf nach rechts bewegen. Damit wird erkannt, ob das erste
Zeichen ein öffnendes Anführungszeichen ist.
\item Prüfe, ob das Zeichen in {\tt \textquotedblright} ist.
Falls ja, Kopf nach rechts bewegen, weiter bei (5).
Wenn dieser Test zutrifft, liegt ein schliessendes Anführungszeichen
vor, die Zeichenkette müsste jetzt fertig sein, das wird in Schritt
(5) geprüft.
\item Prüfe, ob das Zeichen ein {\tt \textbackslash} ist. Falls
ja, Kopf nach rechts bewegen und weiter bei (2).
Falls nein, Kopf nach rechts bewegen und weiter bei (3).
In diesem Schritt werden alle Zeichen akzeptiert ausser das
{\tt\textbackslash}-Zeichen, bei diesen muss (im nächsten Schritt)
noch überprüft werden, ob ein erlaubtes Zeichen folgt.
\item Prüfe, ob das Zeichen ein {\tt \textquotedblright} oder ein
{\tt \textbackslash} ist.
Falls nein: $q_{\text{reject}}$, falls ja, Kopf nach rechts und weiter
bei (2).
\item Prüfe, ob das Zeichen ein {\blank} ist. Falls ja $q_{\text{accept}}$,
andernfalls $q_{\text{reject}}$.
\end{enumerate}
Das folgende Zustandsdiagramm implementiert diese Turing-Maschine:
\[
\entrymodifiers={++[o][F]}
\xymatrix{
*+\txt{}\ar[r]
	&q_0\ar[r]^{{\tt\textquotedblright},R}
		\ar@/_/[ddr]_{\text{\tt .},R}
		&q_1 \ar@/^/[d]^{{\text{\tt\textbackslash}},R}
			\ar@(ur,ul)_{{\tt .},R}
			\ar[r]^{{\tt\textquotedblright},R}
			&q_3\ar[r]^{\text{\blank} ,R}
				\ar@/^/[ddl]^{\text{\tt .},R}
				&*++[o][F=]{q_{\text{accept}}}
\\
*+\txt{}
	&*+\txt{}
		&q_2 \ar@/^/[u]^{{{\text{\tt\textbackslash}},R}\atop{{\text{\tt\textquotedblright}},R}}
		\ar[d]_{\text{\tt .},R}
\\
*+\txt{}
	&*+\txt{}
		&*++[o][F=]{q_{\text{reject}}}
}
\]
Darin bedeutet die Beschriftung $\text{\tt .},R$ eines Pfeiles, dass
dieser "Ubergang genommen werden soll für alle Zeichen, die nicht
durch andere "Ubergänge von diesem Zustand aus schon bestimmt sind.
\item
Die Sprache ist regulär, was man daran erkennen kann, dass die
Turing-Maschine nur ein einziges Mal von links nach rechts durch den
String fährt, und dabei niemals auf das Band schreibt. Sie liest also
nur Zeichen vom Band, ändert ihren Zustand und befindet sich am Ende 
Des Programmlaufs entweder in einem Akzeptierzustand oder eben nicht.
Das ist genau, was ein DEA macht. Ein möglicher regulärer Ausdruck
für diese Sprache ist
\verbatiminput{re}
\qedhere
\end{teilaufgaben}
\end{loesung}
