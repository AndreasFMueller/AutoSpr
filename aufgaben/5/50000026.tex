Das erste Beispiel einer Turing-Maschine,
die Alan Turing in seinem grundlegenden Paper
über Berechenbarkeit und das Entscheidungsproblem
beschreibt, hat das folgende Zustandsdiagramm:
\[
\entrymodifiers={++[o][F]}
\xymatrix @+10mm {
*+\txt{} \ar[r]
	&q_0 \ar[rrr]^{\texttt{\blank}\to\texttt{0},R}
		\ar[dr]_{\texttt{0}\to\texttt{\blank},L}
		\ar[drr]^{\texttt{1}\to\texttt{\blank},L}
		&*+\txt{}
			&*+\txt{}
				&q_1 \ar[dd]^{\blank\to\blank,R}
					\ar[dll]_{\texttt{0}\to\texttt{\blank},L}
					\ar[dl]^{\texttt{1}\to\texttt{\blank},L}
\\
*+\txt{}
	&*+\txt{}
		&*++[o][F=]{q_{\text{accept}}}
			&*++[o][F=]{q_{\text{reject}}}
				&*+\txt{}
					&*+\txt{}
\\
*+\txt{}
	&q_3 \ar[uu]^{\blank\to\blank,R}
		\ar[ur]^{\texttt{0}\to\texttt{\blank},L}
		\ar[urr]_{\texttt{1}\to\texttt{\blank},L}
		&*+\txt{}
			&*+\txt{}
				&q_2 \ar[lll]^{\blank\to\texttt{1},R}
					\ar[ull]^{\texttt{0}\to\texttt{\blank},L}
					\ar[ul]_{\texttt{1}\to\texttt{\blank},L}
}
\]
Im Gegensatz zu unserer späteren Konvention befindet sich bei Alan Turing's
ursprünglicher Definition einer Turing-Maschine
der Schreib-/Lese-Kopf zu Beginn der Berechnung nicht auf dem ersten
Zeichen das Inputwortes, sondern irgendwo auf dem Band.
\begin{teilaufgaben}
\item
Was tut diese Maschine, wenn das Band leer ist?
\item
Hält die Maschine auf leerem Band an?
\item
Wenn das Band nicht leer ist, hält die Maschine dann an?
Wenn ja, in welchem Zustand?
\item
Wenn die Maschine anhält, was steht dann auf dem Band?
\end{teilaufgaben}

\thema{Turing-Maschine}
\thema{Zustandsdiagramm}

\begin{loesung}
\begin{teilaufgaben}
\item 
Wenn das Band leer ist, spielen nur die Übergänge am Umfang des Rechtecks
eine Rolle.
Die Übergängen überschreiben abwechselnd jedes zweite Leerzeichen mit
\texttt{0} bzw.~\texttt{1}.
\item
Wenn das Band leer ist, kommen die Übergängen zu den Akzeptierzuständen
gar nicht zu Zug, die Maschine hält also nicht an.
\item
Solange der Schreib-/Lese-Kopf auf ein leeres Feld zeigt, bleiben
weiterhin nur die Übergänge am Umfang massgebend.
Sobald der Schreib-/Lese-Kopf auf ein anderes Zeichen trifft, wird das
Zeichen gelöscht und die Maschine vollzieht einen Überhang in einen
Akzeptierzustand.
Falls das Zeichen ein \texttt{0} ist, erfolgt der Übergang nach
$q_\text{accept}$, falls das Zeichen ein \texttt{1}, erfolgt der Übergang
nach $q_{\text{reject}}$.
\item
In diesem Fall stehen auf dem Band Folgen von $\texttt{0\blank1\blank}$
bis zum ersten nicht-\blank-Zeichen.
Dieses wird überschrieben von \blank{} und danach befindet sich der
unveränderte Inhalt des Bandes.
\qedhere
\end{teilaufgaben}
\end{loesung}

\begin{bewertung}
\begin{teilaufgaben}
\item
({\bf A}) 2 Punkt.
\item
({\bf B}) 1 Punkte.
\item
({\bf C}) 2 Punkte.
\item
({\bf D}) 1 Punkt.
\end{teilaufgaben}
\end{bewertung}

