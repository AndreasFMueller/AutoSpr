Die Sprache $L$ besteht aus Wörtern über dem Alphabet
$\Sigma=\{\texttt{0},\texttt{1}\}$ mit folgender Eigenschaft:
jedes Zeichen \texttt{0} muss unmittelbar von einem Zeichen \texttt{1}
gefolgt sein.

\thema{Turing-Maschine}

\begin{teilaufgaben}
%\item
%Zeichnen Sie das Zustandsdiagramm eines DEA, welcher $L$ akzeptiert.
\item
Zeichnen Sie das Zustandsdiagramm einer Turing-Maschine, die $L$ entscheidet.
\item
Wörter in $L$ haben immer mindestens soviele Zeichen \texttt{1} wie \texttt{0}.
Zeichnen Sie das Zustandsdiagramm einer Turing-Maschine, die genau die
Wörter aus $L$ akzeptiert und ein Wort aus so vielen Zeichen \texttt{1}
auf dem Band stehen lässt, wie das Inputwort mehr \texttt{1} als \texttt{0}
hat.
Es bleibt also das Wort
$\texttt{1}^{|w|_\texttt{1}-|w|_\texttt{0}}$ auf dem Band stehen
\end{teilaufgaben}

\begin{loesung}
\begin{teilaufgaben}
\item
\[
\entrymodifiers={++[o][F]}
\xymatrix @+8mm {
*+\txt{}\ar[r]
	&{q_0} \ar[d]_{\blank\rightarrow\blank,\text{R}}
		\ar@/_/[r]_{\texttt{0}\rightarrow\texttt{0},\text{R}}
		\ar@(ur,ul)_{\texttt{1}\to\texttt{1},\text{R}}
		&{q_1}	\ar@/_/[l]_{\texttt{1}\rightarrow\texttt{1},\text{R}}
			\ar[d]_{\scriptstyle\texttt{0}\to\texttt{0},\text{R}
				\atop
				\scriptstyle\blank\to\blank,\text{R}}
\\
*+\txt{}
	&*++[o][F=]{q_{\text{accept}}}
		&*++[o][F=]{q_{\text{reject}}}
}
\]
\item
Wir modifizieren die Turing-Maschine von Teilaufgaben a) wie folgt.
Immer wenn wir die auf eine \texttt{0} folgende \texttt{1} verarbeiten,
ersetzen wir sie durch eine \texttt{0}.
Wenn wir am Ende des Wortes ankommen, stehen dann nur noch so viele 
\texttt{1} auf dem Band, wie wir am Schluss haben wollen, aber es kann
dazwischen noch \texttt{0}en haben.

Wir müssen also alle \texttt{1} nach vorne schieben und dann alle \texttt{0}
löschen.
Dies kann mit der folgenden Turing-Maschine geschehen:
\[
\entrymodifiers={++[o][F]}
\xymatrix @+8mm {
*+\txt{}\ar[r]
	&{q_0}	\ar[d]_{\blank\rightarrow\blank,\text{R}}
		\ar@/_/[r]_{\texttt{0}\rightarrow\texttt{0},\text{R}}
		\ar@(ur,ul)_{\texttt{1}\to\texttt{1},\text{R}}
		&{q_1}	\ar@/_/[l]_{\texttt{1}\rightarrow\texttt{\color{red}0},\text{R}}
			\ar[r]^{\scriptstyle\texttt{0}\to\texttt{0},\text{R}
				\atop
				\scriptstyle\blank\to\blank,\text{R}}
			&*++[o][F=]{q_{\text{reject}}}
\\
*+\txt{}
	&{q_3}	\ar@(dl,ul)^{\scriptstyle \texttt{0}\to\texttt{0},\text{L}
			\atop
			\scriptstyle \texttt{1}\to\texttt{1},\text{L}}
		\ar[d]_{\blank\to\blank,\text{R}}
		&*+\txt{}
\\
*+\txt{}
	&{q_4} \ar@(dl,ul)^{\texttt{1}\to\texttt{1},\text{R}}
		\ar[d]_{\blank\to\blank,\text{R}}
		\ar[r]^{\texttt{0}\to\texttt{0},\text{R}}
		&{q_5}	\ar[r]^{\texttt{1}\to\texttt{0},\text{L}}
			\ar@(dl,d)_{\texttt{0}\to\texttt{0},\text{R}}
			\ar[dr]^{\blank\to\blank,\text{L}}
			&{q_6}	\ar[r]^{\texttt{0}\to\texttt{1},\text{L}}
				\ar[uu]_{
					\scriptstyle\texttt{1}\to\texttt{1},\text{R}
					\atop
					\scriptstyle\blank\to\blank,\text{R}
				}
				&{q_7}	\ar@/_40pt/[lll]_{\blank\to\blank,\text{R}}
				\ar@(dr,ur)_{
					\scriptstyle\texttt{0}\to\texttt{0},\text{L}
					\atop
					\scriptstyle\texttt{1}\to\texttt{1},\text{L}
				}
\\
*+\txt{}
	&*++[o][F=]{q_{\text{accept}}}
		&*\txt{}
			&{q_8}	\ar@(dr,ur)_{\texttt{0}\to\blank,\text{L}}
				\ar[ll]_{\scriptstyle \texttt{1}\to\texttt{1},\text{L}
					\atop
					\scriptstyle \blank\to\blank,\text{L}}
}
\]
Im Zustand $q_3$ fährt die Turingmaschine den Schreib-/Lesekopf wieder zurück
auf den Wortanfang.

Auf der Zeile der Zustände $q_4$ bis $q_7$ werden zunächst jeweils die
\texttt{1} am Anfang des Wortes überfahren, dann alle \texttt{0} bis zur
nächsten \texttt{1}, die durch eine \texttt{0} überschrieben wird
(Übergang von $q_5$ nach $q_6$).
Die davor stehende Null wird durch eine \texttt{1} ersetzt
(Übergang von $q_6$ nach $q_7$), die \texttt{1} wird also eine
Stelle nach links geschoben.
Dies wird gemacht, bis nach der ersten Null keine \texttt{1} mehr gefunden
wird, was den Übergang von $q_5$ nach $q_8$ erzwingt.

Im Zustand $q_8$ werden alle \texttt{0} durch \blank{} ersetzt, so dass
am Schluss nur noch die \texttt{1} auf dem Band stehen bleiben.
\qedhere
\end{teilaufgaben}
\end{loesung}

\begin{bewertung}
\begin{teilaufgaben}
\item
Turing-Maschine kann auf \texttt{0} folgende \texttt{1} von anderen
\texttt{1} unterscheiden ({\bf 1}) 1 Punkt,
vollständige Turing-Maschine ($\textbf{T}_\text{a}$) 1 Punkt.
\item
Funktionierender Lösungsalgorithmus ({\bf L}) 1 Punkt,
Zusammensammeln der \texttt{1} ({\bf S}) 1 Punkt,
Löschen der \texttt{0} ({\bf N}) 1 Punkt,
vollständige Turing-Maschine ($\textbf{T}_\text{b}$) 1 Punkt.
\end{teilaufgaben}
\end{bewertung}

