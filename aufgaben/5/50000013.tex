Welche der folgenden Mengen sind abzählbar unendlich, welche sind
überabzählbar.
\begin{teilaufgaben}
\item Die Menge aller kontextfreien Sprachen.
\item Die Menge aller Folgen $a_1,a_2,\dots$ von rationalen Zahlen,
die nach endlich vielen Folgengliedern konstant sind.
\item Die Menge aller konvergenten Folgen $a_1,a_2,a_3,\dots$ von
rationalen Zahlen.
\item Die Menge aller Entscheider.
\item Die Menge aller Aufzähler.
\item Die Menge aller Polynome mit ganzzahligen Koeffizienten:
${\mathbb Z}[X]$
\end{teilaufgaben}

\thema{abzählbar}

\begin{loesung}
\begin{teilaufgaben}
\item Die kontextfreien Grammatiken kann man zum Beispiel nach
Anzahl Regeln ihrer Chomsky-Normalform aufzählen, also ist
die Menge der kontextfreien Grammatiken abzählbar.
\item Die Menge der Folgen, die nach endlich vielen Gliedern konstant
werden, kann man zerlegen in die Mengen $F_k$, der Folgen, die ab dem
$k$-ten Glied konstant sind. Die Menge $F_k$ besteht aus folgen, die
$k$ verschiedene rationale Glieder haben, also gleich mächtig wie
$\mathbb Q^{k+1}$. Als abzählbare Vereinigung von abzählbaren Mengen
ist die Menge der Folgen, die nach endlich vielen Gliedern konstant sind,
also abzählbar.
\item Jede reelle Zahl lässt sich durch eine Folge rationaler Zahlen
approximieren. Daher ist die Menge der konvergenten Folgen mindestens
so gross wie die Menge der reellen Zahlen, also überabzählbar.
\item Entscheider sind Turing-Maschinen, davon gibt es nur abzählbar viele.
\item Aufzähler sind Turing-Maschinen, davon gibt es nur abzählbar viele.
\item Die Menge ${\mathbb Z}[X]$ lässt sich schreiben als Vereinigung
der Mengen $P_k$, wobei $P_k$ die Polynome vom Grad $k$ enthält.
Ein Polynom vom Grad $k$ mit ganzzahligen Koeffizienten wird beschrieben
durch die Koeffizienten, $P_k$ ist also gleich mächtig wie $\mathbb Z^{k+1}$.
Damit ist aber ${\mathbb Z}[X]$ als Vereinigung von abzählbar vielen
abzählbaren Mengen wieder abzählbar.
\qedhere
\end{teilaufgaben}
\end{loesung}

