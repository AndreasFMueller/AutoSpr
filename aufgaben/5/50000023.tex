Sei $\Sigma=\{\texttt{0},\texttt{1}\}$ und
\[
L
=
\left\{
w\in\Sigma^*
\;\left|\;
\begin{minipage}{0.5\hsize}
$w$ besteht abwechselnd aus \texttt{0} und \texttt{1} und hört mit
dem gleichen Zeichen auf wie es beginnt
\end{minipage}
\right.
\right\}.
\]
Finden Sie das Zustandsdiagramm einer Turing-Maschine, die $L$ entscheidet.

\thema{Turing-Maschine}
\thema{Zustandsdiagramm}

\begin{loesung}
Die Turing-Maschine muss die beiden Fälle unterscheiden, dass das Wort 
mit \texttt{0} oder mit \texttt{1} beginnt, dazu dienen der obere
und der untere Ast des Diagramms.
\begin{center}
\begin{tikzpicture}[>=latex,thick]
\def\l{3}
\def\zustand#1#2{
	\draw #1 circle[radius=0.35];
	\node at #1 {$#2\mathstrut$};
}
\def\pfeil#1#2{
	\draw[->,shorten >= 0.35cm,shorten <= 0.35cm] #1 -- #2;
}
\coordinate (q0) at ({-1.4*\l},0);
\coordinate (qaccept) at (0,0);
\coordinate (q1) at (0,\l);
\coordinate (q2) at ({1.4*\l},\l);
\coordinate (q3) at (0,-\l);
\coordinate (q4) at ({1.4*\l},-\l);
\coordinate (qreject) at ({1.4*\l},0);
\zustand{(q0)}{q_0}
\zustand{(q1)}{q_1}
\zustand{(q2)}{q_2}
\zustand{(q3)}{q_3}
\zustand{(q4)}{q_4}
\node[ellipse,draw,double] (qa) at (qaccept) {$q_{\text{accept}}$};
\node[ellipse,draw,double] (qr) at (qreject) {$q_{\text{reject}}$};
\pfeil{(q0)}{(q1)}
\pfeil{(q0)}{(q3)}

\draw[->,shorten <= 0.35cm] (q1) -- (qr.north west);
\draw[->,shorten <= 0.35cm] (q2) -- (qr.north);
\draw[->,shorten <= 0.35cm] (q3) -- (qr.south west);
\draw[->,shorten <= 0.35cm] (q4) -- (qr.south);
\draw[->,shorten <= 0.35cm] (q1) -- (qa.north);
\draw[->,shorten <= 0.35cm] (q0) -- (qa.west);
\draw[->,shorten <= 0.35cm] (q3) -- (qa.south);
\draw[<-,shorten <= 0.35cm] (q0) -- +(-1.5,0);
\draw[->,shorten <= 0.35cm,shorten >= 0.35cm] (q1) to[out=10,in=170] (q2);
\draw[->,shorten <= 0.35cm,shorten >= 0.35cm] (q2) to[out=-170,in=-10] (q1);
\draw[->,shorten <= 0.35cm,shorten >= 0.35cm] (q3) to[out=10,in=170] (q4);
\draw[->,shorten <= 0.35cm,shorten >= 0.35cm] (q4) to[out=-170,in=-10] (q3);

\node at ($0.5*(q0)+0.5*(q1)$)
	[above left] {$\scriptstyle\texttt{0}\to\texttt{0},R$};
\node at ($0.5*(q0)+0.5*(q3)$)
	[below left] {$\scriptstyle\texttt{1}\to\texttt{1},R$};
\node at ($0.5*(q0)+0.5*(qa.west)$) [above] {$\scriptstyle\blank\to\blank,R$};
\node at ($0.4*(q1)+0.6*(qa.north)$) [right] {$\scriptstyle\blank\to\blank,R$};
\node at ($0.4*(q3)+0.6*(qa.south)$) [right] {$\scriptstyle\blank\to\blank,R$};
\node at ($0.5*(q2)+0.5*(qr.north)$) [right] {$*$};
\node at ($0.5*(q4)+0.5*(qr.south)$) [right] {$*$};
\node at ($0.4*(q1)+0.6*(qr.north west)$)
	[above right] {$\scriptstyle\texttt{0}\to\texttt{0},R$};
\node at ($0.4*(q3)+0.6*(qr.south west)$)
	[below right] {$\scriptstyle\texttt{0}\to\texttt{0},R$};
\node at ($0.5*(q1)+0.5*(q2)+(0,0.2)$)
	[above] {$\scriptstyle\texttt{1}\to\texttt{1},R$};
\node at ($0.5*(q1)+0.5*(q2)+(0,-0.2)$)
	[below] {$\scriptstyle\texttt{0}\to\texttt{0},R$};
\node at ($0.5*(q3)+0.5*(q4)+(0,-0.2)$)
	[below] {$\scriptstyle\texttt{1}\to\texttt{1},R$};
\node at ($0.5*(q3)+0.5*(q4)+(0,0.2)$)
	[above] {$\scriptstyle\texttt{0}\to\texttt{0},R$};
\end{tikzpicture}
\end{center}
%\[
%\entrymodifiers={++[o][F]}
%\xymatrix @+15mm {
%*+\txt{}
%	&*+\txt{}
%		&{q_1}	\ar@/^/[r]^{\texttt{1}\to\texttt{1},R}
%			\ar[dr]^{\texttt{0}\to\texttt{0},R}
%			\ar[d]^{\blank\to\blank,R}
%			&{q_2} \ar@/^/[l]^{\texttt{0}\to\texttt{0},R}
%				\ar[d]^{\texttt{*}}
%\\
%*+\txt{} \ar[r]
%	&{q_0}
%		\ar[r]^{\blank\to\blank,R}
%		\ar[ur]^{\texttt{0}\to\texttt{0},R}
%		\ar[dr]_{\texttt{1}\to\texttt{1},R}
%		&*++[o][F=]{q_{\text{accept}}}
%			&*++[o][F=]{q_{\text{reject}}}
%\\
%*+\txt{}
%	&*+\txt{}
%		&{q_3} \ar@/^/[r]^{\texttt{0}\to\texttt{0},R}
%			\ar[u]_{\blank\to\blank,R}
%			\ar[ur]_{\texttt{1}\to\texttt{1},R}
%			&{q_4} \ar@/^/[l]^{\texttt{1}\to\texttt{1},R}
%				\ar[u]_{\texttt{*}}
%}
%\]
Darin bedeuten die mit \texttt{*} bezeichneten Pfeile alle Übergänge, für die
es ausgehend von diesem Zustand noch keinen Übergang gibt.

Im Zustand $q_1$ bzw.~$q_3$ hat die Maschine jeweils das gleiche Zeichen
verarbeitet, mit dem das Wort begonnen hat.
Nur von diesen Zuständen aus ist daher ein Übergang zu $q_{\text{accept}}$
möglich, aber auch nur mit dem Zeichen \blank.
Alle anderen Zeichen müssen zu $q_{\text{reject}}$ führen.
\end{loesung}

\begin{bewertung}
Maschine ist deterministisch ({\bf D}) 1 Punkt,
Maschine erkennt Wortende ({\bf E}) 1 Punkt,
Maschine akzeptiert nur alternierende Folgen ({\bf A}) 2 Punkte,
Maschine akzeptiert nur Folgen, die mit dem gleichen Zeichen enden wie
sie beginne ({\bf G}) 2 Punkt.
\end{bewertung}



