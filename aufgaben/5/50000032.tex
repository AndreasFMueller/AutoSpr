Betrachten Sie die Turing-Maschine mit Bandalphabet
$\Gamma=\{\texttt{0},\texttt{1},\blank\}$
und Zustandsdiagramm
\[
\entrymodifiers={++[o][F]}
\xymatrix @+5mm {
*+\txt{}\ar[r]
	&{q_0}	\ar[r]^{\texttt{0}\to \texttt{0},R}
		\ar@(ur,ul)_{\texttt{1}\to\texttt{1},R}
		\ar[d]_{\blank\to\blank,L}
		&{q_1}	\ar[r]^{\texttt{0}\to\texttt{0},R}
			\ar@(ur,ul)_{\texttt{1}\to\texttt{1},R}
			\ar[d]^{\blank\to\blank,L}
			&{q_2}	\ar@/_33pt/[ll]_{\texttt{0}\to\texttt{0},R}
				\ar@(ur,ul)_{\texttt{1}\to\texttt{1},R}
				\ar[d]^{\blank\to\blank,L}
\\
*++[o][F=]{q_{\text{accept}}}
	&{q_3}	\ar@(dr,dl)^{\texttt{0}\to\texttt{0},L}
		\ar[l]_{\blank\to\blank,L}
		&{q_4}	\ar[l]_{\texttt{1}\to \texttt{1},L}
			\ar@(dr,dl)^{\texttt{0}\to\texttt{0},L}
			\ar[r]^{\blank\to\blank,L}
			\ar@/_33pt/[rr]_{\blank\to\blank,L}
			&{q_5}	\ar@/^33pt/[ll]^{\texttt{1}\to \texttt{1},L}
				\ar@(dr,dl)^{\texttt{0}\to\texttt{0},L}
				\ar[r]^{\blank\to\blank,L}
				&*++[o][F=]{q_{\text{reject}}}
}
\]
\begin{teilaufgaben}
\item
Wie verarbeitet die Maschine das Wort \texttt{010}?
\item
Welche der Wörter
\texttt{0110},
\texttt{10100},
\texttt{100100},
\texttt{101111}
und
\texttt{1101000}
werden akzeptiert?
\item
Welche Sprache wird von der Turing-Maschine akzeptiert?
\end{teilaufgaben}

\thema{Turing-Maschine}
\thema{Zustandsdiagramm}

\begin{loesung}
\begin{teilaufgaben}
\item
Die folgende Tabelle beschreibt die Berechnung
\begin{center}
\begin{tabular}{>{$}r<{$}|>{$}l<{$}>{$}l<{$}>{$}l<{$}>{$}l<{$}>{$}l<{$}}
q_0 & \blank & \color{red}\texttt{0} & \texttt{1} & \texttt{0} & \blank \\
q_1 & \blank & \texttt{0} & \color{red}\texttt{1} & \texttt{0} & \blank \\
q_1 & \blank & \texttt{0} & \texttt{1} & \color{red}\texttt{0} & \blank \\
q_2 & \blank & \texttt{0} & \texttt{1} & \texttt{0} & \color{red}\blank \\
q_5 & \blank & \texttt{0} & \texttt{1} & \color{red}\texttt{0} & \blank \\
q_5 & \blank & \texttt{0} & \color{red}\texttt{1} & \texttt{0} & \blank \\
q_3 & \blank & \color{red}\texttt{0} & \texttt{1} & \texttt{0} & \blank \\
q_3 & \color{red}\blank & \texttt{0} & \texttt{1} & \texttt{0} & \blank \\
q_{\text{accept}} &&&&
\end{tabular}
\end{center}
Das Wort \texttt{010} wird akzeptiert.
\item
Offenbar fehlt beim Zustand $q_3$ ein Übergang für das Zeichen \texttt{1},
dies TM ist nicht deterministisch.
Keines der Wörter wird akzeptiert, wie man aus der Beschreibung der
Sprache ablesen kann.
\item
Im oberen Teil des Diagramms wird ermittelt, ob die Anzahl der Nullen
durch drei teilbar ist (Zustand $q_0$) oder ob ein Rest $k$ stehen
bleibt (Zustand $q_k$).
Im unteren Teil wird dann das Wort von rechts nach links nochmals
durchgelesen.
Die folgenden Fälle können eintreten:
\begin{enumerate}
\item Rest 0: Das Wort darf nur \texttt{0} enthalten, weil sonst kein
Übergang mehr möglich ist.
\item Rest $1$ und $2$: Das Wort enthält genau eine \texttt{1}.
\end{enumerate}
Die akzeptierte Sprache ist daher
\[
L
=
\{
w\in\{\texttt{0},\texttt{1}\}^*
\;|\;
|w|_{\texttt{0}}\equiv 0\mod 3
\veebar
|w|_{\texttt{1}}=1
\}
\]
Das Zeichen $\veebar$ bedeutet exklusives ODER.
\qedhere
\end{teilaufgaben}
\end{loesung}

\begin{bewertung}
\begin{teilaufgaben}
\item
Berechnungsgeschichte ({\bf B}) 1 Punkt,
akzeptiert ({\bf A}) 1 Punkt,
\item
Kein Wort wird akzeptiert ({\bf K}) 1 Punkt,
Begründung ({\bf G}) 1 Punkt,
\item
Teilbarkeit ({\bf T}) 1 Punkt,
nur eine Eins ({\bf E}) 1 Punkt.
\end{teilaufgaben}
\end{bewertung}
