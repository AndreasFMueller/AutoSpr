Warum ist der folgende Algorithmus kein Entscheider?

\begin{center}
\begin{tikzpicture}[>=latex,thick]
\coordinate (q0) at (0,0);
\coordinate (q1) at (3,0);
\coordinate (qa) at (0,{-sqrt(3)*1.5});
\coordinate (qr) at (3,{-sqrt(3)*1.5});
\draw (q0) circle[radius=0.4];
\node at (q0) {$q_0$};
\draw (q1) circle[radius=0.4];
\node at (q1) {$q_1$};
\draw (qa) ellipse (0.6cm and 0.3cm);
\draw (qa) ellipse (0.65cm and 0.35cm);
\node at (qa) {$q_{\text{accept}}$};
\draw (qr) ellipse (0.6cm and 0.3cm);
\draw (qr) ellipse (0.65cm and 0.35cm);
\node at (qr) {$q_{\text{reject}}$};
\draw[->,shorten >= 0.4cm,shorten <= 0.4cm] (q0) to[out=30,in=150] (q1);
\draw[->,shorten >= 0.4cm,shorten <= 0.4cm] (q1) to[out=-150,in=-30] (q0);
\draw[->,shorten >= 0.4cm] (-1.5,0) -- (q0);
\draw[->,shorten >= 0.4cm,shorten <= 0.4cm] (q0) -- (qa);
\draw[->,shorten >= 0.4cm,shorten <= 0.4cm] (q1) -- (qr);
\draw[->,shorten >= 0.4cm,shorten <= 0.4cm]
	(q0) to[out=60,in=120,distance=1.5cm] (q0);
\draw[->,shorten >= 0.4cm,shorten <= 0.4cm]
	(q1) to[out=30,in=-30,distance=1.5cm] (q1);
\node at ($(q1)+(1.1,0)$) [right] {$\texttt{1}\to\texttt{0},L$};
\node at ($0.5*(q0)+0.5*(q1)+(0,0.4)$) [above] {$\blank\to\blank,L$};
\node at ($0.5*(q0)+0.5*(q1)-(0,0.4)$) [below] {$\blank\to\blank,R$};
\node at ($(q0)+(0,1)$) [above] {$\texttt{0}\to\texttt{1},R$};

\node at ($(q0)+(0,{-sqrt(3)*0.75})$) [left] {$\texttt{1}\to\texttt{0},R$};
\node at ($(q1)+(0,{-sqrt(3)*0.75})$) [right] {$\texttt{0}\to\texttt{1},R$};

\end{tikzpicture}
\end{center}

\thema{Turing-Maschine}
\thema{Entscheider}

\begin{loesung}
Beim Zustand $q_0$ werden alle Nullen durch Einsen ersetzt.
Wenn das Wort kein Eins enthält, dann geschieht das, bis der Schreib-/Lesekopf
am Ende des Wortes ankommt.
Beim Zustand $q_1$ werden dann alle Einsen wieder durch Nullen
ersetzt bis der Schreib-/Lese-Kopf am Wortanfang ankommt.
Die einzige Möglichkeit für die Maschine, anzuhalten ist also, dass
das Wort eine Eins enthalten muss.
Enthält das Wort keine Eins, hält die Maschine nicht.
Daher ist die Maschine kein Entscheider.
\end{loesung}
