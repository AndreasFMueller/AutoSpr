%
% aufgabensammlung.tex -- Skript zur Vorlesung Automaten und Sprachen
%                         an der Hochschule Rapperswil
%
% (c) 2009-2017 Prof. Dr. Andreas Mueller, HSR
%
\documentclass[a4paper,12pt]{book}
\usepackage{german}
\usepackage[utf8]{inputenc}
\usepackage[T1]{fontenc}
\usepackage{CJKutf8}
\usepackage{times}
\usepackage{geometry}
\geometry{papersize={210mm,297mm},total={160mm,240mm},top=31mm,bindingoffset=15mm,marginparwidth=9mm}
\usepackage{alltt}
\usepackage{verbatim}
\usepackage{fancyhdr}
\usepackage{amsmath}
\usepackage{amssymb}
\usepackage{amsfonts}
\usepackage{amsthm}
\usepackage{textcomp}
\usepackage{graphicx}
\usepackage{array}
%\usepackage{picins}
\usepackage{ifthen}
\usepackage{multirow}
\usepackage{txfonts}
%\usepackage[basic]{circ}
\usepackage[all]{xy}
\usepackage{algorithm}
\usepackage{algorithmic}
\usepackage{makeidx}
\usepackage{paralist}
\usepackage{color}
\usepackage[colorlinks=true]{hyperref}
\usepackage{environ}
\usepackage{menukeys}
\usepackage{pifont}
\usepackage{etoolbox}
\usetikzlibrary{arrows,scopes}
\makeindex
\begin{document}
\pagestyle{fancy}
\lhead{Aufgabensammlung}
\rhead{}
\frontmatter
\newcommand\HRule{\noindent\rule{\linewidth}{1.5pt}}
\begin{titlepage}
\vspace*{\stretch{1}}
\HRule
\vspace*{2pt}
\begin{flushright}
{\Huge
Automaten und Sprachen:\\
\bigskip
Aufgabensammlung}
\end{flushright}
\HRule
\begin{flushright}
\vspace{30pt}
\LARGE
Andreas Müller
\end{flushright}
\vspace*{\stretch{2}}
\begin{center}
Hochschule für Technik, Rapperswil, 2011-2020
\end{center}
\end{titlepage}
\hypersetup{
        colorlinks=true,
        linktoc=all,
        linkcolor=blue
}
\rhead{Inhaltsverzeichnis}
\tableofcontents
\newenvironment{beispiel}[1][Beispiel]{%
\begin{proof}[#1]%
\renewcommand{\qedsymbol}{$\bigcirc$}
}{\end{proof}}
\mainmatter
\input uebungen.tex
\allowdisplaybreaks
\openthemaindex
\setboolean{loesungen}{true}
\aufgabetoplevel{./}

%
% vorwort.tex
%
% (c) 2019 Prof Dr Andreas Müller, Hochschule Rapperswil
%
\chapter*{Vorwort}
Diese Aufgabensammlung enthält alle Aufgaben, die je in Übungen oder
Prüfungen im Fach {\em Automaten und Sprachen} gestellt worden sind.
Aktuelle Übungen werden jeweils aus Aufgaben dieser Aufgabensammlung
zusammengestellt.
Unmittelbar nach einer Prüfung werden die Prüfungsaufgaben samt
Lösung und Bewertungsinformation in die Aufgabensammlung aufgenommen.

Die Aufgaben sind nach Skript-Kapiteln sortiert, das Skript ist
jedoch im Github nicht mehr vorhanden, um Urheberrechtsprobleme
mit dem im Springer-Verlag publizierten Buch zu vermeiden.
Die folgende Tabelle zeigt, wie die Skriptkapitel auf die Buchkapitel
verteilt worden sind.
\begin{center}
\begin{tabular}{lll}
\hline
\multicolumn{2}{l}{Skriptkapitel}&Buchkapitel \\
\hline
1 &Grundlagen                               &  A          \\
2 &Sprachen                                 &  1          \\
3 &Endliche Automaten und reguläre Sprachen &  1, 2, 3, 4 \\
4 &Stackautomaten und kontextfreie Sprachen &  5, 6, 7, 8 \\
5 &Turing-Maschinen                         &  9, 10      \\
6 &Entscheidbarkeit                         &  11         \\
7 &Komplexitätstheorie                      &  12, 13     \\
8 &Turing-Vollständigkeit                   &  14         \\
\hline
\end{tabular}
\end{center}
Innerhalb der Kapitel sind die Aufgaben chronologisch gesammelt,
so wie sie in Prüfungen gestellt worden sind.
Die jüngsten Prüfungsaufgaben finden sich also ganz am Schluss jedes
Kapitels.
Nach jeder Aufgabe finden sich unter dem Stichwort {\em Thema} Links
ins Themenverzeichnis ganz am Schluss der Aufgabensammlung.
Diese sollen ermöglichen, für die Vertiefung und Prüfungsvorbereitung
weitere Aufgaben zum gleichen Thema zu finden.

Ehemalige Prüfungsaufgaben enthalten auch Information zur Bewertung.
In einem Block {\em Bewertung} wird angegeben, welche Teilschritte
der Lösung Punkte geben.
Diese Information kann helfen zu lernen, worauf bei der Dokumentation
einer Lösung zu achten ist.

Neben jeder Aufgabe steht auch die Nummer des Aufgabenfiles im
Github Repository.



\chapter{Grundlagen}
\lhead{Kapitel~\thechapter}
\rhead{Grundlagen}
\input 1.tex
\chapter{Sprachen}
\lhead{Kapitel~\thechapter}
\rhead{Sprachen}
\input 2.tex
\chapter{Endliche Automaten und regul"are Sprachen}
\lhead{Kapitel~\thechapter}
\rhead{Endliche Automaten und regul"are Sprachen}
\input 3.tex
\chapter{Stackautomaten und kontextfreie Sprachen}
\lhead{Kapitel~\thechapter}
\rhead{Stackautomaten und kontextfreie Sprachen}
\input 4.tex
\chapter{Turing Maschinen}
\lhead{Kapitel~\thechapter}
\rhead{Turing Maschinen}
\input 5.tex
\chapter{Entscheidbarkeit}
\lhead{Kapitel~\thechapter}
\rhead{Entscheidbarkeit}
\input 6.tex
\chapter{Komplexit"atstheorie}
\lhead{Kapitel~\thechapter}
\rhead{Komplexit"atstheorie}
\input 7.tex
\chapter{Turing-Vollst"andigkeit}
\lhead{Kapitel~\thechapter}
\rhead{Turing-Vollst"andigkeit}
\input 8.tex
\closethemaindex
\printthemata
\input aufgabensammlung.ind
\end{document}
