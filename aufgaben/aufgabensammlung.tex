%
% aufgabensammlung.tex -- Skript zur Vorlesung Automaten und Sprachen
%                         an der Hochschule Rapperswil
%
% (c) 2009 Prof. Dr. Andreas Mueller, HSR
% $Id: skript.tex,v 1.34 2008/11/02 22:46:16 afm Exp $
%
\documentclass[a4paper,12pt]{book}
\usepackage{german}
\usepackage{times}
\usepackage{geometry}
\geometry{papersize={210mm,297mm},total={160mm,240mm},top=31mm,bindingoffset=15mm}
\usepackage{alltt}
\usepackage{verbatim}
\usepackage{fancyhdr}
\usepackage{amsmath}
\usepackage{amssymb}
\usepackage{amsfonts}
\usepackage{amsthm}
\usepackage{textcomp}
\usepackage{graphicx}
\usepackage{array}
%\usepackage{picins}
\usepackage{ifthen}
\usepackage{multirow}
\usepackage{txfonts}
%\usepackage[basic]{circ}
\usepackage[all]{xy}
\usepackage{algorithm}
\usepackage{algorithmic}
\usepackage{makeidx}
\usepackage{paralist}
\usepackage{color}
\usepackage{hyperref}
\usepackage{environ}
\usepackage{menukeys}
\usepackage{pifont}
\usepackage[utf8]{inputenc}
\makeindex
\begin{document}
\pagestyle{fancy}
\lhead{Aufgabensammlung}
\rhead{}
\frontmatter
\newcommand\HRule{\noindent\rule{\linewidth}{1.5pt}}
\begin{titlepage}
\vspace*{\stretch{1}}
\HRule
\vspace*{2pt}
\begin{flushright}
{\Huge
Automaten und Sprachen:\\
\bigskip
Aufgabensammlung}
\end{flushright}
\HRule
\begin{flushright}
\vspace{30pt}
\LARGE
Andreas M"uller
\end{flushright}
\vspace*{\stretch{2}}
\begin{center}
Hochschule f"ur Technik, Rapperswil, 2011-2016
\end{center}
\end{titlepage}
\hypersetup{
        colorlinks=true,
        linktoc=all,
        linkcolor=blue
}
\rhead{Inhaltsverzeichnis}
\tableofcontents
\newenvironment{beispiel}[1][Beispiel]{%
\begin{proof}[#1]%
\renewcommand{\qedsymbol}{$\bigcirc$}
}{\end{proof}}
\mainmatter
\allowdisplaybreaks
\input uebungen.tex
\setboolean{loesungen}{true}
\aufgabetoplevel{./}

\chapter{Grundlagen}
\lhead{Kapitel~\thechapter}
\rhead{Grundlagen}
\input 1.tex
\chapter{Sprachen}
\lhead{Kapitel~\thechapter}
\rhead{Sprachen}
\input 2.tex
\chapter{Endliche Automaten und regul"are Sprachen}
\lhead{Kapitel~\thechapter}
\rhead{Endliche Automaten und regul"are Sprachen}
\input 3.tex
\chapter{Stackautomaten und kontextfreie Sprachen}
\lhead{Kapitel~\thechapter}
\rhead{Stackautomaten und kontextfreie Sprachen}
\input 4.tex
\chapter{Turing Maschinen}
\lhead{Kapitel~\thechapter}
\rhead{Turing Maschinen}
\input 5.tex
\chapter{Entscheidbarkeit}
\lhead{Kapitel~\thechapter}
\rhead{Entscheidbarkeit}
\input 6.tex
\chapter{Komplexit"atstheorie}
\lhead{Kapitel~\thechapter}
\rhead{Komplexit"atstheorie}
\input 7.tex
\chapter{Turing-Vollst"andigkeit}
\lhead{Kapitel~\thechapter}
\rhead{Turing-Vollst"andigkeit}
\input 8.tex
\input aufgabensammlung.ind
\end{document}
