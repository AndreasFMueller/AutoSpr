Die Goldbach-Vermutung besagt, jede gerade Zahl $> 2$ könne als Summe von
zwei Primzahlen geschrieben werden. Sie ist immer noch offen.
\begin{teilaufgaben}
\item Formulieren Sie das Entscheidungsproblem ``Kann die gerade Zahl $n$
als Summe von zwei Primzahlen geschrieben werden?'' als Sprachproblem.
\item Ist die in a) definierte Sprache entscheidbar?
\item Jemand hat eine Funktion \texttt{boolean gb(BigInteger n)} geschrieben,
welche herausfinden kann, ob die natürlich Zahl $n$ gerade ist und sich
als Summe von zwei Primzahlen schreiben lässt. Gibt es eine Möglichkeit,
maschinell zu überprüfen, ob eine solche Funktion in allen Fällen
korrekt arbeitet?
\end{teilaufgaben}

\thema{Sprachproblem}
\thema{Entscheidbarkeit}
\thema{Satz von Rice}

\begin{loesung}
\begin{teilaufgaben}
\item Sei $\Sigma=\{\texttt{0},\texttt{1}\}$. Die Sprache
\[
L=\left\{ w\in\Sigma^*\,\left|\, \begin{minipage}{4truein}
$w$ ist die Binärcodierung einer einer geraden Zahl $n$, die
als Summe von zwei Primzahlen geschrieben werden kann.
\end{minipage}
\right.\right\}
\]
enthält genau die Binärdarstellungen derjenigen geraden Zahlen,
die sich als Summe von zwei Primzahlen schreiben lassen.
\item Ein Entscheidungsalgorithmus für $L$ geht wie folgt vor:
\begin{compactenum}
\item Für jede Zahl $p_1<n$ überprüfe, ob $p_1$ und $n-p_1$ Primzahlen
sind.
\item Falls ja: $q_{\text{accept}}$, andernfalls $q_{\text{reject}}$.
\end{compactenum}
Da dieser Algorithmus immer anhält, ist er ein Entscheider.
\item Dieses Problem ist nicht entscheidbar.
Es geht darum, zu entscheiden, ob eine Turing-erkennbare Sprache
die Eigenschaft
``enthält genau die geraden Zahlen, die sich als Summe von zwei Primzahlen
schreiben lassen''
hat.
Diese Eigenschaft ist nicht trivial, denn die leere Menge $\emptyset$ hat
diese Eigenschaft nicht, die Menge
\[
\left\{ n\in\mathbb N\,\left|\,
\begin{minipage}{3truein}
$n$ ist eine gerade Zahl und sie ist also Summe von zwei Primzahlen
darstellbar.
\end{minipage}\right.\right\}
\]
hat die Eigenschaft.
Nach dem Satz von Rice ist diese Eigenschaft also nicht entscheidbar.
\qedhere
\end{teilaufgaben}
\end{loesung}

\begin{bewertung}
\begin{teilaufgaben}
\item Sprachproblem (\textbf{S}) 1 Punkt,
\item Algorithmus (\textbf{A}) 1 Punkt, Entscheider (\textbf{E}) Entscheider
\item Anwendung des Satzes von Rice (\textbf{R}) 1 Punkt,
Sprachbeispiele (\textbf{B}) 1 Punkt,
Schlussfolgerung Nichtentscheidbarkeit (\textbf{N}) 1 Punkt.
\end{teilaufgaben}
\end{bewertung}
