Eine ``Programmierumgebung'' für jedermann erlaubt, Berechnung mit
Graphen zu beschreiben, die vereinfachte Flussdiagramme sind.
Der Berechnung steht ein im Wesentlichen unbeschränkter Speicher
bestehend aus einzelnen Variablen $x_i$
zur Verfügung.
Die Knoten des Graphen beschreiben arithmetische Operationen auf beliebigen
Variablen $x_i$ oder Entscheidungen basierend auf beliebigen Variablenwerten,
bei welchem Nachbarknoten die Berechnung fortgesetzt werden.
Jetzt soll ein Programm geschrieben werden, welches ermittelt,
ob so beschriebene Programme anhalten werden.
Ist dies möglich?

\thema{Halteproblem}
\thema{Reduktion}
\thema{Turing-vollständig}

\begin{loesung}
Nein, denn ein solches Programm  würde das Halteproblem lösen.

Die Programme, die von solchen Flussdiagrammen beschrieben werden
können, entsprechen genau den Fähigkeiten der Sprache GOTO.
Man kann dies einsehen indem man alle Knoten des Graphen numeriert und
die Verbindungen zwischen Knoten durch GOTO-Anweisungen realisiert.
Die Sprache GOTO ist Turing-vollständig.
Insbesondere ist das Halteproblem für GOTO-Programme nicht lösbar.
\end{loesung}

\begin{bewertung}
Vergleich mit GOTO ({\bf G}) 2 Punkte,
Vergleich mit Halteproblem ({\bf H}) 2 Punkte,
Unlösbarkeit des Halteproblems ({\bf U}) 2 Punkte.
\end{bewertung}




