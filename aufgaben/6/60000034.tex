Eine ``Programmierumgebung'' f"ur jedermann erlaubt, Berechnung mit
Graphen zu beschreiben, die vereinfachte Flussdiagramme sind.
Der Berechnung steht ein im Wesentlichen unbeschr"ankter Speicher
bestehend aus einzelnen Variablen $x_i$
zur Verf"ugung.
Die Knoten des Graphen beschreiben arithmetische Operationen auf beliebigen
Variablen $x_i$ oder Entscheidungen basierend auf beliebigen Variablenwerten,
bei welchem Nachbarknoten die Berechnung fortgesetzt werden.
Jetzt soll ein Programm geschrieben werden, welches ermittelt,
ob so beschriebene Programme anhalten werden.
Ist dies m"oglich?

\begin{loesung}
Nein, denn ein solches Programm  w"urde das Halteproblem l"osen.

Die Programme, die von solchen Flussdiagrammen beschrieben werden
k"onnen, entsprechen genau den F"ahigkeiten der Sprache GOTO.
Man kann dies einsehen indem man alle Knoten des Graphen numeriert und
die Verbindungen zwischen Knoten durch GOTO-Anweisungen realisiert.
Die Sprache GOTO ist Turing-vollst"andig.
Insbesondere ist das Halteproblem f"ur GOTO-Programme nicht l"osbar.
\end{loesung}

\begin{bewertung}
Vergleich mit GOTO ({\bf G}) 2 Punkte,
Vergleich mit Halteproblem ({\bf H}) 2 Punkte,
Unl"osbarkeit des Halteproblems ({\bf U}) 2 Punkte.
\end{bewertung}




