In vielen Anwendungen wird verlangt, dass der Output sortiert wird.
Es wäre daher nützlich für die Qualitätssicherung, wenn man ein Tool
schreiben könnte, welches von einem Programm entscheiden kann,
ob sein Output korrekt sortiert wird.
Viele Programmiersprachen haben zum Sortieren Funktionen oder Klassen,
die Daten sortieren können, man könnte testen, ob diese Klassen vom
Code verwendet werden.
Manchmal fallen die Daten aber auch automatisch sortiert an, zum
Beispiel wenn in einer Datenbank ein Index verwendet wird.
Es reicht also nicht, nur die Verwendung der genannten Klassen zu
prüfen.
Ist es möglich, so ein Tool zu schreiben?

Nehmen Sie der Einfachheit halber an, dass das Tool nur auf Programme
angewendet werden soll, welche Wörter aus Zeichen in
$\Sigma=\{\texttt{a},\texttt{b},\dots,\texttt{z}\}$ erzeugen, und
die Zeichen in einem Wort sollen alphabetisch aufsteigend sortiert sein.

\thema{Reduktion}
\thema{Satz von Rice}

\begin{loesung}
Es ist nicht möglich, ein solches Tool zu schreiben, wie man mit dem
Satz von Rice zeigen kann.
Die Eigenschaft \textsl{SORTIERT}, die der Output haben muss, ist,
dass die Wörter im Output sortiert sind.
Dies ist eine nichttriviale Eigenschaft, den von den Sprachen
\begin{align*}
L_1&=\{ \texttt{ab} \}\\
L_2&=\{ \texttt{ba} \}
\end{align*}
besteht die eine aus sortierten Wörtern, die andere nicht. 
Nach dem Satz von Rice ist es nicht möglich, zu entscheiden, ob die
akzeptierte Sprache einer Turing-Maschine die Eigenschaft \textsl{SORTIERT}
hat.
\end{loesung}

\begin{bewertung}
Erkenntnis, dass es um eine Eigenschaft der erkannten Sprache geht
(Wörter sind sortiert) ({\bf E}) 1 Punkt,
Bezug zum Satz von Rice ({\bf R}) 1 Punkt,
zwei Beispielsprachen ({\bf L}) 2 Punkte,
Schlussfolgerung ({\bf S}) 2 Punkte.
\end{bewertung}

