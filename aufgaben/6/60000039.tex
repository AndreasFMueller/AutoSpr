Am 2.~Juli 2019 waren während etwa einer halben Stunde Millionen von
Websites nicht erreichbar, die beim Provider Cloudflare gehostet sind.
In einem Blog-Post von Cloudflare auf
\url{https://blog.cloudflare.com/cloudflare-outage/} kann man
lesen, dass dafür ein regulärer Ausdruck in einer Regel der verwendeten
Web Application Firewall (WAF) verantwortlich gewesen sei.
Dieser reguläre Ausdruck habe dazu geführt, dass die CPU-Last auf allen
WAFs auf 100\% geschnellt sei, was dann zu Ausfällen geführt habe.

\begin{teilaufgaben}
\item
Was lässt sich auf Grund dieses Berichts über die Implementation der
regulären Ausdrücke in der von Cloudflare verwendeten WAF schliessen?
\item
Der CEO des Konkurrenten Fogstorm\footnote{Jede Ähnlichkeit mit real 
existierenden Firmen ist rein zufällig} diktiert seinen Technikern
sofort neue Vorgehensweisen für das Deployment von WAF-Regeln in seiner
Firma, damit so etwas nicht passieren kann.
Er fordert, dass alle Regeln vor dem Deployment automatisch daraufhin
gescannt werden müssen, ob sie beim Matching exponentielle Laufzeit haben.
Wenn ja, muss das Deployment verhindert werden.
Ist so etwas möglich?
\item
Es stellt sich heraus, dass man in den WAF-Regeln bei Fogstorm
nicht nur reguläre Ausdrücke spezifizieren kann, sondern auch
Grammatik-Regeln.
Kann man diese daraufhin scannen, ob sie exponentielle Laufzeit haben?
\item
Es können auch in Javascript geschriebene Regeln verwendet werden.
Der CEO hat in einem Youtube-Video gesehen, dass damit noch schlimmere
Situationen eintreten können, nämlich dass die Regeln unendliche Laufzeit
haben.
Er verlangt daher auch einen Scanner für diese Art von Regeln.
Was meinen Sie dazu?
\end{teilaufgaben}

\begin{loesung}
\begin{teilaufgaben}
\item
Reguläre Ausdrücke können mit DEA implementiert werden,
die immer Laufzeit $O(|w|)$ haben. 
Man muss daher schliessen, dass die WAF bei Cloudflare eine
Regex-Engine verwendet, welche nur mit einem NEA implementiert
werden kann.
\item
Eine gute Regex-Engine verwendet DEAs und hat damit immer lineare Laufzeit,
es ist keine Massnahme erforderlich.
Bei einer Regex-Engine, die NEAs verwendet, ist es implementationsabhängig,
ob das Matching eines regulären Ausdrucks exponentiell lange dauern wird.
Man kann daher keinen von der Engine unabhängigen Algorithmus angeben.
\item
Parsen auf der Basis einer Grammatik ist mit dem CYK-Algorithmus immer
in Zeit $O(n^3)$ möglich, also kann der Fall exponentieller Laufzeit
gar nicht auftreten, es ist kein Scanner für diesen Fall notwendig.
\item
Javascript ist Turing-vollständig, das Halteproblem lässt sich daher
für Javascript-Programm nicht lösen.
Einen Scanner der verlangten Art kann es daher nicht geben.
\qedhere
\end{teilaufgaben}
\end{loesung}

\begin{bewertung}
\begin{teilaufgaben}
\item NEA ({\bf N}) 1 Punkt,
\item Keine Massnahme ({\bf K}) 1 Punkt,
\item CYK ({\bf C}) 1 Punkt, Laufzeit $O(n^3)$ ({\bf L}) 1 Punkt,
\item Halteproblem ({\bf H}) 1 Punkt, nicht  möglich ({\bf M}) 1 Punkt.
\end{teilaufgaben}
\end{bewertung}

