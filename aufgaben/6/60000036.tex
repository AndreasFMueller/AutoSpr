In den letzten Wochen haben die neu entdeckten Angriffe {\em Meltdown} und
{\em Spectre} im Management von IT-Abteilungen Panik verbreitet.
An {\em Meltdown} ist interessant, dass man mit diesem Trick den
Kernel-Adressraum auslesen kann.
Er funktioniert so: das Programm versucht auf eine Kernel-Adresse
zuzugreifen, das wird vom Betriebssystem verhindert, aber die Daten werden
trotzdem in den Prozessorcache geladen.
Durch Messen der Cache-Zugriffszeit kann das Programm dann
herausfinden, was gelesen worden ist.

Ein verängstigter Manager verlangt, einen Scanner einzusetzen, mit dem 
jede Software daraufhin gescannt werden kann, ob sie zur Laufzeit
den {\em Meltdown}-Angriff verwendet.
Was geben Sie ihm zur Antwort?

\thema{Halteproblem}
\thema{Reduktion}

\begin{loesung}
Einen solchen Scanner kann es nicht geben.
Normalerweise hält das Betriebssystem einen Prozess an, der versucht
auf den Kernel-Adressraum zuzugreifen.
Ein Scanner, der in der Lage ist, einen {\em Meltdown}-Angriff vorherzusehen
ist also auch in der Lage, das Anhalten eines Programms zu entscheiden,
löst also das Halteproblem.
Da das Halteproblem nicht entscheidbar ist, kann es einen solchen
Scanner nicht geben.
\end{loesung}

\begin{diskussion}
Die Tatsache, dass der Manager die Forderung überhaupt stellt,
berechtigt Zweifel an der fachlichen Kompetenz für seine Funktion.
\end{diskussion}

\begin{bewertung}
Reduktion auf das Halteproblem ({\bf R}) 4 Punkt
Verweis auf Nichtentscheidbarkeit des Halteproblems ({\bf H}) 1 Punkte,
Schlussfolgerung ({\bf S}) 1 Punkt.
\end{bewertung}

