Zeigen Sie, dass eine reguläre Sprache $L$ Turing-entscheidbar ist.

\thema{DEA}
\thema{Turing-entscheidbar}

\begin{loesung}
Da $L$ regulär ist, gibt es einen DEA $A$, der $L$ akzeptiert.
Aus diesem DEA konstruieren wir jetzt eine Turing-Maschine.
Wir ersetzen dazu jeden "Ubergang
\[
\entrymodifiers={++[o][F]}
\xymatrix{
p\ar[r]^a
        &q
}
\]
des DEA durch einen Turing-Maschinen-Übergang
\[
\entrymodifiers={++[o][F]}
\xymatrix{
p\ar[r]^{a\to a,R}
        &q
}
\]
Ausserdem fügen wir zwei neue Akzeptierzuständge $q_{\text{accept}}$ und
$q_{\text{reject}}$ hinzu, zusammen mit "Ubergängen
\[
\entrymodifiers={++[o][F]}
\xymatrix{
*++[o][F=]{p}\ar[r]^{\blank\to\blank, R}
        &q_{\text{accept}}
                &{p}\ar[r]^{\blank\to\blank, R}
                        &q_{\text{reject}}
}
\]
Diese Turing-Maschine verarbeitet den Input auf dem Band mit dem
DEA, und erkennt am Ende des Wortes genau diejenigen Wörter, die
vom DEA akzpetiert werden. Also ist $L(M)=L(A)$. Die TM hält genau
nach $|w|+1$ Schritten an, also ist $M$ ein Entscheider.
\end{loesung}
