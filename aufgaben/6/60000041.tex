Die Firma Leibacher liefert ihre Biber in einer Verpackung, die auf dem
Deckel alle zyklischen Permutationen des Wortes \texttt{BIBER} zeigt
(Abbildung~\ref{60000041:fig}).
Die zyklischen Permutationen eines Wortes sind mit einem Computerprogramm
leicht zu finden: Man bewegt wie in Abbildung~\ref{60000041:fig} 
einen Buchstaben um den anderen vom Anfang des Wortes an sein Ende.
Wir nennen eine Sprache {\em zyklisch invariant}, wenn sie mit jedem
Wort auch alle seinen zyklischen Permutationen enthält.
Ist es möglich, ein Programm zu schreiben, welches auf Programme
angewendet werden kann, um herauszufinden,
ob die akzeptierte Sprache zyklisch invariant ist, und welches immer
anhält?
\begin{figure}[h]
\centering
\includeagraphics[width=0.5\hsize]{biber2.jpg}
\caption{Zyklisch invariante Sprache auf einer Biberverpackung
\label{60000041:fig}}
\end{figure}

\begin{loesung}
Die Eigenschaft, zyklisch invariant zu sein, ist eine nichttriviale
Eigenschaft.
Die Sprache auf der Schachtel in Abbildung~\ref{60000041:fig} ist
zyklisch invariant, die Sprache bestehend nur aus dem Wort
\texttt{BIBER} hat diese Eigenschaft nicht.
Nach dem Satz von Rice kann nicht entschieden werden, ob die akzeptierte
Sprache die Eigenschaft ``zyklisch invariant'' hat, ein Programm der 
verlangten Art kann es also nicht geben.
\end{loesung}

\begin{diskussion}
Man beachte, dass sich das ``anhält'' am Ende auf das zu schreibende
Analyseprogramm bezieht, nicht auf die untersuchten Programme.
Daher geht es nicht um die Frage, ob die untersuchten Programme
anhalten, also das Halteproblem, sondern nur um die Frage, ob die
untersuchten Programme zyklisch invariante Sprachen akzeptieren.
\end{diskussion}

\begin{bewertung}
Satz von Rice ({\bf R}) 1 Punkt,
Eigenschaft ({\bf E}) 1 Punkt,
zwei Sprachen ({\bf Z}) 1 Punkt,
Eigenschaft ist nicht trivial ({\bf T}) 1 Punkt,
Schlussfolgerung nicht entscheidbar ({\bf N}) 2 Punkt.
\end{bewertung}


