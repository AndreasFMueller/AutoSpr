Zeigen Sie mit Hilfe des Satzes von Rice, dass die Frage, ob eine
Turing-Maschine eine kontextfreie Sprache akzeptiert, nicht entscheidbar
ist.

\begin{loesung}
Ob eine Sprache kontextfrei ist oder nicht ist eine nichttriviale
Eigenschaft.
Um dies zu zeigen, muss man zwei Sprachen angeben, von denen die eine
Kontextfrei ist und die andere nicht.
Zwei mögliche Sprachen sind
\begin{align*}
L_1
&=
\Sigma^*&&\text{alle Wörter, kontextfrei dank der Grammatik}
\qquad
\left\{
\;
\begin{aligned}
S&\to SZ\mid \varepsilon\\
Z&\to \Sigma
\end{aligned}
\right.
\\
L_2
&=
\{\texttt{a}^n\texttt{b}^n\texttt{c}^n
\mid
n\ge 0
\}
&&\text{nicht kontextfrei, Standardbeispiel}
\end{align*}
Da die Spracheigenschaft, kontextfrei zu sein, eine nichttriviale
Eigenschaft ist, folgt nach dem Satz von Rice, dass nicht entscheidbar
ist, ob eine Turing-Maschine eine kontextfreie Sprache akzeptiert.
\end{loesung}
