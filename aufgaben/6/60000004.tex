Seien $L_1$ und $L_2$ zwei Turing-erkennbare Sprachen. Zeigen Sie, dass
die folgenden Sprachen Turing-erkennbar sind (mit Hilfe von TM $M_1$
bzw.~$M_2$):
\begin{teilaufgaben}
\item $L_1\cup L_2$
\item $L_1L_2$
\item $L_1^*$
\item $L_1\cap L_2$
\end{teilaufgaben}

\thema{Turing-erkennbar}

\begin{loesung}
Wir müssen eine Turing Maschine angeben, die die entsprechende
Sprache erkennen kann.
Dazu können im Prinzip die gleichen Ideen verwendet werden, wie sie
zur Lösung von Aufgabe~\ref{60000005} verwendet wurden.
Allerdings ist jetzt nicht mehr sichergestellt, dass der Aufruf
der Funktionen \texttt{l1} oder \texttt{l2} je zurückkehrt.
Statt die Funktionen nacheinander aufzurufen, kann man aber in allen
Algorithmen alle Aufrufe parallel ausführen, und die Threads abbrechen,
sobald genügend Resultate für eine positive Antwort gefunden worden
sind. Im schlimmsten Fall wird kein Thread je fertig, dann ist das
Inputwort aber auch nicht akzeptabel.

Im Detail kann man dies für Turing-Maschinen auch wie folgt
beschreiben.
\begin{teilaufgaben}
\item
Wir können nicht einfach beide Turing Maschinen nach einander
auf dem Input laufen lassen, da die erste ja nicht terminieren könnte und
somit verunmöglichen könnte, dass die zweite das Wort akzeptieren kann.
Wir konstruieren daher eine Turing-Maschine mit zwei Bändern, welche
beide mit dem Input-Wort initialisiert werden.
In jedem Schritt der Maschine wird auf Band 1 ein Schritt von $M_1$
ausgeführt, und auf Band 2 ein Schritt von $M_2$.
Die Maschine akzeptiert, wenn eine der Teilmaschinen
akzeptiert.

In einer modernen Turing-Maschine, auch bekannt unter der Bezeichnung
Computer könnte man die Parallelität auch durch zwei parallele
Prozesse realisieren, dann würde das Betriebssystem die Umschaltung
zwischen den beiden Prozessen übernehmen.

In Java könnte man das realisieren, indem man zwei Threads startet,
in jedem Thread wird eine der Turing-Maschinen ausgeführt.
Sobald einer der Threads terminiert, kann der andere ``abgeschossen'' werden,
das Resultat steht ja dann fest.
\item
Um zu verhindern, dass die Turing Maschine auf einem Wort der
Sprachen nicht hält, lassen wir $M_1$ und $M_2$ jeweils nur für eine
beschränkte Anzahl Schritte $l$ laufen.
Für jede Berechnungslänge $l$ und jede Aufteilung des Wortes in zwei
Teilwörter $w=w_1w_2$ lassen wir auf zwei Tapes parallel die
Maschinen $M_1$ und $M_2$ auf dem Input $w_1$ bzw.~$w_2$ während
maximal $l$ Schritten laufen.
Akzeptieren beide, akzeptieren wir das Wort.
Falls nicht, probieren wir die nächste der $|w|+1$ Aufteilungen,
oder, falls alle Aufteilungen bereits probiert wurden, die nächste
Berechnungslänge $l+1$.
\item
Für jede Berechnungslänge $l$ und
jede Aufteilung eines Wortes $w$ in Teilstücke $x_1,\dots,x_n$
lassen
wir auf jedem Teilstück die Turing-Maschine während $l$ Schritten
laufen, die
$L_1$ erkennt. Akzeptiert sie auf allen Teilstücken, akzeptieren wir
das Wort.
Falls nicht, versuchen wir die nächste Aufteilung, oder falls
bereits alle Aufteilungen durchprobiert worden sind, die nächste
Berechnungslänge $l$.
\item Wir verwenden eine Turing-Maschine analog zu der in Teilaufgaben a)
konstruierten, akzeptieren aber erst, wenn beide Teilmaschinen
$M_1$ und $M_2$ akzeptiert haben.
\qedhere
\end{teilaufgaben}
\end{loesung}
