Sei $L$ eine Turing-erkennbare Sprache.
Man sagt, die Sprache sei links-kürzbar, wenn man von einem Wort ein beliebiges
Anfangsstück entfernen kann und das verkürzte Wort immer noch ein Wort
der Sprache ist.
Also zum Beispiel
\[
\texttt{BIBER}\in L
\quad\Rightarrow\quad
\texttt{IBER},
\texttt{BER},
\texttt{ER},
\texttt{R},
\varepsilon \in L.
\]
Wie könnte man ein Programm aufbauen, welches immer anhält und mit
welchem man andere Programme analysieren kann, ob deren akzeptierte Sprache
links-kürzbar ist?

\thema{Reduktion}
\thema{Satz von Rice}

\begin{loesung}
Die Eigenschaft einer Sprache, links-kürzbar zu sein, ist eine nichttriviale 
Eigenschaft.
Die Liste der Wörter in der Aufgabenstellung bildet eine links-kürzbare
Sprache, die Sprache bestehend nur aus dem Wort \texttt{BIBER} hat diese
Eigenschaft nicht.
Der Satz von Rice besagt jetzt, dass man kein Programm schreiben kann, welches
entscheiden könnte, ob die akzeptierte Sprache die Eigenschaft hat,
links-kürzbar zu sein.
\end{loesung}

\begin{bewertung}
Satz von Rice ({\bf R}) 1 Punkt,
Eigenschaft ({\bf E}) 1 Punkt,
zwei Sprachen ({\bf Z}) 1 Punkt,
Eigenschaft ist nicht trivial ({\bf T}) 1 Punkt,
Schlussfolgerung nicht entscheidbar ({\bf N}) 2 Punkt.
\end{bewertung}




