Ein {\it e}-Learning-System soll Sch"ulern arithmetische Ausdr"ucke
zur Auswertung geben, und die Antworten der Sch"uler "uberpr"ufen.
Die Qualit"atssicherung verlangt vom Programmierer dieses Moduls, dass
er einen unabh"angigen Test schreibt, welcher aus dem Source-Code
des Moduls ableiten kann, ob je ein inkorrekter arithmetischer Ausdruck
als Aufgabe gestellt werden k"onnte. Kann der Programmierer dieses
Problem l"osen?

\begin{loesung}
Das Modul soll nur W"orter einer Sprache von korrekten arithmetischen
Ausdr"ucken produzieren.
Die vom Modul akzeptierte Sprache soll also die Eigenschaft
\[
P=\text{``enth"alt nur korrekte arithmetische Ausdr"ucke''}
\]
haben. 
Diese Eigenschaft ist nicht trivial. Die Sprache
\[
L_1=\{w\,|\,\text{$w$ ist ein korrekter arithemtischer Ausdruck}\}
\]
ist Turing erkennbar (sogar Turing-entscheidbar, dank des CYK-Algorithmus),
und hat die Eigenschaft $P$.
\index{CYK-Algorithmus}
\index{Satz von Rice}
Die Sprache
\[
L_2=\{ \text{``$7-$''}\}
\]
ist als endliche und damit regul"are Sprache ebenfalls Turing-erkennbar,
hat aber die Eigenschaft $P$ nicht. Nach dem Satz von Rice folgt daher,
dass nicht entscheidbar ist, ob die von dem Modul akzeptierte Sprache
von W"ortern nur aus korrekten arithmetischen Ausdr"ucken besteht.
So einen Test kann der Programmierer also nicht schreiben.
\end{loesung}

\begin{loesung}
Man kann auch direkt eine Reduktion von $A_{\text{TM}}$ auf das vorliegende
Problem konstruieren.  Wir nehmen also an, wir h"atten einen Entscheider,
der herausfinden kann, ob eine Turingmaschine ausschliesslich korrekte
arithmetische Ausdr"ucke akzeptiert. Wir konstruieren jetzt die
Reduktionsabbildung, die $\langle M,w\rangle$ auf das folgende
Programm $M'$ mit Input $u$ abbildet:
\begin{enumerate}
\item Teste, ob $u$ kein arithmetischer Ausdruck ist. Falls 
$u$ ein arithemtischer Ausdruck ist: $q_\text{reject}$
\item Lasse $M$ auf $w$ laufen. Falls $M$ akzeptiert: $q_\text{accept}$,
andernfalls $q_\text{reject}$
\end{enumerate}
Dieses Programm akzeptiert genau die Sprache, die aus lauter W"ortern
besteht, die keine arithemetischen Ausdr"ucke sind, oder die leere
Sprache, und zwar abh"angig davon ob $M$ das Wort $w$ akzeptiert
oder nicht. Indem man den Entscheider f"ur arithmetische Ausdr"ucke
auf $M'$ anwendet kann man jetzt also entscheiden, ob $M$ auf $w$
anhält, man hat einen Entscheider f"ur $A_\text{TM}$ gefunden.
Da es einen solchen nicht geben kann, ist auch das urspr"ungliche
Problem nicht entscheidbar.
\end{loesung}

\begin{bewertung}
Anwendung des Satzes von Rice ({\bf R}) 1 Punkt,
Eigenschaft ({\bf P}) 2 Punkt,
zwei Beispiele f"ur Sprachen ({\bf B}) je 1 Punkt,
Schlussfolgerung ({\bf F}) 1 Punkt.
\end{bewertung}
