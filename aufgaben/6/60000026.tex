Ein {\it e}-Learning-System soll Schülern arithmetische Ausdrücke
zur Auswertung geben, und die Antworten der Schüler überprüfen.
Die Qualitätssicherung verlangt vom Programmierer dieses Moduls, dass
er einen unabhängigen Test schreibt, welcher aus dem Source-Code
des Moduls ableiten kann, ob je ein inkorrekter arithmetischer Ausdruck
als Aufgabe gestellt werden könnte. Kann der Programmierer dieses
Problem lösen?

\thema{Satz von Rice}
\thema{CYK-Algorithmus}
\thema{Entscheidbarkeit}

\begin{loesung}
Das Modul soll nur Wörter einer Sprache von korrekten arithmetischen
Ausdrücken produzieren.
Die vom Modul akzeptierte Sprache soll also die Eigenschaft
\[
P=\text{``enthält nur korrekte arithmetische Ausdrücke''}
\]
haben. 
Diese Eigenschaft ist nicht trivial. Die Sprache
\[
L_1=\{w\,|\,\text{$w$ ist ein korrekter arithemtischer Ausdruck}\}
\]
ist Turing erkennbar (sogar Turing-entscheidbar, dank des CYK-Algorithmus),
und hat die Eigenschaft $P$.
\index{CYK-Algorithmus}
\index{Satz von Rice}
Die Sprache
\[
L_2=\{ \text{``$7-$''}\}
\]
ist als endliche und damit reguläre Sprache ebenfalls Turing-erkennbar,
hat aber die Eigenschaft $P$ nicht. Nach dem Satz von Rice folgt daher,
dass nicht entscheidbar ist, ob die von dem Modul akzeptierte Sprache
von Wörtern nur aus korrekten arithmetischen Ausdrücken besteht.
So einen Test kann der Programmierer also nicht schreiben.
\end{loesung}

\begin{loesung}
Man kann auch direkt eine Reduktion von $A_{\text{TM}}$ auf das vorliegende
Problem konstruieren.  Wir nehmen also an, wir hätten einen Entscheider,
der herausfinden kann, ob eine Turing-Maschine ausschliesslich korrekte
arithmetische Ausdrücke akzeptiert. Wir konstruieren jetzt die
Reduktionsabbildung, die $\langle M,w\rangle$ auf das folgende
Programm $M'$ mit Input $u$ abbildet:
\begin{enumerate}
\item Teste, ob $u$ kein arithmetischer Ausdruck ist. Falls 
$u$ ein arithemtischer Ausdruck ist: $q_\text{reject}$
\item Lasse $M$ auf $w$ laufen. Falls $M$ akzeptiert: $q_\text{accept}$,
andernfalls $q_\text{reject}$
\end{enumerate}
Dieses Programm akzeptiert genau die Sprache, die aus lauter Wörtern
besteht, die keine arithemetischen Ausdrücke sind, oder die leere
Sprache, und zwar abhängig davon ob $M$ das Wort $w$ akzeptiert
oder nicht. Indem man den Entscheider für arithmetische Ausdrücke
auf $M'$ anwendet kann man jetzt also entscheiden, ob $M$ auf $w$
anhält, man hat einen Entscheider für $A_\text{TM}$ gefunden.
Da es einen solchen nicht geben kann, ist auch das ursprüngliche
Problem nicht entscheidbar.
\end{loesung}

\begin{bewertung}
Anwendung des Satzes von Rice ({\bf R}) 1 Punkt,
Eigenschaft ({\bf P}) 2 Punkt,
zwei Beispiele für Sprachen ({\bf B}) je 1 Punkt,
Schlussfolgerung ({\bf F}) 1 Punkt.
\end{bewertung}
