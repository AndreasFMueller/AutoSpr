Der chinesische Präsident Xi Jinping hat sich nicht darüber gefreut, dass
er auf dem Internet mit Winnie the Pooh verglichen wird.
Er hat seine
Zensoren angewiesen, Bilder von Winnie the Pooh zu blockieren, sogar in
der in China verbreiteten Chat-Anwendung WeChat werden Meldungen, die
die Zeichenkette {\tt 
Winnie the Pooh} enthalten, mit einer Fehlermeldung quittiert. 
Trotzdem gibt es natürlich immer noch zahllose Apps, die Winnie the Pooh
nicht blockieren und damit die chinesischen Zensoren brüskieren.
Gibt es eine Möglichkeit, Apps zu blockieren, die zur Laufzeit die
Zeichenkette \texttt{Winnie the Pooh} erzeugen können?

\thema{Halteproblem}
\thema{Reduktion}

\begin{loesung}
Sei $P$ irgend ein Programm, welches wir daraufhin untersuchen
wollen, ob es den String {\tt Winnie the Pooh} produzieren kann.
Mit dem Debugger könnten wir feststellen, ob ein Programm den
genannten String produziert und könnten das Programm anhalten lassen.
Herauszufinden, ob der String produziert wird, ist also gleichbedeutend
damit, herauszufinden, ob ein Programm anhält.
Letzteres ist das Halteproblem, welches nicht entscheidbar ist.

Etwas formaler können wir auch mit Hilfe einer Reduktion vom Halteproblem
nachweisen, dass Problem nicht entscheidbar ist.
Sei also $Q$ ein Programm welches keinen Output produziert, so wie das
in der Theorie zum Halteproblem vorausgesetzt wurde.
Von $Q$ wollen wir entscheiden, ob es anhält oder nicht.
Wir konstruieren das folgende neue Programm $Q'$:
\begin{enumerate}
\item Lasse $Q$ laufen
\item Gebe die Zeichenkette {\tt Winnie the Pooh} aus
\end{enumerate}
Dieses Programm produziert genau dann den Output {\tt Winnie the Pooh},
wenn $Q$ anhält.
Die Abbildung $Q\to Q'$ ist daher eine Reduktion vom Halteproblem auf
das Winnie-the-Pooh-Problem.
Da das Halteproblem nicht entscheidbar ist, ist auch das
Winnie-the-Pooh-Problem nicht entscheidbar.
\end{loesung}

\begin{bewertung}
Verbindung zum Halteproblem ({\bf H}) 2 Punkte,
Konstruktion einer Reduktion ({\bf R}) 2 Punkte,
Schluffolgerung ``nicht entscheidbar'' ({\bf E}) 2 Punkte.
\end{bewertung}


