Endlosschleifen gehören zu den immer wieder vorkommenden Programmierfehlern.
Da ein Java-Compiler den vollständigen Überblick darüber hat, was in einem
Java-Programm passiert, könnte man versuchen, die Änderungen, die der Code
an den Variablen vornimmt, zu verfolgen.
Dadurch könnte man dem Compiler ein Feature hinzufügen, mit dem der
Compiler mögliche Endlosschleifen erkennen kann.
Ist dies möglich?

\themaL{Turing-vollstandig}{Turing-vollständig}
\thema{Halteproblem}
\thema{Reduktion}

\begin{loesung}
Ein solcher Compiler wäre in der Lage, das Halteproblem für alle
Programme zu lösen, die in dieser Sprache verfasst sind.
Da Java Turing-vollständig ist, wäre damit das Halteproblem gelöst.
Dies ist ein Widerspruch zum Haltetheorem.
\end{loesung}

\begin{bewertung}
Bezug zum Halteproblem ({\bf H}) 2 Punkte,
Nichtentscheidbarkeit des Halteproblems ({\bf E}) 2 Punkte,
Turing-Vollständigkeit von Java ({\bf T}) 2 Punkte.
\end{bewertung}

