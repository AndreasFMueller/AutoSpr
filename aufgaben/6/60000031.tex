Man findet im Internet Websites, die anbieten, Word-Dokumente in PDF
umzuwandeln.
Ein Jungunternehmer möchte einen ähnlichen Dienst anbieten, um XML Files
mit Hilfe von XSLT Stylesheets in PDF Files umzuwandeln.
Die Kunden sollen einen Satz von XML-Files und XSLT Stylesheets zum
Beispiel als ZIP-File hochladen können, und als Antwort ein fertig
formattiertes PDF erhalten.
Im Apache-Projekt FOP findet er geeignete freie Software dafür.
Sie sind in diesem Projekt als Sicherheitsverantwortliche(r) dafür zuständig,
dass alle XML-/XSLT-File-Kombinationen zurückgewiesen werden, die dazu führen
könnten, dass der Server beliebig lange mit diesem Auftrag beschäftigt ist.
Wie stehen Ihre Aussichten, dies zu realisieren?

\thema{Turing-Entscheidbarkeit}
\themaL{Turing-vollstandig}{Turing-vollständig}
\thema{Halteproblem}
\thema{Satz von Rice}

\begin{loesung}
Die Sprache XSLT ist Turing-vollständig, man kann in ihr also alles
programmieren, was mit einer Turing-Maschine gemacht werden kann.
Die gestellte Aufgabe besteht also darin, herauszufinden, ob ein
gegebenes Programm (XML-Files und XSLT Stylesheets) je anhalten wird.
Dies ist das Halteproblem, welches nicht entscheidbar ist.
Es gibt also keine Möglichkeit, einem Auftrag anzusehen, ob er dazu
führen wird, dass der PDF-Renderer nicht terminieren wird.
\end{loesung}

\begin{diskussion}
Die Aussage, dass XSLT Turing-vollständig ist, ist ein wesentlicher Schritt.
Ohne diesen Schritt wäre es denkbar, dass, wegen des begrenzten Sprachumfanges
der Sprache XSLT, alle Programme terminieren, so wie dies bei der Sprache LOOP
festgestellt wurde.
Erst wenn eine Sprache Turing-vollständig ist, kann man mit dem Halte-Theorem
argumentieren, welches ja ausschliesslich für Turing-Maschinen gilt.

Man könnte versucht sein, den Satz von Rice anwenden zu wollen.
Dieser ist jedoch nur auf Turing-erkennbare Sprachen anwendbar.
Es muss eine Eigenschaft formuliert werden, die eine solche Sprache hat. 
Der Satz von Rice würde uns schliessen lassen, dass es nicht möglich ist
zu entscheiden, ob eine Sprache die Eigenschaft hat.
Zu jeder XML/XSLT-Kombination müsste eine Sprache gehören, doch
in der Problemstellung kommt eine solche Sprache gar nicht vor.
Die XML/XSLT-Programme haben ja keinen Input, der akzeptiert oder
verworfen werden könnte, und so eine Sprache definieren könnte.
\end{diskussion}

\begin{bewertung}
XSLT ist Turing-vollständig ({\bf V}) 2 Punkte,
Anwendung des Halte-Theorems ({\bf H}) 4 Punkte.
\end{bewertung}
