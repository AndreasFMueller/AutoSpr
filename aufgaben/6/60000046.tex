Bei der Übertragung von Zeichen könnte man den Ausfall einzelner
Zeichen erkennen, wenn man sich darauf einigt, dass immer eine ungerade
Anzahl von Zeichen übermittelt wird.
Wird eine gerade Zahl von Zeichen empfangen, bedeutet dies, dass mindestens
ein Zeichen, vielleicht aber auch ein grössere, ungerade Zahl von Zeichen,
nicht übermittelt worden sind.
Ein Empfängerprogramm darf also nur Meldungen akzeptieren, welche eine
ungerade Länge haben.

Man möchte ein Testprogramm schreiben, welches Empfängerprogramme
zertifizieren kann.
Es soll genau jene Empfängerprogramme akzeptieren, die in allen Fällen
korrekt arbeiten.
Ist so ein Testprogramm möglich?

\begin{loesung}
Das Empfängerprogramm muss eine Sprache akzeptieren, welche die Eigenschaft
{\em ODD} hat, d.~h.~sie enthält nur Wörter ungerader Länge.
Diese Eigenschaft ist nichttrivial, denn die Sprache $L_1=\{\texttt{00}\}$
hat die Eigenschaft nicht, $L_2=\{\texttt{0}\}$ hat sie.
Nach dem Satz von Rice ist daher nicht entscheidbar, ob eine Turing-Maschine
eine Sprache mit der Eigenschaftt {\em ODD} akzeptiert.
Die Sprache $\text{\em ODD}_{\text{TM}}$ ist nicht entscheidbar.
\end{loesung}

\begin{bewertung}
Satz von Rice ({\bf R}) 2 Punkt,
Zwei Beispiele von Sprachen ({\bf L}) 2 Punkt,
Schlussfolgerung ({\bf S}) 2 Punkte.
\end{bewertung}

