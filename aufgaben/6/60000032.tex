Die SlemBunk Malware für Android gehört zu einem ausgefeilten
System, mit dem Internet Banking unterwandert werden kann.
Die App enthält keine ``verdächtige'' Funktionalität als direkt
einsehbaren Code. 
Vielmehr ist der Code, der die endgültige Malware herunterlädt,
unscheinbar verpackt, und wird von der App erst zur Laufzeit
ausgepackt und in den Speicher geladen, wo er dann auch ausgeführt
werden kann.
Wegen dieses komplizierten Verfahrens schöpft der App-Scanner von
Android keinen Verdacht.
Gibt es eine Möglichkeit, den Scanner so zu erweitern, dass er 
von jeder App erkennen kann, ob sie je einen Internet-Download
ausführen wird?

\begin{loesung}
Nein, den damit liesse sich das Halteproblem lösen.
Ein Programm $P_1$ kann man immer in ein grösseres Programm $P_2$
einbetten, welches einen Internet-Download ausführt, sobald $P_1$
anhält.
Wendet man den Scanner auf $P_2$ an,
kann man herausfinden, ob $P_2$
je einen Internet-Download ausführen wird, und damit auch, ob
$P_1$ je anhalten wird.
Damit ist das Halteproblem gelöst.
Da das Halteproblem nicht lösbar ist, kann es auch keinen solchen
Scanner geben.
\end{loesung}

\begin{bewertung}
Reduktion auf Halteproblem ({\bf R}) 3 Punkte,
Halteproblem ist nicht entscheidbar ({\bf H}) 3 Punkte.
\end{bewertung}



