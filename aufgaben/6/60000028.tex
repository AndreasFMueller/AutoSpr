Formulieren Sie die folgenden Probleme als Sprachprobleme
\begin{teilaufgaben}
\item Welche natürlichen Zahlen sind Quadrate einer natürlichen Zahl?
\item Falls $n\in\mathbb N$ eine Quadratzahl ist, finde man die Wurzel.
\item Hat die quadratische Gleichung $ax^2+bx+c=0$ mit $a,b,c\in\mathbb N$
reelle Lösungen?
\item Hat die Gleichung $a^n+b^n=c^n$ ganzzahlige Lösungen, wobei mindestens
eine der Zahlen $a$, $b$ oder $c$ grösser als $1$ sein muss?
\item Man finde die Primfaktoren einer Zahl $n$.
\end{teilaufgaben}

\thema{Sprachproblem}

\begin{hinweis}
Es wird nicht verlangt, eine Lösungsalgorithmus für das Problem zu
formulieren.
\end{hinweis}

\begin{loesung}
\begin{teilaufgaben}
\item Sie $L$ die Sprache über dem Alphabet $\{\texttt{0}, \texttt{1}\}$
gegeben durch
\[
L=\{ w\in\Sigma^*
\mid
\text{$w$ ist die Binärdarstellung einer Quadratzahl}
\}
\]
Das Problem wird entscheiden von einer Turing-Maschine, die Binärzahlen auf
dem Band analysiert, ob sie Quadratzahlen sind, und im Zustand
$q_{\text{accept}}$ stehenbleibt, falls dies zutrifft.
\item
Sei $\Sigma=\{\texttt{0},\texttt{1},\texttt{:}\}$ und
\[
L=\left\{w\in\Sigma^*\;\left|\;
\begin{minipage}{4truein}
$w$ ist von der Form $w_1\texttt{:}w_2$, wobei $w_i$ Binardarstellungen
von Zahlen $n_i$ sind mit $n_1=n_2^2$.
\end{minipage}
\right.\right\}.
\]
Diese Sprache kann von einer TM entschieden werden, welche $n_2^2$ 
berechnet und genau dann im Zustand $q_{\text{accept}}$ anhält, wenn
das Resultat mit $n_1$ übereinstimmt.
\item 
Sei $\Sigma=\{\texttt{0},\texttt{1},\texttt{:}\}$ und 
\[
L=\left\{w\in\Sigma^*\;\left|\;
\begin{minipage}{4truein}
$w$ ist von der Form $a\texttt{:}b\texttt{:}c$, wobei $a$, $b$ und $c$
Binärdarstellungen der Koeffizienten einer quadratischen Gleichung sind,
die reelle Lösungen hat.
\end{minipage}
\right.\right\}
\]
Diese Sprache kann mit einer TM entschieden werden, die die Diskriminante
$b^2-4ac$ berechnet und im Zustand $q_{\text{accept}}$ stehen bleibt genau
dann, wenn die Diskriminante $\ge 0$ ist.
\item 
Sei $\Sigma=\{\texttt{0},\texttt{1}\}$ und
\[
L=\left\{ w\in\Sigma^*\;\left|\;
\begin{minipage}{4truein}
$w$ ist die Binärdarstellung einer natürlichen Zahl $n$, für die
die Gleichung $a^n+b^n=c^n$ eine ganzzahlige Lösung hat, wobei mindestens
eine der Zahlen $a$, $b$ oder $c$ grösser als $1$ sein muss.
\end{minipage}
\right.\right\}.
\]
Dank der Leistung von Andrew Wiles, der das Fermatsche Problem gelöst hat,
ist es heutzutage möglich, einen
Entscheider für $L$ anzugeben: für $w=2$ muss er im Zustand
$q_{\text{accept}}$ anhalten, andernfalls im Zustand $q_{\text{reject}}$.
Ohne dieses Wissen bleibt nur, Tripel von ganzen Zahlen $(a,b,c)$
durchzuprobieren. Falls eine Lösung existiert, wird sie früher oder später
gefunden. Falls keine Lösung existiert, wird dieser Algorithmus nicht
anhalten. Die Sprache kann also auf diese naive Weise zwar erkannt, aber
nicht entschieden werden.
\item 
Sei $\Sigma=\{\texttt{0},\texttt{1},\texttt{:}\}$ und 
\[
L=\left\{ w\in\Sigma^*\;\left|\;
\begin{minipage}{4truein}
$w$ ist von der Form
$n\texttt{:}p_1\texttt{:}n_1\texttt{:}p_2\texttt{:}n_2\texttt{:}\dots\texttt{:}p_k\texttt{:}n_k$, und es gilt $n=p_1^{n_1}p_2^{n_2}\dots p_k^{n_k}$,
wenn man die $p_i$ und $n_i$ als Binärzahlen interpretiert.
\end{minipage}
\right.\right\}.
\]
Die Sprache $L$ wird entschieden von einer Turing-Maschine, die das Produkt
$p_1^{n_1}p_2^{n_2}\dots p_k^{n_k}$ ausrechnet und genau dann im Zustand
$q_{\text{accept}}$, wenn das Resultat mit $n$ übereinstimmt.
\qedhere
\end{teilaufgaben}
\end{loesung}

