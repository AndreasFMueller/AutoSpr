Seien $L_1$ und $L_2$ zwei Turing-entscheidbare Sprachen. Zeigen Sie, dass
die folgenden Sprachen Turing-entscheidbar sind:
\begin{teilaufgaben}
\item $L_1\cup L_2$
\item $L_1L_2$
\item $L_1^*$
\item $\Sigma^*\setminus L_1$
\item $L_1\cap L_2$
\end{teilaufgaben}

\thema{Turing-entscheidbar}

\begin{hinweis}
Turing-entscheidbar heisst, dass es zwei Funktionen 
\begin{verbatim}
boolean l1(String w);
boolean l2(String w);
\end{verbatim}
gibt, welche genau dann \texttt{true} zurückgeben, wenn
\texttt{w} in $L_1$ bzw.~$L_2$ ist. Die Aufgabe
besteht dann darin, für jede Sprache $L$ in den Teilaufgaben eine neue
Funktion
\begin{verbatim}
boolean l(String w) {
  ....
}
\end{verbatim}
zu programmieren, welche genau dann \texttt{true} zurückgibt, wenn
\texttt{w} in $L$ ist.
\end{hinweis}

\begin{loesung}
\begin{teilaufgaben}
\item Um zu entscheiden, ob ein Wort $w$ in $L_1\cup L_2$ ist,
lassen wir zuerst $M_1$ auf dem Input $w$ laufen, und falls
diese das Wort verwirft, $M_2$. Akzeptiert eine der Maschinen
das Wort, wird es akzeptiert, andernfalls wird es verworfen.

Mit den Notationen des Hinweises könnte man dieses Problem mit folgendem
Code lösen:
\verbatimainput{a.java}
\item
Für jede der $|w|+1$ Unterteilungen des Wortes $w$ in zwei
Teilwörter $xy=w$ lassen wir $M_1$ auf $x$ und $M_2$ auf
$y$ laufen. Akzeptieren die Maschinen für eine der
Unterteilungen das Wort, akzeptieren wir, wir verwerfen, wenn
für keine Unterteilung beide Maschinen akzeptieren.

Mit den Notationen des Hinweises könnte man dieses Problem mit folgendem
Code lösen:
\verbatimainput{b.java}
\item
Wenn $w\in L_1^*$, dann kann man $w$ schreiben als
$w=x_1\dots x_n$, wobei $x_i\in L_1$ für alle $i$ und
$x_i\ne \varepsilon$.
Es gibt endlich viele Unterteilungen von $w$ in maximal $n$ Teilwörter
mit einer Länge von mindestens einem Zeichen.
Für alle solchen Unterteilungen testen wir jede Komponente $x_i$ mit der
Turingmaschine $M_1$.
Falls $M_1$ alle Komponenten einer Unterteilung
akzeptiert, akzeptieren wir das Wort, wenn dies in keinem Fall
vorkommt, verwerfen wir das Wort.

Man kann dieses Problem sehr elegant mit Rekursion lösen.
Die Funktion untersucht alle Unterteilungen des Wortes in
zwei Teile $x_1$ und $x_2$, indem sie zuerst den Entscheider
für $L_1$ mit Input $x_1$ aufruft, und dann sich selbst
mit Input $x_2$ aufruft. Diese Rekursion ist endlich.

Mit den Notationen des Hinweises könnte man dieses Problem mit folgendem
Code lösen:
\verbatimainput{c.java}
\item
Für die Sprache $\Sigma^*\setminus L_1$ verwenden wir die
Turing Maschine $\bar M_1$, in welcher gegenüber $M_1$
die Zustände $q_{\text{accept}}$ und $q_{\text{reject}}$
ihre Rollen vertauschen. $\bar M_1$ ist ein Entscheider,
der genau die Wörter von $\Sigma^*\setminus L_1$ akzeptiert.

Mit den Notationen des Hinweises könnte man dieses Problem mit folgendem
Code lösen:
\verbatimainput{d.java}
\item
Wir lassen $M_1$ auf $w$ laufen und anschliessen $M_2$ auf $w$.
Wenn beide Maschinen akzeptieren, akzeptieren wir das Wort, andernfalls
verwerfen wir.

Mit den Notationen des Hinweises könnte man dieses Problem mit folgendem
Code lösen:
\verbatimainput{e.java}
\end{teilaufgaben}
\end{loesung}

