Seien $L_1$ und $L_2$ zwei Turing-entscheidbare Sprachen. Zeigen Sie, dass
die folgenden Sprachen Turing-entscheidbar sind:
\begin{teilaufgaben}
\item $L_1\cup L_2$
\item $L_1L_2$
\item $L_1^*$
\item $\Sigma^*\setminus L_1$
\item $L_1\cap L_2$
\end{teilaufgaben}

\begin{hinweis}
Turing-entscheidbar heisst, dass es zwei Funktionen 
\begin{verbatim}
boolean l1(String w);
boolean l2(String w);
\end{verbatim}
gibt, welche genau dann \texttt{true} zur"uckgeben, wenn
\texttt{w} in $L_1$ bzw.~$L_2$ ist. Die Aufgabe
besteht dann darin, f"ur jede Sprache $L$ in den Teilaufgaben eine neue
Funktion
\begin{verbatim}
boolean l(String w) {
  ....
}
\end{verbatim}
zu programmieren, welche genau dann \texttt{true} zur"uckgibt, wenn
\texttt{w} in $L$ ist.
\end{hinweis}

\begin{loesung}
\begin{teilaufgaben}
\item Um zu entscheiden, ob ein Wort $w$ in $L_1\cup L_2$ ist,
lassen wir zuerst $M_1$ auf dem Input $w$ laufen, und falls
diese das Wort verwirft, $M_2$. Akzeptiert eine der Maschinen
das Wort, wird es akzeptiert, andernfalls wird es verworfen.

Mit den Notationen des Hinweises k"onnte man dieses Problem mit folgendem
Code l"osen:
\verbatimainput{a.java}
\item
F"ur jede der $|w|+1$ Unterteilungen des Wortes $w$ in zwei
Teilw"orter $xy=w$ lassen wir $M_1$ auf $x$ und $M_2$ auf
$y$ laufen. Akzeptieren die Maschinen f"ur eine der
Unterteilungen das Wort, akzeptieren wir, wir verwerfen, wenn
f"ur keine Unterteilung beide Maschinen akzeptieren.

Mit den Notationen des Hinweises k"onnte man dieses Problem mit folgendem
Code l"osen:
\verbatimainput{b.java}
\item
Wenn $w\in L_1^*$, dann kann man $w$ schreiben als
$w=x_1\dots x_n$, wobei $x_i\in L_1$ f"ur alle $i$ und
$x_i\ne \varepsilon$.
Es gibt endlich viele Unterteilungen von $w$ in maximal $n$ Teilw"orter
mit einer L"ange von mindestens einem Zeichen.
F"ur alle solchen Unterteilungen testen wir jede Komponente $x_i$ mit der
Turingmaschine $M_1$.
Falls $M_1$ alle Komponenten einer Unterteilung
akzeptiert, akzeptieren wir das Wort, wenn dies in keinem Fall
vorkommt, verwerfen wir das Wort.

Man kann dieses Problem sehr elegant mit Rekursion l"osen.
Die Funktion untersucht alle Unterteilungen des Wortes in
zwei Teile $x_1$ und $x_2$, indem sie zuerst den Entscheider
f"ur $L_1$ mit Input $x_1$ aufruft, und dann sich selbst
mit Input $x_2$ aufruft. Diese Rekursion ist endlich.

Mit den Notationen des Hinweises k"onnte man dieses Problem mit folgendem
Code l"osen:
\verbatimainput{c.java}
\item
F"ur die Sprache $\Sigma^*\setminus L_1$ verwenden wir die
Turing Maschine $\bar M_1$, in welcher gegen"uber $M_1$
die Zust"ande $q_{\text{accept}}$ und $q_{\text{reject}}$
ihre Rollen vertauschen. $\bar M_1$ ist ein Entscheider,
der genau die W"orter von $\Sigma^*\setminus L_1$ akzeptiert.

Mit den Notationen des Hinweises k"onnte man dieses Problem mit folgendem
Code l"osen:
\verbatimainput{d.java}
\item
Wir lassen $M_1$ auf $w$ laufen und anschliessend $M_2$ auf $w$.
Wenn beide Maschinen akzeptieren, akzeptieren wir das Wort, andernfalls
verwerfen wir.

Mit den Notationen des Hinweises k"onnte man dieses Problem mit folgendem
Code l"osen:
\verbatimainput{e.java}
\end{teilaufgaben}
\end{loesung}

