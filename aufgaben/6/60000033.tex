Der Herausgeber eines Konferenz-Bandes muss die einzelnen Beiträge der 
Konferenzteilnehmer zu einem PDF File zusammenstellen, welches dem Drucker
übergeben werden kann.
Die Beiträge werden also \LaTeX-Sourcecode angeliefert.
Der Herausgeber sorgt sich, dass die \LaTeX-Files nicht nur den Text
der Artikel enthalten, sondern auch Code, der auf dem PC des Herausgebers
Schaden anrichten könnte.
Daher sucht er auf Google nach einer Scanner-Software, der solchen Schadcode  in
\LaTeX-Files entdecken könnte.
Er findet zwar einige Anfragen in exotischen Foren, die Forenthreads
enthalten wie so oft nichts ausser ``das würde mich auch interessieren'' und,
meist nach Jahren, ``hast Du dafür schon eine Lösung gefunden?''
Warum ist das nicht überraschend?

\thema{Halteproblem}
\themaL{Turing-vollstandig}{Turing-vollständig}
\thema{Reduktion}

\begin{loesung}
Die Sprache \LaTeX{} ist Turing-vollständig.
Die Aufgabe, Schadcode zu erkennen läuft also darauf hinaus, das Halteproblem
zu lösen.
Da das Halteproblem nicht entscheidbar ist, kann es einen solchen Scanner
nicht geben.
\end{loesung}

\begin{bewertung}
Turing-Vollständigkeit von \LaTeX~{\bf L} 2 Punkte (nur ein Punkt, wenn
als Annahme formuliert),
(informelle) Reduktion auf das Halte- oder Akzeptanz-Problem ({\bf R}) 2 Punkte,
Halteproblem nicht entscheidbar ({\bf H}) 1 Punkt,
Schlussfolgerung für die Aufgabe ({\bf S}) 1 Punkt.
\end{bewertung}

