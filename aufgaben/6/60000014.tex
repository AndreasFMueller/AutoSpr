Ein Sudoku-Tester ist ein Programm, welches bei einem vollständig
ausgefüllten Sudoku-Spielplan entscheidet, ob es richtig ausgefüllt
worden ist, d.~h.~ob alle Regeln eingehalten worden sind. Können Sie
einen Algorithmus angegeben, mit dem man Sudoku-Tester auf Korrektheit
prüfen kann? Dabei ist das mathematische Sudoku-Problem nicht auf
Spielpläne mit $9\times 9$ Feldern beschränkt,
sondern kann für beliebig grosse
$n^2\times n^2$-Spielpläne ($n\times n$ Unterquadrate von je $n\times n$
Feldern) gestellt werden.

\begin{loesung}
Nein.  Die Eigenschaft \textsl{SUDOKU} sei die Eigenschaft einer
Sprache $L$, dass jedes Wort $w\in L$ ein korrektes Sudoku-Feld
beschreibt. Dabei verwenden wir folgende einfache
Codierung für ein Sudoku-Feld: wir schreiben die Ziffern in
einem Sudoko-Feld zeilenweise in ein einziges Wort.
Die Aufgabe fragt danach, ob \textsl{SUDOKU} entscheidbar sei.

Wir wollen zeigen, dass die
Eigenschaft $\textsl{SUDOKU}$ nicht trivial ist.
Das einzige Wort der Sprache $L_1=\{w\}$  steht für das korrekt
ausgfüllte Sudoku
\[
\begin{tabular}{|c|c|c|c|c|c|c|c|c|}
\hline
\raisebox{0pt}[13pt][5pt]{5}&3&4&6&7&8&9&1&2\\
\hline
\raisebox{0pt}[13pt][5pt]{6}&7&2&1&9&5&3&4&8\\
\hline
\raisebox{0pt}[13pt][5pt]{1}&9&8&3&4&2&5&6&7\\
\hline
\raisebox{0pt}[13pt][5pt]{8}&6&9&7&6&1&4&2&3\\
\hline
\raisebox{0pt}[13pt][5pt]{4}&2&6&8&5&3&7&9&1\\
\hline
\raisebox{0pt}[13pt][5pt]{7}&1&3&9&2&4&8&5&6\\
\hline
\raisebox{0pt}[13pt][5pt]{9}&6&1&5&3&7&2&8&4\\
\hline
\raisebox{0pt}[13pt][5pt]{2}&8&7&4&1&9&6&3&5\\
\hline
\raisebox{0pt}[13pt][5pt]{3}&4&5&2&8&6&1&7&9\\
\hline
\end{tabular}
\]
also
\[
w=
534678912672195348198342567869761423426853791713924856961537284287419635345286179
\]
Andererseits hat
\[
L_2=\{1^92^93^94^95^96^97^98^99^9\}
\]
die Eigenschaft \textsl{SUDOKU} nicht. Nach dem Satz
von Rice ist also \textsl{SUDOKU} nicht entscheidbar.

Genau genommen haben wir sogar eine stärkere Aussage bewiesen.
Die untersuchte Eigenschaft besagte, dass ein Sudoku-Tester-Programm nicht
alle korrekt ausgefüllten Sudokus erkennt, es darf auch einige
korrekt ausgefüllte Sudokus ``übersehen''. Unser Resultat sagt dann,
dass wir nicht einmal solche ``lückenhaften'' Sudoko-Tester von solchen
unterscheiden können, die auch ein inkorrekt ausgefülltes Sudoku
akzeptieren.
\end{loesung}
