In einer verteilten Anwendung sollen Log-Meldungen zentral gesammelt
und ausgewertet werden.
Die Meldungen könnten leicht mit etwa einem Dutzend ziemlich
komplexer regulärer Ausdrücke klassifiziert werden.
Die zugehörigen endlichen Automaten sind jedoch komplex genug, dass
sich bereits Performance-Engpässe abzeichnen.
Daher sollen jetzt alternative Algorithmen untersucht werden.
Um sicherzustellen, dass durch diesen Umbau nicht neue Fehler
eingeführt werden, wird verlangt, dass ein Testprogramm geschrieben
wird, welches vorgeschlagene Algorithmen darauf hin prüfen soll,
ob sie genau die Log-Messages akzeptieren, die auch die genannten
regulären Ausdrücke akzeptieren würden.
Ist so etwas möglich?

\thema{Reduktion}
\thema{Satz von Rice}

\begin{loesung}
Nein.
Die Algorithmen müssen eine Sprache akzeptieren, die die Eigenschaft
\begin{center}
P = \text{``besteht nur aus Wörtern, die von den regulären Ausdrücken akzeptiert werden''}
\end{center}
haben.
Sofern die regulären Ausdrücke nicht alle Wörter akzeptieren, was man
hier annehmen darf, ist dies eine nichttriviale Eigenschaft.
Nach dem Satz von Rice kann es daher kein Programm geben, welches
entscheiden kann, ob eine Maschine eine Sprache mit der Eigenschaft
$P$ akzeptiert.
\end{loesung}





