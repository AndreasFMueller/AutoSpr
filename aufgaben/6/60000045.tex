Eine {\em endliche} Sprache hat nur endlich viele Wörter.
\begin{teilaufgaben}
\item
Wie kann man entscheiden, ob ein regulärer Ausdruck eine endliche
Sprache akzeptiert?
\item
Wie kann man mit einem Algorithmus entscheiden, ob ein DEA
eine endliche Sprache akzeptiert?
\item
Man nennt eine Turing-Maschine {\em endlich}, wenn sie eine endliche
Sprache akzeptiert.
Können Sie einen Algorithmus konstruieren, der entscheidet, ob eine
Turing-Maschine $M$ endlich ist, d.~h.~eine endliche Sprache akzeptiert?
\end{teilaufgaben}

\thema{Entscheidbarkeit}
\thema{Satz von Rice}

\begin{loesung}
\begin{teilaufgaben}
\item
Ein regulärer Ausdruck akzeptiert genau dann eine endliche Sprache, 
wenn er keinen Stern enthält.
\item
Ist $r$ der reguläre Ausdruck, der die gleiche Sprache akzeptiert
wie der DEA $A$, dann ist die Sprache $L(r)=L(A)$ genau dann endlich,
wenn der Ausdruck $r$ keinen Stern enthält.
\item
Nein, dies ist wegen des Satzes von Rice nicht möglich.
Die Eigenschaft, endlich zu sein, ist eine nichttriviale Eigenschaft
von Sprachen.
Die leere Sprache $L_1=\emptyset$ ist endlich, die Sprache $L_2=\Sigma^*$
ist unendlich.
Nach dem Satz von Rice ist die Eigenschaft, eine endliche Sprache zu
akzeptieren, nicht entscheidbar, es kann also keinen Algorithmus der
verlangten Art geben.
\qedhere
\end{teilaufgaben}
\end{loesung}

\begin{bewertung}
\begin{teilaufgaben}
\item
Detektieren von ``Sternen'' ({\bf S}) 1 Punkt.
\item
Umwandlung in einen regulären Ausdruck ({\bf R}) 1 Punkt.
\item
Satz von Rice ({\bf R}) 1 Punkt,
nichttriviale Eigenschaft ({\bf T}) 1 Punkt,
zwei Beispiele für Sprachen ({\bf L}) 1 Punkte,
Nichtentscheidbarkeit ({\bf N}) 1 Punkt.
\end{teilaufgaben}
In der Prüfung haben viele Studenten die Aufgabe anders interpretiert.
Sie haben angenommen, dass es z.~B.~in a) darum geht, herauszufinden,
ob der reguläre Ausdruck $r$ eine andere, endliche Sprache $L$ akzeptiert,
also ob $L(r) \supset L$ gilt.
Dies kann man natürlich sehr leicht entscheiden, indem man alle Wörter
der Sprache $L$ ausprobiert.
Da die Sprache $L$ nur endlich viele Wörter hat, ist das ein
Algorithmus, der terminiert, also ein Entscheider.
Das gleiche gilt für b).
Falls aus der Lösung erkennbar war, dass die Aufgabe so verstanden wurde,
wurde in a) und b) je maximal ein Punkt gegeben und mit ({\bf X})
gekennzeichnet.
\end{bewertung}
