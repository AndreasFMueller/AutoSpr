Sei $\Sigma=\{\text{\tt 0},\text{\tt 1}\}$, und sei $L$ eine Sprache
mit dem Alphabet $\Sigma$. Zu jeder natürlichen Zahl $n\in\mathbb N$
kann man zählen, wieviele Wörter mit Länge $\le n$ die Sprache
hat:
\[
l(n)=|\{w\in L|\;|w|\le n\}|.
\]
Man sagt, die Sprache $L$ habe polynomielles Wachstum, wenn
\[
l(n)=O(n^k)
\]
für ein geeignetes $k$. Ist entscheidbar, ob eine Sprache polynomielles
Wachstum hat?

\thema{Entscheidbarkeit}
\thema{Satz von Rice}

\begin{loesung}
Wir zeigen mit dem Satz von Rice, dass die Eigenschaft, polynomielles
Wachstum zu haben, entscheidbar ist. Dazu müssen wir zwei Sprachen
angeben, eine Sprache, die die Eigenschaft hat, und eine, die die
Eigenschaft nicht hat.

Die Sprache $L_0=\emptyset$ enthält gar keine Wörter, für sie
gilt also $l(n)=0$, diese Sprache hat also sicher polynomielles
Wachstum, es ist sogar $l(n)=O(1)$.

Die Sprache $L_1=\{\text{\tt 1}\}^*$ ist sicher auch eine Sprache mit
dem Alphabet $\Sigma$, also $L_1\subset \Sigma^*$. Die Wörter
von $L_1$ sind die natürlichen Zahlen in unärer Darstellung,
es gibt also immer genau $n+1$ Wörter der Länge $\le n$, also
$l(n)=n+1=O(n)$. Auch die Sprache $L_1$ hat also polynomielles Wachstum.

Man kann sogar für jeden beliebigen Exponenten $k\ge 1$ eine Sprache $L_k$
mit Wachstum $O(n^k)$ angeben. Dazu sei $w$ ein Wort in $\Sigma^*$,
wir bezeichnen mit $v(w)$ den Zahlenwert, den $w$ im Zweiersystem
darstellt. Dann nehmen wir nur jene Wörter in die Sprache,
deren Wert kleiner als die Länge hoch $k$ ist:
\[
L_k=\{ w\in\Sigma^*| v(w) < |w|^{k-1}\}
\]
Offenbar gibt es von jeder Länge $n$ genau $n^{k-1}$ Wörter.
Somit ist
\[
l(n)=\sum_{i=0}^ni^{k-1}\le n\cdot n^{k-1}=n^k\quad\Rightarrow\quad l(n)=O(n^k).
\]
Auch ist bekannt, dass die Summe der $(k-1)$-ten Potenzen ein
Polynom vom Grad $k$ ist, was wiederum zeigt, dass $l(n)=O(n^k)$.
Setzt man übrigens $k=1$ ergibt
sich wieder das bereits im letzten Abschnitt betrachtete
$L_1$.

Die Sprache $L_\infty=\Sigma^*$ besteht aus allen {\tt 0}-{\tt 1}-Folgen.
Es gibt jeweils $2^n$ Wörter der Länge $n$, oder $2^{n+1}-1$ Wörter
der Länge $n$, also $l(n)=2^{n+1}-1$. $2^n$ wächst aber schneller
als jede Potenz von $n$, somit hat die Sprache $L_2$ kein polynomielles
Wachstum.

$L_k$ und $L_\infty$ sind offenbar zwei Sprachen, wie sie der Satz von
Rice verlangt, somit kann man schliessen, dass die Eigenschaft
nicht entscheidbar ist.
\end{loesung}
