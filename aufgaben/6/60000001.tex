Die Sprache
\[
L=\{a^nb^nc^n\,|\,n\in \mathbb N\}
\]
ist nicht kontextfrei.
Konstruieren sie eine TM, die $L$ entscheidet.

\thema{Turing-Maschine}

\begin{loesung}
Die Maschine verwendet als Bandalphabet $\{a,b,c,\blank,x\}$,
und arbeitet nach folgendem Algorithmus:
\begin{enumerate}
\item Falls das nächste Zeichen ein $\blank$ ist:
$q_{\text{accept}}$
\item Falls das nächste Zeichen ein $b$, $c$ oder $x$ ist:
$q_{\text{reject}}$.
\item Falls das nächste zu lesende Zeichen ein $a$ ist, ersetze
es durch $\blank$ und fahre nach rechts.
\item Nach rechts fahrend, überspringe alle $a$ und $x$.
\item Falls das nächste Zeichen kein $b$ ist:
$q_{\text{reject}}$.
\item Falls das nächste Zeichen ein $b$ ist, ersetze es durch $x$,
bewege den Kopf nach rechts.
\item Nach rechts fahrend, überspringe alle $b$
\item Falls das nächste Zeichen kein $c$ ist:
$q_{\text{reject}}$.
\item Nach rechts fahrend, überspringe alle $c$.
\item Falls das nächste Zeichen kein $\blank$ ist: $q_{\text{reject}}$
\item Falls das nächste Zeichen ein $\blank$ ist, fahre nach
links.
\item Falls das Zeichen kein $c$ ist: $q_{\text{reject}}$
\item Ersetze $c$ durch $\blank$, fahre nach links.
\item Nach links fahrend, überspringe alle Zeichen bis zum nächsten
$\blank$.
\item Bewege den Kopf nach rechts, und fahre weiter bei 1.
\end{enumerate}
Offenbar ist diese TM auch ein Entscheider.

Das Zustandsdiagramm dieser TM ist
\[
\entrymodifiers={++[o][F]}
\xymatrix{
*+\txt{}
        &*++[o][F=]{q_{\text{accept}}}
\\
*+\txt{}
        &{q_6}  \ar@(ur,dr)^{x\to x,R}
                \ar[u]^{\blank\to\blank,R}
                \ar@/_80pt/[ddd]
\\
*+\txt{}\ar[r]
        &q_0    \ar[r]^{a\to\blank,R}
                \ar[u]^{x\to x,R}
                \ar[dd]
                &q_1    \ar@(ur,ul)_{a\to a,R}
                        \ar@(dr,dl)^{x\to x,R}
                        \ar[r]^{b\to x,R}
                        \ar[ddl]
                        &q_2    \ar@(ur,ul)_{b\to b,R}
                                \ar@(dr,dl)^{c\to c,R}
                                \ar[r]^{c\to c,R}
                                \ar[ddll]
                                &q_3    \ar[d]^{\blank\to\blank,L}
                                        \ar@(u,r)^{c\to c,R}
                                        \ar[ddlll]
\\
*+\txt{}
        &*+\txt{}
                &*+\txt{}
                        &*+\txt{}
                                &q_4    \ar[d]^{c\to\blank,L}
                                        \ar[dlll]
\\
*+\txt{}
        &*++[o][F=]{q_{\text{reject}}}
                &*+\txt{}
                        &*+\txt{}
                                &q_5    \ar[llluu]^{\blank\to\blank,R}
                                        \ar@(ur,dr)^{.\to.,L}
}\]
Der "Ubergang bei $q_5$ ist zu lesen als: für alle Zeichen ausser dem
$\blank$, bewege den Kopf nach links.
\end{loesung}
