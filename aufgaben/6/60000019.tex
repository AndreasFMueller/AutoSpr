Es ist bekannt, dass es eine Sprache
$L\subset \{\text{\tt 1}\}^*$ gibt, die nicht entscheidbar ist.
Man kann dies zum Beispiel so einsehen:
jede solche Sprache entspricht einer Teilmenge von $\mathbb N$,
und man verwendet, dass die Menge der Teilmengen von $\mathbb N$,
die Potenzmenge $P(\mathbb N)$ überabzählbar ist.
Verwenden Sie dieses
Resultat um zu zeigen, dass es eine reelle Zahl $r$ gibt, die von keinem
Computer berechnet werden kann in dem Sinne, dass es keine Turing-Maschine
gibt, die nacheinander beliebig viele Stellen der Binärdarstellung von $r$
berechnen
kann.

\begin{loesung}
Sei $L$ eine nicht entscheibare Sprache. Natürlich muss $L$ eine
unendliche Sprache sein, denn endliche Sprachen sind ja alle regulär
also erst recht entscheidbar.

Wir ordnen die Wörter der Grösse nach, und bilden dann folgende
Binärzahl $r$. Hinter dem Komma schreiben wir die Wörter von $L$
getrennt durch jeweils eine {\tt 0} hin. Wenn also $L$ mit
den Wörtern
\[
\{
\text{\tt 1},
\text{\tt 111},
\text{\tt 111111},
\text{\tt 1111111},
\text{\tt 11111111111},
\dots
\}
\]
beginnt, dann ist
\[
r=\text{\tt 0},\text{\tt 101110111111011111110111111111110}\dots
\]
Nehmen wir an, eine Turing-Maschine $M$ könnte beliebig viele Stellen
der Zahl ausrechnen. Dann können wir daraus einen Entscheider $M'$ für
die Sprache konstruieren. Dieser muss zu einem Wort $w=1^n$ sagen,
ob es zur Sprache $L$ gehört. $M'$ arbeitet wie folgt:

\begin{quote}
Wir lassen daher die Maschine $M$ laufen, die die Stellen von $r$ berechnet.
Jedesmal, wenn die Maschine eine
{\tt 0} schreibt, vergleichen wir die Länge $l$ der vorngegangenen
{\tt 1}er-Folge mit $n$. Falls $l=n$, haben wir ein Wort der Sprache,
und wir akzeptieren $w$. Falls $l<n$ sind wir sicher, müssen wir
die Maschine noch länger laufen lassen. Falls aber $l>n$ haben wir
das Wort nicht gefunden. Weil die {\tt 1}er-Folgen in $r$ immer
länger werden, wird es auch nie mehr gefunden werden können, weshalb
wir $w$ verwerfen müssen.
\end{quote}

Da die Länge der {\tt 1}er-Folgen in $r$ immer zunimmt, muss $l=n$
oder $l>n$ irgendwann eintreten, $M'$ wird als auf jeden Fall
anhalten. $M'$ wird genau dann akzeptieren, wenn $w\in L$, somit
ist $M'$ eine Entscheider.

Somit haben wir einen Entscheider $M'$ für die nicht entscheidbare Sprache
$L$ gefunden. Dieser Widerspruch zeigt, dass die ursprüngliche Annahme,
nämlich dass es eine die Maschine $M$ gäbe, nicht richtig sein kann.
Es gibt also keine Turing-Maschine, welche die Stellen der Zahl $r$ berechnen
kann.

Ein noch einfacheres Argument ist das folgende. Zur Berechnung von
rellen Zahlen braucht man Turing-Maschinen, denen man als Input auf
dem Band die gewünschte Anzahl stellen geben kann. Es gibt nur
abzählbar viele Turing-Maschinen, also können nur abzählbar viele
reelle Zahlen berechnet werden. Da es überabzählbar viele reelle
Zahl gibt, können die meisten gar nicht berechnet werden.
\end{loesung}
