Es gibt verschiedene Algorithmen, mit denen man herausfinden kann, ob
eine natürliche Zahl eine Quadratzahl ist.
Ein Programm heisst Quadratzahldetektor, wenn es Quadratzahlen
als Input meistens erkennt, aber niemals eine Nicht-Qua\-dratzahl
als Quadratzahl ausgibt. Gibt es eine Software, mit der man automatisiert
herausfinden kann, ob so ein Quadratzahldetektor korrekt arbeitet?

\thema{Entscheidbarkeit}
\thema{Satz von Rice}

\begin{loesung}
Dies ist nicht möglich. 
Natürliche Zahlen können mit Hilfe der Dezimaldarstellung als 
Zeichenketten über dem Alphabet $\Sigma=\{0,1,2,3,\dots,9\}$
betrachtet werden.
Die Menge der Zeichenketten, die ein Programm $M$ als Quadratzahlen
erkennt, bildet eine Teilmenge $L_M\subset\Sigma^*$, also eine Sprache
über dem Alphabet $\Sigma$. Ein Quadratzahldetektor liegt vor, wenn
die Menge $L_M$ ausschliesslich Quadratzahlen enthält, möglicherweise
aber nicht alle. Quadratzahldetektoren $M$ sind also durch Eigenschaft
$Q$ der Sprache $L_M$ charakterisiert, dass $L_M$ nur Quadratzahlen
enthält. Eine Sprache $L$ hat die Eigenschaft $Q$ genau dann, wenn
alle Zeichenketten in $L$ als Dezimalzahlen gelesen Quadratzahlen sind.

Die Eigenschaft $Q$ ist nicht trivial, denn die Sprache 
$L_2=\{2\}$ enthält eine Nichtquadratzahl, hat also die Eigenschaft $Q$
nicht, $L_1=\{1\}$ enthält dagegen aus\-schliess\-lich Quadratzahlen, hat
also die Eigenschaft $Q$.

Nach dem Satz von Rice ist nicht entscheidbar, ob eine Turing-erkennbare
Sprache die Eigenschaft $Q$ hat. Gleichbedeutend damit ist
nicht entscheidbar, ob ein Programm $M$ ein Quadratzahldetektor ist.
\end{loesung}
