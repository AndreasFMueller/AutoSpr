%
% pruefung.tex -- Uebungen zur Vorlesung Mathematische Grundlagen der Informatik 2
%
% (c) 2006 Prof. Dr. Andreas Mueller, HSR
% $Id: pruefung.tex,v 1.1 2009/02/26 22:29:55 afm Exp $
%
\input a-begin.tex
\input macros-uebungen.tex
\lhead{Probepr"ufung}
\rhead{Automaten und Sprachen}
\phantom{a}
\vspace{0.8cm}
{\parindent0pt\hbox to\hsize{%
Name: \hbox to7cm{\dotfill} Vorname: \dotfill}}
\vspace{1.5cm}
\pagenumbering{gobble}

\begin{center}
{\LARGE Automaten und Sprachen}

\vspace{0.5cm}
{\Large Probepr"ufung}

\vspace{1.5cm}
\end{center}

{\parindent 0pt
{\bf Abgabetermin:}
Donnerstag, 19.~Mai 2016.
\smallskip

{\bf Erlaubte Hilfsmittel:} Zusammenfassung maximal 16 Seiten A4.
Taschenrechner, aber keine Computer.
\smallskip

{\bf L"osungen:} Alle Antworten sind zu begr"unden.
Es werden vollst"andige, mit Papier und Bleistift
nachvollziehbare L"o\-sungs\-wege erwartet. Resultate ohne Begr"undung
z"ahlen nicht.

Die Aufgaben m"ussen
auf den ausgeteilten Aufgabenbl"attern gel"ost werden. Wenn zus"atzliche
Bl"atter verwendet werden, muss diese auf den Aufgabenbl"attern notiert
werden und die Zusatzbl"atter m"ussen mit Name und Aufgabennummer versehen
sein. Beginnen Sie jede Aufgabe auf einem neuen Blatt.

Dies Probepr"ufung hat den halben Umfang einer regul"aren Pr"ufung.
Es wird nicht kontrolliert, ob Sie sich an die erlaubten Hilfsmittel
oder die zur Verf"ugung stehende Zeit halten, der ``Pr"ufungsdruck'' ist
daher nicht wirklich gegeben.
Bitte ber"ucksichtigen Sie dies bei der Beurteilung Ihrer Leistung
und der Planung Ihrer Pr"ufungsvorbereitung.
}

\vspace{1cm}

\begin{center}
\begin{tabular}{|l|c|c|c|c|c|c|c|c|c|c|}
\hline
\raisebox{0pt}[13pt][6pt]{\phantom{XX}}&1&2&3&4&Total&Note\\
\hline
Punkte:&
\raisebox{0pt}[20pt][15pt]{\phantom{XX}}&
\raisebox{0pt}[20pt][15pt]{\phantom{XX}}&
\raisebox{0pt}[20pt][15pt]{\phantom{XX}}&
\raisebox{0pt}[20pt][15pt]{\phantom{XX}}&\phantom{XXX}&\phantom{XXXX}\\
\hline
\end{tabular}
\end{center}


\pagebreak
\keineloesungen

\input aufgaben.tex

\end{document}
