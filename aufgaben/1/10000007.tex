Zeigen Sie, dass die logischen Formeln
\[
x\vee y\qquad\text{und}\qquad (x\vee z)\wedge (y\vee \neg z)
\]
``äquivalent'' sind im folgenden Sinne: die Variablen der einen Formel
können genau dann so mit Wahrheitswerten
belegt werden, dass die Formel wahr wird, wenn dies für die andere Formel
möglich ist.
Man sagt, die Formeln sind {\em erfüllungsäquivalent}.
\index{erfüllungsäquivalent}%

\themaL{Erfullungsaquivalenz}{Erfüllungsäquivalenz}

\begin{loesung}
Wenn $x\vee y$ wahr ist, dann ist eine der Variablen, zum Beispiel
$x$ wahr.
Dann kann mann $z$ auf ``falsch'' setzen, was die zweite Formel wahr macht.
Ist stattdessen $y$ wahr, kann man $z$ auf wahr setzen, auch so wird die
zweite Formel wahr.
Ist die erste Formel dagegen falsch, dann ist weder $x$ noch $y$ wahr,
die zweite Formel ist also im wesentlichen $z\wedge \neg z$, was nie
wahr sein kann.

Ist umgekehrt die zweite Formel wahr, und auch $z$, dann ist $\neg z$ nicht 
wahr, also muss $y$ wahr sein. Dann ist auch $x\vee y$ wahr. Dasselbe
Argument funktioniert auch, wenn $z$ falsch ist, dann folgt, dass $x$
wahr sein muss, und wieder ist $x\vee y$ wahr.
Ist die zweite Formel immer falsch, dann gibt es keine Belegung von
$z$ so, dass beide Terme wahr sind.
D.\,h.~aber, dass $x$ und $y$ beide falsch sein müssen, also ist auch
$x\vee y$ falsch.

Alternativ kann man die ``"Aquivalenz'' im obigen Sinne auch mit Hilfe
einer Tabelle von Wahrheitswerten prüfen:
\begin{center}
\begin{tabular}{|>{$}c<{$}>{$}c<{$}|>{$}c<{$}|>{$}c<{$}|>{$}c<{$}>{$}c<{$}>{$}c<{$}|}
\hline
                   x&                   y&x\vee y
	&z&x\vee z&y\vee\neg z&(x\vee z)\wedge(y\vee\neg z)\\
\hline
\multirow{2}{*}{$t$}&\multirow{2}{*}{$t$}&\multirow{2}{*}{$\color{red}t$}
	&t&   t   &     t     &   \color{red}t           \\
                    &                    &                   
	&f&   t   &     t     &   \color{red}t           \\
\hline
\multirow{2}{*}{$t$}&\multirow{2}{*}{$f$}&\multirow{2}{*}{$\color{red}t$}
	&t&   t   &     f     &              f           \\
                   & &            &f&   t   &     t     &   \color{red}t           \\
\hline
\multirow{2}{*}{$f$}&\multirow{2}{*}{$t$}&\multirow{2}{*}{$\color{red}t$}
	&t&   t   &     t     &   \color{red}t           \\
                    &                    &            
	&f&   f   &     t     &              f           \\
\hline
\multirow{2}{*}{$f$}&\multirow{2}{*}{$f$}&\multirow{2}{*}{$f$}
	&t&   t   &     f     &              f           \\
                    &                    &            
	&f&   f   &     t     &              f           \\
\hline
\end{tabular}
\end{center}
Für jedes rote $\color{red}t$ muss es möglich sein, mit Hilfe eines
geeigneten Wahrheitswertes für $z$ in der Spalte ganz rechts ebenfalls
ein rotes $\color{red}t$ zu erzeugen.
Und wenn $x\vee y$ nicht wahr ist, wie in der letzten Zeile der Tabelle,
dann darf auch kein Wahrheitswert für $z$ den Ausdruck
$(x\vee z)\wedge(y\vee \neg z)$ wahr machen.
\end{loesung}

