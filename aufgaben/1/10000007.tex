Zeigen Sie, dass die logischen Formeln
\[
x\vee y\qquad\text{und}\qquad (x\vee z)\wedge (y\vee \neg z)
\]
"aquivalent sind, d.\,h.~die Variablen der einen Formel k"onnen
genau dann so mit Wahrheitswerten
belegt werden, dass die Formel wahr wird, wenn dies f"ur die andere Formel
m"oglich ist.

\begin{loesung}
Wenn $x\vee y$ wahr ist, dann ist eine der Variablen, zum Beispiel
$x$ wahr.
Dann kann mann $z$ auf ``falsch'' setzen, was die zweite Formel wahr macht.
Ist stattdessen $y$ wahr, kann man $z$ auf wahr setzen, auch so wird die
zweite Formel wahr.
Ist die erste Formel dagegen falsch, dann ist weder $x$ noch $y$ wahr,
die zweite Formel ist also im wesentlichen $z\wedge \neg z$, was nie
wahr sein kann.

Ist umgekehrt die zweite Formel wahr, und auch $z$, dann ist $\neg z$ nicht 
wahr, also muss $y$ wahr sein. Dann ist auch $x\vee y$ wahr. Dasselbe
Argument funktioniert auch, wenn $z$ falsch ist, dann folgt, dass $x$
wahr sein muss, und wieder ist $x\vee y$ wahr.
Ist die zweite Formel immer falsch, dann gibt es keine Belegung von
$z$ so, dass beide Terme wahr sind. D.\,h.~aber, dass $x$ und $y$
beide falsch sein m"ussen, also ist auch $x\vee y$ falsch.
\end{loesung}
