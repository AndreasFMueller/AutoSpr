Ein Internet-Meme von
\url{https://mathematicsart.com/solved-exercises/derivative-of-x-squared-and-derivative-of-sumx-x-times/}
schlägt einen Beweis für $1=0$ vor.
Was ist falsch am folgenden Beweis.
Zunächst ist
\begin{align*}
x^2
&=
\underbrace{x+x+\dots+x}_{\displaystyle\text{$x$ times}}
\intertext{Durch Ableiten erhält man}
\frac{d}{dx}x^2
&=
\frac{d}{dx}\bigl(
\underbrace{x+x+\dots+x}_{\displaystyle\text{$x$ times}}.
\bigr)
\intertext{Andererseits kennt man die Ableitungen von $x^2$ und $x$ aus
der Analysis.
Durch Einsetzen dieser bekannten Ableitungen ergibt sich}
2x
&=
\underbrace{1+1+\dots+1}_{\displaystyle\text{$x$ times}}
=
x
\intertext{Wegen $x\ne 0$ darf man durch $0$ dividieren und 
erhält}
2&=1
\intertext{oder  nach Subtraktion von $1$}
1&=0
\end{align*}

\begin{loesung}
Die Interpretation von $x^2$ als eine Summe von $x$ Faktoren $x$ 
funktioniert nur für ganzzahlige $x$.
Um eine Ableitung zu berechnen, muss man die unabhängige Variable 
aber in $\mathbb{R}$ varieren können.

Doch selbst wenn das möglich wäre, besteht der nächste Fehler darin,
dass man auf der rechten Seite nach jedem Vorkommen von $x$ ableiten
müsste, d.~h.~auch nach dem $x$ in ``$x$ times`` ableiten können.
Formal könnte man das so beschreiben: die rechte Seite ist eine
Funktion $f(x,y)$ zweier Variablen $x$ und $y$, definiert durch
\[
f(x,y) = \underbrace{x+x+\dots+x}_{\displaystyle \text{$y$ times}},
\]
wobei man aber $y=x$ setzen muss.
Nach den Ableitungsregeln für Funktionen mehrerer Variablen müsste
man die Ableitung gemäss
\[
\frac{d}{dx}f(x,x)
=
\frac{\partial f}{\partial x}(x,x)
+
\frac{\partial f}{\partial y}(x,x)\frac{dy}{dx}\bigg|_{y=x}
\]
berechnen.
Der zweite Term fehlt in der ``Ableitungsberechnung'' des Memes.
\end{loesung}
