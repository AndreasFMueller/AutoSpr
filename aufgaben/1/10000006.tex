Betrachten Sie die Behauptung:
\begin{quote}
In einem Graphen mit mindestens zwei Knoten gibt es zwei verschiedene Knoten
mit dem gleichen Grade (der gleichen Anzahl Kanten, die von diesem
Knoten ausgehen).
\end{quote}
Ist die Behauptung wahr oder falsch? Warum (Beweis
oder Gegenbeispiel)?

\begin{hinweis}
Gehen Sie von einem zusammenhängenden Graphen
aus, und überlegen sie sich, welche Grade möglich sind.
\end{hinweis}

\thema{Graph}
\thema{Grad eines Vertex}

\begin{loesung}
Die Behauptung ist wahr.

In einem zusammenhängenden Graphen mit $n$
Ecken kann der Grad einer Ecke nicht grösser als $n-1$ sein.
Möglich sind also nur die $n-1$ verschiedenen Grade
$1,2,\dots,n-1$. Daher müssen zwei Ecken den gleichen Grad
haben.

Falls der Graph nicht zusammenhängend ist, nimmt man eine
Zusammenhangskomponente mit mindestens zwei Ecken, nach dem
eben bewiesenen muss diese zwei Ecken mit dem gleichen
Grad haben.

Falls es keine Zusammenhangskomponente mit mit mindestens
zwei Ecken gibt, besteht der Graph nur aus Vertices und hat
kein Kanten, also haben alle Vertices den Grad $0$, auch
in diesem Fall ist die Aussage also richtig.
\end{loesung}
