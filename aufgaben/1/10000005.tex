Der Grad eines Vertex ist die Zahl der Kanten, die im Vertex enden.
Zeigen Sie, dass die Summe der Grade aller Vertices eines Graphen
eine gerade Zahl ist.

\thema{Grad eines Vertex}
\thema{Graph}

\begin{loesung}
Entfernt man aus dem Graphen Kante für Kante, dann wird die
Summe der Grade in jedem Schritt um zwei kleiner. Sie ändert
dabei also ihren Rest bei Teilung durch zwei nicht. Wenn alle
Kanten entfernt sind, sind auch alle Grade 0, die Summe der
Grade ist also gerade. Daher muss die ursprüngliche Summe der
Grade auch gerade gewesen sein.

Wir können diesen konstruktiven Beweis auch induktiv formulieren.
Dazu zeigen wir, dass die Summe der Grade aller Vertizes eines Graphen
mit $n$ Kanten $2n$ ist. Für einen Graphen ohne Kanten ist dies
sicher richtig, dies ist die Induktionsverankerung. Nehmen wir
jetzt an, dass die Behauptung für einen Graphen mit $n$ Kanten
richtig ist. Sei jetzt $G$ ein Graph mit $n+1$ Kanten. Wir können
eine Kante entfernen und erhalten einen Graphen $G_1$ mit $n$
Kanten, für den natürlich die Induktionsannahme gilt. Die Summe
der Grade ist also $2n$. Fügen wir jetzt die Kante wieder hinzu,
wird der Grad von Anfangs- und Endpunkt der Kante um genau $1$
zunehmen, die Summe der Grade von $G$ ist also $2n+2=2(n+1)$,
was die Behauptung für $n+1$ beweist.
\end{loesung}

