Zählen Sie mindestens 5 Wörter in jeder der folgenden Sprachen auf.
\begin{teilaufgaben}
\item $\Sigma= \{\texttt{0},\texttt{1},\dots,\texttt{9}\}$ und
\[
L=\left\{ w\in\Sigma^*\;\left|\;
\begin{minipage}{8.4cm}
$w$ ist Dezimaldarstellung des Beitrittsjahres eines Kantons zur
Eidgenossenschaft
\end{minipage}
\right.\right\}.
\]
\item Sei das Alphabet der Tasten 
$\Sigma = \{
\text{\keys{{$+$}}},
\text{\keys{{$-$}}},
\text{\keys{{$\times$}}},
\text{\keys{{$\div$}}},
\text{\keys{{$=$}}},
\text{\keys{0}},
\dots,
\text{\keys{9}},
\text{\keys{C}},
\text{\keys{AC}}
\}
$
eines Taschenrechners gebeben und
\[
L=\left\{
w\in\Sigma^*\;
\left|
\;
\begin{minipage}{10cm}\raggedright
$w$ ist eine Tastenfolge, die ohne Fehler zu einer erfolgreichen Berechnung
auf einem Taschenrechner führt.
\end{minipage}
\right.
\right\}.
\]
\item
Sei $\Sigma = \{
\texttt{I},
\texttt{V},
\texttt{X},
\texttt{L},
\texttt{C},
\texttt{M},
\texttt{D}\}$ und
\[
L=\{
w\in\Sigma^*\;
|
\;
\text{$w$ ist eine römische Zahl}
\}
\]
\end{teilaufgaben}

\thema{Sprache}

\begin{loesung}
\begin{teilaufgaben}
\item Gemäss der Tabelle auf
\url{https://de.wikipedia.org/wiki/Kanton_(Schweiz)}
ist 
\[
L=\{
\texttt{1291},
\texttt{1332},
\texttt{1351},
\texttt{1352},
\texttt{1353},
\texttt{1481},
\texttt{1501},
\texttt{1513},
\texttt{1803},
\texttt{1815},
\texttt{1979}
\}.
\]
\item
Solche Tastenfolgen bestehen aus Folgen von Ziffern, getrennt durch
Operationszeichen
und abgeschlossen durch ein Gleichheitszeichen:
\begin{align*}
L=\mathstrut&\{
\\
&\qquad
\text{\keys{1} \keys{{$+$}} \keys{1} \keys{{$=$}}}\,,
\\
&\qquad
\text{\keys{AC} \keys{9} \keys{{$\div$}} \keys{3} \keys{{$=$}}}\,,
\\
&\qquad
\text{\keys{AC} \keys{1} \keys{2} \keys{9} \keys{1}
\keys{{$\times$}} \keys{7} \keys{{$=$}}}\,,
\\
&\qquad
\text{\keys{0} \keys{0} \keys{0} \keys{1}
\keys{{$-$}}
\keys{1} \keys{1} \keys{1} \keys{1}
\keys{{$=$}}}\,,
\\
&\qquad
\text{\keys{2}
\keys{{$\times$}}
\keys{{$=$}}
\keys{{$=$}}
\keys{{$=$}}
\keys{{$=$}}}\,,
\\
&\qquad\dots
\\
&\}
\end{align*}
Die letzte Tastenfolge berechnet die Zweierpotenzen durch wiederholte 
Multiplikation mit $2$.
\item
Die römischen Zahlen kennen keine Null, sie sind also der Reihe nach
\[
L=\{
\texttt{I},
\texttt{II},
\texttt{III},
\texttt{IV},
\texttt{V},
\texttt{VI},
\texttt{VII},
\texttt{IIX},
\texttt{IX},
\texttt{X},
\texttt{XI},
\texttt{XII},
\dots
\}
\qedhere
\]
\end{teilaufgaben}
\end{loesung}
