Wieviele Wörter enthalten die folgenden Sprachen über dem Alphabet 
$\Sigma=\{{\tt 0}, {\tt 1}\}$?
\begin{teilaufgaben}
\item $L=\{ w\in\Sigma^*\,|\, |w|\le 5\}$
\item $L=\{ w\in\Sigma^*\,|\, |w|\le n\}$
\item $L=\{ w\in\Sigma^*\,|\,
|w|_{\tt 0}\le 2
\wedge
|w|_{\tt 1}\le 3
\}$
\item $L=\{ w\in\Sigma^*\,|\, |w|_{\tt 0}\le 1291\}$
\end{teilaufgaben}

\begin{loesung}
\begin{teilaufgaben}
\item
Es gibt jeweils $2^l$ Wörter der Länge $l$, also
\[
|L|=2^0 + 2^1+2^2+2^3+2^4+2^5=2^6-1=63.
\]
\item
Wie in a) ist die Kardinalität von $L$:
\[
|L|=\sum_{k=0}^n2^k=2^{n+1}-1.
\]
\item
Wir zählen die Anzahl möglicher Wörter für jede Länge, länger als
$5$ Zeichen kann ein Wort in $L$ ja offenbar nicht werden. Wir bekommen
folgende Tabelle
\begin{center}
\begin{tabular}{|c|l|c|}
\hline
Länge&Wörter&Anzahl\\
\hline
0&$\varepsilon$&1\\
1&{\tt 0}, {\tt 1}&2\\
2&{\tt 00}, {\tt 01}, {\tt 10}, {\tt 11}&4\\
3&{\tt 001}, {\tt 010}, {\tt 011}, {\tt 100}, {\tt 101}, {\tt 110}, {\tt 111}&7\\
4&{\tt 0111}, {\tt 1011}, {\tt 1101}, {\tt 1110}&4\\
 &{\tt 0011}, {\tt 0101}, {\tt 0110}, {\tt 1001}, {\tt 1010}, {\tt 1100}&6\\
5&Wähle 2 aus 5 Plätzen für die {\tt 0}: $\binom{5}{2}$&10\\
\hline
\end{tabular}
\end{center}
Insgesamt hat die Sprache $L$ also $1+2+4+7+10+10=34$ Wörter.
\item Die Wörter aus $L$ können beliebig viele {\tt 1} enthalten,
die Sprache hat also unendlich viele Wörter.
\qedhere
\end{teilaufgaben}
\end{loesung}
