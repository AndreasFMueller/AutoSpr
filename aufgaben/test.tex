%
% test.tex -- Skript zur Vorlesung Automaten und Sprachen
%                         an der Hochschule Rapperswil
%
% (c) 2009-2017 Prof. Dr. Andreas Mueller, HSR
%
\documentclass[a4paper,12pt]{book}
\usepackage{german}
\usepackage[utf8]{inputenc}
\usepackage[T1]{fontenc}
\usepackage{CJKutf8}
\usepackage{times}
\usepackage{geometry}
\geometry{papersize={210mm,297mm},total={160mm,240mm},top=31mm,bindingoffset=15mm,marginparwidth=9mm}
\usepackage{alltt}
\usepackage{verbatim}
\usepackage{fancyhdr}
\usepackage{amsmath}
\usepackage{amssymb}
\usepackage{amsfonts}
\usepackage{amsthm}
\usepackage{textcomp}
\usepackage{graphicx}
\usepackage{array}
%\usepackage{picins}
\usepackage{ifthen}
\usepackage{multirow}
\usepackage{txfonts}
%\usepackage[basic]{circ}
\usepackage[all]{xy}
\usepackage{algorithm}
\usepackage{algorithmic}
\usepackage{makeidx}
\usepackage{paralist}
\usepackage{color}
\usepackage[colorlinks=true]{hyperref}
\usepackage{environ}
\usepackage{menukeys}
\usepackage{pifont}
\usepackage{etoolbox}
\usepackage{bm}
\usetikzlibrary{arrows,scopes}
\makeindex
\begin{document}
\pagestyle{fancy}
%
% uebung.tex -- gemeinsame Makros fuer Uebungsblaetter
%
% (c) 2006-2017 Prof. Dr. Andreas Mueller, HSR
%
\definecolor{darkgreen}{rgb}{0,0.6,0}
\newcounter{uebungsaufgabe}
\newboolean{loesungen}
% environment fuer uebungsaufgaben
\newenvironment{uebungsaufgaben}{
\begin{list}{\arabic{uebungsaufgabe}.}
  {\usecounter{uebungsaufgabe}
  \setlength{\labelwidth}{2cm}
  \setlength{\leftmargin}{0pt}
  \setlength{\labelsep}{5mm}
  \setlength{\rightmargin}{0pt}
  \setlength{\itemindent}{0pt}
}}{\end{list}\vfill\pagebreak}
% Teilaufgaben
\newenvironment{teilaufgaben}{
\begin{enumerate}
\renewcommand{\labelenumi}{\alph{enumi})}
}{\end{enumerate}}
% Loesung
\NewEnviron{loesung}{%
\begin{proof}[Lösung]%
\renewcommand{\qedsymbol}{$\bigcirc$}
\BODY
\end{proof}}
\NewEnviron{diskussion}{
\BODY
\bigskip
}
\def\keineloesungen{%
\RenewEnviron{loesung}{\relax}
\RenewEnviron{diskussion}{\relax}
}
% Hinweis
\newenvironment{hinweis}{%
\renewcommand{\qedsymbol}{}
\begin{proof}[Hinweis]}{\end{proof}}
% Aufgabe aus der Sammlung wiedergeben
\newboolean{themastatus}
\setboolean{themastatus}{false}
\newcounter{problemcounter}[chapter]
\def\aufgabepath{./}
\def\ainput#1{\input\aufgabepath/#1}
\def\verbatimainput#1{\expandafter\verbatiminput{\aufgabepath/#1}}
\def\aufgabetoplevel#1{%
\expandafter\def\expandafter\inputpath{#1}%
\let\aufgabepath=\inputpath
}
\def\includeagraphics[#1]#2{\expandafter\includegraphics[#1]{\aufgabepath#2}}
% \aufgabe
\newcommand{\aufgabe}[1]{%
\StrRemoveBraces{#1}[\FirstChar]%
\StrChar{\FirstChar}{1}[\FirstChar]%
  \expandafter\def\csname themalist\endcsname{}
  \setboolean{themastatus}{false}
  \refstepcounter{problemcounter}%
  \label{#1}
  \bigskip{\parindent0pt\strut}\hbox{\bf\theproblemcounter. }%
  \marginpar{\raggedright\tiny #1}%
  \expandafter\def\csname currentaufgabe\endcsname{#1}%
  \expandafter\def\csname aufgabepath\endcsname{\inputpath/\FirstChar/#1/}%
  \expandafter\input{\inputpath\FirstChar/#1.tex}
  %\medskip
  \ifthenelse{\boolean{themastatus}}{
    \parindent 0pt
    {\sc Thema:} {\small \themalist.}}{%
  }
  \bigskip

}
\renewcommand\theproblemcounter{\thechapter.\arabic{problemcounter}}
% Bewertung
\NewEnviron{bewertung}{\footnotesize
\renewcommand{\qedsymbol}{}
\begin{proof}[Bewertung]
\BODY
\end{proof}}
% oft benutzte Macros
\def\blank{\text{\textvisiblespace}}
%
% macros fuer den thema-Index
%
\newcommand{\openthemaindex}{%
  \newwrite\themaindex
  \immediate\openout\themaindex=thema.tix
}
%
\newcommand{\closethemaindex}{\immediate\closeout\themaindex}
%
\def\themalink#1#2{\hyperref[thema:#1]{#2}}
\def\themaL#1#2{%
  \ifthenelse{\boolean{themastatus}}{%
    \xappto{\themalist}{, \noexpand\themalink{#1}{#2}}
  }{%
    \xdef\themalist{\noexpand\themalink{#1}{#2}}
    \setboolean{themastatus}{true}
  }
  \immediate\write\themaindex%
  {{\thechapter}{#1}{#2}{\arabic{problemcounter}}{\thechapter.\arabic{problemcounter}}{\currentaufgabe}}%
}
\def\thema#1{\themaL{#1}{#1}}
%\def\thema#1{%
%  \ifthenelse{\boolean{themastatus}}{%
%    \xappto{\themalist}{, \noexpand\themalink{#1}}
%  }{%
%    \xdef\themalist{\noexpand\themalink{#1}}
%    \setboolean{themastatus}{true}
%  }
%  \immediate\write\themaindex%
%  {{\thechapter}{#1}{\arabic{problemcounter}}{\thechapter.\arabic{problemcounter}}{\currentaufgabe}}%
%}
% Thema-Information anzeigen
\def\themasection#1#2{
\item[#2] \label{thema:#1}
}
\newcommand{\printthemata}{
  \IfFileExists{./thema.tex}{
    \chapter*{Aufgaben nach Themen}
    \addcontentsline{toc}{chapter}{Themenindex}
    \begin{description}
    \input{thema.tex}
    \end{description}
  }{}
}
\newenvironment{beispiel}[1][Beispiel]{%
\begin{proof}[#1]%
\renewcommand{\qedsymbol}{$\bigcirc$}
}{\end{proof}}


\setboolean{loesungen}{true}
\aufgabetoplevel{./}
\noindent
%
% testaufgabe.tex
%
% (c) 2023 Prof Dr Andreas Müller, OST Ostschweizer Fachhochschule
%
Ein Formattierprogramm für Wordle-Lösungen nimmt eine Spezifikation
der Lösung als Zeichenketten entgegen, in der jeweils ein Buchstabe
gefolgt wird von einem Zeichen {\blank} falls, das Zeichen nicht 
vorkommt, \texttt{+} wenn es vorkommt und \texttt{*} wenn es am richtigen
Platz steht.
Am Ende jeder Zeile wird ein \texttt{|} platziert.
Die Zeilen werden unmittelbar hintereinander geschrieben, natürlich
müssen Sie alle die gleiche Länge haben, die aber auch verschieden
von den üblichen 5 Zeichen sein kann.
Bevor mit der Formattierung begonnen wird, soll mit einem regulären
Ausdruck geprüft werden, ob das Format stimmt.
Ist dies möglich?

\themaL{Pumping Lemma fur regulare Sprachen}{Pumping Lemma für reguläre Sprachen}
\themaL{regular}{regulär}

\begin{loesung}
Nein, denn der zugehörige endliche Automat müsste immer wieder ``nachzählen''
können, ob die Zeilen die richtige Länge haben.
Beweisen kann man es mit dem Pumping Lemma für reguläre Sprachen.
\begin{enumerate}
\item Die Sprache wird mit $L$ bezeichnet und wir nehmen an, dass sie
regulär ist.
\item
Das Pumping Lemma garantiert, dass es die Pumping Length $N$ gibt.
\item
Wir konstruieren das Wort
\begin{equation*}
w
=
\underbrace{\texttt{A\blank A\blank\dots A\blank}}_N\texttt{|}%
\underbrace{\texttt{A\blank A\blank\dots A\blank}}_N\texttt{|}
\end{equation*}
welches sicher in der Sprache $L$ liegt,
den es repräsentiert zwei Zeilen der Länge $N$ aus Zeichen \texttt{A},
die im gesuchten Wort nicht vorkommen.
\item
Nach dem Pumping-Lemma muss es eine Unterteilung geben, deren
aufpumpbarer Teil $y$ vollständig im ersten Block aus
``$\texttt{A\blank}$'' liegen muss.
\item
Beim Pumpen des Teils $y$ wird die Anzahl der \texttt{A} in diesem
ersten Block erhöht, die Anzahl im zweiten Block bleibt gleich, somit
liegt keine korrekte Spezifikation mehr vor, das gepumpte Wort ist
nicht mehr in $L$, im Widerspruch zur Aussage des Pumping Lemma.
\item
Dieser Widerspruch zeigt, dass $L$ nicht regulär sein kann.
\qedhere
\end{enumerate}
\end{loesung}

\begin{bewertung}
6 Schritte des Pumping-Lemma-Beweises:
Annahme ({\bf A}) 1 Punkt,
Pumping Length ({\bf N}) 1 Punkt,
Wort ({\bf W}) 1 Punkt,
Unterteilung ({\bf U}) 1 Punkt,
Widerspruch beim Pumpen ({\bf P}) 1 Punkt,
Folgerung ({\bf F}) 1 Punkt.
\end{bewertung}


\end{document}
