%
% test.tex -- Skript zur Vorlesung Automaten und Sprachen
%                         an der Hochschule Rapperswil
%
% (c) 2009-2017 Prof. Dr. Andreas Mueller, HSR
%
\documentclass[a4paper,12pt]{book}
\usepackage{german}
\usepackage[utf8]{inputenc}
\usepackage[T1]{fontenc}
\usepackage{CJKutf8}
\usepackage{times}
\usepackage{geometry}
\geometry{papersize={210mm,297mm},total={160mm,240mm},top=31mm,bindingoffset=15mm,marginparwidth=9mm}
\usepackage{alltt}
\usepackage{verbatim}
\usepackage{fancyhdr}
\usepackage{amsmath}
\usepackage{amssymb}
\usepackage{amsfonts}
\usepackage{amsthm}
\usepackage{textcomp}
\usepackage{graphicx}
\usepackage{array}
%\usepackage{picins}
\usepackage{ifthen}
\usepackage{multirow}
\usepackage{txfonts}
%\usepackage[basic]{circ}
\usepackage[all]{xy}
\usepackage{algorithm}
\usepackage{algorithmic}
\usepackage{makeidx}
\usepackage{paralist}
\usepackage{color}
\usepackage[colorlinks=true]{hyperref}
\usepackage{environ}
\usepackage{menukeys}
\usepackage{pifont}
\usepackage{etoolbox}
\usepackage{bm}
\usetikzlibrary{arrows,scopes}
\makeindex
\begin{document}
\pagestyle{fancy}
%
% uebung.tex -- gemeinsame Makros fuer Uebungsblaetter
%
% (c) 2006 Prof. Dr. Andreas Mueller, HSR
% $Id: uebung.tex,v 1.3 2008/01/06 14:05:56 afm Exp $
%
\newcounter{uebungsaufgabe}
\newboolean{loesungen}
% environment fuer uebungsaufgaben
\newenvironment{uebungsaufgaben}{
\begin{list}{\arabic{uebungsaufgabe}.}
  {\usecounter{uebungsaufgabe}
  \setlength{\labelwidth}{2cm}
  \setlength{\leftmargin}{0pt}
  \setlength{\labelsep}{5mm}
  \setlength{\rightmargin}{0pt}
  \setlength{\itemindent}{0pt}
}}{\end{list}\vfill\pagebreak}
% Teilaufgaben
\newenvironment{teilaufgaben}{
\begin{enumerate}
\renewcommand{\labelenumi}{\alph{enumi})}
}{\end{enumerate}}
% Loesung
\NewEnviron{loesung}{%
\begin{proof}[L"osung]%
\renewcommand{\qedsymbol}{$\bigcirc$}
\BODY
\end{proof}}
\NewEnviron{diskussion}{
\BODY
}
\def\keineloesungen{%
\RenewEnviron{loesung}{\relax}
\RenewEnviron{diskussion}{\relax}
}
% Hinweis
\newenvironment{hinweis}{%
\renewcommand{\qedsymbol}{}
\begin{proof}[Hinweis]}{\end{proof}}
% Aufgabe aus der Sammlung wiedergeben
\newcounter{problemcounter}[chapter]
\def\aufgabepath{./}
\def\ainput#1{\input\aufgabepath/#1}
\def\verbatimainput#1{\expandafter\verbatiminput{\aufgabepath/#1}}
\def\aufgabetoplevel#1{%
\expandafter\def\expandafter\inputpath{#1}%
\let\aufgabepath=\inputpath
}
\def\includeagraphics[#1]#2{\expandafter\includegraphics[#1]{\aufgabepath#2}}
% \aufgabe
\newcommand{\aufgabe}[2]{%
\refstepcounter{problemcounter}%
\label{#2}%
\bigskip{\parindent0pt\strut}\hbox{\bf\theproblemcounter. }%
\marginpar{\raggedright\tiny #2}%
\expandafter\def\csname aufgabepath\endcsname{\inputpath/#1/#2/}%
\expandafter\input{\inputpath#1/#2.tex}
\bigskip
}
\renewcommand\theproblemcounter{\thechapter.\arabic{problemcounter}}
% oft benutzte Macros
\def\blank{\text{\textvisiblespace}}

\setboolean{loesungen}{true}
\aufgabetoplevel{./}
\noindent
%
% testaufgabe.tex
%
% (c) 2023 Prof Dr Andreas Müller, OST Ostschweizer Fachhochschule
%
%\aufgabe{70000041}
%\aufgabe{60000035}
\aufgabe{50000030}

\end{document}
