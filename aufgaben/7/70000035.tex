Student Xaver Tecco soll im Rahmen einer Big-Data-Studienarbeit die 
Kunden einer grossen Shop-Website untersuchen und klassifizieren.
Es steht eine grosse Zahl von binären Eigenschaften zur Verfügung,
zum Beispiel ob Kunden ein bestimmtes Produkt gekauft haben, oder
ob ein Kunde nur im Dezember einkauft.
Herr Tecco soll herausfinden, ob es eine Teilmenge von Kriterien derart
gibt, dass jeder Kunde genau eine der Eigenschaften hat.
Die Abgabe der Arbeit steht in zwei Tagen bevor, und er hat noch keinen
funktionierenden Algorithmus.
Muss er sich Sorgen machen?

\themaL{NP-vollstandig}{NP-vollständig}
\thema{polynomielle Reduktion}

\begin{loesung}
Dieses Problem ist äquivalent mit dem bekanntermassen NP-vollständigen
Problem \textsl{EXACT-COVER}:
\begin{align*}
\text{Eigenschaft}&\leftrightarrow \text{Menge $S_j$}\\
\text{Teilmenge von Eigenschaften}&\leftrightarrow \text{Unterfamilie $S_{j_i}$}\\
\text{genau eine der Eigenschaften}&\leftrightarrow S_{j_i}\cap S_{j_k}=\emptyset\quad\forall i\ne k\\
\text{alle Kunden erfasst}&\leftrightarrow \bigcup_{j=1}^nS_j=\bigcup_{i=1}^m S_{j_i}
\end{align*}
Da nach heutigem Wissen NP-vollständige Problem nicht mit einem polynomiellen
Algorithmus gelöst werden können, ist nicht mit einer schnellen Lösung
des Problems zu rechnen.

Es sind auch andere Vergleiche möglich.
Zu jeder Eigenschaft $i$ gibt es eine Menge $M_i$ von Kunden, die diese
Eigenschaft haben.
Diese Menge kann man im $n$-adischen System codieren.
Wenn Kunden $k$ die Eigenschaft $i$ hat, dann wird die $k$-te Stelle
auf $1$ gesetzt, sonst auf $0$.
Zur Eigenschaft $i$ gibt es also die Zahl $n_i$, die Menge $M_i$
codiert.
Sei $S$ die Menge all dieser Zahlen.
Gesucht wird jetzt eine Teilmenge, so dass deren Summe die Zahl
bestehend aus lauter $1$ ist, d.~h.~jeder Kunde wird abgedeckt, aber
nur genau einmal.
Dies ist das Problem \textsl{SUBSET-SUM}.
Damit hat man aber nur gezeigt, dass das gestellte Problem leichter ist
als \textsl{SUBSET-SUM}, man müsste die umgekehrte Richtung haben.
Dies reicht also nicht als Nachweis, dass das gestellte Problem
NP-vollständig ist.

Das Problem klingt auch sehr ähnlich wie \textsl{HITTING-SET}.
Man könnte mit $S$ die Menge aller Eigenschaften bezeichnen,
und mit $S_i$ die Menge der Eigenschaften, die Kunden $i$ hat.
Das Problem \textsl{HITTING-SET} versucht dann eine Menge $H\subset S$
von Eigenschaften zu finden, so dass jeder Kunde genau eine der
Eigenschaften in $H$ hat.
Dies ist natürlich die Technik, mit der \textsl{HITTING-SET}
als NP-vollständig bewiesen wird.
\end{loesung}

\begin{diskussion}
Der Name \textsl{Xaver Tecco} ist eine Anagramm von \textsl{EXACT-COVER}.
\end{diskussion}

\begin{bewertung}
Reduktionsprinzip ({\bf R}) 1 Punkt,
Vergleichsproblem \textsl{EXACT-COVER} ({\bf V}) 1 Punkt,
Mapping für Eigenschaften ({\bf E}) 1 Punkt,
für genau eine Eigenschaft ({\bf G}) 1 Punkt,
für alle Kunden erfasst ({\bf A}) 1 Punkt,
Schlussfolgerung NP-vollständig ({\bf N}) 1 Punkt.
\end{bewertung}


