Die Bewohner eines Landes lassen sich nach einer grossen Zahl
von Kriterien in Gruppen einteilen: nach Berufsgruppen, nach
Altersgruppen, nach kulturellen Interessen, nach Hobbies usw.
Jetzt soll eine Interessenvertretung gebildet werden so, dass
jede dieser Gruppen mit genau einem Mitglied vertreten ist.
Dabei ist es durchaus zulässig, dass die Gruppe der Raketenmodellbauer
und die Gruppe der Freunde von Kammermusik von der gleichen Person
vertreten werden. Aber es darf nicht sein, dass sich der Vertreter
der Astrophotographen auch noch für Kammermusik interessiert, denn
dann hätten die Kammermusikfreunde zwei Vertreter.
Gibt es einen effizienten Algorithmus, der auch bei einer grossen
Zahl von Bewohnern und Gruppierungen entscheiden kann, ob eine
solche Zusammenstellung einer Interessenvertretung überhaupt
möglich ist?

\begin{loesung}
Nein, denn das Problem ist NP-vollständig, wie wir gleich
zeigen werden, nach unserem aktuellen
Wissen also höchstwahrscheinlich nicht mit einem effizienten
Algorithmus lösbar. Und gäbe es einen effizienten Algorithmus,
würde dies automatisch $\text{P}=\text{NP}$ zur Folge haben.

Seien die $i\in I$ die Kriterien, nach denen gruppiert wird,
und sei $S_i$ die Menge der Bewohner, die zur Gruppe $i$ gehört.
Gesucht wird jetzt ein Teilmenge $H\subset\bigcup_{i\in I}S_i$
von Interessenvertretern so, dass $H$ mit jeder der Mengen
$S_i$ genau einen Vertreter gemeinsam hat: $|H\cap S_i|=1$
für alle $i$. Dies ist das Problem {\textsl{HITTING-SET}}. 

Umgekehrt lässt sich jedes {\textsl{HITTING-SET}} Problem umformulieren
als Problem, eine Interessenvertretung der genannten Art zu finden.
Folglich ist das vorliegende Problem äquivalent zum {\textsl{HITTING-SET}},
welches bekanntermassen NP-vollständig ist. Also ist auch das
Interessenvertretungsproblem NP-vollständig.
\end{loesung}
