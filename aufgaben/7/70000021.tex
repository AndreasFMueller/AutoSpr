Die Bewohner eines Landes lassen sich nach einer grossen Zahl
von Kriterien in Gruppen einteilen: nach Berufsgruppen, nach
Altersgruppen, nach kulturellen Interessen, nach Hobbies usw.
Jetzt soll eine Interessenvertretung gebildet werden so, dass
jede dieser Gruppen mit genau einem Mitglied vertreten ist.
Dabei ist es durchaus zul"assig, dass die Gruppe der Raketenmodellbauer
und die Gruppe der Freunde von Kammermusik von der gleichen Person
vertreten werden. Aber es darf nicht sein, dass sich der Vertreter
der Astrophotographen auch noch f"ur Kammermusik interessiert, denn
dann h"atten die Kammermusikfreunde zwei Vertreter.
Gibt es einen effizienten Algorithmus, der auch bei einer grossen
Zahl von Bewohnern und Gruppierungen entscheiden kann, ob eine
solche Zusammenstellung einer Interessenvertretung "uberhaupt
m"oglich ist?

\begin{loesung}
Nein, denn das Problem ist NP-vollst"andig, wie wir gleich
zeigen werden, nach unserem aktuellen
Wissen also h"ochstwahrscheinlich nicht mit einem effizienten
Algorithmus l"osbar. Und g"abe es einen effizienten Algorithmus,
w"urde dies automatisch $\text{P}=\text{NP}$ zur Folge haben.

Seien die $i\in I$ die Kriterien, nach denen gruppiert wird,
und sei $S_i$ die Menge der Bewohner, die zur Gruppe $i$ geh"ort.
Gesucht wird jetzt ein Teilmenge $H\subset\bigcup_{i\in I}S_i$
von Interessenvertretern so, dass $H$ mit jeder der Mengen
$S_i$ genau einen Vertreter gemeinsam hat: $|H\cap S_i|=1$
f"ur alle $i$. Dies ist das Problem {\textsl{HITTING-SET}}. 

Umgekehrt l"asst sich jedes {\textsl{HITTING-SET}} Problem umformulieren
als Problem, eine Interessenvertretung der genannten Art zu finden.
Folglich ist das vorliegende Problem "aquivalent zum {\textsl{HITTING-SET}},
welches bekanntermassen NP-vollst"andig ist. Also ist auch das
Interessenvertretungsproblem NP-vollst"andig.
\end{loesung}
