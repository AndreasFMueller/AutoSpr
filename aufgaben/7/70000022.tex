Das Spiel Hashiwokakero wird auf einem $n\times n$-Gitter gespielt,
auf dem einige Gitterpunkte als Inseln markiert und mit einer Zahl
versehen sind. Die Zahl gibt an, wieviele Br"ucken auf der
betreffenden Insel enden.
Br"ucken verlaufen immer parallel zu den Koordinatenachsen, und
k"onnen keine Inseln "uberspringen.
Zwei Inseln k"onnen mit maximal zwei
Br"ucken verbunden sein. Der Spieler soll Br"ucken so bauen,
dass die Zahl der Br"ucken auf jeder Insel mit der Vorgabe
"ubereinstimmt. Im Bild links die Aufgabe, rechts die L"osung.
\begin{center}
\begin{tabular}{cc}
\includeagraphics[width=0.4\hsize]{HashiwokakeroBeispiel.pdf}&
\includeagraphics[width=0.4\hsize]{HashiwokakeroLoesung.pdf}
\end{tabular}
\end{center}
Zeigen Sie, dass eine nichtdeterministische Maschine in
polynomieller Zeit entscheiden kann, ob ein Hashiwokakero-R"atsel
eine L"osung hat.

\begin{loesung}
Zunächst muss man sicherstellen, dass HASHIWOKAKERO entscheidbar ist.
Dazu muss man alle Brückenkombinationen druchproberen, wovon es sicher
weniger als $O(n^4)$ gibt.
Dies ist immer in endlicher Zeit möglich, also ist das Problem
entscheidbar.

Nun muss man zeigen, dass HASHIWOKAKERO in NP ist. Dazu gen"ugt es, einen
polynomiellen Verifizierer zu finden. Als Zertifikat f"ur den
Verifizierer verlangen wir die Anzahl der Br"ucken zwischen jedem 
Paar von Inseln des R"atsels.
Der Verifizierer muss folgende Dinge "uberpr"ufen:
\begin{center}
\begin{tabular}{l|c}
Pr"ufung&Laufzeit\\
\hline
Br"ucken sind horizontal oder vertikal&$O(n^4)$\\
Br"ucken enden in Inseln&$O(n^4)$\\
Br"ucken kreuzen sich nicht&$O(n^8)$\\
Br"ucken, die in einer Insel enden, haben die ``richtige'' Summe&$O(n^6)$\\
\hline
Maximale Laufzeit& $O(n^8)$
\end{tabular}
\end{center}
Dabei haben wir die ganz grobe Absch"atzung verwendet, dass es h"ochstens
so viele Br"ucken geben kann wie Punktpaare, also $O(n^2)\cdot O(n^2)=O(n^4)$.
Bei der Kontrolle der ``richtigen'' Summe muss der Verifizierer zum
Beispiel f"ur jede Insel und jede Br"ucke die gleiche Arbeit leisten und
"uberpr"ufen, ob die
Br"ucke in der Insel endet, die Br"ucken z"ahlen und mit dem Inselwert
vergleichen.  Die Komplixit"at dieses Teils ist also $O(n^6)$, wie in
obiger Tabelle angegeben.
Die Verifikation ist also in polynomieller Zeit durchf"uhrbar.
\end{loesung}

\begin{bewertung}
Entscheidbar ({\bf E}) 1 Punkt,
Beobachtung, dass es reicht, einen polynomiellen Verifizerer ({\bf V})
zu finden 1 Punkt,
Definition des Zertifikats ({\bf Z}) 1 Punkt,
Definition des Verifikationsalgorithmus ({\bf A}) 2 Punkte,
Absch"atzung der Laufzeitkomplexit"at ({\bf L}) 1 Punkte.
\end{bewertung}

\begin{diskussion}
Daniel Andersson hat nachgewiesen, dass das Br"uckenspiel
NP-vollst"andig ist: {\it Hashiwokakero is NP-complete},
Daniel Anderrsson, Journal of Information Processing Letters {\bf 109} (19),
2009, pp. 1145-1146.
F"ur den Beweis muss eine Reduktion von einem bekanntermassen
NP-vollst"andigen Problem konstruiert werden.
Im vorliegenden Fall wird folgendes Problem verwendet:
In einem Gitter $\mathbb Z$ ist eine Menge von Punkten $P\subset \mathbb Z^2$
ausgew"ahlt worden.
Daraus entsteht ein Graph, indem alle Strecken zwischen Punkten von $P$
hinzugef"ugt werden, die L"ange $1$ haben,
der sogenannte Einheits-Distanz-Graph.
Das Problem zu entscheiden,
ob dieser Graph einen hamiltonschen Zyklus besitzt ist NP-vollst"andig.
Der Beweis von Anderson konstruiert zu jedem Einheits-Distanz-Graphen
ein Hashiwokakero-Spiel, welches genau dann l"osbar ist, wenn der
Graph einen hamiltonschen Zyklus hat.
\end{diskussion}

