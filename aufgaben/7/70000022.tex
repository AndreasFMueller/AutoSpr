Das Spiel Hashiwokakero wird auf einem $n\times n$-Gitter gespielt,
auf dem einige Gitterpunkte als Inseln markiert und mit einer Zahl
versehen sind. Die Zahl gibt an, wieviele Brücken auf der
betreffenden Insel enden.
Brücken verlaufen immer parallel zu den Koordinatenachsen, und
können keine Inseln überspringen.
Zwei Inseln können mit maximal zwei
Brücken verbunden sein. Der Spieler soll Brücken so bauen,
dass die Zahl der Brücken auf jeder Insel mit der Vorgabe
übereinstimmt. Im Bild links die Aufgabe, rechts die Lösung.
\begin{center}
\begin{tabular}{cc}
\includeagraphics[width=0.4\hsize]{HashiwokakeroBeispiel.pdf}&
\includeagraphics[width=0.4\hsize]{HashiwokakeroLoesung.pdf}
\end{tabular}
\end{center}
Zeigen Sie, dass eine nichtdeterministische Maschine in
polynomieller Zeit entscheiden kann, ob ein Hashiwokakero-Rätsel
eine Lösung hat.

\thema{polynomieller Verifizierer}
\thema{NP}

\begin{loesung}
Zunächst muss man sicherstellen, dass HASHIWOKAKERO entscheidbar ist.
Dazu muss man alle Brückenkombinationen druchproberen, wovon es sicher
weniger als $O(n^4)$ gibt.
Dies ist immer in endlicher Zeit möglich, also ist das Problem
entscheidbar.

Nun muss man zeigen, dass HASHIWOKAKERO in NP ist. Dazu genügt es, einen
polynomiellen Verifizierer zu finden. Als Zertifikat für den
Verifizierer verlangen wir die Anzahl der Brücken zwischen jedem 
Paar von Inseln des Rätsels.
Der Verifizierer muss folgende Dinge überprüfen:
\begin{center}
\begin{tabular}{l|c}
Prüfung&Laufzeit\\
\hline
Brücken sind horizontal oder vertikal&$O(n^4)$\\
Brücken enden in Inseln&$O(n^4)$\\
Brücken kreuzen sich nicht&$O(n^8)$\\
Brücken, die in einer Insel enden, haben die ``richtige'' Summe&$O(n^6)$\\
\hline
Maximale Laufzeit& $O(n^8)$
\end{tabular}
\end{center}
Dabei haben wir die ganz grobe Abschätzung verwendet, dass es höchstens
so viele Brücken geben kann wie Punktpaare, also $O(n^2)\cdot O(n^2)=O(n^4)$.
Bei der Kontrolle der ``richtigen'' Summe muss der Verifizierer zum
Beispiel für jede Insel und jede Brücke die gleiche Arbeit leisten und
überprüfen, ob die
Brücke in der Insel endet, die Brücken zählen und mit dem Inselwert
vergleichen.  Die Komplixität dieses Teils ist also $O(n^6)$, wie in
obiger Tabelle angegeben.
Die Verifikation ist also in polynomieller Zeit durchführbar.
\end{loesung}

\begin{bewertung}
Entscheidbar ({\bf E}) 1 Punkt,
Beobachtung, dass es reicht, einen polynomiellen Verifizerer ({\bf V})
zu finden 1 Punkt,
Definition des Zertifikats ({\bf Z}) 1 Punkt,
Definition des Verifikationsalgorithmus ({\bf A}) 2 Punkte,
Abschätzung der Laufzeitkomplexität ({\bf L}) 1 Punkte.
\end{bewertung}

\begin{diskussion}
Daniel Andersson hat nachgewiesen, dass das Brückenspiel
NP-vollständig ist: {\it Hashiwokakero is NP-complete},
Daniel Anderrsson, Journal of Information Processing Letters {\bf 109} (19),
2009, pp. 1145-1146.
Für den Beweis muss eine Reduktion von einem bekanntermassen
NP-vollständigen Problem konstruiert werden.
Im vorliegenden Fall wird folgendes Problem verwendet:
In einem Gitter $\mathbb Z$ ist eine Menge von Punkten $P\subset \mathbb Z^2$
ausgewählt worden.
Daraus entsteht ein Graph, indem alle Strecken zwischen Punkten von $P$
hinzugefügt werden, die Länge $1$ haben,
der sogenannte Einheits-Distanz-Graph.
Das Problem zu entscheiden,
ob dieser Graph einen hamiltonschen Zyklus besitzt ist NP-vollständig.
Der Beweis von Anderson konstruiert zu jedem Einheits-Distanz-Graphen
ein Hashiwokakero-Spiel, welches genau dann lösbar ist, wenn der
Graph einen hamiltonschen Zyklus hat.
\end{diskussion}

