Ein etwas unerfahrener Event-Manager hat sich für seinen nächsten
Event ein Spiel einfallen lassen, für das er die Gäste in
Teams mit jeweils drei Mitgliedern einteilen muss. 
Damit die Bildung der Teams schnell vonstatten gehen kann, schickt
er den Teilnehmern die Team-Nummer bereits mit der Einladung.
Dummerweise lässt die Datenqualität seiner Adressdatenbank
sehr zu wünschen übrig. Die meisten Teilnehmer bekommen mehrere
Einladungen, mit verschiedenen Team-Nummern. Am Event kommt es zum Eklat.
Der Manager versucht die Situation noch zu retten, indem er den
Teilnehmern sagt, sie sollen die Dreierteams einfach mit einer
der Teamnummern bilden, mit welcher spiele keine Rolle. Die Teilnehmer
finden jedoch keine Lösung, der Event platzt, einige Teilnehmer 
gehen frühzeitig nach Hause, andere ersäuffen ihren Frust an der Bar.
Warum ist es so schwierig, nach Anweisung des Managers Teams zu bilden?

\themaL{NP-vollstandig}{NP-vollständig}
\thema{polynomielle Reduktion}

\begin{loesung}
Dieses Problem, welches wir \textsl{TEAMBILDUNG} nennen wollen,
ist NP-vollständig, wie wir durch Vergleich mit dem bekanntermassen
NP-vollständigen Problem \textsl{EXACT-COVER} nachweisen wollen.

Je der vom Manager verschickten Nummern $j$ definiert eine Teilmenge
$S_j$ der Menge aller Teilnehmer $U$. Gesucht ist eine Unterfamilie
Teams $S_{j_i}$, $1\le i\le m$, so dass keine zwei Teams sich überschneiden
(d.~h.~kein Teilnehmer ist in mehr als einem Team) und die ganze Menge $U$
überdecken (d.~h.~jeder Teilnehmer ist in einem Team).

\medskip
Reduktion auf \textsl{HITTING-SET}: Dazu bildet man für jeden
Teilnehmer $i$ die Menge $S_i$ der Teams, für die Teilnehmer $i$
vorgesehen ist. Es ist klar, dass $S_i\subset S$, wobei $S$ die Menge
aller Teams ist. Gesucht ist jetzt eine Auswahl von Teams so, dass
jeder Teilnehmer in genau einem Team ist, also $H\subset S$ so,
dass $|H\cap S_i|=1$. Das ist das Problem \textsl{HITTING-SET}.

\medskip
Es ist in diesem Fall auch möglich, eine Reduktion auf SAT anzugeben.
Dazu muss man beachten, dass eine Auswahl der Teams, die auf den
Einladungen definiert wurden, ja tatsächlich eine Lösung des
Problems darstellen würden, man muss nur noch herausfinden, welche
Teams gebildet werden müssen, und welche nicht. Seien also $i\in I$
die Teamnummern, und sei für jedes $i$ eine boolsche Variable $x_i$
gegeben, welche angibt, ob ein Team gebildet werden soll oder nicht.
Jetzt muss aus den Daten nur noch eine Formel $\varphi$ konstruiert werden,
welche angibt, ob die Belegung der $x_i$ mit Wahrheitswerten, also die Auswahl
der Teams, eine Lösung des Problems darstellt.

Zu jedem Teilnehmer $j\in T$ können wir die Megen $I_j$ der Nummern
der Teams bilden, für die er vorgesehen ist.

Die Formel $\varphi$ besteht aus mehreren Teilen, welche für jeden
Teilnehmer $j$ ausdrücken sollen, dass 
\begin{enumerate}
\item er in höchstens einem der realisierten Teams ist (Formel $\varphi_{1i}$),
und dass
\item er in mindestens einem der realisierten Teams ist (Formel $\varphi_{2i}$).
\end{enumerate}
Die gesucht Formel ist dann
$\bigwedge_{j\in T} \varphi_{1i}\wedge \varphi_{2i}$.

Die Formel $\varphi_{2j}$ drückt aus, dass mindestens ein Team realisiert
wird, für das der Teilnehmer $j$ vorgesehen ist:
\begin{equation}
\varphi_{2j}=x_{i_1}\vee x_{i_2}\vee\dots\vee x_{i_k} = \bigvee_{i\in I_j}x_i.
\label{70000025:1}
\end{equation}

Die Formel
\begin{equation}
x_{i_1}\wedge\overline{(x_{i_2}\vee\dots\vee x_{i_k})}
\label{70000025:2}
\end{equation}
ist genau dann wahr, wenn Team $i_1$ das einzige Team von $j$ ist,
welches realisiert wird. Die Formel ist falsch, wenn ausser $i_1$ auch
noch ein anderes Team realisiert wird. Sie ist aber auch falsch, wenn 
gar keines der Teams realisiert wird.

Entsprechend kann man eine Formel bilden, die dann war wird, wenn $i_2$ als 
einziges Team realisiert wird. Oder allgemein ist
die Formel
\begin{equation}
x_i\wedge \overline{\biggl(\bigvee_{i'\in I_j\setminus\{i\}}x_{i'}\biggr)}
\label{70000025:3}
\end{equation}
wahr, wen Team $i\in I_j$ als einziges der Teams von $j$ realisiert wird.
Damit eine Lösung gefunden wird, muss irgend eines der Teams von $j$ als
einziges realisiert werden,
die gesuchte Formel $\varphi_{1j}$ ist daher
\begin{equation}
\varphi_{1j} = 
\bigvee_{i\in I_j} \biggl(x_i\wedge
	\overline{\biggl(\bigvee_{i'\in I_j\setminus \{i\}}x_{i'}\biggr)}\biggr)
\label{70000025:4}
\end{equation}
Damit kann man jetzt auch die Gesamtformel $\varphi$ zusammensetzen:
\begin{equation}
\varphi=
\bigwedge_{j\in T}
\biggl(
\bigvee_{i\in I_j} \biggl(x_i\wedge
	\overline{\biggl(\bigvee_{i'\in I_j\setminus \{i\}}x_{i'}\biggr)}\biggr)
\wedge
\biggl(
\bigvee_{i\in I_j}x_i
\biggr)
\biggr)
\label{70000025:5}
\end{equation}
Damit ist zwar eine Reduktion von \textsl{TEAMBILDUNG} auf \textsl{SAT}
geschafft, es ist aber noch nicht gezeigt, dass dies auch eine polynomielle
Reduktion ist. Sei $n$ die Anzahl der Teilnehmer und $m$ die Anzahl der
Teams, die gebildet werden sollten. In unserem Beispiel gilt ürlich $n=3m$,
aber obige Reduktion ist auch möglich wenn die Team-Grösse nicht $3$
ist.

Jede der Formeln (\ref{70000025:1}) hat Länge $O(m)$.
Da es in der Schlussformel $O(n)$ Teilformeln dieser Art gibt, 
tragen diese mit Länge $O(nm)$ zur Länge von $\varphi$ bei.

Die Formel (\ref{70000025:3}) hat Länge $O(m)$, davon werden $|I_j|=O(n)$
Stück
in (\ref{70000025:4}) zu einer Formel der Länge $O(mn)$ zusammengesetzt.
In (\ref{70000025:5}) kommt eine solche Formel für jeden Teilnehmer vor,
die Gesamtlänge von $\varphi$ ist $O(n^2m) + O(nm)=O(n^2m)$, also auf
jeden Fall polynomiell in der Grösse des Problems. Damit ist gezeigt,
dass die Reduktion auf \textsl{SAT} auch eine polynomielle Reduktion ist.
\end{loesung}

\begin{diskussion}
Bei oberflächlicher Betrachtung scheint \textsl{TEAMBILDUNG} auch
zum Problem \textsl{3D-MATCHING} äquivalent zu sein. Da alle NP-vollständigen
Problem äquivalent sind, ist das natürlich immer richtig, aber
es ist nicht so einfach, eine "Aquivalenz zwischen 
\textsl{TEAMBILDUNG} und 
\textsl{3D-MATCHING} zu beschreiben.

Im Problem \textsl{3D-MATCHING} geht es um eine endliche Menge $T$,
die den Teilnehmern im Problem \textsl{TEAMBILDUNG} entspricht.
Jede Team-Nummer, die der Manager verschickt hat, erzeugt ein Tripel
von jeweils drei Teilnehmern, also eine Menge $U$ von Tripeln,
$U\subset T\times T\times T$. Die Aufgabe der Teilnehmer besteht
jetzt darin, nur einen Teil der Team-Nummern zu verwenden, um
die Teams zu bilden, also eine Auswahl von Tripeln $W$ aus der Menge $U$
zu treffen. Die Bedingungen sind jedoch nicht exakt die gleichen.
In \textsl{3D-MATCHING} müssen gleich viele Tripel $|W|$
gewählt werden, wie es Teilnehmer $|T|$ gibt, in 
\textsl{TEAMBILDUNG} werden nur $\frac13$ so viele Teams gebildet
wie es Teilnehmer gibt. \textsl{3D-MATCHING} verlangt auch nicht,
dass sich die Tripel nicht überschneiden dürfen, sie dürfen sich
nur in keiner Koordinate überschneiden.
\end{diskussion}

\begin{bewertung}
Lösungsprinzip Reduktion ({\bf R}) 1 Punkt,
Auswahl eines geeigneten Vergleichsproblems ({\bf V}) 1 Punkt,
Reduktionsabbildung für Gäste ({\bf G}) 1 Punkt,
Teams ({\bf T}) 1 Punkt,
Ausschlussbedingung (jeder Teilnehmer in genau einem realisierten Team,
{\bf B}) 1 Punkt,
Schlussfolgerung NP-vollständig ({\bf S}) 1 Punkt.
\end{bewertung}
