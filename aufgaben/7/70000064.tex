% NP-vollständiges Problem
Ein Youtube-Fan hat eine lange Playlist von Videos gesammelt, die er
gerne ansehen möchte.
Er hasst es aber, wenn er beim Ansehen eines Videos unterbrochen wird.
Vor einer längeren Zugfahrt macht er sich daher Gedanken darüber,
welche Videos er sich ansehen soll.
Zu diesem Zweck stellt er eine neue Playlist zusammenzustellen,
in die er diejenigen Videos aus der ursprünglichen Liste kopiert,
die er auf der Zugfahrt ansehen will.
Natürlich will er die ganze zur Verfügung stehende Zeit vollständig
ausnutzen, aber das letzte Video soll nicht von der Ankunft am Ziel
unterbrochen werden.
Das stellt sich als schwieriger heraus als erwartet. Warum?

\begin{loesung}
Wir nennen das beschriebene Problem {\em TRAIN-RIDE-PLAYLIST} und 
zeigen, dass es sich eins zu eins auf das Problem {\em SUBSET-SUM}
reduzieren lässt.
Eine solche Reduktion ist
\begin{center}
\renewcommand{\arraystretch}{1.3}
\begin{tabular}{r>{$}c<{$}p{8cm}}
{\em SUBSET-SUM}&&{\em TRAIN-RIDE-PLAYLIST}\\
\hline
Liste $S$ von Zahlen
&\leftrightarrow&
Playlist mit Videos bekannter Länge
\\
Wert $t$
&\leftrightarrow&
Dauer der Zugfahrt
\\
Teilliste $T\subset S$
&\leftrightarrow&
Playlist für die Zugfahrt
\\
Summenbedingung $\displaystyle\sum_{s\in S} s = t$
&\leftrightarrow&
Gesamtdauer der Zugfahrt-Playlist ist gleich lang wie die Zugfahrt
\end{tabular}
\end{center}
Die Reduktion auf {\em SUBSET-SUM} zeigt, dass {\em TRAIN-RIDE-PLAYLIST}
ein NP-vollständiges Problem ist.
Nach aktuellem Wissen gibt es dafür keinen polynomiellen Algorithmus,
es skaliert daher nicht auf lange Playlists, wie sie in der Aufgabe
vorliegt.
\end{loesung}

\begin{bewertung}
NP-Vollständigkeit ({\bf N}) 1 Punkt,
Reduktionsprinzip ({\bf R}) 1 Punkt,
Vergleichsproblem wählen ({\bf V}) 1 Punkt,
Abbildung von $S$ ({\bf S}) 1 Punkt,
Abbildung von $T$ ({\bf T}) 1 Punkt,
Abbildung der Summenbedingung ({\bf B}) 1 Punkt.
\end{bewertung}

