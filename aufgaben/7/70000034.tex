Das Spiel {\em Light Up} wird auf einem $n\times m$ Feld gespielt.
Einzelne Felder sind schwarz gefärbt, manche davon sind zusätzlich
mit einer Zahl versehen.
Auf den weissen Quadraten müssen Glühbirnen plaziert werden,
so dass die folgenden Regeln eingehalten werden: 
\begin{itemize}
\item
Die Zahlen auf den schwarzen Quadraten geben an, wieviele Glühbirnen
auf weissen Feldern sind, die über eine Kante an dieses schwarze Feld grenzen. 
\item
Jedes weisse Feld wird von mindestens einer Glühbirne beleuchtet.
Eine Glühbirne leuchtet waagrecht und senkrecht bis zu einem schwarzen Feld
oder zum Rand des Spielfeldes. 
\item
Glühbirnen dürfen sich nicht gegenseitig beleuchten. 
\end{itemize}
Hier ein {\em Light Up}-Rätsel mit Lösung:
\begin{center}
\includeagraphics[width=0.33\hsize]{lightup.png}
\qquad
\includeagraphics[width=0.33\hsize]{lightup-solution.png}
\end{center}

Kann eine nichtdeterministische Turing-Maschine in polynomieller Zeit
entscheiden, ob ein {\em Light Up}-Rätsel eine Lösung hat?

\begin{loesung}
Eine nicht deterministische Turing-Maschine kann genau dann in polynomieller
Zeit entscheiden, ob ein {\em Light Up}-Rätsel eine Lösung hat,
wenn es einen polynomiellen Verifizierer gibt.
Als Lösungszertifikat verwenden wir die Platzierung der Glühbirnen.
Der Verifizierer muss dann folgendes überprüfen:
\begin{center}
\begin{tabular}{rll}
  &Verifikation&Aufwand\\
\hline
1.&\begin{minipage}[t]{12cm}\strut
Für jedes schwarze Feld mit einer Zahl überprüfe, ob die Anzahl
der Glühbirnen auf den vier Nachbarfeldern der Zahl entspricht.
\strut\end{minipage}&$O(4nm)$\\
2.&\begin{minipage}[t]{12cm}\strut
Für jedes weisse Feld, welches eine Glühbirne enthält kontrolliere,
ob in der gleichen Zeile oder Spalte eine weitere Glühbirne vorkommt.
\strut\end{minipage}&$O(mn(m+n))$\\
3.&\begin{minipage}[t]{12cm}\strut
Für jedes weisse Feld überprüfe, ob in der gleichen Zeile oder
Spalte eine Glühbiren vorhanden ist.
\strut\end{minipage}&$O(mn(m+n))$\\
\hline
&Gesamtaufwand&$O(mn(m+n))$
\end{tabular}
\end{center}
Da der Gesamtaufwand $O(mn(m+n)$ hängt polynomiell von $m$ und $n$
ab, somit haben wir einen polynomiellen Verifizierer konstruiert.
\end{loesung}

\begin{diskussion}
Brandon McPhail hat bewiesen, dass {\em Light Up} NP-vollständig ist,
wobei er eine ähnliche Technik verwendet hat wie sie im Skript zur
Beweisskizze der NP-Vollständigkeit von \textsl{MINESWEEPER-CONSISTENCY}
verwendet wurde.
Sein Paper ist unter
\url{http://www.mountainvistasoft.com/docs/lightup-is-np-complete.pdf}
herunterzuladen.
\end{diskussion}

\begin{bewertung}
Polynomieller Verifizierer ({\bf P}) 1 Punkt,
Lösungszertifikat ist definiert ({\bf Z}) 1 Punkt,
Verifikation schwarze Zahlfelder ({\bf S}) 1 Punkt,
Verifikation nur eine Glühbirne beleuchtet ein weisses Feld ({\bf 1}) 1 Punkt,
Verifikation jedes weisse Feld ist beleuchtet ({\bf W}) 1 Punkt,
Komplexitätsabschätzung für Verifikationsschritte ({\bf K}) 1 Punkt.
\end{bewertung}

