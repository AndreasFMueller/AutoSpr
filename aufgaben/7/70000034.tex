Das Spiel {\em Light Up} wird auf einem $n\times m$ Feld gespielt.
Einzelne Felder sind schwarz gef"arbt, manche davon sind zus"atzlich
mit einer Zahl versehen.
Auf den weissen Quadraten m"ussen Gl"uhbirnen plaziert werden,
so dass die folgenden Regeln eingehalten werden: 
\begin{itemize}
\item
Die Zahlen auf den schwarzen Quadraten geben an, wieviele Gl"uhbirnen
auf weissen Feldern sind, die "uber eine Kante an dieses schwarze Feld grenzen. 
\item
Jedes weisse Feld wird von mindestens einer Gl"uhbirne beleuchtet.
Eine Gl"uhbirne leuchtet waagrecht und senkrecht bis zu einem schwarzen Feld
oder zum Rand des Spielfeldes. 
\item
Gl"uhbirnen d"urfen sich nicht gegenseitig beleuchten. 
\end{itemize}
Hier ein {\em Light Up}-R"atsel mit L"osung:
\begin{center}
\includeagraphics[width=0.33\hsize]{lightup.png}
\qquad
\includeagraphics[width=0.33\hsize]{lightup-solution.png}
\end{center}

Kann eine nichtdeterministische Turing-Maschine in polynomieller Zeit
entscheiden, ob ein {\em Light Up}-R"atsel eine L"osung hat?

\begin{loesung}
Eine nicht deterministische Turing-Maschine kann genau dann in polynomieller
Zeit entscheiden, ob ein {\em Light Up}-R"atsel eine L"osung hat,
wenn es einen polynomiellen Verifizierer gibt.
Als L"osungszertifikat verwenden wir die Platzierung der Gl"uhbirnen.
Der Verifizierer muss dann folgendes "uberpr"ufen:
\begin{center}
\begin{tabular}{rll}
  &Verifikation&Aufwand\\
\hline
1.&\begin{minipage}[t]{12cm}\strut
F"ur jedes schwarze Feld mit einer Zahl "uberpr"ufe, ob die Anzahl
der Gl"uhbirnen auf den vier Nachbarfeldern der Zahl entspricht.
\strut\end{minipage}&$O(4nm)$\\
2.&\begin{minipage}[t]{12cm}\strut
F"ur jedes weisse Feld, welches eine Gl"uhbirne enth"alt kontrolliere,
ob in der gleichen Zeile oder Spalte eine weitere Gl"uhbirne vorkommt.
\strut\end{minipage}&$O(mn(m+n))$\\
3.&\begin{minipage}[t]{12cm}\strut
F"ur jedes weisse Feld "uberpr"ufe, ob in der gleichen Zeile oder
Spalte eine Gl"uhbiren vorhanden ist.
\strut\end{minipage}&$O(mn(m+n))$\\
\hline
&Gesamtaufwand&$O(mn(m+n))$
\end{tabular}
\end{center}
Da der Gesamtaufwand $O(mn(m+n)$ h"angt polynomiell von $m$ und $n$
ab, somit haben wir einen polynomiellen Verifizierer konstruiert.
\end{loesung}

\begin{diskussion}
Brandon McPhail hat bewiesen, dass {\em Light Up} NP-vollst"andig ist,
wobei er eine "ahnliche Technik verwendet hat wie sie im Skript zur
Beweisskizze der NP-Vollst"andigkeit von \textsl{MINESWEEPER-CONSISTENCY}
verwendet wurde.
Sein Paper ist unter
\url{http://www.mountainvistasoft.com/docs/lightup-is-np-complete.pdf}
herunterzuladen.
\end{diskussion}

\begin{bewertung}
Polynomieller Verifizierer ({\bf P}) 1 Punkt,
L"osungszertifikat ist definiert ({\bf Z}) 1 Punkt,
Verifikation schwarze Zahlfelder ({\bf S}) 1 Punkt,
Verifikation nur eine Gl"uhbirne beleuchtet ein weisses Feld ({\bf 1}) 1 Punkt,
Verifikation jedes weisse Feld ist beleuchtet ({\bf W}) 1 Punkt,
Komplexit"atsabsch"atzung f"ur Verifikationsschritte ({\bf K}) 1 Punkt.
\end{bewertung}

