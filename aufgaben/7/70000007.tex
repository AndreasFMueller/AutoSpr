Das Management eines Warenhauses möchte Lagerkosten sparen, und
kam daher auf folgende neue Lagerstrategie: es soll nur noch eine
so kleine Teilmenge von Artikeln an Lager gehalten werden, dass
jeder Kunde mindestens einen Artikel sofort mitnehmen kann, während
man in Kauf nimmt, dass einige Artikel nachgeliefert werden müssen.
Zur Vorbereitung hat man Bestellungen von vielen Kunden gesammelt.
Jetzt sind diejenigen Artikel zu bestimmen, die auch in Zukunft
noch an Lager gehalten werden sollen. Zeigen Sie, dass das Problem
zu entscheiden, ob man mit $k$ Artikeln am Lager bereits alle
Bestellungen abdecken kann, ein NP-vollständiges Problem ist.

\thema{NP-vollständig}
\thema{polynomielle Reduktion}

\begin{loesung}
Man kann das Problem wie folgt auf das Mengenüberdeckungsproblem
abbilden.
Wir denken uns die Produkte nummeriert mit natuerlichen Zahlen,
und die Menge der Kunden nennen wir $K$.
Zum Produkt mit der Nummer $i$ bilden wir die Menge
$S_i\subset K$ derjenigen Kunden, die dieses Produkt bestellt haben.
Gesucht wird jetzt eine Teilmenge $I$ der Produkte, die an Lager
genommen werden soll. Dass jeder Kunden aus dieser Teilmenge
bedient werden kann ist gleich bedeutend damit, dass die Teilmenge
der Kundenmengen $S_i, i\in I,$ bereits alle Kunden abdecken, also
\[
\bigcup_{i\in I}S_i=K.
\]
Umgekehrt kann man natürlich auch aus einer Familie von Mengen
$S_i$ die Bestellungen eines Kunden wieder rekonstruieren. Ein
Kunde $k$ bestellt
\[
\{i\in \mathbb N\;|\;k\in S_i\}.
\]
Die Frage, ob es $k$ Artikel gibt, mit denen man bereits alle
Kunden befriedigen kann ist als gleichbedeutend mit der Frage
nach einer $k$-"Uberdeckung mit Mengen aus $\{S_i\}$. Die
Existenz einer $k$-"Uberdeckung ist aber ein bekanntes NP-vollständiges
Problem.
\end{loesung}
