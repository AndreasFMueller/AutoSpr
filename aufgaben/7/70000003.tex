Beim Spiel Mine-Sweeper, wird dem Spieler von einigen der noch nicht
aufgedeckten Felder die Anzahl benachbarter Felder angezeigt, unter
denen sich eine Bombe befindet. Der Spieler soll dann nur diejenigen
Felder betreten, unter denen sich keine Bombe versteckt, und alle
Felder markieren, unter denen eine Bombe liegt. Betrachten Sie das
Problem {\it MINE-SWEEPER-CONSISTENCY}, in dem zu einer Belegung der
Felder mit Zahlen (der Anzahl Bomben auf Nachbarfeldern) entschieden
werden muss, ob diese Zahlen konsistent sind, ob also Bomben so
auf dem Spielfeld verteilt werden können, dass die Zahlen stimmen.
Zeigen sie, dass
\[
\text{\it MINE-SWEEPER-CONSISTENCY}\in \text{NP}.
\]

\begin{loesung}
Zunächst müssen wir sicherstellen, dass das Problem tatsächlich
entscheidbar sind.
Ist $N$ die Zahl der Felder, dann gibt es $2^N$ Bombenkonfigurationen.
Diese können in endlicher Zeit durchprobiert werden, also ist das
Problem entscheidbar.

Um herauszufinden, ob das Problem in NP ist, konstruieren wir
einen polynomiellen Verifizierer.
Als Lösungszertifikat brauchen wir die Verteilung der Bomben.
Zur Verifikation müssen wir dann die Anzahl der Bomben auf den
Nachbarfeldern zählen, was in Laufzeit $O(n^2)$ möglich ist,
wenn $n$ die Länge der längeren Seite des Spielfeldes ist.
Also hat
{\it MINE-SWEEPER-CONSISTENCY} einen polynomiellen Verifizierer und
ist damit in NP.
\end{loesung}
