Das 8-Damenproblem verlangt vom Spieler, 8 Damen so auf einem Schachfeld
zu platzieren, dass sie sich gegenseitig nicht schlagen können.
Eine Dame kann eine andere Figure in der gleichen Zeile, Spalte oder
in diagonaler Richtung schlagen.
Es kann aufwendig sein, eine Platzierung der Damen zu finden.
Daher wird in dieser Aufgabe nur das Problem studiert, eine Platzierung
der Damen zu codieren und zu entscheiden, ob sie eine Lösung ist.
\begin{teilaufgaben}
\item
Formulieren Sie das Problem als Sprachproblem.
\item
Ist es entscheidbar?
\end{teilaufgaben}

\begin{loesung}
\begin{teilaufgaben}
\item
Eine Lösung des $n$-Damenproblems enthält in jeder Zeile und
jeder Spalte des $n\times n$-Schachfeldes je genau eine Dame.
Ausserdem muss sichergestellt sein, dass sich die Damen über
die Diagonale nicht gegenseitig schlagen können.
Eine Lösung des $n$-Damenproblems kann daher geschrieben werden
als die Zeichenkette
\[
(n,r_1,r_2,\dots,r_n) \in \mathbb{N}^{n+1}
\]
codiert, wobei die Zahl $r_i$ die Zeilennummer der Dame in Spalte $i$
ist.
Nicht alle diese Zeichenketten sind Lösungen, die Zeichenketten, die
Lösungen beschreiben bilden die Sprache \textit{QUEENS}.
\item
Man kann alle möglichen Stellungen von Damen durchprobieren und prüfen,
ob sie Lösungen sind.
Da es nur endlich viele sind, nämlich $n^n$, ist dies möglich.
Die Laufzeit ist bei dieser Vorgehensweise natürlich exponentiell, aber
bei der Entscheidbarkeit steht dies nicht zur Diskussion.
\qedhere
\end{teilaufgaben}
\end{loesung}
