Das 8-Damen-Problem verlangt vom Spieler, 8 Damen so auf einem Schachfeld
zu platzieren, dass sie sich gegenseitig nicht schlagen können.
Eine Dame kann eine andere Figure in der gleichen Zeile, Spalte oder
in diagonaler Richtung schlagen.
Formulieren Sie das Problem als Sprachproblem.

\begin{loesung}
Eine Lösung des $n$-Damen-Problems enthält in jeder Zeile und
jeder Spalte des $n\times n$-Schachfeldes je genau eine Dame.
Ausserdem muss sichergestellt sein, dass sich die Damen über
die Diagonale nicht gegenseitig schlagen können.
Eine Lösung des $n$-Damen-Problems kann daher geschrieben werden
als die Zeichenkette
\[
(n,r_1,r_2,\dots,r_n) \in \mathbb{N}^{n+1}
\]
codiert, wobei die Zahl $r_i$ die Zeilennummer der Dame in Spalte $i$
ist.
Nicht alle diese Zeichenketten sind Lösungen, die Zeichenketten, die
Lösungen beschreiben bilden die Sprache \textit{QUEENS}.
\end{loesung}
