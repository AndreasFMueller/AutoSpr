Ein Tourist besucht ein fernes Land und kauft sich bei der Ankunft
am Flughafen einen Führer, der all die zahlreichen Sehenswürdigkeiten enthält,
die er gerne besuchen möchte.
Er sagt sich, dass er seine Besichtigungspläne am besten so organisiert,
dass er an jedem Tag eine Gruppe von Sehenswürdigkeiten  besucht, die
nahe beeinander liegen.
Als nahe beeinander liegend klassifiziert er jeweils solche, die in kurzer
Zeit mit öffentlichen Verkehrsmitteln erreichbar sind.
Für jeden Tag seines Aufenthalts könnte er dann genau eine Gruppe von
Sehenswürdigkeiten auswählen, so dass er am Ende alle gewünschten
Sehenswürdigkeiten gesehen hat und keine zweimal besucht.
Dank eines Sabbaticals war er zeitlich frei und stellte sich vor,
seinen Aufenhalt nötigenfalls zu verlängern, sollte er mit seinen
Besichtigungen nicht durchkommen.

Nach einigen fehlgeschlagenen Versuchen, das Problem im Kopf zu lösen,
setzt er sich in eine Caffeteria am Flughafen und arbeitet an dem Problem
weiter.
Er sagt sich, dass eine sorgfältige Planung durchaus einen ersten Tag
ohne Besuch einer Sehenswürdigkeit wert ist.
Als es dunkel wird ohne dass er eine Lösung gefunden hat, muss er sich
eingestehen, dass das Problem doch nicht ganz einfach ist und beschliesst,
am nächsten Morgen weiterzuarbeiten.
Schliesslich geht nichts über eine sorgfältige Planung.
Warum besteht die Gefahr, dass der arme Mann seine Heimreise antreten wird,
ohne auch nur eine einzige Sehenswürdigkeit besucht zu haben?

\thema{polynomielle Reduktion}
\themaL{NP-vollstandig}{NP-vollständig}

\begin{loesung}
Das beschriebene Problem {\em BESICHTIGUNGS-PLANUNG}
ist das NP-vollständige Problem {\em EXACT-COVER}, 
für das es nach aktuellem Wissen keinen polynomiellen Algorithmus gibt.
Die Menge $U$ besteht aus allen Sehenswürdigkeiten des Führers.
Die Teilmengen $S_j$ sind die Sehenswürdigkeiten, die der Tourist 
als nahe beeinander klassifiziert hat.
Die Vereinigung $\bigcup_{j=1}^n S_j$ umfasst die Sehenswürdigkeiten, die
der Tourist in seine Planung einbezogen hat.
Gesucht ist eine disjunkte Teilfamilie $S_{j_i}$, die alle Sehenswürdigkeiten
umfasst, also
\begin{align*}
S_{j_i}\cap S_{j_k}&=\emptyset \qquad \forall i\ne k
\\
\bigcup_{j=1}^n S_j &= \bigcup_{i=1}^m S_{j_i}.
\end{align*}
Die Reduktionsabbildung von {\em EXACT-COVER} auf 
{\em BESICHTIGUNGS-PLANUNG} ist
\begin{align*}
\text{\em EXACT-COVER}
&\leftrightarrow
\text{\em BESICHTIGUNGS-PLANUNG}
\\[5pt]
U&\leftrightarrow\{\text{alle Sehenswürdigkeiten}\}
\\
S_j&\leftrightarrow\{\text{nahe beeinander liegende Sehenswürdigkeiten}\}
\\
i&\leftrightarrow\text{Nummer des Besuchtstages}
\\
S_{j_i}&\leftrightarrow\text{Für Tag $i$ ausgewählte Gruppe von Sehenswürdigkeiten}
\end{align*}
Dies beweist, dass
$\text{\em EXACT-COVER}\le_{P}\text{\em BESICHTIGUNGS-PLANUNG}$
und damit dass {\em BESICHTIGUNGS-PLANUNG} NP-vollständig ist.
\end{loesung}

\begin{bewertung}
Reduktionsprinzip ({\bf R}) 1 Punkt,
Wahl eines Vergleichsproblems ({\bf V}) 1 Punkt,
Mapping ({\bf M}) 3 Punkte.
Folgerung NP-vollständig und Folgerung für Laufzeit ({\bf NP}) 1 Punkt.
\end{bewertung}


