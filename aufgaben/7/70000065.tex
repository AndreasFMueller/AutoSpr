Moderne massenproduzierte Puzzles setzen sich zusammen aus ungefähr
quadratischen Teilen, die an allen vier Seiten vorspringende {\em Nasen} oder
einspringende {\em Löcher} haben.
Die Grösse der Nasen und Löcher ist bei jedem Teil anders.
Die Teile lassen sich aber auf einem $n\times m$-Gitter so anordnen,
dass die Kanten der Teile zusammenpassen und so ein rechteckiges Bild
vom Format $n\times m$ ergeben.

\begin{center}
\begin{tikzpicture}[>=latex,thick]
\clip (-3,-2.1) rectangle (3,2.1);
%\node at (0,0) {\includeagraphics[width=6cm]{puzzle-mit-fehlendem-stueck-fehlende-puzzleteile.jpg}};
\node at (0,0) {\includeagraphics[width=6cm]{puzzle.jpg}};
\node[color=white] at (3.1,-2.1) [above left] {\tiny Bild von jcomp auf Freepik};
\end{tikzpicture}
\end{center}

In einem Sack befinden sich Teile für eine grosse Zahl von Puzzles der
selben Grösse $n\times m$.
Die Puzzles sind alle verschieden, keines ist vollständig.
Besonders an den Teilen ist, dass sie alle mit der Position beschriftet
sind, an der sie auf dem $n\times m$-Gitter platziert werden müssen.
Für jede Gitter-Position findet man mindestens ein Teil in dem Sack.
Die Teile sind zum Teil zwischen den Puzzles austauschbar, d.~h.~einzelne
Teile passen an der angegebenen Position in mehr als eines der Puzzles,
aber nicht unbedingt alle.

Man möchte herausfinden, ob man aus den Teilen ein Puzzle auswählen kann,
welches das $n\times m$-Bild vollständig bedeckt.
Die aufgedruckten Bilder müssen dabei nicht zusammenpassen.
Es stellt sich heraus, dass dies ziemlich zeitaufwendig ist.
Gibt es dafür eine Erklärung?

\begin{loesung}
Wir nennen das Problem, aus einem Sack zusammengemischter Puzzles
ein vollständiges Puzzle auszuwählen, {\em MIXED-PUZZLE-SELECTION}-Problem.
Man möchte ein bekanntes NP-vollständiges Problem darauf reduzieren.
Es muss sich um ein Problem handeln, welches von Mengen spricht, die eine
Menge überdecken.
Zur Auswahl stehen
\begin{itemize}
\item
{\em SET-COVERING}: Mengen überdecken das ganze Gebiet.
Das trägt dem Wunsch Rechnung, dass das ganze Puzzlefeld gefüllt werden
soll.
In {\em SET-COVERING} dürfen sich die Mengen überschneiden, die
Puzzle-Teile müssen aber passen.
{\em SET-COVERING} passt also nicht.
\item
{\em SET-PACKING}: Hier wird sichergestellt, dass sich die
Mengen nicht überschneiden, aber es ist nicht mehr garantiert,
dass die Mengen alles überdecken.
{\em SET-PACKING} passt also auch nicht.
\item
{\em EXACT-COVER}: Dieses Problem vereint die beiden Bedingungen,
dass alles abgedeckt werden muss und dass sich die Puzzle-Teile
nicht im Weg sein dürfen.
Dies ist das Problem, von dem man aus eine Reduktion versuchen kann.
\end{itemize}

Die folgende Eins-zu-Eins Reduktion zeigt, dass {\em MIXED-PUZZLE-SELECTION}
reduziert werden kann, das Problem {\em EXACT-COVER}:
\begin{center}
\renewcommand{\arraystretch}{1.3}
\begin{tabular}{r>{$}c<{$}l}
{\em EXACT-COVER}&&{\em MIXED-PUZZLE-SELECTION}
\\
\hline
Teilmenge $S_j\subset U$
&\leftrightarrow&
Puzzleteil
\\
Familie $(S_j)_{1\le j\le n}$ von Teilmengen
&\leftrightarrow&
alle Puzzleteile
\\
Unterfamilie $(S_{j_i})_{1\le i\le m}$
&\leftrightarrow&
Auswahl von Puzzleteilen
\\
Vereinigung $\bigcup_{j=1}^n S_j$
&\leftrightarrow&
Auswahl ergibt ein vollständiges Puzzle
\\
Keine Überlappung: $S_{j_i}\cap S_{j_k}=\emptyset$
&\leftrightarrow&
ausgewählte Teile passen exakt zusammen
\end{tabular}
\end{center}
Da {\em EXACT-COVER} NP-vollständig ist, zeigt die Reduktion, dass
auch {\em MIXED-PUZZLE-SELECTION} NP-vollständig ist.
Nach aktuellem Wissen gibt es dafür also keinen polynomiellen
Algorithmus, was den hohen Zeitaufwand für einen grossen
Sack mit Puzzleteilen erklärt.
\end{loesung}

\begin{bewertung}
NP-Vollständigkeit ({\bf N}) 1 Punkt,
Reduktionsprinzip ({\bf R}) 1 Punkt,
eines der oben diskutierten Vergleichsprobleme wählen ({\bf V}) 1 Punkt,
Familie $(S_j)$ ({\bf S}) 1 Punkt,
Teilfamilie $(S_{j_i})_i$ ({\bf T}) 1 Punkt,
Überdeckungsbedingung und Überlappungsverbot ({\bf U}) 1 Punkt.
\end{bewertung}


