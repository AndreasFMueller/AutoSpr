Formulieren Sie das $n$-Damenproblem von Aufgabe~\ref{70000067} als
polynomielles Ausfüllrätsel.

\themaL{polynomielles Ausfullratsel}{polynomielles Ausfüllrätsel}

\begin{loesung}
Das $n$-Damenproblem verlangt, in einem $n\times n$-Feld
Zeichen $\times$ einzutragen, welche die Platzierung der Damen
beschreiben.
Für jedes Feld, welches ein Zeichen $\times$ enthält, müssen
folgende Regeln erfüllt sein:
\begin{enumerate}
\item Jede Zeile enthält genau ein Zeichen $\times$
\item Jede Spalte enthält genau ein Zeichen $\times$
\item In der gleichen Diagonale kann nur ein Zeichen $\times$
vorhanden sein.
\end{enumerate}
Damit ist gezeigt, dass \textit{QUEENS} ein Ausfüllrätsel ist.
Es muss noch nachgewiesen werden, dass die Regeln 1--3 in 
polynomieller Zeit überprüft werden können.
\begin{center}
\begin{tabular}{r|p{10cm}|>{$}r<{$}}
Bedingung&Kontrolle&\text{Laufzeit}\\
\hline
1&In jeder Zelle, die ein $\times$ enthält, muss die gleiche Zeile
nach einem weiteren $\times$ durchsucht werden
&O(n^2) \\
2&In jeder Zelle, die ein $\times$ enthält, muss die gleiche Spalte
nach einem weiteren $\times$ durchsucht werden
&O(n^2) \\
3&In jeder Zelle, die ein $\times$ enthält, müssen die beiden
Diaogonalen, die sich in der Zelle kreuzen, nach einem weiteren
$\times$ durchsucht werden
&O(n^2)\\
\hline
&Total&O(n^2)
\end{tabular}
\end{center}
Der Rechenaufwand hängt somit polynomiell von der Grösse des Problems
ab, damit liegt ein polynomielles Ausfüllrätsel vor.
\end{loesung}
