In einem Entwicklungsland sollen die aus dem Ausland erhaltenen
Unterstützungsmittel dazu verwendet werden, endlich alle Ortschaften
mit mindestens 100 Einwohnern ans Stromnetz anzuschliessen.
Der Bau von Leitungen zwischen einzelnen Ortschaften ist je nach
Gelände unterschiedlich teuer, zum Teil auch schlicht unmöglich.
Es wird entschieden, dass man in einer ersten Phase auf Redundanz des
neu zu erstellenden Netzes verzichten will.
Der Minister möchte endlich wissen,
ob das vorhandene Geld für das Projekt ausreicht, und ist sehr ungehalten
darüber, dass die Verwaltung so lange braucht, diese Frage zu beantworten.
Kann man dies erklären?

\themaL{NP-vollstandig}{NP-vollständig}
\thema{polynomielle Reduktion}

\begin{loesung}
Dieses \textsl{STROMNETZ} genannte Problem ist äquivalent zu
\textsl{STEINER-TREE} wie folgt.
Die Ortschaften des Landes entsprechen der Menge $V$ aller Vertizes des
Graphen. Die untereinander zu verbindenden Ortschaften bilden eine Teilmenge
$R\subset V$.
Eine Kante im Graphen $G$ entspricht zwei verbindbaren Ortschaften,
jeder solchen Kante sind die Kosten für den Bau einer Verbindungsleitung
zugeordnet.
Es ist nun eine Menge von Verbindungsleitungen zu wählen, die einen Baum
(keine Redundanz) bilden und so, dass die Summe der Gewichte das 
Budget nicht übersteigt.
\begin{align*}
\textsl{STEINER-TREE}&\leftrightarrow\textsl{STROMNETZ}\\
\text{Knoten}&\leftrightarrow\text{Ortschaften}\\
\text{Knoten in $R$}&\leftrightarrow\text{zu erschliessende Ortschaften}\\
\text{Gewicht $w$ einer Kante}&\leftrightarrow \text{Baukosten einer Verbindungsleitung}\\
\text{maximales Gewicht $k$}&\leftrightarrow\text{Budget}
\end{align*}
Da das Problem \textsl{STEINER-TREE} bekanntermassen NP-vollständig ist
und daher keinen effizienten Lösungsalgorithmus besitzt, ist nicht
überraschend, dass das äquivalente Problem \textsl{STROMNETZ} von
der Verwaltung nicht effizient gelöst werden konnte.
\end{loesung}

\begin{diskussion}
\textsl{HAMPATH} ist nicht geeignet, denn Stromnetze werden ja nicht als eine einzige
Leitung gebaut, die nacheinander durch alle Ortschaften gezogen wird. Ein Stromnetz
ist nicht die Tour de Suisse.

\textsl{MAX-CUT} ist ebenfalls nicht geeignet. \textsl{MAX-CUT} sucht die maximalen
Investitionen, die man in den Sand setzen kann, indem man eine Menge von Verbindungen
durchschneidet. Das ist so ziemlich das Gegenteil von dem, was man tatsächlich machen
will.

Die Tatsache, dass man ein gewisses Budget $k$ ausgeben kann, könnte einen dazu
verleiten, eine Reduktion auf \textsl{SUBSET-SUM} zu versuchen. Aber in
\textsl{SUBSET-SUM} kann man nicht abbilden, dass die gebauten Leitungen ein
zusammenhängendes Netz bilden müssen. \textsl{SUBSET-SUM} wäre die richige
Wahl, wenn es darum ginge herauszufinden, ob man genau das gesamte Budget ausgeben
kann, unabhängig davon, ob damit die Ortschaften tatsächlich mit Strom versorgt
werden.
\end{diskussion}

\begin{bewertung}
Lösungsprinzip Reduktion ({\bf R}) 1 Punkt,
Auswahl eines geeigneten Vergleichsproblems ({\bf V}) 1 Punkt,
Reduktionsabbildung für Ortschaften ({\bf O}) 1 Punkt,
Leitungslänge ({\bf L}) 1 Punkt,
Budget {\bf B}) 1 Punkt,
Schlussfolgerung NP-vollständig ({\bf S}) 1 Punkt.
\end{bewertung}

