In einem Entwicklungsland sollen die aus dem Ausland erhaltenen
Unterst"utzungsmittel dazu verwendet werden, endlich alle Ortschafen
mit mindestens 100 Einwohnern ans Stromnetz anzuschliessen.
Der Bau von Leitungen zwischen einzelnen Ortschaften ist je nach
Gel"ande unterschiedlich teuer, zum Teil auch schlicht unm"oglich.
Es wird entschieden, dass man in einer ersten Phase auf Redundanz des
neu zu erstellenden Netzes verzichten will.
Der Minister m"ochte endlich wissen,
ob das vorhandene Geld f"ur das Projekt ausreicht, und ist sehr ungehalten
dar"uber, dass die Verwaltung so lange braucht, diese Frage zu beantworten.
Kann man dies erkl"aren?

\begin{loesung}
Dieses \textsl{STROMNETZ} genannte Problem ist "aquivalent zu
\textsl{STEINER-TREE} wie folgt.
Die Ortschaften des Landes entsprechen der Menge $V$ aller Vertizes des
Graphen. Die untereinander zu verbindenden Ortschaften bilden eine Teilmenge
$R\subset V$.
Eine Kante im Graphen $G$ entspricht zwei verbindbaren Ortschaften,
jeder solchen Kante sind die Kosten f"ur den Bau einer Verbindungsleitung
zugeordnet.
Es ist nun eine Menge von Verbindungsleitungen zu w"ahlen, die einen Baum
(keine Redundanz) bilden und so, dass die Summe der Gewichte das 
Budget nicht "ubersteigt.
\begin{align*}
\textsl{STEINER-TREE}&\leftrightarrow\textsl{STROMNETZ}\\
\text{Knoten}&\leftrightarrow\text{Ortschaften}\\
\text{Knoten in $R$}&\leftrightarrow\text{zu erschliessende Ortschaften}\\
\text{Gewicht $w$ einer Kante}&\leftrightarrow \text{Baukosten einer Verbindungsleitung}\\
\text{maximales Gewicht $k$}&\leftrightarrow\text{Budget}
\end{align*}
Da das Problem \textsl{STEINER-TREE} bekanntermassen NP-vollst"andig ist
und daher keinen effizienten L"osungsalgorithmus besitzt, ist nicht
"uberraschend, dass das "aquivalente Problem \textsl{STROMNETZ} von
der Verwaltung nicht effizient gel"ost werden konnte.
\end{loesung}

\begin{diskussion}
\textsl{HAMPATH} ist nicht geeignet, denn Stromnetze werden ja nicht als eine einzige
Leitung gebaut, die nacheinander durch alle Ortschaften gezogen wird. Ein Stromnetz
ist nicht die Tour de Suisse.

\textsl{MAX-CUT} ist ebenfalls nicht geeignet. \textsl{MAX-CUT} sucht die maximalen
Investitionen, die man in den Sand setzen kann, indem man eine Menge von Verbindungen
durchschneidet. Das ist so ziemlich das Gegenteil von dem, was man tats"achlich machen
will.

Die Tatsache, dass man ein gewisses Budget $k$ ausgeben kann, k"onnte einen dazu
verleiten, eine Reduktion auf \textsl{SUBSET-SUM} zu versuchen. Aber in
\textsl{SUBSET-SUM} kann man nicht abbilden, dass die gebauten Leitungen ein
zusammenh"angendes Netz bilden m"ussen. \textsl{SUBSET-SUM} w"are die richige
Wahl, wenn es darum ginge herauszufinden, ob man genau das gesamte Budget ausgeben
kann, unabh"angig davon, ob damit die Ortschaften tats"achlich mit Strom versorgt
werden.
\end{diskussion}

\begin{bewertung}
L"osungsprinzip Reduktion ({\bf R}) 1 Punkt,
Auswahl eines geeigneten Vergleichsproblems ({\bf V}) 1 Punkt,
Reduktionsabbildung f"ur Ortschaften ({\bf O}) 1 Punkt,
Leitungsl"ange ({\bf L}) 1 Punkt,
Budget {\bf B}) 1 Punkt,
Schlussfolgerung NP-vollst"andig ({\bf S}) 1 Punkt.
\end{bewertung}

