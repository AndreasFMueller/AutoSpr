Das Spiel Hitori () wird auf einem $n\times n$ Spielfeld gespielt.
(japanisch \begin{CJK}{UTF8}{min}ひとり\end{CJK})
In jedem Feld ist eine Zahl eingetragen.
Jetzt müssen so viele Felder schattiert werden, dass in jeder
Zeile und Spalte eine nicht schattierte Zahl nur einmal vorkommt.
Die schattierten Felder dürfen sich horizontal und vertikal nicht
berühren.
Es muss möglich sein, von einem hellen Feld zu jedem anderen hellen
Feld zu gelangen, indem man einem Weg über lauter helle Felder folgt,
die horizontal oder vertikal benachbart sind.

Hier ist ein Beispiel eines Hitori-Rätsels (links) mit Lösung (rechts).
Ebenfalls eingezeichnet ein Beispiel eines Weges, der nur auf den
hellen Feldern vom Feld $(2,1)$ links unten zum Feld $(8,1)$ führt.

\begin{center}

\begin{tikzpicture}[>=latex,thick,scale=0.6]

\def\gitter{
	\zelle{1}{1}{3}
	\zelle{2}{1}{1}
	\zelle{3}{1}{5}
	\schattiert{4}{1}{1}
	\zelle{5}{1}{7}
	\zelle{6}{1}{2}
	\zelle{7}{1}{6}
	\schattiert{8}{1}{2}

	\schattiert{1}{2}{2}
	\zelle{2}{2}{8}
	\zelle{3}{2}{7}
	\zelle{4}{2}{4}
	\zelle{5}{2}{5}
	\zelle{6}{2}{6}
	\schattiert{7}{2}{5}
	\zelle{8}{2}{2}

	\zelle{1}{3}{7}
	\zelle{2}{3}{3}
	\zelle{3}{3}{8}
	\schattiert{4}{3}{6}
	\zelle{5}{3}{4}
	\schattiert{6}{3}{2}
	\zelle{7}{3}{2}
	\zelle{8}{3}{1}

	\schattiert{1}{4}{2}
	\zelle{2}{4}{7}
	\schattiert{3}{4}{7}
	\zelle{4}{4}{8}
	\zelle{5}{4}{6}
	\zelle{6}{4}{5}
	\zelle{7}{4}{4}
	\zelle{8}{4}{3}

	\zelle{1}{5}{8}
	\zelle{2}{5}{4}
	\zelle{3}{5}{6}
	\schattiert{4}{5}{1}
	\zelle{5}{5}{2}
	\zelle{6}{5}{1}
	\zelle{7}{5}{7}
	\schattiert{8}{5}{1}

	\zelle{1}{6}{2}
	\zelle{2}{6}{5}
	\zelle{3}{6}{4}
	\zelle{4}{6}{6}
	\schattiert{5}{6}{4}
	\zelle{6}{6}{3}
	\schattiert{7}{6}{4}
	\zelle{8}{6}{7}

	\zelle{1}{7}{4}
	\schattiert{2}{7}{6}
	\zelle{3}{7}{2}
	\schattiert{4}{7}{5}
	\zelle{5}{7}{1}
	\zelle{6}{7}{7}
	\zelle{7}{7}{8}
	\zelle{8}{7}{6}

	\schattiert{1}{8}{2}
	\zelle{2}{8}{2}
	\zelle{3}{8}{3}
	\zelle{4}{8}{5}
	\schattiert{5}{8}{3}
	\zelle{6}{8}{8}
	\schattiert{7}{8}{3}
	\zelle{8}{8}{4}
}

\def\netz{
	\foreach \x in {0.5,...,8.5}{
		\draw (\x,0.5)--(\x,8.5);
	}

	\foreach \y in {0.5,...,8.5}{
		\draw (0.5,\y)--(8.5,\y);
	}
}

\def\zelle#1#2#3{
%\fill[color=white] ({#1-0.5},{9-#2-0.5}) rectangle ({#1+0.5},{9-#2+0.5});
\node[color=black] at (#1,{9-#2}) {$#3$};
}
\def\schattiert#1#2#3{
%\fill[color=white] ({#1-0.5},{9-#2-0.5}) rectangle ({#1+0.5},{9-#2+0.5});
\node[color=black] at (#1,{9-#2}) {$#3$};
}

\begin{scope}[xshift=-5cm]
\netz
\gitter
\end{scope}

\def\schattiert#1#2#3{
\fill[color=black] ({#1-0.5},{9-#2-0.5}) rectangle ({#1+0.5},{9-#2+0.5});
\node[color=white] at (#1,{9-#2}) {$#3$};
}

\begin{scope}[xshift=5cm]
\netz
\draw[color=red!20,line width=10pt,round cap-round cap,line join=round]
	(1.7,1)--(3,1)--(3,4)--(2,4)--(2,7)--(5,7)--(5,4)--(6,4)--(6,2)
	--(8,2)--(8,0.7);
\gitter
\end{scope}

\end{tikzpicture}

\end{center}
Kann eine nichtdeterministische Turing-Maschine in polynomieller Zeit
entscheiden, ob ein Hitori-Rätsel eine Lösung hat?

\thema{NP}
\thema{polynomieller Verifizierer}

\begin{loesung}
Die Frage ist gleichbedeutend damit, ob die Entscheidung, ob ein Hitori-Rätsel
eine Lösung hat ein Problem in der Klasse N ist.
Sie ist ausserdem gleichbedeutend damit, dass es einen polynomiellen
Verifizierer gibt.
Zur Konstruktion eines solchen Verifizierer müssen wir erst klarstellen,
was für ein Lösungszertifikat verwendet werden soll.
Die schattierten Felder bilden das Lösungszertifikat.
Der Verifikationsalgorithmus muss dann die folgenden Prüfungen vornehmen
\begin{center}
\renewcommand{\arraystretch}{1.15}
\begin{tabular}{c|p{10cm}|>{$}c<{$}}
%\hline
Schritt&Verifikation&\text{Laufzeit}\\
\hline
1&Für jedes helle Feld, überprüfe dass dieselbe Zahl in
keinem anderen Feld der Zeile vorkommt&n^2 \cdot O(n) \\
2&Für jedes helle Feld, überprüfe dass dieselbe Zahl in
keinem anderen Feld der Spalte vorkommt&n^2 \cdot O(n) \\
3&Mit Hilfe eines Markierungsalgorithmus finde alle hellen Felder, die
mit dem ersten hellen Feld verbunden sind.
Dazu muss man höchstens $n^2$ mal durch das ganze Spielfeld der
Grösse $n^2$ hindurchgehen
und noch nicht markierte helle Felder markieren (Aufwand $O(1)$),
bis keine weiteren Felder markiert werden können.
Am Ende muss man prüfen, ob alle hellen Felder markiert worden sind,
was in $O(n^2)$ möglich ist.
&O(n^4)\\
\hline
&Total&O(n^4)\\
\hline
\end{tabular}
\end{center}
Somit ist die Verifikation in polynomieller Zeit unabhängig von der
Grösse des Zertifikats möglich.
\end{loesung}

\begin{diskussion}
Man kann zeigen, dass Hitori NP-vollständig ist.
\end{diskussion}

\begin{bewertung}
Entscheidbarkeit ({\bf E}) 1 Punkt,
Verifizierer ({\bf V}) 1 Punkt,
Zertifikat ({\bf Z}) 1 Punkt,
Verifikationsschritte ({\bf S}) 2 Punkte,
Komplexitätsabschätzung ({\bf K}) 1 Punkt.
\end{bewertung}

