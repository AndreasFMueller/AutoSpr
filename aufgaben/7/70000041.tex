Die japanische Designerin Non Ishida hat im Jahre 1986 eine Art von
Logikrätsel erfunden, die ihr zu Ehren Nonogramme genannt werden.
In einem $n\times m$ Spielfeld sind die einzelnen Felder mit einer Farben
zu füllen.
Am Rand des Feldes wird angegeben, wie lange Folgen gleichfarbiger
Felder in einer Zeile oder Spalte jeweils sind.

Das Problem {\em NONOGRAM} ist also die Aufgabe, zu entscheiden, ob
ein Nonogramm-Rätsel überhaupt lösbar ist.
\begin{teilaufgaben}
\item Ist das Problem entscheidbar?
\item Kann eine nichtdeterministische Turing-Maschine das Problem in
polynomieller Zeit lösen?
\end{teilaufgaben}

\begin{loesung}
\begin{teilaufgaben}
\item
\item
\end{teilaufgaben}
\end{loesung}




