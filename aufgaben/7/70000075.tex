Verwandeln Sie die Formel
\[
(x_1\vee x_2\vee\overline{x}_3\vee\overline{x}_4\vee x_5)
\wedge
(x_6\vee x_7\vee \overline{x}_8\vee x_9)
\wedge
(x_{10}\vee \overline{x}_{11})
\wedge
x_{12}
\]
in eine erfüllungsäquivalente 3CNF-Formel.

\begin{loesung}
Die Terme mit mehr als drei Variablen müssen mit Hilfe zusätzlicher
Variablen aufgespaltet werden:
\begin{align*}
&(x_1\vee x_2\vee\overline{x}_3\vee\overline{x}_4\vee x_5)
\intertext{wird zu}
&
(x_1\vee x_2\vee z_1)
\wedge
(\overline{z}_1\vee \overline{x}_3\vee z_2)
\wedge(
\overline{z}_2\vee\overline{x}_4\vee z_3)
\wedge
(\overline{z}_3\vee x_5\vee z_4),
\intertext{aus}
&(x_6\vee x_7\vee \overline{x}_8\vee x_9)
\intertext{wird}
&
(x_6\vee x_7\vee z) \wedge(\overline{z}\vee \overline{x}_8\vee x_9).
\end{align*}
Die Terme mit weniger als drei Variablen können durch Wiederholung
einzelner Variablen passend gemacht werden:
\begin{align*}
x_{10}\vee\overline{x}_{11}
&\rightarrow
x_{10}\vee \overline{x}_{11} \vee \overline{x}_{11}
\\
x_{12}
&\rightarrow
x_{12}\vee x_{12}\vee x_{12}.
\end{align*}
Alles zusammen gibt die erfüllungsäquivalente Formel
\begin{align*}
&(x_1\vee x_2\vee z_1)
\wedge
(\overline{z}_1\vee \overline{x}_3\vee z_2)
\wedge(
\overline{z}_2\vee\overline{x}_4\vee z_3)
\wedge
(\overline{z}_3\vee x_5\vee z_4)
\\
\mathstrut\wedge\mathstrut
&(x_6\vee x_7\vee z)
\wedge
(\overline{z}\vee \overline{x}_8\vee x_9)
\\
\mathstrut\wedge\mathstrut
&(x_{10}\vee \overline{x}_{11} \vee \overline{x}_{11})
\\
\mathstrut\wedge\mathstrut
&(x_{12}\vee x_{12}\vee x_{12}).
\qedhere
\end{align*}
\end{loesung}
