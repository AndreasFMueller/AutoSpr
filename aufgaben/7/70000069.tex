Formulieren Sie das $n$-Damenproblem als $k$-\textit{CLIQUE}-Problem.

\begin{loesung}
In einem $k$-\textit{CLIQUE}-Problem müssen Knoten eines Graphen
ausgewählt werden, die alle miteinander verbunden sind.
Für das Problem \textit{QUEENS} kann man für jedes Feld des Spielfeldes
einen Knoten des Graphen anlegen.
Zwei Knoten werden miteinander verbunden, wenn Damen auf den zugehörigen
Feldern sich nicht gegenseitig schlagen können.
So entsteht ein Graph mit $n^2$ Knoten.
Eine Lösung des $n$-Damenproblems ist eine Auswahl von $n$ Feldern,
auf denen Damen platziert werden können, die sich gegenseitig
nicht schlagen können.
Dies entspricht der Auswahl von $n$ Knoten des Graphen, die sich
gegenseitig nicht schlagen können.
Letzteres ist gleichbedeutend damit, dass die ausgewählten Knoten
alle miteinander verbunden sind.

Dem $n$-Damenproblem entspricht daher eine Instanz des
$n$-\textit{CLIQUE}-Problems.
\end{loesung}
