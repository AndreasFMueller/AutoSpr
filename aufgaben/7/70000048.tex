{\em Nurimaze}
(\text{\begin{CJK}{UTF8}{min}ぬりめいず\end{CJK}})
ist ein japanisches Logikrätsel, welches auf einem
$n\times m$-Feld gespielt wird,
welches von fetten Linien in kleine Gebiete unterteilt wird.
Der Spieler muss nun
Gebiete grau einfärben, die keines der Zeichen $S$, $Z$, Dreieck oder
Kreis enthalten.
Die verbleibenden weissen Felder bilden ein Labyrinth.
Weder die weissen noch die grauen Felder dürfen irgendwo einen Bereich
der Grösse $2\times 2$ enthalten.
Die weissen Felder müssen einen zusammenhängenden Bereich bilden.
Sie bilden ein Wegnetz im Labyrinth, welches keinen Rundweg enthalten darf.
Es muss einen Weg von $S$ nach $Z$ geben, der alle mit Kreis markierten 
Felder enthält aber keines der mit Dreieck markierten Felder.

Das folgende Beispiel zeigt links das Rätsel und rechts die Lösung.
Der gesuchte Weg ist rosa (hellgrau) eingezeichnet.

\begin{center}
\includeagraphics[]{nurimaze.pdf}
\end{center}
Kann eine nichtdeterministische Turingmaschine in polynomieller Zeit
entscheiden, ob ein Nurimaze-Rätsel lösbar ist?

\thema{NP}
\thema{polynomieller Verifizierer}

\begin{loesung}
Das Problem ist sicher entscheidbar, indem man jede beliebige Einfärbung
von Gebieten durchprobiert.

Es ist von einer nichtdeterministischen Turingmaschine in polynomieller 
Zeit genau dann entscheidbar, wenn es einen polynomiellen Verifizierer
gibt.
Als Lösungszertifikat $c$ verlangen wir die grau einzufärbenden Felder
und den rosa eingezeichneten Weg von $S$ nach $Z$.
Dann müssen folgende Verifikationen vorgenommen werden:
\begin{center}
\begin{tabular}{c|p{13cm}|>{$}c<{$}}
&Verifikation&\text{Aufwand}\\
\hline
1&Kein $2\times 2$-Bereich im weissen Gebiet&O(mn)\\
2&Kein $2\times 2$-Bereich im grauen Gebiet&O(mn)\\
3&Alle Kreise auf dem rosa Weg&O(mn)\\
4&Keine Dreiecke auf dem rosa Weg&O(mn)\\
5&Keine Kreise oder Dreiecke im grauen Gebiet&O(mn)\\
6&$S$ und $Z$ auf dem rosa Weg&O(mn)\\
7&Mit einem Markierungsalgorithmus, der ausgehend von $S$ im weissen Bereich
iterativ alle benachbarten Felder bereits markierter Felder markiert,
feststellen, ob das weisse Gebiet zusammenhängend ist.&O(m^2n^2)\\
8&Überprüfen, ob das weisse Gebiet keine Rundwege enthält. Dies
kann dadurch geschehen, indem man von jedem schwarzen Teilgebiet prüft, ob
es von einem Weg von weissen Feldern umgeben ist.&O(m^2n^2)\\
\hline
&Total&O(m^2n^2)\\
\hline
\end{tabular}
\end{center}
Der Verifizierer hat also polynomielle Laufzeit.
\end{loesung}

\begin{bewertung}
Entscheidbarkeit ({\bf E}) 1 Punkt,
Verifizierer ({\bf V}) 1 Punkt,
Zertifikat ({\bf Z}) 1 Punkt,
Verifikationsschritte ({\bf S}) 2 Punkte,
Komplexitätsabschätzung ({\bf K}) 1 Punkt.
\end{bewertung}



