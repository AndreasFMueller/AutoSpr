Ein Graph kann mit Hilfe einer Tabelle wie folgt codiert werden.
Die Tabelle hat für jeden Knoten des Graphen eine Zeile und eine Spalte.
Das Feld in Zeile $i$ und Spalte $j$ beschreibt also das Paar der
Knoten $i$ und $j$.
Alle Felder der Diagonale werden schwarz gefüllt.
Das Feld $(i,j)$  wird genau dann weiss gelassen,
wenn es eine Kante von $i$ nach $j$ gibt, sonst wird es schwarz gefüllt.

\begin{center}
\includeagraphics[]{graph.pdf}
\includeagraphics[]{graph-loesung.pdf}
\end{center}

\thema{NP}
\thema{polynomielle Reduktion}
\thema{polynomieller Verifizierer}
\themaL{polynomielles Ausfullratsel}{polynomielles Ausfüllrätsel}

\begin{teilaufgaben}
\item
Ein hamiltonscher Zyklus ist ein geschlossener Weg durch den Graphen, der
jeden Knoten des Graphen genau einmal besucht.
Er besteht aus einer Folge von Kanten, so dass jeder Knoten genau 
einmal als Anfangs- und einmal als Endpunkt einer Kante vorkommt.
Wie kann man einen hamiltonschen Zyklus mit Hilfe der Tabelle codieren?
\item
Ist es möglich, mit einer nicht deterministischen Maschine in polynomieller
Zeit zu entscheiden, ob ein Graph einen hamiltonschen Zyklus enthält?
\end{teilaufgaben}

\begin{loesung}
\begin{teilaufgaben}
\item
Ein Pfad im Graphen kann festgelegt werden, indem man die Kanten, die Teil
des Pfades sind, mit roten Punkten in der Tabelle markiert.
Ein hamiltonscher Zyklus existiert dann genau dann, wenn sich in der Tabelle
rote Punkte in den weissen Feldern platzieren lassen so, dass in jeder
Zeile und Spalte genau zwei rote Punkte liegen und dass ausserdem die
Punkteanordnung symmetrisch bezüglich Spiegelung an der Diagonalen ist.
\item
Offenbar ist das finden eines hamiltonschen Zyklus ein polynomielles
Ausfüllrätsel.
Die Frage, ob ein hamiltonscher Zyklus existiert ist daher gleichbedeutend
mit der Frage, ob das zugehörige polynomielle Ausfüllrätsel eine Lösung hat
und ist damit in NP.
\qedhere
\end{teilaufgaben}
\end{loesung}

\begin{bewertung}
\begin{teilaufgaben}
\item Markierungen für Kanten des hamiltonschen Pfades ({\bf M}) 1 Punkt,
genau zwei Punkte in jeder Zeile/Spalte ({\bf Z}) 1 Punkt.
\item Polynomielles Ausfüllrätsel ({\bf P}) 4 Punkte. Alternativ kann
man darauf hinweisen, dass es einen polynomiellen Verifizierer braucht
({\bf V}) 1 Punkt, man muss das Zertifikat spezifizieren ({\bf C}) 1 Punkt,
den Verifikationsalgorithmus ({\bf A}) 1 Punkt und die Laufzeitabschätzung
({\bf L}) 1 Punkt.
\end{teilaufgaben}
\end{bewertung}




