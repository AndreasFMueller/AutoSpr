Das Heron-Verfahren zur numerischen Berechnung der Quadratwurzel einer Zahl
$a$ wird im Wikipedia-Artikel
\url{https://de.wikipedia.org/wiki/Heron-Verfahren}
beschrieben.
Er beruht auf der Iteration
\[
x_{n+1}= \frac12\biggl(x_n+\frac{a}{x_n}\biggr),\qquad x_0=1.
\]
Im Wikipedia-Artikel findet man auch Information darüber, wieviele
Iterationen notwendig sind, um eine bestimmte Genauigkeit zu erreichen.
Ist dieser Algorithmus in P?

\thema{Berechenbarkeit}

\begin{loesung}
Im Wikipedia-Artikel steht, dass für eine Zahl mit Länge $n$ etwa $n$
Iterationen notwendig sind, um $n$ Signifkante Stellen zu finden.
Danach mögen zusätzliche Iterationen nötig sein, um die volle Genauigkeit
zu erhalten, doch deren Anzahl ist geringer.
Für die Abschätzung des Aufwandes reicht es daher, davon auszugehen, dass
$O(n)$ Iterationen notwendig sind.

Eine Iteration besteht in einer Division $a/x_n$, einer Addition und
einer Division durch $2$. 
Addition und Division durch $2$
fallen nicht ins Gewicht, da sie in $O(n)$
implementiert werden können.
Aufwendiger ist die Division, der aus $O(n)$ Schritten besteht, wobei
in jedem Schritt eine Multiplikation durchgeführt werden muss, die
maximal $O(n^2)$ dauert.
Ein Iterationsschritt lässt sich also sicher in der Zeit $O(n^3)$ durchführen,
der ganze Algorithmus daher in $O(n^4)$.
Insbesondere ist die Laufzeit polynomiell.
\end{loesung}
