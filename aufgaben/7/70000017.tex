Ein verrückter Elektriker hat in einem Labyrinth Lampen installiert.
Am Eingang des Labyrinthes befindet sich eine Schalttafel mit einer
Anzahl Schalter. Jeder Schalter ist ein Umschalter, er ist mit
zwei Stromkreisen verbunden. Jeder Stromkreis ist mit Lampen in
den einzelnen Räumen des Labyrinthes verbunden, nicht unbedingt in
allen.
Wenn der
Schalter umgelegt wird, gehen die Lampen des einen
Stromkreises aus, dafür gehen die Lampen des anderen Stromkreises an.
In jedem Raum gibt es mindestens eine Lampe.

Zeigen Sie: Das Problem, eine Schalterstellung zu finden, so
dass in allen Räumen eines solchen Labyrinthes Licht brennt,
ist NP-vollständig.

\themaL{NP-vollstandig}{NP-vollständig}
\thema{polynomielle Reduktion}

\begin{loesung}
Wir zeigen, dass dieses Problem äquivalent ist mit SAT. Jedem Schalter
entspricht eine logische Variable $x_i$. Die zwei Stromkreise entsprechen
den beiden Werten der Variable $x_i$. Die Lampen des ersten Kreises leuchten,
wenn $x_i$ wahr ist, die Lampen des zweiten Kreises leuchten, wenn
$\overline x_i$ wahr ist.
Jedem Raum des Labyrinthes entspricht eine logische Formel der Form
\[
x_{i_1}\vee \cdots \vee x_{i_k}\vee \overline x_{i_{k+1}}\vee \cdots\vee \overline x_{i_{k+l}}.
\]
$x_i$ kommt in der Formel vor, wenn der erste Stromkreis von Schalter $i$
mit einer Lampe in diesem Raum verbunden ist, $\overline x_i$ kommt vor,
wenn der zweite Stromkreis dieses Schalters mit einer Lampe verbunden ist.
Die Formel wird wahr, wenn mindestens eine Lampe leuchtet. In jedem Raum
des Labyrinthes hat es also Licht, wenn die Und-Verknüpfung all dieser
Ausdrücke wahr ist.

Da sich jede logische Formel in konjunktive Normalform bringen lässt,
die sich wieder als Beleuchtungsproblem in einem Labyrinth interpretieren
lässt, ist das Erfüllbarkeitsproblem für logische Formeln (also SAT)
äquivalent zum Ausleuchtungsproblem für ein Labyrinth, somit ist
das gestellte Problem NP-vollständig.
\end{loesung}
