Das Miniatur Wunderland\footnote{\url{https://www.miniatur-wunderland.de/}}
in Hamburg beherbergt die mit
16138\,m Gleislänge auf 1545\,m$\mathstrut^2$ Anlagefläche grösste
Modelleisenbahnanlage der Welt.
Auf der Anlage sind 1120 Züge, 10250 Autos und 47 Flugzeuge unterwegs.
Die ganze Anlage mit 1392 Signalen, 3454 Weichen und über 497000 LEDs
wird von 50 Computern gesteuert.

Um die Betriebssicherheit im Dauerbetrieb zu gewährleisten, müssen
die Gleise ständig sauber gehalten werden.
Dafür stehen Reinigungszüge zur Verfügung, die in regelmässigen
Abständen auf vorprogrammierten Routen über die Anlage geführt
werden.
Natürlich sind auch noch viele andere Züge ebenfalls auf dem Schienennetz
unterwegs, allerdings nicht ständig, die Loks sind Spielzeug, welches
nicht für den Dauerbetrieb ausgelegt ist, sie müssen daher regelmässig
Pausen zum Abkühlen einlegen.
Die zeitliche Planung der Zugfahrten soll so gestaltet sein, dass
innerhalb vorgegebener Zeitintervalle jeder Zug seine Fahrt durchführt.
Die Zugfahrten der Reinigungszüge sind besonders wichtig.
Fallen diese Fahren aus oder finden Sie zu spät statt, 
kann es zu Ausfällen kommen, die den Besuch des Miniatur Wunderlandes
weniger attraktiv machen.
Man kann sogar versuchen, den dadurch verursachten Schaden zu beziffern.
Da die Reinigungszüge nicht so interessant sind wie die normalen
Modellzüge, soll immer nur ein Reinigungszug aufs Mal unterwegs sein.
Ziel ist natürlich, den Schaden durch Ausfälle möglichst klein zu halten.

Trotzdem scheint das MiWuLa keine Anstrengungen unternommen zu haben, die
Reinigungszugfahrten zum Beispiel in ein kleines Interval vor der Öffnung
am Morgen zu planen und trotzdem die Kosten für Betriebsausfälle
unter der Schranke $k$ zu behalten.
Möglicherweise ist das einfach für eine so umfangreiche Anlage ein
zu schwieriges Problem.
Woran könnte das liegen?

\themaL{NP-vollstandig}{NP-vollständig}
\thema{polynomielle Reduktion}

\begin{loesung}
Das Planungsproblem für die Reinigungszüge ist äquivalent zum
NP-vollständigen Problem \textsl{SEQUENCING}, wie man mit Hilfe
einer polynomiellen Reduktion zeigen kann.
Dazu seien die $t_i$ die Fahrzeiten für jeden der $p$ Reinigungszüge,
$d_i$ ist die späteste Fahrzeit, zu der die Zugfahrt von Zug
$i$ abgeschlossen sein muss, und $s_i$ sind die Kosten für eine verspätete
Reinigung durch den Zug $i$.
\begin{align*}
\text{Fahrzeit von Zug $i$}
&
\leftrightarrow\text{Ausführungszeit $t_i$ von Job $i$}
\\
\text{Deadline für die Fahrt von Zug $i$}
&
\leftrightarrow d_i
\\
\text{Kosten für Verspätete Reinigungsfahrt $i$}
&
\leftrightarrow s_i
\\
\text{Reihenfolge der Fahrten}&
\leftrightarrow \text{Permutation $\pi$}
\end{align*}
Die Aufgabe besteht darin, die Reihenfolge $\pi$ der Reinigungszugfahrten so
zu bestimmen, dass die Kosten für verspätte Reinigungs unter $k$ zu halten,
dies ist genau die Aufgabe des \textsl{SEQUENCING}-Problems.
Damit ist gezeigt, dass das Reinigungszugsplanungsproblem NP-vollständig ist.
Der Aufwand zur Planung der Reinigungszugsplanung steigt daher nach heutigem
Wissen exponentiell mit der Grösse des Problems.
\end{loesung}

\begin{bewertung}
Reduktionsidee ({\bf R}) 1 Punkt,
Auswahl eines Vergleichsproblems ({\bf V}) 2 Punkte,
Mapping ({\bf M}) 2 Punkte,
NP-Vollständigkeit und Schlussfolgerung ({\bf S}) 1 Punkt.
\end{bewertung}
