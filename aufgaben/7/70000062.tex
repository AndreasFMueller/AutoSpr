Das Miniatur Wunderland\footnote{\url{https://www.miniatur-wunderland.de/}}
in Hamburg beherbergt die mit
15715\,m Gleislänge auf 1499\,m$\mathstrut^2$ Anlagefläche grösste
Modelleisenbahnanlage der Welt.
Auf der Anlage sind 1040 Züge, 9250 Autos und 52 Flugzeuge unterwegs.
Die ganze Anlage mit 1380 Signalen, 3454 Weichen und über 479000 LEDs
wird von 50 Computern gesteuert.
Die Steuerung einer solchen Anlage teilt das Gleisnetz in Blöcke ein.
Die Züge fahren auf den Gleisen von Block zu Block, wobei oft nur
die Fahrt in einer Richtung möglich ist.

fuxbeck\footnote{Playlist der Kamerafahrten von fuxbeck im Miniatur Wunderland:
\url{https://youtube.com/playlist?list=PLCuBWutVb71b5fWFnQ-LyPo4xqt30kaBe}}
ist ein Youtuber, der viele Modelleisenbahnanlagen besucht,
mit Einwilligung der Betreiber einen Videokamera-Wagen auf der Anlage
fahren lässt und die Videos auf seinem Kanal veröffentlicht.
Am einfachsten wäre es natürlich, wenn die Steuerung der Anlage den Zug
mit dem Kamera-Wagen so führt, dass jeder Block genau einmal durchfahren
wird.
Bei einer grossen Anlage wie der des Miniatur Wunderlandes ist dies
vielleicht nicht möglich, weil nicht alle Anlageteile auch tatsächlich
miteinander verbunden werden.
Sicher ist es aber möglich innerhalb eines Abschnitts wie zum Beispiel
des 2005 eröffneten Skandinavien-Abschnittes mit 2000\,m Gleislänge,
600 Weichen, 150 Zügen und 16 Schiffen, gesteuert von 7 Computern.
Trotzdem hat fuxbeck zwei verschiedene Videos zum Skandinavien-Abschnitt
auf seinem Kanal.
Möglicherweise stand nicht genügend Zeit zur Planung einer einzigen Fahrt
durch alle Blöcke des Abschnitts zur Verfügung.
Warum könnte das so sein?

\themaL{NP-vollstandig}{NP-vollständig}
\thema{polynomielle Reduktion}

\begin{loesung}
Das Problem zu entscheiden, ob es einen Pfad gibt, ist äquivalent zum
NP-vollständigen Problem \textsl{HAMPATH}, wie die folgende polynomielle
Eins-zu-Eins-Reduktion zeigt.
Die Blöcke bilden die Knoten eines Graphen, die möglichen Fahrten von Block
zu Block bilden die gerichteten Kanten eines gerichteten Fragen.
Die Aufgabe, eine Fahrt durch alle Blöcke zu finden besteht jetzt darin,
einen Hamiltonschen Pfad durch diesen Graphen zu planen:
\begin{align*}
\text{Knoten} &\leftrightarrow \text{Block}
\\
\text{gerichtete Kante} &\leftrightarrow \text{mögliche Fahrt von einem Block zu anderen}
\\
\text{Hamiltonscher Pfad}&\leftrightarrow \text{Kamerafahrt}
\end{align*}
Damit ist gezeigt, dass das Planungsproblem für die Kamerafahrt ebenfalls
NP-vollständig ist.
Nach aktuellem Wissen muss man davon ausgehen, dass die Laufzeit für die
Lösung des Problems exponentiell mit der Grösse des Anlagenabschnitts
wächst.
\end{loesung}

\begin{diskussion}
Es wurde auch versucht, das Problem mit {\sl FEEDBACK-NODE-SET} zu vergleichen.
{\sl FEEDBACK-NODE-SET} verlangt aber, dass jeder Zyklus durch mindestens einen
Knoten aus einer zu bestimmenden $k$-elementigen Mengen führt. 
Man kann versuchen, $k=|V|$ zu wählen, dann ist die gesuchte $k$-elementige
Menge $=V$.
Dann fragt aber {\sl FEEDBACK-NODE-SET}, ob jeder Zyklus einen Knoten in $V$
enthält. 
Dies ist natürlich immer der Fall!
{\sl FEEDBACK-NODE-SET} ist also nur dann ein interessantes Problem,
wenn $k\ll |V|$
ist, wenn man also mit viel weniger Knoten auskommen muss.
Das ist ganz offensichtlich ein ganz anderes Problem als {\sl HAMPATH}
\end{diskussion}

\begin{bewertung}
Reduktionsidee ({\bf R}) 1 Punkt,
Auswahl eines Vergleichsproblems ({\bf V}) 2 Punkte,
Mapping ({\bf M}) 2 Punkte,
NP-Vollständigkeit und Schlussfolgerung ({\bf S}) 1 Punkt.
\end{bewertung}
