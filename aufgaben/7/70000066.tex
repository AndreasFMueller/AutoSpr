Im Skigebiet Arosa Lenzerheide fand folgende Challenge statt.
Wer schafft es am genausten, mit den 34 zur Verfügung stehenden
Liften und Bahnen ingesamt 8888 Höhenmeter zu fahren?
Auf dem Weg zur Talstation entscheidet sich eine Informatik-Studentin
der OST zu versuchen, das Problem exakt zu lösen, also die Anlagen
so zu wählen, dass der zurückgelegte Höhenunterschied genau
$8888\,\text{m}$ wird.
Doch trotz intensiven Nachdenkens während des unvermeidlichen Anstehens und
der ersten Bahnfahrt weiss sie noch nicht einmal, ob es überhaupt möglich
ist, eine solche Lösung zu finden.
Warum ist dieses Problem so schwierig?

\themaL{NP-vollstandig}{NP-vollständig}
\thema{polynomielle Reduktion}

\begin{hinweis}
Nehmen Sie der Einfachheit halber an, dass man jede Bahn nach jeder
anderen Bahn fahren kann.
In der Realität kann man nicht jede Talstation von jeder Bergstation aus
erreichen, was das Problem verkompliziert.
Nehmen Sie also an, dass dies möglich ist.
\end{hinweis}

\begin{loesung}
Wir nennen das gestellte Problem das \textit{8888-CHALLENGE}-Problem.
Es ist gleichbedeutend mit dem \textit{SUBSET-SUM}-Problem, wie die folgende
Eins-zu-eins-Reduktion zeigt:
\begin{center}
\begin{tabular}{rcl}	
\textit{SUBSET-SUM}&&\textit{8888-CHALLENGE}\\
\hline
Zahl             & $\leftrightarrow$ &Höhenunterschied eines Liftes\\
Liste von Zahlen & $\leftrightarrow$ &Lifte (mehrmals)\\
Summe $t$        & $\leftrightarrow$ &$8888$, der zu erreichende Höhenunterschied
\end{tabular}
\end{center}
Die Lifte bzw.~ihr Höhenunterschied müssen mehrfach in der Liste der
Zahlen vertreten sein, da es ja auch möglich ist, einen Lift mehrmals
zu fahren.
Das \textit{SUBSET-SUM}-Problem erlaubt aber nur, jede Zahl der Liste
genau einmal zu verwenden.

Da \textit{SUBSET-SUM} NP-vollständig ist, gibt es nach heutigem Wissen
keinen polynomiell skalierenden Algorithmus zu seiner Lösung.
\end{loesung}

\begin{bewertung}
Reduktionsprinzip ({\bf R}) 1 Punkt,
Vergleichsproblem \textit{SUBSET-SUM} ({\bf V}) 1 Punkt,
Reduktionsabbildung für Zahlen ({\bf Z}) 1 Punkt,
Total ({\bf T}) 1 Punkt und
Auswahl mit Summenbedingung ({\bf S}) 1 Punkt.
NP-Vollständigkeit ({\bf N}) 1 Punkt.
\end{bewertung}
