Das untenstehende Zustandsdiagramm gehört zu einer Turingmaschine, die
die Sprache $\{\texttt{0}^n\texttt{1}^n\}$ entscheidet.
Gehören die nachstehenden Ausschnitte zu einer korrekten
Berechnungsgeschichte?
Wenn ja, können Sie den möglicherweise nicht sichtbaren
Zustand ermitteln?

\bgroup
\def\h{0.6}
\def\punkt#1#2{({(#1)*\h},{-(#2)*\h})}
\def\zeichen#1#2#3{\node at \punkt{#1}{#2} {$#3\mathstrut$};}
\newdimen\diagrammbreite
\diagrammbreite\dimexpr 6.2cm\relax
\newdimen\punktelaenge
\punktelaenge\dimexpr .5\textwidth-.5\diagrammbreite\relax
\hbox to\textwidth{%
\parbox[t]{0.9\punktelaenge}{
\begin{center}
\begin{tikzpicture}[>=latex,thick]
\fill[color=white] (0,0) rectangle ({0.9\punktelaenge},-3);
\node at (0,0) [below right] {a)};
\begin{scope}[xshift=1.3cm,yshift=-4mm]
\ifthenelse{\boolean{loesungen}}{
\fill[color=blue!10] \punkt{-0.5}{-0.5} rectangle \punkt{2.5}{1.5};
}{}
\draw \punkt{-0.5}{-0.5} rectangle \punkt{2.5}{1.5};
\zeichen{0}{0}{\texttt{0}}
\zeichen{1}{0}{q_1}
\zeichen{2}{0}{\texttt{1}}
\zeichen{0}{1}{\texttt{0}}
\zeichen{1}{1}{\texttt{x}}
\zeichen{2}{1}{q_2}
\end{scope}
\node at (0,-1.5) [below right] {c)};
\begin{scope}[xshift=1.3cm,yshift=-1.9cm]
\ifthenelse{\boolean{loesungen}}{
\fill[color=darkgreen!10] \punkt{-0.5}{-0.5} rectangle \punkt{2.5}{1.5};
}{}
\draw \punkt{-0.5}{-0.5} rectangle \punkt{2.5}{1.5};
\zeichen{0}{0}{q_3}
\zeichen{1}{0}{\texttt{x}}
\zeichen{2}{0}{\texttt{x}}
\zeichen{0}{1}{\blank}
\zeichen{1}{1}{\texttt{x}}
\zeichen{2}{1}{\texttt{x}}
\end{scope}
\end{tikzpicture}
\end{center}}
\hfill
\parbox[t]{0.9\punktelaenge}{
\begin{center}
\begin{tikzpicture}[>=latex,thick]
\fill[color=white] (0,0) rectangle ({0.9\punktelaenge},-3);
\node at (0,0) [below right] {b)};
\begin{scope}[xshift=1.3cm,yshift=-4mm]
\ifthenelse{\boolean{loesungen}}{
\fill[color=red!10] \punkt{-0.5}{-0.5} rectangle \punkt{2.5}{1.5};
}{}
\draw \punkt{-0.5}{-0.5} rectangle \punkt{2.5}{1.5};
\zeichen{0}{0}{\texttt{1}}
\zeichen{1}{0}{\texttt{1}}
\zeichen{2}{0}{\blank}
\zeichen{0}{1}{q_2}
\zeichen{1}{1}{\texttt{1}}
\zeichen{2}{1}{\blank}
\end{scope}
\node at (0,-1.5) [below right] {d)};
\begin{scope}[xshift=1.3cm,yshift=-1.9cm]
\draw \punkt{-0.5}{-0.5} rectangle \punkt{2.5}{1.5};
\zeichen{0}{0}{\texttt{x}}
\zeichen{1}{0}{\texttt{x}}
\zeichen{2}{0}{q_3}
\zeichen{0}{1}{\texttt{x}}
\zeichen{1}{1}{\texttt{x}}
\zeichen{2}{1}{\blank}
\end{scope}
\end{tikzpicture}
\end{center}}
\hfill
\parbox[t]{\diagrammbreite}{
\begin{center}
\begin{tikzpicture}[>=latex,thick]
\begin{scope}
\coordinate (q0) at (0,0);
\coordinate (q1) at (1.7,0);
\coordinate (q2) at (3.4,0);
\coordinate (q3) at (1.7,-1);
\coordinate (accept) at (0,1);
\coordinate (reject) at (3.4,1);
\node at (q0) {$q_0$};
\node at (q1) {$q_1$};
\node at (q2) {$q_2$};
\node at (q3) {$q_3$};
\node at (accept) {$q_{\text{accept}}$};
\node at (reject) {$q_{\text{reject}}$};
\draw (q0) circle[radius=0.25cm];
\draw (q1) circle[radius=0.25cm];
\draw (q2) circle[radius=0.25cm];
\draw (q3) circle[radius=0.25cm];
\draw (accept) ellipse (0.6 and 0.25);
\draw (accept) ellipse (0.65 and 0.3);
\draw (reject) ellipse (0.6 and 0.25);
\draw (reject) ellipse (0.65 and 0.3);
\draw[->,shorten >= 0.65cm,shorten <= 0.25cm] (q0) to[out=60,in=180] (reject);
\draw[->,shorten >= 0.65cm,shorten <= 0.25cm] (q1) to[out=90,in=180] (reject);
\draw[->,shorten >= 0.25cm,shorten <= 0.25cm] (q2) -- (reject);
\draw[->,shorten >= 0.25cm] (-1,0)--(q0);
\node at ($0.5*(q0)+0.5*(q1)+(0,-0.1)$)
	[above] {\tiny $\texttt{0}\to\texttt{x},R$};
\node at ($0.5*(q1)+0.5*(q2)+(0,-0.1)$)
	[above] {\tiny $\texttt{1}\to\texttt{x},R$};
\node at ($(q0)+(0,0.5)$) [left] {\tiny $\blank \to \blank,R$};
\node at ($(q3)+(0,0.5)$) {\tiny $.\to.,L$};
\node at ($(q2)+(-0.5,-0.7)$) [below right] {\tiny $\blank\to\blank,L$};
\node at ($(q0)+(0.5,-0.7)$) [below left] {\tiny $\blank\to\blank,R$};
\draw[->,shorten >= 0.3cm,shorten <= 0.25cm] (q0) -- (accept);
\draw[->,shorten >= 0.25cm,shorten <= 0.25cm] (q0) -- (q1);
\draw[->,shorten >= 0.25cm,shorten <= 0.25cm] (q1) -- (q2);
\draw[->,shorten >= 0.25cm,shorten <= 0.25cm] (q2) to[out=-90,in=0] (q3);
\draw[->,shorten >= 0.25cm,shorten <= 0.25cm] (q3) to[out=180,in=-90] (q0);
\draw[->,shorten >= 0.25cm,shorten <= 0.25cm] (q3) to[out=45,in=135,distance=0.6cm] (q3);
\draw[->,shorten >= 0.25cm,shorten <= 0.25cm]
	(q0) to[out=-100,in=-130,distance=0.8cm] (q0);
\node at ($(q0)+(-0.2,-0.4)$) [left] {\tiny$\texttt{x}\to\texttt{x},R$};
\draw[->,shorten >= 0.25cm,shorten <= 0.25cm]
	(q2) to[out=-50,in=-80,distance=0.8cm] (q2);
\node at ($(q2)+(0.2,-0.4)$) [right] {\tiny$\texttt{1}\to\texttt{1},R$};
\draw[->,shorten >= 0.25cm,shorten <= 0.25cm]
	(q1) to[out=100,in=140,distance=0.7cm] (q1);
\node at ($(q1)+(0.1,0.4)$) [above left] {\tiny$\texttt{0}\to\texttt{0},R$};
\draw[->,shorten >= 0.25cm,shorten <= 0.25cm]
	(q1) to[out=80,in=40,distance=0.7cm] (q1);
\node at ($(q1)+(0.1,0.4)$) [above right] {\tiny$\texttt{x}\to\texttt{x},R$};
\end{scope}
\end{tikzpicture}%
\end{center}
}}
\egroup


\begin{loesung}
Die Ausschnitte a), b) und c) können in der Berechnungsgesschichte
für das Wort $\texttt{0}^n\texttt{1}^n$ gefunden werden, sie werden
durch Rechtecke gleicher Farbe angezeigt:
\begin{center}
\def\h{0.5}
\def\punkt#1#2{({(#1)*\h},{-(#2)*\h})}
\def\o{\texttt{0}}
\def\x{\texttt{x}}
\def\l{\texttt{1}}
\def\b{\blank}
\def\zeichen#1#2#3{\node at \punkt{#1}{#2} {$#3\mathstrut$};}
\def\zeile#1#2#3#4#5#6#7#8{
	\node[color=gray] at \punkt{-0.1}{#1} [left] {\tiny #1:};
	\zeichen{1}{#1}{#2}
	\zeichen{2}{#1}{#3}
	\zeichen{3}{#1}{#4}
	\zeichen{4}{#1}{#5}
	\zeichen{5}{#1}{#6}
	\zeichen{6}{#1}{#7}
	\zeichen{7}{#1}{#8}
}
\begin{tikzpicture}[>=latex,thick]
\foreach \x in {0,...,6}{
	\node[color=gray] at \punkt{\x+1}{-1} {\tiny\x};
}
\fill[color=blue!20,opacity=0.5] \punkt{2.5}{1.5} rectangle \punkt{5.5}{3.5};
\fill[color=red!20,opacity=0.5] \punkt{4.5}{1.5} rectangle \punkt{7.5}{3.5};
\fill[color=darkgreen!20,opacity=0.5] \punkt{1.5}{17.5} rectangle \punkt{4.5}{19.5};
\zeile{ 0}{\b }{q_0}{\o }{\o }{\l }{\l }{\b }
\zeile{ 1}{\b }{\x }{q_1}{\o }{\l }{\l }{\b }
\zeile{ 2}{\b }{\x }{\o }{q_1}{\l }{\l }{\b }
\zeile{ 3}{\b }{\x }{\o }{\x }{q_2}{\l }{\b }
\zeile{ 4}{\b }{\x }{\o }{\x }{\l }{q_2}{\b }
\zeile{ 5}{\b }{\x }{\o }{\x }{q_3}{\l }{\b }
\zeile{ 6}{\b }{\x }{\o }{q_3}{\x }{\l }{\b }
\zeile{ 7}{\b }{\x }{q_3}{\o }{\x }{\l }{\b }
\zeile{ 8}{\b }{q_3}{\x }{\o }{\x }{\l }{\b }
\zeile{ 9}{q_3}{\b }{\x }{\o }{\x }{\l }{\b }
\zeile{10}{\b }{q_0}{\x }{\o }{\x }{\l }{\b }
\zeile{11}{\b }{\x }{q_0}{\o }{\x }{\l }{\b }
\zeile{12}{\b }{\x }{\x }{q_1}{\x }{\l }{\b }
\zeile{13}{\b }{\x }{\x }{\x }{q_1}{\l }{\b }
\zeile{14}{\b }{\x }{\x }{\x }{\x }{q_2}{\b }
\zeile{15}{\b }{\x }{\x }{\x }{q_3}{\x }{\b }
\zeile{16}{\b }{\x }{\x }{q_3}{\x }{\x }{\b }
\zeile{17}{\b }{\x }{q_3}{\x }{\x }{\x }{\b }
\zeile{18}{\b }{q_3}{\x }{\x }{\x }{\x }{\b }
\zeile{19}{q_3}{\b }{\x }{\x }{\x }{\x }{\b }
\zeile{20}{\b }{q_0}{\x }{\x }{\x }{\x }{\b }
\zeile{21}{\b }{\x }{q_0}{\x }{\x }{\x }{\b }
\zeile{22}{\b }{\x }{\x }{q_0}{\x }{\x }{\b }
\zeile{23}{\b }{\x }{\x }{\x }{q_0}{\x }{\b }
\zeile{24}{\b }{\x }{\x }{\x }{\x }{q_0}{\b }
\zeile{25}{\b }{\x }{\x }{\x }{\x }{\b }{q_{\text{a}\rlap{$\scriptstyle\text{ccept}$}}}
\end{tikzpicture}
\end{center}
\begin{teilaufgaben}
\item
Beide Zustände sind sichtbar.
\item
Der Zustand in der ersten Zeile ist gemäss der Berechnungsgeschichte
$q_1$, dies ist aber nicht die einzige Möglichkeit.
Wenn das Wort mehr Einsen hat, könnte sich die Maschine auch bereits
Auf der ersten Zeile im Zustand $q_2$ befinden, in dem alle Einsen
und \texttt{x} übersprungen werden.
\item
Der Kopf bewegt sich nach links, um den Wortanfang zu finden.
Gemäss Zustandsdiagramm ändert sich ausgehend von $q_3$ der Zustand
nur dann, wenn das aktuelle Zeichen ein Leerzeichen ist, was hier
nicht der Fall ist.
Der zur zweiten Zeile gehörige Zustand muss daher $q_3$ sein.
\item
Der Ausschnitt zeigt eine Kopfbwegung nach rechts, daher muss
rechts von $q_3$ ein Leerzeichen stehen.
Der neue, nicht sichtbare Zustand ist $q_0$.
Der Übergang ist korrekt, kommt aber in keiner Berechnungsgeschichte
vor.
\qedhere
\end{teilaufgaben}
\end{loesung}
