Lösen Sie das \textit{SUBSET-SUM} Problem mit den Zahlen
\begin{align*}
y_1&=100100,&
y_2&=10110,&
y_3&=1100,&
g_1&=100,&
g_2&=10,&
g_3&=1,
\\
z_1&=100011,&
z_2&=10001,&
z_3&=1011,&
h_1&=100,&
h_2&=10,&
h_3&=1
\end{align*}
und der Summe $t=111333$ und leiten Sie daraus eine Lösung für 
das \textit{3SAT}-Problem
\[
\varphi
=
(x_1\vee x_2\vee x_3)
\wedge
(\overline{x}_1\vee x_2\vee \overline{x}_3)
\wedge
(\overline{x}_1\vee \overline{x}_2\vee \overline{x}_3)
\]
ab.

\themaL{NP-vollstandig}{NP-vollständig}
\thema{polynomielle Reduktion}

\begin{loesung}
Eine mögliche Lösung ist
\begin{align*}
y_1+z_2+z_3
&=
100100+10001+1011
\\
&=111112
\\
g_1+h_1+g_2+h_2+g_3
&= \phantom{000}221
\\
y_1+z_2+z_3
+
g_1+h_1+g_2+h_2+g_3
&=
111333=t
\end{align*}
Nach der Konstruktion der Zahlen $y_i$, $z_i$, $g_i$ und $h_i$ bedeutet
diese Auswahl, dass $x_1=\texttt{t}$, $x_2=\texttt{f}$ und
$x_3=\texttt{f}$ gesetzt werden muss.
Tatsächlich ist
\begin{align*}
\varphi
&=
(x_1\vee x_2\vee x_3)
\wedge
(\overline{x}_1\vee x_2\vee \overline{x}_3)
\wedge
(\overline{x}_1\vee \overline{x}_2\vee \overline{x}_3)
\\
&=
(
\texttt{t}
\vee
\texttt{f}
\vee
\texttt{f}
)
\wedge
(
\overline{\texttt{t}}
\vee
\texttt{f}
\vee
\overline{\texttt{f}}
)
\wedge
(
\overline{\texttt{t}}
\vee
\overline{\texttt{f}}
\vee
\overline{\texttt{f}}
)
\\
&=
(
\underbrace{
\texttt{t}
\vee
\texttt{f}
\vee
\texttt{f}
}_{\displaystyle \texttt{t}}
)
\wedge
(
\underbrace{
\texttt{f}
\vee
\texttt{f}
\vee
\texttt{t}
}_{\displaystyle \texttt{t}}
)
\wedge
(
\underbrace{
\texttt{f}
\vee
\texttt{t}
\vee
\texttt{t}
}_{\displaystyle \texttt{t}}
)
=
\texttt{t}.
\end{align*}
Diese Belegung der Variablen $x_i$ mit Wahrheitswerten ist also
tatsächlich eine Lösung des \textit{3SAT}-Problems.
\end{loesung}
