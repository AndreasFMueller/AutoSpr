Ein Orchester soll Konzertprogramme für eine Saison zusammenstellen.
Zur Verfügung steht eine Menge von Werken, die das Orchester im Repertoire hat.
Aus diesen Werken sollen Programme mit jeweils drei Werken zusammengesetzt
werden.
Je ein Werk zur Eröffnung, für den Mittelteil und zum Abschluss des
Konzertes.
Die Werke lassen sich nicht beliebig kombinieren, weil zum Beispiel die
Tonarten nicht zu verschieden sein dürfen.
Die Anzahl der Werke und der Konzertabende der Saison ist gleich.
Es stellt sich heraus, dass es ziemlich schwierig ist, Programme so
zu konstruieren, dass kein Werk in der gleichen Phase eines Konzerts
mehr als einmal gespielt wird.
Warum?

\themaL{NP-vollstandig}{NP-vollständig}
\thema{polynomielle Reduktion}

\begin{loesung}
Dies ist eine Instanz des 3D-Matching-Problems.
Die Menge der Werke ist $T$, die Menge der möglichen Programme (Einschränkungen
zum Beispiel durch Tonart) ist $U\subset T\times T\times T$.
Gesucht werden solle eine Menge von Programmen $W\subset U$, mit
$|W|=|T|$, so dass keine zwei Elemente in irgend einer Koordinate
übereinstimmen.
Da 3D-Matching NP-vollständig ist, ist auch das Konzertproblem NP-vollständig,
nach aktuellem Wissen gibt es dafür daher keinen polynomiellen Algorithmus.
\end{loesung}

\begin{bewertung}
Reduktionsansatz ({\bf R}) 1 Punkt,
Wahl des Vergleichsproblems ({\bf V}) 1 Punkt,
Mapping für Werke ({\bf W}) 1 Punkt,
Mapping für Programme ({\bf U}) 1 Punkt,
Mapping für Bedingung ({\bf P}) 1 Punkt,
NP-Vollständigkeit ({\bf N}) 1 Punkt.
\end{bewertung}

