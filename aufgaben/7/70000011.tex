Für eine Gruppenarbeit sollen $k$ Gruppen gebildet werden.
Um die Zeit für das gegenseitige Kennenlernen möglichst
kurz zu halten, sollen sich die Leute einer Gruppe bereits
gegenseitig kennen. Alle Leute sollen beschäftigt sein.
Können Sie einen effizienten Algorithmus
formulieren, mit dem eine solche Gruppeneinteilung auch bei
einer grossen Teilnehmerzahl gefunden werden kann?

\themaL{NP-vollstandig}{NP-vollständig}
\thema{polynomielle Reduktion}

\begin{loesung}
Das Problem ist äquivalent zu \textsl{CLIQUE-COVER}, also
NP-vollständig. Nach aktuellem Wissen gibt es also keine
effizienten Algorithmus, der das Problem lösen würde. Die
"Aquivalenz wird durch folgende Abbildung vermittelt
\begin{align*}
\text{Teilnehmer}&\mapsto\text{Knoten}\\
\text{kenne sich}&\mapsto\text{Kante}\\
\text{Anzahl Gruppen}&\mapsto k\\
\text{Gruppe}&\mapsto\text{Clique}
\qedhere
\end{align*}
\end{loesung}
