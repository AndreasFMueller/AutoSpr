Konstruieren Sie einen polynomiellen Verifizierer für das $n$-Damen-Problem
von Aufgabe~\ref{70000067}.

\begin{loesung}
Wir wissen bereits, dass das Problem entscheidbar ist, allerdings
mit einem Algorithmus von exponentieller Komplexität.

Um einen Verifizierer zu konstruieren, muss zuerst festgelegt
werden, was als Zertifikat verwendet werden soll.
Das Wort
\[
(n,c_1,c_2,\dots,c_n) \in \textit{QUEENS}
\]
ist ein geeignetes Zertifikat.

Der Verifikationsalgorithmus muss folgende Dinge überprüfen:
\begin{center}
\begin{tabular}{r|p{10cm}|>{$}r<{$}}
Schritt&Was&\text{Laufzeit}\\
\hline
1&Enthält das Wort genau $n+1$ natürliche Zahlen zwischen $1$ und $n$?&O(n\log n)\\
2&Sind die Zahlen $c_i$ alle verschieden?&O(n^2\log(n)^2)\\
3&Sind keine Damen in der gleichen Diagonalen platziert?&O(n^2\log(n)^2)\\
\hline
&Total&O(n^2\log(n)^2)\\
\end{tabular}
\end{center}
Die Laufzeit ist schneller als $O(n^3)$, also polynomiell.
Damit ist der oben beschriebene Algorithmus ein polynomieller
Verifizierer.
\end{loesung}
