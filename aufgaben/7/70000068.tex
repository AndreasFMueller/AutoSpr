Konstruieren Sie einen polynomiellen Verifizierer für das $n$-Damenproblem
von Aufgabe~\ref{70000067}.
Es geht also darum, zu entscheiden, ob es zu einer gegebenen Zahl $n$
eine Lösung gibt.
\begin{teilaufgaben}
\item
Was verlangen Sie als Zertifikat?
\item
Wie gehen Sie bei der Verifikation vor?
\end{teilaufgaben}

\begin{loesung}
Wir wissen bereits, dass das Problem entscheidbar ist, allerdings
mit einem Algorithmus von exponentieller Komplexität.

Um einen Verifizierer zu konstruieren, muss zuerst festgelegt
werden, was als Zertifikat verwendet werden soll.
Das Wort
\[
(n,r_1,r_2,\dots,r_n) \in \textit{QUEENS},
\]
wobei $r_i$ die Zeilennummer der Dame in Spalte $i$ ist,
ist ein geeignetes Zertifikat.

Der Verifikationsalgorithmus muss folgende Dinge überprüfen:
\begin{center}
\begin{tabular}{r|p{10cm}|>{$}r<{$}}
Schritt&Was&\text{Laufzeit}\\
\hline
1&Enthält das Wort genau $n+1$ natürliche Zahlen zwischen $1$ und $n$?&O(n\log n)\\
2&Sind die Zahlen $r_i$ alle verschieden?&O(n^2\log(n)^2)\\
3&Sind keine Damen in der gleichen Diagonalen platziert?&O(n^2\log(n)^2)\\
\hline
&Total&O(n^2\log(n)^2)\\
\end{tabular}
\end{center}
Die Faktoren $\log(n)$ kommen daher, dass mit $n$ auch die Länge
der Zahlen wie $\log(n)$ anwächst.
Entsprechend nimmt auch die Zeit zu, jede dieser Zahlen zu verarbeiten.

Die gesamte Laufzeit ist schneller als $O(n^3)$, also polynomiell.
Damit ist der oben beschriebene Algorithmus ein polynomieller
Verifizierer.
\end{loesung}
