Ein aufstrebendes Film-Festival ist derart gewachsen, dass der 
Vorführsaal nicht mehr reicht. Daher müssen jetzt zwei gleich
grosse Säle verwendet werden, und trotzdem ist das Festival
wieder ausverkauft, und zwar in einem Masse, dass überhaupt nur Stars und Prominente
samt ihrer Entourage eingelassen werden können, für
einzelne Besucher gibt es keine Plätze.

Doch die Stars stören sich daran, dass sie möglicherweise nicht
ihre ganze Entourage im gleichen Saal haben können. Daher muss
kurzfristig eine Aufteilung der Festival-Gäste gefunden werden,
so dass die beiden Säle so gefüllt werden können, dass
jede Entourage vollständig in einem der Säle Platz nimmt.

Der Festival-Direktor ist jedoch sehr überrascht, dass die
Bestimmung einer solchen Aufteilung so lange dauert. Warum
sind Sie nicht überrascht?

\thema{polynomielle Reduktion}
\thema{NP-vollständig}

\begin{loesung}
Zu jedem Star $i\in I$ gibt es eine Entourage mit $c_i$
Mitgliedern. Diese Menge muss jetzt in zwei Teilmengen
$A$ (Stars samt Entourage, die in Saal A Platz nehmen) und
$B$ (Stars samt Entourage, die in Saal B Platz nehmen) aufgeteilt
werden, so dass $I=A\cup B$. Die Aufteilung muss so sein, dass
in beiden Sälen gleich viele Leute Platz nehmen, also
\[
\sum_{i\in A}c_i=
\sum_{i\in B}c_i.
\]
Dies ist das Problem {\it PARTITION}. Das gestellte Problem ist also
äquivalent zum NP-vollständigen Problem {\it PARTITION}, und ist
daher ebenfalls NP-vollständig. Man kann daher nach aktuellem
Wissen nicht erwarten, dass es dafür einen effizienten Algorithmus
gibt.
\end{loesung}

\begin{diskussion}
Man kann auch versuchen, das Problem Festival-Problem mit
{\it SUBSET-SUM} vergleichen.
Zu jedem Star $i\in I$ gehört die Zahl $s_i$ der Mitglieder in der
Entourage des Stars. Sei $S=\{s_i\,|\,i\in I\}$ die Menge all dieser
Mitgliederzahlen. Sei $t$ die Platzzahl eines der Vorführsäle.
Man muss jetzt eine Teilmenge von $I'\subset I$ auswählen, so dass
\[
\sum_{i\in I'} s_i=t.
\]
Dies sieht auf den ersten Blick aus wie des {\it SUBSET-SUM}-Problem,
beim genaueren Hinschauen erkennt man jedoch den Unterschied: In der
Liste der $s_i$ können einzelne Zahlen mehrfach vorkommen. Es könnte
sogar sein, dass die Menge $S$ überhaupt nur eine Zahl enthält,
zum Beispiel wenn dass Filmfestival allen Stars die gleiche Zahl von
Freikarten für die Entourage abgegeben hat. Dann ist das erzeugte
{\it SUBSET-SUM}-Problem gar nicht mehr lösbar, während das ursprüngliche
Problem genau dann lösbar ist, wenn $t$ durch die immer gleiche Grösse
der Entourage teilbar ist. 

Eine allgemeinere Formulierung des Rucksack-Problems hingegen wäre
durchaus ein mögliches Vergleichsproblem, denn oben haben wir ja
eine Reduktion des Festival-Problems auf das Problem konstruiert,
für eine Familie (nicht Menge!) $(s_i)_{i\in I}$ von Zahlen eine
Teilfamilie (nicht Teilmenge!) $(s_i)_{i\in I'}$ zu finden, so dass
die Summe der Zahlen einen bestimmten Wert $t=\sum_{i\in I'}s_i$
erreicht.
\end{diskussion}

\begin{bewertung}
Lösungsansatz mit Reduktion ({\bf R}) 1 Punkt,
Auswahl eines geeigneten Vergleichsproblems ({\bf V}) 1 Punkt.
Reduktionsabbildung für Stars ({\bf S}) 1 Punkt,
für die Entourage ({\bf E}) 1 Punkt,
für die Zuordnung zu den beiden Sälen ({\bf A}) 1 Punkt,
für die Aufteilung in zwei gleich grosse Räume ({\bf P}) 1 Punkt,
maximal drei von den letztgenannten vier möglichen Punkten.
Schlussfolgerung, dass NP-vollständige Probleme nach aktuellem
Wissen nicht effizient gelöst werden können ({\bf N}) 1 Punkt.
\end{bewertung}
