Die COVID-19 Pandemie verlangt von Läden Sicherheitsmassnahmen, die
sicherstellen sollen, dass die Kunden in sicheren Abständen verbleiben.
Ein Einkaufszentrum entscheidet daher, dass ein Einbahnbetrieb eingeführt werden soll.
Die Marketing-Abteilung legt fest, in welcher Richtung die einzelnen
Gänge zwischen den Gestellen durchlaufen werden können sollen.
Es soll nämlich sichergestellt werden, dass die Kunden immer noch 
in der ``richtigen'' Reihenfolge mit überflüssiger Werbung zu
genauso überflüssigen Spontankäufen verleitet werden sollen.
Schliesslich will der Sicherheitsverantwortliche wissen, ob es möglich
ist, an einigen
Kreuzungsstellen Desinfektionsstationen aufzustellen so,
dass jeder Kunde, der sich auf einem geschlossenen Weg durch das
Einkaufszentrum bewegt, mindestens einmal an einer Desinfektionsstation
vorbeikommt.
Dabei dürfen nicht mehr Desinfektionsstationen verwendet werden, als
das vorgegebene Budget erlaubt.
Dies stürtzt die Planer in eine Krise, auch nach stundenlangem
Probieren können Sie keine definitive Antwort geben.
Warum?

\begin{loesung}
Das beschriebene Problem ist das Problem \textsl{FEEDBACK-NODE-SET},
wie die folgenden Eins-zu-eins-Reduktion zeigt:
\begin{align*}
\text{Graph} &\leftrightarrow \text{Plan des Einkaufszentrums}
\\
\text{Knoten}&\leftrightarrow \text{Kreuzungsstellen}
\\
\text{Kanten}&\leftrightarrow \text{Gänge}
\\
\text{Richtung}&\leftrightarrow \text{Einbahnrichtung in jedem Gang}
\\
\text{Anzahl $k$}&\leftrightarrow \text{Budget für Desinfektionsstationen}
\\
\text{Node set}&\leftrightarrow \text{Platzierung der Desinfektionsstationen}
\end{align*}
Das gestellte Problem ist also NP-vollständig, nach aktuellem Wissen gibt
es daher keinen polynomiellen Algorithmus, der die Frage entscheiden könnte.
\end{loesung}

\begin{bewertung}
NP-Vollständigkeit ({\bf N}) 1 Punkt,
Reduktionsprinzip ({\bf R}) 1 Punkt,
Vergleichsproblem wählen ({\bf V}) 1 Punkt,
Mapping Knoten, Kanten, Stationen ({\bf M}) 3 Punkte.
\end{bewertung}
