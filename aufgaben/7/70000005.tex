Job-Planungs-Problem:
Gegeben sei eine Menge von $n$ Jobs $J_1,\dots,J_n$, welche zu ihrer
Ausführung einige der $m$ Resourcen $r_1,\dots,r_m$ für sich reservieren
müssen. Jeder Task hat die gleiche Laufzeit $1$, jeweils zu ganzzahligen
Zeitpunkten können Jobs gestartet werden. Zeigen Sie, dass die
Aufgabe, die Jobs ohne Resourcenkonflikte so zu planen, dass innerhalb
gegebener Zeit $N$ alle Jobs abgearbeitet werden, ein NP-vollständiges Problem
ist.

\themaL{NP-vollstandig}{NP-vollständig}
\thema{polynomielle Reduktion}

\begin{loesung}
Wir zeigen, dass das Problem äquivalent ist zum Einfärbe-Problem für
einen Graphen. Wir müssen für jeden Job ein Zeitinterval $I_k=[k,k+1]$
auswählen, während dem er laufen soll. Dabei dürfen zwei Jobs nicht
gleichzeitig laufen, wenn Sie auf die gleiche Resource angewiesen sind.
Wir konstruieren daher einen Graphen $G$ mit den Jobs als Ecken. Falls
zwei Jobs die gleiche Resource beanspruchen, fügen wir eine Kante
zwischen diesen beiden Jobs hinzu. Gesucht ist jetzt eine Zuteilung
von Intervallen $I_k$ mit $0\le k< N$ zu jeder Ecke, so dass keine
zwei benachbarten Ecken dem gleichen Interval zugeteilt sind. Bezeichnen
wir die Intervalle $I_k$ als ``Farben'', haben wir aus dem Job-Planungs-Problem
ein Färbeproblem hergestellt.

Ist umgekehrt ein Graph und eine Menge von Farben vorgegeben, können
wir daraus wie folgt ein Planungs-Problem konstruieren. Die Ecken
des Graphen nennen wir Jobs, die Kanten nennen wir Resourcen. Die
mit einer Ecke inzidenten Kanten stellen die Resourcen dar, die ein
Job braucht, um laufen zu können. Der Graph kann mit $N$ Farben
eingefärbt werden, wenn das Planungsproblem in Zeit $N$ lösbar ist.

Die beiden Probleme sind somit äquivalent. Da das Färbeproblem
bekanntermassen NP-vollständig ist, ist auch das Job-Planungs-Problem
NP-vollständig.
\end{loesung}
