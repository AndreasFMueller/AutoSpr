Beim Spiel Flood-It sind die Felder eines $n\times m$-Spielfeld mit 
mindestens drei verschiedenen Farben eingefärbt.
Man sagt, zwei Felder seine verbunden, wenn sie die gleiche Farbe
haben und über eine Kante benachbart sind.
Ein Gebiet besteht aus allen untereinander verbundenen Feldern.
Der Spieler kann jetzt jeweils die Farbe des Feldes in der linken
oberen Ecke ändern, dabei wird auch die Farbe des ganzen zugehörigen
Gebietes geändert.
Da an das Gebiet Felder der neuen Farbe anstossen können, kann das
Gebiet durch den Farbwechsel grösser werden.
Ziel ist, das Gebiet der linken oberen Ecke in möglichst wenigen
Farbwechseln so anwachsen zu lassen, dass es das ganze Spielfeld überdeckt.

Im folgenden Bild sind vier Schritte des Spiels dargestellt, das Gebiet
des Feldes in der linken oberen Ecke ist jeweils fett ausgezogen.
\begin{center}
\definecolor{darkred}{rgb}{0.8,0,0}
\def\h{0.4}
\def\farbfeld#1#2#3{
	\fill[color=#3!40] ({(#1-1)*\h},{(#2-1)*\h}) rectangle ++(\h,\h);
}
\def\r#1#2{\farbfeld{#1}{#2}{darkred}}
\def\g#1#2{\farbfeld{#1}{#2}{darkgreen}}
\def\b#1#2{\farbfeld{#1}{#2}{blue}}
\begin{tikzpicture}[>=latex,thick]
\begin{scope}
\r{1}{6} \g{2}{6} \r{3}{6} \b{4}{6} \r{5}{6} \g{6}{6}
\b{1}{5} \r{2}{5} \b{3}{5} \g{4}{5} \b{5}{5} \g{6}{5}
\b{1}{4} \b{2}{4} \g{3}{4} \b{4}{4} \b{5}{4} \b{6}{4}
\b{1}{3} \g{2}{3} \r{3}{3} \b{4}{3} \r{5}{3} \g{6}{3}
\b{1}{2} \g{2}{2} \r{3}{2} \g{4}{2} \g{5}{2} \g{6}{2}
\r{1}{1} \b{2}{1} \r{3}{1} \g{4}{1} \b{5}{1} \r{6}{1}
\draw (0,0) rectangle ++({6*\h},{6*\h});
\draw[line width=1pt] (0,{5*\h}) rectangle ++(\h,\h);
\draw[->] ({6.2*\h},{3*\h}) -- ++({1.3*\h},0);
\end{scope}
\begin{scope}[xshift=3.1cm]
\b{1}{6} \g{2}{6} \r{3}{6} \b{4}{6} \r{5}{6} \g{6}{6}
\b{1}{5} \r{2}{5} \b{3}{5} \g{4}{5} \b{5}{5} \g{6}{5}
\b{1}{4} \b{2}{4} \g{3}{4} \b{4}{4} \b{5}{4} \b{6}{4}
\b{1}{3} \g{2}{3} \r{3}{3} \b{4}{3} \r{5}{3} \g{6}{3}
\b{1}{2} \g{2}{2} \r{3}{2} \g{4}{2} \g{5}{2} \g{6}{2}
\r{1}{1} \b{2}{1} \r{3}{1} \g{4}{1} \b{5}{1} \r{6}{1}
\draw (0,0) rectangle ++({6*\h},{6*\h});
\draw[line width=1pt] (0,{6*\h}) -- ++(\h,0) -- ++(0,{-2*\h})
	-- ++(\h,0) -- ++(0,-\h) -- ++(-\h,0) -- ++(0,{-2*\h})
	-- ++(-\h,0) -- cycle;
\draw[->] ({6.2*\h},{3*\h}) -- ++({1.3*\h},0);
\end{scope}
\begin{scope}[xshift=6.2cm]
\g{1}{6} \g{2}{6} \r{3}{6} \b{4}{6} \r{5}{6} \g{6}{6}
\g{1}{5} \r{2}{5} \b{3}{5} \g{4}{5} \b{5}{5} \g{6}{5}
\g{1}{4} \g{2}{4} \g{3}{4} \b{4}{4} \b{5}{4} \b{6}{4}
\g{1}{3} \g{2}{3} \r{3}{3} \b{4}{3} \r{5}{3} \g{6}{3}
\g{1}{2} \g{2}{2} \r{3}{2} \g{4}{2} \g{5}{2} \g{6}{2}
\r{1}{1} \b{2}{1} \r{3}{1} \g{4}{1} \b{5}{1} \r{6}{1}
\draw (0,0) rectangle ++({6*\h},{6*\h});
\draw[line width=1pt] (0,{6*\h}) -- ++({2*\h},0) -- ++(0,{-1*\h})
	-- ++(-\h,0) -- ++(0,-\h) -- ++({2*\h},0) -- ++(0,{-1*\h})
	-- ++(-\h,0) -- ++(0,{-2*\h}) -- ++({-2*\h},0) -- cycle;
\draw[->] ({6.2*\h},{3*\h}) -- ++({1.3*\h},0);
\end{scope}
\begin{scope}[xshift=9.3cm]
\r{1}{6} \r{2}{6} \r{3}{6} \b{4}{6} \r{5}{6} \g{6}{6}
\r{1}{5} \r{2}{5} \b{3}{5} \g{4}{5} \b{5}{5} \g{6}{5}
\r{1}{4} \r{2}{4} \r{3}{4} \b{4}{4} \b{5}{4} \b{6}{4}
\r{1}{3} \r{2}{3} \r{3}{3} \b{4}{3} \r{5}{3} \g{6}{3}
\r{1}{2} \r{2}{2} \r{3}{2} \g{4}{2} \g{5}{2} \g{6}{2}
\r{1}{1} \b{2}{1} \r{3}{1} \g{4}{1} \b{5}{1} \r{6}{1}
\draw (0,0) rectangle ++({6*\h},{6*\h});
\draw[line width=1pt] (0,{6*\h}) -- ++({3*\h},0) -- ++(0,{-1*\h})
	-- ++(-\h,0) -- ++(0,-\h) -- ++({1*\h},0) -- ++(0,{-4*\h})
	-- ++(-\h,0) -- ++(0,{1*\h}) -- ++({-1*\h},0) -- ++(0,-\h)
	-- ++(-\h,0) -- cycle;
\draw[->] ({6.2*\h},{3*\h}) -- ++({1.3*\h},0);
\end{scope}
\begin{scope}[xshift=12.4cm]
\b{1}{6} \b{2}{6} \b{3}{6} \b{4}{6} \r{5}{6} \g{6}{6}
\b{1}{5} \b{2}{5} \b{3}{5} \g{4}{5} \b{5}{5} \g{6}{5}
\b{1}{4} \b{2}{4} \b{3}{4} \b{4}{4} \b{5}{4} \b{6}{4}
\b{1}{3} \b{2}{3} \b{3}{3} \b{4}{3} \r{5}{3} \g{6}{3}
\b{1}{2} \b{2}{2} \b{3}{2} \g{4}{2} \g{5}{2} \g{6}{2}
\b{1}{1} \b{2}{1} \b{3}{1} \g{4}{1} \b{5}{1} \r{6}{1}
\draw (0,0) rectangle ++({6*\h},{6*\h});
\draw[line width=1pt] (0,{6*\h}) -- ++({4*\h},0) -- ++(0,{-1*\h})
	-- ++(-\h,0) -- ++(0,-\h) -- ++(\h,0) -- ++(0,\h)
	-- ++(\h,0) -- ++(0,-\h) -- ++(\h,0) -- ++(0,{-1*\h})
	-- ++({-2*\h},0) -- ++(0,-\h) -- ++(-\h,0) -- ++(0,{-2*\h})
	-- ++({-3*\h},0) -- cycle;
\end{scope}
\end{tikzpicture}
\end{center}
Kann eine nichtdeterministische Turing-Maschine in polynomieller Zeit 
herausfinden, ob zu einer vorgegeben Zahl $k$ ein Flood-It
Rätsel in höchstens $k$ Schritten gelöst werden kann?

\thema{polynomieller Verifizierer}
\thema{NP}

\begin{loesung}
Es gibt $c^k$ Abfolgen von Farbwechseln, wenn $c$ die Anzahl der Farben ist.
Indem man alle diese Abfolgen durchprobiert, kann man entscheiden, ob
das Problem eine Lösung hat.
Das Problem ist also entscheidbar.

Um nachzuweisen, dass das Problem von einer nichtdeterministischen
Turing-Maschine in polynomieller Zeit entschieden werden kann,
suchen wir jetzt einen polynomiellen Verifizierer.

Als Lösungszertifikat für den Verifizierer verlangen wir die Abfolge
von Farben, die für die Lösung des Flood-It-Rätsels nötig ist.
Dann simulieren wir das Spiel, wobei wir in jedem Schritt
die $nm$ Felder überprüfen müssen und feststellen müssen, welche
Felder jetzt das Gebiet ausmachen, also welche Felder mit der neuen
Farbe eingefärbt werden müssen.
Dazu sind maximal $nm$ Prüfungen nötig.
Nach maximal $k$ Schritten müssen wir überprüfen, ob das ganze Feld
mit der gleichen Farbe eingefärbt ist.
Dazu sind nochmals $nm$ Prüfungen nötig.
Der Rechenaufwand für Verifikation ist also $O((k+1)nm)$.
Wir wissen ausserdem, dass $k\le nm-1$, also ist der Aufwand für die
Verifikation $O((nm)^2)$ und damit sicher polynomiell.
Somit ist nachgewiesen, dass das Problem in polynomieller Zeit von
einer nichtdeterministischen Turing-Maschine entscheidbar ist.
\end{loesung}

\begin{diskussion}
Clifford, Jalsenius, Montanaro und Sach haben in
\url{https://arxiv.org/pdf/1001.4420v3.pdf} bewiesen, dass die oben
beschriebene Version von Flood-It NP-vollständig ist.
Sie haben auch weitere Version diskutiert, und approximative Lösungen
mit polynomieller Laufzeit angegeben.
\end{diskussion}

\begin{bewertung}
Verifizierer ({\bf V}) 1 Punkt,
Zertifikat ({\bf Z}) 1 Punkt,
Verifikationsalgorithmus ({\bf A}) 2 Punkte,
Nachweis, dass Aufwand polynomiell ist ({\bf P}) 2 Punkte.
\end{bewertung}

