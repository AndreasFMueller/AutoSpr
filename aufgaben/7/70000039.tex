Beim Spiel Flood-It sind die Felder eines $n\times m$-Spielfeld mit 
mindestens drei verschiedenen Farben eingef"arbt.
Man sagt, zwei Felder seine verbunden, wenn sie die gleiche Farbe
haben und "uber eine Kante benachbart sind.
Ein Gebiet ist besteht aus allen untereinander verbundenen Feldern.
Der Spieler kann jetzt jeweils die Farbe des Feldes in der linken
oberen Ecke "andern, dabei wird auch die Farbe des ganzen Gebietes 
ge"andert.
Da an das Gebiet Felder der neuen Farbe anstossen k"onnen, kann das
Gebiet durch den Farbwechsel gr"osser werden.
Ziel ist, das Gebiet der linken oberen Ecke in m"oglichst wenigen
Farbwechseln so anwachsen zu lassen, dass es das ganze Spielfeld "uberdeckt.

\begin{center}
\includeagraphics[width=\hsize]{floodit-1.pdf}
\end{center}


Kann eine nichtdeterministische Turing-Maschine in polynomieller Zeit 
herausfinden, ob zu einer vorgegeben Zahl $k$ eine L"osung eins Flood-It
R"atsels in h"ochstens $k$ Schritten m"oglich ist?

\begin{loesung}
Es gibt $c^k$ Abfolgen von Farbwechseln, wenn $c$ die Anzahl der Farben ist.
Indem man alle diese Abfolgen durchprobiert, kann man entscheiden, ob
das Problem eine L"osung hat.
Das Problem ist also entscheidbar.

Um nachzuweisen, dass das Problem von einer nichtdeterministischen
Turing-Maschine in polynomieller Zeit entschieden werden kann,
suchen wir jetzt einen polynomiellen Verifizierer.

Als L"osungszertifikat f"ur den Verifizierer verlangen wir die Abfolge
von Farben, die f"ur die L"osung des Flood-It-R"atsels n"otig ist.
Dann simulieren wir das Spiel, wobei wir in jedem Schritt
die $nm$ Felder "uberpr"ufen m"ussen und feststellen m"ussen, welche
Felder jetzt das Gebiet ausmachen, also welche Felder mit der neuen
Farbe eingef"arbt werden m"ussen.
Dazu sind maximal $nm$ Pr"ufungen n"otig.
Nach maximal $k$ Schritten m"ussen wir "uberpr"ufen, ob das ganze Feld
mit der gleichen Farbe eingef"arbt ist.
Dazu sind nochmals $nm$ Pr"ufungen n"otig.
Der Rechenaufwand f"ur Verifikation ist also $O((k+1)nm)$.
Wir wissen ausserdem, dass $k\le nm-1$, also ist der Aufwand f"ur die
Verifikation $O((nm)^2)$ und damit sicher polynomiell.
Somit ist nachgewiesen, dass das Problem in polynomieller Zeit von
einer nichtdeterministischen Turing-Maschine entscheidbar ist.
\end{loesung}

\begin{diskussion}
Clifford, Jalsenius, Montanaro und Sach haben in
\url{https://arxiv.org/pdf/1001.4420v3.pdf} bewiesen, dass die oben
beschriebene Version von Flood-It NP-vollst"andig ist.
Sie haben auch weitere Version diskutiert, und approximative L"osungen
mit polynomieller Laufzeit angegeben.
\end{diskussion}

\begin{bewertung}
\end{bewertung}

