Der neue CEO einer grossen Fluggesellschaft möchte das Personal in allen
Flughäfen persönlich besuchen, die die Fluggesellschaft anfliegt.
Er bittet seine Sekretärin, einen optimalen Besuchsplan zusammenzustellen,
bei der er jede Destination nur genau einmal besuchen muss.
Als der Plan eine Woche später immer noch nicht fertig ist, wird er
ungeduldig.
Warum dauert es so lange, einen Besuchsplan zusammenzustellen?

\themaL{NP-vollstandig}{NP-vollständig}
\thema{polynomielle Reduktion}

\begin{loesung}
Die Flüge, die die Destinationen der Fluggesellschaft miteinander
verbinden, bilden einen gerichteten Graphen.
Der CEO verlangt die Lösung des Problems HAMPATH, welches einen
hamiltonschen Pfad sucht, also einen Pfad, der jede Ecke des Graphen
genau einmal besucht.
Dieses Problem ist NP-vollständig, es gibt also keine effizienten
Algorithmen für grosse solche Probleme.
Eine Reduktion $\text{HAMPATH}\le_P\text{BESUCH}$ ist:
\begin{align*}
\text{Knoten}    & \to \text{Destination}\\
\text{Kante}     & \to \text{Flug} \\
\text{hamiltonscher Pfad}&\to\text{Besuchsplan}
\end{align*}
\end{loesung}

\begin{diskussion}
Man könnte auch das Travelling-Salesman-Problem (TSP) als Vergleichsproblem
heranziehen.
Dies geht leider nicht ganz, weil das übliche TSP von einem ungerichteten
Graphen ausgeht.
Es gibt allerdings auch das {\em asymmetrische TSP}, welches von einem 
gerichteten Graphen mit richtungsabhängigen Kantengewichten ausgeht.
Das asymmetrische TSP unterscheidet sich von HAMPATH durch die
zusätzlichen Kantengewichte, die für die Lösung eine möglichst
kleine Summe ergeben.
Eine solche Lösung muss also zusätzlich ein Mapping für die
Kantengewichte spezifizieren.
\end{diskussion}

\begin{bewertung}
Reduktionsmethode ({\bf R}) 1 Punkt,
HAMPATH als Vergleichsproblem ({\bf H}) 3 Punkte,
Reduktionsabbildung für Knoten ({\bf D}, 1 Punkt) und Kanten
({\bf F}, 1 Punkt).
\end{bewertung}

