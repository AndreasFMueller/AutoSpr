Das Acht-Damen-Problem ist die Schach-Aufgabe, acht Damen so auf einem
Schachbrett aufzustellen, dass sie sich nicht gegenseitig schlagen k"onnen.
Eine Dame kann eine andere Figur schlagen, die sich in der gleichen Zeile,
Spalte oder Diagonale befindet.
Offenbar ist acht die maximale Anzahl von Damen, die man auf diese Art auf
einem Schachbrett platzieren kann.

Betrachten Sie jetzt das allgemeinere Problem
\[
\textsl{DAMEN}
=
\left\{\; n
\;\left|\;
\begin{minipage}{0.7\hsize}
Es gibt eine Platzierung von $n$ Damen auf einem $n\times n$-Schachbrett,
die sich nicht gegenseitig schlagen k"onnen
\end{minipage}\right.\right\}.
\]
\begin{teilaufgaben}
\item
Konstruieren Sie eine Reduktion von \textsl{DAMEN} auf \textsl{SAT}.
\item
Bestimmen Sie den Rechenaufwand Ihrer Konstruktion in Abh"angigkeit von $n$.
\end{teilaufgaben}

\begin{hinweis}
Verwenden Sie als Codierung von $D$ die logischen Variablen $x_{ij}$
mit $1\le i,j\le n$, die angeben, ob auf dem Feld $(i,j)$ eine Dame
steht.
\end{hinweis}

\begin{loesung}
\begin{teilaufgaben}
\item
Wir m"ussen f"ur jede Zahl $n$ eine Formel $\varphi_n$ in den Variablen
$x_{ij}$ mit $1\le i,j\le n$ konstruieren, die genau dann erf"ullbar ist,
wenn sich $n$ Damen auf dem Feld platzieren lassen, die sich nicht schlagen
lassen.

Diese Formel muss ausdr"ucken, dass in der gleichen Zeile, Spalte und
Diagonalen keine weitere Dame steht.
Wenn auf dem Feld $(i,j)$ eine Dame steht, dann wird dies gem"ass Hinweis
dadurch ausgedr"uckt, dass $x_{ij}$ wahr ist.
Wir m"ussen daher eine Formel bauen, die sicherstellt, dass in diesem Fall
alle $x_{kl}$ falsch sind, die zur gleichen Zeile, Spalte oder Diagonalen
geh"oren.

Wir schreiben $\oplus$ f"ur die XOR-Verkn"upfung.
Die Bedingung, dass keine weiteren Damen in der gleichen Zeile stehen,
kann man durch die Formel
\begin{align}
\varphi_{ij,\text{Zeile}}
&=
x_{ij}\oplus(x_{i1}\vee x_{i2}\vee \dots \vee \hat x_{ij}\vee\dots\vee x_{in})
\label{70000036:oder}
\\
&=
x_{ij}\oplus\biggl(\bigvee_{k\ne j} x_{ik}\biggr)
\notag
\intertext{ausdr"ucken. Dabei bedeutet $\hat x_{ij}$, dass die Variable
$x_{ij}$ in der Klammer auf der rechten Seite weggelassen werden soll.
Analog kann man auch die Bedingungen f"ur die Spalten und die Diagonalen
ausdr"ucken:}
\varphi_{ij,\text{Spalte}}
&=
x_{ij}\oplus(x_{1j}\vee\dots\vee\hat x_{ij}\vee\dots\vee x_{nj})
\notag
\\
&=
x_{ij}\oplus\biggl(\bigvee_{l\ne i} x_{lj}\biggr),
\notag
\\
\varphi_{ij,\text{Diagonalen}}
&=
x_{ij}\oplus\biggl(\bigvee_{\text{$k,l$ in Diagonalen von $i,j$}}x_{kl}\biggr).
\notag
\end{align}
Die Variablen $x$ repr"asentieren genau dann eine akzeptable Platzierung
wenn f"ur jedes Paar $(i,j)$ alle drei soeben entwickelten Formeln wahr werden:
\[
\varphi_{ij}
=
\varphi_{ij,\text{Zeile}}
\wedge
\varphi_{ij,\text{Spalte}}
\wedge
\varphi_{ij,\text{Diagonalen}}.
\]
Die gesuchte Formel ist daher
\[
\varphi = \bigwedge_{i,j=1}^n \varphi_{ij}.
\]
\item
F"ur den Aufbau dieser Formel braucht f"ur jedes Paar $(i,j)$ den gleichen
Aufwand $O(n)$.
Es gibt $n^ 2$ solche Paare, der gesamte Aufwand ist daher $O(n^3)$.
\qedhere
\end{teilaufgaben}
\end{loesung}

\begin{diskussion}
Es stellt sich vielleicht die Frage, warum diese Formeln garantieren, dass
genau $n$ Damen auf dem Feld platziert worden sind.
In der Formel~\eqref{70000036:oder} kommt jede Variable einer Zeile vor.
Würde die Zeile keine Damen enhalten, wäre auch keine dieser
Formeln $\varphi_{ij,\text{Zeile}}$ wahr, und damit auch nicht $\varphi_{ij}$
und $\varphi$.
Es muss also in jeder Zeile und Spalte eine Dame vorkommen, also $n$
Damen.

Diese Reduktion ist je nach Codierung von $n$ keine polynomielle Reduktion.
Zwar ist der Aufwand f"ur die Konstruktion $O(n^3)$, doch f"ur eine
polynomielle Reduktion brauchen wir, dass die Laufzeit polynomiell in
der Inputl"ange ist.
Codiert man $n$ bin"ar, dann ist die Inputl"ange $N=\log_2n$, die Laufzeit
der Reduktion ist daher $O(2^{3N})$, also exponentiell.
W"ahlt man aber die un"are Darstellung, in der die Zahl $n$ durch $n$ Einsen
codiert wird, dann ist die Inputl"ange von $n$ auch $n$, und die Laufzeit
ist $O(n^ 3)$, also polynomiell.
\end{diskussion}


