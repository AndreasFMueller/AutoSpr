{\em Corral} (auch bekannt als {\em Cave} oder {\em Bag}, japanisch
\begin{CJK}{UTF8}{min}バッグ\end{CJK})
ist ein Logikrätsel, welches auf einem Gitter mit $n\times n$
Knoten gespielt wird.
In einzelne Feldern des Gitters sind Zahlen eingetragen.
Der Spieler muss einen Rundweg enlang der Rasterlinien einzeichnen, der keinen
Rasterpunkt mehrfach durchläuft, aber nicht notwendigerweise durch alle
Rasterpunkte führt.
Folgende Regeln müssen eingehalten werden:
\begin{enumerate}
\item
Alle Zahlenfelder müssen im Inneren des Rundwegs liegen.
\item
Eine Zahl gibt an, wie viele Felder man insgesamt innerhalb des
Rundwegs sieht, wenn man vom Zahlenfeld ausgehend nach Norden,
Süden, Osten und Westen blickt.
Das Feld mit der Zahl zählt dabei einmal mit.
\item
Andere Zahlen behindern die Sicht nicht, aber man kann nicht durch
Felder ausserhalb des Rundwegs in andere Teile des Rundwegs blicken.
\end{enumerate}
Hier ein Corral-Rätsel (links) mit Lösung (rechts):
\begin{center}
\begin{tikzpicture}[>=latex,thick,scale=0.7]

\def\vorgabe#1#2#3{
	\node at ({#1+0.5},{#2+0.5}) {$#3$};
}

\def\vorgaben{
	\vorgabe{0}{3}{3}
	\vorgabe{0}{6}{3}
	\vorgabe{1}{0}{9}
	\vorgabe{1}{4}{15}
	\vorgabe{2}{1}{2}
	\vorgabe{2}{5}{3}
	\vorgabe{3}{7}{7}
	\vorgabe{4}{2}{3}
	\vorgabe{4}{6}{3}
	\vorgabe{5}{0}{4}
	\vorgabe{6}{2}{4}
	\vorgabe{7}{3}{10}
}

\def\spielfeld{
	\foreach \x in {0,...,8}{
		\draw[line width=0.3pt] (\x,0) -- (\x,8);
		\draw[line width=0.3pt] (0,\x) -- (8,\x);
		\foreach \y in {0,...,8}{
			\fill (\x,\y) circle[radius=0.08];
		}
	}
}

\def\loesungspfad{
	   (0,0) -- (2,0) -- (2,1) -- (3,1) -- (3,2) -- (2,2) -- (2,4)
	-- (3,4) -- (3,3) -- (4,3) -- (4,2) -- (5,2) -- (5,3) -- (6,3)
	-- (6,2) -- (7,2) -- (7,1) -- (4,1) -- (4,0) -- (8,0) -- (8,6)
	-- (7,6) -- (7,5) -- (3,5) -- (3,6) -- (2,6) -- (2,7) -- (4,7)
	-- (4,6) -- (6,6) -- (6,7) -- (7,7) -- (7,8) -- (0,8) -- (0,6)
	-- (1,6) -- (1,5) -- (0,5) -- (0,3) -- (1,3) -- (1,1) -- (0,1)
	-- cycle
}

\def\kreuz#1{
	\node at ($#1+(0.5,0.5)$) {$\times$};
}

\def\loesung{
	\fill[color=gray!50] \loesungspfad;
	\draw[line width=2.0pt] \loesungspfad;
	\kreuz{(2,0)}
	\kreuz{(3,0)}

	\kreuz{(0,1)}
	\kreuz{(3,1)}
	\kreuz{(4,1)}
	\kreuz{(5,1)}
	\kreuz{(6,1)}

	\kreuz{(0,2)}
	\kreuz{(2,2)}
	\kreuz{(3,2)}
	\kreuz{(5,2)}

	\kreuz{(2,3)}

	\kreuz{(0,5)}
	\kreuz{(3,5)}
	\kreuz{(4,5)}
	\kreuz{(5,5)}
	\kreuz{(6,5)}

	\kreuz{(2,6)}
	\kreuz{(3,6)}
	\kreuz{(6,6)}
	\kreuz{(7,6)}

	\kreuz{(7,7)}
}

\begin{scope}[xshift=-6cm]
\spielfeld
\vorgaben
\end{scope}

\begin{scope}[xshift=6cm]
\spielfeld
\loesung
\vorgaben
\end{scope}

\end{tikzpicture}
\end{center}
Kann eine nichtdeterministische Turing-Maschine in polynomieller Zeit
entscheiden, ob ein Corral-Rätsel eine Lösung hat?

\thema{NP}
\thema{polynomieller Verifizierer}

\begin{loesung}
Das Problem ist sicher entscheidbar, da man alle endlich vielen Abfolgen
von Gitterpunkten daraufhin testen kann, ob sie einen geschlossenen Pfad
entlang von Gitterlinien bilden und die drei Regeln einhalten.

Das Problem ist durch eine nichtdeterministische Turing-Maschine in 
polynomieller Zeit entscheidbar, wenn es einen polynomiellen Verifizierer
gibt.
Ein solcher verlangt als Zertifikat den Pfad von Gitterpunkten,
von denen es $n^2$ gibt und führt damit folgende Verifikationen durch:
\begin{center}
\begin{tabular}{>{$}c<{$}|p{8cm}|>{$}c<{$}}
\text{Regel}&Verifikation&\text{Laufzeit}\\
\hline
0
&Entlang des Pfades überprüfen, ob aufeinanderfolgende Punkte des
Pfades auf dem Gitter benachbart sind
&O(n^2)
\\
1
&Mit einem der bekannten Algorithmen für das Problem, ob ein Punkt
im Inneren eines Pfades liegt, überprüfen, ob jede der maximal $n^2$
Zahlen im Inneren des Pfades liegt.
&O(n^2\cdot n^2)
\\
2
&Die Felder in der Zeile und Spalte jeder Zahl bis zu einem Pfadsegment
auszählen und mit der Zahl vergleichen.
Regel 3 ist damit automatisch auch überprüft.
&O(n^2\cdot 2n)
\\
\hline
&Total&O(n^4)
\end{tabular}
\end{center}
Damit ist gezeigt, dass der Verifizierer polynomielle Laufzeit hat.
Somit ist Corral in NP.
\end{loesung}

\begin{bewertung}
Entscheidbarkeit ({\bf E}) 1 Punkt,
Prinzip Verifizierer ({\bf V}) 1 Punkt,
Zertifikat spezifiziert ({\bf Z}) 1 Punkt,
Laufzeitschätzung ({\bf L}) 2 Punkte,
Schlussfolgerung polynomieller Verifizierer ({\bf S}) 1 Punkt.
\end{bewertung}

\begin{diskussion}
Erich Friedmann hat beweisen, dass Corral NP-vollständig ist.
\end{diskussion}
