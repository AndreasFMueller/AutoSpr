Konstruieren Sie eine Reduktion
$\textsl{SUDOKU}\le_P\textsl{VERTEX-COLORING}$.

\begin{hinweis}
Es reicht, die Reduktion für den Fall $n=2$ durchzuführen, also für
das $2^2\times 2^2$-Sudoku, wenn daraus klar
wird, wie die Verallgemeinerung für grösser $n$ zu handhaben ist.
\end{hinweis}

\begin{loesung}
Zu einem Sudoku-R"atsel konstruieren wir einen Graphen wie folgt:
Jedes Feld des Sudoku-R"atsel wird zu einem Vertex des Graphen.
Der Graph erh"alt eine Kante f"ur jedes Paar von Feldern, die nicht
das gleiche Zeichen enthalten d"urfen, also wenn die beiden Felder in der
gleichen Zeile, der gleichen Spalte oder dem gleichen Unterfeld liegen.

F"ur ein $2^2\times 2^2$-Sudoku wie das in Abbildung~\ref{70000028:sudoku}
ergibt sich so der Graph:
\begin{equation}
\entrymodifiers={++[o][F]}
\xymatrix {
{}\ar@{-}[r] \ar@{-}[d] \ar@{-}[dr]
	\ar@/^10pt/@{-}[rr]
	\ar@/^15pt/@{-}[rrr]
	\ar@/_10pt/@{-}[dd]
	\ar@/_15pt/@{-}[ddd]
	&{} \ar@{-}[d]
		\ar@{-}@/^2pt/[r]
		\ar@{-}@/^10pt/[rr]
		\ar@/^10pt/@{-}[dd]
		\ar@/^15pt/@{-}[ddd]
		&{}\ar@{-}[r] \ar@{-}[d] \ar@{-}[dr]
			\ar@/_10pt/@{-}[dd]
			\ar@/_15pt/@{-}[ddd]
			\color{red}
			&{} \ar@{-}[d]
				\ar@/^10pt/@{-}[dd]
				\ar@/^15pt/@{-}[ddd]
				\color{red}
\\
{}\ar@{-}[r] \ar@{-}[ur]
	\ar@/_10pt/@{-}[rr]
	\ar@/_15pt/@{-}[rrr]
	\ar@{-}@/_2pt/[d]
	\ar@{-}@/_10pt/[dd]
	&{}
		\ar@{-}@/^2pt/[d]
		\ar@{-}@/^10pt/[dd]
		\ar@{-}@/_2pt/[r]
		\ar@{-}@/_10pt/[rr]
		&{}\ar@{-}[r] \ar@{-}[ur]
			\ar@{-}@/_2pt/[d]
			\ar@{-}@/_10pt/[dd]
			&{}
				\ar@{-}@/^2pt/[d]
				\ar@{-}@/^10pt/[dd]
\\
{}\ar@{-}[r] \ar@{-}[d] \ar@{-}[dr]
	\ar@/^10pt/@{-}[rr]
	\ar@/^15pt/@{-}[rrr]
	&{} \ar@{-}[d]
		\ar@{-}@/^2pt/[r]
		\ar@{-}@/^10pt/[rr]
		&{}\ar@{-}[r] \ar@{-}[d] \ar@{-}[dr]
			&{} \ar@{-}[d]
\\
{}\ar@{-}[r] \ar@{-}[ur]
	\ar@/_10pt/@{-}[rr]
	\ar@/_15pt/@{-}[rrr]
	&{}
		\ar@{-}@/_2pt/[r]
		\ar@{-}@/_10pt/[rr]
		&{}\ar@{-}[r] \ar@{-}[ur]
			&{}
}
\label{70000028:graph}
\end{equation}

Ausserdem m"ussen wir die Vorgaben abbilden. Dazu konstruieren wir noch
zus"atzliche Vertices, die wir mit den Zeichen aus $\Sigma$ anschreiben.
Wir nennen diese Vertizes die {\em Zeichenvertices}.
Die Vorgabefelder werden mit all den Zeichenvertices verbunden, die verschieden
sind vom Vorgabezeichen.
F"ur das $2^2\times 2^2$-Sudoku aus Abbildung \ref{70000028:sudoku}
\begin{figure}
\begin{center}
\includeagraphics[width=0.25\hsize]{sudoku-1.pdf}
\end{center}
\caption{$2^2\times 2^2$-Sudoku zu Aufgabe \ref{70000028}
\label{70000028:sudoku}}
\end{figure}
m"ussen wir zum Graphen (\ref{70000028:graph}) noch die folgenden Kanten
hinzuf"ugen:
\begin{equation}
\entrymodifiers={++[o][F]}
\xymatrix {
{1}
	&*+\txt{}
		&*+\txt{}
			&*+\txt{}
				&*+\txt{}
					&{2}
\\
*+\txt{}
	&{}%\ar@{-}[r] \ar@{-}[d] \ar@{-}[dr]
		%\ar@/^10pt/@{-}[rr]
		%\ar@/^15pt/@{-}[rrr]
		%\ar@/_10pt/@{-}[dd]
		%\ar@/_15pt/@{-}[ddd]
		&{} %\ar@{-}[d]
			%\ar@{-}@/^2pt/[r]
			%\ar@{-}@/^10pt/[rr]
			%\ar@/^10pt/@{-}[dd]
			%\ar@/^15pt/@{-}[ddd]
			&{}%\ar@{-}[r] \ar@{-}[d] \ar@{-}[dr]
				%\ar@/_10pt/@{-}[dd]
				%\ar@/_15pt/@{-}[ddd]
				\ar@{-}[urr]
				\ar@{-}@/_9pt/[ddddrr]
				\ar@{-}@/_25pt/[ddddlll]
				&{} %\ar@{-}[d]
					%\ar@/^10pt/@{-}[dd]
					%\ar@/^15pt/@{-}[ddd]
					\ar@{-}[ullll]
					\ar@{-}[ur]
					\ar@{-}@/^12pt/[ddddllll]
					&*+\txt{}
\\
*+\txt{}
	&{}%\ar@{-}[r] \ar@{-}[ur]
		%\ar@/_10pt/@{-}[rr]
		%\ar@/_15pt/@{-}[rrr]
		%\ar@{-}@/_2pt/[d]
		%\ar@{-}@/_10pt/[dd]
		&{}
			%\ar@{-}@/^2pt/[d]
			%\ar@{-}@/^10pt/[dd]
			%\ar@{-}@/_2pt/[r]
			%\ar@{-}@/_10pt/[rr]
			&{}%\ar@{-}[r] \ar@{-}[ur]
				%\ar@{-}@/_2pt/[d]
				%\ar@{-}@/_10pt/[dd]
				&{}
					%\ar@{-}@/^2pt/[d]
					%\ar@{-}@/^10pt/[dd]
					&*+\txt{}
\\
*+\txt{}
	&{}%\ar@{-}[r] \ar@{-}[d] \ar@{-}[dr]
		%\ar@/^10pt/@{-}[rr]
		%\ar@/^15pt/@{-}[rrr]
		&{} %\ar@{-}[d]
			%\ar@{-}@/^2pt/[r]
			%\ar@{-}@/^10pt/[rr]
			&{}%\ar@{-}[r] \ar@{-}[d] \ar@{-}[dr]
				&{} %\ar@{-}[d]
					&*+\txt{}
\\
*+\txt{}
	&{}%\ar@{-}[r] \ar@{-}[ur]
		%\ar@/_10pt/@{-}[rr]
		%\ar@/_15pt/@{-}[rrr]
		\ar@{-}[uuuul]
		\ar@{-}[dl]
		\ar@{-}[drrrr]
		&{}
			%\ar@{-}@/_2pt/[r]
			%\ar@{-}@/_10pt/[rr]
			\ar@{-}@/_9pt/[uuuull]
			\ar@{-}@/_25pt/[uuuurrr]
			\ar@{-}[dll]
			&{}%\ar@{-}[r] \ar@{-}[ur]
				&{}
					&*+\txt{}
\\
{3}
	&*+\txt{}
		&*+\txt{}
			&*+\txt{}
				&*+\txt{}
					&{4}
}
\label{70000028:vorgaben}
\end{equation}
Die Vereinigung der Kanten in (\ref{70000028:graph}) und
(\ref{70000028:vorgaben}) liefert den Graphen, der dem gegebenen
Sudoku-R"atsel zugeordnet wird.

F"ur die Zahl der Farben setzen wir $k=|\Sigma|$, im Beispiel des
$2^2\times 2^2$-Sudokus aus Abbildung~\ref{70000028:sudoku} ist $k=4$.

Ist das Sudoku-R"atsel l"osbar, kann man man eine beliebige
Zuordnung von Farben zu den einzelnen Zeichen von $\Sigma$ verwenden,
um die Vertizes des Graphen einzuf"arben. 

Ist umgekehrt der konstruierte Graph mit $k$ Farben einf"arbbar, dann kann man
die Farben der Zeichen-Vertizes verwenden, um allen anderen Vertizes
ebenfalls ein Zeichen zuzuordnen. Diese Zeichenbelegung ist automatisch
eine L"osung. Die Kanten innerhalb des Sudoku-Feldes sorgen daf"ur, dass
die Sudoku-Regeln eingehalten werden, die Kanten zu den Zeichen-Vertizes
sorgen daf"ur, dass die Vorgaben erf"ullt sind.

Wir haben damit gezeigt, dass der konstruierte Graph genau dann mit
$k$ Farben einf"arbbar ist, wenn das Sudoku-R"atsel l"osbar ist. 
Ausserdem ist der Aufwand f"ur die Konstruktion des Graphen von der
Gr"ossenordnung des urspr"unglichen Sudoku, also $O(n)$. Die konstruierte
Reduktion ist also sogar eine polynomielle Reduktion.
\end{loesung}


