Ein grosses Softwareentwicklungsprojekt ist wegen der aus dem Ruder gelaufenen
Kosten gezwungen, zu redimensionieren.
Zu diesem Zweck stellt der Projektleiter eine Liste von Skills zusammen,
die er nach der Entlassungswelle in seinem Team noch haben muss und
kämpft nun seit Tagen damit, ein Auswahl von Mitarbeitern zu finden,
in der jeder Skill in genau einem Teammitglied vertreten ist.
Warum fällt ihm das so schwer?

\themaL{NP-vollstandig}{NP-vollständig}

\begin{loesung}
Dies ist das Problem HITTING-SET.
Wir bezeichnen die Skills mit $i\in I$, $S_i$ ist die Menge der
Team-Mitglieder, die Skill $i$ haben.
Gesucht ist eine Menge
\[
H\subset \bigcup_{i\in I} S_i
\]
derart, dass $|H\cap S_i|=1$ für jeden Skill.
Wir haben also eine 1-1-Reduktion
\[
\begin{aligned}
&\text{Skills}                      &&\leftrightarrow &&I\\
&\text{Skill}                       &&\leftrightarrow &&i\\
&\text{Teammitglieder mit Skill $i$}&&\leftrightarrow &&S_i\\
&\text{neues Team}                  &&\leftrightarrow &&H
\end{aligned}
\]
Da HITTING-SET NP-vollständig ist, muss davon ausgegangen werden, dass
es keinen Algorithmus mit polynomieller Laufzeit zur Bestimmung der
Menge $H$ gibt.
\end{loesung}

\begin{bewertung}
Reduktionsmethode ({\bf R}) 1 Punkt,
Vergleichsproblem ({\bf V}) 1 Punkt,
eins zu eins Mapping ({\bf M}) 2 Punkte,
Vergleichsproblem ist NP-vollständig ({\bf N}) 1 Punkt,
Schlussfolgerung ({\bf S} 1 Punkt.
\end{bewertung}


