In einer Firma mit $n$ Mitarbeitern nutzen viele Mitarbeiter die
Möglichkeit des Home Office und kommen nur an wenigen Tagen ins Büro.
Daher steht auch nicht für jeden Mitarbeiter ein Arbeitsplatz zur
Verfügung.
Die Mitarbeiter haben aber feste Wochentage, an denen sie ins Büro
kommen.
Es soll jetzt eine Zuordnung von Mitarbeitern zu Arbeitsplätzen
vorgenommen werden, die einem Mitarbeiter ermöglicht, jeweils am
gleichen Arbeitsplatz zu arbeiten.
Mehrere Mitarbeiter können sich einen Arbeitsplatz teilen, was möglich
ist, wenn sie nie am gleichen Tag im Büro sind.
Der CEO möchte wissen, ob die $k$ zur Verfügung stehenden Arbeitsplätze
ausreichen.
Warum ist es schwierig, ihm rasch eine Antwort zu geben?

\begin{loesung}
Das Arbeitsplatzproblem \textit{WORKPLACE} lässt sich wie folgt eins-zu-eins
auf das bekanntermassen NP-vollständige Problem $k$-\textit{VERTEX-COLORING}
reduzieren:
\begin{center}
\begin{tabular}{r>{$}c<{$}l}
\textit{WORKPLACE} && \textit{$k$-VERTEX-COLORING}  \\
\hline
Mitarbeiter                   &\leftrightarrow& Knoten \\
am gleichen Wochentag im Büro &\leftrightarrow& Kante \\
Arbeitsplatz                  &\leftrightarrow& Farbe \\
Anzahl $k$ der Arbeitsplätze  &\leftrightarrow& Anzahl $k$ der Farben \\
\end{tabular}
\end{center}
Dies zeigt, dass \textit{WORKPLACE} ein NP-vollständiges Problem ist.
Ein aktuellem Stand des Wissens gibt es keinen polynomiellen Algorithmus,
der die Frage des CEO beantworten könnte.
\end{loesung}
