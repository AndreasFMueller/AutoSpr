Das Spiel {\em Hitori} wird auf einem $n\times n$-Feld gespielt, in jeder
Zelle des Spielfeldes ist eine Zahl zwischen 1 und $n$ eingetragen.
Der Spieler muss nun Zellen schwärzen, so dass zwei Regeln
eingehalten werden:
\begin{enumerate}
\item In einer Zeile darf keine (nicht geschwärzte) Zahl mehr als
einmal vorkommen.
\item Benachbarte Zellen dürfen nicht geschwärzt werden.
\end{enumerate}
Die folgende Abbildung zeigt ein {\em Hitori} (links) mit Lösung (rechts).
\begin{center}
\includeagraphics[width=0.3\hsize]{Hitori.png}
\qquad
\includeagraphics[width=0.3\hsize]{Hitoricompleted.png}
\end{center}
Kann eine nichtdeterministische Turing-Maschine in polynomieller Zeit
entscheiden, ob ein {\em Hitori} eine Lösung hat?

\begin{loesung}
Das Problem ist sicher entscheidbar, man kann alle $2^{n^2}$
möglichen Schwärzungen des Spielfeldes daraufhin testen, ob sie die
Regeln 1.~und 2.~erfüllen.

Es ist für eine nicht deterministische Maschine genau dann möglich,
die Lösbarkeit eines {\em Hitori} zu entscheiden, wenn man einen
polynomiellen Verifizierer finden kann.

Als Lösungszertifikat für einen solchen Verifizierer verwendet man die
Zellen, die geschwärzt werden müssen.
Darauf ist der folgende Verifikationsalgorithmus anzuwenden:
\begin{compactenum}
\item Für jede geschwärzte Zelle kontrolliere, ob die maximal
vier Nachbarzellen nicht geschwärzt sind.
\item Für jede nicht geschwärzte Zelle kontrolliere, ob die anderen
Zellen in der gleichen Zeile eine andere Zahl enthalten.
\item Für jede nicht geschwärzte Zelle kontrolliere, ob die anderen
Zellen in der gleichen Spalte eine andere Zahl enthalten.
\end{compactenum}
Der Rechenaufwand für die einzelnen Schritte ist:
\begin{center}
\begin{tabular}{c|l|>{$}c<{$}}
1.&Nachbarzellen nicht geschwärzt&O(n^2)\\
2.&Keine gleichen Werte in einer Zeile&O(n^3)\\
3.&Keine gleichen Werte in einer Spalte&O(n^3)\\
\hline
  &Lösung für Hitori&O(n^3)
\end{tabular}
\end{center}
Daraus kann man ablesen, dass die Verifikation in polynomieller Zeit 
möglich ist. Das Problem zu entscheiden, ob ein {\em Hitori} eine 
Lösung hat, ist also in NP.
\end{loesung}

\begin{bewertung}
Zertifikat ({\bf Z}) 1 Punkt,
Drei wesentliche Schritte des Entscheidungsalgorithmus ({\bf E}) je 1 Punkt,
maximal 3 Punkte,
Abschätzung der Laufzeitkomplexität für alle Schritte ({\bf L}) 1 Punkt,
Schlussfolgerung, dass das Problem in NP ist ({\bf NP}) 1 Punkt.
\end{bewertung}

