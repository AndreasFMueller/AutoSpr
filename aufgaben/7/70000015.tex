Nurikabe ist ein japanisches Logikrätsel, welches auf einem rechteckigen
Gitter mit $u\times v$ Zellen gespielt wird. In einigen Feldern dieses Gitters
sind Zahlen eingetragen.
der Spieler muss einige der leeren Felder auswählen und schwarz
färben. Die gefärbten Felder heissen ``Fluss''. Es entstehen so
weisse Gebiete, bestehend aus weissen Feldern, die entlang einer Kante
benachbart sind. Es müssen folgende Regeln beachtet werden:
\begin{enumerate}
\item In jedem weissen Gebiet gibt es genau ein Zahlfeld, und die Zahl darin
gibt an, wieviele weisse Felder das Gebiet umfasst.
\item Es darf nur einen Fluss geben (das schwarze Gebiet muss zusammenhängend
sein), und es darf keine schwarzen $2\times 2$-Gebiete geben (der Fluss hat
keine Tümpel).
\end{enumerate}
Die Abbildung zeigt ein korrekt gelöstes Nurikabe.
\begin{center}
\includeagraphics[width=2truein]{nurikabe}
\end{center}
Ist die Sprache
\[
\text{\textsl{NURIKABE}}=\{\langle N\rangle\,|\,\text{$N$ ist ein lösbares Nurikabe-Rätsel}\}
\]
in NP?

\thema{NP}
\thema{polynomieller Verifizierer}

\begin{loesung}
Wir schreiben $n=uv$ für die Anzahl der Felder.
Sie ist mit Sicherheit entscheidbar, denn man kann alle $2^{n}$
möglichen Belegungen des Spielfeldes mit schwarzen Feldern daraufhin
überprüfen, ob sie die Regeln einhalten. Dieser Algorithmus braucht
zwar exponentielle Zeit, zeigt aber, dass die Sprache entscheidbar ist.

Es ist zu zeigen, dass es einen polynomiellen Verifizierer gibt. Dieser
braucht ein Zertifikat $c$, für welches wir eine Liste der schwarzen
Felder verlangen. Damit wird folgender Verifikationsalgorithmus
durchgeführt:
\begin{center}
\begin{tabular}{rll}
&Schritt&Aufwand\\
\hline
1&\begin{minipage}[t]{4.2truein}\strut
Für jedes Zahlfeld ($O(n)$) verwendet man einen Markieralgorithmus,
der alle zum Gebiet dieses Zahlfeldes gehörenden weissen
Felder bestimmt ($O(n)$ Durchgänge mit Aufwand
$O(n)$).\strut\end{minipage}&$O(n^3)$\\
2&\begin{minipage}[t]{4.2truein}\strut
Trifft dieser Algorithmus auf ein weiteres
Zahlfeld, ist der erste Teil von Regel 1 verletzt
($\to q_{\text{reject}}$).
\strut\end{minipage}&$O(n)$\\
3&\begin{minipage}[t]{4.2truein}\strut
Weicht die Zahl der gefundenen weissen Felder vom Inhalt des Zahlfeldes
ab, ist der zweite Teil von Regel 1 verletzt
($\to q_{\text{reject}}$).\strut\end{minipage}&$n\cdot O(n)$\\
4&\begin{minipage}[t]{4.2truein}\strut
Ebenfalls mit einem Markieralgorithmus wird überprüft, ob das
schwarze Gebiet zusammenhängend ist.
\strut\end{minipage}&$O(n^2)$\\
5&\begin{minipage}[t]{4.2truein}\strut
Für jedes schwarze Feld wird überprüft, ob es Teil eines
$2\times 2$-Tümpels ist.
\strut\end{minipage}&$O(n)$\\
\hline
&Total&$O(n^3)$
\end{tabular}
\end{center}
In den Aufwandschätzungen verwenden wir für $n$ die Länge der
Beschreibung $\langle N\rangle$. Da der Aufwand für die Verifikation
polynomiell in der Grösse des Nurikabe ist, ist
$\text{\textsl{NURIKABE}}\in\operatorname{NP}$.

In der Tat ist sogar wahr, dass \textsl{NURIKABE} NP-vollständig
ist. Dazu wurde eine ähnlich Technik verwendet wie in dem in
der Vorlesung skizzierten Beweis
der NP-Vollständigkeit von \textsl{MINESWEEPER-CONSISTENCY}:
Ein \textsl{SAT}-Problem wurde polynomiell in eine Schaltung und dann in
ein Nurikabe-Problem übersetzt:
\url{http://www.cs.umass.edu/\~\ mcphailb/papers/2003thesis.pdf}
\end{loesung}
