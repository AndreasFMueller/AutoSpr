Konstruieren Sie ein binäres Additionsprogramm für eine geeignet
ausgebaute Turing-Maschine, welches Laufzeit
$O(n)$ hat, wobei $n$ die Anzahl Stellen der Summanden ist.
Warum ist Ihre Maschine so viel schneller als die in der Vorlesung
im Video gezeigte Maschine?

\thema{Komplexität}

\begin{loesung}
Wir verwenden eine Maschine mit drei Bändern. Zunächst schreiben
wir den zweiten Summanden auf das zweite Band, was in $O(n)$
Schritten durchgeführt werden kann.
Dann beginnt die
eigentlich Addition, wir lesen den ersten Summanden vom ersten Band,
beginnend beim niederwertigsten Bit, und den zweiten Summanden
vom zweiten Band, ebenfalls beginnend beim niederwertigsten Bit.
Die Summe wird jeweils auf Band 3 geschrieben. Nach $O(n)$
solchen Operationen steht das Resultat auf Band 3.

Nach einem in der Vorlesung bewiesenen Satz benötigt
die Simulation dieser Maschine mit drei Bändern auf einer Standardmaschine
die Laufzeit $O(n^2)$, was sich mit der Laufzeit der Maschine
aus der Vorlesung deckt.
\end{loesung}


