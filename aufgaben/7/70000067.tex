Das 8-Damen-Problem verlangt vom Spieler, 8 Damen so auf einem Schachfeld
zu platzieren, dass sie sich gegenseitig nicht schlagen können.
Eine Dame kann eine andere Figure in der gleichen Zeile, Spalte oder
in diagonaler Richtung schlagen.
Überlegen Sie sich einen einfachen Algorithmus, mit dem Sie Lösungen
für das $n$-Damen-Problem finden können, und bestimmen sie die
Laufzeit in Abhängigkeit von $n$.

\begin{loesung}
Die Bedingungen des $n$-Damen-Problems besagen, dass man in jede Zeile
und Spalte des $n\times n$-Schachfeldes genau eine Dame platzieren muss.
Zusätzlich muss die Bedingung erfüllt sein, dass sich die Damen über die
Diagonalen auch nicht schlagen können.

Solche Lösungen kann man finden, indem man die Damen 
der Reihe nach in den Spalten $1$--$n$ platziert.
Für die Dame in Spalte $1$ hat man $n$ Möglichkeiten, für die Dame
in Spalte $2$ bleiben $n-1$ Möglichkeiten usw.~bis zur Dame $n$, für
die nur noch eine Möglichkeit übrig bleibt.
Insgesamt sind dies $n\cdot (n-1)\cdot\ldots\cdot 3\cdot 2\cdot 1=n!$
Möglichkeiten.
Jede dieser Möglichkeiten muss jetzt noch darauf getestet werden,
ob sich die Damen über die Diagonale gegenseitig schlagen können.

Die Fakultät $n!$ wächst exponentiell mit $n$ an, also wächst die
Laufzeit für diesen Algorithmus ebenfalls exponentiell an.
\end{loesung}
