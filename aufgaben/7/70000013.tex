Mitglieder des sozialen Netzwerkes Twitter {\it folgen} einander,
sie stehen zueinander in einer asymmetrischen Beziehungen. Ein Mitglied
kann ein {\it Follower} eines anderen Mitglieds sein, das Umgekehrte
muss aber nicht zutreffen. Die Twitter-Mitglieder bilden also
einen sehr grossen gerichteten Graphen.

Mitteilungen einzelner Mitglieder heissen {\it Tweets}, sie werden
allen Followern angezeigt. Die Follower können die Tweets {\it retweeten},
so dass ihre eigenen Follower diese Tweets auch sehen können.
Ein Tweet kann also potentiell eine sehr grosse Menge von Twitterern
erreichen.

Gibt es einen effizienten Algorithmus, mit dem man entscheiden kann,
ob es $k$ Twitterer gibt, die zusammen potentiell alle Twitter-Mitglieder
erreichen könnten?

\thema{NP-vollständig}
\thema{polynomielle Reduktion}

\begin{loesung}
Sei $S_i$ die Menge der Twitterer, die Twitterer $i$ erreichen kann,
wenn alle seine Tweets retweetet werden. Es ist klar, dass
\[
\bigcup_{i\in I}S_i,\qquad I=\{\text{Twitter-Mitglieder}\}
\]
alle Twitterer umfasst. Gefragt ist eine $k$-elementige Teilmenge
$I'\subset I$, so dass die Vereinigung der $S_i$ mit $i\in I'$
ebenfalls alle Twitterer umfasst:
\[
\bigcup_{i\in I}S_i
=
\bigcup_{i\in I'}S_i.
\]
Dies ist eine Instanz des Problems $k$-{\it SET-COVER} (Elemente sind
die Twitterer $i$, Teilmengen sind die von $i$ aus erreichbaren
Twitterer $S_i$). Dieses Problem ist bekanntermassen NP-vollständig,
nach aktuellem Wissenstand gibt es daher keinen effizienten Algorithmus, um
die Frage zu entscheiden.

Man könnte versuchen, andere Problem aus dem Katalog von Karp für
die Reduktion zu verwenden. Dadurch wird die Sache aber vor allem
schwieriger. Zum Beispiel könnte man versuchen, {\it VERTEX-COVER} zu
verwenden: die Konten sind die Twitterer und die Kanten drücken
die Beziehung aus, dass ein Tweet einen Twitterer erreichen kann.
Die Kanten drücken jedoch nicht mehr aus, dass der Tweet nur zu
den Followern fliesst, nicht umgekehrt.
Daher ist {\it VERTEX-COVER} nicht direkt nutzbar.

Auch {\it HAMPATH} ist nicht geeignet.
Zwar ist in {\it HAMPATH} die Rede davon, dass alle Knoten erreicht
werden können, aber Tweets müssen sich ja nicht auf einem einzigen
Rundgang durch den Twitterer Graphen verbreiten lassen, sie können
sich auch ``teilen''. Ebensowenig ist {\it FEEDBACK-NODE-SET}
geeignet, denn da ist von Zyklen die Rede, zu denen es im Twitterproblem
keine Entsprechung gibt.
\end{loesung}
