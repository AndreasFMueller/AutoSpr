Im $n$-Damen-Problem (Aufgabe~\ref{70000067}) können die Positionen
der Damen mit einer booleschen Variable $x_{ik}$ für jedes Feld 
in Zeile $i$ und Spalte $k$ des Spielfeldes codiert werden.
Ist die Variable \texttt{true}, befindet sich eine Dame auf dem
Feld, ist sie \texttt{false}, ist das Feld leer.
\begin{teilaufgaben}
\item
Finden Sie eine logische Formel $\varphi_{ik}$, die besagt, dass
eine Dame im Feld $ik$ steht, aber keine weitere Dame in der
gleichen Zeile.
\item
Ist $\varphi_{ik}$ wahr, wenn die Dame in einem anderen Feld der Zeile $i$
steht?
\item
Finden Sie eine logische Formel, die besagt, dass eine Dame in genau einem
Feld der Zeile $i$ steht.
\item
Finden Sie eine logische Formel, die besagt, dass in jeder Zeile genau
eine Dame steht.
\ifthenelse{\boolean{loesungen}}{
\item
Formulieren Sie die Bedingungen für eine Lösung des $n$-Damen-Problems
als logische Formel in den Variablen $x_{ik}$.
}{}
\end{teilaufgaben}

\begin{loesung}
\begin{teilaufgaben}
\item
Die Variable $x_{ik}$ muss wahr sein und keine der anderen Variablen
$x_{il}$, $l\ne k$, dürfen wahr sein.
\begin{align*}
\varphi_{ik}
&=
x_{ik}\wedge \neg x_{i1} \wedge
\ldots
\wedge \widehat{\neg x_{ik}}\wedge
\ldots
\wedge \neg x_{in}
\\
&=
x_{ik} \wedge \bigwedge_{l\ne k} \neg x_{il}
=
x_{ik} \wedge \neg \bigvee_{l\ne k} x_{il}.
\end{align*}
Darin meint das Dach in $\widehat{\neg x_{ik}}$, dass dieser Term weggelassen
werden soll.
\item
Wenn $x_{ik}$ falsch ist, kann die Formel nicht mehr wahr sein, auch wenn 
eine andere der Variablen $x_{il}$ wahr ist.
\item
Da mindestens eine der Formeln $\varphi_{ik}$ wahr sein muss, ist
die gesuchte Formel
\begin{equation}
\varphi_{\text{Zeile}_i}
=
\bigvee_{k=1}^n
\biggl(
x_{ik} \wedge \neg \bigvee_{l\ne k} x_{il}
\biggr).
\label{70000072:zeile}
\end{equation}
\item
Damit auf jeder Zeile genau eine Dame steht, muss $\varphi_{\text{Zeile}_i}$
für jedes $i=1,\dots,n$ wahr sein:
\[
\varphi_{\text{Zeilen}}
=
\bigwedge_{i=1}^n
\varphi_{\text{Zeile}_i}
=
\bigwedge_{i=1}^n
\biggl(
\bigvee_{k=1}^n
\biggl(
x_{ik} \wedge \neg \bigvee_{l\ne k} x_{il}
\biggr)
\biggr).
\]
\item
Die Formel $\varphi_{\text{Zeilen}}$ stellt sicher, dass genau
$n$ Damen auf dem Feld stehen.
Es ist daher nur noch sicherzustellen, dass in einer Spalte
keine weiteren Damen stehen, wenn in einem Feld schon eine
Dame steht.
Wenn $x_{ik}$ wahr ist, darf keine andere der Variablen $x_{lk}$
wahr sein, also
\[
\varphi_{\text{Spalte}_k}
=
\bigwedge_{i=1}^n
\biggl(
x_{ik}\Rightarrow \neg \bigvee_{l\ne i}x_{lk}
\biggr).
\]
Für alle Spalten bedeutet das
\[
\varphi_{\text{Spalten}}
=
\bigwedge_{k=1}^n
\varphi_{\text{Spalte}_k}
=
\bigwedge_{i,k=1}^n
\biggl(
x_{ik}\Rightarrow \neg \bigvee_{l\ne i}x_{lk}
\biggr).
\]
Es bleibt noch sicherzustellen, dass sich die Damen nicht gegenseitig
über die Diagonale schlagen können.
Falls die Dame in Feld $x_{ik}$ platziert ist, dann darf es auf der
gleichen Diagonalen keine weitere Dame haben.
Wir schreiben $ik\sim rs$, wenn die Felder $ik$ und $rs$ auf der 
gleichen Diagonalen sind.
Damit 
\[
\varphi_{\text{Diagonalen}_{ik}}
=
x_{ik}\Rightarrow \neg\bigvee_{rs\sim ik\wedge rs\ne ik}  x_{rs}
\]
wahr sein, damit auf den gleichen Diagonalen keine weitere Dame steht.

Diese Formeln kann man jetzt zu einer Formel zusammensetzen, die 
alle Bedingungen einschliesst:
\begin{align*}
\varphi
&
\varphi_{\text{Zeilen}}
\wedge
\varphi_{\text{Spalten}}
\wedge
\varphi_{\text{Diagonalen}}
\\
&=
\bigwedge_{i=1}^n \varphi_{\text{Zeile}_i}
\wedge
\bigwedge_{k=1}^n \varphi_{\text{Spalte}_k}
\wedge
\bigwedge_{ik} \varphi_{\text{Diagonalen}_{ik}}
\\
&=
\bigwedge_{i=1}^n
\biggl(
\bigvee_{k=1}^n 
\biggl(
x_{ik}\wedge\neg\bigvee_{l\ne k} x_{il}
\biggr)
\biggr)
\wedge
\bigwedge_{i,k=1}^n
\biggl(
x_{ik}\Rightarrow\neg\bigvee_{l\ne i} x_{lk}
\biggr)
\wedge
\bigwedge_{i,k=1}^n
\biggl(
x_{ik}\Rightarrow \neg \bigvee_{rs\sim ik\wedge rs\ne ik}  x_{rs}
\biggr).
\qedhere
\end{align*}
\end{teilaufgaben}
\end{loesung}
