Das Spiel {\em Slither Link} wird auf einem $n\times m$-Gitter von Punkten 
gespielt.
In einzelnen der Gitterquadrate stehen Zahlen zwischen 0 und 4.
Der Spieler muss benachbarte Punkte des Gitters so verbinden, dass ein
geschlossener Pfad entsteht. 
Der Pfad darf aber dem Rand eines Gitterquadrates, welches eine Zahl enthält,
nur an so vielen Seiten folgen, wie die Zahl angibt.
\begin{center}
\includeagraphics[width=0.3\hsize]{example.png}
\qquad
\qquad
\qquad
\includeagraphics[width=0.3\hsize]{answer.png}
\end{center}
Kann eine nichtdeterministische Turing-Maschine in polynomieller Zeit
entscheiden, ob ein {\em Slither Link}-Rätsel eine Lösung hat?

\begin{loesung}
Es ist für eine nicht deterministische Maschine möglich, die
Lösbarkeit eines {\em Slither Link}-Rätsels zu entscheiden, wenn man
einen polynomiellen Verifizierer finden kann.

Als Lösungszertifikat $c$ für so einen Verifizierer verwendet man die
Kanten, die den gesuchten geschlossenen Pfad ergeben. Darauf ist folgender
Verifikationsalgorithmus anzuwenden:
\begin{compactenum}
\item Kontrolliere, ob jedes Gitterquadrat mit einer Zahl die richtige
Anzahl von Kanten im Pfad hat.
\item Kontrolliere, ob jede Ecke, die von den Kanten in $c$ berührt werden,
genau von zwei Kanten getroffen wird. Damit ist sichergestellt, dass 
der $c$ keine Enden hat, es ist aber immer noch möglich, dass $c$
aus mehreren geschlossenen Pfad-Teilen besteht.
\item Schritt 2 ist Voraussetzung dafür, dass man beginnend bei einer
Kante jeweils zur nächsten Kante gehen kann.
Wähle jetzt eine Kante und folge der den Nachbarkannten, markiere
dabei jede gefundene Kante.
Kontrolliere, dass alle Kanten in $c$ markiert sind. Dies bedeutet,
dass $c$ nur einen einzigen geschlossenen Pfad enthält.
\end{compactenum}
Der Rechenaufwand für die einzelnen Schritt ist:
\begin{center}
\begin{tabular}{c|l|>{$}c<{$}}
1.&Richtige Anzahl für Gitterquadrate mit Zahl&O(nm)\\
2.&Keine Enden und Verzweigungen&O(nm)\\
3.&Nur eine Pfadkomponenten&O(mn)\\
\hline
&geschlossener Pfad&O(mn)\\
\hline
\end{tabular}
\end{center}
Insbesondere ist die Verifikation in polynomieller Zeit möglich. 
Das Problem, {\em Slither Link} zu entscheiden, ist also in NP.
\end{loesung}

\begin{diskussion}
Takayuki Yato hat nachgewiesen, dass {\em Slither Link}
NP-vollständig ist:
\url{http://www-imai.is.s.u-tokyo.ac.jp/~yato/data2/SIGAL74-3.pdf}
\end{diskussion}

\begin{bewertung}
Verifizierer (\textbf{V}) 1 Punkt,
Zertifikat (\textbf{Z}) 1 Punkt,
Verfikationsalgorithmus (\textbf{A}) 2 Punkte,
Aufwandschätzung ($O(?)$) (\textbf{O}) 1 Punkt,
Schlussfolgerung (\textbf{NP}) 1 Punkt.
\end{bewertung}
