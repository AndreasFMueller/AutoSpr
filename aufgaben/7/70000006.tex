Job-Parallelisierbarkeit:
Gegeben sei eine Menge von $n$ Jobs $J_1,\dots,J_n$, die jeweils exklusiv
auf $m$ Resourcen $r_1,\dots,r_m$ zugreifen, diese Jobs dürfen
nicht gleichzeitig laufen. Zeigen Sie, dass das Problem zu entscheiden,
ob mit diesen Jobs zu irgend einem Zeitpunkt mehr als $k$ Prozessoren
ausgelastet werden können, NP-vollständig ist.

\themaL{NP-vollstandig}{NP-vollständig}
\thema{polynomielle Reduktion}

\begin{loesung}
Wir zeigen, dass das Problem äquivalent ist mit dem $k$-Cliquen-Problem,
welches in der Vorlesung bereits als NP-vollständig erkannt worden ist.

Aus einem Job-Parallelisierbarkeitsproblem konstruieren wir einen Graphen $G$,
dessen Ecken die Jobs sind. Zwischen je zwei Jobs fügen wir eine
Kante hinzu, wenn die Jobs gleichzeitig laufen können, wenn also
keine der Resourcen von beiden Jobs beansprucht wird. Eine $k$-Clique
ist eine Auswahl von $k$ Prozessen, die gegenseitig keine Resourcen-Konflikte
haben, also gleichzeitig laufen können. Gibt es eine $k$-Clique in $G$,
lassen sich mit den Jobs der Clique $k$ Prozessoren auslasten.

Sei umgekehrt ein Graph gegeben. Wir nennen die Ecken ``Jobs''. Die
Paare $(J_i,J_j)$, für die im Graph keine Kante existiert,  nennen
wir Resourcen $r_{ij}$. Die Kanten des Graphen drücken also aus, dass
die beiden Jobs an den Enden der Kante gleichzeitig laufen können.
Falls sich $k$ der Jobs gleichzeitig starten lassen, enthält der
Graph jede Verbindung zwischen diesen Jobs, die $k$ Jobs bilden also
eine $k$-Clique des Graphen.

Somit sind das Job-Parallelitäts-Problem und das Cliquen-Problem
äquivalent, insbesondere ist Job-Parallelität ein NP-vollständiges
Problem.
\end{loesung}
