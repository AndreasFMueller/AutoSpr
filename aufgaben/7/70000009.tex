Facebook hat einige hundert Millionen aktiver Mitglieder.
Jeder Teilnehmer kann mit jedem anderen Teilnehmer ``befreundet''
sein, oder auch nicht.
\begin{teilaufgaben}
\item
Wie aufwendig ist es, die Liste aller ``Freunde''
eines Teilnehmers so zu sortieren, dass zwei aufeinanderfolgende ``Freunde''
in der Liste untereinander ebenfalls befreundet sind.
\item
Wie aufwendig
ist es herauszufinden, ob es eine Liste aller Facebookteilnehmer gibt,
die so sortiert ist, dass je zwei aufeinanderfolgende Teilnehmer
auch ``befreundet'' sind.
\end{teilaufgaben}

\thema{NP-vollständig}
\thema{polynomielle Reduktion}

\begin{loesung}
Wir nennen die Aufgabe, eine solche Liste zu produzieren das ``Facebook
Freundelisten-Problem'' FFLP. Wir zeigen, dass das Problem
UHAMPATH, einen Hamiltonschen Pfad in einem Graphen zu finden,
polynomiell auf FFLP reduziert werden:
$\text{UHAMPATH}\le_p\text{FFLP}$.
Sei $G$ ein Graph, dann konstruieren wir daraus ein FFLP wie
folgt. Die Vertizes von $G$ werden als Teilnehmer definiert,
zwei Teilnehmer sind genau dann befreundet, wenn die zugehörigen
Vertizes in $G$ mit einer Kante verbunden sind. Die Aufgabe, einen
Hamiltonschen Pfad zu finden läuft dann darauf hinaus, eine Liste
von Vertizes anzugeben, in der jeder Vertex genau einmal vorkommt,
und aufeinanderfolgende Vertizes im Graphen mit einer Kante verbunden
sind. "Ubersetzt bedeutet das, dass eine Liste von Teilnehmern gefunden
werden muss, so dass zwei aufeinanderfolgende Teilnehmer auch Freunde sind.
Da UHAMPATH NP-vollständig ist folgt, dass auch \text{FFLP} NP-vollständig
ist.
\begin{teilaufgaben}
\item Die Freunde eines Teilnehmers bilden ein FFLP mit relativ wenigen
Teilnehmern, nach aktuellem Wissen muss aber damit gerechnet werden,
dass jeder Algorithmus, der FFLP löst, länger als in der Freundezahl
polynomielle Laufzeit braucht, um die Frage zu entscheiden.
\item Angesichts der grossen Zahl von Facebook-Teilnehmern ist es
wegen der langen Laufzeit eines Algorithmus, der FFLP löst,
praktisch unmöglich zu entscheiden, ob es eine Liste der verlangten
Art gibt.
\qedhere
\end{teilaufgaben}
\end{loesung}
