CalcuDoku ist ein japanisches Logikr"atsel, in dem in einem $4\times 4$-Feld
die Ziffern 1 bis 4 eingetragen werden m"ussen, so dass in jeder
Zeile und Spalte jede Ziffer genau einmal vorkommt.
Ausserdem sind Gruppen von Zahlen mit punktierten Linien, einem arithmetischen
Operationszeichen und einem Resultat markiert.
F"uhrt man die arithmetische Operation mit den Ziffern der Gruppe durch,
muss sich das angegeben Result ergeben.  Zum Beispiel muss die Summe
der ersten drei Ziffern in der ersten Zeile des folgenden CalcuDoku
$6$ ergeben. Die rechte Abbildung zeigt das vollst"andig gel"oste CalcuDoku.
\begin{center}
\begin{tabular}{cc}
\includeagraphics[width=0.4\hsize]{calcudoku-problem.png}&%
\includeagraphics[width=0.4\hsize]{calcudoku-solved.png}
\end{tabular}
\end{center}
Es k"onnen CalcuDokus der Gr"osse $n\times n$ aufgestellt werden, in 
deren Felder die Zahlen von $1$ bis $n$ eingetragen werden m"ussen.
\begin{teilaufgaben}
\item Ist entscheidbar, ob ein CalcuDoku "uberhaupt eine L"osung hat?
\item Zeigen Sie, dass CalcuDoku auf einer nicht deterministischen
Turingmaschine in polynomieller Zeit gel"ost werden kann.
\end{teilaufgaben}

\begin{loesung}
\begin{teilaufgaben}
\item
Der folgende Algorithmus entscheidet, ob ein CalcuDoku eine L"osung
hat:
\begin{compactenum}
\item[1.] Konstruiere eine Liste der $n^{n^2}$ Belegungen des CalcuDoku-Feldes
mit den Zahlen $1$ bis $n$.
\item[2.] W"ahle die erste Belegung aus der Liste
\item[3.] Pr"ufe, ob in jeder der $n$ Zeilen und $n$ Spalten jede der
$n$ Zahlen genau einmal vorkommt. Falls nicht, gehe zu 6.
\item[4.] Pr"ufe, ob in jeder der punktiert eingezeichneten Gruppen
(davon gibt es h"ochstens $\frac{n^2}2$, weil eine Gruppe mindestens
zwei Mitglieder haben muss), die Rechenoperationen das verlangte Resultat
geben. Falls nicht, gehe zu 6.
\item[5.]
Eine L"osung wurde gefunden: terminiere im Zustand $q_{\text{accept}}$.
\item[6.]
W"ahle die n"achste Belegung aus der Liste und fahre weiter bei 3.
\end{compactenum}
Damit ist gezeigt, dass entscheidbar ist, ob CalcuDoku eine L"osung
hat.
\item
Wir m"ussen einen polynomiellen Verifizerer finden. Als L"osungszertifikat
$c$ verlangen wir die Belegung des CalcuDoku-Feldes mit den Zahlen
$1$ bis $n$.
Die Schritte 3 und 4 im obigen Algorithmus verfizieren, ob eine solche
Belegung eine L"osung des CalcuDokus ist. Es ist also nur noch zu verfizieren,
dass die Berechnung in polynomieller Zeit durchf"uhrbar ist.

In Schritt 3 muss jede der $n$ Zeilen muss jede der $n$ Zahlen mit den
$n-1$ anderen Zahlen verglichen werden. Dies braucht Zeit $O(n^2(n-1))=O(n^3)$.
In Schritt 4 muss in jeder der h"ochstens $\frac{n^2}2$ Gruppen die 
Rechenoperation mit den h"ochstens $n^2$ Operanden durchgef"uhrt und das
Resultat "uberpr"uft werden. Dies braucht Zeit $O(\frac{n^2}2\cdot n^2)=O(n^4)$.
In jedem Fall sind also die Pr"ufungen in Schritt 3 und 4 in Zeit $O(n^4)$,
mithin in polynomieller Zeit durchf"uhrbar. Wir haben also einen polynomiellen
Verifizierer gefunden.
Auf einer nicht deterministischen Turing-Maschine ist CalcuDoku also
in polynomieller Zeit l"osbar.
\end{teilaufgaben}
\end{loesung}
