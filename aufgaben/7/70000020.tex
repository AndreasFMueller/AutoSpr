CalcuDoku ist ein japanisches Logikrätsel, in dem in einem $4\times 4$-Feld
die Ziffern 1 bis 4 eingetragen werden müssen, so dass in jeder
Zeile und Spalte jede Ziffer genau einmal vorkommt.
Ausserdem sind Gruppen von Zahlen mit punktierten Linien, einem arithmetischen
Operationszeichen und einem Resultat markiert.
Führt man die arithmetische Operation mit den Ziffern der Gruppe durch,
muss sich das angegeben Result ergeben.  Zum Beispiel muss die Summe
der ersten drei Ziffern in der ersten Zeile des folgenden CalcuDoku
$6$ ergeben. Die rechte Abbildung zeigt das vollständig gelöste CalcuDoku.
\begin{center}
\begin{tabular}{cc}
\includeagraphics[width=0.4\hsize]{calcudoku-problem.png}&%
\includeagraphics[width=0.4\hsize]{calcudoku-solved.png}
\end{tabular}
\end{center}
Es können CalcuDokus der Grösse $n\times n$ aufgestellt werden, in 
deren Felder die Zahlen von $1$ bis $n$ eingetragen werden müssen.
\begin{teilaufgaben}
\item Ist entscheidbar, ob ein CalcuDoku überhaupt eine Lösung hat?
\item Zeigen Sie, dass CalcuDoku auf einer nichtdeterministischen
Turingmaschine in polynomieller Zeit gelöst werden kann.
\end{teilaufgaben}

\thema{polynomieller Verifizierer}
\thema{NP}

\begin{loesung}
\begin{teilaufgaben}
\item
Der folgende Algorithmus entscheidet, ob ein CalcuDoku eine Lösung
hat:
\begin{compactenum}
\item[1.] Konstruiere eine Liste der $n^{n^2}$ Belegungen des CalcuDoku-Feldes
mit den Zahlen $1$ bis $n$.
\item[2.] Wähle die erste Belegung aus der Liste
\item[3.] Prüfe, ob in jeder der $n$ Zeilen und $n$ Spalten jede der
$n$ Zahlen genau einmal vorkommt. Falls nicht, gehe zu 6.
\item[4.] Prüfe, ob in jeder der punktiert eingezeichneten Gruppen
(davon gibt es höchstens $\frac{n^2}2$, weil eine Gruppe mindestens
zwei Mitglieder haben muss), die Rechenoperationen das verlangte Resultat
geben. Falls nicht, gehe zu 6.
\item[5.]
Eine Lösung wurde gefunden: terminiere im Zustand $q_{\text{accept}}$.
\item[6.]
Wähle die nächste Belegung aus der Liste und fahre weiter bei 3.
\end{compactenum}
Damit ist gezeigt, dass entscheidbar ist, ob CalcuDoku eine Lösung
hat.
\item
Wir müssen einen polynomiellen Verifizerer finden. Als Lösungszertifikat
$c$ verlangen wir die Belegung des CalcuDoku-Feldes mit den Zahlen
$1$ bis $n$.
Die Schritte 3 und 4 im obigen Algorithmus verfizieren, ob eine solche
Belegung eine Lösung des CalcuDokus ist. Es ist also nur noch zu verfizieren,
dass die Berechnung in polynomieller Zeit durchführbar ist.

In Schritt 3 muss jede der $n$ Zeilen muss jede der $n$ Zahlen mit den
$n-1$ anderen Zahlen verglichen werden. Dies braucht Zeit $O(n^2(n-1))=O(n^3)$.
In Schritt 4 muss in jeder der höchstens $\frac{n^2}2$ Gruppen die 
Rechenoperation mit den höchstens $n^2$ Operanden durchgeführt und das
Resultat überprüft werden. Dies braucht Zeit $O(\frac{n^2}2\cdot n^2)=O(n^4)$.
In jedem Fall sind also die Prüfungen in Schritt 3 und 4 in Zeit $O(n^4)$,
mithin in polynomieller Zeit durchführbar. Wir haben also einen polynomiellen
Verifizierer gefunden.
Auf einer nichtdeterministischen Turing-Maschine ist CalcuDoku also
in polynomieller Zeit lösbar.
\qedhere
\end{teilaufgaben}
\end{loesung}
