F"ur ein medizinische Studie ist eine grosse Zahl von Probanden rekrutiert
worden.
Sie sind bereits auf Allergien getestet worden, man weiss also von jedem
Probanden, auf welche Allergene (Pollen, Katzenhaare, Hausstaub, Lactose,\dots)
er allergisch reagiert.
Die Untersuchung soll sich auf eine Teilmenge von $k=17$ oder noch
mehr ausgew"ahlten
Allergenen beschr"anken, die so beschaffen ist, dass kein Proband auf mehr als 
eines der ausgew"ahlten Allergene reagiert.
Es stellt sich als schwierig heraus, eine solche Teilmenge zu finden.
Warum?

\begin{loesung}
Dies ist das Problem {\em SET-PACKING}, wenn man folgende Identifikation
vornimmt:
\begin{align*}
\text{Allergene}                            &\quad\leftrightarrow\quad I\\
\text{auf Allergen $i$ allergische Probanden}&\quad\leftrightarrow\quad S_i\\
\text{ausgew"ahlte Allergene}               &\quad\leftrightarrow\quad J\\
\text{Ausschlussbedingung zwischen Allergenen $i$ und $j$}&\quad\leftrightarrow\quad S_i\cap S_j = \emptyset
\end{align*}
Es wird verlangt, $k$ Allergene auszuw"ahlen, also eine Teilmenge
$J\subset I$ mit $|J|=k$ zu finden.
\end{loesung}

\begin{bewertung}
Reduktionsmethode ({\bf R}) 1 Punkt,
Vergleichsproblem ({\bf V}) 1 Punkt,
Mappings f"ur Allergene ({\bf A}), allergisch reagierende Probanden ({\bf S})
und Auswahl ({\bf W}) je 1 Punkt,
Mapping f"ur $k$ ({\bf K}) 1 Punkt.
\end{bewertung}

