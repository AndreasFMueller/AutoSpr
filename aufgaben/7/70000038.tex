Für eine medizinische Studie ist eine grosse Zahl von Probanden rekrutiert
worden.
Sie sind bereits auf Allergien getestet worden, man weiss also von jedem
Probanden, auf welche Allergene (Pollen, Katzenhaare, Hausstaub, Lactose,\dots)
er allergisch reagiert.
Die Untersuchung soll sich auf eine Teilmenge von $k=17$ oder noch
mehr ausgewählten
Allergenen beschränken, die so beschaffen ist, dass kein Proband auf mehr als 
eines der ausgewählten Allergene reagiert.
Es stellt sich als schwierig heraus, eine solche Teilmenge zu finden.
Warum?

\thema{NP-vollständig}
\thema{polynomielle Reduktion}

\begin{loesung}
Dies ist das Problem {\em SET-PACKING}, wenn man folgende Identifikation
vornimmt:
\begin{align*}
\text{Allergene}                            &\quad\leftrightarrow\quad I\\
\text{auf Allergen $i$ allergische Probanden}&\quad\leftrightarrow\quad S_i\\
\text{ausgewählte Allergene}               &\quad\leftrightarrow\quad J\\
\text{Ausschlussbedingung zwischen Allergenen $i$ und $j$}&\quad\leftrightarrow\quad S_i\cap S_j = \emptyset
\end{align*}
Es wird verlangt, $k$ Allergene auszuwählen, also eine Teilmenge
$J\subset I$ mit $|J|=k$ zu finden.
\end{loesung}

\begin{diskussion}
Man\footnote{zum Beispiel viele Prüfungsteilnehmer im FS 16}
könnte auch eine Reduktion auch auf {\em HITTING-SET} versuchen, etwa
mit der folgenden Identifikation:
\begin{align*}
\text{Allergene}                    &\qquad\leftrightarrow\qquad S             \\
\text{Probanden}                    &\qquad\leftrightarrow\qquad I             \\
\text{Allergien eines Probanden $i$}&\qquad\leftrightarrow\qquad S_i, i\in I   \\
\text{Auswahl von Allergenen}       &\qquad\leftrightarrow\qquad H\subset S    \\
\text{Proband hat nur eine der Allergien}&\qquad\leftrightarrow\qquad |S_i\cap H|=1\forall i\in I
\end{align*}
Leider funktioniert das nicht, weil im Aufgabenproblem die Anzahl der Allergene
vorgegeben ist, das entspräche hier der Vorgabe der Anzahl der Elemente
von $H$.
Es fehlt also ein Element des Aufgabenproblems in {\em HITTING-SET}.

Manchmal ist es schwierig {\em SET-COVERING}, {\em EXACT-COVER} und
{\em SET-PACKING} auseinander zu halten.
Man beachte:
\begin{itemize}
\item
In {\em SET-COVERING} und in {\em SET-PACKING} kommt eine Zahl $k$ vor,
nicht aber in {\em EXACT-COVER}.
\item
In {\em SET-COVERING} dürfen sich die Mengen schneiden, müssen aber auch
alles abdecken. 
In {\em SET-PACKING} dürfen sich die Mengen nicht schneiden, müssen aber
auch nicht alles abdecken.
\item
In {\em EXACT-COVER} dürfen sich die Mengen nicht schneiden, müssen alles
abdecken, aber es kommt nicht auf ihre Anzahl an.
\end{itemize}
Oder tabellarisch:
\[
\begin{tabular}{r|ccc}
                       &{\em SET-COVERING}&{\em SET-PACKING}&{\em EXACT-COVER}\\
\hline
Anzahl Mengen          &$k$               &$k$              &           \\
Vereinigung            &ganze Menge       &                 &ganze Menge\\
paarweise Schnittmengen&                  &$\emptyset$      &$\emptyset$
\end{tabular}
\]
\end{diskussion}

\begin{bewertung}
Reduktionsmethode ({\bf R}) 1 Punkt,
Vergleichsproblem ({\bf V}) 1 Punkt,
Mappings für Allergene ({\bf A}), allergisch reagierende Probanden ({\bf S})
und Auswahl ({\bf W}) je 1 Punkt,
Mapping für $k$ ({\bf K}) 1 Punkt.
Eine Reduktion auf die naheliegenden Vergleichsprobleme {\em HITTING-SET},
{\em EXACT-COVER} oder {\em SET-COVERING} gab 3 Punkte.
\end{bewertung}

