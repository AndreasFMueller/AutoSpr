n einem
$n$-Sudoku m"ussen die Zahlen $1$ bis $n^2$ auf die Felder eines
Spielfeld verteilt werden, welches sich aus $n\times n$ quadratisch
angeordneten Quadraten von je $n\times n$ Feldern besteht.
Ein keiner Zeile und Spalte und in keinem $n\times n$ Teilfeld
darf eine Zahl mehr als einmal vorkommen. Die bekannten Sudokus
sind 3-Sudokus.

Zeigen sie: wenn $\text{P}=\text{NP}$, dann gibt es einen Algorithmus,
der auch grosse Sudokus in polynomialer Zeit l"osen kann.

\ifthenelse{\boolean{loesungen}}{
\begin{loesung}
Wir m"ussen zun"achst zeigen, dass Sudoko ein NP-Problem ist.
Das Sudoku-Feld hat $n^4$ Felder. F"ur jedes Feld m"ussen die
folgenden Tests durchgef"uhrt werden. Zun"achst muss getestet
werden, ob die Zahl in dem Feld zwisch $1$ und $n^2$ liegt.
Dann muss die Zahl mit den $n^2$ Feldern der Zeile, der Spalte
und des gleichen Teilfeldes verglichen werden. Insgesamt sind
also $1+3n^2$ Vergleichsoperationen notwendig. Jeder solche Vergleich
braucht $O(\log n^)2$ Ziffernvergleiche, da eine Zahl $<n^2$ $O(\log n^2)$
Stellen hat.
Insgesamt braucht es also $O(n^2(1+3n^2)\log n^2)< O(n^4\log n)<O(n^5)$
Operationen. Damit ist gezeigt, dass das Sudoku-Problem einen polynomialen
Verifizierer hat, also in NP ist.

Wenn $\text{P}=\text{NP}$ ist, dann muss es auch einen polynomialen
Algorithmus geben, der Sudokus l"ost.
\end{loesung}

}{}

