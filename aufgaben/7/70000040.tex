An einer Veranstaltung werden verschiedene m"ogliche Aktivit"aten angeboten,
allerdings steht nicht gen"ugend Zeit zur Verf"ugung, so dass jeder
Teilnehmer nur an einer Aktivit"at teilnehmen kann.
Der Veranstalter bittet die Teilnehmer in der Anmeldung, auf einem
Formular alle Aktivit"aten anzukreuzen, f"ur die sie sich interessieren.
Aus Kostengr"unden will der Veranstalter dann aber nur $k$ Angebote
auch realisieren, die dann auch genutzt werden sollen.
Diese sollten zudem so sein, dass kein Teilnehmer sich f"ur mehr als
eines der realisierten Angebote angemeldet hat.
Ausserdem ist ihm egal, dass m"oglicherweise einige Teilnehmer nichts tun
werden.

Das Veranstaltungs-Sekretariat soll jetzt die Aktivit"aten ermitteln, 
die realisiert werden sollen.
Dies stellt sich als schwierig heraus, warum?

\begin{loesung}
Dies ist das NP-vollst"andige Problem \textsl{SET-PACKING}, wie man 
mittels der Reduktion
\begin{align*}
\textsl{SET-PACKING}&\phantom{\mathstrut\mapsto\mathstrut}\textrm{Aktivit"aten an einer Veranstaltung}
\\[5pt]
i&\mapsto \textrm{Aktivit"at}
\\
S_i&\mapsto \textrm{Teilnehmer, die sich f"ur $i$ angemeldet haben}
\\
k&\mapsto \textrm{Anzahl realisierte Aktivit"aten}
\\
J&\mapsto \textrm{tats"achlich realisierte Aktivit"aten}
\\
S_i\cap S_j=\emptyset\forall i\ne j
&\mapsto \textrm{Teilnehmer sind f"ur h"ochstens eine realisierte Aktivit"at angemeldet}
\end{align*}
erkennen kann.
\end{loesung}

\begin{bewertung}
1-1-Reduktionsmethode ({\bf R}) 1 Punkt,
richtiges Vergleichsproblem ({\bf V}) 1 Punkt,
Mapping ({\bf M}) 4 Punkte.
\end{bewertung}

