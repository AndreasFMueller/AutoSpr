An einer Veranstaltung werden verschiedene mögliche Aktivitäten angeboten,
allerdings steht nicht genügend Zeit zur Verfügung, so dass jeder
Teilnehmer nur an einer Aktivität teilnehmen kann.
Der Veranstalter bittet die Teilnehmer in der Anmeldung, auf einem
Formular alle Aktivitäten anzukreuzen, für die sie sich interessieren.
Aus Kostengründen will der Veranstalter dann aber nur $k$ Angebote
auch realisieren, die dann auch genutzt werden sollen.
Diese sollten zudem so sein, dass kein Teilnehmer sich für mehr als
eines der realisierten Angebote angemeldet hat.
Ausserdem ist ihm egal, dass möglicherweise einige Teilnehmer nichts tun
werden.

Das Veranstaltungs-Sekretariat soll jetzt die Aktivitäten ermitteln, 
die realisiert werden sollen.
Dies stellt sich als schwierig heraus, warum?

\thema{polynomielle Reduktion}
\thema{NP-vollständig}

\begin{loesung}
Dies ist das NP-vollständige Problem \textsl{SET-PACKING}, wie man 
mittels der Reduktion
\begin{align*}
\textsl{SET-PACKING}&\phantom{\mathstrut\mapsto\mathstrut}\textrm{Aktivitäten an einer Veranstaltung}
\\[5pt]
i&\mapsto \textrm{Aktivität}
\\
S_i&\mapsto \textrm{Teilnehmer, die sich für $i$ angemeldet haben}
\\
k&\mapsto \textrm{Anzahl realisierte Aktivitäten}
\\
J&\mapsto \textrm{tatsächlich realisierte Aktivitäten}
\\
S_i\cap S_j=\emptyset\forall i\ne j
&\mapsto \textrm{Teilnehmer sind für höchstens eine realisierte Aktivität angemeldet}
\end{align*}
erkennen kann.
\end{loesung}

\begin{bewertung}
1-1-Reduktionsmethode ({\bf R}) 1 Punkt,
richtiges Vergleichsproblem ({\bf V}) 1 Punkt,
Mapping ({\bf M}) 4 Punkte.
\end{bewertung}

