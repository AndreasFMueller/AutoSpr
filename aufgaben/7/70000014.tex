Im Zahlenrätsel Kakuro müssen die weissen Felder eines Spielplanes
so mit Zahlen zwischen 1 und 9 gefüllt werden, dass die Summe der
vertikal bzw.~horizontal zusammenhängenden weissen Felder den Wert im
schwarzen Feld darüber oder links daneben ergeben, wie im folgenden
Beispiel. 
\begin{center}
\includeagraphics[width=0.4\hsize]{kakuro.pdf}
\end{center}
\begin{teilaufgaben}
\item 
Ist entscheidbar, ob ein Kakuro-Rätsel überhaupt eine Lösung
hat?
\item
Zeigen Sie, dass Kakuro mit einer nicht deterministischen Turingmaschine
in polynomieller Zeit gelöst werden kann.
\end{teilaufgaben}

\thema{NP-vollständig}
\thema{polynomielle Reduktion}

\begin{loesung}
\begin{teilaufgaben}
\item
Um zu entscheiden, ob ein Kakuro-Rätsel überhaupt gelöst werden
kann, muss man alle $9^n$ möglichen Belegungen der $n$ weissen
Felder mit Zahlen zwischen 1 und 9 durchtesten. Dazu ist zwar
exponentielle Zeit $O(9^n)$ erforderlich ist, das tut der Entscheidbarkeit
jedoch keinen Abbruch.
Ein effizienterer Algorithmus verwendet Backtracking.
\item
Um zu zeigen, dass Kakuro in NP ist, muss man einen polynomiellen
Verifizierer angeben. Der Verifizierer braucht ein Lösungszertifikat,
in diesem Fall bietet sich an, dafür die in den weissen Feldern eingetragenen
Zahlen zu verlangen. Die Verifikation muss jetzt in jeder Zeile und
Spalte die Inhalte der zusammenhängenden weissen Felder addieren und 
mit der Vorgabe in den schwarzen Feldern vergleichen. Ist $n$ die Anzahl
der Felder, gibt es weniger als $n$ Summen, die überprüft werden müssen,
und jede dieser Summen umfasst weniger als $n$ Summanden. Die Verifikation
ist also in Zeit $O(n^2)$ möglich, mithin in polynomieller Zeit.
Also ist Kakuro in NP.
\qedhere
\end{teilaufgaben}
\end{loesung}
