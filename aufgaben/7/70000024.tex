Das Rätselspiel {\it Fillomino} wird auf einem $n\times m$ Spielfeld 
gespielt. In einzelnen Zellen des Spielfeldes sind Zahlen eingetragen.
Der Spieler muss die leeren Felder mit Zahlen füllen, so dass
zusammenhängende Gebiete soviele Felder enthalten
wie die Zahl angibt. Ein zusammenhängendes Gebiet besteht aus Feldern,
die sich entlang einer Kante berühren und alle die gleiche Zahl enthalten.
Die Abbildung zeigt ein Fillomino-Rätsel (links) mit Lösung (rechts).
\begin{center}
\includeagraphics[width=0.4\hsize]{fill1.png}
\qquad
\includeagraphics[width=0.4\hsize]{fill8.png}
\end{center}
Zeigen Sie, dass eine nichtdeterministische Maschine in polynomieller
Zeit entscheiden kann, ob ein Fillomino eine Lösung hat.

\thema{NP}
\thema{polynomieller Verifizierer}

\begin{loesung}
Es genügt zu zeigen, dass es einen polynomiellen Verifizierer gibt.

Ein polynomieller Verifizierer braucht ein Lösungszertifikat als 
zusätzlichen Input. Im vorliegenden Fall verlangen wir das vollständig
ausgefüllte Fillomino-Feld als Zertifikat.

Mit dem vollständig ausgefüllten Feld kann der Verifizierer wie folgt
vorgehen, um die Korrektheit der Lösung zu überprüfen.
\begin{enumerate}
\item "Uberprüfe, ob jede Zelle mit einer Zahl gefüllt ist.
\item Für jede Zelle des Spielfeldes prüfe wie folgt, ab die Zahl der
Zellen des zuammenhängenden Gebietes der Zelle die richtige Anzahl
Zellen enthält
\begin{enumerate}
\item Markiere die Startzelle
\item \label{70000024:loop} Gehe durch das Spielfeld und markiere alle Zellen, die zu bereits
markierten Zellen benachbart sind, und die die gleiche Zahl enthalten wie
die Startzelle.
\item Wiederhole \ref{70000024:loop} bis sich nichts mehr ändert
\item Zähle die markierten Zellen, brich mit $q_{\text{reject}}$ ab,
wenn die Zahl der markierten Zellen nicht mit der Zahl in der Startzelle
übereinstimmt.
\end{enumerate}
\end{enumerate}
Es muss jetzt nur noch gezeigt werden, dass der Verifizierer tatsächlich
in polynomieller Zeit fertig wird. Die einzelnen Schritte des Verifzierers
brauchen Zeit wie folgt
\begin{enumerate}
\item Zellen prüfen: $O(mn)$
\item In diesem Schritt werden $mn$ mal die folgenden Schritte ausgeführt:
\begin{enumerate}
\item Startzelle markieren: $O(1)$
\item Nachbarzellen markieren: $O(mn)$
\item Abbruch falls keine "Anderung: $O(1)$
\item Markierte Zellen zählen: $O(mn)$
\end{enumerate}
\end{enumerate}
Insgesamt ist also die Verifikation in maximal $mnO(mn)=O(m^2n^2)$ möglich,
insbesondere ist die Arbeit polynomiell in der Grösse des Spielfeldes.
Damit ist gezeigt, dass es einen polynomiellen Verifizierer gibt,
also kann eine nichtdeterministische Maschine das Problem in polynomieller
Zeit lösen.
\end{loesung}

\begin{diskussion}
Takayuki Yato hat 2003 in seiner Masterarbeit in Tokyo nachgewiesen,
dass Fillomino NP-vollständig ist:
\url{http://www-imai.is.s.u-tokyo.ac.jp/~yato/data2/MasterThesis.pdf}.
\end{diskussion}

\begin{bewertung}
Beobachtung, dass es reicht, einen polynomiellen Verifizerer ({\bf V})
zu finden 1 Punkt,
Definition des Zertifikats ({\bf Z}) 1 Punkt,
Definition des Verifikationsalgorithmus ({\bf A}) 2 Punkte,
Abschätzung der Laufzeitkomplexität ({\bf L}) 2 Punkte.
\end{bewertung}

