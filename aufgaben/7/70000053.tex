Ein Koch hat je $n$ Rezepte für Vorspeisen, Hauptspeisen und Desserts.
Nicht alle Vorspeisen lassen sich mit jeder Hauptspeise kombinieren,
dasselbe gilt auch für Desserts.
Damit jedes seiner Rezepte regelmässig zum Einsatz kommt, möchte der
Koch eine Folge von $n$ Menus zusammenstellen, so dass jedes Rezept
in genau einem der Menus vorkommt.
Nach längerem tüfteln gibt er jedoch frustriert auf.
Können Sie erklären, warum ihm die Menugestaltung so schwer gefallen ist.

\themaL{NP-vollstandig}{NP-vollständig}
\thema{polynomielle Reduktion}

\begin{loesung}
Es handelt sich hier um das Problem 3D-MATCHING.
Die Menge $T$ sind die Nummern der Rezepte, die Tripel aus $T\times T\times T$
sind die Menuzusammenstellungen, bestehend aus je einer Nummer für
ein Vorspeisen-, ein Hauptspeisen- und ein Dessert-Rezept.
Die Menge $U$ ist die Menge der möglichen Menukombinationen.
Die gesuchte Teilmenge $W$ ist eine Auswahl von $n=|T|=|W|$ Menus
derart, dass keine Vorspeise (erste Komponente), keine Hauptspeise
(zweite Komponente) und kein Dessert (dritte Komponente) mehr als
einmal vorkommt.
Das Problem 3D-MATCHING ist NP-vollständig, es ist daher kein
Algorithmus bekannt, der das Problem in polynomieller Zeit lösen könnte.
\end{loesung}

\begin{bewertung}
Reduktionsansatz ({\bf R}) 1 Punkt,
Wahl des Vergleichsproblems ({\bf V}) 1 Punkt,
Mapping für Rezepte ({\bf W}) 1 Punkt,
Mapping für Menus ({\bf U}) 1 Punkt,
Mapping für Bedingung ({\bf P}) 1 Punkt,
NP-Vollständigkeit ({\bf N}) 1 Punkt.
\end{bewertung}



