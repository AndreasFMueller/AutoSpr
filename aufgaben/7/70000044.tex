Eine Zeitschrift für Alpinisten möchte eine Artikelserie über selten bestiegene
Berggipfel veröffentlichen.
Sie hat bereits eine Liste von Gipfeln zusammengestellt
und bittet jetzt einen Alpinistenverein um eine Liste von
Alpinisten als mögliche Interviewpartner, die über ihre Erfahrung
beim Besteigen dieser Gipfel berichten können.
Um Mehrspurigkeiten aus dem Weg zu gehen, soll
kein Alpinist auf der Liste mehr als einen der aufgelisteten Gipfel
bestiegen haben.
Dem Verein fällt es sehr schwer, eine solche Liste zusammenzustellen,
woran könnte das liegen?

\themaL{NP-vollstandig}{NP-vollständig}

\begin{loesung}
Dies ist das Problem {\em HITTING-SET}.
Die Liste der Gipfel ist die Menge $I$.
Die Menge $S_i$ besteht aus allen Vereinsmitgliedern, die den Gipfel
$i$ bestiegen haben.
\begin{align*}
\text{Liste der Gipfel}&\leftrightarrow I
\\
\text{Besteiger von Gipfel $i$}&\leftrightarrow S_i
\\
\text{Alle in Frage kommenden Mitglieder}&\leftrightarrow S = \bigcup_{i\in I}S_i
\\
\text{Ausgewählte Besteiger}&\leftrightarrow H
\end{align*}
Gesucht wird eine Menge $H$ von Mitgliedern, so dass $|H\cap S_i|=1$.

Da {\em HITTING-SET}  NP-vollständig ist, müssen wir nach aktuellem
Wissen davon ausgehen, dass es keinen polynomiellen Algorithnus zur
Lösung des Problems gibt.
\end{loesung}

\begin{bewertung}
Reduktionsmethode ({\bf R}) 1 Punkt,
Vergleichsproblem ({\bf V}) 1 Punkt,
eins zu eins Mapping ({\bf M}) 2 Punkte,
Vergleichsproblem ist NP-vollständig ({\bf N}) 1 Punkt,
Schlussfolgerung ({\bf S} 1 Punkt.
\end{bewertung}

