Eine Firma konnte in einer feindlichen "Ubernahme einen wichtigen Konkurrenten
aufkaufen. Der Konkurrent ist eigentlich eine Vereinigung von sehr vielen,
sehr intensiv und effizient zusammenarbeitenden, aber im "Ubrigen weitgehend
selbständigen Abteilungen, alle unter dem selben Dach.
Das Ziel der "Ubernahme war, daraus die ``Rosinen'' herauszupicken, und
den Rest zu schliessen. Die Wettbewerbsbehörden hatten dies vorausgesehen,
und als Auflage für die "Ubernahme gemacht, dass keine einzige der Abteilungen
geschlossen werden dürfe.
Daher dachten sich die neuen Eigentümer den folgenden bösartigen Plan aus,
um den gleichen Zweck zu erreichen. Sie teilten die Abteilungen auf zwei 
verschiedene Standorte auf, und sorgten dafür, dass jede Kommunikation zwischen
den beiden Standorten so ineffizient wurde, dass die Abteilungen kaum mehr
sinnvoll zusammenarbeiten konnten.
Dadurch würden die einzelnen Abteilungen wirtschaftlich ruiniert, und man
müsste sie trotz allem schliessen.
Die neuen Eigentümer beauftragten daher eine Beratungsfirma, eine
Aufteilung zu finden, mit der die Kommunikation zwischen den Abteilungen
möglichst stark behindert würde.
Die Beratungsfirma brauchte dafür sehr lange. Warum ist das nicht
überraschend?

\thema{NP-vollständig}
\thema{polynomielle Reduktion}

\begin{loesung}
Es handelt sich hier um das Problem \textsl{MAX-CUT}.
Wir beschreiben eine Reduktion des Problems auf \textsl{MAX-CUT}:
\begin{align*}
\text{Abteilung}&\leftrightarrow \text{Vertex} \\
\text{Kommunikationsbeziehung}&\leftrightarrow \text{Kante} \\
\text{Kommunikationsvolumen}&\leftrightarrow \text{Gewicht einer Kante}
\end{align*}
Die neuen Firmeneigentümer wollen die Menge der Vertices so in zwei
Mengen aufteilen, dass die Summe der Gewichte der Kanten, die durch die
Aufteilung zerschnitten werden, möglichst
gross wird. Dies ist genau die Beschreibung des Problems \textsl{MAX-CUT}.

Das Problem \textsl{MAX-CUT} ist NP-vollständig, nach aktuellem Wissen
gibt es also keinen effizienten (polynomiellen) Algorithmus, der ein
\textsl{MAX-CUT} Problem lösen könnte.
\end{loesung}

\begin{bewertung}
Reduktionsansatz ({\bf R}) 1 Punkt,
Vergleichsproblem ({\bf M}) 1 Punkt,
Reduktionsabbildung ({\bf A}) je ein Punkt, maximal 3 Punkte,
Schlussfolgerung NP-Vollständigkeit ({\bf N}) 1 Punkt.
\end{bewertung}

