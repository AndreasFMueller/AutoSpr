Betrachten Sie den folgenden, in Tabellenform gegebenen
Automaten über dem Alphabet $\Sigma=\{0,1\}$:
\begin{center}
\begin{tabular}{|c|cc|}
\hline
&{\tt 0}&{\tt 1}\\
\hline
$z$&$e$&$z_1$\\
$z_0/E$&$z_0$&$z_1$\\
$z_1$&$z_2$&$z_0$\\
$z_2$&$z_1$&$z_3$\\
$z_3$&$z_1$&$z_2$\\
$e$&$e$&$e$\\
\hline
\end{tabular}
\end{center}
Zeichnen Sie das Zustandsdiagramm des Automaten. Konstruieren Sie den
zugehörigen minimalen Automaten.

\thema{minimaler Automat}
\thema{Zustandsdiagramm}

\begin{loesung}
Das folgende Zustandsdiagramm beschreibt den DEA:
%\[
%\UseTips
%\entrymodifiers={++[o][F]}
%\xymatrix @-1mm {
%*+\txt{}
%        &*+\txt{}
%                &*+\txt{}
%                        &z_2 \ar@/_/[d]^{\tt 1} \ar@/^/[dl]_{\tt 0}
%\\
%*+\txt{}\ar[r]
%        &z \ar[r]^{1} \ar[d]_{\tt 0}
%                &z_1 \ar@/_/[d]_{\tt 1}  \ar@/^/[ur]^{\tt 0}
%                        &z_3 \ar[l]^{\tt 0} \ar@/_/[u]_{\tt 1}
%\\
%*+\txt{}
%        &e \ar@(dr,dl)^{{\tt 0},{\tt 1}}
%                &*++[o][F=]{z_0} \ar@(dr,dl)^{\tt 0} \ar@/_/[u]_{\tt 1}
%}
%\]
\begin{center}
\begin{tikzpicture}[>=latex,thick]
\def\l{1.6}
\coordinate (z) at (0,0);
\coordinate (z1) at (\l,0);
\coordinate (z3) at ({2*\l},0);
\coordinate (z2) at ({2*\l},\l);
\coordinate (e) at (0,-\l);
\coordinate (z0) at (\l,-\l);
\node at (z) {$z$};
\node at (z0) {$z_0$};
\node at (z1) {$z_1$};
\node at (z2) {$z_2$};
\node at (z3) {$z_3$};
\node at (e) {$e$};
\draw (z) circle[radius=0.35];
\draw (z0) circle[radius=0.35];
\draw (z0) circle[radius=0.30];
\draw (z1) circle[radius=0.35];
\draw (z2) circle[radius=0.35];
\draw (z3) circle[radius=0.35];
\draw (e) circle[radius=0.35];
\draw[->,shorten <= 0.35cm,shorten >= 0.35cm] (-\l,0) -- (z);
\draw[->,shorten <= 0.35cm,shorten >= 0.35cm] (z) -- (z1);
\draw[->,shorten <= 0.35cm,shorten >= 0.35cm] (z3) -- (z1);
\draw[->,shorten <= 0.35cm,shorten >= 0.35cm] (z) -- (e);
\draw[->,shorten <= 0.35cm,shorten >= 0.35cm]
	(e) to[out=-60,in=-120,distance=1.2cm] (e);
\draw[->,shorten <= 0.35cm,shorten >= 0.35cm]
	(z0) to[out=-60,in=-120,distance=1.2cm] (z0);
\draw[->,shorten <= 0.35cm,shorten >= 0.35cm]
	(z1) to[out=-70,in=70] (z0);
\draw[->,shorten <= 0.35cm,shorten >= 0.35cm]
	(z0) to[out=110,in=-110] (z1);
\draw[->,shorten <= 0.35cm,shorten >= 0.35cm]
	(z2) to[out=-70,in=70] (z3);
\draw[->,shorten <= 0.35cm,shorten >= 0.35cm]
	(z3) to[out=110,in=-110] (z2);
\draw[->,shorten <= 0.35cm,shorten >= 0.35cm]
	(z2) -- (z1);
\draw[->,shorten <= 0.35cm,shorten >= 0.35cm]
	(z1) to[out=80,in=-165] (z2);
\node at ($0.5*(z2)+0.5*(z3)$) {\texttt{1}};
\node at ($0.5*(z0)+0.5*(z1)$) {\texttt{1}};
\node at ($0.5*(z)+0.5*(z1)$) [above] {\texttt{1}};
\node at ($0.5*(z1)+0.5*(z2)+(-0.15,0.15)$) {\texttt{0}};
\node at ($0.5*(z1)+0.5*(z3)$) [below] {\texttt{0}};
\node at ($0.5*(z)+0.5*(e)$) [left] {\texttt{0}};
\node at ($(z0)+(0,-1.1)$) {\texttt{0}};
\node at ($(e)+(0,-1.1)$) {\texttt{0},\texttt{1}};

\end{tikzpicture}
\end{center}
Für die Bestimmung der äquivalenten Zustände bauen wir die Tabellen
der nicht äquivalenten Zustände auf. Im ersten Schritt werden die
Paare aus einem Akzeptierzustand und einem Nicht-Akzeptierzustand markiert:
\begin{center}
\begin{tabular}{|c|cccccc|}
\hline
&$z$&$z_0/E$&$z_1$&$z_2$&$z_3$&$e$\\
\hline
$z$&$\equiv$&$\times$&&&&\\
$z_0/E$&$\times$&$\equiv$&$\times$&$\times$&$\times$&$\times$\\
$z_1$&&$\times$&$\equiv$&&&\\
$z_2$&&$\times$&&$\equiv$&&\\
$z_3$&&$\times$&&&$\equiv$&\\
$e$&&$\times$&&&&$\equiv$\\
\hline
\end{tabular}
\end{center}
In den folgenden Schritten prüft man jedes noch nicht markierte Paar
von Zuständen, ob sie sich durch Anwendung der "Ubergangsfunktion
in ein Paar überführen lassen, welches bereits markiert ist. Ist dies
der Fall, wird das entsprechende Feld ebenfalls markiert. In den folgenden
Tabellen sind die jeweils bereits bekannten Markierungen mit dem Zeichen
$\otimes$ ausgeführt.
\begin{center}
\begin{tabular}{|c|cccccc|}
\hline
&$z$&$z_0/E$&$z_1$&$z_2$&$z_3$&$e$\\
\hline
$z$&$\equiv$&$\otimes$&$\times$&&&\\
$z_0/E$&$\otimes$&$\equiv$&$\otimes$&$\otimes$&$\otimes$&$\otimes$\\
$z_1$&$\times$&$\otimes$&$\equiv$&$\times$&$\times$&$\times$\\
$z_2$&&$\otimes$&$\times$&$\equiv$&&$\times$\\
$z_3$&&$\otimes$&$\times$&&$\equiv$&$\times$\\
$e$&&$\otimes$&$\times$&$\times$&$\times$&$\equiv$\\
\hline
\end{tabular}
\end{center}
In der zweiten Iteration können noch sechs Paare markiert werden:
\begin{center}
\begin{tabular}{|c|cccccc|}
\hline
&$z$&$z_0/E$&$z_1$&$z_2$&$z_3$&$e$\\
\hline
$z$&$\equiv$&$\otimes$&$\otimes$&$\times$&$\times$&$\times$\\
$z_0/E$&$\otimes$&$\equiv$&$\otimes$&$\otimes$&$\otimes$&$\otimes$\\
$z_1$&$\otimes$&$\otimes$&$\equiv$&$\otimes$&$\otimes$&$\otimes$\\
$z_2$&$\times$&$\otimes$&$\otimes$&$\equiv$&&$\otimes$\\
$z_3$&$\times$&$\otimes$&$\otimes$&&$\equiv$&$\otimes$\\
$e$&$\times$&$\otimes$&$\otimes$&$\otimes$&$\otimes$&$\equiv$\\
\hline
\end{tabular}
\end{center}
Aber in der letzten Iteration bleiben die Felder $(z_2,z_3)$ und
$(z_3,z_2)$ unmarkiert, die beiden Zustände sind also äquivalent:
\begin{center}
\begin{tabular}{|c|cccccc|}
\hline
&$z$&$z_0/E$&$z_1$&$z_2$&$z_3$&$e$\\
\hline
$z$&$\equiv$&$\otimes$&$\otimes$&$\otimes$&$\otimes$&$\otimes$\\
$z_0/E$&$\otimes$&$\equiv$&$\otimes$&$\otimes$&$\otimes$&$\otimes$\\
$z_1$&$\otimes$&$\otimes$&$\equiv$&$\otimes$&$\otimes$&$\otimes$\\
$z_2$&$\otimes$&$\otimes$&$\otimes$&$\equiv$&&$\otimes$\\
$z_3$&$\otimes$&$\otimes$&$\otimes$&&$\equiv$&$\otimes$\\
$e$&$\otimes$&$\otimes$&$\otimes$&$\otimes$&$\otimes$&$\equiv$\\
\hline
\end{tabular}
\end{center}
Durch Zusammenlegen im Zustandsdiagramm erhält man jetzt als
kleinsmöglichen Automaten:
%\[
%\UseTips
%\entrymodifiers={++[o][F]}
%\xymatrix @-1mm {
%*+\txt{}\ar[r]
%        &z \ar[r]^{1} \ar[d]_{\tt 0}
%                &z_1 \ar@/_/[d]_{\tt 1}  \ar@/^/[r]^{\tt 0}
%                        &z_2,z_3 \ar@/^/[l]^{\tt 0} \ar@(dr,dl)^{\tt 1}
%\\
%*+\txt{}
%        &e \ar@(dr,dl)^{{\tt 0},{\tt 1}}
%                &*++[o][F=]{z_0} \ar@(dr,dl)_{\tt 0} \ar@/_/[u]_{\tt 1}
%}
%\qedhere
%\]
\begin{center}
\begin{tikzpicture}[>=latex,thick]
\def\l{1.8}
\coordinate (z) at (0,0);
\coordinate (z1) at (\l,0);
\coordinate (z2) at ({2.4*\l},0);
\coordinate (e) at (0,-\l);
\coordinate (z0) at (\l,-\l);
\node at (z) {$z$};
\node at (z0) {$z_0$};
\node at (z1) {$z_1$};
\node at (z2) {$z_2,z_3$};
\node at (e) {$e$};
\draw (z) circle[radius=0.35];
\draw (z0) circle[radius=0.35];
\draw (z0) circle[radius=0.30];
\draw (z1) circle[radius=0.35];
\draw (z2) ellipse (0.65 and 0.35);
\draw (e) circle[radius=0.35];
\draw[->,shorten <= 0.35cm,shorten >= 0.35cm] (-\l,0) -- (z);
\draw[->,shorten <= 0.35cm,shorten >= 0.35cm] (z) -- (e);
\draw[->,shorten <= 0.35cm,shorten >= 0.35cm] (z) -- (z1);
\draw[->,shorten <= 0.35cm,shorten >= 0.35cm] (z1) to[out=-60,in=60] (z0);
\draw[->,shorten <= 0.35cm,shorten >= 0.35cm] (z0) to[out=120,in=-120] (z1);
\draw[->,shorten <= 0.35cm,shorten >= 0.35cm]
	(e) to[out=-60,in=-120,distance=1.2cm] (e);
\draw[->,shorten <= 0.35cm,shorten >= 0.35cm]
	(z0) to[out=-60,in=-120,distance=1.2cm] (z0);
\draw[->,shorten <= 0.35cm,shorten >= 0.59cm]
	(z1) to[out=30,in=160] (z2);
\draw[->,shorten <= 0.59cm,shorten >= 0.35cm]
	(z2) to[out=-160,in=-30] (z1);
\draw[->,shorten <= 0.50cm,shorten >= 0.50cm]
	(z2) to[out=-30,in=30,distance=1.6cm] (z2);
\node at ($0.5*(z1)+0.5*(z0)$) {\texttt{1}};
\node at ($0.5*(z1)+0.5*(z2)$) {\texttt{0}};
\node at ($(z0)+(0,-1.1)$) {\texttt{0}};
\node at ($(e)+(0,-1.1)$) {\texttt{0},\texttt{1}};
\node at ($0.5*(e)+0.5*(z)$) [left] {\texttt{0}};
\node at ($0.5*(z)+0.5*(z1)$) [above] {\texttt{1}};
\node at ($(z2)+(1.4,0)$) {\texttt{1}};
\end{tikzpicture}
\end{center}
\qedhere
\end{loesung}

