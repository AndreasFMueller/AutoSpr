Untersuchen Sie die Sprache $L_\infty$, die aus den Wörtern besteht, die
in allen Sprachen $L_n$ aus Aufgabe \ref{30000022} vorkommen:
\[
L_{\infty} = \bigcap_{n>0}L_n.
\]
Sie besteht also aus Wörtern, für die
$|w|_{\tt 0}$ und
$|w|_{\tt 1}$ immer den gleichen Rest bei Teilung durch jede beliebige
Zahl $n$ haben.

\begin{teilaufgaben}
\item Warum haben die Wörter in $L_\infty$ gleich viele {\tt 0}
wie {\tt 1}?
\item Aus a) folgt, dass
$$L_\infty = \{w\in\Sigma^*|\;|w|_0=|w|_1\}.$$
Ist die Sprache $L_\infty$ regulär?
\end{teilaufgaben}

\thema{Pumping Lemma für reguläre Sprachen}
\thema{regulär}

\begin{loesung}
\begin{teilaufgaben}
\item Nehmen wir an, ein Wort $w$ habe $a$ {\tt 0}, also $a=|w|_{\tt 0}$ und
$b$ {\tt 1}, also $|w|_{\tt 1}=b$.
Sei $N=\max(a,b) + 1$. Das Wort $w$ ist auch in $L_N$ enthalten,
die Anzahlen $n$ und $m$ müssen aber auch den gleichen Rest bei
Teilung durch $N$ haben. Da $a<N$ ist $a$ auch gerade dieser Rest,
und analog für $b$. Somit muss $a=b$ sein. Die Wörter in
$L_\infty$ haben also gleich viele {\tt 0} und {\tt 1},
\[
L_\infty=\{w|\,|w|_{\tt 0}=|w|_{\tt 1}\}.
\]

Man könnte auch wie folgt argumentieren. Wenn zwei Zahlen $a$ und $b$
bei Teilung durch jede beliebige Zahl gleichen Rest ergeben müssen,
dann müssen sie das auch bei Teilung durch $b$ und $a$ tun. In beiden
Fällen ist der eine Rest $0$, also müssen beide Zahlen sowohl durch
$b$ wie auch durch $a$ teilbar sein. $b$ muss also Vielfaches von $a$
sein, und $a$ ein Vielfaches von $b$. Also einerseits $b\ge a$ und andererseits
$b\le a$, also $b=a$. Allerdings muss man sich den Fall $b=0$ oder $a=0$
noch separat überlegen, weil man nicht durch $0$ teilen darf. Wenn
aber eine der Zahlen, sagen wir $b$, gleich $0$ ist, bedeutet das, dass
die andere, also $a$, bei Teilung durch jede beliebige Zahl $0$ ergeben
muss. Teilt man $a$ durch $a+1$ ergibt sich $a$ als Rest, also ist $a=0$.

Schliesslich könnte man auch mit der Zifferndarstellung der Zahlen
$a$ und $b$ operieren. Den Rest von $a$ und $b$ bei Teilung durch
$10^k$ bekommt man, indem man nur die letzten $k$ Stellen der
Dezimaldarstellung der beiden Zahlen berücksichtigt. Wählt man
jetzt $k$ grösser als die Stellenzahl von $a$ und $b$ folgt,
dass alle Stellen von $a$ und $b$ übereinstimmen müssen.

\item $L_\infty$ ist nicht regulär. Wäre $L_\infty$ regulär,
müsste es nach dem Pumping Lemma eine Zahl $N$ geben, so dass
Wörter $w$ mit Länge $|w|\ge N$ ``aufgepumpt'' werden können.
Das Wort $w={\tt 0}^N{\tt 1}^N$ könnte also geschrieben werden
als $uxy=w$ mit $|ux|\le N$, $|x|>0$. Insbesondere dürfte $x$ nur
aus {\tt 0} bestehen. Nach dem Pumping Lemma müssten auch alle
Wörter $ux^ky\in L_\infty$ sein, aber
\begin{align*}
|ux^kv|_{\tt 0}
&=
|uxv|_{\tt 0}
+
(k-1) |x^k|_{\tt 0}
=
|w|_0+(k-1)|x|
\\
|ux^kv|_{\tt 1}
&=
|u|_{\tt 1} + k|x|_{\tt 1} + |v|_{\tt 1}
=
|v|_{\tt 1}
=|w|_{\tt 1}
\\
|ux^kv|_{\tt 0}
-
|ux^kv|_{\tt 1}
&=
(k-1)|x|>0.
\end{align*}
Die aufgepumpten Wörter haben also nicht mehr gleich viele
{\tt 0} und {\tt 1}. Dieser Widerspruch zeigt, dass $L_\infty$
nicht regulär sein kann.

Es reicht allerdings nicht zu argumentieren, $L=\{ 0^n1^n|n \ge 0\}$
sei in $L_\infty$ enthalten und bekanntermassen nicht regulär,
also müsse auch $L_\infty$ nicht regulär sein. Denn $L$ ist auch
in $\Sigma^*$ enhalten, welches offensichtlich regulär ist.
\qedhere
\end{teilaufgaben}
\end{loesung}


