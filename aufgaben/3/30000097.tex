Konstruieren Sie einen endlichen Automaten, der Wörter gerader Länge
über dem Alphabet $\Sigma=\{\texttt{0},\texttt{1}\}$ akzeptiert,
und bereiten Sie ihn auf die Umwandlung in einen regulären Ausdruck vor.

\begin{loesung}
Wir brauchen zwei Zustände $q_e$ und $q_o$ für Wörter gerader und
ungerader Länge.
Startzustand ist $q_e$ für Wörter gerader Länge, denn das leere Wort
$\varepsilon$ hat gerade Länge.
Jedes verarbeitete Zeichen wechselt die Parität:
\def\l{3}
\def\r{0.4}
\def\zustand#1#2#3{
        \draw[color=#3] #1 circle[radius=\r];
        \node[color=#3] at #1 {$#2\mathstrut$};
}
\def\akzeptierzustand#1#2#3{
        \zustand{#1}{#2}{#3}
        \draw[color=#3] #1 circle[radius={\r-0.05}];
}
\begin{center}
\begin{tikzpicture}[>=latex,thick]
\coordinate (qe) at (0,0);
\coordinate (qo) at (\l,0);
\draw[->,shorten >= 0.4cm] (-2,0) -- (qe);
\draw[->,shorten >= 0.4cm,shorten <= 0.4cm] (qo) to[out=-150,in=-30] (qe);
\node at ({0.5*\l},0.8) {$\texttt{0},\texttt{1}$};
\node at ({0.5*\l},-0.8) {$\texttt{0},\texttt{1}$};
\draw[->,shorten >= 0.4cm,shorten <= 0.4cm] (qe) to[out=30,in=150] (qo);
\akzeptierzustand{(qe)}{q_e}{black}
\zustand{(qo)}{q_o}{black}
\end{tikzpicture}
\end{center}
Dieser Automat muss jetzt umgewandelt werden in einen Automaten, genau
einen Start- und einen Akzeptierzustand hat, die nicht übereinstimmen,
und so, dass vom Startzustand nur Pfeile weggehen und im Akzeptierzustand
nur Pfeile ankommen.
Wir bezeichnen die neuen Zustände mit $S$ und $E$ und verbinden sie mit
$\varepsilon$-Übergängen mit dem bisherigen Automaten:
\begin{center}
\begin{tikzpicture}[>=latex,thick]
\coordinate (S) at (-\l,0);
\coordinate (qe) at (0,0);
\coordinate (E) at (\l,0);
\coordinate (qo) at (0,-\l);
\zustand{(S)}{S}{red}
\zustand{(qe)}{q_e}{black}
\zustand{(qo)}{q_o}{black}
\akzeptierzustand{(E)}{E}{red}
\draw[->,shorten >= 0.4cm,color=red] ($(S)+(-2,0)$) -- (S);
\draw[->,shorten >= 0.4cm,shorten <= 0.4cm,color=red] (S) -- (qe);
\draw[->,shorten >= 0.4cm,shorten <= 0.4cm,color=red] (qe) -- (E);
\draw[->,shorten >= 0.4cm,shorten <= 0.4cm] (qe) to[out=-60,in=60] (qo);
\draw[->,shorten >= 0.4cm,shorten <= 0.4cm] (qo) to[out=120,in=-120] (qe);
\node[color=red] at ($0.5*(S)+0.5*(qe)$) [above] {$\varepsilon$};
\node[color=red] at ($0.5*(E)+0.5*(qe)$) [above] {$\varepsilon$};
\node at (-0.5,{-0.5*\l}) [left] {$\texttt{0},\texttt{1}$};
\node at (0.5,{-0.5*\l}) [right] {$\texttt{0},\texttt{1}$};
\end{tikzpicture}
\end{center}
\end{loesung}

