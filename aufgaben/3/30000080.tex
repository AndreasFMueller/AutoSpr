\definecolor{darkgreen}{rgb}{0,0.6,0}%
Finden Sie den Minimalautomaten zum deterministischen endlichen Automaten
mit dem Zustandsdiagramm
\begin{center}
\def\l{1.6}
\def\s{0.3}
\def\S{0.4}
\def\zustand#1#2{
	\ifthenelse{\boolean{loesungen}}{
		\fill[color=#2!20] #1 circle[radius=0.3];
	}{
		\fill[color=white] #1 circle[radius=0.3];
	}
	\draw #1 circle[radius=0.3];
}
\def\akzeptierzustand#1#2{
	\ifthenelse{\boolean{loesungen}}{
		\fill[color=#2!20] #1 circle[radius=0.3];
	}{
		\fill[color=white] #1 circle[radius=0.3];
	}
	\draw #1 circle[radius=0.3];
	\draw #1 circle[radius=0.25];
}
\def\nullpfeil#1{
	\draw[->,shorten >= 0.3cm,shorten <= 0.3cm] #1 -- ($#1+(\l,0)$);
	\node at ($#1+({0.5*(\s+1)*\l},0)$) [above] {\small\texttt{0}};
}
\def\nulllang#1{
	\draw[->,shorten >= 0.3cm,shorten <= 0.3cm]
		#1
		-- 
		($#1+({\S*\l},0)$)
		to[out=0,in=0]
		($#1+({\S*\l},{-\l*\s})$)
		--
		($#1+({(-2-\S)*\l},{-\l*\s})$)
		to[out=180,in=180]
		($#1+({(-2-\S)*\l},0)$)
		--
		($#1+({-2*\l},0)$);
	\node at ($#1+({\S*\l+0.1},{-0.5*\s*\l})$) [right] {\small\texttt{0}};
}
\def\einspfeil#1{
	\draw[->,shorten >= 0.3cm,shorten <= 0.3cm] #1 -- ($#1+(0,-\l)$);
	\node at ($#1+(0,{-0.5*(\s+1)*\l})$) [left] {\small\texttt{1}};
}
\def\einslang#1{
	\draw[->,shorten >= 0.3cm,shorten <= 0.3cm]
		#1
		-- 
		($#1+(0,{-\S*\l})$)
		to[out=-90,in=-90]
		($#1+({\l*\s},{-\S*\l})$)
		--
		($#1+({\l*\s},{(2+\S)*\l})$)
		to[out=90,in=90]
		($#1+(0,{(2+\S)*\l})$)
		--
		($#1+(0,{2*\l})$);
	\node at ($#1+({0.5*\s*\l},{-\S*\l-0.1})$) [below] {\small\texttt{1}};
}
\begin{tikzpicture}[>=latex,thick]
\coordinate (A) at (-\l,-\l);
\coordinate (B) at (0,-\l);
\coordinate (C) at (\l,-\l);

\coordinate (D) at (-\l,0);
\coordinate (E) at (0,0);
\coordinate (F) at (\l,0);

\coordinate (G) at (-\l,\l);
\coordinate (H) at (0,\l);
\coordinate (I) at (\l,\l);

\draw[<-,shorten <= 0.3cm] (G) -- ++({-\l*0.5},{\l*0.5});

\akzeptierzustand{(A)}{darkgreen}
\zustand{(B)}{red}
\zustand{(C)}{blue}

\zustand{(D)}{blue}
\akzeptierzustand{(E)}{darkgreen}
\zustand{(F)}{red}

\zustand{(G)}{red}
\zustand{(H)}{blue}
\akzeptierzustand{(I)}{darkgreen}

\node at (A) {$6$};
\node at (B) {$7$};
\node at (C) {$8$};
\node at (D) {$3$};
\node at (E) {$4$};
\node at (F) {$5$};
\node at (G) {$0$};
\node at (H) {$1$};
\node at (I) {$2$};

\nullpfeil{(A)}
\nullpfeil{(B)}
\nulllang{(C)}
\nullpfeil{(D)}
\nullpfeil{(E)}
\nulllang{(F)}
\nullpfeil{(G)}
\nullpfeil{(H)}
\nulllang{(I)}

\einspfeil{(G)}
\einspfeil{(H)}
\einspfeil{(I)}
\einspfeil{(D)}
\einspfeil{(E)}
\einspfeil{(F)}
\einslang{(A)}
\einslang{(B)}
\einslang{(C)}

\end{tikzpicture}
\end{center}


\begin{loesung}
Lösung durch Anstarren: Vom Startzustand kann man mit genau zwei Zeichen
zu einem Akzeptierzustand kommen.
Mit genau drei weiteren Zeichen, unabhängig davon, ob es \texttt{0} oder
\texttt{1} ist, kann man wieder zu einem Akzeptierzustand kommen.
Der einfachste endliche Automat, der genau dies auch ausdrückt, ist
\begin{center}
\def\r{0.35}
\def\zustand#1#2{
	\ifthenelse{\boolean{loesungen}}{
		\fill[color=#2!20] #1 circle[radius=\r];
	}{
		\fill[color=white] #1 circle[radius=\r];
	}
	\draw #1 circle[radius=\r];
}
\def\akzeptierzustand#1#2{
	\ifthenelse{\boolean{loesungen}}{
		\fill[color=#2!20] #1 circle[radius=\r];
	}{
		\fill[color=white] #1 circle[radius=\r];
	}
	\draw #1 circle[radius=\r];
	\draw #1 circle[radius={\r-0.05}];
}
\def\pfeil#1{
	\draw[->,shorten >= 0.35cm,shorten <= 0.35cm] #1 -- ($#1+(2,0)$);
	\node at ($#1+(1,0)$) [above] {\texttt{0},\texttt{1}};
}
\begin{tikzpicture}[>=latex,thick]
\coordinate (S) at (-2,0);
\coordinate (A) at (0,0);
\coordinate (B) at (2,0);
\coordinate (C) at (4,0);
\zustand{(A)}{red}
\zustand{(B)}{blue}
\akzeptierzustand{(C)}{darkgreen}

\draw[->,shorten >= 0.35cm,shorten <= 0.35cm] (S) -- (A);
\pfeil{(A)}
\pfeil{(B)}
\draw[->,shorten >= 0.35cm,shorten <= 0.35cm] (C) to[out=-120,in=-60] (A);
\node at ($(B)+(0,-1)$) [below] {\texttt{0},\texttt{1}};

\end{tikzpicture}
\end{center}
Dies ist der der Minimalautomat, den man auch mit dem ``Tabellenalgorithmus''
finden können sollte.

Lösung mit der ``Kreuzchentabelle'': Die Starttabelle bezeichnet
Paare von Akzeptier- und Nichtakzeptierzuständen als nicht äquivalent:
\begin{center}
\def\e{\equiv}
\def\t{\times}
\def\o{\otimes}
\begin{tabular}{|>{$}c<{$}|>{$}c<{$}>{$}c<{$}>{$}c<{$}>{$}c<{$}>{$}c<{$}>{$}c<{$}>{$}c<{$}>{$}c<{$}>{$}c<{$}|}
\hline
  &0  &1  &2  &3  &4  &5  &6  &7  &8  \\
\hline
 0&\e &   &\t &   &\t &   &\t &   &   \\
 1&   &\e &\t &   &\t &   &\t &   &   \\
 2&\t &\t &\e &\t &   &\t &   &\t &\t \\
 3&   &   &\t &\e &\t &   &\t &   &   \\
 4&\t &\t &   &\t &\e &\t &   &\t &\t \\
 5&   &   &\t &   &\t &\e &\t &   &   \\
 6&\t &\t &   &\t &   &\t &\e &\t &\t \\
 7&   &   &\t &   &\t &   &\t &\e &   \\
 8&   &   &\t &   &\t &   &\t &   &\e \\
\hline
\end{tabular}
\end{center}
In der zweiten Generation sucht man Paare, die durch einen Übergang in ein
nicht äquivalentes Paar übergeführt werden:
\begin{center}
\def\e{\equiv}
\def\t{\times}
\def\o{\otimes}
\def\f#1{\raisebox{-0.14cm}{%
\begin{tikzpicture}
\fill[color=#1!20] (0,0) rectangle (0.4,0.4);
\node at (0.2,0.1) {\clap{\smash{$\equiv$}}};
\end{tikzpicture}}%
}
\begin{tabular}{|>{$}c<{$}|>{$}c<{$}>{$}c<{$}>{$}c<{$}>{$}c<{$}>{$}c<{$}>{$}c<{$}>{$}c<{$}>{$}c<{$}>{$}c<{$}|}
\hline
  &0  &1  &2  &3  &4  &5  &6  &7  &8  \\
\hline
 0&\e &\t &\o &\t &\o &\f{red}   &\o &\f{red}   &\t \\
 1&\t &\e &\o &\f{blue}   &\o &\t &\o &\t &\f{blue}   \\
 2&\o &\o &\e &\o &\f{darkgreen}   &\o &\f{darkgreen}   &\o &\o \\
 3&\t &\f{blue}   &\o &\e &\o &\t &\o &\t &\f{blue}   \\
 4&\o &\o &\f{darkgreen}   &\o &\e &\o &\f{darkgreen}   &\o &\o \\
 5&\f{red}   &\t &\o &\t &\o &\e &\o &\f{red}   &\t \\
 6&\o &\o &\f{darkgreen}   &\o &\f{darkgreen}   &\o &\e &\o &\o \\
 7&\f{red}   &\t &\o &\t &\o &\f{red}   &\o &\e &\t \\
 8&\t &\f{blue}   &\o &\f{blue}   &\o &\t &\o &\t &\e \\
\hline
\end{tabular}
\end{center}
Beim nächsten Durchgang können keine neuen Kreuzchen mehr eingetragen
werden, die leeren Felder bedeuten also, dass diese Zustände zusammengelegt
werden können.
Damit haben wir den Minimalautomaten gefunden:
\begin{center}
\def\r{0.35}
\def\zustand#1#2{
	\fill[color=#2!20] #1 circle[radius=\r];
	\draw #1 circle[radius=\r];
}
\def\akzeptierzustand#1#2{
	\fill[color=#2!20] #1 circle[radius=\r];
	\draw #1 circle[radius=\r];
	\draw #1 circle[radius={\r-0.05}];
}
\def\pfeil#1{
	\draw[->,shorten >= 0.35cm,shorten <= 0.35cm] #1 -- ($#1+(2,0)$);
	\node at ($#1+(1,0)$) [above] {\texttt{0},\texttt{1}};
}
\begin{tikzpicture}[>=latex,thick]
\coordinate (S) at (-2,0);
\coordinate (A) at (0,0);
\coordinate (B) at (2,0);
\coordinate (C) at (4,0);
\zustand{(A)}{red}
\zustand{(B)}{blue}
\akzeptierzustand{(C)}{darkgreen}

\draw[->,shorten >= 0.35cm,shorten <= 0.35cm] (S) -- (A);
\pfeil{(A)}
\pfeil{(B)}
\draw[->,shorten >= 0.35cm,shorten <= 0.35cm] (C) to[out=-120,in=-60] (A);
\node at ($(B)+(0,-1)$) [below] {\texttt{0},\texttt{1}};

\end{tikzpicture}
\end{center}
\end{loesung}
