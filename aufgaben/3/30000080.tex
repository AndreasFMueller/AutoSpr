Finden Sie den Minimalautomaten zum deterministischen endlichen Automaten
mit dem Zustandsdiagramm
\begin{center}
\def\l{1.6}
\def\s{0.3}
\def\S{0.4}
\def\zustand#1{
	\fill[color=white] #1 circle[radius=0.3];
	\draw #1 circle[radius=0.3];
}
\def\akzeptierzustand#1{
	\fill[color=white] #1 circle[radius=0.3];
	\draw #1 circle[radius=0.3];
	\draw #1 circle[radius=0.25];
}
\def\nullpfeil#1{
	\draw[->,shorten >= 0.3cm,shorten <= 0.3cm] #1 -- ($#1+(\l,0)$);
	\node at ($#1+({0.5*(\s+1)*\l},0)$) [above] {\small\texttt{0}};
}
\def\nulllang#1{
	\draw[->,shorten >= 0.3cm,shorten <= 0.3cm]
		#1
		-- 
		($#1+({\S*\l},0)$)
		to[out=0,in=0]
		($#1+({\S*\l},{-\l*\s})$)
		--
		($#1+({(-2-\S)*\l},{-\l*\s})$)
		to[out=180,in=180]
		($#1+({(-2-\S)*\l},0)$)
		--
		($#1+({-2*\l},0)$);
	\node at ($#1+({\S*\l+0.1},{-0.5*\s*\l})$) [right] {\small\texttt{0}};
}
\def\einspfeil#1{
	\draw[->,shorten >= 0.3cm,shorten <= 0.3cm] #1 -- ($#1+(0,-\l)$);
	\node at ($#1+(0,{-0.5*(\s+1)*\l})$) [left] {\small\texttt{1}};
}
\def\einslang#1{
	\draw[->,shorten >= 0.3cm,shorten <= 0.3cm]
		#1
		-- 
		($#1+(0,{-\S*\l})$)
		to[out=-90,in=-90]
		($#1+({\l*\s},{-\S*\l})$)
		--
		($#1+({\l*\s},{(2+\S)*\l})$)
		to[out=90,in=90]
		($#1+(0,{(2+\S)*\l})$)
		--
		($#1+(0,{2*\l})$);
	\node at ($#1+({0.5*\s*\l},{-\S*\l-0.1})$) [below] {\small\texttt{1}};
}
\begin{tikzpicture}[>=latex,thick]
\coordinate (A) at (-\l,-\l);
\coordinate (B) at (0,-\l);
\coordinate (C) at (\l,-\l);

\coordinate (D) at (-\l,0);
\coordinate (E) at (0,0);
\coordinate (F) at (\l,0);

\coordinate (G) at (-\l,\l);
\coordinate (H) at (0,\l);
\coordinate (I) at (\l,\l);

\draw[<-,shorten <= 0.3cm] (G) -- ++({-\l*0.5},{\l*0.5});

\akzeptierzustand{(A)}
\zustand{(B)}
\zustand{(C)}

\zustand{(D)}
\akzeptierzustand{(E)}
\zustand{(F)}

\zustand{(G)}
\zustand{(H)}
\akzeptierzustand{(I)}

\node at (A) {$0$};
\node at (B) {$1$};
\node at (C) {$2$};
\node at (D) {$3$};
\node at (E) {$4$};
\node at (F) {$5$};
\node at (G) {$6$};
\node at (H) {$7$};
\node at (I) {$8$};

\nullpfeil{(A)}
\nullpfeil{(B)}
\nulllang{(C)}
\nullpfeil{(D)}
\nullpfeil{(E)}
\nulllang{(F)}
\nullpfeil{(G)}
\nullpfeil{(H)}
\nulllang{(I)}

\einspfeil{(G)}
\einspfeil{(H)}
\einspfeil{(I)}
\einspfeil{(D)}
\einspfeil{(E)}
\einspfeil{(F)}
\einslang{(A)}
\einslang{(B)}
\einslang{(C)}

\end{tikzpicture}
\end{center}


\begin{loesung}
\end{loesung}
