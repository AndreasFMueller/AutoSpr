Betrachten Sie die Sprache
\[
L
=
\{
wxw^t
\mid
w,x\in\Sigma^*
\wedge
|w|\le 2
\}
\]
über dem Alphabet $\Sigma=\{\texttt{a},\texttt{b}\}$.
Das Wort $w^t$ ist die gespiegelte Version des Wortes $w$.
Ist die Sprache $L$ regulär?

\themaL{regular}{regulär}

\begin{loesung}
Die schnellste Begründung ist, dass $w=\varepsilon$ sein darf und daher
$w=x\in\Sigma^*$ sein kann.
Es folgt $L=\Sigma^*$, diese Sprache ist regulär.

Wenn man das nicht unmittelbar sieht, und das betraff die meisten
Prüfungsteilnehmer, dann kann man einen NEA konstruieren, der genau
die Bedingungen in der Sprachedefinition implementiert.
Der folgende NEA akzeptiert die Sprache $L$.
\begin{center}
\begin{tikzpicture}[>=latex,thick]
\def\h{0.7}

\coordinate (S) at (-4.5,0);
\coordinate (A) at (-3,0);

\coordinate (B1) at (-1.5,{-4*\h});
\coordinate (B2) at (-1.5,{4*\h});

\coordinate (C0) at (0,{-6*\h});
\coordinate (C1) at (0,{-4*\h});
\coordinate (C2) at (0,{-2.5*\h});
\coordinate (C3) at (0,0);
\coordinate (C4) at (0,{2.5*\h});
\coordinate (C5) at (0,{4*\h});
\coordinate (C6) at (0,{6*\h});

\coordinate (D1) at (1.5,{-4*\h});
\coordinate (D2) at (1.5,{4*\h});

\coordinate (E) at (3,0);

\draw (A) circle[radius=0.3];

\draw (B1) circle[radius=0.3];
\draw (B2) circle[radius=0.3];

\draw (C0) circle[radius=0.3];
\draw (C1) circle[radius=0.3];
\draw (C2) circle[radius=0.3];
\draw (C3) circle[radius=0.3];
\draw (C4) circle[radius=0.3];
\draw (C5) circle[radius=0.3];
\draw (C6) circle[radius=0.3];

\draw (D1) circle[radius=0.3];
\draw (D2) circle[radius=0.3];

\draw (E) circle[radius=0.3];
\draw (E) circle[radius=0.25];

\draw[->,shorten >= 0.3cm,shorten <= 0.3cm] (S) -- (A);

\draw[->,shorten >= 0.3cm,shorten <= 0.3cm] (A) -- (B1);
\draw[->,shorten >= 0.3cm,shorten <= 0.3cm] (A) -- (B2);
\draw[->,shorten >= 0.3cm,shorten <= 0.3cm] (A) -- (C3);

\draw[->,shorten >= 0.3cm,shorten <= 0.3cm] (B1) -- (C0);
\draw[->,shorten >= 0.3cm,shorten <= 0.3cm] (B1) -- (C1);
\draw[->,shorten >= 0.3cm,shorten <= 0.3cm] (B1) -- (C2);
\draw[->,shorten >= 0.3cm,shorten <= 0.3cm] (B2) -- (C4);
\draw[->,shorten >= 0.3cm,shorten <= 0.3cm] (B2) -- (C5);
\draw[->,shorten >= 0.3cm,shorten <= 0.3cm] (B2) -- (C6);

\draw[->,shorten >= 0.3cm,shorten <= 0.3cm] (C0) -- (D1);
\draw[->,shorten >= 0.3cm,shorten <= 0.3cm] (C1) -- (D1);
\draw[->,shorten >= 0.3cm,shorten <= 0.3cm] (C2) -- (D1);
\draw[->,shorten >= 0.3cm,shorten <= 0.3cm] (C4) -- (D2);
\draw[->,shorten >= 0.3cm,shorten <= 0.3cm] (C5) -- (D2);
\draw[->,shorten >= 0.3cm,shorten <= 0.3cm] (C6) -- (D2);

\draw[->,shorten >= 0.3cm,shorten <= 0.3cm] (C3) -- (E);
\draw[->,shorten >= 0.3cm,shorten <= 0.3cm] (D1) -- (E);
\draw[->,shorten >= 0.3cm,shorten <= 0.3cm] (D2) -- (E);

\node at ($0.5*(A)+0.5*(B1)$) [left] {\texttt{a}};
\node at ($0.5*(A)+0.5*(B2)$) [left] {\texttt{b}};


\node at ($0.5*(B1)+0.5*(C0)$) [below left] {\texttt{a}};
\node at ($0.5*(B1)+0.5*(C2)$) [above left] {\texttt{b}};
\node at ($0.5*(B2)+0.5*(C4)$) [below left] {\texttt{a}};
\node at ($0.5*(B2)+0.5*(C6)$) [above left] {\texttt{b}};

\node at ($0.5*(D1)+0.5*(C0)$) [below right] {\texttt{a}};
\node at ($0.5*(D1)+0.5*(C2)$) [above right] {\texttt{b}};
\node at ($0.5*(D2)+0.5*(C4)$) [below right] {\texttt{a}};
\node at ($0.5*(D2)+0.5*(C6)$) [above right] {\texttt{b}};

\node at ($0.5*(E)+0.5*(D1)$) [right] {\texttt{a}};
\node at ($0.5*(E)+0.5*(D2)$) [right] {\texttt{b}};

\node at ($0.5*(A)+0.5*(C3)$) [above] {$\varepsilon$};
\node at ($0.5*(E)+0.5*(C3)$) [above] {$\varepsilon$};

\node at ($0.5*(B1)+0.5*(C1)$) [above] {$\varepsilon$};
\node at ($0.5*(D1)+0.5*(C1)$) [above] {$\varepsilon$};
\node at ($0.5*(B2)+0.5*(C5)$) [above] {$\varepsilon$};
\node at ($0.5*(D2)+0.5*(C5)$) [above] {$\varepsilon$};

\draw[->,shorten >= 0.3cm,shorten <= 0.3cm]
	(C0) to[out=-60,in=-120,distance=0.8cm] (C0);
\node at ($(C0)+(0,-0.5)$) [below] {$*$};

\draw[->,shorten >= 0.3cm,shorten <= 0.3cm]
	(C1) to[out=-60,in=-120,distance=0.8cm] (C1);
\node at ($(C1)+(0,-0.5)$) [below] {$*$};

\draw[->,shorten >= 0.3cm,shorten <= 0.3cm]
	(C2) to[out=60,in=120,distance=0.8cm] (C2);
\node at ($(C2)+(0,0.5)$) [above] {$*$};

\draw[->,shorten >= 0.3cm,shorten <= 0.3cm]
	(C3) to[out=60,in=120,distance=0.8cm] (C3);
\node at ($(C3)+(0,0.5)$) [above] {$*$};

\draw[->,shorten >= 0.3cm,shorten <= 0.3cm]
	(C4) to[out=-60,in=-120,distance=0.8cm] (C4);
\node at ($(C4)+(0,-0.5)$) [below] {$*$};

\draw[->,shorten >= 0.3cm,shorten <= 0.3cm]
	(C5) to[out=60,in=120,distance=0.8cm] (C5);
\node at ($(C5)+(0,0.5)$) [above] {$*$};

\draw[->,shorten >= 0.3cm,shorten <= 0.3cm]
	(C6) to[out=60,in=120,distance=0.8cm] (C6);
\node at ($(C6)+(0,0.5)$) [above] {$*$};

\end{tikzpicture}
\end{center}
Es gibt genau $7$ Wörter $w$ der Länge $\le 2$, die Zustände in der
vertikalen Symmetrieachse des Automaten repräsentieren diese Wörter.
Die Schleifen bei den Zuständen stellen den Teil $x$ des Wortes dar.

Die horizontale Achse des Diagramms zeigt aber auch, dass jedes beliebige
Wort akzeptiert werden kann.
Es ist also $L=\Sigma^*$, und diese Sprache ist auch regulär.

Man kann auch direkt einen regulären Ausdruck angeben
\[
\def\x{\texttt{.*}}
r=
\texttt{aa}\x\texttt{aa|ab}\x\texttt{ba|ba}\x\texttt{ab|bb}\x\texttt{bb|a}\x\texttt{a|b}\x\texttt{b}|\x
\]
Dies zeigt ebenfalls, dass die Sprache regulär ist.
\end{loesung}

\begin{bewertung}
Endlicher Automat ({\bf E}) 1 Punkt,
syntaktisch korrektes Zustandsdiagramm ({\bf D}) 1 Punkt,
erstes und letztes Zeichen gleich ({\bf G}) 1 Punkt,
zweites und zweitletzes Zeichen gleich ({\bf Z}) 1 Punkt,
beliebig langes Zwischenstück $x$ ({\bf X}) 1 Punkt,
Schlussfolgerung regulär ({\bf R}) 1 Punkt.
\end{bewertung}
