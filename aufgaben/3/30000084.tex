Verwendeln Sie den Thompson-NEA um den folgenden NEA über dem Alphabet
$\Sigma=\{\texttt{0},\texttt{1}\}$ in einen DEA
umzuwandeln.
\begin{center}
\def\l{2}
\def\r{0.35}
\def\punkt#1#2{({(#1)*\l},{(#2)*\l})}
\def\zustand#1#2{
	\fill[color=white] #1 circle[radius=\r];
	\draw #1 circle[radius=\r];
	\node at #1 {$#2\mathstrut$};
}
\def\akzeptierzustand#1#2{
	\zustand{#1}{#2}
	\draw #1 circle[radius={\r-0.05}];
}
\def\pfeil#1#2{
	\draw[->,shorten >= 0.35cm,shorten <= 0.35cm] #1 -- #2;
}
\begin{tikzpicture}[>=latex,thick]
\coordinate (q0) at (0,0);
\coordinate (q1) at (\l,0);
\zustand{(q0)}{q_0}
\akzeptierzustand{(q1)}{q_1}
%\pfeil{(q0)}{(q1)}
\draw[->,shorten >= 0.35cm, shorten <= 0.35cm] 
	(q0) to[out=20,in=160] (q1);
\draw[->,shorten >= 0.35cm, shorten <= 0.35cm] 
	(q1) to[out=-160,in=-20] (q0);
\pfeil{(-1.5,0)}{(q0)}
\draw[->,shorten >= 0.35cm,shorten <= 0.35cm]
	(q0) to[out=60,in=120,distance=1.2cm] (q0);
\node at ($(q0)+(0,0.8)$) [above] {\texttt{1}};
\draw[->,shorten >= 0.35cm,shorten <= 0.35cm]
	(q1) to[out=60,in=120,distance=1.2cm] (q1);
\node at ($(q1)+(0,0.8)$) [above] {$\Sigma$};
\node at ($0.5*(q0)+0.5*(q1)+(0,0.2)$) [above] {\texttt{1}};
\node at ($0.5*(q0)+0.5*(q1)+(0,-0.2)$) [below] {\texttt{0}};
\end{tikzpicture}
\end{center}
Zeichnen Sie auch die nicht erreichbaren Zustände und Übergänge.

\begin{loesung}
Da der NEA zwei Zustände $Q=\{q_0,q_1\}$ hat, hat der DEA höchstens
die vier Zustände
\[
Q' = P(Q) = \{ \emptyset, \{q_0\}, \{q_1\}, Q \}.
\]
Das Zustandsdiagramm ist
\begin{center}
\def\l{1.5}
\def\r{0.55}
\def\zustand#1#2{
	\fill[color=white] #1 circle[radius=\r];
	\draw #1 circle[radius=\r];
	\node at #1 {$#2\mathstrut$};
}
\def\akzeptierzustand#1#2{
	\zustand{#1}{#2}
	\draw #1 circle[radius={\r-0.05}];
}
\def\pfeil#1#2{
	\draw[->,shorten >= 0.55cm,shorten <= 0.55cm] #1 -- #2;
}
\begin{tikzpicture}[>=latex,thick]
\coordinate (e) at (0,-\l);
\coordinate (q0) at (-\l,0);
\coordinate (q1) at (\l,0);
\coordinate (q01) at (0,\l);
\zustand{(e)}{\emptyset}
\zustand{(q0)}{q_0}
\akzeptierzustand{(q01)}{q_0,q_1}
\pfeil{(-4,0)}{(q0)}
\pfeil{(q0)}{(q01)}
\node at ($0.5*(q0)+0.5*(q01)$) [above left] {\texttt{1}};
\pfeil{(q0)}{(e)}
\node at ($0.5*(q0)+0.5*(e)$) [below left] {\texttt{0}};

\begin{scope}
\color{gray}
\akzeptierzustand{(q1)}{q_1}
\pfeil{(q1)}{(q01)}
\node at ($0.5*(q1)+0.5*(q01)$) [below left] {\texttt{0}};
\draw[->,shorten >= 0.55cm,shorten <= 0.55cm]
	(q1) to[out=-30,in=30,distance=1.5cm] (q1);
\node at ($(q1)+(1,0)$) [right] {\texttt{1}};
\end{scope}

\draw[->,shorten >= 0.55cm,shorten <= 0.55cm]
	(e) to[out=-30,in=30,distance=1.5cm] (e);
\node at ($(e)+(1,0)$) [right] {$\Sigma$};
\draw[->,shorten >= 0.55cm,shorten <= 0.55cm]
	(q01) to[out=-30,in=30,distance=1.5cm] (q01);
\node at ($(q01)+(1,0)$) [right] {\texttt{1}};
\end{tikzpicture}
\end{center}
Der Zustand $\{q_1\}$ ist nicht erreichbar und ist daher zusammen mit
seinen Übergängen grau gezeichnet.
\end{loesung}
