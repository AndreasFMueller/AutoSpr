Finden Sie die Zustände eines endlichen Automaten für die Sprache
\[
L
=
\{
\texttt{0}^n\texttt{10}
\mid
n>0
\}
\]
über dem Alphabet $\Sigma=\{\texttt{0},\texttt{1}\}$ mit der Methode
von Myhill-Nerode.

\begin{loesung}
Wir tragen die möglichen Zustandsmengen für verschiedene Anfangswörter
zusammen
\begin{center}
\begin{tabular}{|>{$}l<{$}|>{$}l<{$}|>{$}l<{$}|}
\hline
w            & L(w)                                   & Q   \\
\hline
\varepsilon  & L(\varepsilon)  = L                    & q_0 \\
\texttt{0}   & L(\texttt{0})   = L\cup\{\texttt{10}\} & q_1 \\
\texttt{1}   & L(\texttt{1})   = \emptyset            & e   \\
\texttt{00}  & L(\texttt{00})  = L\cup\{\texttt{10}\} & q_1 \\
\texttt{01}  & L(\texttt{01})  = \{ \texttt{0} \}     & q_2 \\
\texttt{10}  & L(\texttt{10})  = \emptyset            & e   \\
\texttt{11}  & L(\texttt{11})  = \emptyset            & e   \\
\texttt{000} & L(\texttt{000}) = L\cup\{\texttt{10}\} & q_1 \\
\texttt{001} & L(\texttt{001}) = \{ \texttt{0} \}     & q_2 \\
\texttt{010} & L(\texttt{010}) = \{ \varepsilon \}    & q_3 \\
\texttt{011} & L(\texttt{011}) = \emptyset            & e   \\
\texttt{100} & L(\texttt{100}) = \emptyset            & e   \\
\texttt{101} & L(\texttt{101}) = \emptyset            & e   \\
\texttt{110} & L(\texttt{110}) = \emptyset            & e   \\
\texttt{111} & L(\texttt{111}) = \emptyset            & e   \\
\dots        & \dots                                  &\dots\\
\hline
\end{tabular}
\end{center}
Da vier verschiedene mögliche Mengen $L(w)$ auftreten, kann man
daraus ablesen, dass es vier Zustände braucht.
Das Zustandsdiagramm wird
\begin{center}
\begin{tikzpicture}[>=latex,thick]
\coordinate (q0) at (0,0);
\coordinate (q1) at (3,0);
\coordinate (q2) at (6,0);
\coordinate (q3) at (9,0);
\coordinate (e) at (4.5,-3);
\def\pfeil#1#2{
	\draw[->,shorten >= 0.5cm,shorten <= 0.5cm] #1 -- #2;
}
\def\zustand#1#2{
	\draw #1 circle[radius=0.5];
	\node at #1 {$#2$};
}
\def\akzeptierzustand#1#2{
	\draw #1 circle[radius=0.5];
	\draw #1 circle[radius=0.45];
	\node at #1 {$#2$};
}
\zustand{(q0)}{q_0}
\zustand{(q1)}{q_1}
\zustand{(q2)}{q_2}
\akzeptierzustand{(q3)}{q_3}
\zustand{(e)}{e}
\pfeil{(-2,0)}{(q0)}
\pfeil{(q0)}{(q1)}
\pfeil{(q1)}{(q2)}
\pfeil{(q2)}{(q3)}
\pfeil{(q0)}{(e)}
\pfeil{(q2)}{(e)}
\pfeil{(q3)}{(e)}
\node at ($0.5*(q3)+0.5*(e)$) [below right] {$\texttt{1}$};
\node at ($0.5*(q0)+0.5*(e)$) [below left] {$\texttt{1}$};
\node at ($0.5*(q2)+0.5*(e)$) [left] {$\texttt{1}$};
\node at ($0.5*(q0)+0.5*(q1)$) [above] {$\texttt{0}$};
\node at ($0.5*(q1)+0.5*(q2)$) [above] {$\texttt{1}$};
\node at ($0.5*(q2)+0.5*(q3)$) [above] {$\texttt{0}$};
\draw[->,shorten >= 0.5cm,shorten <= 0.5cm]
	(q1) to[out=45,in=135,distance=2cm] (q1);
\node at ($(q1)+(0,1.2)$) [above] {\texttt{0}};
\draw[->,shorten >= 0.5cm,shorten <= 0.5cm]
	(e) to[out=-45,in=-135,distance=2cm] (e);
\node at ($(e)+(0,-1.2)$) [below] {$\Sigma$};
\end{tikzpicture}
\end{center}
\end{loesung}
