Betrachten Sie eine Sprache $L$ über dem Alphabet $\Sigma=\{\texttt{1}\}$.
Die Wörter sind durch ihre Länge vollständig beschrieben, sie können also
der Länge nach sortiert werden.
Zusätzlich sei bekannt, dass jedes Wort $w\in L$ weniger als halb so lang
ist wie das nächstlängere Wort in $L$.
Für das längere Wort $w_1$ gilt also $|w| < \frac12|w_1|$ oder gleichbedeutend 
$2|w| < |w_1|$.
Warum ist diese Sprache nicht regulär?

\begin{hinweis}
Überlegen Sie, wie sich die Länge eines Wortes beim Aufpumpen ändert.
\end{hinweis}

\themaL{Pumping Lemma fur regulare Sprachen}{Pumping Lemma für reguläre Sprachen}
\themaL{regular}{regulär}

\begin{loesung}
Nein, wie man mit dem Pumping-Lemma zeigen kann.
\begin{enumerate}
\item
Wir nehmen an, die Sprache $L$ sei regulär.
\item
Nach dem Pumping-Lemma gibt es die Pumping Length $N$
\item
Wir wählen das kürzeste Wort $w\in L$, welches Länge $|w|\ge N$ hat.
\item
Nach dem Pumping-Lemma lässt sich das Wort in drei Teile $xyz$ aufteilen,
die natürlich alle aus \texttt{1} besteht.
\item
Beim Pumpen wird das Wort um den Teil $y$ länger, also um eine Länge
$|y|\le N$.
Die Länge des gepumpten Wortes ist daher $|w|+|y|\le 2|w|$.
In der Beschreibung des Wortes steht aber auch, dass das gepumpte Wort
eine Länge $>2|w|$ haben muss, ein Widerspruch.
\item
Dieser Widerspruch zeigt, dass die Sprache $L$ nicht regulär sein kann.
\qedhere
\end{enumerate}
\end{loesung}

\begin{bewertung}
6 Schritte des Pumping-Lemma-Beweises:
Annahme ({\bf A}) 1 Punkt,
Pumping Length ({\bf N}) 1 Punkt,
Wort ({\bf W}) 1 Punkt,
Unterteilung ({\bf U}) 1 Punkt,
Widerspruch beim Pumpen ({\bf P}) 1 Punkt,
Folgerung ({\bf F}) 1 Punkt.
\end{bewertung}
