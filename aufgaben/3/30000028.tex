Zeigen Sie, dass die Sprache
\[
L=\{
\texttt{0}^m\texttt{1}^n\mid m\ne n
\}
\]
über dem Alphabet $\Sigma=\{\texttt{0},\texttt{1}\}$ nicht regulär ist.

\themaL{Mengenoperationen fur regulare Sprachen}{Mengenoperationen für reguläre Sprachen}
\thema{Zustandsdiagramm}
\thema{DEA}
\themaL{Pumping Lemma fur regulare Sprachen}{Pumping Lemma für reguläre Sprachen}

\begin{loesung}
Der einfachste Beweis verwendet die Mengenoperationen:
Die Sprache
\[
L'=\{\texttt{0}^n\texttt{1}^m\,|\,n,m\ge 0\}
\]
ist regulär, denn sie wird von dem Automaten
\begin{center}
\begin{tikzpicture}[>=latex,thick]
	\coordinate (s) at (-2,0);
	\coordinate (z0) at (0,0);
	\coordinate (z1) at (2,0);
	\coordinate (z2) at (4,0);
	\node at (z0) {$z_0$};
	\node at (z1) {$z_1$};
	\node at (z2) {$z_2$};
	\draw (z0) circle[radius=0.4];
	\draw (z0) circle[radius=0.35];
	\draw (z1) circle[radius=0.4];
	\draw (z1) circle[radius=0.35];
	\draw (z2) circle[radius=0.4];
	\draw[->,shorten >= 0.4cm,shorten <= 0.4cm] (s) -- (z0);
	\draw[->,shorten >= 0.4cm,shorten <= 0.4cm] (z0) -- (z1);
	\draw[->,shorten >= 0.4cm,shorten <= 0.4cm] (z1) -- (z2);
	\draw[->,shorten >= 0.4cm,shorten <= 0.4cm]
		(z0) to[out=60,in=120,distance=1.4cm] (z0);
	\draw[->,shorten >= 0.4cm,shorten <= 0.4cm]
		(z1) to[out=60,in=120,distance=1.4cm] (z1);
	\draw[->,shorten >= 0.4cm,shorten <= 0.4cm]
		(z2) to[out=30,in=-30,distance=1.4cm] (z2);
	\node at ($0.5*(z0)+0.5*(z1)$) [above] {\texttt{1}};
	\node at ($0.5*(z1)+0.5*(z2)$) [above] {\texttt{0}};
	\node at ($(z0)+(0,1.3)$) {\texttt{0}};
	\node at ($(z1)+(0,1.3)$) {\texttt{1}};
	\node at ($(z2)+(1.5,0)$) {\texttt{0},\texttt{1}};
\end{tikzpicture}
\end{center}
akzeptiert.
Wäre $L$ regulär, müsste auch $L'\setminus L$ regulär sein.
Doch $L'\setminus L$ ist die Sprache
\[
L'\setminus L
=
\{\texttt{0}^n\texttt{1}^n\,|\,n\ge 0\},
\]
von der in der Vorlesung gezeigt wurde, dass sie nicht regulär
ist. Dieser Widerspruch zeigt, dass auch $L$ nicht regulär sein kann.

Man kann die Behauptung aber auch direkt mit dem Pumping-Lemma beweisen.
\ding{182}
Wir nehmen also an, die Sprache sei regulär.
\ding{183}
Sei $N$ die Pumping Length.
Jedes genügend lange Wort $x$ der Sprache kann dann geschrieben werden als $uvw$,
wobei $|uv|\le N$, und alle Wörter $uv^nw\in L$, für alle $n\ge 1$.
Dem Teilwort $v$ entspricht eine Schleife im endlichen Automaten, der
$L$ akzeptiert.

\ding{184}
Wir wählen jetzt ein Wort $x=\texttt{0}^N\texttt{1}^{N+b}$.
Es hat Länge $2N+b$ und ist für $b>0$ in $L$, es erfüllt also die
Voraussetzungen des Pumping Lemma.
\ding{185}
Also kann es als $x=uvw$ geschrieben werden, wobei dem Teilwort $v$ eine
Schleife im Graphen des Automaten entspricht.
Das Wort $v$ enthält nur Nullen, da $|uv|\le N$ sein muss.
Vergrössert man $b$, ist der erste Teil des Wortes davon unberührt,
es gilt also immer noch die gleiche Aufteilung in $u$ und $v$,
$w$ wird um einige Einsen länger.

\ding{186}
Wir versuchen jetzt das Wort genau einmal aufzupumpen, also $uv^ky$ mit
$k=2$ zu bilden.
Wir möchten erreichen, dass dieses Wort nicht mehr in $L$ ist.
Da durch das Aufpumpten genau so viele \texttt{0} hinzukommen wie das
Teilwort $v$ lang ist, das Wort $uv^2w$ beginnt also mit $N+|v|$ Nullen.
Damit $uv^2w$ nicht mehr in der Sprache ist, müssen gleich viele Nullen
wie Einsen vorhanden sein, es muss also $N+|v|=N+b$ gelten.
\begin{center}
\def\h{0.7}
\definecolor{darkred}{rgb}{0.8,0,0}
\begin{tikzpicture}[>=latex,thick]
\begin{scope}
	\draw (0,0) rectangle ({19*\h},\h);
	\draw ({8*\h},0) -- ++(0,\h);
	\draw[line width=0.3pt] ({16*\h},0) -- ++(0,\h);
	\node at ({4*\h},{0.5*\h}) {\texttt{0}};
	\node at ({13.5*\h},{0.5*\h}) {\texttt{1}};
	\node at ({8*\h},\h) [above] {$N\mathstrut$};
	\node at ({16*\h},\h) [above] {$2N\mathstrut$};
	\node at ({19*\h},\h) [above] {$2N+b\mathstrut$};
\end{scope}
\begin{scope}[yshift=-1.8cm]
	\draw (0,0) rectangle ({19*\h},\h);
	\draw ({8*\h},0) -- ++(0,\h);
	\draw[line width=0.3pt] ({16*\h},0) -- ++(0,\h);
	\fill[color=darkgreen!30,opacity=0.5] ({0.07*\h},{0.07*\h})
		rectangle ({1.965*\h},{0.93*\h});
	\draw[color=darkgreen] ({0.07*\h},{0.07*\h})
		rectangle ({1.965*\h},{0.93*\h});
	\node[color=darkgreen] at ({\h},{0.5*\h}) {$u\mathstrut$};
	\fill[color=darkred!30,opacity=0.5] ({2.035*\h},{0.07*\h})
		rectangle ({4.965*\h},{0.93*\h});
	\draw[color=darkred] ({2.035*\h},{0.07*\h})
		rectangle ({4.965*\h},{0.93*\h});
	\node[color=darkred] at ({3.5*\h},{0.5*\h}) {$v\mathstrut$};
	\fill[color=blue!30,opacity=0.5] ({5.035*\h},{0.07*\h})
		rectangle ({18.93*\h},{0.93*\h});
	\draw[color=blue] ({5.035*\h},{0.07*\h})
		rectangle ({18.93*\h},{0.93*\h});
	\node[color=blue] at ({12*\h},{0.5*\h}) {$w\mathstrut$};
	\node[color=gray!40] at ({4*\h},{0.5*\h}) {\texttt{0}};
	\node[color=gray!20] at ({13.5*\h},{0.5*\h}) {\texttt{1}};
	\node at ({8*\h},\h) [above] {$N\mathstrut$};
	\node at ({16*\h},\h) [above] {$2N\mathstrut$};
	\node at ({19*\h},\h) [above] {$2N+b\mathstrut$};
\end{scope}
\begin{scope}[yshift=-3.6cm]
	\draw (0,0) rectangle ({22*\h},\h);
	\draw ({11*\h},0) -- ++(0,\h);
	\draw[line width=0.3pt] ({19*\h},0) -- ++(0,\h);
	\draw[line width=0.2pt] (0,0) -- ++(0,-0.9);
	\draw[line width=0.2pt] ({11*\h},0) -- ++(0,-0.9);
	\draw[line width=0.2pt] ({22*\h},0) -- ++(0,-0.9);
	\draw[<->,line width=0.7pt] (0,-0.8) -- ++({11*\h},0);
	\node at ({5.5*\h},-0.9) [above] {$N+|v|\mathstrut$};
	\draw[<->,line width=0.7pt] ({11*\h},-0.8) -- ++({11*\h},0);
	\node at ({16.5*\h},-0.9) [above] {$N+b\mathstrut$};
	\fill[color=darkgreen!30,opacity=0.5] ({0.07*\h},{0.07*\h})
		rectangle ({1.965*\h},{0.93*\h});
	\draw[color=darkgreen] ({0.07*\h},{0.07*\h})
		rectangle ({1.965*\h},{0.93*\h});
	\node[color=darkgreen] at ({\h},{0.5*\h}) {$u\mathstrut$};
	\fill[color=darkred!30,opacity=0.5] ({2.035*\h},{0.07*\h})
		rectangle ({4.965*\h},{0.93*\h});
	\draw[color=darkred] ({2.035*\h},{0.07*\h})
		rectangle ({4.965*\h},{0.93*\h});
	\node[color=darkred] at ({3.5*\h},{0.5*\h}) {$v\mathstrut$};
	\fill[color=darkred!30,opacity=0.5] ({5.035*\h},{0.07*\h})
		rectangle ({7.965*\h},{0.93*\h});
	\draw[color=darkred] ({5.035*\h},{0.07*\h})
		rectangle ({7.965*\h},{0.93*\h});
	\node[color=darkred] at ({6.5*\h},{0.5*\h}) {$v\mathstrut$};
	\fill[color=blue!30,opacity=0.5] ({8.035*\h},{0.07*\h})
		rectangle ({21.93*\h},{0.93*\h});
	\draw[color=blue] ({8.035*\h},{0.07*\h})
		rectangle ({21.93*\h},{0.93*\h});
	\node[color=blue] at ({15*\h},{0.5*\h}) {$w\mathstrut$};
	\node[color=gray!40] at ({5.5*\h},{0.5*\h}) {\texttt{0}};
	\node[color=gray!20] at ({16.5*\h},{0.5*\h}) {\texttt{1}};
\end{scope}
\end{tikzpicture}
\end{center}
Wenn man also $b=|v|$ wählt, ist das einmal aufgepumpte Wort nicht
mehr in $L$, im Widerspruch zur Behauptung des Pumping Lemma.

\ding{187}
Somit kann $L$ nicht regulär sein.
\end{loesung}
