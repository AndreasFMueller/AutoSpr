Zeigen Sie, dass die Sprache
\[
L=\{
0^m1^n|m\ne n
\}
\]
"uber dem Alphabet $\Sigma=\{0,1\}$ nicht regul"ar ist.

\begin{loesung}
Der einfachste Beweis verwendet die Mengenoperationen:
W"are $L$ regul"ar, m"usste auch das Komplement regul"ar sein.
Das Komplement der Sprache $L$ ist die Sprache
$$\{0^n1^n|n\ge 0\},$$
von der in der Vorlesung gezeigt wurde, dass sie nicht regul"ar
ist. Dieser Widerspruch zeigt, dass auch $L$ nicht regul"ar sein kann.

Man kann die Behauptung aber auch direkt mit dem Pumping-Lemma beweisen.
Wir nehmen also an, die Sprache sei regul"ar. Sei $N$ die Pumping Length.
Jedes gen"ugend lange Wort $x$ der Sprache kann dann geschrieben werden als $uvw$,
wobei $|uv|\le N$, und alle W"orter $uv^nw\in L$, f"ur alle $n\ge 1$.
Dem Teilwort $v$ entspricht eine Schleife im endlichen Automaten, der
$L$ akzeptiert.

Wir w"ahlen jetzt ein Wort $x=0^N1^{N+b}$. Dieses Wort kann
$x=uvw$ geschrieben werden, wobei dem Teilwort $v$ eine Schleife im
Graphen des Automaten entspricht.
Das Wort $v$ enth"alt nur Nullen, da $|uv|\le N$ sein muss. Vergr"ossert
man $b$, ist der erste Teil des Wortes davon unber"uhrt, es gilt also immer
noch die gleiche Aufteilung in $u$ und $v$, $w$ wird um einige Einsen l"anger.

Versuchen wir jetzt, das Wort aufzupumpen, also $uv^nw$ zu bilden, wird es
die Form
$$uv^nw=0^{|u|+n|v|+|w|_0}1^{|w|_1}
=
0^{|u|+n|v|+N-|u|-|v|}1^{N+b}
=
0^{(n-1)|v|+N}1^{N+b}
$$
haben. Wir m"ussen zeigen, dass wir erreichen k"onnen, dass die beiden
Exponenten "ubereinstimmen, also $(n-1)|v|=b$. Wenn $|v|=1$ ist, kann man
einfach $b$ mal aufpumpen (also $n=b+1$ setzen), um ein Wort zu erhalten,
welches nicht in $L$ ist.
Wenn aber $|v|>1$ ist, dann m"ussen wir zum Beispiel $b=|v|$ w"ahlen,
wie oben argumentiert "andert sich dadurch das $v$ nicht. Dann wird f"ur
$n=2$ das Wort $0^{N+b}1^{N+b}\not\in L$ entstehen, ein Widerspruch. Somit
kann $L$ nicht regul"ar sein.
\end{loesung}
