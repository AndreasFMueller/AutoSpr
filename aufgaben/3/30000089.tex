Wandeln Sie den NEA
\begin{center}
\def\l{2}
\def\r{0.4}
\def\zustand#1#2{
	\draw #1 circle[radius=\r];
	\node at #1 {$#2\mathstrut$};
}
\def\akzeptierzustand#1#2{
	\zustand{#1}{#2}
	\draw #1 circle[radius={\r-0.05}];
}
\def\punkte{
	\coordinate (S) at (-\l,0);
	\coordinate (q0) at (0,0);
	\coordinate (q1) at (\l,0);
	\coordinate (E) at ({2*\l},0);
}
\def\pfeil#1#2{
	\draw[->,shorten >= 0.4cm,shorten <= 0.4cm] #1 -- #2;
}
\begin{tikzpicture}[>=latex,thick]
\punkte
\zustand{(S)}{S}
\zustand{(q0)}{q_0}
\zustand{(q1)}{q_1}
\pfeil{({-1.8*\l},0)}{(S)}
\pfeil{(S)}{(q0)}
\node at ($0.5*(S)+0.5*(q0)$) [above] {$\varepsilon$};
\draw[->,shorten >= 0.4cm,shorten <= 0.4cm] (q0) to[out=20,in=160] (q1);
\node at ($0.5*(q0)+0.5*(q1)+(0,0.2)$) [above] {\texttt{0}};
\draw[->,shorten >= 0.4cm,shorten <= 0.4cm] (q1) to[out=-160,in=-20] (q0);
\node at ($0.5*(q0)+0.5*(q1)+(0,-0.2)$) [below] {\texttt{0}};
\pfeil{(q1)}{(E)}
\node at ($0.5*(q1)+0.5*(E)$) [above] {$\varepsilon$};
\akzeptierzustand{(E)}{E}
\draw[->,shorten >= 0.4cm,shorten <= 0.4cm]
	(q0) to[out=60,in=120,distance=1.5cm] (q0);
\node at ($(q0)+(0,1)$) [above] {\texttt{1}};
\draw[->,shorten >= 0.4cm,shorten <= 0.4cm]
	(q1) to[out=60,in=120,distance=1.5cm] (q1);
\node at ($(q1)+(0,1)$) [above] {\texttt{1}};
\end{tikzpicture}
\end{center}
der Wörter mit einer ungeraden Anzahl Nullen akzeptiert,
in einen VNEA um, indem Sie nacheinander die Zustände $q_0$ und $q_1$
entfernen, und lesen Sie den regulären Ausdruck ab, der genau auf die
Wörter passt, die der NEA akzeptiert

\begin{loesung}
\def\l{2}
\def\r{0.4}
\def\zustand#1#2{
	\draw #1 circle[radius=\r];
	\node at #1 {$#2\mathstrut$};
}
\def\akzeptierzustand#1#2{
	\zustand{#1}{#2}
	\draw #1 circle[radius={\r-0.05}];
}
\def\punkte{
	\coordinate (S) at (-\l,0);
	\coordinate (q0) at (0,0);
	\coordinate (q1) at (\l,0);
	\coordinate (E) at ({2*\l},0);
}
\def\pfeil#1#2{
	\draw[->,shorten >= 0.4cm,shorten <= 0.4cm] #1 -- #2;
}
Entfernung des Zustandes $q_0$ liefert
\begin{center}
\begin{tikzpicture}[>=latex,thick]
\punkte
\zustand{(S)}{S}
\zustand{(q1)}{q_1}
\pfeil{({-1.8*\l},0)}{(S)}
\pfeil{(S)}{(q1)}
\node at ($0.5*(S)+0.5*(q1)$) [above] {\texttt{1*0}};
\pfeil{(q1)}{(E)}
\node at ($0.5*(q1)+0.5*(E)$) [above] {$\varepsilon$};
\akzeptierzustand{(E)}{E}
\draw[->,shorten >= 0.4cm,shorten <= 0.4cm]
	(q1) to[out=60,in=120,distance=1.5cm] (q1);
\node at ($(q1)+(0,1)$) [above] {\texttt{1|01*0}};
\end{tikzpicture}
\end{center}
Entfernung des Zustands $q_1$ liefert:
\begin{center}
\begin{tikzpicture}[>=latex,thick]
\punkte
\zustand{(S)}{S}
\pfeil{({-1.8*\l},0)}{(S)}
\pfeil{(S)}{(E)}
\node at ($0.5*(S)+0.5*(E)$) [above] {\texttt{1*0(1|01*0)*}};
\akzeptierzustand{(E)}{E}
\end{tikzpicture}
\end{center}
Man kann sich davon überzeugen, dass der reguläre Ausdruck
\texttt{1*0(1|01*0)*} tatsächlich genau die Wörter mit einer
ungeraden Anzahl Nullen akzeptiert.
\end{loesung}

