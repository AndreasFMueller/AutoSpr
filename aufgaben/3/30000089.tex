Reduzieren Sie den DEA
\vspace*{-10pt}
\begin{center}
\def\l{1.3}
\def\r{0.30}
\def\zustand#1#2{
	\fill[color=white] #1 circle[radius=\r];
	\node at #1 {$#2\mathstrut$};
	\draw #1 circle[radius=\r];
}
\def\akzeptierzustand#1#2{
	\zustand{#1}{#2}
	\draw #1 circle[radius={\r-0.05}];
}
\def\pfeil#1#2{
	\draw[->,shorten >= 0.3cm,shorten <= 0.3cm] #1 -- #2;
}
\def\punkt#1#2#3{($#3*#1+#2-#3*#2$)}
\begin{tikzpicture}[>=latex,thick]
\coordinate (S) at ({0.8*\l},{0.8*\l});
\coordinate (q0) at (0,0);
\coordinate (q1) at (\l,0);
\coordinate (q2) at (\l,\l);
\coordinate (q3) at (0,\l);
\coordinate (e) at ({(1+sqrt(3)/2)*\l},{0.5*\l});
\pfeil{(S)}{(q0)}
\akzeptierzustand{(q0)}{q_0}
\akzeptierzustand{(q3)}{q_3}
\zustand{(q1)}{q_1}
\zustand{(q2)}{q_2}
\zustand{(e)}{e}

\draw[->,shorten >= 0.30cm,shorten <= 0.30cm]
	(q0) to[out=150,in=-150,distance=1.5cm] (q0);
\node at ($(q0)+(-1,0)$) [left] {\texttt{0}};

\draw[->,shorten >= 0.30cm,shorten <= 0.30cm]
	(q3) to[out=150,in=-150,distance=1.5cm] (q3);
\node at ($(q3)+(-1,0)$) [left] {\texttt{1}};

\draw[->,shorten >= 0.30cm,shorten <= 0.30cm]
	(e) to[out=-30,in=30,distance=1.5cm] (e);
\node at ($(e)+(1.0,0)$) [right] {$\Sigma$};

\pfeil{(q0)}{(q1)}
\node at \punkt{(q0)}{(q1)}{0.5} [below] {\texttt{1}};

\pfeil{(q1)}{(q2)}
\node at \punkt{(q1)}{(q2)}{0.5} [right] {\texttt{1}};

\pfeil{(q2)}{(q3)}
\node at \punkt{(q2)}{(q3)}{0.5} [above] {\texttt{1}};

\pfeil{(q3)}{(q0)}
\node at \punkt{(q3)}{(q0)}{0.5} [left] {\texttt{0}};

\pfeil{(q1)}{(e)}
\node at \punkt{(q1)}{(e)}{0.7} [below right] {\texttt{0}};

\pfeil{(q2)}{(e)}
\node at \punkt{(q2)}{(e)}{0.7} [above right] {\texttt{0}};

\end{tikzpicture}
\end{center}
\vspace*{-10pt}
von Aufgabe~\ref{30000087}, der Wörter
akzeptiert, in denen Einsen immer mindestens als Dreiergruppen
auftreten, auf einen NEA.

\thema{NEA}
\thema{DEA}

\begin{loesung}
Der NEA enhält nur noch die Zustände und Übergänge, die zum akzeptieren
nötig sind.
Wir verdeutlichen die Teile, die weggelassen wurden, dadurch, dass wir 
sie hellgrau stehen lassen:
\begin{center}
\def\l{1.3}
\def\r{0.30}
\def\zustand#1#2{
	\fill[color=white] #1 circle[radius=\r];
	\node at #1 {$#2\mathstrut$};
	\draw #1 circle[radius=\r];
}
\def\akzeptierzustand#1#2{
	\zustand{#1}{#2}
	\draw #1 circle[radius={\r-0.05}];
}
\def\pfeil#1#2{
	\draw[->,shorten >= 0.3cm,shorten <= 0.3cm] #1 -- #2;
}
\def\punkt#1#2#3{($#3*#1+#2-#3*#2$)}
\begin{tikzpicture}[>=latex,thick]
\coordinate (S) at ({0.8*\l},{0.8*\l});
\coordinate (q0) at (0,0);
\coordinate (q1) at (\l,0);
\coordinate (q2) at (\l,\l);
\coordinate (q3) at (0,\l);
\coordinate (e) at ({(1+sqrt(3)/2)*\l},{0.5*\l});
\pfeil{(S)}{(q0)}
\akzeptierzustand{(q0)}{q_0}
\akzeptierzustand{(q3)}{q_3}
\zustand{(q1)}{q_1}
\zustand{(q2)}{q_2}

\draw[->,shorten >= 0.30cm,shorten <= 0.30cm]
	(q0) to[out=150,in=-150,distance=1.5cm] (q0);
\node at ($(q0)+(-1,0)$) [left] {\texttt{0}};

\draw[->,shorten >= 0.30cm,shorten <= 0.30cm]
	(q3) to[out=150,in=-150,distance=1.5cm] (q3);
\node at ($(q3)+(-1,0)$) [left] {\texttt{1}};

\begin{scope}
\color{gray!40}
\zustand{(e)}{e}
\draw[->,shorten >= 0.30cm,shorten <= 0.30cm]
	(e) to[out=-30,in=30,distance=1.5cm] (e);
\node at ($(e)+(1.0,0)$) [right] {$\Sigma$};
\pfeil{(q1)}{(e)}
\node at \punkt{(q1)}{(e)}{0.7} [below right] {\texttt{0}};

\pfeil{(q2)}{(e)}
\node at \punkt{(q2)}{(e)}{0.7} [above right] {\texttt{0}};
\end{scope}

\pfeil{(q0)}{(q1)}
\node at \punkt{(q0)}{(q1)}{0.5} [below] {\texttt{1}};

\pfeil{(q1)}{(q2)}
\node at \punkt{(q1)}{(q2)}{0.5} [right] {\texttt{1}};

\pfeil{(q2)}{(q3)}
\node at \punkt{(q2)}{(q3)}{0.5} [above] {\texttt{1}};

\pfeil{(q3)}{(q0)}
\node at \punkt{(q3)}{(q0)}{0.5} [left] {\texttt{0}};


\end{tikzpicture}
\end{center}
\end{loesung}
