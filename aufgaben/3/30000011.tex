Welche der folgenden Aussagen ist wahr~bzw.~falsch:
\begin{teilaufgaben}
\item Jede Teilmenge einer regulären Sprache ist regulär.
\item Jede Obermenge einer regulären Sprache ist regulär.
\item Die Vereinigung zweier nicht regulärer Sprachen ist nicht regulär.
\item Die Schnittmenge zweier nicht regulärer Sprachen ist nicht
regulär.
\end{teilaufgaben}

\thema{Mengenoperationen für reguläre Sprachen}
\thema{regulär}

\begin{loesung}
\begin{teilaufgaben}
\item $\Sigma^*$ ist regulär, wäre a) wahr, müsste jede Sprache regulär
sein, aber wir kennen nicht reguläre Sprachen, zum Beispiel
$\{0^n1^n\,|\,n\in\mathbb N\}$.
\item $\emptyset$ ist regulär, da aber $\emptyset$ in jeder anderen
Teilmenge von $\Sigma^*$ enthalten ist, müssten alle Sprachen regulär
sein, im Widerspruch zur Tatsache, dass wir nicht reguläre Sprachen
kennen.
\item Wenn $L$ nicht regulär ist, dann ist auch $\bar L$ nicht regulär.
Wäre nämlich $\bar L$ regulär, müsste auch $\bar{\bar L}=L$ regulär
sein. Es ist aber auch $L\cup \bar L=\Sigma^*$ die Vereinigung zweier
nicht regulärer Sprachen, trotzdem ist $\Sigma^*$ natürlich regulär.
\item Das Komplement einer nicht regulären Sprache ist ebenfalls nicht
regulär. Wäre die Vereinigung zweier nicht regulärer Sprachen $L_1$
und $L_2$ nicht regulär müsste auch $\bar L_1\cup \bar L_2$ nicht
regulär sein, und damit auch $\overline{\bar L_1\cup \bar L_2}=L_1\cap L_2$.
Dass die Schnittmenge nicht regulärer Sprachen aber trotzdem regulär
sein kann haben wir bereits in b) eingesehen.
\qedhere
\end{teilaufgaben}
\end{loesung}

