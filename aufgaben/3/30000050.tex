Ist die Sprache
\[
L=\{w\in\{\texttt{1}\}^*\;|\;\text{$|w|$ ist prim}\}
\]
regulär?

\themaL{Pumping Lemma fur regulare Sprachen}{Pumping Lemma für reguläre Sprachen}
\themaL{regular}{regulär}

\begin{loesung}
Nein, wie wir mit dem Pumping Lemma zeigen können.
Dazu nehmen wir an, die Sprache $L$ sei regulär.
Nach dem Pumping Lemma gibt es daher die Pumping Length $N$, Wörter
mit Länge mindestens $N$ haben die Pump-Eigenschaft.
Sei also 
\[
w=\texttt{1}^p\qquad\text{mit einer Primzahl $p\ge N$},
\]
so eine Primzahl $p$ gibt es, weil es unendlich viele Primzahlen gibt.
Nach dem Pumping Lemma gibt es jetzt eine Unterteilung $w=xyz$, wobei
$|y|>0$, mit der Eigenschaft, dass auch alle aufgepumpten Wörter
$xy^kz$ in $L$ sind, d.~h.~prime Länge haben.
Wenn man aber genau mit $k=p+1$ pumpt, erhält man als Länge
\[
|xy^kz|
=
|x| + k|y| + |z|
=
|x| + (p+1)|y| + |z|
=
p|y| + \underbrace{|x|+|y|+|z|}_{\textstyle=p}
=
p(|y|+1)
\]
Die Länge von $xy^{(p+1)}z$ ist daher durch $p$ und $|y|+1$ teilbar, kann also
keine Primzahl sein, $xy^{(p+1)}z\not\in L$.
Dieser Widerspruch zeigt, dass die Annahme, $L$ sei regulär, nicht
aufrecht erhalten werden kann.
\end{loesung}


