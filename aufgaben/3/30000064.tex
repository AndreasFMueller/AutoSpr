Auf der Website \url{http://www.vogella.com} findet man ein Tutorial
zu regulären Audrücken, aus dem die folgenden Beispiele stammen.

\begin{teilaufgaben}
\item
Was für Strings akzeptiert der folgende reguläre Ausdruck?
\verbatimainput{ausdruck.txt}
Gesucht ist eine inhaltliche Spezifikation in Worten, die nicht einfach
nur eine Wiedergabe des regulären Ausdrucks in Worten ist.
\item
Die Klasse \texttt{LinkGetter} enthält den folgenden Konstruktor, mit
dem reguläre Ausdrücke in NEAs kompiliert werden:
\verbatimainput{30000064.java}
Das Metazeichen \texttt{\textbackslash b} steht für einen Wortanfang
oder ein Wortende (wie hier).
Damit wird erreicht, dass das Zeichen \texttt{a} alleine stehen muss.
Wozu dient der erste dieser regulären Ausdrücke?
\item
Die Idee des regulären Ausdrucks \texttt{link} war, dass damit der Wert
des \texttt{href}-Attributes isoliert werden kann.
Wenn der reguläre Ausdruck eine Zeichenkette $s$ akzeptiert, dann ist der 
Teil zwischen dem ersten und letzten Anführungszeichen der Wert des
Attributes, so die Erklärung.
Warum ist dies falsch?
\item
Können Sie sich einen String vorstellen, bei dem die regulären Ausdrücke
nicht das machen, was deren Autor von ihnen erwartet hat?
\end{teilaufgaben}

\themaL{regulare Ausdrucke}{reguläre Ausdrücke}

\begin{loesung}
\begin{teilaufgaben}
\item
Akzeptiert werden beliebige Zeichenketten, die irgendwo genau eine höchstens
dreistellige ganze Zahl enthalten, die kleiner als 300 sein muss.
\item
Es werden korrekt geschlossene \texttt{a}-Elemente gefunden, die ein 
\texttt{href}-Attribut haben, dessen Wert mit \texttt{\textquotedbl}
begrenzt ist.
Das \texttt{a}-Element kann auch noch weitere Attribute enthalten.
\item
Der Teilausdruck
\texttt{[\textasciicircum>]*}
akzeptiert auch Anführungszeichen.
Daher kann der Teil zwischen dem ersten und letzten Anführungszeichen
auch ganze Attributespezifikationen enthalten, die nach dem
\texttt{href}-Attribute angegeben wurden.
\item
Zusätzlich zu dem in c) erwähnten Problem können
Attributewerte in HTML auch mit Apostroph \texttt{\textquotesingle}
statt mit dem Anführungszeichen \texttt{\textquotedbl} begrenzt werden.
In diesem Fall erkennen die regulären Ausdrücke weder das \texttt{<a>}-Tag
noch das \texttt{href}-Attribut.
Ein Attribute mit dem Namen \texttt{xyhref} wird als \texttt{href}
interpretiert.
\qedhere
\end{teilaufgaben}
\end{loesung}

\begin{diskussion}
Das erwähnte Tutorial ist zu finden auf
\url{https://www.vogella.com/tutorials/JavaRegularExpressions/article.html}.
\end{diskussion}

\begin{bewertung}
\begin{teilaufgaben}
\item String enthält genau eine Zahl ({\bf Z}) 1 Punkt,
Zahl $<300$ ({\bf 3}) 1 Punkt.
\item \texttt{A}-Element ({\bf E}) 1 Punkt,
\texttt{href}-Element muss geschossen sein ({\bf H}) oder 
Element muss geschlossen sein ({\bf G}) 1 Punkt.
\item Zusätzliche Attribute ({\bf A}) 1 Punkt.
\item Quoting-Problem ({\bf Q}) 1 Punkt.
\end{teilaufgaben}
\end{bewertung}

