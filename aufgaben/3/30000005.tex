Gegeben sei ein deterministischer endlicher Automat $A=(Q,\Sigma, \delta,
z_0, F)$ mit $|Q|=|F|$. Wieviele Zustände enthält der minimale Automat?

\thema{DEA}
\thema{minimaler Automat}

\begin{loesung}
Da $Q$ and $F$ endliche Mengen sind folgt aus $|Q|=|F|$, dass $Q=F$.
Alle Zustände sind also Akzeptierzustände. Im Algorithmus zur Bestimmung
des Minimalautomaten können daher schon im ersten Schritt überhaupt
keine Zustände als äquivalent markiert werden, also erst recht nicht
in späteren Schritten.

Alternativ kann man auch argumentieren, dass von jedem Zustand $z$ aus
$$L(z)=\Sigma^*,$$
also
$$L(z)=L(z'),$$
d.~h.~die Zustände sind äquivalent.

Oder man kann sagen, dass dieser Automat, da er nur Akzeptierzustände
besitzt, notwendigerweise alle Wörter akzeptieren muss. $L(A)=\Sigma^*$,
der minimale Automat, der $\Sigma^*$ akzeptiert, ist aber
\[
\entrymodifiers={++[o][F]}
\xymatrix @-1mm {
*+\txt{} \ar[r]
        &*++[o][F=]{}\ar@(ur,dr)^{\Sigma}
}
\]
Dieser Automat hat offensichtlich nur einen Zustand.
\end{loesung}

