Gegeben sei ein deterministischer endlicher Automat $A=(Q,\Sigma, \delta,
z_0, F)$ mit $|Q|=|F|$. Wieviele Zust"ande enth"alt der minimale Automat?

\begin{loesung}
Da $Q$ and $F$ endliche Mengen sind folgt aus $|Q|=|F|$, dass $Q=F$.
Alle Zust"ande sind also Akzeptierzust"ande. Im Algorithmus zur Bestimmung
des Minimalautomaten k"onnen daher schon im ersten Schritt "uberhaupt
keine Zust"ande als "aquivalent markiert werden, also erst recht nicht
in sp"ateren Schritten.

Alternativ kann man auch argumentieren, dass von jedem Zustand $z$ aus
$$L(z)=\Sigma^*,$$
also
$$L(z)=L(z'),$$
d.~h.~die Zust"ande sind "aquivalent.

Oder man kann sagen, dass dieser Automat, da er nur Akzeptierzust"ande
besitzt, notwendigerweise alle W"orter akzeptieren muss. $L(A)=\Sigma^*$,
der minimale Automat, der $\Sigma^*$ akzeptiert, ist aber
\[
\entrymodifiers={++[o][F]}
\xymatrix @-1mm {
*+\txt{} \ar[r]
        &*++[o][F=]{}\ar@(ur,dr)^{\Sigma}
}
\]
Dieser Automat hat offensichtlich nur einen Zustand.
\end{loesung}

