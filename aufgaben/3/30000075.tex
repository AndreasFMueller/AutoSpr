Sei $f\colon \mathbb{N}\to\mathbb{N}$ eine streng monoton wachsende Funktion.
Eine solche Funktion nimmt beliebig grosse Werte an und es 
gilt $f(n) \ge n$ für alle $n\in\mathbb{N}$.
Ist die Sprache
\[
L=
\{
w\in\Sigma^*
\;|\;
|w|_{\texttt{0}}
\ge f(|w|_{\texttt{1}})
\}
\]
über dem Alphabet $\Sigma=\{\texttt{0},\texttt{1}\}$ regulär?

\begin{hinweis}
Die Sprache $L$ besteht aus Wörtern, die immer mehr \texttt{0} als \texttt{1}
enthalten.
Es gilt aber ausserdem, dass die Anzahl der Nullen mindestens so schnell
zunimmt wie die Anzahl der Einsen.
\end{hinweis}

\begin{loesung}
Die Sprache ist nicht regulär, wie man mit dem Pumping-Lemma für
reguläre Sprachen zeigen kann.
\begin{enumerate}
\item Annahme: $L$ ist regulär.
\item Nach dem Pumping Lemma gibt es die Pumping Length $N$.
\item Wir wählen das Wort $\texttt{0}^{f(N)}\texttt{1}^N\in L$.
\item Nach dem Pumping Lemma lässt sich das Wort $w$ in drei Teile
$w=xyz$
zerlegen mit $|xy|\le N$ und $|y|>0$.
Weil $f(N)\ge N$ enthält $y$ ausschliesslich Nullen.
\begin{center}
\definecolor{darkgreen}{rgb}{0,0.6,0}
\begin{tikzpicture}[>=latex,thick,scale=0.5]

\fill[color=gray!40] (0,-0.5) rectangle (20,0.5);
\draw (0,-0.5) rectangle (20,0.5);
\draw (16,-0.5) -- (16,0.5);

\node at (18,0) {$\texttt{1}^N$};
\node at (8,0) {$\texttt{0}^{f(N)}$};

\node at (0,0.5) [above] {$0$};
\node at (4,0.5) [above] {$N$};
\node at (16,0.5) [above] {$f(N)$};
\node at (20,0.5) [above] {$f(N)+N$};

\fill[color=darkgreen!40,opacity=0.5] (0.1,-0.4) rectangle (1.4,0.4);
\draw[color=darkgreen] (0.1,-0.4) rectangle (1.4,0.4);
\node[color=darkgreen] at (0.7,0) {$x\mathstrut$};

\fill[color=red!40,opacity=0.5] (1.5,-0.4) rectangle (2.4,0.4);
\draw[color=red] (1.5,-0.4) rectangle (2.4,0.4);
\node[color=red] at (1.95,0) {$y\mathstrut$};

\fill[color=blue!40,opacity=0.5] (2.5,-0.4) rectangle (19.9,0.4);
\draw[color=blue] (2.5,-0.4) rectangle (19.9,0.4);
\node[color=blue] at (11.2,0) {$z\mathstrut$};

\node at (21.5,0) [right] {$\in L$};

\begin{scope}[yshift=-3.5cm]

	\fill[color=gray!40] (0,-0.5) rectangle (21,0.5);
	\draw (0,-0.5) rectangle (21,0.5);
	\draw (17,-0.5) -- (17,0.5);

	\node at (19,0) {$\texttt{1}^N$};
	\node at (8.5,0) {$\texttt{0}^{f(N)+|y|}$};

	\node at (0,0.5) [above] {$0$};
	\node at (4,0.5) [above] {$N$};
	\node at (17,0.5) [above] {$f(N)+|y|\ge f(N)$};
	%\node at (21,0.5) [above] {$2^N+N$};

	\fill[color=darkgreen!40,opacity=0.5] (0.1,-0.4) rectangle (1.4,0.4);
	\draw[color=darkgreen] (0.1,-0.4) rectangle (1.4,0.4);
	\node[color=darkgreen] at (0.7,0) {$x\mathstrut$};

	\fill[color=red!40,opacity=0.5] (1.5,-0.4) rectangle (2.4,0.4);
	\draw[color=red] (1.5,-0.4) rectangle (2.4,0.4);
	\node[color=red] at (1.95,0) {$y\mathstrut$};

	\fill[color=red!40,opacity=0.5] (2.5,-0.4) rectangle (3.4,0.4);
	\draw[color=red] (2.5,-0.4) rectangle (3.4,0.4);
	\node[color=red] at (2.95,0) {$y\mathstrut$};

	\fill[color=blue!40,opacity=0.5] (3.5,-0.4) rectangle (20.9,0.4);
	\draw[color=blue] (3.5,-0.4) rectangle (20.9,0.4);
	\node[color=blue] at (12.2,0) {$z\mathstrut$};

	\node at (21.5,0) [right] {$\in L$};

\end{scope}

\begin{scope}[yshift=-7.0cm]

	\fill[color=gray!40] (0,-0.5) rectangle (19,0.5);
	\draw (0,-0.5) rectangle (19,0.5);
	\draw (15,-0.5) -- (15,0.5);

	\node at (17,0) {$\texttt{1}^N$};
	\node at (7.5,0) {$\texttt{0}^{f(N)-|y|}$};

	\node at (0,0.5) [above] {$0$};
	\node at (15,0.5) [above] {$f(N)-|y|<f(N)$};
	%\node at (20,0.5) [above] {$2^N-|y|+N$};

	\fill[color=darkgreen!40,opacity=0.5] (0.1,-0.4) rectangle (1.4,0.4);
	\draw[color=darkgreen] (0.1,-0.4) rectangle (1.4,0.4);
	\node[color=darkgreen] at (0.7,0) {$x\mathstrut$};

	%\fill[color=red!40,opacity=0.5] (1.5,-0.4) rectangle (2.4,0.4);
	%\draw[color=red] (1.5,-0.4) rectangle (2.4,0.4);
	%\node[color=red] at (1.95,0) {$y\mathstrut$};

	\fill[color=blue!40,opacity=0.5] (1.5,-0.4) rectangle (18.9,0.4);
	\draw[color=blue] (1.5,-0.4) rectangle (18.9,0.4);
	\node[color=blue] at (10.2,0) {$z\mathstrut$};

	\node at (21.5,0) [right] {$\not\in L$};

\end{scope}

\end{tikzpicture}
\end{center}
\item Beim Aufpumpen wird die Anzahl der Nullen grösser, das bedeutet
aber nicht, dass das Wort nicht mehr in der Sprache enthalten ist,
da nur $|w|_{\texttt{0}} \ge f(|w|_{\texttt{1}})$ verlangt ist.
Beim Abpumpen wird allerdings die Anzahl der Nullen kleiner, $xz$ enthält
nur noch $f(N)-|y|$ Nullen.
Damit haben wir 
\[
|xz|_{\texttt{0}} = f(N)-|y| < f(N) = |xz|_{\texttt{1}},
\]
das Wort $xz$ ist also nicht mehr in $L$, im Widerspruch zur Aussage
des Pumping Lemmas.
\item Der Widerspruch zeigt, dass Annahme, $L$ sie regulär, nicht
haltbar ist.
Also ist $L$ nicht regulär.
\qedhere
\end{enumerate}
\end{loesung}

\begin{bewertung}
Pumping-Lemma-Beweis: jeder Schritt ein Punkt:
Annahme regulär ({\bf PL}) 1 Punkt,
Pumping Length ({\bf N}) 1 Punkt,
Wort ({\bf W}) 1 Punkt,
Aufteilung ({\bf A}) 1 Punkt,
Widerspruch beim Pumpen ({\bf P}) 1 Punkt,
Schlussfolgerung ({\bf S}) 1 Punkt.
\end{bewertung}
