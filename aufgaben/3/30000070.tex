\begin{teilaufgaben}
\item
In einem Regex-Dialekt bezeichnet das Metazeichen \texttt{\textbackslash{}b}
eine Wortgrenze.
Auf was für Zeichenketten passt der reguläre Ausdruck
\texttt{\textbackslash{}bb[ou]y\textbackslash{}b}
\item
Im gleichen Dialekt wird nach Zeichenketten gesucht, die auf den Ausdruck
\begin{verbatim}
\b[A-Z0-9._%+-]+@([A-Z0-9-]+\.)+[A-Z]{2,6}\b
\end{verbatim}
Was für Zeichenketten werden gefunden?
\item
In einem Konfigurationsfile für einen Apache-Webserver wird die folgende
Direktive verwendet:
\begin{verbatim}
RewriteCond %{REMOTE_ADDR} 192\.168\.\d\d.*
\end{verbatim}
Auf welche IP-Adressen passt der Ausdruck?
\item
In einem Textfile stehen zeilenweise Informationen über Flughäfen bestehend
aus Stadt, Kurzzeichen (IATA airport code, drei Grossbuchsatben) und Land
im Format:
\begin{verbatim}
Zurich (ZRH) Switzerland
San Francisco (SFO) USA
Sydney (SYD) Australia
Auckland (AKL) New Zealand
\end{verbatim}
Ausserdem gibt es Zeilen, die nicht auf dieses Format passen.
Mit einem regulären Ausdruck soll nun der Teil vor der Kennung extrahiert
werden. 
Geben Sie einen regulären Ausdruck an, der genau auf so strukturierte
Zeilen passt.
\end{teilaufgaben}

\begin{loesung}
\begin{teilaufgaben}
\item
Zeichenketten, die das \texttt{boy} oder \texttt{buy} als
alleinstehendes Wort enhalten.
\item
Dieser Ausdruck findet gross geschriebene Email-Adressen.
Der Top-Level-Domain darf allerdings höchstens 6 Zeichen enthalten,
was für viele TLDs nicht funktioniert, zum Beispiel für \texttt{.university}.
\item
Auf diesen Ausdruck passen auf die Klasse-C Adressen mit Netznummern
zwischen 192.168.0 und 192.168.99.
Man beachte, dass führende Nullen in IP-Adressen erlaubt sind.
\item
Der Ausdruck muss sicherstellen, dass vor und nach der Klammer Blanks stehen
und dass in der Klammer genau drei Grossbuchstaben stehen.
Stadtname und Land dürfen Leerzeichen enthalten:
\verbatimainput{air.txt}
Die Klammern \texttt{[A-Za-z]} stellen sicher, dass Strasse und Land
mindesstens ein Zeichen enthalten.
\end{teilaufgaben}
\end{loesung}

\begin{bewertung}
\begin{teilaufgaben}
\item ({\bf A}) 1 Punkt
\item ({\bf B}) 2 Punkte
\item ({\bf C}) 1 Punkt
\item ({\bf D}) 2 Punkte
\end{teilaufgaben}
\end{bewertung}


