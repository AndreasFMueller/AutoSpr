Konstruieren Sie einen nichtdeterministischen endlichen Automaten
mit drei Zuständen
über dem Alphabet $\{{\tt 0},{\tt 1}\}$, welcher die Sprache
\[
L=\{w\in\Sigma^*\;|\; \text{$w$ endet mit {\tt 00}}\}.
\]
akzeptiert.
Wandeln Sie diesen anschliessend mit dem Standardalgorithmus
in einen deterministischen endlichen Automaten um.

\thema{NEA}
\thema{DEA}

\begin{loesung}
\def\zustand#1#2{
	\draw #1 circle[radius=0.4];
	\node at #1 {#2};
}
\def\akzeptierzustand#1#2{
	\draw #1 circle[radius=0.34];
	\zustand{#1}{#2}
}
\def\pfeil#1#2{
	\draw[->,shorten >= 0.4cm,shorten <= 0.4cm] #1 -- #2;
}
Ein besonders einfacher NEA ist
\begin{center}
\begin{tikzpicture}[>=latex,thick]
\coordinate (s) at (-2,0);
\coordinate (z1) at (0,0);
\coordinate (z2) at (2,0);
\coordinate (z3) at (4,0);
\zustand{(z1)}{$z_1$}
\zustand{(z2)}{$z_2$}
\akzeptierzustand{(z3)}{$z_3$}
\pfeil{(s)}{(z1)}
\pfeil{(z1)}{(z2)}
\pfeil{(z2)}{(z3)}
\draw[->,shorten >= 0.4cm,shorten <= 0.4cm]
	(z1) to[out=-60,in=-120,distance=1.4cm] (z1);
\node at ($0.5*(z1)+0.5*(z2)$) [above] {\texttt{0}};
\node at ($0.5*(z2)+0.5*(z3)$) [above] {\texttt{0}};
\node at ($(z1)+(0,-1.3)$) {\texttt{0},\texttt{1}};
\end{tikzpicture}
\end{center}
Dabei haben die beiden explizit gezeigten Nullen eine spezielle
Bedeutung: es sind ``Endnullen'', im Gegensatz zu gewöhnlichen
Nullen, die nicht zu den letzten zwei Nullen eines akzeptierten Wortes
gehören.
Der Nichtdeterminismus besteht darin, dass man für die
Entscheidung, welcher "Ubergang bei der nächsten Null zu wählen
ist, die Hilfe eines Orakels braucht, welches weiss, ob die
nächste Null bereits eine Endnull ist.

Die Umwandlung mit Hilfe des Standardalgorithmus führt auf folgenden
Zwischenautomaten
\begin{center}
\begin{tikzpicture}[>=latex,thick]
\coordinate (s) at (-2,0);
\coordinate (q) at (-2,-2);
\coordinate (q1) at (0,0);
\coordinate (q2) at (0,-2);
\coordinate (q3) at (0,-4);
\coordinate (q12) at (2,0);
\coordinate (q13) at (2,-2);
\coordinate (q23) at (2,-4);
\coordinate (q123) at (4,-2);
\pfeil{(s)}{(q1)}
\zustand{(q1)}{$1$}
\zustand{(q12)}{$12$}
\akzeptierzustand{(q123)}{$123$}
\pfeil{(q12)}{(q123)}
\pfeil{(q123)}{(q1)}

\begin{scope}
\color{gray}
\zustand{(q)}{$\emptyset$}
\zustand{(q2)}{$2$}
\akzeptierzustand{(q3)}{$3$}
\akzeptierzustand{(q13)}{$13$}
\akzeptierzustand{(q23)}{$23$}
\pfeil{(q2)}{(q)}
\pfeil{(q23)}{(q3)}
\pfeil{(q23)}{(q)}
\pfeil{(q2)}{(q3)}
\pfeil{(q3)}{(q)}
\pfeil{(q13)}{(q12)}
\pfeil{(q13)}{(q1)}
\draw[->,shorten >= 0.4cm,shorten <= 0.4cm]
	(q) to[out=150,in=-150,distance=1.4cm] (q);
\node at ($(q)+(-1.35,0)$) {\texttt{0},\texttt{1}};
\node at ($0.7*(q12)+0.3*(q13)$) [right] {\texttt{0}};
\node at ($0.8*(q23)+0.2*(q)$) [above right] {\texttt{1}};
\node at ($0.5*(q)+0.5*(q3)$) [below left] {\texttt{0},\texttt{1}};
\node at ($0.5*(q)+0.5*(q2)$) [above] {\texttt{1}};
\node at ($0.5*(q1)+0.5*(q13)$) [below left] {\texttt{1}};
\node at ($0.5*(q3)+0.5*(q23)$) [below] {\texttt{0}};
\node at ($0.7*(q2)+0.3*(q3)$) [right] {\texttt{0}};
\end{scope}

\draw[->,shorten >= 0.4cm,shorten <= 0.4cm]
	(q1) to[out=60,in=120,distance=1.4cm] (q1);
\draw[->,shorten >= 0.4cm,shorten <= 0.4cm]
	(q123) to[out=30,in=-30,distance=1.4cm] (q123);
\draw[->,shorten >= 0.4cm,shorten <= 0.4cm]
	(q1) to[out=45,in=135] (q12);
\pfeil{(q12)}{(q1)}
\node at ($0.5*(q1)+0.5*(q12)+(0,0.7)$) {\texttt{0}};
\node at ($0.5*(q1)+0.5*(q12)$) [above] {\texttt{1}};
\node at ($(q1)+(0,1.3)$) {\texttt{1}};
\node at ($(q123)+(1.25,0)$) {\texttt{0}};
\node at ($0.5*(q12)+0.5*(q123)$) [above right] {\texttt{0}};
\node at ($0.85*(q123)+0.15*(q1)$) [below left] {\texttt{1}};
\end{tikzpicture}
\end{center}
Offenbar werden nur die Zustände $1$, $12$  und $123$ benötigt,
die anderen sind vom Startzustand aus nicht erreichbar. Der DEA
ist daher
\begin{center}
\begin{tikzpicture}[>=latex,thick]
\coordinate (s) at (-2,0);
\coordinate (a) at (0,0);
\coordinate (b) at (2,0);
\coordinate (c) at (4,0);
\zustand{(a)}{$a$}
\zustand{(b)}{$b$}
\akzeptierzustand{(c)}{$c$}
\pfeil{(s)}{(a)}
\pfeil{(b)}{(c)}
\draw[->,shorten >= 0.4cm,shorten <= 0.4cm]
	(a) to[out=20,in=160] (b);
\draw[->,shorten >= 0.4cm,shorten <= 0.4cm]
	(b) to[out=-160,in=-20] (a);
\draw[->,shorten >= 0.4cm,shorten <= 0.4cm]
	(c) to[out=30,in=-30,distance=1.4cm] (c);
\draw[->,shorten >= 0.4cm,shorten <= 0.4cm]
	(c) to[out=-135,in=-45] (a);
\draw[->,shorten >= 0.4cm,shorten <= 0.4cm]
	(a) to[out=-60,in=-120,distance=1.4cm] (a);
\node at ($0.5*(b)+0.5*(c)$) [above] {\texttt{0}};
\node at ($0.5*(a)+0.5*(b)+(0,0.5)$) {\texttt{0}};
\node at ($0.4*(a)+0.7*(b)+(0,-0.4)$) {\texttt{1}};
\node at ($(c)+(1.2,0)$) {\texttt{0}};
\node at ($(a)+(0,-1.3)$) {\texttt{1}};
\node at ($0.5*(a)+0.5*(c)+(0,-1.2)$) {\texttt{1}};
\end{tikzpicture}
\end{center}
\end{loesung}
