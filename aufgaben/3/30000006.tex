Konstruieren Sie einen nichtdeterministischen endlichen Automaten
mit drei Zuständen
über dem Alphabet $\{{\tt 0},{\tt 1}\}$, welcher die Sprache
\[
L=\{w\in\Sigma^*\;|\; \text{$w$ endet mit {\tt 00}}\}.
\]
akzeptiert.
Wandeln Sie diesen anschliessend mit dem Standardalgorithmus
in einen deterministischen endlichen Automaten um.

\thema{NEA}
\thema{DEA}

\begin{loesung}
Ein besonders einfacher NEA ist
\[
\entrymodifiers={++[o][F]}
\xymatrix{
*+\txt{} \ar[r]
        &{z_1} \ar[r]^{\tt 0} \ar@(dr,dl)^{\Sigma}
                &{z_2} \ar[r]^{\tt 0}
                        &*++[o][F=]{z_3}
}
\]
Dabei haben die beiden explizit gezeigten Nullen eine spezielle
Bedeutung: es sind ``Endnullen'', im Gegensatz zu gewöhnlichen
Nullen, die nicht zu den letzten zwei Nullen eines akzeptierten Wortes
gehören.
Der Nichtdeterminismus besteht darin, dass man für die
Entscheidung, welcher "Ubergang bei der nächsten Null zu wählen
ist, die Hilfe eines Orakels braucht, welches weiss, ob die
nächste Null bereits eine Endnull ist.

Die Umwandlung mit Hilfe des Standardalgorithmus führt auf folgenden
Zwischenautomaten
\[
\entrymodifiers={++[o][F]}
\xymatrix{
*+\txt{}\ar[r]
        &{1} \ar@/^/[r]^{\tt 0} \ar@(ur,ul)_{\tt 1}
                &{12}\ar[dr]^{\tt 0} \ar@/^/[l]^{\tt 1}
                        &*+\txt{}
\\
{\emptyset}\ar@(ul,dl)_{\Sigma}
        &{2}\ar[l]_{\tt 1} \ar[d]^{\tt 0}
                &*++[o][F=]{13} \ar[u]^{\tt 0} \ar[ul]^{\tt 1}
                        &*++[o][F=]{123} \ar@(ur,dr)^{\tt 0} \ar[ull]_{\tt 1}
\\
*+\txt{}
        &*++[o][F=]{3} \ar[ul]^{\Sigma}
                &*++[o][F=]{23} \ar[l]_{\tt 0} \ar[ull]_{\tt 1}
                        &*+\txt{}
}
\]
Offenbar werden nur die Zustände $1$, $12$  und $123$ benötigt,
die anderen sind vom Startzustand aus nicht erreichbar. Der DEA
ist daher
\[
\entrymodifiers={++[o][F]}
\xymatrix{
*\txt{}\ar[r]
        &{a} \ar@/^/[r]^{\tt 0} \ar@(dr,dl)^{\tt 1}
                &{b} \ar[r]^{\tt 0} \ar@/^/[l]_{\tt 1}
                        &*++[o][F=]{c} \ar@(ur,dr)^{\tt 0} \ar@/^15pt/[ll]^{\tt 1}
}
\]
\end{loesung}
