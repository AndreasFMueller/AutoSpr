Ein Wort $w\in\Sigma^*$ ($\Sigma=\{0,1\}$) heisst periodisch, wenn
es von der Form $w^k$ mit $k\ge 2$  ist, also aus einem wiederholten
Teil besteht. Sei $L$ die Sprache der periodischen Wörter, also
\[
L=\{w^k \,|\,w\in\Sigma^*,k\ge 2\}.
\]
Die folgenden Wörter sind periodisch (in Klammern jeweils die Länge
des wiederholten Teiles):
\[
\varepsilon\;(0),\quad
\text{\tt 011011011}\;(3),\quad
\text{\tt 1001010010}\;(5),\quad
\text{\tt 11111}\;(1).
\]
Die folgenden Wörter sind nicht periodisch:
\[
\text{\tt 011011001},\quad
\text{\tt 1001110010},\quad
\text{\tt 11110}.
\]
Ist $L$ regulär?

\thema{regulär}
\thema{Pumping Lemma für reguläre Sprachen}

\begin{loesung}
$L$ ist nicht regulär, wie man mit dem Pumping Lemma zeigen kann. Zum Beweis
nehmen wir an $L$ sei regulär, dann sagt das Pumping Lemma für
reguläre Sprachen, dass es
eine Zahl $N$ gäbe, die Pumping Length, so dass für Wörter mit
Länge $N$ oder grösser spezielle Eigenschaften gelten.
Ein periodisches Wort dieser Länge ist
\[
w=0^N1\quad\Rightarrow\quad
W=w^2=\underbrace{0\dots 0}_N1\underbrace{0\dots 0}_N1
\]
$W$ ist ein Wort mit Periode $N+1$. Das Pumping Lemma besagt jetzt, dass
sich $W$ in drei Teile $xyz$ aufteilen lässt mit folgenden Eigenschaften.
Zunächst ist $|xy|\le N$, d.~h.~sowohl $x$ als auch $y$ bestehen aus
lauter Nullen. Ausserdem ist $|y|>0$, es hat also mindestens eine Null
in $y$. Das Pumping Lemma behauptet ausserdem, dass auch das Wort
$xy^kz$ in der Sprache $L$ enthalten ist. Dieses Wort hat aber die
Form
\[
xy^kz=\underbrace{0\dots0}_{N+(k-1)|y|}1\underbrace{0\dots 0}_N1
\]
Falls $k\ne 1$ ist, der Nuller-Teil vor der ersten $1$ verschieden vom
Nullerteil danach,
\[
N+(k-1)|y|\ne N,
\]
das Wort kann also nicht mehr periodisch
sein. Dies steht im Widerspruch zur Folgerung des Pumping-Lemmas, also
konnte $L$ nicht regulär gewesen sein.
\end{loesung}
