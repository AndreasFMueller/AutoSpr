Verwenden Sie den Thompson-NEA um den folgenden $\text{NEA}_\varepsilon$
über dem Alphabet
$\Sigma=\{\texttt{0},\texttt{1}\}$ in einen DEA umzuwandeln.
\begin{center}
\def\l{2}
\def\r{0.35}
\def\punkt#1#2{({(#1)*\l},{(#2)*\l})}
\def\zustand#1#2{
	\fill[color=white] #1 circle[radius=\r];
	\draw #1 circle[radius=\r];
	\node at #1 {$#2\mathstrut$};
}
\def\akzeptierzustand#1#2{
	\zustand{#1}{#2}
	\draw #1 circle[radius={\r-0.05}];
}
\def\pfeil#1#2{
	\draw[->,shorten >= 0.35cm,shorten <= 0.35cm] #1 -- #2;
}
\begin{tikzpicture}[>=latex,thick]
\coordinate (q0) at (0,0);
\coordinate (q1) at (\l,0);
\coordinate (q2) at ({0.5*\l},{-sqrt(3)*\l/2});
\zustand{(q0)}{q_0}
\zustand{(q2)}{q_2}
\akzeptierzustand{(q1)}{q_1}
\pfeil{(q1)}{(q2)}
\pfeil{(q2)}{(q0)}
\pfeil{(q0)}{(q1)}
\pfeil{(-1.5,0)}{(q0)}
\draw[->,shorten >= 0.35cm,shorten <= 0.35cm]
	(q0) to[out=60,in=120,distance=1.2cm] (q0);
\node at ($(q0)+(0,0.8)$) [above] {\texttt{1}};
\draw[->,shorten >= 0.35cm,shorten <= 0.35cm]
	(q1) to[out=60,in=120,distance=1.2cm] (q1);
\node at ($(q1)+(0,0.8)$) [above] {$\Sigma$};
\node at ($0.5*(q0)+0.5*(q1)$) [above] {\texttt{1}};
\node at ($0.5*(q1)+0.5*(q2)$) [below right] {$\varepsilon$};
\node at ($0.5*(q0)+0.5*(q2)$) [below left] {\texttt{0}};
\end{tikzpicture}
\end{center}

\begin{loesung}
Da der NEA zwei Zustände $Q=\{q_0,q_1\}$ hat, hat der DEA höchstens
die vier Zustände
\[
Q' = P(Q) = \{ \emptyset, \{q_0\}, \{q_1\}, Q \}.
\]
Das Zustandsdiagramm ist
\begin{center}
\def\l{2.2}
\def\r{0.55}
\def\zustand#1#2{
	\fill[color=white] #1 circle[radius=\r];
	\draw #1 circle[radius=\r];
	\node at #1 {$#2\mathstrut$};
}
\def\akzeptierzustand#1#2{
	\zustand{#1}{#2}
	\draw #1 circle[radius={\r-0.05}];
}
\def\pfeil#1#2{
	\draw[->,shorten >= 0.55cm,shorten <= 0.55cm] #1 -- #2;
}
\begin{tikzpicture}[>=latex,thick]
\coordinate (e) at ({-1.5*\l},0);
\zustand{(e)}{\emptyset}
\coordinate (q0) at ({-0.5*\l},\l);
\zustand{(q0)}{q_0}
\pfeil{({-1.5*\l},\l)}{(q0)}
\coordinate (q1) at ({-0.5*\l},0);
\coordinate (q2) at ({-0.5*\l},-\l);
\coordinate (q12) at ({0.5*\l},\l);
\coordinate (q02) at ({0.5*\l},0);
\coordinate (q01) at ({0.5*\l},-\l);

\begin{scope}
\color{gray}
\akzeptierzustand{(q1)}{q_1}
\zustand{(q2)}{q_2}
\akzeptierzustand{(q01)}{q_0,q_1}
\zustand{(q02)}{q_0,q_2}
\akzeptierzustand{(q12)}{q_1,q_2}
\end{scope}

\coordinate (q012) at ({1.5*\l},0);
\akzeptierzustand{(q012)}{Q}

% from {q0}
\pfeil{(q0)}{(e)}
\node at ($0.5*(e)+0.5*(q0)$) [above left] {\texttt{0}};
\pfeil{(q0)}{(q012)}
\node at ($0.2*(q0)+0.8*(q012)+(0,0.1)$) [below left] {\texttt{1}};

% from {}
\draw[->,shorten >= 0.55cm,shorten <= 0.55cm]
	(e) to[out=150,in=-150,distance=1.5cm] (e);
\node at ($(e)+(-1,0)$) [left] {$\Sigma$};

% from q = {q0,q1,q2}
\draw[->,shorten >= 0.55cm,shorten <= 0.55cm]
	(q012) to[out=-30,in=30,distance=1.5cm] (q012);
\node at ($(q012)+(1,0)$) [right] {\texttt{1},\texttt{0}};

\end{tikzpicture}
\end{center}
Die grau gezeichneten Zustände sind nicht erreichbar.
Es entsteht der gleiche DEA wie in Aufgabe~\ref{30000090}.
\end{loesung}
