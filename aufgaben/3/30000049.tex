Finden Sie einen regulären Ausdruck für die Sprache
\[
L=\{w\in\Sigma^*\,|\, \text{$|w|_{\texttt{0}}$ ist gerade}\}
\]
über dem Alphabet $\Sigma=\{\texttt{0},\texttt{1}\}$.

\thema{reguläre Ausdrücke}
\thema{regulär}

\begin{loesung}
Wir konstruieren zunächst einen DEA für die Sprache $L$:
\[
\entrymodifiers={++[o][F]}
\xymatrix{
*+\txt{}\ar[r]
        &*++[o][F=]{0}\ar@/^/[r]^{\texttt{0}} \ar@(ur,ul)_{\texttt{1}}
                &{1} \ar@(ur,dr)^{\texttt{1}} \ar@/^/[l]^{\texttt{0}}
}
\]
Diesen Automaten müssen wir nun mit separaten Start- und Akzeptierzuständen
ausstatten:
\[
\entrymodifiers={++[o][F]}
\xymatrix{
*+\txt{}\ar[r]
	&{S}\ar[r]^{\varepsilon}
        	&{0}\ar[d]^{\varepsilon}\ar@/^/[r]^{\texttt{0}} \ar@(ur,ul)_{\texttt{1}}
                	&{1} \ar@(ur,dr)^{\texttt{1}} \ar@/^/[l]^{\texttt{0}}
\\
*+\txt{}
	&*+\txt{}
		&*++[o][F=]{A}
}
\]
Jetzt entfernen wir nacheinander die Zwischenzustände, beginnend mit dem
Zustand $1$:
\[
\entrymodifiers={++[o][F]}
\xymatrix{
*+\txt{}\ar[r]
	&{S}\ar[r]^{\varepsilon}
        	&{0}\ar[d]^{\varepsilon}
			\ar@(ur,dr)^{\texttt{1}|\texttt{01*0}}
\\
*+\txt{}
	&*+\txt{}
		&*++[o][F=]{A}
}
\]
gefolgt vom Zustand $0$:
\[
\entrymodifiers={++[o][F]}
\xymatrix{
*+\txt{}\ar[r]
	&{S}\ar[dr]^{\texttt{(1|01*0)*}}
\\
*+\txt{}
	&*+\txt{}
		&*++[o][F=]{A}
}
\]
Der reguläre Ausdruck besagt, dass ein Wort in $L$ aus Teilen besteht, die
entweder nur aus Einsen bestehen, oder aus Paaren von Nullen, zwischen
denen beliebige viel Einsen eingeschoben sein können.

Beginnt man mit dem Enfernen von Zwischenzuständen beim Zustand $0$, wird 
der reguläre Ausdruck ganz anders aussehen:
\[
\entrymodifiers={++[o][F]}
\xymatrix{
*+\txt{}\ar[r]
	&{S}\ar[rr]^{\texttt{1*0}} \ar[dr]_{\texttt{1*}}
		&*+\txt{}
                	&{1} \ar@(ur,dr)^{\texttt{1|01*0}} \ar[dl]_{\texttt{01*}}
\\
*+\txt{}
	&*+\txt{}
		&*++[o][F=]{A}
}
\]
Jetzt entfernt man den Zustand $1$ und erhält
\[
\entrymodifiers={++[o][F]}
\xymatrix{
*+\txt{}\ar[r]
	&{S} \ar[dr]^{\texttt{1*}|\texttt{1*0(1|01*0)*01*}}
		&*+\txt{}
\\
*+\txt{}
	&*+\txt{}
		&*++[o][F=]{A}
}
\]
Auch dieser reguläre Ausdruck besagt, dass man zusätzliche Nullen immer
nur paarweise einschieben kann, und dass dazwischen jeweils Einsen stehen
können.
\end{loesung}

