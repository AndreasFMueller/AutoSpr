Sei $\Sigma=\{\texttt{1}\}$ und die Sprache $L$ über dem Alphabet $\Sigma$
bestehe aus Wörtern, deren Länge eine Fakultät ist:
\[
L=\{
a^{k!}\,|\, k\in \mathbb N
\}.
\]
Ist $L$ regulär?

\thema{regulär}
\thema{Pumping Lemma für reguläre Sprachen}

\begin{loesung}
Nein, wie man mit dem Pumping-Lemma beweisen kann.
Dazu nehmen wir an, dass $L$ regulär ist.
Sei $N$ die Pumping length von $L$ und $k$ die kleinste ganze Zahl
derart, dass $k!>N$ ist.
Dann ist das Wort
\[
w=\texttt{1}^{k!} \in L.
\]
Nach dem Pumping Lemma lässt sich $w$ in drei Teile $w=xyz$ aufteilen
derart, dass $|y|>0$ und $xy^rz\in L$ für jedes $r$.
Nun ist aber $|y|\le k!$ und damit $|xy^2z|\le k!\cdot 2 < k!\cdot (k+1) = (k+1)!$, 
das Wort $xy^2z$ ist also nicht das Wort $\texttt{1}^{k!}$ und ausserdem
ist es zu kurz,
um $\texttt{1}^{(k+1)!}$ zu sein, es kann daher nicht in $L$ sein,
im Widerspruch zur Aussage des Pumping-Lemmas.
Dieser Widerspruch zeigt, dass $L$ nicht regulär sein kann.
\end{loesung}

\begin{bewertung}
Pumping Lemma ({\bf PL}) 1 Punkt,
Pumping Length ({\bf N}) 1 Punkt,
Beispielwort ({\bf W}) 1 Punkt,
Aufteilung ({\bf A}) 1 Punkt,
Pumpeigenschaft ({\bf P}) 1 Punkt,
Widerspruch und Schlussfolgerung ({\bf R}) 1 Punkt.
\end{bewertung}

