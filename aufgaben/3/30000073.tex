Finden Sie einen deterministischen endlichen Automaten, der
die Sprache
\[
L
=
\{
w = \{\texttt{0},\texttt{1}\}^*
\;|\;
|w|_{\texttt{0}}
\not
\equiv
|w|_{\texttt{1}}
\mod 5
\}
\]
bestehend aus Wörtern, deren Anzahl von Nullen und Einsen verschiedenen 
Rest bei Teilung durch fünf haben.

\begin{loesung}
Das Zustandsdiagramm 
\begin{center}
\begin{tikzpicture}[>=latex,thick]

\coordinate (O) at (0,0);
\coordinate (A) at (2,0);
\coordinate (B) at (4,0);
\coordinate (C) at (6,0);
\coordinate (D) at (8,0);
\coordinate (E) at (10,0);

\node at (A) {$q_0$};
\node at (B) {$q_1$};
\node at (C) {$q_2$};
\node at (D) {$q_3$};
\node at (E) {$q_4$};

\draw[->,shorten >= 0.3cm,shorten <= 0.3cm] (O) -- (A);

\draw (A) circle[radius=0.3];
\draw (B) circle[radius=0.3];
\draw (B) circle[radius=0.25];
\draw (C) circle[radius=0.3];
\draw (C) circle[radius=0.25];
\draw (D) circle[radius=0.3];
\draw (D) circle[radius=0.25];
\draw (E) circle[radius=0.3];
\draw (E) circle[radius=0.25];

\draw[->,shorten >= 0.3cm,shorten <= 0.3cm]
	(A) to[out=30,in=150] (B);
\draw[->,shorten >= 0.3cm,shorten <= 0.3cm]
	(B) to[out=30,in=150] (C);
\draw[->,shorten >= 0.3cm,shorten <= 0.3cm]
	(C) to[out=30,in=150] (D);
\draw[->,shorten >= 0.3cm,shorten <= 0.3cm]
	(D) to[out=30,in=150] (E);

\draw[->,shorten >= 0.3cm,shorten <= 0.3cm]
	(E) to[out=110,in=0] ($(E)+(-1.0,1.3)$)
	--
	($(A)+(1.0,1.3)$) to[out=180,in=70] (A);

\node at ($0.5*(A)+0.5*(B)+(0,0.3)$) [above] {\texttt{0}};
\node at ($0.5*(B)+0.5*(C)+(0,0.3)$) [above] {\texttt{0}};
\node at ($0.5*(C)+0.5*(D)+(0,0.3)$) [above] {\texttt{0}};
\node at ($0.5*(D)+0.5*(E)+(0,0.3)$) [above] {\texttt{0}};
\node at ($(C)+(0,1.3)$) [above] {\texttt{0}};

\draw[<-,shorten >= 0.3cm,shorten <= 0.3cm]
	(A) to[out=-30,in=-150] (B);
\draw[<-,shorten >= 0.3cm,shorten <= 0.3cm]
	(B) to[out=-30,in=-150] (C);
\draw[<-,shorten >= 0.3cm,shorten <= 0.3cm]
	(C) to[out=-30,in=-150] (D);
\draw[<-,shorten >= 0.3cm,shorten <= 0.3cm]
	(D) to[out=-30,in=-150] (E);

\draw[<-,shorten >= 0.3cm,shorten <= 0.3cm]
	(E) to[out=-110,in=0] ($(E)-(1.0,1.3)$)
	--
	($(A)-(-1.0,1.3)$) to[out=180,in=-70] (A);

\node at ($0.5*(A)+0.5*(B)-(0,0.3)$) [below] {\texttt{1}};
\node at ($0.5*(B)+0.5*(C)-(0,0.3)$) [below] {\texttt{1}};
\node at ($0.5*(C)+0.5*(D)-(0,0.3)$) [below] {\texttt{1}};
\node at ($0.5*(D)+0.5*(E)-(0,0.3)$) [below] {\texttt{1}};
\node at ($(C)-(0,1.3)$) [below] {\texttt{1}};

\end{tikzpicture}
\end{center}
beschreibt diesen Automaten.
Der Zustand $q_k$ codiert den Fünferrest $k$ der Differenz
$|w|_{\texttt{0}}-|w|_{\texttt{1}}$.
\end{loesung}

\begin{bewertung}
Korrekte Idee für ein Konstruktionsprinzip ({\bf I}) 1 Punkt,
Startzustand ({\bf S}) 1 Punkt,
Akzeptierzustände ({\bf A}) 1 Punkt,
Automat ist deterministisch ({\bf D}) 1 Punkt,
Reihenfolge von \texttt{0} und \texttt{1} spielt keine Rolle ({\bf R}) 1 Punkt,
Automat akzeptiert genau die Sprache $L$ ({\bf L}) 1 Punkt.
\end{bewertung}
