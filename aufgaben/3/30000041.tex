Ist die Sprache 
\[
L=\{\texttt{a}^n\texttt{b}^m\;|\; \text{$n$ und $m$ sind teilerfremd}\}
\]
regulär?

\themaL{regular}{regulär}
\themaL{Pumping Lemma fur regulare Sprachen}{Pumping Lemma für reguläre Sprachen}

\begin{loesung}
Die Sprache ist nicht regulär, wie man mit dem Pumping-Lemma zeigen
kann. Nehmen wir an, $L$ sei regulär und habe Pumping Length $N$.
Dann wählen wir ein Wort $\texttt{a}^N\texttt{b}^n$, wobei
$n$ eine beliebige zu $N$ teilerfremde Zahl sein muss.
Nach dem Pumping-Lemma kann man das Wort aufpumpen, was mit einem
aus lauter Zeichen $\texttt{a}$ bestehenden Wort der Länge $l$
geschieht, $l\le N$. Die aufgepumpten Wörter haben die Form
\[
\texttt{a}^{N+kl}\texttt{b}^n.
\]
Diese Wörter sind nur dann alle in $L$, wenn $N+kl$ und $n$ für alle k
teilerfremd sind.

Wir können $n$ beliebig wählen, insbesondere können wir für $n$ eine
beliebige Primazahl $p$ wählen, die $N$ nicht teilt. Ausserdem können
wir $p$ so wählen, dass sie auch $l$ nicht teilt, einfach indem
wir sie noch weiter vergrössern. Wir möchten wissen, ob $N+kl$ ein
Vielfaches von $p$ sein kann. Dies ist gleichbedeutend mit der Gleichung
\[
N+kl=0\quad \text{in $\mathbb F_p$}
\]
Da aber sowohl $N$ wie auch $l$ in $\mathbb F_p$ invertierbar sind,
muss es ein $k\in\mathbb F_p$ geben, mit $k=N/l$. Also gibt es eine 
Zahl $k\le p$ so, dass $N+kl$ durch $p$ teilbar ist, womit ein Widerspruch
zu der Annahme erreicht wurde, dass $L$ regulär sei. Also ist $L$
nicht regulär.
\end{loesung}

