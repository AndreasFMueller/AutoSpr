Der Universally Unique Identifier (UUID) wurde im Distributed Computing
Environment definiert und in RFC4122 standardisiert.
Er besteht aus den folgenden Komponenten
\begin{center}
\begin{tabular}{|c|l|l|l|}
\hline
Nummer&Feldname                          &Datentyp          &Beschreibung\\
\hline
  1   &time\_low                         &\texttt{uint32\_t}&Zeitstempel niederwertigste 32bits\\
  2   &time\_mid                         &\texttt{uint16\_t}&Zeitstempel, mittlere 16bits\\
  3   &time\_hi\_and\_version            &\texttt{uint16\_t}&Zeitstempel, Versionsnummer\\
  4   &clock\_seq\_high\_and\_reserverved&\texttt{uint8\_t} &Clocksequenz\\
  5   &clock\_seq\_low                   &\texttt{uint8\_t} &Clocksequenz\\
  6   &node                              &\texttt{uint48\_t}&Node-ID, z.~B.~MAC Adresse\\
\hline
\end{tabular}
\end{center}
Eine UUID ist also genau 16 Bytes lang.
Mit Hilfe verschiedener Werte von clock\_seq können auf einer Maschine
(d.~h.~mit gleicher Node-ID) verschiedene UUIDs mit dem gleichen Zeitstempel
erzeugt werden.
Solche UUIDs werden zum Beispiel verwendet im Distributed Computing
Environment (DCE), in Microsofts GUID,
in der Java Klasse \texttt{java.util.UUDI},
in der Qt-Klasse \texttt{QUuid} oder
im ITU Standard X.667.

Für die praktische Anwendung wurde im RFC4122 eine String-Darstellung
von UUIDs definiert.
Sie wird dadurch erzeugt, dass alle Zahlen wenn nötig mit führenden
Nullen als Hexadezimalzahlen codiert und verkettet werden.
Zur besseren Lesbarkeit wird nach den Feldern 1, 2, 3 und 5 jeweils 
ein einzelner Bindestrich `\texttt{-}' eingefügt.
Gross- und Kleinschreibung spielt keine Rolle.

\begin{teilaufgaben}
\item
Geben Sie einen regulären Ausdruck, der genau die korrekten UUIDs akzeptiert.
\item
Microsofts Component Object Model verwendet
UUIDs zur Identifikation von Objekten, sie heissen GUID.
Allerdings sind die Werte des Feldes clock\_seq\_high\_and\_reserved
jetzt nicht mehr beliebig, die ersten drei Bits dieses Feldes sind immer
\texttt{100}.
Geben Sie einen regulären Ausdruck, der genau die korrekten GUID
akzeptiert.
\end{teilaufgaben}

\begin{loesung}
\begin{teilaufgaben}
\item
Hex-Zahlen sind Zeichenketten aus einer geraden Anzahl von
Ziffern \texttt{[0-9]} und den Buchstaben \text{[a-fA-F]}.
Die Anzahl Ziffern ist jeweils ein Viertel der Bitlänge des Feldes.
Wir können die Länge $n$ mit Konstrukt \texttt{\{$n$\}} angeben,
und erhalten somit den regulären Ausdruck:
\begin{center}
\texttt{[0-9a-fA-F]\{8\}-%
[0-9a-fA-F]\{4\}-%
[0-9a-fA-F]\{4\}-%
[0-9a-fA-F]\{4\}-%
[0-9a-fA-F]\{12\}}
\end{center}
\item
Das erste Zeichen des zweitletzten Feldes kann jetzt nur noch
\texttt{8} (binär \texttt{1000}) oder \texttt{9} (binär \texttt{1001})
sein, der reguläre Ausdruck muss daher wie folgt abgeändert
werden:
\begin{center}
\texttt{[0-9a-fA-F]\{8\}-%
[0-9a-fA-F]\{4\}-%
[0-9a-fA-F]\{4\}-%
(8|9)[0-9a-fA-F]\{3\}-%
[0-9a-fA-F]\{12\}}
\end{center}
\qedhere
\end{teilaufgaben}
\end{loesung}

\begin{bewertung}
nur Hex-Ziffern ({\bf H}) 1 Punkt,
Gross- und Kleinschreibung ({\bf K}) 1 Punkt,
Längen der Hex-Blöcke festgelegt ({\bf L}) 1 Punkt,
Bindestriche als Trennzeichen ({\bf B}) 1 Punkt,
Hex-Codierung von \texttt{1000} und \texttt{1001} ({\bf 8}) 1 Punkt,
Einschränkung in Teilaufgabe b) ({\bf E}) 1 Punkt.
\end{bewertung}

