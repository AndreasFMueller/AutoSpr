Eine Sprache besteht aus denjenigen Strings
über dem Alphabet $\Sigma=\{{\tt 0},{\tt 1}\}$, in welchen Folgen von
aufeinanderfolgenden {\tt 0}
immer gerade Länge und Folgen von aufeinanderfolgenden
{\tt 1} immer durch drei teilbare Länge haben.
Zur Sprache gehören also beispielsweise
\[
\text{\tt 001110000111},\;
\text{\tt 111111},\;
\text{\tt 001110011100},\;
\text{\tt 00000000111000000},
\]
nicht dazu gehören dagegen
\[
\text{\tt 10101},\;
\text{\tt 1100},\;
\text{\tt 1110},\;
\text{\tt 0}.
\]
Ist diese Sprache regulär? Wenn nein, warum nicht? Wenn ja, zeichnen
Sie das Zustandsdiagramm eines DEA, der diese Sprache akzeptiert.

\thema{regulär}
\thema{reguläre Ausdrücke}
\thema{NEA}

\begin{loesung}
Der reguläre Ausdruck
\[
((\text{\tt 00})|(\text{\tt 111}))*
\]
beschreibt diese Sprache, woraus mindestens zu erkennen ist, dass die
Sprache regulär ist, also auch ein deterministischer endlicher Automat
existiert, der sie akzeptiert.

Um das Zustandsdiagramm eines endlichen Automaten zu zeichnen,
brauchen wir auf jeden Fall die folgenden
Teilautomaten, die Folgen aus einer geraden Anzahl von {\tt 0}
bzw.~aus einer durch drei teilbaren Anzahl von {\tt 1} erkennen:
\[
\entrymodifiers={++[o][F]}
\xymatrix @-1mm {
*+\txt{}\ar[r]
        &*++[o][F=]{e}\ar@/^/[r]^{\tt 0}
                &{o} \ar@/^/[l]^{\tt 0}
\\
*+\txt{}\ar[r]
        &*++[o][F=]{0}\ar@/^/[r]^{\tt 1}
                &{1}\ar@/^/[r]^{\tt 1}
                        &{2} \ar@/^10pt/[ll]^{\tt 1}
}
\]
Ein "Ubergang mit einer {\tt 1} ist im ersten Automaten nur
zulässig, wenn er im Zustand $e$ ist. Im zweiten Automaten ist
ein "Ubergang mit {\tt 0} nur zulässig, wenn er im Zustand $0$ ist.
Wir brauchen also noch einen zusätzlichen Zustand $x$, mit dem wir
ausdrücken, dass das gelesene Wort niemals zur Sprache gehören
kann:
\[
\entrymodifiers={++[o][F]}
\xymatrix @-1mm {
*+\txt{}
        &*+\txt{}
                &{o} \ar@/^/[dl]^{\tt 0} \ar[dr]^{\tt 1}
\\
*+\txt{}\ar[r]
        &*++[o][F=]{0/e}\ar@/^/[ur]^{\tt 0} \ar@/^/[dr]^{\tt 1}
                &*+\txt{}
                        &{x}\ar@(ur,dr)^{\tt 0,1}
\\
*+\txt{}
        &*+\txt{}
                &{1}\ar@/^/[dr]^{\tt 1} \ar[ur]^{\tt 0}
\\
*+\txt{}
        &*+\txt{}
                &*+\txt{}
                        &{2} \ar@/^10pt/[uull]^{\tt 1} \ar[uu]^{\tt 0}
}
\]
Dieser Automat ist deterministisch, in jedem Zustand entspringen
zwei Pfeile für {\tt 0} und {\tt 1}. Somit ist er der gesuchte DEA.

Alternativ liesse sich der Automat durch Anwendung des Algorithmus zur
Umwandlung eines regulären Ausdruck in einen DEA gewinnen. Oder
indem man die Teil-NEAs zuerst in DEAs umwandelt und dann die
Konstruktion des kartesischen Produkt-Automaten anwendet, um die
Wörter zu akzeptieren, die beide Bedingungen erfüllen.
\end{loesung}
