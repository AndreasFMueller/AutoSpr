Betrachten Sie über dem Alphabet $\Sigma = \{{\tt 0},{\tt 1},\blank\}$ die
Sprache $L$, deren Wörter aus zwei durch genau ein Leerzeichen getrennten
Binärzahlen bestehen, wobei die zweite grösser sein muss als die erste.
Zur Sprache gehören also zum Beispiel
\[
{\tt 10000}\blank{\tt 10001},\quad
{\tt 0}\blank{\tt 10001}
\]
aber nicht
\[
{\tt 10000}\blank{\tt 10000},\quad
{\tt 10000}\blank\blank\blank{\tt 10001},\quad
{\tt 10001}\blank{\tt 0}.
\]
Ist $L$ regulär?

\themaL{regular}{regulär}
\themaL{Pumping Lemma fur regulare Sprachen}{Pumping Lemma für reguläre Sprachen}

\begin{loesung}
Nein, wie man mit dem Pumping Lemma nachweisen kann.
Wäre $L$ regulär, müsste es eine Pumping Length $N$ geben, so dass
Wörter mit mindestens Länge $N$ aufgepumpt werden können. Wir wählen
das Wort
\begin{align*}
w&= {\tt 1}^N\blank{\tt 10}^N
&
w&=\underbrace{\tt 1111\dots 1}_{\text{$N$ Einsen}}\blank{\tt 1}\underbrace{\tt 0000\dots 0}_{\text{$N$ Nullen}}
\end{align*}
es erfüllt die Bedingung $w\in L$. Da $|w| = 2N+2>N$ muss es eine
Unterteilung $w=xyz$ geben so, dass $|xy|\le N$ ist, also $x$ und $y$
müssen vollständig aus Einsen bestehen. Es ist also $y={\tt 1}^n$
mit $n > 0$.
Pumpt man diese Wort auf, erhöht sich nur die Zahl der Einsen im Teil
vor dem Leerzeichen:
\begin{align*}
xy^2z&={\tt 1}^{N+n}\blank{\tt 10}^N
&
xy^2z&=
\underbrace{\tt 1111\dots 1}_{N+n}\blank{\tt 1}\underbrace{\tt 0000\dots 0}_{N}
\end{align*}
Die Zahl links vom Blank ist $2^{N+n}-1$, die Zahl rechts ist $2^N$,
aber $2^{N+n}-1 > 2^N$, das aufgepumpte Wort kann also nicht mehr
in $L$ sein.

Dieser Widerspruch zeigt, dass $L$ nicht regulär sein kann.
\end{loesung}

\begin{bewertung}
Pumping Lemma ({\bf PL}) 1 Punkt,
Pumping Length ({\bf N}) 1 Punkt,
Konstruktion eines geeigneten Wortes ausgehend von der Pumping Length
({\bf W}) 1 Punkt,
Zerlegung des Wortes $w=xyz$ ({\bf Z}) 1 Punkt,
Konsequenzen bei Auf- oder Abpumpen ({\bf A}) 1 Punkt,
Schlussfolgerung ({\bf S}) 1 Punkt.
\end{bewertung}

