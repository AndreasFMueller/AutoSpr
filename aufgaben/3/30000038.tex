Finden Sie einen regul"aren Ausdruck sowie einen deterministischen endlichen
Automaten f"ur die Sprache $L$ "uber dem 
Alphabet $\Sigma = \{\texttt{a},\texttt{b}\}$ bestehend aus W"ortern,
in denen der Buchstabe \texttt{a} nie einzeln dasteht, sondern immer
nur in Gruppen von mindestens $2$ Zeichen \texttt{a}.

\begin{loesung}
Eine Gruppe von mindestens zwei Zeichen \texttt{a} kann durch den
regul"aren Ausdruck \texttt{aaa*} dargestellt werden.
Solche Gruppen k"onnen jetzt mit beliebig vielen Zeichen \texttt{b} verbunden werden:
\texttt{(aaa*b*)*}. Dieser Regul"are Ausdruck verlangt aber, dass W"orter
mit \texttt{a} beginnen m"ussen, was nicht den urspr"unglichen Forderungen
entspricht. Es ist auch erlaubt, dass ein Wort mit beliebig vielen
Zeichen \texttt{b} beginnt. Der fertige regul"are Ausdruck wird damit zu
\[
\texttt{b*(aaa*b*)*}
\]
Ein deterministischer endlicher Automat f"ur die Sprache $L$ ist
\[
\entrymodifiers={++[o][F]}
\xymatrix @-1mm {
*+\txt{}\ar[r]
	&*++[o][F=]{q_0}\ar[r]^{\texttt{a}}
	      \ar@(dr,dl)^{\texttt{b}}
		&{q_1}\ar[r]^{\texttt{a}} \ar[d]^{\texttt{b}}
			&*++[o][F=]{q_2} \ar@/_20pt/[ll]_{\texttt{b}}
				\ar@(ur,dr)^{\texttt{a}}
\\
*+\txt{}
	&*+\txt{}
		&{e}\ar@(dl,dr)_{\texttt{a},\texttt{b}}
}
\]
Selbstverst"andlich kann man aus diesem endlichen Automaten auch einen
regul"aren Ausdruck gewinnen. Dazu ist zun"achst der endlich Automat
mit einem neuen Start- und einem einzigen Akzeptierzustand umzuformen:
\[
\entrymodifiers={++[o][F]}
\xymatrix @-1mm {
*+\txt{}\ar[r]
	&{s}\ar[r]^{\varepsilon}
		&{q_0}\ar[d]^{\texttt{a}} \ar[rr]^{\varepsilon}
			\ar@(ul,ur)^{\texttt{b}}
			&*+\txt{}
				&*++[o][F=]{a}
\\
*+\txt{}
	&{e} \ar@(ul,dl)_{\texttt{a}|\texttt{b}}
		&{q_1}\ar[r]^{\texttt{a}} \ar[l]_{\texttt{b}}
			&{q_2}\ar[ur]_{\varepsilon} \ar[ul]_{\texttt{b}}
				\ar@(dl,dr)_{\texttt{a}}
}
\]
Da keine Pfade "uber $e$ zu einem Akzeptierzustand f"uhren, kann man
in dem VNEA zur Ermittlung des regul"aren Ausdrucks diesen Zustand
weglassen:
\[
\entrymodifiers={++[o][F]}
\xymatrix @-1mm {
*+\txt{}\ar[r]
	&{s}\ar[r]^{\varepsilon}
		&{q_0}\ar[d]^{\texttt{a}} \ar[rr]^{\varepsilon}
			\ar@(ul,ur)^{\texttt{b}}
			&*+\txt{}
				&*++[o][F=]{a}
\\
*+\txt{}
	&*+\txt{}
		&{q_1}\ar[r]^{\texttt{a}}
			&{q_2}\ar[ur]_{\varepsilon} \ar[ul]_{\texttt{b}}
				\ar@(dl,dr)_{\texttt{a}}
}
\]
Jetzt muss man alle Zust"ande entfernen, wir beginnen mit $q_1$:
\[
\entrymodifiers={++[o][F]}
\xymatrix @-1mm {
*+\txt{}\ar[r]
	&{s}\ar[r]^{\varepsilon}
		&{q_0} \ar[rr]^{\varepsilon}
			\ar@(ul,ur)^{\texttt{b}}
			\ar@/_/[dr]_{\texttt{aa}}
			&*+\txt{}
				&*++[o][F=]{a}
\\
*+\txt{}
	&*+\txt{}
		&*\txt{}
			&{q_2}\ar[ur]_{\varepsilon} \ar@/_/[ul]_{\texttt{b}}
				\ar@(dl,dr)_{\texttt{a}}
}
\]
Jetzt entfernen wir Zustand $q_2$:
\[
\entrymodifiers={++[o][F]}
\xymatrix @-1mm {
*+\txt{}\ar[r]
	&{s}\ar[r]^{\varepsilon}
		&{q_0} \ar[rr]^{\varepsilon|\texttt{aaa*}}
			\ar@(ul,ur)^{\texttt{b}|\texttt{aaa*b}}
			&*+\txt{}
				&*++[o][F=]{a}
}
\]
Und zum Schluss wird jetzt noch $q_0$ entfernt:
\[
\entrymodifiers={++[o][F]}
\xymatrix @-1mm {
*+\txt{}\ar[r]
	&{s}\ar[rrr]^{\texttt{(b|aaa*b)*(|aaa*)}}
		&*+\txt{}
			&*+\txt{}
				&*++[o][F=]{a}
}
\]
Der regul"are Ausdruck
\texttt{(b|aaa*b)*(|aaa*)}
ist also auch eine m"ogliche L"osung der Aufgabe.

Die urspr"ungliche Formulierung verlangte, dass \texttt{a} in Gruppen von mindestens
zwei Zeichen auftreten musste, was man auch so verstehen konnte, dass einzig das
alleinstehende Zeichen \texttt{a} nicht zul"assig ist. Diese Sprache ist nat"urlich
auch regul"ar.

$L=\Sigma^* \setminus \{\texttt{a}\}$. Da die Sprache $\{\texttt{a}\}$
endlich ist, ist sie regul"ar. $\Sigma^*$ ist auch regul"ar. Die Differenz regul"arer
Sprachen ist ebenfalls regul"ar. Man muss also nur noch einen endlichen Automaten
und einen regul"aren Ausdruck daf"ur finden.

Der regul"are Ausdruck muss ausdr"ucken, dass ein mit \texttt{a} beginnendes Wort
noch mindestens ein Zeichen haben muss. F"ur W"orter, die mit \texttt{b} beginnen,
gibt es keine weiteren Bedingungen. Also:
\[
\texttt{a.+|b.*}
\]
Auf dieser Basis l"asst sich auch leicht ein endlicher Automat formulieren
\[
\entrymodifiers={++[o][F]}
\xymatrix @-1mm {
*+\txt{}\ar[r]
	&*++[o][F=]{q_0} \ar[r]^{\texttt{a}} \ar[d]_{\texttt{b}}
		&{q_a} \ar[dl]^{\Sigma}
\\
*+\txt{}
	&*++[o][F=]{q_b} \ar@(dl,dr)_{\Sigma}
}
\]
\end{loesung}

\begin{bewertung}
Regul"arer Ausdruck: stellt sicher, dass \texttt{a} immer mindestens als
Gruppen \text{aa} vorkommt ($\textbf{A}_r$) 1 Punkt, \texttt{b} kann beliebig
vorkommen ($\textbf{B}_r$) 1 Punkt, W"orter k"onnen mit \texttt{b} beginnen und
aufh"oren ($\textbf{E}_r$) 1 Punkt. Analog  
$\textbf{A}_r$ und
$\textbf{B}_r$ f"ur den deterministischen endlichen Automaten.
Automat ist eine DEA (kein NEA) (\textbf{D}) 1 Punkt.
\end{bewertung}


