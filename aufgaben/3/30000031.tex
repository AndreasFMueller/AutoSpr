Die ``International Bank Account Number'' vereinfacht den
grenzüberschreitenden Zahlungsverkehr durch ein vereinheitlichtes
Format. Die IBAN beginnt immer mit der zweistelligen ISO-Länderkennzeichnung
({\tt CH} für die Schweiz, {\tt GR} für Griechenland,$\dots$). Danach
folgen zwei bis drei Stellen für eine Prüfzahl (Italien verwendet
drei Stellen, der Rest Europas zwei).
Danach folgen weitere Ziffern, die die Bank und das Konto bezeichnen.
Der kontospezifische Teil kann weitere bankspezifische Prüfziffernteile
enthalten, die jedoch nicht standardisiert sind, und uns hier nicht
interessieren.
Insgesamt ist die IBAN höchstens 34 Stellen lang.
Beispielsweise ist
\begin{center}
{\tt DE88200800000970375700}
\end{center}
eine gültige IBAN Nummer.

Die Berechnung der Prüfziffer funktioniert wie folgt.
Zunächst
werden die beiden Buchstaben des Ländercodes durch Zahlenwerte
beginnend mit $\text{\tt A}=10$ ersetzt, und für die zweistellige
Prüfzahl werden {\tt 0} eingesetzt. Dann werden diese sechs Stellen
vorne entfernt und hinten an die IBAN angehängt. Für obiges Beispiel
wird also {\tt D} durch $13$ und {\tt E} durch $14$ ersetzt, so dass dass die 
IBAN zunächst zu
\begin{center}
{\tt 131488200800000970375700}
\end{center}
Dann werden die ersten 6 Stellen nach hinten verschoben:
\begin{center}
{\tt 200800000970375700131400}
\end{center}
Dann wird von dieser Zahl der Rest bei Division durch 97 ermittelt:
in diesem Fall 10. Die Prüfzahl ist die Differenz zu 98, in diesem
Fall also $98-10=88$, die korrekte IBAN ist also
\begin{center}
{\tt DE88200800000970375700}
\end{center}

Die Modulo-Berechnung für grosse Zahlen ist etwas unhandlich.
Gibt es eventuell einen endlichen Automaten, mit dem man die IBANs
auch auf Korrektheit prüfen kann?

\begin{hinweis}
Es wird nicht verlangt, dass Sie einen solchen Zustandsautomaten angeben,
sondern höchstens dass sie darlegen, wie man gegebenenfalls zu einem solchen
Zustandsautomaten kommen könnte, sollte es einen geben.
\end{hinweis}

\themaL{regular}{regulär}
\thema{DEA}

\begin{loesung}
Ja, da IBANs eine endliche, wenn auch sehr grosse ($25^2\cdot 10^{32}$)
Sprache bilden, ist die Menge der IBANs regulär.
Insbesondere gibt es einen regulären Ausdruck, der genau die 
korrekten IBANs akzeptiert.

Einen solchen regulären Ausdruck zu finden, ist jedoch nicht einfach.
Man kann aber wie folgt vorgehen: In der Vorlesung wurde gezeigt,
dass man den Rest bei einer Division durch 97 mit einem endlichen
Automaten bestimmen kann. Die obige Beschreibung der Prüfzahl
läuft darauf hinaus, dass man überprüfen muss, ob die ersten
vier Stellen den zum zweiten Teil ``passenden'' Rest bei Teilung
durch 97 liefert.

Man formuliert also zunächst einen Automaten $A_1$,
welcher genau die Reste von der ersten vier Stellen ermittelt, seine
Zustände sind die möglichen Reste $0$ bis $96$.
Dann konstruiert man einen Automaten $A_2$, welcher den Rest 
des zweiten Teils ermittelt, wenn diesem noch 6 Stellen mit $0$
folgen würden.

Dieser Rest muss jetzt zusammen mit dem von $A_1$
bestimmten Rest einen bestimmten Wert ergeben. Aus $A_1$ und $A_2$ bildet
man den Produktautomaten und markiert alle Paare von Zuständen die
in diesem Sinne ``zusammenpassen'' also Akzeptiertzustände.
Um einen regulären Ausdruck zu finden, wandelt man diesen endlichen
Automaten mit dem Standardalgorithmus in einen regulären Automaten
um.
\end{loesung}
