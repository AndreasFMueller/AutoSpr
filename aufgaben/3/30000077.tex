Vor der Wordle-Craze des Jahres 2022 gabe es als sehr erfolgreichen
Vorläufer das 1970 erfundene Spiel Mastermind, welches in den 1970er-Jahren
eines der erfolgreichsten Spiele war.
\begin{center}
\includeagraphics[width=6cm]{Mastermind.jpeg}
\end{center}
Der eine Spieler, der Codierer, gibt verdeckt einen Code aus
vier farbigen Stiften vor.
Der andere Spieler, der Rater, versucht jetzt, diesen Farbcode zu ermitteln.
Er schlägt eine mögliche Lösung vor, indem er farbige Stifte in die 
Lochplatte steckt (siehe Abbildung).
Der Codierer antwortet dann, indem er mit kleinen Stiften bekannt gibt,
wieviele der farbigen Stifte am richtigen Ort (schwarzer kleiner Stift)
sind oder nur die richtige Farbe haben (weisser kleiner Stift).
Im Gegensatz zu Wordle erfährt der Rater also nicht, an welcher Position
sich die Stifte mit der richtigen Farbe befinden.
Der Rater versucht dann die nächste Kombination, wobei er natürlich das
bereits gewonnene Wissen über den Farbcode wird berücksichtigen wollen.

Jede Antwort des Codierers definiert eine Einschränkung für die noch
möglichen Wörter.
Sie definiert damit eine Sprache von Farbcodes über dem Alphabet
der verfügbaren Farben.

\begin{teilaufgaben}
\item
Ist die Sprache regulär?
\item
Nehmen Sie an, dass der Rater die Farbkombination \texttt{RRGB}
versucht hat und der Codierer die Antwort \texttt{ss} gegeben hat,
was bedeuten soll, dass zwei Stifte mit der richtigen Farbe am richtigen
Ort sind.
Konstruieren Sie einen endlichen Automaten, der
genau die Wörter akzeptiert, die die Bedingung erfüllen.
\item
Nehmen Sie an, dass Sie wie in Teilaufgabe b)
auch für die weissen kleinen Stifte (eine Antwort \texttt{w} zum Beispiel)
einen endlichen Automaten konstruieren können (Sie müssen diesen Automaten
nicht finden).
Gibt es eine Möglichkeit,
aus diesen beiden Automaten (dem von Teilaufgabe b) und dem neuen für die
Bedingung \texttt{w}) einen Automaten zu konstruieren, der genau
diejenigen Wörter akzeptiert, die die kombinierte Bedingung \texttt{ssw}
akzeptiert?
\end{teilaufgaben}

\begin{hinweis}
Sie dürfen zur Vereinfachung annehmen, dass es nur drei Farben
$\Sigma=\{\texttt{R},\texttt{G},\texttt{B}\}$ gibt.
\end{hinweis}

\begin{loesung}
\begin{teilaufgaben}
\item
Die Sprache ist endlich und daher auch regulär.
Alternativ kann auch aus der Konstruktion der nachfolgenden Teilaufgaben
schliessen, dass es einen endlichen Automaten gibt, der die Sprache
akzeptiert.
Damit ist sie ebenfalls als regulär nachgewiesen.
\item
Man kann einen NEA in ``Baum-Form'' aufbauen, in dem alle Wörter mit Länge
vier vorkommen  und dann Akzeptierzuständ dorthin setzen, wo ein Wort
hinführt, welches auf die Bedingung passt.
Dies sind die Wörter, die genau zwei Farben am richtigen Ort haben.
Alle Wörter, die Farben nicht am richtigen Ort haben, zeichnen wir gar
nicht erst ein, da es sich ja um einen NEA handelt.

Man könnte, da die Sprache endlich ist, auch eine Liste der Wörter machen,
die passen.
Dazu überlegt man sich, dass es sechs Möglichkeiten gibt, wo die beiden
richtigen Farben sind:
\[
\texttt{RR\blank\blank}
\quad
\texttt{R\blank G\blank}
\quad
\texttt{R\blank\blank B}
\quad
\texttt{\blank RG\blank}
\quad
\texttt{\blank R\blank B}
\quad
\texttt{\blank \blank GB}
\]
Die übrigen Plätze kann man jetzt mit Zeichen ausstatten, die falsch sind aber auch
kein \texttt{w} auslösen:
\begin{center}
\def\b{\blank}
\begin{tabular}{|cccccc|}
\hline
\texttt{RR\b\b}&\texttt{R\b G\b}&\texttt{R\b\b B}&\texttt{\b RG\b}&\texttt{\b R\b B}&\texttt{\b\b GB}\\
\hline
\texttt{RRRR}&\texttt{RGGG}&\texttt{RBBB}&\texttt{GRGG}&\texttt{BRBB}&\texttt{GGGB}\\
             &             &             &             &             &\texttt{BBGB}\\
             &             &             &             &             &\texttt{BGGB}\\
             &             &             &             &             &\texttt{GBGB}\\
\hline
\end{tabular}
\end{center}
Man kann zum Beispiel in den ersten 5 Fällen kein weitere $\texttt{R}$ platzieren,
weil dies zu einem \texttt{w} führen würde.
Für diese Wörter kann man einen endlichen Automaten angeben:
\begin{center}
\def\zustand#1{\draw #1 circle[radius=0.2];}
\def\azustand#1{\draw #1 circle[radius=0.2]; \draw #1 circle[radius=0.13];}
\def\pfeil#1#2#3{
	\draw[->,shorten >= 0.2cm,shorten <= 0.2cm] #1 -- #2;
	\node at ($0.5*#1+0.5*#2$) [above] {\texttt{#3}};
}
\begin{tikzpicture}[>=latex,thick]
\coordinate (S) at (0,0);	\zustand{(S)} \pfeil{(-2,0)}{(S)}{}

\coordinate (R) at (2,3);	\zustand{(R)} \pfeil{(S)}{(R)}{R}
\coordinate (RR) at (4,4);	\zustand{(RR)} \pfeil{(R)}{(RR)}{R}
\coordinate (RRR) at (6,4);	\zustand{(RRR)} \pfeil{(RR)}{(RRR)}{R}
\coordinate (RRRR) at (8,4);	\azustand{(RRRR)} \pfeil{(RRR)}{(RRRR)}{R}
\coordinate (RG) at (4,3);	\zustand{(RG)} \pfeil{(R)}{(RG)}{G}
\coordinate (RGG) at (6,3);	\zustand{(RGG)} \pfeil{(RG)}{(RGG)}{G}
\coordinate (RGGG) at (8,3);	\azustand{(RGGG)} \pfeil{(RGG)}{(RGGG)}{G}
\coordinate (RB) at (4,2);	\zustand{(RB)} \pfeil{(R)}{(RB)}{B}
\coordinate (RBB) at (6,2);	\zustand{(RBB)} \pfeil{(RB)}{(RBB)}{B}
\coordinate (RBBB) at (8,2);	\azustand{(RBBB)} \pfeil{(RBB)}{(RBBB)}{B}

\coordinate (G) at (2,0);	\zustand{(G)} \pfeil{(S)}{(G)}{G}
\coordinate (GR) at (4,1);	\zustand{(GR)} \pfeil{(G)}{(GR)}{R}
\coordinate (GRG) at (6,1);	\zustand{(GRG)} \pfeil{(GR)}{(GRG)}{G}
\coordinate (GRGG) at (8,1);	\azustand{(GRGG)} \pfeil{(GRG)}{(GRGG)}{G}
\coordinate (GG) at (4,0);	\zustand{(GG)} \pfeil{(G)}{(GG)}{G}
\coordinate (GGG) at (6,0);	\zustand{(GGG)} \pfeil{(GG)}{(GGG)}{G}
\coordinate (GGGB) at (8,0);	\azustand{(GGGB)} \pfeil{(GGG)}{(GGGB)}{B}
\coordinate (GB) at (4,-1);	\zustand{(GB)} \pfeil{(G)}{(GB)}{B}
\coordinate (GBG) at (6,-1);	\zustand{(GBG)} \pfeil{(GB)}{(GBG)}{G}
\coordinate (GBGB) at (8,-1);	\azustand{(GBGB)} \pfeil{(GBG)}{(GBGB)}{B}

\coordinate (B) at (2,-3);	\zustand{(B)} \pfeil{(S)}{(B)}{B}
\coordinate (BR) at (4,-2);	\zustand{(BR)} \pfeil{(B)}{(BR)}{R}
\coordinate (BRB) at (6,-2);	\zustand{(BRB)} \pfeil{(BR)}{(BRB)}{B}
\coordinate (BRBB) at (8,-2);	\azustand{(BRBB)} \pfeil{(BRB)}{(BRBB)}{B}
\coordinate (BG) at (4,-3);	\zustand{(BG)} \pfeil{(B)}{(BG)}{G}
\coordinate (BGG) at (6,-3);	\zustand{(BGG)} \pfeil{(BG)}{(BGG)}{G}
\coordinate (BGGB) at (8,-3);	\azustand{(BGGB)} \pfeil{(BGG)}{(BGGB)}{B}
\coordinate (BB) at (4,-4);	\zustand{(BB)} \pfeil{(B)}{(BB)}{B}
\coordinate (BBG) at (6,-4);	\zustand{(BBG)} \pfeil{(BB)}{(BBG)}{G}
\coordinate (BBGB) at (8,-4);	\azustand{(BBGB)} \pfeil{(BBG)}{(BBGB)}{B}
\end{tikzpicture}
\end{center}
\item
Mit Hilfe des Produktautomaten können die beiden Automaten so kombiniert
werden, dass genau diejenigen Wörter akzeptiert werden, die beide 
Bedingungen erfüllen.
Dies interpretiert die Fragestellung als Anforderung an eine UND-Verknüpfung.

Dies ist jedoch nicht ganz richtig, dazu muss man aber etwas vertrauter sein
mit dem Spiel als man in einer Prüfung erwarten kann.
Der Grund ist, dass eine Farbe an einem falschen Ort nicht mehr als \texttt{w}
zählt, wenn ein dieselbe Farbe schon am richtigen Ort platziert worden ist.
Im Beispiel: zwei \texttt{R} am Anfang erzeugen die beiden \texttt{s}, aber ein
weiteres \texttt{R} erzeugt kein \texttt{w} mehr.
Man kann natürlich wieder hingehen und die Wörter zusammenstellen, die die
kombinierte Bedingung erfüllen:
\begin{center}
\def\b{\blank}
\begin{tabular}{|cccccc|}
\hline
\texttt{RR\b\b}&\texttt{R\b G\b}&\texttt{R\b\b B}&\texttt{\b RG\b}&\texttt{\b R\b B}&\texttt{\b\b GB}\\
\hline
\texttt{RRRG}&\texttt{RBGR}&\texttt{RBRB}&\texttt{BRGG}&\texttt{GRBB}&             \\
\texttt{RRBR}&\texttt{RBGG}&\texttt{RGBB}&\texttt{GRGR}&\texttt{BRRB}&             \\
\hline
\end{tabular}
\end{center}
Aus dieser Tabelle lässt sich dann wieder nach obigem Muster ein
Automat erzeugen.

Ein weiteres Indiz, dass die Erzeugung eines DEA etwas schwieriger
ist, ist die Tatsache, dass es den in der Aufgabe postulierten
Automaten für \texttt{w} gar nicht gibt, wenn nur die drei Farben
\texttt{R}, \texttt{G} und \texttt{B} möglich sind.
Da man nicht alle Positionen mit nur einer Farbe belegen kann,
kommen mindestens zwei Farben am falschen Ort vor, es muss also
mindestens zwei \texttt{w} haben.
\end{teilaufgaben}
\end{loesung}


\begin{bewertung}
\begin{teilaufgaben}
\item
Endlichkeit oder anderes Regularitätsargument ({\bf R}) 1 Punkt,
\item
Konstruktion eines Automaten für Teilaufgabe b):
Startzustand vorhanden ({\bf S}) 1 Punkt,
Zwei Farben am richtigen Platz ({\bf Z}) 1 Punkt,
keine richtigen Farben am falschen Platz ({\bf F}) 1 Punkt,
Akzeptierzustände vorhanden ({\bf A}) 1 Punkt.
\item
Produktautomat oder ähnliches Argument ({\bf C}) 1 Punkt.
\end{teilaufgaben}
\end{bewertung}





