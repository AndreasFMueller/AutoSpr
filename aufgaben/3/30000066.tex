Auf der Website \url{http://regular-expression.info} findet man die folgenden
Beispiele von regulären Ausdrücken.
\begin{teilaufgaben}
\item
Der reguläre Ausdruck
\verb+<TAG [^>]*>.*</TAG>+
soll HTML-Tags und das zugehörige schliessende Element finden.
Warum funktioniert er nicht?
\item
Finden Sie einen regulären Ausdruck, welcher ein öffnendes Tag mit
dem Namen
\verb+TAG+
und das zugehörige schliessende Tag findet unter
der Annahme, dass der Inhalt nur aus gewöhnlichen Zeichen besteht und
keine geschachtelten Tags enthalten sind.
Ausserdem wird verlangt, dass das Tag ein \texttt{style}-Attribut hat.
\item
Was soll der reguläre Ausdruck
\begin{verbatim}
(19|20)\d\d[- /.](0[1-9]|1[012])[- /.](0[1-9]|[12][0-9]|3[01])
\end{verbatim}
wohl bezwecken (die Abkürzung \verb+\d+ steht für die Zeichenklasse
\verb+[0-9]+%
)?
Finden Sie mindestens zwei akzeptierte Wörter, die nicht diesem Zweck
entsprechen.
\end{teilaufgaben}

\themaL{regulare Ausdrucke}{reguläre Ausdrücke}
\themaL{regular}{regulär}

\begin{loesung}
\begin{teilaufgaben}
\item
Es wird nicht überprüft, ob Tags korrekt geschachtelt sind,
zum Beispiel wird
\texttt{<TAG ></TAG></TAG>}
akzeptiert.
Andererseits wird
\texttt{<TAG></TAG>}
nicht akzeptiert, weil der reguläre Ausdruck ein Leerzeichen nach dem
Tag-Namen im öffnenden Tag verlangt.
\item
Der Ausdruck
\begin{center}
\texttt{<(TAG|TAG [\char`\^>]*) style=\char`\"[\^{}\char`\"]*\char`\"[\char`\^{}>]*>.*</TAG>}
\end{center}
tut das gewünschte.
Die erste Alternative stellt sicher, dass der Ausdruck auch funktioniert,
wenn zusätzliche Attribute vor dem \texttt{style}-Attribute vorkommen.

Der Ausdruck könnte viel einfacher werden, wenn es ein Zeichen gäbe, mit dem
man eine Wortgrenze erkennen könnte.
Einige Regex-Dialekte verstehen das Zeichen \texttt{\textbackslash b}, welches
genau dies tut.
Damit kann man sicherstellen, dass nach \texttt{TAG} und vor \texttt{style}
eine Wortgrenze kommt.
Damit kann man sich die Alternative sparen, der Ausdruck wird dann
\begin{center}
\texttt{<TAG\textbackslash
b[\char`\^>]*\textbackslash b%
style=\char`\"[\^{}\char`\"]*\char`\"[\char`\^{}>]*>.*</TAG>%
}
\end{center}
\item
Der Ausdruck versucht, gültige Datumsangaben zu akzeptieren.
Dabei sind Jahreszahlen zwischen 1900 und 2099 akzeptiert,
Monatsangaben zwischen 01 und 12 Tagesangaben zwischen 01 und 31.
Es wird allerdings nicht überprüft, ob die Tageszahl zum Monat passt,
der String
\texttt{2020-02-31}
wird akzeptiert, obwohl der Februar höchstens
29 Tage haben kann.
Auch werden Separatoren akzeptiert, die nicht zu einem gültigen Datum
gehören.
So ist
\texttt{2020-02-02}
ein gültiges Datum, aber
\texttt{2020.02.02}
ist nicht sinnvoll genau so wenig wie gemischte Separator-Zeichen.
\qedhere
\end{teilaufgaben}
\end{loesung}

\begin{bewertung}
\begin{teilaufgaben}
\item
Schachtelung oder Leerzeichen ({\bf A}) 1 Punkt.
\item
Grundlegende Struktur ({\bf G}) 1 Punkt, 
\texttt{style}-Attribut ({\bf S}) 1 Punkt,
Problem der Leerzeichen im Element ({\bf L}) 1 Punkt.
\item
Zweck ({\bf Z}) 1 Punkt, Gegenbeispiele ({\bf B}) 1 Punkt.
\end{teilaufgaben}
\end{bewertung}
