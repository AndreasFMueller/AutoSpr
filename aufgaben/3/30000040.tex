In der Astronomie werden Punkte am Himmel mit Hilfe der Winkel
der geographischen Länge und Breite angegeben, die im astronomischen
Zusammenhang auch
Rektaszension und Deklination genannt werden.
Allerdings sind viele verschiedene Formate für die
Winkelangabe gebräuchlich, manchmal werden sogar im gleichen Sternkatalog
verschiedene Format verwendet.
Die Astronomen haben eine gute Intuition für einen Winkel in Minuten,
da der Vollmonddurchmesser ziemlich genau 30 Winkelminuten beträgt.
"Ubliche Formate sind
\begin{center}
\begin{tabular}{ll}
Format&Beispiel\\
\hline
dezimale Grade&\texttt{65.4321}\\
Grad, dezimale Minuten&\texttt{65 25.926}\\
Grad, Minuten, dezimale Sekunden&\texttt{65 25 55.56}\\
\hline
\end{tabular}
\end{center}
Natürlich kann der Nachkommateil auch fehlen, und es ist auch möglich, dass
ein Winkel negativ ist.
Stellen Sie einen regulären Ausdruck auf, der genau diese Formate
akzeptiert. 

\themaL{regular}{regulär}
\themaL{regulare Ausdrucke}{reguläre Ausdrücke}

\begin{loesung}
Eine Winkelangabe endet immer mit einem optionalen Nachkommateil
\begin{center}
\tt($\backslash$.[0-9]*)?
\end{center}
Davor stehen maximal drei Gruppen von Ziffern, die jeweils durch
Leerzeichen getrennt sind.
\begin{center}
\tt[0-9]+( *[0-9]+)\{0,2\}
\end{center}
Zusammengesetzt, und mit einem optionalen Vorzeichen versehen:
\begin{center}
\tt (|-|$\backslash$+)[0-9]+( *[0-9]+)\{0,2\}($\backslash$.[0-9]*)?
\end{center}

Für die Praxis wäre aber eine andere Form nützlicher, nämlich ein
regul"rare Ausdruck, der nicht nur die Gültigkeit des Formates
erkennt, sondern auch die einzelnen Teile identifiziert, damit man
diese Teilstrings gleich zur Berechnung des Winkelwertes heranziehen
kann. Dieser Ausdruck beginnt immer mit einer Gruppe von Ziffern
\begin{center}
\tt [0-9]+
\end{center}
Diese Gruppe kann jetzt entweder von einem Nachkommateil gefolgt sein
oder von einer weiteren Gruppe
\begin{center}
\tt [0-9]+($\backslash$.[0-9]*| {\rm weitere Gruppen } )?
\end{center}
Die weiteren Gruppen haben aber genau das gleiche Format wie das,
was der bisherige Ausdruck darstellt, man kann also den Ausdruck einfach
rekursiv nochmals einpacken
\begin{center}
\tt
(|-|$\backslash$+)[0-9]+($\backslash$.[0-9]*| [0-9]+($\backslash$.[0-9]*| [0-9]+($\backslash$.[0-9]*)?)?)?
\end{center}
Bei der letzten Gruppe kann nur ein Kommateil folgen, etwas kleineres
als Sekunden gibt es ja nicht.
\end{loesung}

\begin{bewertung}
Regulärer Ausdruck akzeptiert die drei Beispiele ({\bf B}) je 1 Punkt,
regulärer Ausdruck azkeptiert Vorzeichen +, - ({\bf V}) 1 Punkt,
regulärer Ausdruck akzeptiert Zahl von Nachkommateil ({\bf N}) 1 Punkt,
regulärer Ausdruck akzeptiert keine anderen Zahlen ({\bf X}) 1 Punkt.
\end{bewertung}

