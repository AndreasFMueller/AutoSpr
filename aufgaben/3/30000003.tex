Untersuchen Sie, welche der Zustände im Automaten
%\[
%\entrymodifiers={++[o][F]}
%\xymatrix{
%*+\txt{}\ar[r]
%        &{z_0}\ar[r]^{\tt 1} \ar@(ur,ul)_{\tt 0}
%                &*++[o][F=]{z_1} \ar@/^/[dl]^{\tt 0} \ar@(dr,ur)_{\tt 1}
%\\
%*+\txt{}
%        &{z_2} \ar@/^/[ur]^{\tt 1} \ar[u]^{\tt 0}
%}
%\]
\begin{center}
\begin{tikzpicture}[>=latex,thick]
\def\l{1.8}
\coordinate (z0) at (0,0);
\coordinate (z1) at (\l,0);
\coordinate (z2) at (0,-\l);
\draw (z0) circle[radius=0.35];
\draw (z1) circle[radius=0.35];
\draw (z1) circle[radius=0.30];
\draw (z2) circle[radius=0.35];
\node at (z0) {$z_0$};
\node at (z1) {$z_1$};
\node at (z2) {$z_2$};
\draw[->,shorten <= 0.35cm,shorten >= 0.35cm] (-\l,0) -- (z0);
\draw[->,shorten <= 0.35cm,shorten >= 0.35cm] (z0) -- (z1);
\draw[->,shorten <= 0.35cm,shorten >= 0.35cm] (z2) -- (z0);
\draw[->,shorten <= 0.35cm,shorten >= 0.35cm] (z2) -- (z1);
\draw[->,shorten <= 0.35cm,shorten >= 0.35cm] (z1) to[out=-100,in=10] (z2);
\draw[->,shorten <= 0.35cm,shorten >= 0.35cm]
	(z1) to[out=30,in=-30,distance=1.2cm] (z1);
\draw[->,shorten <= 0.35cm,shorten >= 0.35cm]
	(z0) to[out=60,in=120,distance=1.2cm] (z0);
\node at ($0.5*(z0)+0.5*(z1)$) [above] {\texttt{1}};
\node at ($0.5*(z0)+0.5*(z2)$) [left] {\texttt{0}};
\node at ($(z0)+(0,1.1)$) {\texttt{0}};
\node at ($(z1)+(1.1,0)$) {\texttt{1}};
\node at ($0.5*(z1)+0.5*(z2)+(0.05,-0.05)$) [above left] {\texttt{1}};
\node at ($0.5*(z1)+0.5*(z2)+(0.25,-0.25)$) [below right] {\texttt{0}};
\end{tikzpicture}
\end{center}
äquivalent sind, und finden Sie den minimalen Automaten.

\thema{minimaler Automat}
\thema{Zustandsdiagramm}

\begin{loesung}
$z_0$ und $z_2$ sind äquivalent, denn von beiden aus
werden genau die Wörter akzeptiert, die mit beliebig
vielen {\tt 0}  gefolgt von einer {\tt 1} beginnen.
$z_1$ ist mit keinem anderen Zustand äquivalent, weil im
Zustand $z_1$ das leere Wort akzeptiert werden kann, nicht
aber in den anderen Zuständen.

Mit dem in der Vorlesung besprochenen ``Kreuzchen-Algorithmus'' lässt sich
der Minimalautomat natürlich auch finden:
\begin{center}
\begin{tabular}{|>{$}c<{$}|>{$}c<{$}>{$}c<{$}>{$}c<{$}|}
\hline
              &z_0    &z_1/\texttt{E}&z_2    \\
\hline
z_0           &\equiv &\times        &       \\
z_1/\texttt{E}&\times &\equiv        &\times \\
z_2           &       &\times        &\equiv \\
\hline
\end{tabular}
\end{center}
Der \texttt{0}-"Ubergang von $z_0$ und $z_2$ führt zu $z_0$, der
\texttt{1}-"Ubergang zu $z_1$, in beiden Fällen kann man also
kein neues Kreuz im Feld $(z_0,z_2)$ machen, der Algorithmus endet.
Man schliesst, dass $z_0$ und $z_2$ äquivalent sein müssen.

Legt man $z_0$ und $z_2$ zusammen, wird der Automat vereinfacht zu
%\[
%\entrymodifiers={++[o][F]}
%\xymatrix{
%*+\txt{}\ar[r]
%        &{z_0,z_2}\ar@/^/[r]^{\tt 1} \ar@(ur,ul)_{\tt 0}
%                &*++[o][F=]{z_1} \ar@/^/[l]^{\tt 0} \ar@(dr,ur)_{\tt 1}
%}
%\qedhere
%\]
\begin{center}
\begin{tikzpicture}[>=latex,thick]
\coordinate (z0) at (0,0);
\coordinate (z1) at (2.4,0);
\draw (z1) circle[radius=0.35];
\draw (z1) circle[radius=0.30];
\draw (z0) ellipse (0.6 and 0.35);
\node at (z0) {$z_0,z_2$};
\node at (z1) {$z_1$};
\draw[->,shorten <= 0.55cm,shorten >= 0.35cm] (z0) to[out=20,in=160] (z1);
\draw[->,shorten <= 0.35cm,shorten >= 0.55cm] (z1) to[out=-160,in=-20] (z0);
\draw[->,shorten <= 0.35cm,shorten >= 0.65cm] (-1.9,0) -- (z0);
\draw[->,shorten <= 0.40cm,shorten >= 0.40cm]
	(z0) to[out=60,in=120,distance=1.2cm] (z0);
\draw[->,shorten <= 0.35cm,shorten >= 0.35cm]
	(z1) to[out=-30,in=30,distance=1.2cm] (z1);
\node at ($(z0)+(0,1.15)$) {\texttt{0}};
\node at ($(z1)+(1.1,0)$) {\texttt{1}};
\node at ($0.5*(z0)+0.5*(z1)+(0,0.55)$) {\texttt{1}};
\node at ($0.5*(z0)+0.5*(z1)+(0,-0.55)$) {\texttt{0}};
\end{tikzpicture}
\end{center}
\qedhere
\end{loesung}


