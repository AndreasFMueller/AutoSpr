Untersuchen Sie, welche der Zustände im Automaten
\[
\entrymodifiers={++[o][F]}
\xymatrix{
*+\txt{}\ar[r]
        &{z_0}\ar[r]^{\tt 1} \ar@(ur,ul)_{\tt 0}
                &*++[o][F=]{z_1} \ar@/^/[dl]^{\tt 0} \ar@(dr,ur)_{\tt 1}
\\
*+\txt{}
        &{z_2} \ar@/^/[ur]^{\tt 1} \ar[u]^{\tt 0}
}
\]
äquivalent sind, und finden Sie den minimalen Automaten.

\thema{minimaler Automat}
\thema{Zustandsdiagramm}

\begin{loesung}
$z_0$ und $z_2$ sind äquivalent, denn von beiden aus
werden genau die Wörter akzeptiert, die mit beliebig
vielen {\tt 0}  gefolgt von einer {\tt 1} beginnen.
$z_1$ ist mit keinem anderen Zustand äquivalent, weil im
Zustand $z_1$ das leere Wort akzeptiert werden kann, nicht
aber in den anderen Zuständen.

Mit dem in der Vorlesung besprochenen ``Kreuzchen-Algorithmus'' lässt sich
der Minimalautomat natürlich auch finden:
\begin{center}
\begin{tabular}{|>{$}c<{$}|>{$}c<{$}>{$}c<{$}>{$}c<{$}|}
\hline
              &z_0    &z_1/\texttt{E}&z_2    \\
\hline
z_0           &\equiv &\times        &       \\
z_1/\texttt{E}&\times &\equiv        &\times \\
z_2           &       &\times        &\equiv \\
\hline
\end{tabular}
\end{center}
Der \texttt{0}-"Ubergang von $z_0$ und $z_2$ führt zu $z_0$, der
\texttt{1}-"Ubergang zu $z_1$, in beiden Fällen kann man also
kein neues Kreuz im Feld $(z_0,z_2)$ machen, der Algorithmus endet.
Man schliesst, dass $z_0$ und $z_2$ äquivalent sein müssen.

Legt man $z_0$ und $z_2$ zusammen, wird der Automat vereinfacht zu
\[
\entrymodifiers={++[o][F]}
\xymatrix{
*+\txt{}\ar[r]
        &{z_0,z_2}\ar@/^/[r]^{\tt 1} \ar@(ur,ul)_{\tt 0}
                &*++[o][F=]{z_1} \ar@/^/[l]^{\tt 0} \ar@(dr,ur)_{\tt 1}
}
\qedhere
\]
\end{loesung}


