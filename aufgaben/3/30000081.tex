Betrachten Sie die Sprache
\[
L
=
\{
wuw^t
\mid
w,u\in\Sigma^*\wedge |u|\le 3
\}
\]
über dem Alphabet $\Sigma=\{\texttt{a},\texttt{b}\}$.
Dabei bezeichnet $w^t$ das gespiegelte Wort von $w$.
Die Sprache $L$ besteht also aus Wörtern, die bis auf einen Block
der Länge $\le 3$ genau in der Mitte spiegelsymmetrisch sind.
Ist $L$ regulär?

\themaL{nicht regular}{nicht regulär}
\themaL{Pumping Lemma fur regulare Sprachen}{Pumping Lemma für reguläre Sprachen}

\begin{loesung}
Diese Sprache ist nicht regulär, da sich der endliche Automat an das
beliebig lange Wort $w$ erinnern müsste.
Mit dem Pumping Lemma kann man dieses heuristische Argument auch streng
beweisen.
\begin{enumerate}
\item
Wir nehmen an, die Sprache $L$ sei regulär.
\item
Nach dem Pumping Lemma gibt es dann die Pumping Length $N$.
\item
Wir wählen das Wort
$v=\texttt{a}^N\texttt{b}^3\texttt{a}^N\in L$.
Der Vergleich mit der Spezifikation der Sprache zeigt, dass der erste
Block aus Buchstaen \texttt{a} der Teil $w$ sein muss und der Block aus 
Buchstaben \texttt{b} muss $u$ sein.
\item
Nach dem Pumping Length lässt sich das Wort aufteilen in drei Teile
$v=xyz$ derart, dass $|xy|\le N$ und $|y|>0$.
\item
Der Teil $y$ besteht ausschliesslich aus \texttt{a}.
Beim Pumpen nimmt die Anzahl \texttt{a} im ersten Block zu, nicht
aber im zweiten.
Der Teil $u$ muss genau in der Mitte des gepumpten Wortes sein,
der Teil aus \texttt{b} verschiebt sich beim Pumpen aber weg von
der Mitte des Wortes, das Wort hat daher nicht mehr die verlangte
Symmetrie und ist damit nicht mehr in $L$.
\item
Der Widerspruch zeigt, dass die Sprache $L$ nicht regulär sein kann.
\qedhere
\end{enumerate}
\end{loesung}

\begin{bewertung}
6 Schritte des Pumping-Lemma-Beweises:
Annahme ({\bf A}) 1 Punkt,
Pumping Length ({\bf N}) 1 Punkt,
Wort ({\bf W}) 1 Punkt,
Unterteilung ({\bf U}) 1 Punkt,
Widerspruch beim Pumpen ({\bf P}) 1 Punkt,
Folgerung ({\bf F}) 1 Punkt.
\end{bewertung}

