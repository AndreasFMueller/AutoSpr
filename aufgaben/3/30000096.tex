In Aufgabe~\ref{30000089} in der Vorlesung wurde gezeigt, dass die
reguläre Sprache bestehend aus den Wörtern mit einer ungeraden
Anzahl Nullen vom regulären Ausdruck \texttt{1*0(1|01*0)*} akzeptiert
werden.
Der Ragel-Compiler produziert mit der Kommandozeilenoption
\texttt{-n} aus diesem regulären Ausdruck den nicht minimierten
deterministischen endlichen Automaten
\begin{center}
\includeagraphics[width=\textwidth]{ungerade.pdf}
\end{center}
Bestimmen Sie den zugehörigen Minimalautomaten und zeigen Sie, dass er
mit dem Automaten übereinstimmt, von dem ausgehend in Aufgabe \ref{30000089}
der reguläre Ausdruck gefunden wurde.

\thema{minimaler Automat}

\begin{loesung}
Der Algorithmus startet mit der Tabelle, die die Paare aus
Akzeptierzuständen und Nichtakzeptierzuständen als nicht zusammenlegbar
ausweist:
\def\t{\times}
\def\e{\equiv}
\def\o{\otimes}
\begin{center}
\begin{tabular}{|>{$}c<{$}|>{$}c<{$}>{$}c<{$}>{$}c<{$}>{$}c<{$}>{$}c<{$}>{$}c<{$}>{$}c<{$}|}
\hline
 &1     &2     &3     &4     &5     &6     &7     \\
\hline
1&\e    &      &      &      &\t    &\t    &\t    \\
2&      &\e    &      &      &\t    &\t    &\t    \\
3&      &      &\e    &      &\t    &\t    &\t    \\
4&      &      &      &\e    &\t    &\t    &\t    \\
5&\t    &\t    &\t    &\t    &\e    &      &      \\
6&\t    &\t    &\t    &\t    &      &\e    &      \\
7&\t    &\t    &\t    &\t    &      &      &\e    \\
\hline
\end{tabular}
\end{center}
Jetzt muss für alle freien Paare untersucht werden, ob sie von einem
Übergang auf ein Zustandspaar abgebildet werden, welches bereits als
nicht zusammenlegbar erkannt worden ist.
Es stellt sich heraus, dass das bei keinem Paar eintrifft.
Somit können alle Nichtakzeptierzustände in einen Zustand zusammengelegt
werden.
Ebenso können die drei Akzeptierzustände in einen Zustand
zusammengelegt werden.
So ergibt sich der minimierte Automat
\begin{center}
\includeagraphics[]{ungerade_minimized.pdf}
\end{center}
\end{loesung}
