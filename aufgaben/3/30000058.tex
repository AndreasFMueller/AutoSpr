In Google Forms kann man Formulare erstellen. 
Bei jedem Eingabefeld kann man zur Validierung der Benutzereingabe
einen regulären Ausdruck angeben.
Geben Sie in den folgenden Teilaufgaben jeweils einen regulären Ausdruck
an, der genau die spezifizierten Eingaben akzeptiert.
\begin{teilaufgaben}
\item Geburtsjahr: eine vierstellige Zahl zwischen 1900 und 2019.
\item Telefonnummer: optionale internationale Vorwahl in der Form
\texttt{+xx}, wobei \text{x} für eine beliebige Ziffer steht,
gefolgt von mindestens 7 und maximal 10 beliebigen Ziffern,
zwischen denen beliebig viele Leerzeichen stehen können.
\item Eine Internet-Email Adresse. Man beachte, dass der Domainname der
Email-Adresse aus mit Punkten getrennten, nicht leeren Komponenten
besteht, die nur Buchstaben, Ziffern und Bindestriche enthalten dürfen.
\end{teilaufgaben}

\thema{reguläre Ausdrücke}
\thema{regulär}

\begin{loesung}
\begin{teilaufgaben}
\item
\texttt{19[0-9][0-9]|20[01][0-9]}
\item
\texttt{(+[0-9][0-9])?(\blank*)([0-9]\blank*)\{7,10\}}
\item
\texttt{[a-zA-Z0-9\textbackslash.]*@([-a-zA-Z0-9]+\textbackslash.)+[-a-zA-Z0-9]+}
\qedhere
\end{teilaufgaben}
\end{loesung}

\begin{bewertung}
In jeder Teilaufgabe ein Punkt für eine Teillösung und ein weiterer Punkt für
eine vollständige Lösung.
\end{bewertung}

