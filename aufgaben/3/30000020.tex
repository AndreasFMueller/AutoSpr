Für jede positive natürliche Zahl $0<p\in\mathbb N$ und jedes Zeichen
$z\in \Sigma=\{\text{\tt 0},\text{\tt 1}\}$
sei $L_{p,z}$ die Sprache über dem Alphabet
$\Sigma$, deren Wörter das Zeichen $z$ eine Anzahl mal
enthalten, die ein Vielfaches von $p$ ist. Zum Beispiel enthält
$L_{3,\text{\tt 1}}$ die Wörter
\[
\text{\tt 00},
\text{\tt 01011},
\text{\tt 11000001},
\text{\tt 110101101},
\]
nicht aber die Wörter
\[
\text{\tt 101},
\text{\tt 000},
\text{\tt 000000},
\text{\tt 1010101},
\text{\tt 11}.
\]
In Formeln kann man $L_{p,z}$ auch schreiben als
\[
L_{p,z}=\{w\in\Sigma^*|\; |w|_z\equiv 0\mod p\}
\]
(Zur Erinnerung: $|w|_z$ ist die Anzahl der Zeichen $z$ im Wort $w$.)
Das leere Wort ist in jeder Sprache $L_{p,z}$ enthalten.
\begin{teilaufgaben}
\item Zeichnen Sie das Zustandsdiagramm je eines DEA, welcher
$L_{2,\text{\tt 0}}$ bzw.~$L_{3,\text{\tt 1}}$ akzeptiert.
\item Zeichnen Sie das Zustandsdiagramm eines DEA, welcher
$L_{2,\text{\tt 0}}\cap L_{3,\text{\tt 1}}$ akzeptiert.
\item
Bestimmen Sie
$L_{p,\text{\tt 0}} \cap L_{p,\text{\tt 0}}$
für zwei verschiedene Primzahlen $p$ und $q$.
\item Welche Wörter enthält
\[
L_{\infty,\text{\tt 0}}
=
\bigcap_{\text{$p$ prim}}L_{p,\text{\tt 0}}
=
L_{2,\text{\tt 0}}\cap
L_{3,\text{\tt 0}}\cap
L_{5,\text{\tt 0}}\cap
L_{7,\text{\tt 0}}\cap
L_{11,\text{\tt 0}}\cap\dots
\]
\end{teilaufgaben}

\themaL{regular}{regulär}
\thema{DEA}

\begin{loesung}
\begin{teilaufgaben}
\item Als Zustände eines Automaten für $L_{p,z}$ verwenden wir den Rest
von $|w|_z$ modulo $p$, also für $L_{2,\text{\tt 0}}$:
\[
\entrymodifiers={++[o][F]}
\xymatrix @+1mm {
*+\txt{}\ar[r]
        &*++[o][F=]{0} \ar@/^/[r]^{\tt 0} \ar@(ur,ul)^{\tt 1}
                &{1}\ar@/^/[l]^{\tt 0} \ar@(ur,ul)^{\tt 1}
}
\]
und für $L_{3,\text{\tt 1}}$:
\[
\entrymodifiers={++[o][F]}
\xymatrix @+1mm {
*+\txt{}\ar[r]
        &*++[o][F=]{0}\ar[r]^{\tt 1} \ar@(ur,ul)^{\tt 0}
                &{1}\ar[r]^{\tt 1} \ar@(ur,ul)^{\tt 0}
                        &{2}\ar@/^10pt/[ll]^{\tt 1} \ar@(ur,ul)^{\tt 0}
}
\]
\item Man kann einen Automaten zusammenbauen aus den beiden
Automaten für $L_{2,\text{\tt 0}}$ und $L_{3,\text{\tt 1}}$.
Die Zustände sind Paare von Zustände der beiden Teilautomaten:
\[
\entrymodifiers={++[o][F]}
\xymatrix @+1mm {
*+\txt{}\ar[dr]
\\
*+\txt{}
        &*++[o][F=]{0,0} \ar@/^/[d]^{\tt 0} \ar[r]_{\tt 1}
                &{0,1} \ar@/^/[d]^{\tt 0} \ar[r]_{\tt 1}
                        &{0,2} \ar@/^/[d]^{\tt 0} \ar@/_15pt/[ll]_{\tt 1}
\\
*+\txt{}
        &{1,0} \ar@/^/[u]^{\tt 0} \ar[r]^{\tt 1}
                &{1,1} \ar@/^/[u]^{\tt 0} \ar[r]^{\tt 1}
                        &{1,2} \ar@/^/[u]^{\tt 0} \ar@/^15pt/[ll]^{\tt 1} 
}
\]
\item
Ein Wort in $L_{p,\text{\tt 0}} \cap L_{p,\text{\tt 0}}$ enthält
eine Anzahl Nullen, die sowohl durch $p$ als auch durch $q$ teilbar
ist, also ein Vielfaches von $pq$ ist. Es ist also
\[
L_{p,\text{\tt 0}} \cap L_{p,\text{\tt 0}}
=
L_{pq,\text{\tt 0}}
\]
Man kann dieses Resultat auch für das Beispiel $p=2$, $q=3$ aus
den oben gezeigten Automaten ableiten, und bekommt folgenden Automaten
für die Schnittsprache:
\[
\entrymodifiers={++[o][F]}
\xymatrix @+1mm {
*+\txt{}\ar[dr]
\\
*+\txt{}
        &*++[o][F=]{0,0} \ar[dr]^{\tt 0} \ar@(ur,ul)_{\tt 1}
                &{0,1} \ar[dr]^{\tt 0}\ar@(ur,ul)_{\tt 1}
                        &{0,2} \ar[dll]^{\tt 0}\ar@(ur,ul)_{\tt 1}
\\
*+\txt{}
        &{1,0} \ar[ur]^{\tt 0} \ar@(dr,dl)^{\tt 1}
                &{1,1} \ar[ur]^{\tt 0}\ar@(dr,dl)^{\tt 1}
                        &{1,2} \ar[ull]^{\tt 0}\ar@(dr,dl)^{\tt 1}
}
\]
Indem man den Pfeilen folgt kann man verifizieren, dass man den
Akzeptierzustand jeweils erst wieder erreicht, wenn man alle
anderen Zustände besucht hat, also ein Vielfaches von $6$ Nullen
gelesen hat.
\item Ein Wort $w$ in $L_{\infty,\text{\tt 0}}$ muss eine Anzahl Nullen
enthalten, die durch alle Primzahlen teilbar ist. Dies ist nur möglich,
wenn $w$ gar keine Nullen enthält. $L_{\infty,\text{\tt 0}}$ besteht
also aus den Wörtern, die ausschliesslich Einsen enthalten:
\[
L_{\infty,\text{\tt 0}}=\{\text{\tt 1}\}^*.
\qedhere
\]
\end{teilaufgaben}
\end{loesung}
