Konstruieren Sie einen endlichen Automaten mit dem Alphabet
$\Sigma=\{\texttt{0},\texttt{1}\}$, der Zeichenketten akzeptiert, in denen
Einsen (\texttt{1}) immer allein stehen, also zum Beispiel
\[
\texttt{1},\;
\texttt{010},\;
\texttt{0100},\;
\texttt{0100001}
\]
aber nicht
\[
\texttt{11},\;
\texttt{0110},\;
\texttt{0100011}.
\]

\begin{loesung}
Wir brauchen zwei Zustände um auszudrücken, ob gerade eine Eins verarbeitet
worden ($q_1$) ist oder eben nicht ($q_0$).
Ein dritter Zustand $q_2$ ist nötig, um Übergänge abzufangen, die zu nicht
akzeptablen Zeichenketten führen würden.
Gestartet wird im Zustand $q_0$, eine \texttt{1} führt über in den 
Zustand $q_1$.
Folgt nach einer \texttt{1} eine weitere \texttt{1}, wechselt man
in den Zustand $q_2$, der kein Akzeptierzustastand ist.
Folgt eine \texttt{0} ist man wieder in der Situation $q_0$.
Beide Zustände $q_0$ und $q_1$ sind also Akzeptierzustände:
\begin{center}
\def\zustand#1#2{
	\fill[color=white] #1 circle[radius=0.5];
	\node at #1 {$#2$};
	\draw #1 circle[radius=0.5];
}
\def\akzeptierzustand#1#2{
	\fill[color=white] #1 circle[radius=0.5];
	\node at #1 {$#2$};
	\draw #1 circle[radius=0.5];
	\draw #1 circle[radius=0.45];
}
\def\pfeil#1#2#3{
	\draw[->,shorten >= 0.5cm,shorten <= 0.5cm] #1 -- #2;
	\node at ($0.5*#1+0.5*#2$) [above] {$#3$};
}
\begin{tikzpicture}[>=latex,thick]
\coordinate (A) at (-3,0);
\coordinate (B) at (0,0);
\coordinate (C) at (3,0);
\akzeptierzustand{(A)}{q_0}
\akzeptierzustand{(B)}{q_1}
\zustand{(C)}{q_2}
\pfeil{(A)}{(B)}{\texttt{1}}
\pfeil{(B)}{(C)}{\texttt{1}}
\draw[->,shorten >= 0.5cm, shorten <= 0.5cm] (C) to[out=-45,in=45,distance=2cm] (C);
\node at ($(C)+(1.2,0)$) [right] {$\texttt{0},\texttt{1}$};
\draw[->,shorten >= 0.5cm,shorten <= 0.5cm] (A) to[out=45,in=135,distance=2cm] (A);
\node at ($(A)+(0,1.2)$) [above] {$\texttt{0}$};
\draw[->,shorten >= 0.5cm,shorten <= 0.5cm] (B) to[out=-135,in=-45] (A);
\node at ($0.5*(A)+0.5*(B)+(0,-0.8)$) [below] {\texttt{0}};
\draw[->,shorten >= 0.5cm] (-5,0) -- (A);
\end{tikzpicture}
\end{center}
\end{loesung}
