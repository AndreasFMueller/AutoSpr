Konstruieren Sie einen endlichen Automaten mit dem Alphabet
$\Sigma=\{\texttt{0},\texttt{1}\}$, der Zeichenketten akzeptiert, in denen
Einsen (\texttt{1}) immer allein stehen, also zum Beispiel
\[
\texttt{1},\;
\texttt{010},\;
\texttt{0100},\;
\texttt{0100001}
\]
aber nicht
\[
\texttt{11},\;
\texttt{0110},\;
\texttt{0100011}.
\]

\begin{loesung}
Wir brauchen zwei Zustände um auszudrücken, ob gerade eine Eins verarbeitet
worden ist oder eben nicht.
Ein dritter Zustand ist nötig, um Übergänge abzufangen, die zu nicht
akzeptablen Zeichenketten führen würden.
\begin{center}
\begin{tikzpicture}[>=latex,thick]
\coordinate (A) at (-1.5,0);
\coordinate (B) at (1.5,0);
\coordinate (C) at (0,{-\sqrt{3}*3/2});
\def\zustand#1#2{
	\fill[color=white] #1 circle[radius=0.5];
	\node at #1 {$#2$};
	\draw #1 circle[radius=0.5];
}
%\def\akzeptierzustand#1#2{
%	\fill[color=white] #1 circle[radius=0.5];
%	\node at #1 {$#2$};
%	\draw #1 circle[radius=0.5];
%	\draw #1 circle[radius=0.45];
%}
%\def\pfeil#1#2#3{
%	\draw[->,shorten >= 0.5cm,shorten <= 0.5cm] #1 -- #2;
%	\node at ($0.5*#1+0.5*#2$) [above] {$#3$};
%}
%\akzeptierzustand{(A)}{q_0}
%\akzeptierzustand{(B)}{q_1}
\zustand{(C)}{q_2}
%\pfeil{(A)}{(B)}{\texttt{1}}
%\pfeil{(B)}{(C)}{\texttt{1}}
\end{tikzpicture}
\end{center}
\end{loesung}
