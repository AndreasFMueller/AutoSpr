Im Jahr 2022 erlangte das Spiel Wordle von Josh Wardle grosse Popularität.
Obwohl es erst 2021 veröffentlich wurde, spielten es im Januar 2022
bereits Millionen von Spielern jeden Tag.

Der Spieler muss ein Wort mit vorgegebener Länge erraten.
Nach jedem Versuch wird ihm angezeigt, welche Buchstaben er bereits
richtig erraten hat und welche Buchstaben zwar im Wort vorkommen,
aber noch nicht am richtigen Platz stehen.
%Richtig platzierte Buchstaben werden grün angezeigt,
%Buchstaben, die zwar im Wort vorkommen, aber nicht am richtigen Platz stehen,
%werden oliv angezeigt.
Ausserdem müssen die Wörter, die man eingebend darf, in einer
natürlichen Sprache enthalten sein, man kann also nicht mit
beliebigen, Zeichenkombinationen probieren und so möglichst effizient
die vorkommenden Buchstaben ermitteln.

\strut\hbox to\textwidth{%
\begin{minipage}[t]{10.8cm}
\strut
Im nebenstehenden Beispiel wurden im ersten Schritt gefunden,
dass der Buchstabe \texttt{A} zwar im Wort vorkommt, aber an einer
anderen Stelle, dies wird mit oliver Farbe angezeigt.
Im fünften Schritt waren dann alle Buchstaben an der richtigen Stelle,
angezeigt durch die grüne Farbe.
Nach Schritt 4 war klar, welche Zeichen vorkommen (das Zeichen \texttt{H}
muss gemäss Zeile 2 vorkommen, ausserdem die vier richtigen Zeichen
von Zeile 4), aber der Spieler
musste noch ein Wort aus diesen Zeichen zusammenstellen.
Dies ist jedoch nicht so einfach, 
insbesondere wenn der Spieler nicht englischer Muttersprache ist.
Natürlich macht dies einen Teil des Reizes dieses Spiels aus.
\end{minipage}%
\hfill
\raisebox{-5.7cm}{\includeagraphics[]{wordle.pdf}}%
}

\smallskip

Ein Wörterbuch, in welchem man nach nur teilweise bekannten Wörtern
suchen könnte, wäre dazu hilfreich.
Files mit Wortlisten sind für fast jede Sprache verfügbar und mit
einer Suchfunktionen basierend auf regulären Ausdrücken könnte man
sie auch sehr schnell durchsuchen, diese Funktion wird zum Beispiel
vom Unix-Befehl \texttt{grep} realisiert.
Man kann sie jedoch nur nutzen, wenn man in der Lage ist, die
bereits bekannten Bedingungen an die vorkommenden Buchstaben
in Form eines regulären Ausdruckes zu formulieren.

% XXX Beispiel

\begin{teilaufgaben}
\item Finden Sie einen endlichen Automaten, der genau Wörter der
Länge $5$ akzeptiert.
\item Finden Sie einen regulären Ausdruck, der Wörter findet, die
den Buchstaben \texttt{A} an der vierten Stelle enthalten.
\item Finden Sie einen deterministischen endlichen Automaten,
der Wörter findet, die den Buchstaben \texttt{A} enthalten.
\item Skizzieren Sie einen Algorithmus, der einen regulären Ausdruck
erzeugt, der alle in den Teilaufgaben a)--c) genannten Bedingungen erfüllt.
\end{teilaufgaben}

\begin{hinweis}
Es wird in Teilaufgabe d) nicht ein regulärer Ausdruck verlangt, sondern
eine Methode, wie man solche reguläre Ausdrücke gewinnen kan.
Die Idee ist, dass man damit das im Spiel bereits erlangte Wissen 
als regulären Ausdruck codieren und so die Suche für den nächsten
Versuch durchführen kann.
\end{hinweis}

\begin{loesung}
\begin{teilaufgaben}
\item
Der Ausdruck $r=\texttt{.....}$ erfüllt dies, doch es ist ein endlicher
Automat gefragt.
Ein NEA, der dies leistet, ist
\begin{center}
\begin{tikzpicture}[>=latex,thick]
\coordinate (S) at (0,0);
\coordinate (A) at (2,0);
\coordinate (B) at (4,0);
\coordinate (C) at (6,0);
\coordinate (D) at (8,0);
\coordinate (E) at (10,0);
\draw[->,shorten >= 0.3cm,shorten <= 0.3cm] (-2,0) -- (S);
\draw[->,shorten >= 0.3cm,shorten <= 0.3cm] (S) -- (A);
\draw[->,shorten >= 0.3cm,shorten <= 0.3cm] (A) -- (B);
\draw[->,shorten >= 0.3cm,shorten <= 0.3cm] (B) -- (C);
\draw[->,shorten >= 0.3cm,shorten <= 0.3cm] (C) -- (D);
\draw[->,shorten >= 0.3cm,shorten <= 0.3cm] (D) -- (E);
\node at (S) {$0$};
\node at (A) {$1$};
\node at (B) {$2$};
\node at (C) {$3$};
\node at (D) {$4$};
\node at (E) {$5$};
\draw (S) circle[radius=0.3];
\draw (A) circle[radius=0.3];
\draw (B) circle[radius=0.3];
\draw (C) circle[radius=0.3];
\draw (D) circle[radius=0.3];
\draw (E) circle[radius=0.3];
\draw (E) circle[radius=0.25];
\node at ($0.5*(S)+0.5*(A)$) [above] {$\Sigma$};
\node at ($0.5*(A)+0.5*(B)$) [above] {$\Sigma$};
\node at ($0.5*(B)+0.5*(C)$) [above] {$\Sigma$};
\node at ($0.5*(C)+0.5*(D)$) [above] {$\Sigma$};
\node at ($0.5*(D)+0.5*(E)$) [above] {$\Sigma$};
\end{tikzpicture}
\end{center}
\item
Der reguläre Ausdruck $r=\texttt{...A.}$ leistet dies.
\item Der folgende Automat ist ein deterministischer Automat, der die
Bedingung erfüllt.
\begin{center}
\begin{tikzpicture}[>=latex,thick]
\coordinate (S) at (-2,0);
\coordinate (A) at (0,0);
\coordinate (B) at (2,0);
\coordinate (C) at (4,0);
\draw[->,shorten >= 0.3cm,shorten <= 0.3cm] (S) -- (A);
\draw[->,shorten >= 0.3cm,shorten <= 0.3cm] (A) -- (B);
\draw[->,shorten >= 0.3cm,shorten <= 0.3cm]
	(A) to[out=60,in=120,distance=1.5cm] (A);
\draw[->,shorten >= 0.3cm,shorten <= 0.3cm]
	(B) to[out=60,in=120,distance=1.5cm] (B);
\draw (A) circle[radius=0.3];
\draw (B) circle[radius=0.3];
\draw (B) circle[radius=0.25];
\node at ($(A)+(0,1.2)$) [above,distance=1.3cm]
	{$\Sigma\setminus\{\texttt{A}\}\mathstrut$};
\node at ($(B)+(0,1.2)$) [above,distance=1.3cm]
	{$\Sigma\mathstrut$};
\node at ($0.5*(A)+0.5*(B)$) [above] {$\texttt{A}$};
\end{tikzpicture}
\end{center}
\item
Der folgende Algorithmus liefert einen regulären Ausdruck der
verlangten Art:
\begin{enumerate}[1.]
\item
Für die vorgegebene Wortlänge, verwende den analog zu Teilaufgabe~a)
gefundenen regulären Ausdruck und erzeuge daraus einen DEA.
\item
Für jedes Zeichen mit bekannter Position erzeuge nach dem Muster von
Teilaufgabe~b) einen DEA, der genau die Wörter akzeptiert, die diese
Bedingung erfüllen.
\item
Für jedes Zeichen, das vorkommt, erzeuge nach dem Muster von
Teilaufgabe~c) einen DE, der genau die Wörter akzeptiert, die diese
Bedingung erfüllen.
\item
Aus den DEAs der Schritte~1--3 erzeuge den Produktautomaten, einen DEA,
der genau die Wörter akzeptiert, die alle Bedingungen erfüllen.
\item
Wandle den DEA von Schritt~4 mit dem in der Vorlesung dargestellten
Algorithmus in einen regulären Ausdruck um.
\qedhere
\end{enumerate}
\end{teilaufgaben}
\end{loesung}

\begin{diskussion}
Das Spiel ist übrigens nicht neu, ein Spiel mit dem gleichen Prinzip
wurde schon in den 1980er-Jahren in der Fernsehshow {\em Lingo} gespielt.
\end{diskussion}

\begin{bewertung}
Teilaufgaben a)--c) je ein Punkt ({\bf A}), ({\bf B}) bzw.~({\bf C}),
Produktautomat ({\bf P}) 1 Punkt,
Umwandlung eines regulären Ausdrucks in einen DEA ({\bf R}) 1 Punkt,
Umwandlung eines VNEA in einer regulären Ausdruck ({\bf V}) 1 Punkt.
\end{bewertung}

