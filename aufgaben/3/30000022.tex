Sei $n>0$ eine natürliche Zahl und sei $L_n$ die Sprache bestehend
aus Wörtern über dem Alphabet
$\Sigma=\{{\tt 0},{\tt 1 }\}$ deren Anzahl {\tt 0} und {\tt 1}
den gleichen Rest bei Teilung durch $n$ haben. In Formeln:
\[
L_n=\{ w\in\Sigma^*| \; |w|_0\equiv |w|_1\mod n\}.
\]
Zur Sprache $L_3$ gehört also beispielsweise dazu:
\[
\text{\tt 1010000},\;
\text{\tt 00101110},\;
\text{\tt 010101},\;
\text{\tt 101010},\;
\]
es gehört aber nicht dazu:
\[
\text{\tt 1110},\;
\text{\tt 0001},\;
\text{\tt 1110001},\;
\text{\tt 10101},\;
\]
Zeigen Sie, dass jede der Sprachen $L_n$ regulär ist.

\begin{hinweis}
Zeichnen Sie einen DEA für $L_8$ und überlegen sie
sich dann, wie ein DEA für $L_n$ mit beliebigem $n$ aussehen müsste.
\end{hinweis}

\thema{DEA}
\themaL{regular}{regulär}

\begin{loesung}
Wir wissen bereits, dass es endliche Automaten gibt, die den Rest
einer Zahl bei Teilung durch $n$ bestimmen können. Als Zustände
wird der Rest verwendet. Da es nur $n$ verschiedene Reste gibt,
kann ein endlicher Automat alle möglichen Reste darstellen.
Daher ist plausibel, dass auch die Gleichheit zweier Reste
mit Hilfe eines endlichen Automaten erkannt werden kann.

Wir zeigen, dass es einen endlichen Automaten gibt, der die Wörter
von $L_n$ akzeptiert. Wenn $r_0(w)\equiv |w|_{\tt 0}\mod n$
der Rest der Anzahl {\tt 0} bei
Teilung durch $n$ und $r_1(w)\equiv |w|_{\tt 1}\mod n$ der Rest der Anzahl
{\tt 1} bei Teilung
durch $n$ ist, dann interessieren uns nur die Wörter, für die die
beiden Zahlen gleich sind, also
\[
L_n=\{w\in\Sigma^*\;|\;r_0(w)-r_1(w)=0\}
\]
Die Differenz kann aber mit folgendem endlichen Automaten bestimmt
werden. Als Zustände dienen die möglichen Differenzen
$\{0,1,2,\dots,n-1\}$.
Startzustand ist $0$.
Jede {\tt 0} in $w$ erhöht den Zustand um
eins, jede {\tt 1} erniedrigt den Zustand um $1$. Akzeptierzustand
ist $0$. Das Zustandsdiagramm für den Fall $n=8$ ist
\[
\entrymodifiers={++[o][F]}
\xymatrix @-1mm {
*+\txt{}\ar[dr]
\\
*+\txt{}
        &*++[o][F=]{0}\ar@/^/[r]^{\tt 0} \ar@/^/[dl]^{\tt 1}
                &{1}\ar@/^/[dr]^{\tt 0} \ar@/^/[l]^{\tt 1}
                        &*+\txt{}
\\
{7}\ar@/^/[ur]^{\tt 0} \ar@/^/[d]^{\tt 1}
        &*+\txt{}
                &*+\txt{}
                        &{2}\ar@/^/[d]^{\tt 0} \ar@/^/[ul]^{\tt 1}
\\
{6}\ar@/^/[u]^{\tt 0} \ar@/^/[dr]^{\tt 1}
        &*+\txt{}
                &*+\txt{}
                        &{3}\ar@/^/[dl]^{\tt 0} \ar@/^/[u]^{\tt 1}
\\
*+\txt{}
        &{5}\ar@/^/[ul]^{\tt 0} \ar@/^/[r]^{\tt 1}
                &{4}\ar@/^/[l]^{\tt 0} \ar@/^/[ur]^{\tt 1}
                        &*+\txt{}
}
\]
Somit ist klar, dass $L_n$ regulär ist.

Alternativ könnte man auch zwei Automaten bilden, deren Zustände
die Reste von $|w|_0$ und $|w|_1$ bei Teilung durch $n$ sind:
\[
\xymatrix @-1mm {
        &
                &
                        &
                                &
                                        &
                                                &
\\
*+\txt{}\ar[r]
        &*++[o][F]{0}\ar[r]^{\tt 0}
                &*++[o][F]{1}\ar[r]^{\tt 0}
                        &*++[o][F]{2}\ar[r]^{\tt 0}
                                &*++[o][F]{3}\ar[r]^{\tt 0}
                                        &{\dots}\ar[r]^{\tt 0}
                                                &*++[o][F]{n-1}\ar `u[l] `[lllll]_{\tt 0} [lllll]
\\
        &
                &
                        &
                                &
                                        &
                                                &
\\
*+\txt{}\ar[r]
        &*++[o][F]{0}\ar[r]^{\tt 1}
                &*++[o][F]{1}\ar[r]^{\tt 1}
                        &*++[o][F]{2}\ar[r]^{\tt 1}
                                &*++[o][F]{3}\ar[r]^{\tt 1}
                                        &{\dots}\ar[r]^{\tt 1}
                                                &*++[o][F]{n-1}\ar `u[l] `[lllll]_{\tt 1} [lllll]
}
\]
Bildet man jetzt den kartesischen Produkt-Automaten, sind genau
die Zustände auf den Diagonalen diejenigen, die Wörter akzeptieren,
die module $n$ die gleiche Zahl von {\tt 0} und {\tt 1} haben:
\[
\xymatrix @-1mm {
*+\txt{}\ar[dr]
        &{\vdots}\ar[d]^{\tt 1}
                &{\vdots}\ar[d]^{\tt 1}
                        &{\vdots}\ar[d]^{\tt 1}
                                &
                                        &{\vdots}\ar[d]^{\tt 1}
\\
{\dots}\ar[r]^{\tt 0}
        &*++[o][F=]{ }\ar[r]^{\tt 0}\ar[d]^{\tt 1}
                &*++[o][F]{ }\ar[r]^{\tt 0}\ar[d]^{\tt 1}
                        &*++[o][F]{ }\ar[r]^{\tt 0}\ar[d]^{\tt 1}
                                &{\dots}\ar[r]^{\tt 0}
                                        &*++[o][F]{ }\ar[r]^{\tt 0}\ar[d]^{\tt 1}
                                                &{\dots}
\\
{\dots}\ar[r]^{\tt 0}
        &*++[o][F]{ }\ar[r]^{\tt 0}\ar[d]^{\tt 1}
                &*++[o][F=]{ }\ar[r]^{\tt 0}\ar[d]^{\tt 1}
                        &*++[o][F]{ }\ar[r]^{\tt 0}\ar[d]^{\tt 1}
                                &{\dots}\ar[r]^{\tt 0}
                                        &*++[o][F]{ }\ar[r]^{\tt 0}\ar[d]^{\tt 1}
                                                &{\dots}
\\
{\dots}\ar[r]^{\tt 0}
        &*++[o][F]{ }\ar[r]^{\tt 0}\ar[d]^{\tt 1}
                &*++[o][F]{ }\ar[r]^{\tt 0}\ar[d]^{\tt 1}
                        &*++[o][F=]{ }\ar[r]^{\tt 0}\ar[d]^{\tt 1}
                                &{\dots}\ar[r]^{\tt 0}
                                        &*++[o][F]{ }\ar[r]^{\tt 0}\ar[d]^{\tt 1}
                                                &{\dots}
\\
        &\vdots\ar[d]^{\tt 1}
                &\vdots\ar[d]^{\tt 1}
                        &\vdots\ar[d]^{\tt 1}
                                &
                                        &\vdots\ar[d]^{\tt 1}
\\
{\dots}\ar[r]^{\tt 0}
        &*++[o][F]{ }\ar[r]^{\tt 0}\ar[d]^{\tt 1}
                &*++[o][F]{ }\ar[r]^{\tt 0}\ar[d]^{\tt 1}
                        &*++[o][F]{ }\ar[r]^{\tt 0}\ar[d]^{\tt 1}
                                &{\dots}\ar[r]^{\tt 0}
                                        &*++[o][F=]{ }\ar[r]^{\tt 0}\ar[d]^{\tt 1}
                                                &{\dots}
\\
        &{\vdots}
                &{\vdots}
                        &{\vdots}
                                &
                                        &{\vdots}
                                                &
}
\]
Darin führen die Pfeile, die das Diagramm am rechten oder unteren Rand
verlassen wieder zu den Zuständen in der gleichen Zeile bzw.~Spalte
am linken bzw.~oberen Rand. Natürlich ist dieser Automat nicht minimal,
denn alle Zustände auf der Diagonalen oder Zustände auf einer Geraden
parallel zur Diagonalen sind äquivalent.

Man kann den Nachweis aber auch führen, ohne einen Automaten zu zeichnen.
Dazu überlegt man sich, dass die Sprachen
\[
L_{ik}=\{w\in\Sigma^*|\; |w|_i\cong k\mod n\}= L( i^k(i^n)^* )
\]
regulär sind, weil sie sich durch den angegeben regulären Ausdruck
beschreiben lassen. Die Sprache
$L_{{\tt 0}k}\cap L_{{\tt 1}k}$
besteht
aus den Wörtern, in denen die Anzahl der {\tt 0} und {\tt 1} den
Rest $k$ bei Teilung durch $n$ haben. Als Durchschnitt regulärer
Sprachen ist sie natürlich auch wieder regulär. In der Sprache $L_n$
sind jetzt aber alle solchen Wörter mit Resten $k=0,\dots,n-1$, also
\[
L_n=\bigcup_{k=0}^{n-1}
L_{{\tt 0}k}\cap L_{{\tt 1}k},
\]
was als Vereinigung regulärer Sprachen natürlich auch wieder regulär ist.
\end{loesung}
