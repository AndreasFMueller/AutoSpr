Ist die Sprache
\[
L=
\{
w\in \Sigma^*
\;|\;
|w|_\texttt{0} \ge 2^{|w|_{\texttt{1}}}
\}
\]
über dem Alphabet $\Sigma = \{ \texttt{0},\texttt{1}\}$ regulär?

\themaL{regular}{regulär}
\themaL{Pumping Lemma fur regulare Sprachen}{Pumping Lemma für reguläre Sprachen}

\begin{loesung}
Die Sprache ist nicht regulär, wie man mit dem Pumping Lemma für reguläre
Sprachen zeigen kann.
\begin{enumerate}
\item Annahme ist regulär.
\item Nach dem Pumping Lemma gibt es die Pumping Length $N$.
\item Wähle das Wort $w=\texttt{0}^{2^N}\texttt{1}^N\in L$
\item Es gibt eine Aufteilung des Wortes $w=xyz$ mit $|xy|\le N$ und $|y|>0$.
Der Teil $y$ besteht ausschliesslich aus Nullen.
\begin{center}
\definecolor{darkgreen}{rgb}{0,0.6,0}
\begin{tikzpicture}[>=latex,thick,scale=0.5]

\fill[color=gray!40] (0,-0.5) rectangle (20,0.5);
\draw (0,-0.5) rectangle (20,0.5);
\draw (16,-0.5) -- (16,0.5);

\node at (18,0) {$\texttt{1}^N$};
\node at (8,0) {$\texttt{0}^{2^N}$};

\node at (0,0.5) [above] {$0$};
\node at (4,0.5) [above] {$N$};
\node at (16,0.5) [above] {$2^N$};
\node at (20,0.5) [above] {$2^N+N$};

\fill[color=darkgreen!40,opacity=0.5] (0.1,-0.4) rectangle (1.4,0.4);
\draw[color=darkgreen] (0.1,-0.4) rectangle (1.4,0.4);
\node[color=darkgreen] at (0.7,0) {$x\mathstrut$};

\fill[color=red!40,opacity=0.5] (1.5,-0.4) rectangle (2.4,0.4);
\draw[color=red] (1.5,-0.4) rectangle (2.4,0.4);
\node[color=red] at (1.95,0) {$y\mathstrut$};

\fill[color=blue!40,opacity=0.5] (2.5,-0.4) rectangle (19.9,0.4);
\draw[color=blue] (2.5,-0.4) rectangle (19.9,0.4);
\node[color=blue] at (11.2,0) {$z\mathstrut$};

\node at (21.5,0) [right] {$\in L$};

\begin{scope}[yshift=-3.5cm]

	\fill[color=gray!40] (0,-0.5) rectangle (21,0.5);
	\draw (0,-0.5) rectangle (21,0.5);
	\draw (17,-0.5) -- (17,0.5);

	\node at (19,0) {$\texttt{1}^N$};
	\node at (8.5,0) {$\texttt{0}^{2^N+|y|}$};

	\node at (0,0.5) [above] {$0$};
	\node at (4,0.5) [above] {$N$};
	\node at (17,0.5) [above] {$2^N+|y|\ge 2^N$};
	%\node at (21,0.5) [above] {$2^N+N$};

	\fill[color=darkgreen!40,opacity=0.5] (0.1,-0.4) rectangle (1.4,0.4);
	\draw[color=darkgreen] (0.1,-0.4) rectangle (1.4,0.4);
	\node[color=darkgreen] at (0.7,0) {$x\mathstrut$};

	\fill[color=red!40,opacity=0.5] (1.5,-0.4) rectangle (2.4,0.4);
	\draw[color=red] (1.5,-0.4) rectangle (2.4,0.4);
	\node[color=red] at (1.95,0) {$y\mathstrut$};

	\fill[color=red!40,opacity=0.5] (2.5,-0.4) rectangle (3.4,0.4);
	\draw[color=red] (2.5,-0.4) rectangle (3.4,0.4);
	\node[color=red] at (2.95,0) {$y\mathstrut$};

	\fill[color=blue!40,opacity=0.5] (3.5,-0.4) rectangle (20.9,0.4);
	\draw[color=blue] (3.5,-0.4) rectangle (20.9,0.4);
	\node[color=blue] at (12.2,0) {$z\mathstrut$};

	\node at (21.5,0) [right] {$\in L$};

\end{scope}

\begin{scope}[yshift=-7.0cm]

	\fill[color=gray!40] (0,-0.5) rectangle (19,0.5);
	\draw (0,-0.5) rectangle (19,0.5);
	\draw (15,-0.5) -- (15,0.5);

	\node at (17,0) {$\texttt{1}^N$};
	\node at (7.5,0) {$\texttt{0}^{2^N-|y|}$};

	\node at (0,0.5) [above] {$0$};
	\node at (15,0.5) [above] {$2^N-|y|<2^N$};
	%\node at (20,0.5) [above] {$2^N-|y|+N$};

	\fill[color=darkgreen!40,opacity=0.5] (0.1,-0.4) rectangle (1.4,0.4);
	\draw[color=darkgreen] (0.1,-0.4) rectangle (1.4,0.4);
	\node[color=darkgreen] at (0.7,0) {$x\mathstrut$};

	%\fill[color=red!40,opacity=0.5] (1.5,-0.4) rectangle (2.4,0.4);
	%\draw[color=red] (1.5,-0.4) rectangle (2.4,0.4);
	%\node[color=red] at (1.95,0) {$y\mathstrut$};

	\fill[color=blue!40,opacity=0.5] (1.5,-0.4) rectangle (18.9,0.4);
	\draw[color=blue] (1.5,-0.4) rectangle (18.9,0.4);
	\node[color=blue] at (10.2,0) {$z\mathstrut$};

	\node at (21.5,0) [right] {$\not\in L$};

\end{scope}

\end{tikzpicture}
\end{center}
\item Beim Aufpumpen nimmt die Zahl der Nullen zu, dies steht aber nicht
im Widerspruch zu der Sprachdefinition.
Beim Abpumpen nimmt jedoch die Zahl der Nullen ab, es ist dann
$|xz|_{\texttt{w}}=2^N-|y|$.
Die Zahl der Einsen ist immer noch $N$, damit haben wir
\[
|xy|_{\texttt{0}} 
=
2^N-|y|
<
2^N
=
|xy|_{\texttt{1}},
\]
im Widerspruch zur Aussage des Pumping Lemmas
\item 
Dieser Widerspruch zeigt, dass die Sprache $L$ nicht regulär ist.
\qedhere
\end{enumerate}
Es gibt natürlich viele weitere Möglichkeiten, den gesuchten Widerspruch
herzustellen. 
Das oben gewählte Wort ist möglicherweise dasjenigen, welches die meisten
Studenten als erstes wählen würden, aber das Wort
$w=\texttt{1}^N\texttt{0}^{2^N}$ ist noch etwas einfacher.
Beim Aufpumpen nimmt die Anzahl der Einsen zu, die Anzahl der Nullen aber
nicht.
Nach Sprachdefinition müsste aber auch die Anzahl der Nullen zunehmen,
da ja $|w|_{\texttt{0}} \ge 2^{|w|_\texttt{1}}$ sein muss.
Der Widerspruch zeigt wieder, dass $L$ nicht regulär sein kann.
\end{loesung}

\begin{bewertung}
Pumping-Lemma-Beweis: jeder Schritt ein Punkt:
Annahme regulär ({\bf PL}) 1 Punkt,
Pumping Length ({\bf N}) 1 Punkt,
Wort ({\bf W}) 1 Punkt,
Aufteilung ({\bf A}) 1 Punkt,
Widerspruch beim Pumpen ({\bf P}) 1 Punkt,
Schlussfolgerung ({\bf S}) 1 Punkt.
Falls das gleiche Wort wie in der Musterlösung gewählt wurde wird
der Punkt für Schritt~5 nur gegeben, wenn erkannt worden ist, dass
nur Abpumpen zu einem Widerspruch führt.
\end{bewertung}
