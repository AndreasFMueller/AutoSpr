Gibt es einen regulären Ausdruck, mit dem sich die Sprache
\[
L=\{
w\in\{\texttt{0},\texttt{1}\}^*
\,|\,
2|w|_{\texttt{0}}
=
|w|_{\texttt{1}}
\}
\]
über dem Alphabet $\Sigma=\{\texttt{0},\texttt{1}\}$ beschreiben lässt.

\begin{loesung}
Nein, wie man mit dem Pumping-Lemma beweisen kann.
Nehmen wir also an, dass die Sprache $L$ regulär ist.
Dann gibt es die Pumping Length $N$.
Wir konstruieren das Wort $w=\texttt{0}^N\texttt{1}^{2N}$, welches
offensichtlich in $L$ ist.
Gemäss Pumping-Lemma gibt es eine Unterteilung des Wortes in
$w=xyz$ derart, dass $|xy|\le N$, d.~h.~die Teile $x$ und $y$ bestehen
aus lauter \texttt{0}.
Beim Pumpen wird die Anzahl der \texttt{0} verändert, da $|y|>0$ sein
muss, aber die Anzahl der \texttt{1} wird nicht verändert.
Damit kann das gepumpte Wort nicht mehr in $L$ sein.
Der Widerspruch zeigt, dass $L$ nicht regulär sein kann, also auch nicht
durch einen regulären Ausdruck beschrieben werden kann.
\end{loesung}


