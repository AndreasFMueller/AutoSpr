Ein etwas primitives Datenübermittlungsprotokoll verwendet
das Zeichen \texttt{/} als Delimiter, dazwischen können Messages beliebiger
Länge stehen, die aber das Zeichen \texttt{/} nicht verwenden dürfen.
Die Erfahrung hat gezeigt, dass es sehr selten vorkommt, dass einzelne
Zeichen nicht übertragen werden, es fehlen immer gleich mindestens drei
Zeichen. Daher kam jemand auf die Idee die "Ubertragung dadurch zu überprüfen,
dass man jeder Message einen String von \texttt{x}-Zeichen anhängt, der
ein Drittel so lange ist wie die Message:
\begin{align*}
&\texttt{/}\underbrace{\texttt{wasjetzt}}_8\texttt{/}\underbrace{\texttt{xx}}_2\texttt{/}
&\texttt{/}\underbrace{\texttt{noch eine etwas laengere message}}_{32}\texttt{/}\underbrace{\texttt{xxxxxxxxxx}}_{10}\texttt{/}
\end{align*}
Bei der Bestimmung der Länge der Folge von \texttt{x}-Zeichen wird ganzzahlige
Division verwendet, also $\lfloor 8 / 3\rfloor = 2$ und $\lfloor 32 / 3 \rfloor = 10$.
Ihre Aufgabe ist, einen regulären Ausdruck zu formulieren, der korrekte
Messages von fehlerhaften unterscheidet.

\themaL{regulare Ausdrucke}{reguläre Ausdrücke}
\themaL{Pumping Lemma fur regulare Sprachen}{Pumping Lemma für reguläre Sprachen}
\themaL{regular}{regulär}

\begin{loesung}
So einen regulären Ausdruck kann es nicht geben, denn die Sprache 
\[
\{ \texttt{/}w\texttt{/}s\texttt{/}\, |\, w\in \Sigma^*, s\in\{\texttt{x}\}^*,
|s| = \lfloor |w|/3\rfloor
\}
\]
ist nicht regulär, wie wir mit dem Pumping-Lemma beweisen können.

Nehmen wir an, die Sprache $L$ sei regulär. Nach dem Pumping-Lemma gibt
es dann die Pumping Length $N$. Wir konstruieren jetzt ein Wort in $L$:
\[
w=
\texttt{/a}^{3N}\texttt{/x}^N\texttt{/}
\]
Das Pumping-Lemma sagt ausserdem, dass $w$ zerlegt werden können muss in
drei Teile $w=xyz$ so, dass $|xy|\le N$, $|y|>0$ und alle aufgepumpten
Wörter $xy^kz$ sollten wieder in $L$ liegen. Dies ist nur möglich,
wenn $x$ und $y$ vollständig im Teile $\texttt{a}^{3N}$ enthalten ist.
Beim Aufpumpen wird $xy^kz$ also zusätzlick $k|y|$ Zeichen \texttt{a}
enthalten, bei dreimaligem Aufpumpen ändert sich also die Zahl der 
\texttt{a}s dermassen, dass auch die Zahl der \texttt{x} vergrössert
werden müsste, um wieder ein Wort in der Sprache zu bekommen. Die Zahl
der \texttt{x} ändert aber beim Pumpen nie, also sind nicht mehr alle
aufgepumpten Wörter in $L$. Dieser Widerspruch zeigt, dass $L$ nicht
regulär sein kann.
\end{loesung}

\begin{bewertung}
Pumping Lemma (\textbf{L}) 1 Punkt, Pumping Length (\textbf{N}) 1 Punkt,
geeignetes Wort (\textbf{W}) 1 Punkt, Zerlegung (\textbf{Z}) 1 Punkt,
Widerspruch beim Pumpen (\textbf{P}) 1 Punkt,
Schlussfolgerung nicht regulär (\textbf{R}) 1 Punkt.
\end{bewertung}
