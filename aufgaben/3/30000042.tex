Sei $p$ eine fest natürliche Zahl. Sind die folgenden Sprachen regulär?
\begin{teilaufgaben}
\item $L_1=\{ \texttt{a}^n\,|\, \text{$n$ ist durch $p$ teilbar}\}$
\item $L_2=\{ \texttt{a}^n\,|\, \text{$n$ ist nicht durch $p$ teilbar}\}$
\item Ist die Sprache $L=\{\texttt{a}^n\texttt{b}^m\,|\,\text{$n$ und $m$ sind nicht beide durch $p$ teilbar}\}$
\end{teilaufgaben}

\themaL{regular}{regulär}
\themaL{Mengenoperationen fur regulare Sprachen}{Mengenoperationen für reguläre Sprachen}

\begin{loesung}
\begin{teilaufgaben}
\item
Teilbarkeit durch $p$ kann man mit dem folgenden Automaten feststellen:
\[
\xymatrix{
*+\txt{}\ar[r]
        &*++[o][F=]{0}\ar[r]^{\texttt{a}} 
                &*++[o][F]{1} \ar[r]^{\texttt{a}}
                	&*++[o][F]{2} \ar[r]^{\texttt{a}}
				&{\dots} \ar[r]^{\texttt{a}}
					&*++[o][F]{p-1} \ar@/^20pt/[llll]^{\texttt{a}}
}
\]
daher ist $L_1$ regulär.
\item
Die Sprache $L_2$ ist das Komplement der Sprache $L_1$ in 
$\{\texttt{a}\}^*$:
\[
L_2=\{\texttt{a}\}^*\setminus L_1.
\]
Da sowohl $\{\texttt{a}\}^*$ als auch $L_1$ regulär sind, ist auch $L_2$
regulär.
\item
Schreiben wir $L_1(\texttt{a})=L_1$ und $L_2(\texttt{b})$ für die Sprache
bestehenden aus Wörter der Form $\texttt{b}^n$, für die $n$ durch $p$ teilbar
ist, und analog für $L_2(\texttt{a})$ und $L_2(\texttt{b})$.
Nach a) und b) sind diese Sprachen alle regulär.
Und natürlich sind auch die Verkettungen $L_i(\texttt{a})L_j(\texttt{b})$
regulär.
Die Sprache $L$ ist daher als Vereinigung
\[
L=
L_1(\texttt{a})L_2(\texttt{b})
\cup
L_2(\texttt{a})L_1(\texttt{b})
\cup
L_2(\texttt{a})L_2(\texttt{b})
\]
ebenfalls regulär.
\qedhere
\end{teilaufgaben}
\end{loesung}

