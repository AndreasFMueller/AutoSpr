F"ur die Input-Validierung in einer Anwendung wurde beschlossen, regul"are
Ausdr"ucke zu verwenden.
Gegen Sie f"ur jede im folgenden beschriebene Art von Eingabefeldern einen
regul"aren Ausdruck an, der genau auf die akzeptable Eingaben passt.
Sie d"urfen annehmen, dass Whitespace am Anfang und Ende eines Eingabefeldes
bereits entfernt worden ist.
\begin{teilaufgaben}
\item
Eingabefeld f"ur eine Pr"ufungsnote: eine Zahl $n$ mit $1\le n\le 6$
und maximal zwei Nachkommastellen.
Zahlen mit Dezimalpunkt aber ohne Nachkommateil m"ussen akzeptiert sein.
\item
Eingabefeld f"ur ein Datum im europ"aischen Format: Tageszahl mit Punkt,
Monatszahl mit Punkt Jahreszahl. Zwischen den Komponenten d"urfen Leerzeichen
vorhanden sein.
Sowohl die Tageszahl wie auch die Monatszahl d"urfen f"uhrende Nullen haben.
Es sollen nur tats"achlich existierende Monatszahlen
und Tageszahlen eingegeben werden k"onnen, es wird aber nicht verlangt,
dass nicht existierende Kombinationen wie der 31.~Februar ausgeschlossen
sind.
\end{teilaufgaben}

\begin{loesung}
\begin{teilaufgaben}
\item In der Standard-Syntax aus der Vorlesung ist dies
\[
\texttt{[1-5](|.(|[0-9](|[0-9])))|6(|.(|0(|0)))}
\]
Mit erweiterer Syntax f"ur regul"are Ausdr"ucke kann des etwas kompakter
ausdr"ucken:
\[
\texttt{[1-5](.[0-9]\{0,2\})?|6(.[0]\{0,2\})?}
\]
\item Die einzeln Teile sind
\begin{align*}
r_{\text{Tageszahl}}
&=
\texttt{(|0)[1-9]|[12][0-9]|3[01].}
\\
r_{\text{Monatszahl}}
&=
\texttt{1[0-2]|(|0)[1-9].}
\\
r_{\text{Jahreszahl}}
&=
\texttt{[12][0-9][0-9][0-9][0-9]}
\\
\end{align*}
Zusammengesetzt zu einem Datum mit optionalem Whitespace
\begin{align*}
r
&=
r_{\text{Tageszahl}}
\texttt{ *}
r_{\text{Monatszahl}}
\texttt{ *}
r_{\text{Jahreszahl}}
\\
&=
\texttt{(|0)[1-9]|[12][0-9]|3[01].}
\texttt{ *}
\texttt{1[0-2]|(|0)[1-9].}
\texttt{ *}
\texttt{[12][0-9][0-9][0-9][0-9]}
\end{align*}
\end{teilaufgaben}
\end{loesung}

\begin{bewertung}
\begin{teilaufgaben}
\item 2 Punkte
\item Tageszahl ({\bf T}) 1 Punkt, Monatszahl ({\bf M}) 1 Punkt,
Jahreszahl ({\bf J}) 1 Punkt, Verkettung (Whitespace) ({\bf V}) 1 Punkt.
\end{teilaufgaben}
\end{bewertung}

