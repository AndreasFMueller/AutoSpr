Die Sprache $L$, bestehend aus Wörtern über dem Alphabet
$\Sigma=\{\texttt{0},\texttt{1}\}$, die Einsen immer mindestens als
Dreiergruppen enthalten, wird vom deterministischen
endlichen Automaten
\vspace*{-10pt}
\begin{center}
\def\l{1.3}
\def\r{0.30}
\def\zustand#1#2{
	\fill[color=white] #1 circle[radius=\r];
	\node at #1 {$#2\mathstrut$};
	\draw #1 circle[radius=\r];
}
\def\akzeptierzustand#1#2{
	\zustand{#1}{#2}
	\draw #1 circle[radius={\r-0.05}];
}
\def\pfeil#1#2{
	\draw[->,shorten >= 0.3cm,shorten <= 0.3cm] #1 -- #2;
}
\def\punkt#1#2#3{($#3*#1+#2-#3*#2$)}
\begin{tikzpicture}[>=latex,thick]
\coordinate (S) at ({0.8*\l},{0.8*\l});
\coordinate (q0) at (0,0);
\coordinate (q1) at (\l,0);
\coordinate (q2) at (\l,\l);
\coordinate (q3) at (0,\l);
\coordinate (e) at ({(1+sqrt(3)/2)*\l},{0.5*\l});
\pfeil{(S)}{(q0)}
\akzeptierzustand{(q0)}{q_0}
\akzeptierzustand{(q3)}{q_3}
\zustand{(q1)}{q_1}
\zustand{(q2)}{q_2}
\zustand{(e)}{e}

\draw[->,shorten >= 0.30cm,shorten <= 0.30cm]
	(q0) to[out=150,in=-150,distance=1.5cm] (q0);
\node at ($(q0)+(-1,0)$) [left] {\texttt{0}};

\draw[->,shorten >= 0.30cm,shorten <= 0.30cm]
	(q3) to[out=150,in=-150,distance=1.5cm] (q3);
\node at ($(q3)+(-1,0)$) [left] {\texttt{1}};

\draw[->,shorten >= 0.30cm,shorten <= 0.30cm]
	(e) to[out=-30,in=30,distance=1.5cm] (e);
\node at ($(e)+(1.0,0)$) [right] {$\Sigma$};

\pfeil{(q0)}{(q1)}
\node at \punkt{(q0)}{(q1)}{0.5} [below] {\texttt{1}};

\pfeil{(q1)}{(q2)}
\node at \punkt{(q1)}{(q2)}{0.5} [right] {\texttt{1}};

\pfeil{(q2)}{(q3)}
\node at \punkt{(q2)}{(q3)}{0.5} [above] {\texttt{1}};

\pfeil{(q3)}{(q0)}
\node at \punkt{(q3)}{(q0)}{0.5} [left] {\texttt{0}};

\pfeil{(q1)}{(e)}
\node at \punkt{(q1)}{(e)}{0.7} [below right] {\texttt{0}};

\pfeil{(q2)}{(e)}
\node at \punkt{(q2)}{(e)}{0.7} [above right] {\texttt{0}};

\end{tikzpicture}
\end{center}
\vspace*{-10pt}
akzeptiert.
Verwenden Sie den Automaten, um die folgenden Wörter wie im Pumping Lemma
versprochen zum Aufpumpen zu zerlegen.
\begin{teilaufgaben}
\item
$w=\texttt{111110000}$
\item
$w=\texttt{011111000}$
\item
$w=\texttt{1110}$
\end{teilaufgaben}

\thema{DEA}
\thema{Pumping Lemma fur regulare Sprachen}{Pumping Lemma für reguläre Sprachen}

\begin{loesung}
\definecolor{darkgreen}{rgb}{0,0.6,0}
Es muss jeweils eine Aufteilung des Wortes $w$ in drei Teile
$w={\color{darkgreen}x}{\color{red}y}{\color{blue}z}$ gefunden
werden derart, dass der Teil $y$ beliebig aufgepumpt werden kann.
\begin{teilaufgaben}
\item
$w=\texttt{{\color{darkgreen}111}{\color{red}1}{\color{blue}10000}}$,
d.~h.~$x=\texttt{\color{darkgreen}111}$, $y=\texttt{\color{red}1}$
und $z=\texttt{\color{blue}10000}$.
\item
$w=\texttt{{\color{red}0}{\color{blue}11111000}}$,
d.~h.~$x=\varepsilon$, $y=\texttt{\color{red}0}$ und
$z=\texttt{\color{blue}11111000}$.
\item
$w=\texttt{\color{red}1110}$, die Teile $x=\varepsilon$ und $z=\varepsilon$
haben Länge $0$.
\qedhere
\end{teilaufgaben}
\end{loesung}
