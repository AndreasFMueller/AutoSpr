%
% wordl.tex -- Wordle-Beispiel
%
% (c) 2019 Prof Dr Andreas Müller, Hochschule Rapperswil
%
\documentclass[tikz,12pt]{standalone}
\usepackage{amsmath}
\usepackage{times}
\usepackage{txfonts}
\usepackage{pgfplots}
\usepackage{csvsimple}
\usetikzlibrary{arrows,intersections,math}
\begin{document}
\definecolor{wordlegreen}{rgb}{0.32,0.55,0.30}
\definecolor{wordleolive}{rgb}{0.71,0.62,0.23}
\def\h{1}
\def\s{0.05}
\def\emptyblock#1#2{
	\fill[color=gray]
		({#1*\h+\s},{-#2*\h+\s})
		rectangle
		({(#1+1)*\h-\s},{(-#2+1)*\h-\s});
	\fill[color=black]
		({#1*\h+2*\s},{-#2*\h+2*\s})
		rectangle
		({(#1+1)*\h-2*\s},{(-#2+1)*\h-2*\s});
}
\def\block#1#2#3#4{
	\fill[color=#4]
		({#1*\h+\s},{-#2*\h+\s})
		rectangle
		({(#1+1)*\h-\s},{(-#2+1)*\h-\s});
	\node[color=white] at ({(#1+0.5)*\h},{(-#2+0.5)*\h-0.04})
		{\Large\texttt{#3}\strut};
}
\begin{tikzpicture}[>=latex,thick]

\block{1}{1}{B}{gray}
\block{2}{1}{R}{gray}
\block{3}{1}{E}{gray}
\block{4}{1}{A}{wordleolive}
\block{5}{1}{K}{gray}

\block{1}{2}{M}{gray}
\block{2}{2}{O}{gray}
\block{3}{2}{U}{gray}
\block{4}{2}{T}{wordleolive}
\block{5}{2}{H}{wordleolive}

\block{1}{3}{P}{gray}
\block{2}{3}{A}{wordleolive}
\block{3}{3}{S}{gray}
\block{4}{3}{T}{wordleolive}
\block{5}{3}{A}{gray}

\block{1}{4}{A}{wordleolive}
\block{2}{4}{N}{wordleolive}
\block{3}{4}{T}{wordleolive}
\block{4}{4}{I}{gray}
\block{5}{4}{C}{wordleolive}

\block{1}{5}{C}{wordlegreen}
\block{2}{5}{H}{wordlegreen}
\block{3}{5}{A}{wordlegreen}
\block{4}{5}{N}{wordlegreen}
\block{5}{5}{T}{wordlegreen}

\foreach \x in {1,2,3,4,5}{
	\emptyblock{\x}{6}
}

\foreach \y in {1,2,3,4,5,6}{
	\node at ({0.5*\h},{(0.5-\y)*\h}) {$\mathstrut\scriptstyle \y$};
}

\end{tikzpicture}
\end{document}

