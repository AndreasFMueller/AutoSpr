Für die Input-Validierung in einer Anwendung wurde beschlossen,
reguläre Ausdrücke zu verwenden.
Geben Sie für jede im folgenden beschriebene Art von Eingabefeldern
einen regulären Ausdruck an, der genau auf die akzeptablen Eingaben passt.
Sie dürfen annehmen, dass Whitespace am Anfang und Ende des Eingabefeldes
bereits entfernt worden ist.
\begin{teilaufgaben}
\item Eingabefeld für Postleitzahl: eine vier- oder fünfstellige Dezimalzahl
ohne führende Nullen.
\item Eingabefeld für natürliche Zahlen in dezimal, oktal oder hexadezimal.
Hexadezimalzahlen verwenden die Ziffern \texttt{0}--\texttt{9} und die
Buchstaben \texttt{a}--\texttt{f} und müssen mit \texttt{0x} eingeleitet
sein.
Oktalzahlen verwenden nur die Ziffern \texttt{0}--\texttt{7} und sind
durch eine führende \texttt{0} gekennzeichnet.
Dezimalzahlen verwenden \texttt{0}--\texttt{9} und beginnen nicht mit einer
\texttt{0}.
\item Eingabefeld für eine Kreditkartennummer bestehende aus 4 Gruppen
von jeweils vier Dezimalziffern, optional getrennt durch ein Leerzeichen.
\end{teilaufgaben}

\themaL{regulare Ausdrucke}{reguläre Ausdrücke}
\themaL{regular}{regulär}

\begin{loesung}
\begin{teilaufgaben}
\item 
Vierstellige Zahl ohne führende Null:
$r_1 = \texttt{[1-9][0-9][0-9][0-9]}$.
Dieser Zahl darf man jetzt noch optional eine weitere Ziffer anhängen:
\[
r=r_1\texttt{(|[0-9])}
=
\texttt{[1-9][0-9][0-9][0-9](|[0-9])}
\]
oder etwas einfacher unter Verwendung einer erweiterten Regex-Syntax:
\texttt{[1-9][0-9]\{3,4\}}
\item
Man beachte, dass die Null nur als Hexadezimalzahl oder als Oktalzahl
geschrieben werden kann, eine Dezimalzahl \texttt{0} im Sinne der Spezifikation
dieser Aufgabe gibt es nicht.
Dies ist aber kein Verlust, denn die Zeichenkette \texttt{0} ist spezifiziert
und ihr kann der korrekte Wert $0$ zugeordnet werden, unabhängig davon, ob
der Benutzer die Eingabe als Oktalzahl oder als Dezimalzahl verstanden hat.
\begin{center}
\begin{tabular}{ll}
Ausdruck für Dezimalzahlen:&\texttt{[1-9][0-9]*}\\
Ausdruck für Oktalzahlen:&\texttt{0[0-7]*}\\
Ausdruck für Hexadezimalzahlen:&\texttt{0x[0-9a-f]+}\\
Zusammen:&\texttt{[1-9][0-9]*|0[0-9]*|0x[0-9a-f]+}
\end{tabular}
\end{center}
\item
Eine Vierergruppe ist $r_1 = \texttt{[0-9][0-9][0-9][0-9]}$,
eine Vierergruppe mit optionalem voranstehendem Leerzeichen ist
$r_2 = \texttt{(|\blank)}r_1$.
Eine Kreditkartennummer setzt sich aus einem $r_1$ und genau drei
$r_2$ zusammen, der gesuchte reguläre Ausdruck ist also
\[
r = r_1r_2r_2r_2 = 
\texttt{[0-9][0-9][0-9][0-9]%
(|\blank)[0-9][0-9][0-9][0-9]%
(|\blank)[0-9][0-9][0-9][0-9]%
(|\blank)[0-9][0-9][0-9][0-9]}
\]
oder etwas einfacher mit erweiterter Regex-Syntax:
\[
\texttt{[0-9]\{4\}(\blank?[0-9]\{4\})\{3\}}
\]
Eine noch kürzere Lösung wäre
\begin{equation}
\texttt{([0-9]\{4\}\blank?)\{4\}}
\label{30000052}
\end{equation}
Dieser Ausdruck lässt zwar zu, dass am Ende des Eingabestrings ein
Leerzeichen akzeptiert wird, doch gemäss Voraussetzung gibt es solche
Leerzeichen im Input nicht.
Der reguläre Ausdruck~\eqref{30000052} ist also immer noch eine korrekte 
Lösung für das Validierungs-Problem.
\end{teilaufgaben}
\end{loesung}

\begin{bewertung}
Für jede Teilaufgabe 2 Punkte.
\end{bewertung}

