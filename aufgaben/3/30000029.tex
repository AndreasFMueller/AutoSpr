Finden sie reguläre Ausdrücke für die folgenden Sprachen
\begin{teilaufgaben}
\item $L=\{w\mid \text{$w$ beginnt mit {\tt 1} und endet mit {\tt 0}}\}$
\item $L=\{w\mid \text{$w$ enthält den Teilstring {\tt 0101}}\}$
\item $L=\{w\mid \text{an jeder ungeraden Position von $w$ befindet sich ein {\tt 1}}\}$
\item $L=\{w\mid |w|\le 5\}$
\end{teilaufgaben}
Alle Sprachen sind über dem Alphabet $\Sigma=\{{\tt 0},{\tt 1}\}$ zu
betrachten.

\themaL{regular}{regulär}
\themaL{regulare Ausdrucke}{reguläre Ausdrücke}

\begin{loesung}
\begin{teilaufgaben}
\item ${\tt 1}({\tt 0}|{\tt 1})^*{\tt 0}$
\item $({\tt 0}|{\tt 1})^* {\tt 0101} ({\tt 0}|{\tt 1})^*$
\item $({\tt 1}({\tt 0}|{\tt 1}))^*(\varepsilon|{\tt 1})$
\item
Am einfachsten wäre diese Sprache durch den regulären Ausdruck 
\[
(\texttt{0}|\texttt{1})\{0,5\}
\]
abzubilden, allerdings ist dies ein erweiterter regulärer Ausdruck.
Wünschen wir stattdessen einen regulären Ausdruck, der nur die
grundlegenden Elemente verwendet, müssen wir eine Konstruktion haben,
welche $r\{m,n\}$ ausdrückt. Dafür gibt es mehrere Möglichkeiten.
Man könnte zum Beispiel die ersten $m$ Kopien von $r$ obligatorisch
erklären, und die verbleibenden fakultativ:
\[
\underbrace{r\dots r}_m\underbrace{(r|\varepsilon)\dots(r|\varepsilon)}_{n-m}
\]
Alternativ könnte man $r\{m,n\}$ als eine Alternative von verschieden langen
Strings aus $r$ aufbauen:
\[
(\underbrace{r\dots r}_m)|
(\underbrace{r\dots r}_{m+1})|
(\underbrace{r\dots r}_{m+2})|\dots
(\underbrace{r\dots r}_{n - 1})|
(\underbrace{r\dots r}_{n})
\]
Der Vorteil der zweiten Konstruktion besteht darin, dass man daraus, welcher
Teil des regulären Ausdrucks gepasst hat, sofort ablesen kann, wieviele
Komponenten $r$ das getestete Wort umfasst, der reguläre Ausdruck kann
die Anzahl Komponenten ``zählen''.

Wendet man die zweite Lösung auf die Aufgabe an, erhält man.
\[
\varepsilon|
(\texttt{0}|\texttt{1})|
((\texttt{0}|\texttt{1}) (\texttt{0}|\texttt{1}))|
((\texttt{0}|\texttt{1}) (\texttt{0}|\texttt{1}) (\texttt{0}|\texttt{1}))|
((\texttt{0}|\texttt{1}) (\texttt{0}|\texttt{1}) (\texttt{0}|\texttt{1}) (\texttt{0}|\texttt{1}))|
((\texttt{0}|\texttt{1}) (\texttt{0}|\texttt{1}) (\texttt{0}|\texttt{1}) (\texttt{0}|\texttt{1}) (\texttt{0}|\texttt{1}))
\]
Der erste Ansatz gibt dagegen als Lösung
\[
(\texttt{0}|\texttt{1}|)
(\texttt{0}|\texttt{1}|)
(\texttt{0}|\texttt{1}|)
(\texttt{0}|\texttt{1}|)
(\texttt{0}|\texttt{1}|)
\qedhere
\]
\end{teilaufgaben}
\end{loesung}
