Sei $\Sigma$ ein Alphabet und $L$ die Sprache
\[
L=\{ w\in\Sigma^* \;|\; |w|_a \ne |w|_b\;\forall a,b\in\Sigma, a\ne b\}
\]
bestehend aus Wörtern, die keine zwei verschiedenen Buchstaben in gleicher
Anzahl enthalten.
Gibt es einen regulären Ausdruck dafür?

\themaL{regulär}{regulär}
\themaL{Pumping Lemma fur regulare Sprachen}{Pumping Lemma für reguläre Sprachen}

\begin{loesung}
Es gibt keinen regulären Ausdruck dafür, weil die Sprache $L$ nicht regulär
ist, wie man mit dem Pumping Lemma einsehen kann:
\begin{enumerate}
\item Angenommen, $L$ ist regulär.
\item Dann gibt es die Pumping Length $N$.
\item Wir konstruieren das Wort $w=\texttt{a}^N\texttt{b}^{N+c}$, wobei
wir die Zahl $c>0$ später bestimmen werden.
Wegen $N\ne N+c$ ist $w$ tatsächlich ein Wort in $L$.
\item Nach dem Pumping-Lemma gibt es eine Unterteilung $w=xyz$, wobei
$x$ und $y$ ausschliesslich aus \texttt{a} bestehen.
Wir wählen $c=|y|$.
\item Pumpt man jetzt das Wort $w$ mit $k=2$ auf, bildet man also
$w_2=xy^2z$, dann wird der \texttt{a}-Teil um $|y|=c$ Zeichen \texttt{a}
verlängert, er bekommt also neu die Länge $N+c$, und ist damit gleich
lang wie der \texttt{b}-Teil.
Das aufgepumpte Wort $w_2$ ist daher nicht mehr in der Sprache $L$.
\item Das Pumping Lemma verlangt, dass jedes aufgepumpte Wort in $L$
sein muss.
Dieser Widerspruch zeigt, dass die Annahme, $L$ sei regulär, nicht
zutreffen kann.
\qedhere
\end{enumerate}
\end{loesung}

\begin{diskussion}
Wörter der Form $a^Nb^{N+1}$ oder $a^Nb^{2N}$ sind nicht geeignet, da
beim Pumpen nicht immer $|w|_a=|w|_b$ erreicht werden kann.
Statt des Wortes $a^Nb^{N+c}$ kann man auch ein Wort der Form $a^Nb^{cN}$
versuchen.
Beim Pumpen wird daraus
$a^{N+(k-1)|y|}b^{cN}$, welches nicht mehr in der Sprache ist, wenn
\[
N+(k-1)|y|=cN
\qquad\Leftrightarrow\qquad
({\color{red}k}-1)|y|=({\color{red}c}-1)N
\]
Die {\color{red}roten} Variablen sind noch zu wählen.
Die Gleichung wird erfüllt, wenn man $c-1=|y|$ und $k-1=N$ wählt, also
\[
c=|y|+1\qquad\text{und}\qquad k=N+1.
\]

Die folgende Idee ist sehr amüsant, funktioniert aber leider nicht ganz.
Man verwendet das Wort
\[
w=\texttt{a}^{N+1}\texttt{b}^N\texttt{c}^{N-1}\texttt{d}^{N-2}\dots
\]
Ausser dem Buchstaben \texttt{a} kommt also jeder andere Buchstabe
mit jeder beliebigen Anzahl $\le N$ vor. 
Pumpt man jetzt ab, wird die Anzahl der \texttt{a} verringert, ihre
Anzahl ist jetzt also auch $\le N$, gleich wie die Anzahl eines der
anderen Buchstaben.
Leider funktioniert dieser Beweis nur, wenn im Alphabet genügend viele
verschiedene Buchstaben zur Verfügung stehen, also $|\Sigma| > N$.
\end{diskussion}

\begin{bewertung}
Jeder Punkt des Pumping-Lemma-Beweises ein Punkt:
Annahme ({\bf A}), Pumping Length ({\bf N}), Beispielwort ({\bf W}),
Zerlegung ({\bf Z}), Widerspruch beim Pumpen ({\bf P}), 
Folgerung ({\bf F}).
Damit der Punkt {\bf P} gegeben wurde, musste in diesem Schritt zum 
Ausdruck gebracht sein, dass beim Pumpen die Gleichheit der Anzahlen
von $a$ und $b$ angestrebt wurde.
\end{bewertung}


