Sei $\Sigma$ ein Alphabet und $L$ die Sprache
\[
L=\{ w\in\Sigma^* \;|\; |w|_a \ne |w|_b\;\forall a,b\in\Sigma, a\ne b\}
\]
bestehend aus W"ortern, die keine zwei verschiedenen Buchstaben in gleicher
Anzahl enthalten.
Gibt es einen regul"aren Ausdruck daf"ur?

\begin{loesung}
Es gibt keinen regul"aren Ausdruck daf"ur, weil die Sprache $L$ nicht regul"ar
ist, wie man mit dem Pumping Lemma einsehen kann:
\begin{enumerate}
\item Angenommen, $L$ ist regul"ar.
\item Dann gibt es die Pumping Length $N$.
\item Wir konstruieren das Wort $w=\texttt{a}^N\texttt{b}^{N+c}$, wobei
wir die Zahl $c>0$ sp"ater bestimmen werden.
Wegen $N\ne N+c$ ist $w$ tats"achlich ein Wort in $L$.
\item Nach dem Pumping-Lemma gibt es eine Unterteilung $w=xyz$, wobei
$x$ und $y$ ausschliesslich aus \texttt{a} bestehen.
Wir w"ahlen $c=|y|$.
\item Pumpt man jetzt das Wort $w$ mit $k=2$ auf, bildet man also
$w_2=xy^2z$, dann wird der \texttt{a}-Teil um $|y|=c$ Zeichen \texttt{a}
verl"angert, er bekommt also neu die L"ange $N+c$, und ist damit gleich
lang wie der \texttt{b}-Teil.
Das aufgepumpte Wort $w_2$ ist daher nicht mehr in der Sprache $L$.
\item Das Pumping Lemma verlangt, dass jedes aufgepumpte Wort in $L$
sein muss.
Dieser Widerspruch zeigt, dass die Annahme, $L$ sei regul"ar, nicht
zutreffen kann.
\qedhere
\end{enumerate}
\end{loesung}

\begin{bewertung}
Jeder Punkt des Pumping-Lemma-Beweises ein Punkt.
\end{bewertung}


