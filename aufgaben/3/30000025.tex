In dieser Aufgabe geht es um die Frage, ob man binäre Ganzzahlen mit
Hilfe eines endlichen Automaten auf Gleichheit testen kann.
Sei $\Sigma=\{\texttt{0},\texttt{1},\texttt{=}\}$. Ist die Sprache
\begin{equation}
L=\{ w\texttt{=}w\,|\,w\in \{\texttt{0},\texttt{1}\}^*\}
\label{30000025:ref}
\end{equation}
regulär?

\thema{Pumping Lemma für reguläre Sprachen}
\thema{regulär}

\begin{hinweis}
Bitte beachten Sie die beiden unterschiedlichen Bedeutungen von $=$
bzw.~\texttt{=} in Formel~\eqref{30000025:def}.
Das eine ist das mathematische Gleichheitszeichen, das andere ein Symbol
im Alphabet $\Sigma$.
Die beiden Zeichen sind zwar in zwei verschiedenen Schriftarten gesetzt,
die normalerweise Zeichen von Formelsymbolen zu unterscheiden gestatten,
aber im Falle des Gleichheitszeichens sind die Unterschiede verschwindend
klein.

Die Gleichung~\eqref{30000025:ref} besagt, dass die Sprache $L$ aus
Zeichenketten von Nullen, Einsen und Glechheitszeichen besteht, die
wie folgt strukturiert sind.
Sie beginnen mit einer Folge von Nullen und Einsen, dann kommt ein
Gleichheitszeichen.
Anschliessend wird die Folge von Nullen und Einsen nochmals wiederholt.
\end{hinweis}

\begin{loesung}
Die Sprache ist nicht regulär, wie man mit Hilfe des Pumping
Lemmas beweisen kann.
\ding{182}
Dazu nehmen wir an, die Sprache sei regulär.
\ding{183}
Das Pumping Lemma garantiert, dass es eine Zahl $N$ gibt, so dass
sich Wörter länger als diese Zahl in einer speziellen Art
zerlegen und ``aufpumpen'' lassen.
\ding{184}
Wir wählen das Wort
\[
w=\texttt{10}^N\texttt{=10}^N,
\]
welches offenbar zur Sprache gehört und auch länger ist als $N$.
\ding{185}
Das Pumping Lemma garantiert, dass $w=xyz$  geschrieben werden kann
mit $|xy|\le N$ und $|y|>0$:
\begin{center}
\includeagraphics[]{pl-1.pdf}
\end{center}
Aus der Konstruktion sieht man, dass $y$ eine der beiden Formen
\[
\texttt{0}^k
\quad\text{oder}\quad
\texttt{10}^k
\]
haben muss (die zweite nur falls $|x|=0$).
\ding{186}
Welche Form auch immer $y$ hat, nach dem Aufpumpen wird
auf der linken Seite des Gleichheitszeichens eine Binärzahl mit
einer führenden \texttt{1} und mit mehr Stellen stehen, die Gleichheit
ist also nicht mehr erfüllt, das aufgepumpte Wort ist nicht in der
Sprache, $xy^lz\not\in L$.
\ding{187}
Dieser Widerspruch zur Aussage des Pumping
Lemmas zeigt, dass die Annahme, $L$ sei regulär, falsch gewesen
sein muss.
\end{loesung}


