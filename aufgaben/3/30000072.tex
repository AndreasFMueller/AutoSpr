Finden Sie einen deterministischen endlichen Automaten, der
die Sprache
\[
L
=
\{
w = \{\texttt{0},\texttt{1}\}^*
\;|\;
|w|_{\texttt{0}}
\equiv
|w|_{\texttt{1}}
\mod 3
\}
\]
akzeptiert,
bestehend aus Wörtern, deren Anzahl von Nullen und Einsen den 
gleichen Rest bei Teilung durch drei haben.

\thema{regulär}
\thema{DEA}

\begin{loesung}
Hier sind zwei mögliche Ansätze, wie man einen solchen endlichen
Automaten konstruieren kann:
\begin{enumerate}
\item
Man kann Automaten $A_1$ und $A_2$ konstruieren, jeweils mit drei Zuständen,
die für den Dreierrest der Anzahl der Nullen bzw.~der Einsen steht.
Dann baut man den Produktautomaten und wählt als Akzeptierzustände
die Paare bestehend aus Zuständen von $A_1$ und $A_2$, die für den gleichen
Rest stehen.
\item
Man kann auch direkt vorgehen, und einen Automaten konstruieren,
dessen drei Zustände für den Dreierrest der Differenz der Anzahl der
Nullen und Einsen steht.
Um die Übergänge zu finden, muss man sich nur bei jeder \texttt{0} und
jeder \texttt{1} überlegen, wie sie den Dreierrest der Differenz der
Anzahl beeinflussen.
\item
Man könnte den Satz von Myhill-Nerode verwenden, um die Zustände des
Automaten zu rekonstruieren.
\end{enumerate}
Wir verwenden für die Lösung den zweiten Ansatz.
Das Zustandsdiagramm 
\begin{center}
\begin{tikzpicture}[>=latex,thick]

\coordinate (O) at (0,0);
\coordinate (A) at (2,0);
\coordinate (B) at (4,0);
\coordinate (C) at (6,0);

\node at (A) {$q_0$};
\node at (B) {$q_1$};
\node at (C) {$q_2$};

\draw[->,shorten >= 0.3cm,shorten <= 0.3cm] (O) -- (A);

\draw (A) circle[radius=0.3];
\draw (A) circle[radius=0.25];
\draw (B) circle[radius=0.3];
\draw (C) circle[radius=0.3];

\draw[->,shorten >= 0.3cm,shorten <= 0.3cm]
	(A) to[out=30,in=150] (B);
\draw[->,shorten >= 0.3cm,shorten <= 0.3cm]
	(B) to[out=30,in=150] (C);

\draw[->,shorten >= 0.3cm,shorten <= 0.3cm]
	(C) to[out=110,in=0] ($(C)+(-0.7,1.0)$)
	--
	($(A)+(0.7,1.0)$) to[out=180,in=70] (A);

\node at ($0.5*(A)+0.5*(B)+(0,0.3)$) [above] {\texttt{0}};
\node at ($0.5*(B)+0.5*(C)+(0,0.3)$) [above] {\texttt{0}};
\node at ($(B)+(0,1.0)$) [above] {\texttt{0}};

\draw[<-,shorten >= 0.3cm,shorten <= 0.3cm]
	(A) to[out=-30,in=-150] (B);
\draw[<-,shorten >= 0.3cm,shorten <= 0.3cm]
	(B) to[out=-30,in=-150] (C);

\draw[<-,shorten >= 0.3cm,shorten <= 0.3cm]
	(C) to[out=-110,in=0] ($(C)-(0.7,1.0)$)
	--
	($(A)+(0.7,-1.0)$) to[out=180,in=-70] (A);

\node at ($0.5*(A)+0.5*(B)-(0,0.3)$) [below] {\texttt{1}};
\node at ($0.5*(B)+0.5*(C)-(0,0.3)$) [below] {\texttt{1}};
\node at ($(B)-(0,1.0)$) [below] {\texttt{1}};

\end{tikzpicture}
\end{center}
beschreibt diesen Automaten.
Der Zustand $q_k$ codiert den Dreierrest $k$ der Differenz
$|w|_{\texttt{0}}-|w|_{\texttt{1}}$.
\end{loesung}

\begin{bewertung}
Korrekte Idee für ein Konstruktionsprinzip ({\bf I}) 1 Punkt,
Startzustand ({\bf S}) 1 Punkt,
Akzeptierzustände ({\bf A}) 1 Punkt,
Automat ist deterministisch ({\bf D}) 1 Punkt,
Reihenfolge von $\texttt{0}$ und $\texttt{1}$ spielt keine Rolle ({\bf R})
1 Punkt,
Automat akzeptiert genau die Sprache $L$ ({\bf L}) 1 Punkt.
\end{bewertung}
