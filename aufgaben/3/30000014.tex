EAN-Code und ISBN-Nummer verwenden zum Schutz vor "Ubertragungsfehlern
eine Prüfziffer. Zur Berechnung der Prüfziffer werden alle
Stellen abwechselnd mit 3 oder 1 multipliziert, diese Produkte werden
addiert. Der Zehner-Rest der Summe ist dann die Prüfziffer.
Beispiel: der Code {\tt 1291} bekommt die Prüfziffer
\[
1\cdot 3+2\cdot1+9\cdot3+1\cdot 1=35\equiv 5\mod10.
\]
Der mit Prüfziffer gesicherte Code {\tt 1291} ist also {\tt 12915}.
Gibt es einen regulären Ausdruck, auf den nur die Codes mit korrekter
Prüfziffer passen?

\themaL{regulare Ausdrucke}{reguläre Ausdrücke}

\begin{loesung}
Ja. Die Sprache der Codes mit korrekter Prüfziffer ist regulär,
denn es gibt einen endlichen Automaten dafür, den wir unten konstruieren
werden. Folglich gibt es auch einen regulären Ausdruck dafür.

Auf den ersten Blick ist vielleicht etwas überraschend, dass
ein endlicher Automat die Prüfziffer ausrechnen können soll,
schliesslich hat man ja gelernt, dass ein endlicher Automat
nicht rechnen könne und sich das Zwischenresultat nicht merken
können. Allerdings gibt es nicht viel, was man sich merken
muss. Von der Summe muss man sich nur an die letzte Ziffer
erinnern, da ja nur der Zehnerrest interessiert. Ausderdem
muss man sicht daran erinnern, ob die nächste Ziffer mit
3 oder mit 1 multipliziert werden muss. An eine solch beschränkte
Informationsmenge kann sich eine DEA sehr wohl erinnern, das sind
seine endlich vielen Zustände, die er hat. Auch muss er nicht
wirklich rechnen können, es reicht, wenn er Reste berechnen
kann, und dass das möglich ist, wurde in der Vorlesung gezeigt.
Auch muss man sich nicht an bereits analysierte Ziffern erinnern,
man braucht nur die bis zu diesem Zeitpunkt berechnete Prüfziffer.

Nehmen wir, wir hätten ein Automaten, der Prüfziffern
berechnen kann. Dieser Automat befindet sich nach dem Lesen eines
Wortes in einem Zustand $z$, und zu jedem Zustand gehört eine
einzige korrekte Prüfziffer $p(z)$. Wenn wir daraus einen
Automaten bauen wollen, der Codes mit korrekter Prüfziffer
akzeptiert, dann müssen wir den bestehenden Automaten für
jeden Zustand um das Fragment
\[
\entrymodifiers={++[o][F]}
\xymatrix{
{z}\ar[r]^{p(z)}
        &*++[o][F=]{a}
}
\]
erweitern.
Diese bedeutet, dass das Wort akzeptiert wird, wenn in diesem
Zustand die zugehörige Prüfziffer $p(z)$ folgt. So entsteht
ein NEA.

Wir müssen jetzt also einen Automaten konstruieren, der die
Prüfziffer berechnen kann.
Ohne die Faktoren $3$ in der Prüfzifferformel
wäre dies einfach. Wir nehmen als Zustände die Prüfziffer,
die folgen müsste, wenn der Code an dieser Stelle fertig wäre.
Der Automat hat dann die Tabellendarstellung
\begin{center}
\begin{tabular}{|c|cccccccccc|}
\hline
&{\tt 0} &{\tt 1} &{\tt 2} &{\tt 3} &{\tt 4} &{\tt 5} &{\tt 6} &{\tt 7} &{\tt 8} &{\tt 9}\\
\hline
0&0&1&2&3&4&5&6&7&8&9\\
1&1&2&3&4&5&6&7&8&9&0\\
2&2&3&4&5&6&7&8&9&0&1\\
3&4&5&6&7&8&9&0&1&2&3\\
4&5&6&7&8&9&0&1&2&3&4\\
5&6&7&8&9&0&1&2&3&4&5\\
6&7&8&9&0&1&2&3&4&5&6\\
7&8&9&0&1&2&3&4&5&6&7\\
8&9&0&1&2&3&4&5&6&7&8\\
9&0&1&2&3&4&5&6&7&8&9\\
\hline
\end{tabular}
\end{center}
Würde stattdessen immer der Faktor $3$ berücksichtigt, wäre
die "Ubergangsfunktion
\begin{center}
\begin{tabular}{|c|cccccccccc|}
\hline
&{\tt 0} &{\tt 1} &{\tt 2} &{\tt 3} &{\tt 4} &{\tt 5} &{\tt 6} &{\tt 7} &{\tt 8} &{\tt 9}\\
\hline
0&0&3&6&9&2&5&8&1&4&7\\
1&1&4&7&0&3&6&9&2&5&8\\
2&2&5&8&1&4&7&0&3&6&9\\
3&3&6&9&2&5&8&1&4&7&0\\
4&4&7&0&3&6&9&2&5&8&1\\
5&5&8&1&4&7&0&3&6&9&2\\
6&6&9&2&5&8&1&4&7&0&3\\
7&7&0&3&6&9&2&5&8&1&4\\
8&8&1&4&7&0&3&6&9&2&5\\
9&9&2&5&8&1&4&7&0&3&6\\
\hline
\end{tabular}
\end{center}
Nun müssen aber beide "Ubergänge gemischt verwendet werden.
Bei den Ziffern mit gerader Nummer müssen wir die erste Tabelle
verwenden, bei der Ziffer mit ungerader Nummer die zweite Tabelle.
Wir müssen also unterscheiden können, ob wir gerade an einer Ziffer
mit gerader oder ungerader Nummer arbeiten. Wir tun dies, indem wir
die Zustande verdoppeln. Zu jedem Zustand $i$ gibt es jetzt noch
einen Zustand $i'$. Das Zeichen {\tt 0} führt dann den Zustand $i$
nicht mehr in den Zustand $i$, sondern $i'$. Und das Zeichen {\tt 1}
führt $i'$ nicht in $(i+3)'$, sondern in $i+3$ über. Der vollständige
Automat zur Berechnung der Prüfziffer ist also
\begin{center}
\begin{tabular}{|>{$}c<{$}|>{$}c<{$}>{$}c<{$}>{$}c<{$}>{$}c<{$}>{$}c<{$}>{$}c<{$}>{$}c<{$}>{$}c<{$}>{$}c<{$}>{$}c<{$}|}
\hline
&{\tt 0} &{\tt 1} &{\tt 2} &{\tt 3} &{\tt 4} &{\tt 5} &{\tt 6} &{\tt 7} &{\tt 8} &{\tt 9}\\
\hline
0&0'&1'&2'&3'&4'&5'&6'&7'&8'&9'\\
1&1'&2'&3'&4'&5'&6'&7'&8'&9'&0'\\
2&2'&3'&4'&5'&6'&7'&8'&9'&0'&1'\\
3&4'&5'&6'&7'&8'&9'&0'&1'&2'&3'\\
4&5'&6'&7'&8'&9'&0'&1'&2'&3'&4'\\
5&6'&7'&8'&9'&0'&1'&2'&3'&4'&5'\\
6&7'&8'&9'&0'&1'&2'&3'&4'&5'&6'\\
7&8'&9'&0'&1'&2'&3'&4'&5'&6'&7'\\
8&9'&0'&1'&2'&3'&4'&5'&6'&7'&8'\\
9&0'&1'&2'&3'&4'&5'&6'&7'&8'&9'\\
0'&0&1&2&3&4&5&6&7&8&9\\
1'&1&2&3&4&5&6&7&8&9&0\\
2'&2&3&4&5&6&7&8&9&0&1\\
3'&4&5&6&7&8&9&0&1&2&3\\
4'&5&6&7&8&9&0&1&2&3&4\\
5'&6&7&8&9&0&1&2&3&4&5\\
6'&7&8&9&0&1&2&3&4&5&6\\
7'&8&9&0&1&2&3&4&5&6&7\\
8'&9&0&1&2&3&4&5&6&7&8\\
9'&0&1&2&3&4&5&6&7&8&9\\
\hline
\end{tabular}
\end{center}
Es gibt also einen DEA, der die Prüfziffer berechnet. Damit können wir
wie zu Beginn festgestellt auch einen NEA zur "Uberprüfung der Prüfziffer
bauen, indem wir die "Ubergänge
\[
\entrymodifiers={++[o][F]}
\xymatrix{
{i}\ar[r]^{i}
        &*++[o][F=]{a}
                &*+\txt{und}
                        &{i'}\ar[r]^{i}
                                &*++[o][F=]{a}
}
\]
hinzufügen.

Vielfach wurde argumentiert, diese Sprache könne nie und nimmer
regulär sein, denn bei Prüfziffer würde ja beim Aufpumpen
kaputt gehen. Das muss nicht sein, aus Teilstück $y$ darf
einfach keinen Einfluss auf die Prüfziffer haben. Solche
Teilstücke hätten für sich allein genommen die Prüfziffer 0.
Dazu gehören zum Beispiel
{\tt 00},
{\tt 13},
{\tt 26},
{\tt 39},
{\tt 42},
{\tt 55},
{\tt 68},
{\tt 71},
{\tt 84}
und
{\tt 97}, wenn sie an einer geraden Stelle beginnen, oder
die gleichen Zeichenfolgen in umgekehrter Reihenfolge, wenn sie
an einer ungeraden Stelle beginnen.
Dazu gibt es noch viele Blöcke mit 4 Zeichen, zum Beispiel
{\tt 1114} oder {\tt 1230}. Es gibt also eine grosse Auswahl von
möglichen Strings, die man aufpumpen kann, und wenn ein Wort
lang genug ist (Pumping-Length), wird man sicher eines finden.

Dass dem so ist kann man sich auch so überlegen. Es gibt 10
verschiedene mögliche Prüfziffern. Nach jeder Ziffer des Codes müsste,
wenn der Code dort zu Ende wäre, eine dieser 10 möglichen Prüfziffern
folgen. Da es nur 10 mögliche Prüfziffern gibt, muss sie sich
wiederholen.
Das Wortstück zwischen der ersten Wiederholung führt hat für sich
genommen Prüfziffer 0 haben. Damit ist es aber natürlich noch
nicht aufpumpbar, es muss auch noch gerade Länge haben. Wenn man
aber ein Wort von mindestens 22 Zeichen Länge hat, dann gibt es
darin nach genau diesem Argument ein Teilstück gerader Länge,
welches für sich genommen die Prüfziffer 0 hat, also ein Teilstück,
welches aufgepumpt werden kann. Dieses Argument ist natürlich
sehr ähnlich wie der ursprüngliche Beweis des Pumping Lemmas.

Als Beispiel betrachten wir das Wort
\[
w=\text{\tt 10101010101010101010}
\]
Die Prüfziffern von Teilwörtern sind in folgender Tabelle
zusammengstellt:
\begin{center}
\begin{tabular}{|l|r|}
\hline
Teilwort&Prüfziffer\\
\hline
$\varepsilon$             &0\\
{\tt 1}                   &1\\
{\tt 10}                  &1\\
{\tt 101}                 &2\\
{\tt 1010}                &2\\
{\tt 10101}               &3\\
{\tt 101010}              &3\\
{\tt 1010101}             &4\\
{\tt 10101010}            &4\\
{\tt 101010101}           &5\\
{\tt 1010101010}          &5\\
{\tt 10101010101}         &6\\
{\tt 101010101010}        &6\\
{\tt 1010101010101}       &7\\
{\tt 10101010101010}      &7\\
{\tt 101010101010101}     &8\\
{\tt 1010101010101010}    &8\\
{\tt 10101010101010101}   &9\\
{\tt 101010101010101010}  &9\\
{\tt 1010101010101010101} &0\\
{\tt 10101010101010101010}&0\\
\hline
\end{tabular}
\end{center}
Daraus kann man ablesen, dass die ersten 20 Zeichen ein Teilstück
bilden, welches aufgepumpt werden kann. Jedes Wort, welches mit
$w$ beginnt und im übrigen eine korrekte Prüfziffer hat, kann
aufgeumpt werden, indem diese 20 Zeichen aufgepumpt werden.
Ein Beispiel eines solchen Wortes wäre
\[
\text{\tt 1010101010101010101012915}.
\qedhere
\]
\end{loesung}
