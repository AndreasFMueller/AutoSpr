Sei $\Sigma=\{\texttt{0},\texttt{1},\texttt{2}\}$.
Finden Sie einen deterministischen endlichen Automaten, der die
Sprache
\[
L
=
\{
w\in \Sigma^*
\mid
\text{$w$ ist eine ungerade Zahl im Dreiersystem}
\}
\]
akzeptiert.

\begin{loesung}
Es braucht zwei Zustände $q_0$ und $q_1$ für gerade und ungerade
Zahlen.
Es muss untersucht werden, wie die Parität (gerade/ungerade) einer
Zahl sich beim Anhängen einer Ziffer ändert.
Anfügen einer \texttt{0} entspricht der Multiplikation mit $3$,
dies ändert die Parität nicht.
Anfügen einer \texttt{1} entspricht der Multiplikation mit $3$
gefolgt von der Addition einer $1$, dies ändert die Parität.
Anfügen einer \texttt{2} dagegen ändert die Parität nicht.
Somit ergibt sich der endliche Automat
\begin{center}
\begin{tikzpicture}[>=latex,thick]
\coordinate (A) at (-2,0);
\coordinate (B) at (2,0);
\node at (A) {$q_0$};
\node at (B) {$q_1$};
\draw (A) circle[radius=0.5];
\draw (B) circle[radius=0.5];
\draw (B) circle[radius=0.45];
\draw[->,shorten >= 0.5cm] (-4,0) -- (A);
\draw[->,shorten >= 0.5cm,shorten <= 0.5cm] (A) to[out=30,in=150] (B);
\draw[->,shorten >= 0.5cm,shorten <= 0.5cm] (B) to[out=-150,in=-30] (A);
\draw[->,shorten >= 0.5cm,shorten <= 0.5cm] (A) to[out=60,in=120,distance=2cm] (A);
\draw[->,shorten >= 0.5cm,shorten <= 0.5cm] (B) to[out=60,in=120,distance=2cm] (B);
\node at (0,0.7) [above] {\texttt{1}};
\node at (0,-0.7) [below] {\texttt{1}};
\node at ($(A)+(0,1.4)$) [above] {\texttt{0},\texttt{2}};
\node at ($(B)+(0,1.4)$) [above] {\texttt{0},\texttt{2}};
\end{tikzpicture}
\end{center}
\end{loesung}
