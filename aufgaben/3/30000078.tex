Ein Formattierprogramm für Wordle-Lösungen nimmt eine Spezifikation
der Lösung als Zeichenketten entgegen, in der jeweils ein Buchstabe
gefolgt wird von einem Zeichen {\blank} falls, das Zeichen nicht 
vorkommt, \texttt{+} wenn es vorkommt und \texttt{*} wenn es am richtigen
Platz steht.
Am Ende jeder Zeile wird ein \texttt{|} platziert.
Die Zeilen werden unmittelbar hintereinander geschrieben, natürlich
müssen Sie alle die gleiche Länge haben, die aber auch verschieden
von den üblichen 5 Zeichen sein kann.
Bevor mit der Formattierung begonnen wird, soll mit einem regulären
Ausdruck geprüft werden, ob das Format stimmt.
Ist dies möglich?

\themaL{Pumping Lemma fur regulare Sprachen}{Pumping Lemma für reguläre Sprachen}
\themaL{regular}{regulär}

\begin{loesung}
Nein, denn der zugehörige endliche Automat müsste immer wieder ``nachzählen''
können, ob die Zeilen die richtige Länge haben.
Beweisen kann man es mit dem Pumping Lemma für reguläre Sprachen.
\begin{enumerate}
\item Die Sprache wird mit $L$ bezeichnet und wir nehmen an, dass sie
regulär ist.
\item
Das Pumping Lemma garantiert, dass es die Pumping Length $N$ gibt.
\item
Wir konstruieren das Wort
\begin{equation*}
w
=
\underbrace{\texttt{A\blank A\blank\dots A\blank}}_N\texttt{|}%
\underbrace{\texttt{A\blank A\blank\dots A\blank}}_N\texttt{|}
\end{equation*}
welches sicher in der Sprache $L$ liegt,
den es repräsentiert zwei Zeilen der Länge $N$ aus Zeichen \texttt{A},
die im gesuchten Wort nicht vorkommen.
\item
Nach dem Pumping-Lemma muss es eine Unterteilung geben, deren
aufpumpbarer Teil $y$ vollständig im ersten Block aus
``$\texttt{A\blank}$'' liegen muss.
\item
Beim Pumpen des Teils $y$ wird die Anzahl der \texttt{A} in diesem
ersten Block erhöht, die Anzahl im zweiten Block bleibt gleich, somit
liegt keine korrekte Spezifikation mehr vor, das gepumpte Wort ist
nicht mehr in $L$, im Widerspruch zur Aussage des Pumping Lemma.
\item
Dieser Widerspruch zeigt, dass $L$ nicht regulär sein kann.
\qedhere
\end{enumerate}
\end{loesung}

\begin{bewertung}
6 Schritte des Pumping-Lemma-Beweises:
Annahme ({\bf A}) 1 Punkt,
Pumping Length ({\bf N}) 1 Punkt,
Wort ({\bf W}) 1 Punkt,
Unterteilung ({\bf U}) 1 Punkt,
Widerspruch beim Pumpen ({\bf P}) 1 Punkt,
Folgerung ({\bf F}) 1 Punkt.
\end{bewertung}
