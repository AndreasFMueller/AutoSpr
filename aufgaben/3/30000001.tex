In einem endlichen Automaten
\[
A=(\{q_1,q_2,q_3,q_4,q_5\}, \{\text{\tt u},\text{\tt d}\}, \delta,
q_3, \{q_3\})
\]
ist die Funktion $\delta$ durch die folgende Tabelle definiert:
\begin{center}
\begin{tabular}{>{$}c<{$}|>{$}c<{$}>{$}c<{$}}
&\text{\tt u}&\text{\tt d}\\
\hline
q_1&q_1&q_2\\
q_2&q_1&q_3\\
q_3&q_2&q_4\\
q_4&q_3&q_5\\
q_5&q_4&q_5\\
\end{tabular}
\end{center}
\begin{teilaufgaben}
\item Zeichnen Sie das Zustandsdiagramm.
\item Akzeptiert der Automat des leere Wort $\varepsilon$?
\item Akzeptiert der Automat ein Wort der Länge $3$?
\item Bestimmen Sie alle Wörter der Länge $\le 4$, die
der Automat akzeptiert.
\item Wieviele verschiedene Wörter akzeptiert der Automat?
\end{teilaufgaben}

\thema{DEA}
\thema{Zustandsdiagramm}

\begin{loesung}
\begin{teilaufgaben}
\item
Das Zustandsdiagramm des Automaten ist
%\[
%\entrymodifiers={++[o][F]}
%\xymatrix @-1mm {
%*+\txt{}&{q_1}\ar@/^/[d]^{\tt d} \ar@(ul,ur)^{\tt u}
%\\
%*+\txt{}&{q_2}\ar@/^/[u]^{\tt u}\ar@/^/[d]^{\tt d}
%\\
%*+\txt{}\ar[r]&*++[o][F=]{q_3}\ar@/^/[u]^{\tt u}\ar@/^/[d]^{\tt d}
%\\
%*+\txt{}&{q_4}\ar@/^/[u]^{\tt u}\ar@/^/[d]^{\tt d}
%\\
%*+\txt{}&{q_5}\ar@/^/[u]^{\tt u} \ar@(dr,dl)^{\tt d}
%}
%\]
\begin{center}
\begin{tikzpicture}[>=latex,thick]
\def\l{1.5}
\coordinate (q1) at (0,{2*\l});
\coordinate (q2) at (0,{1*\l});
\coordinate (q3) at (0,{0*\l});
\coordinate (q4) at (0,{-1*\l});
\coordinate (q5) at (0,{-2*\l});
\draw (q1) circle[radius=0.35];
\draw (q2) circle[radius=0.35];
\draw (q3) circle[radius=0.35];
\draw (q3) circle[radius=0.30];
\draw (q4) circle[radius=0.35];
\draw (q5) circle[radius=0.35];
\node at (q1) {$q_1$};
\node at (q2) {$q_2$};
\node at (q3) {$q_3$};
\node at (q4) {$q_4$};
\node at (q5) {$q_5$};
\draw[->,shorten <= 0.35cm,shorten >= 0.35cm] (q5) to[out=120,in=-120] (q4);
\draw[->,shorten <= 0.35cm,shorten >= 0.35cm] (q4) to[out=120,in=-120] (q3);
\draw[->,shorten <= 0.35cm,shorten >= 0.35cm] (q3) to[out=120,in=-120] (q2);
\draw[->,shorten <= 0.35cm,shorten >= 0.35cm] (q2) to[out=120,in=-120] (q1);
\draw[->,shorten <= 0.35cm,shorten >= 0.35cm] (q1) to[out=-60,in=60] (q2);
\draw[->,shorten <= 0.35cm,shorten >= 0.35cm] (q2) to[out=-60,in=60] (q3);
\draw[->,shorten <= 0.35cm,shorten >= 0.35cm] (q3) to[out=-60,in=60] (q4);
\draw[->,shorten <= 0.35cm,shorten >= 0.35cm] (q4) to[out=-60,in=60] (q5);
\node at ($0.5*(q1)+0.5*(q2)+(-0.5,0)$) {\texttt{u}};
\node at ($0.5*(q2)+0.5*(q3)+(-0.5,0)$) {\texttt{u}};
\node at ($0.5*(q3)+0.5*(q4)+(-0.5,0)$) {\texttt{u}};
\node at ($0.5*(q4)+0.5*(q5)+(-0.5,0)$) {\texttt{u}};
\node at ($0.5*(q1)+0.5*(q2)+(0.5,0)$) {\texttt{d}};
\node at ($0.5*(q2)+0.5*(q3)+(0.5,0)$) {\texttt{d}};
\node at ($0.5*(q3)+0.5*(q4)+(0.5,0)$) {\texttt{d}};
\node at ($0.5*(q4)+0.5*(q5)+(0.5,0)$) {\texttt{d}};
\draw[->,shorten >= 0.35cm] (-\l,0) -- (q3);
\draw[->,shorten <= 0.35cm,shorten >= 0.35cm]
	(q1) to[out=60,in=120,distance=1.2cm] (q1);
\node at ($(q1)+(0,1.1)$) {\texttt{u}};
\draw[->,shorten <= 0.35cm,shorten >= 0.35cm]
	(q5) to[out=-60,in=-120,distance=1.2cm] (q5);
\node at ($(q5)+(0,-1.1)$) {\texttt{d}};
\end{tikzpicture}
\end{center}
\item
Das leere Wort überführt den Startzustand $q_3$ in den Zustand
$q_3\in \{q_3\}$, also einen Akzeptierzustand, somit ist
$\varepsilon\in L(A)$.
\item
Ein Wort der Länge $3$ wird von diesem Automaten nicht akzeptiert.
Man kann das zum Beispiel dadurch einsehen, dass man eine Liste
aller Wörter der Länge drei macht (es gibt nur 8 solche Wörter),
und die dann einzeln prüft.

Oder man sagt sich, dass ein Wort mit der Länge $3$ niemals die kleine
Schleife an den Zuständen $q_1$ oder $q_5$ nutzen kann, weil das Wort
zu kurz ist, um wieder $q_3$ zu erreichen. Die verbleibenden Pfeile,
die das Wort nutzen kann, ändern also mit jedem Zeichen den Rest
der Zustandsnummer bei Teilung durch zwei. Man kann das symbolisieren,
in dem man die $q_1$, $q_3$ und $q_5$ schwarz einfärbt, die anderen
weiss. Ein Zeichen lässt dann immer die Farbe wechseln. Mit drei Zeichen
landet man auf weiss, aber der einzige Akzeptierzustand ist schwarz,
also ist es nicht möglich, ein Wort der Länge drei zu akzeptieren.

Noch ein weiteres Argument verläuft wie folgt:
Zunächst halten wir fest, dass der Automat auch keine Wörter der
Länge $1$ akzeptiert, es gibt keine Pfeile, welche von $q_3$ nach
$q_3$ zurück führen. Ein Wort der Länge drei muss also mindestens
die Zustände $q_2$ oder $q_4$ erreichen. Eine Zustandsänderung von
dort zurück zu $q_3$ ergibt zwar ein akzeptiertes Wort der Länge
$2$, es ist aber nicht möglich, dieses zu einem akzeptierten Wort
der Länge $3$ zu ergänzen.
Bleibt also nur ein Wort, welches
sogar $q_1$ oder $q_5$ erreicht. Da es keine Pfeile von $q_1$
oder $q_5$ direkt zum einzigen Akzeptierzustand $q_3$ gibt,
braucht es also mindestens zwei zusätzliche Zeichen, um den
Akzeptierzustand $q_3$ zu erreichen. Somit muss ein solches
Wort mindestens Länge $4$ haben.
\item Wir müssen alle Wörter finden, die gleich viele {\tt u} wie {\tt d}
enthalten. Die Anzahl solcher Wörter kann man wie folgt bestimmmen.
Man muss auf die vier Stellen eines solchen Wortes zwei Buchstaben
{\tt u} verteilen, die anderen zwei Plätze werden dann mit {\tt d}
gefüllt. Das erste {\tt u} können wir auf Platz 1, 2 oder 3 setzen,
dann bleiben jeweils 3, 2 bzw.~1 Möglichkeiten für das zweite {\tt u},
insgesamt also $3+2+1=6$ Möglichkeiten. Die sechs Wörter sind:
{\tt udud}, {\tt dudu}, {\tt uddu}, {\tt duud}, {\tt uudd}, {\tt dduu}.
Dazu kommen jetzt noch die zwei Wörter der Länge $2$ und das leere Wort,
welches Länge $0$ hat. Insgesamt sind dies $9$ verschiedene Wörter
\[
\{
{\tt udud}, {\tt dudu}, {\tt uddu}, {\tt duud}, {\tt uudd}, {\tt dduu},
{\tt ud},{\tt du}, \varepsilon
\}
\]
\item Der Automat akzeptiert alle Wörter der Form
\[
\text{\tt uu}
\text{\tt u}^n
\text{\tt dd},
\]
für jedes $n\in \mathbb N$,
dies sind unendlich viele verschiedene Wörter. Alternativ kann
man auch die Wörter der Form $({\tt ud})^n$, also die Sprache
$\{({\tt ud})^n\,|\,n\ge 0\}$ anführen,
welche ebenfalls unendlich viele Wörter umfasst.
\qedhere
\end{teilaufgaben}
\end{loesung}

