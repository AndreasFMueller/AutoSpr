Sei $\Sigma=\{\texttt{a},\texttt{b},\dots,\texttt{z}\}$ das Alphabet bestehend
aus allen Kleinbuchstaben.
Die Sprache
\[
L=\{w\in\Sigma^*\;|\;\text{es gibt zwei verschiedene Buchstaben
$a,b\in\Sigma$ mit $|w|_a=|w|_b > 1$}\}
\]
besteht aus Wörtern, die mindestens zwei Buchstaben mehr als einmal
und in gleicher Anzahl enthalten.
Die Wörter
\[
\texttt{essen},\qquad
\texttt{rapperswil},\qquad
\texttt{seenachtfest}
\]
sind in $L$, nicht aber
\[
\texttt{trinken},\qquad
\texttt{pfaeffikon},\qquad
\texttt{montag}.
\]
Ist die Sprache $L$ regulär?

\thema{Pumping Lemma für reguläre Sprachen}
\thema{regulär}

\begin{loesung}
Nein, wie man mit dem Pumping-Lemma für reguläre Sprachen bewesein kann.
Wir nehmen dazu an, dass $L$ regulär sei, das Pumping Lemma für reguläre
Sprachen besagt dann, dass es eine Zahl $N$, die Pumping Length, gibt,
so dass Wörter grösserer Länge die Pumpeigenschaft haben.

Wir konstruieren daher ein Wort in $L$, welches diese Eigenschaft hat:
\[
w=\texttt{a}^N\texttt{b}^N
\]
Das Wort ist in $L$, wenn $N>1$ ist, was wir im Folgenden annehmen.
Das Pumping-Lemma besagt jetzt, dass es eine Zerlegung in drei 
Teile $w=xyz$ gibt, wobei $|xy|\le N$ und $|y|>1$ ist.
Insbesondere bestehen $x$ und $y$ ausschliesslich aus Buchstaben \texttt{a}.
Ein aufgepumptes Wort hat die Form
\[
w_k=xy^kz=\texttt{a}^{N+|y|(k-1)}\texttt{b}^N,
\]
d.~h.~für $k>1$ ist die Zahl der einzigen beiden vorkommenden Buchstaben
\texttt{a} und \texttt{b} nicht mehr gleich, also $w_k\not\in L$,
im Widerspruch zur Behauptung des Pumping-Lemmas.
Dieser Widerspruch zeigt, dass $L$ nicht regulär sein kann.
\end{loesung}

\begin{bewertung}
Pumping-Lemma, Annahme Sprache ist regulär ({\bf R}) 1 Punkt,
Pumping-Length ({\bf N}) 1 Punkt,
Konstruktion eines Wortes in $L$ unter Verwendung von $N$ ({\bf W}) 1 Punkt,
Unterteilung des Wortes ({\bf U}) 1 Punkt,
Nachweis, dass Pumpeigenschaft verletzt sein muss ({\bf P}) 1 Punkt,
Schlussfolgerung, dass $L$ nicht regulär sein kann ({\bf S}) 1 Punkt.
\end{bewertung}

