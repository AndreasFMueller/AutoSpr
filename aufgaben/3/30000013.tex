Ein CSV-File (Comma Separated Values) besteht aus Zeilen, die aus
mit Komma getrennten Feldern bestehen. Jedes Feld ist begrenzt mit
Anführungszeichen.
Die Felder selbst können auch Kommas und Anführungszeichen enthalten,
aber um die Anführungszeichen in einem Feld vom feldbegrenzenden
Anführungszeichen zu unterscheiden, müssen zwei Anführungszeichen
eingegeben werden. Enthält ein Feld weder Komma noch Anführungszeichen,
können die feldbegrenzenden Anführungszeichen weggelassen werden.
Hier ein Beispiel eines syntaktisch korrekten CSV-Files:
\verbatimainput{beispiel.csv}

Die syntaktisch korrekten Zeilen von CSV-Files bilden eine Sprache.
Ist diese Sprache regulär?

\begin{loesung}
Ja, sie ist regulär, denn man kann einen endlichen Automaten dafür
angeben.
Das folgende Zustandsdiagramm beschreibt einen VNEA, der korrekte
CSV-Zeilen akzeptiert.
\[
\entrymodifiers={++[o][F]}
\xymatrix{
*+\txt{}\ar[r]
        &*++[o][F=]{z_0}\ar@/_/[d]_{\text{\tt[\char94,\char34]}} \ar[r]^{\text{\tt \char34}}
            \ar@(ul,ur)^{\text{\tt,}}
                &{z_1}\ar@(ul,ur)^{\text{\tt [\char94{}\char34]}}
                    \ar@/_/[d]_{\text{\tt\char34}}
\\
*+\txt{}
        &*++[o][F=]{z_2}\ar@/_/[u]_{\text{\tt,}}
            \ar@(dl,dr)_{\text{\tt [\char94,\char34]}}
                &*++[o][F=]{z_3}\ar@/_/[u]_{\text{\tt\char34}}
                    \ar[ul]^{\text{\tt,}}
}
\]
Wenn sich der Automat im Zustand $z_0$ befindet, erwartet er als
nächstes ein Feld. Dieses kann leer sein, was durch in Komma ({\tt ,})
angezeigt wird. Es kann mit einem Feldbegrenzer {\tt\char34} beginnen,
was den Automaten in den Zustand $z_1$ bringt. In diesem Zustand
sind alle Zeichen akzeptabel, aber ein Anführungszeichen kann zwei
verschiedene Bedeutungen haben. Es kann gefolgt sein von noch einem
Anführungszeichen, in diesem Fall ist es ein eingebetetes Anführungszeichen.
Es kann aber auch gefolgt sein von einem Komma, in diesem Fall war
es ein Feldbegrenzer. Zu dieser Unterscheidung dient der Zustand $z_3$.

Ein Feld kann im Zustand $z_0$ auch mit einem beliebigen Zeichen
beginnen, das verschieden von Komma und Anführungszeichen ist.
Dazu dient der Zustand $z_2$. In diesem Zustand sind Anführungszeichen
nicht erlaubt, und ein Komma beendet das Feld, führt den Automaten
also wieder in den Zustand $z_0$.

Die Zustände $z_2$ und $z_3$ sind Akzeptierzustände, weil eine Zeile
nur dann noch nicht zu ende sein darf, wenn ein geöffnetes Anführungszeichen
noch nicht geschlossen worden ist. Dies ist genau im Zustand $z_1$ der
Fall, alle Zustände ausser $z_1$ können daher akzeptieren.

Natürlich ist es auch möglich, die Regularität dadurch nachzuweisen,
dass man einen regulären Ausdruck für die Zeilen angibt. Wir bauen
den regulären Ausdruck Schrittweise auf. Wenn $r_f$ ein regulärer Ausdruck
ist, auf den einzelne Felder einer CSV-Zeile passen, dann ist
\[
r_z=r_f\text{\tt (,}r_f\text{\tt )*}
\]
ein regulärer Ausdruck, auf den ganze CSV-Zeilen passen. Ein Feld kann
Feldbegrenzer haben, oder eben nicht, es ist also $r_f=r_b|r_u$, wobei
auf $r_b$ begrenzte Felder passen, und auf $r_u$ nur unbegrenzte.
Ungebegrenzte Felder dürfen keine Kommata oder Anführungszeichen
enthalten, also $r_u=\text{\tt [\char94,\char34]*}$. Begrenzte Felder
dürfen Kommata enthalten. Wenn sie Anführungszeichen enthalten,
dann jeweils in Zweiergruppen:
\[
r_b=\text{\tt
\char34([\char94\char34]|(\char34\char34))*\char34
}
\]
Damit kann man jetzt den regulären Ausdruck zusammensetzen:
\begin{align*}
r_u&=\text{\tt
[\char94,\char34]*
}\\
r_b&=\text{\tt
\char34([\char94\char34]|(\char34\char34))*\char34
}\\
r_f&=\text{\tt
[\char94,\char34]*|\char34([\char94\char34]|(\char34\char34))*\char34
}\\
r_z&=\text{\tt
[\char94,\char34]*|\char34([\char94\char34]|(\char34\char34))*\char34
(,%
[\char94,\char34]*|\char34([\char94\char34]|(\char34\char34))*\char34
)*
}
\qedhere
\end{align*}
\end{loesung}
