Sei $\Sigma$ ein Alphabet und $a\in\Sigma$. Ist die Sprache
\[
L=\{w\in\Sigma^*|\text{$w$ enthält an der drittletzten Stelle das Zeichen $a$}\}
\]
regulär? Falls ja, konstruieren Sie einen nichtdeterministischen
endlichen Automaten, der die Sprache erkennt.

\thema{regulär}
\thema{DEA}
\thema{Zustandsdiagramm}

\begin{loesung}
Die Wörter von $L$ sind genau diejenigen, die auf den
reguläre Ausdruck
$$r=\text{\tt .*a..}$$
passen. Daher ist
$L=L(r)$, somit ist $L$ eine reguläre Sprache.

Einen endlichen Automaten dazu kann man wie folgt konstruieren. Zunächst braucht
man einen Automaten, der beliebige Folgen von Zeichen akzeptiert. Das schafft
der nichtdeterministische Automat
\[
\entrymodifiers={++[o][F]}
\xymatrix @-1mm {
*+\txt{} \ar[r]
&x_0 \ar[r]^{\varepsilon} \ar@(dl,dr)_{.}
&*++[o][F=]{x_1}
}
\]
Nun muss das Wort aber mit einer Folge ganz bestimmter Länge enden,
also mit einem Wort, welches vom Automaten
\[
\entrymodifiers={++[o][F]}
\xymatrix @-1mm {
*+\txt{} \ar[r]
&x_1 \ar[r]^{a} 
&x_2 \ar[r]^{.} 
&x_3 \ar[r]^{.} 
&*++[o][F=]{x_4}
}
\]
Einen nichtdeterminisitischen Automaten kann durch Zusammenhängen
dieser beiden Automaten erreichen:
\[
\entrymodifiers={++[o][F]}
\xymatrix @-1mm {
*+\txt{} \ar[r]
&x_0 \ar[r]^{\varepsilon} \ar@(dl,dr)_{.}
&x_1 \ar[r]^{a} 
&x_2 \ar[r]^{.} 
&x_3 \ar[r]^{.} 
&*++[o][F=]{x_4}
}
\]
Noch etwas sparsamer, aber äquivalent ist der Automat
\[
\entrymodifiers={++[o][F]}
\xymatrix @-1mm {
*+\txt{} \ar[r]
&x_0 \ar[r]^{a} \ar@(dl,dr)_{.}
&x_2 \ar[r]^{.} 
&x_3 \ar[r]^{.} 
&*++[o][F=]{x_4}
}
\]
in dem man den $\varepsilon$-"Ubergang einspart, aber dafür im
Zustand nichtdeterministisch einen "Ubergang zu $x_0$ oder $x_2$
wählen muss, wenn das Zeichen $a$ gelesen wird.
\end{loesung}
