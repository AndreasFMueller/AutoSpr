In einem Textbearbeitungssystem werden Fussnoten durch zwei spezielle
Zeichen codiert: das eine Zeichen, symbolisch als \texttt{L} dargestellt,
zeigt wo im Text auf die Fussnote verwiesen werden soll, während das
andere Zeichen, symbolisch als $\texttt{A}$ dargestellt, den eigentlichen
Fussnoteninhalt markiert.
Dabei soll es möglich sein, Fussnoten nicht nur am Fuss einer Seite, sondern
auf Wunsch auch am Ende eines Kapitels oder sonst irgendwo im Text zu
platzieren.
Ist es möglich, mit einem regulären Ausdruck zu testen, ob zu jedem 
\texttt{L} auch genau ein \texttt{A} vorhanden ist?


\begin{loesung}
Nein, denn dies würde bedeuten, dass die Sprache
\[
L = \{w\in\Sigma^*\;|\; |w|_{\texttt{L}} = |w|_{\texttt{A}}\}
\]
regulär ist, was man mit dem Pumping Lemma widerlegen kann.
\begin{enumerate}
\item Annahme: $L$ ist regulär
\item Nach dem Pumping Lemma gibt es zu $L$ die Pumping Length $N$
\item Wähle das Wort $w=\texttt{L}^N\texttt{A}^N$.
\item Nach dem Pumping Lemma lässt sich das Wort $w$ aufteilen in drei Teile
$w=xyz$ mit den Eigenschaften $|xy|\le N$ und $|y|\ge 1$ derart dass
alle aufgepumpten Versionen $xy^kz\in L$.
\item Der Teil $y$ enthält nur die Zeichen \texttt{L}, beim Pumpen
ändert also die Anzahl dieser Zeichen, nicht aber die Anzahl der Zeichen
\texttt{A}.
Insbesondere sind die gepumpten Wörter nicht mehr in $L$.
\item Der Widerspruch zeigt, dass die Annahme, $L$ sei regulär, nicht
haltbar ist, $L$ ist also nicht regulär.
\qedhere
\end{enumerate}
\end{loesung}

\begin{bewertung}
Verwendung des Pumping Lemmas, Annahme $L$ regulär ({\bf PL}) 1 Punkt,
Puming Length ({\bf N}) 1 Punkt,
Wahl eines Wortes ({\bf W}) 1 Punkt,
Aufteilung des Wortes ({\bf A}) 1 Punkt,
Widerspruch beim Pumpen ({\bf P}) 1 Punkt,
Folgerung $L$ regulär ({\bf R}) 1 Punkt.
\end{bewertung}
