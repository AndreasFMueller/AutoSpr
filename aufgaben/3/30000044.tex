Das Programm {\em ImageMagick} ermöglicht Bildverarbeitung mit
Kommandozeilenbefehlen. Viele Optionen dieses Programms müssen die
Geometrie eines Bildes oder Bildausschnittes spezifizieren können.
Dazu wurde eine Notation entwickelt, die in der Dokumentation etwa
wie folgt zusammengefassst ist:
\begin{center}
\begin{tabular}{>{\tt}lp{5.75in}}
\rm geometry&General description\\
\hline
50\%&Height and width both scaled by specified (50) percentage.\\
47\%x42\%&Height (47\%) and width (42\%) individually scaled by specified
percentages. (Only one \% symbol needed.)\\
47&Width 47 given, height automagically selected to preserve aspect ratio.\\
x42&Height 42 given, width automagically selected to preserve aspect ratio.\\
47x42&Maximum values of height (42) and width (47) given, aspect ratio
preserved.\\
47x42\^&Minimum values of width (47) and height (42) given, aspect ratio
preserved.\\
47x42!&Width (47) and height (42) emphatically given, original
aspect ratio ignored.\\
47x42>&Shrinks an image with dimension(s) larger than the
corresponding width (47) and/or height (42) argument(s).\\
47x42<&Enlarges an image with dimension(s) smaller than the
corresponding width (47) and/or height (42) argument(s).\\
1974@&Resize image to have specified area (1974) in pixels. Aspect ratio is
preserved.\\
\hline
\end{tabular}
\end{center}
Ausserdem kann einer solchen Grössenspezifikation auch noch ein Offset
angehängt werden, der immer von der Form
\texttt{+18-48}
sein muss. Die erste Zahl gibt den Versatz in $x$-Richtung an, die zweite
den Versatz in $y$-Richtung. In beiden Richtungen sind beide Vorzeichen 
\texttt{+} und \texttt{-} möglich.

Finden Sie einen regulären Ausdruck, der genau alle möglichen
Geometrie-Spezifikationen akzeptiert.

\thema{reguläre Ausdrücke}
\thema{regulär}

\begin{loesung}
Ein solcher regulärer Ausdruck kann zum Beispiel in einem Programm dazu
verwendet werden, um mit Hilfe eines Pattern-Matchings die verschiedenen
Fälle zu unterscheiden, die Parameter herauszulesen, und die entsprechend
Aktion einzuleiten.
Dazu ist notwendig, dass der reguläre Ausdruck so aufgebaut wird, dass
er die möglichen Fälle zu unterscheiden erlaubt.
Die folgende Lösung strebt genau dies an.

Es gibt offenbar vier Arten von Geometriespezifikationen: Skalierungen 
(erste zwei Zeilen),
partielle Spezifikationen (Zeilen 3 und 4), vollständige Spezifikationen
und Flächenspezifikationen (letzte Zeile).
Für jede kann man einen regulären Ausdruck angeben:
\begin{align*}
\text{Skalierung:}\qquad&r_1 = \texttt{\color{red}[0-9]+\%|[0-9]+\%?x[0-9]+\%|[0-9]\%+x[0-9]+\%?}
\\
\text{partiell:}\qquad&r_2 = \texttt{\color{green}[0-9]+|x[0-9]+}
\\
\text{vollständig:}\qquad&r_3 = \texttt{\color{blue}[0-9]+x[0-9]+[!><\^{}]?}
\\
\text{Fläche:}\qquad&r_4 = \texttt{\color{yellow}[0-9]+@}
\end{align*}
Man beachte, dass in der Liste der angehängten Sonderzeichen bei den
vollständigen Geometriespezifikation das \texttt{\^{}} ganz am Ende
steht, weil ein \texttt{\^{}} am Anfang einer Zeichenklasse die
Negation bedeutet.


Jedem dieser Ausdrücke kann ausserdem eine Offset-Spezifikation angehängt
werden, die auf den regulären Ausdruck
\begin{center}
\texttt{[-+][0-9]+[-+][0-9]+}
\end{center}
passen muss. Zusammen ergibt das den regulären Ausdruck
\[
\texttt{(}
{\color{red} r_1}
\texttt{|}
{\color{green} r_2}
\texttt{|}
{\color{blue} r_3}
\texttt{|}
{\color{yellow} r_4}
\texttt{)([-+][0-9]+[-+][0-9]+)?}
\]
oder ausgeschrieben
\begin{center}
\texttt{(({\color{red}[0-9]+\%|[0-9]+\%?x[0-9]+\%|[0-9]\%+x[0-9]+\%?})|}%
\texttt{({\color{green}[0-9]+|x[0-9]+})|}\\
\texttt{({\color{blue}[0-9]+x[0-9]+[{}!><\^{}]?}))}%
\texttt{({\color{yellow}[0-9]+@})?}%
\texttt{)([-+][0-9]+[-+][0-9]+)?}
\end{center}
(der reguläre Ausdruck ist aus Platzgründen auf zwei Zeilen aufgeteilt,
beide Teile sind als ein regulärer Ausdruck ohne Zeilenumbruch zu lesen).

Man kann einen regulären Ausdruck auch noch viel direkter bekommen, ohne
die Fälle in der Tabelle auf Gemeinsamkeiten zu untersuchen.
Man schreibt einfach für jeden Ausdruck in der Tabelle einen regulären
Ausdruck hin, und verkettet diese alle, also
\begin{center}
\texttt{([0-9]+\%)|}
\texttt{([0-9]+\%x[0-9]+)|}
\texttt{([0-9]+x[0-9]+\%)|}
\texttt{([0-9]+\%x[0-9]+\%)|}
\\
\texttt{([0-9]+)|}
\\
\texttt{(x[0-9]+)|}
\\
\texttt{([0-9]+x[0-9]+)|}
\\
\texttt{([0-9]+x[0-9]+\^{})|}
\\
\texttt{([0-9]+x[0-9]+!)|}
\\
\texttt{([0-9]+x[0-9]+>)|}
\\
\texttt{([0-9]+x[0-9]+<)|}
\\
\texttt{([0-9]+@)}
\end{center}
(eine Zeile für jede Spezifikation in der Tabelle).
Daran hängt man dann wieder wie oben die Offset\-spezifikation an.
\end{loesung}

\begin{diskussion}
Die ursprüngliche Formulierung konnte so missverstanden werden, dass auch
der Ausdruck \texttt{.*} eine korrekte Antwort gewesen wäre.
Es ist aus dem Zusammenhang der Aufgabe klar, dass dies nicht gemeint war.
\end{diskussion}

\begin{diskussion}
Der Ausdruck $r_2$ könnte auch als \texttt{x?[0-9]+} geschrieben werden,
doch ist die angegebene Form besser.
Typische Pattern-Matcher können nach dem Match auch einzelne markierte
Teile einer Zeichenkette angeben.
In der Standard-Regex-Library kann man zum Beispiel mit Klammern solche
Teile markieren.
Schreibt man $r_2$ dann in der Form \texttt{([0-9]+)|x([0-9]+)}
dann welche kann man am Resultat des Matchings ablesen, welcher der
Teile vorliegt, und kann mit nur einem Match zwischen einer Breitenangabe
und einer Höhenangabe unterscheiden. Das folgende Programm illustriert dies:
{\small
\verbatimainput{part.c.expanded}
}
\end{diskussion}

\begin{bewertung}
Skalierung wiedergegeben ({\bf S}) 1 Punkt,
Partielle Grössenangabe ({\bf P}) 1 Punkt,
vollständige Grössenangabe ({\bf V}) 1 Punkt,
Flächeangabe ({\bf F}) 1 Punkt,
Offset ({\bf O}) 1 Punkt,
Komposition ({\bf K}) 1 Punkt.
\end{bewertung}
