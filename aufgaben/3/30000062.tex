Datums-Angaben können in verschiedenen Formaten erfolgen.
Im deutschen Sprachraum ist die Angabe im Format \texttt{DD. MM. YYYY}
üblich, wobei Tageszahl und Monatszahl auch nur einstellig sein können,
in Datenbanken dagegen eher das ISO-Datum in der Form \texttt{YYYY-MM-DD},
in dem Tageszahl und Monatszahl bei Bedarf eine führende Null haben.
Datumsangaben in offiziellen japanischen Dokumenten beziehen sich dagegen
auf das Regierungsjahr des aktuellen Tenn\=o.
Akihito ist seit 7. Januar 1989 Tenn\=o, der 4.~Februar 2019 fällt daher in
das 31.~Regierungsjahr von Akihito.
Die zugehörige japanische Jahresangabe ist Heisei 31, das Datum ist
\begin{CJK}{UTF8}{min}平成31年2月18日.\end{CJK}
Darin steht 
\begin{CJK}{UTF8}{min}平成\end{CJK} für {\em Heisei}, den Namen der Ära von
Tenn\=o Akihito,
\begin{CJK}{UTF8}{min}年\end{CJK}, gelesen {\em nen}, heisst Jahr,
\begin{CJK}{UTF8}{min}月\end{CJK}, gelesen {\em gatsu}, heisst Monat
und
\begin{CJK}{UTF8}{min}日\end{CJK}, gelesen {\em nichi}, heisst Tag.
Die Angabe der Ära verwendet immer zwei Zeichen.
Japanische Datumsangaben können aber auch internationale Jahreszahlen
verwenden wie in 
\begin{CJK}{UTF8}{min}2018年2月18日.\end{CJK}

\begin{teilaufgaben}
\item Formulieren Sie einen regulären Ausdruck für europäische Datumsangaben,
der nur gültige Monatszahlen akzeptiert.
\item Formulieren Sie einen regulären Ausdruck für japanische Datumsangaben in
beliebigen Ären, der nur gültige Monatsszahlen akzeptiert.
\item Formulieren Sie einen regulären Ausdruck, der jede Form von Datum
akzeptiert.
\end{teilaufgaben}

\begin{hinweis}
Ihre regulären Ausdrücke sollen nur Tageszahlen 1--31 akzeptieren, aber
es ist nicht nötig, dass nur die Tageszahlen des jeweiligen Monats akzeptiert
werden.
\end{hinweis}

\themaL{regulare Ausdrucke}{reguläre Ausdrücke}
\themaL{regular}{regulär}

\begin{loesung}
In beiden Teilaufgaben a) und b) dürfen nur Monatszahlen von 1 bis 12
akzeptiert werden, dies erreichen die regulären Ausdrücke
\begin{align*}
m &= \texttt{[1-9]|1[0-2]}\\
m_0 &= \texttt{0[1-9]|1[0-2]}
\end{align*}
Der Index $0$ gibt an, dass führende Nullen zu verwenden sind.
Tageszahlen sind von der Form
\begin{align*}
d &= \texttt{[1-9]|[12][0-9]|3[01]}\\
d_0 &= \texttt{[012][0-9]|3[01]},
\end{align*}
internationale Jahreszahlen sind vierstellige ganze Zahlen der Form
$y = \texttt{[12][0-9]\{3\}}$.
Diese regulären Ausdrücke werden nun verwendet, um die Teilaufgaben
zu lösen.
\begin{teilaufgaben}
\item
Europäische Daten werden von dem regulären Ausdruck
\[
r_e =
d\texttt{\\. }m\texttt{\\. }y\texttt{|}y\texttt{-}m_0\texttt{-}d_0
\]
\item
Japanische Jahreszahlen sind von der Form
\[
r_j
=
\texttt{(..[1-9][0-9]*|}y\texttt{)}
\text{\begin{CJK}{UTF8}{min}年\end{CJK}}
m
\text{\begin{CJK}{UTF8}{min}月\end{CJK}}
d
\text{\begin{CJK}{UTF8}{min}日\end{CJK}}
\]
\item
Eine beliebige Datumsangabe ist von der Form
\[
r = r_e\texttt{|} r_j
\qedhere
\]
\end{teilaufgaben}
\end{loesung}

\begin{bewertung}
Ausdruck für Tageszahlen 1--31 ({\bf D}) 1 Punkt,
Ausdruck für Monatszahlen mit ({\bf N}) und ohne ({\bf M})
führende Nullen 1 Punkt,
Lösung für jede Teilaufgaben ({\bf A}, {\bf B}, {\bf C}) je ein Punkt.
\end{bewertung}


