Für eine Anwendung, die als Antwort auf äussere Reize (Events)
nach programmierbaren Zeitintervallen (Delays) gewisse Aktionen (Actions)
auslöst, wurde folgende einfache Programmiersprache entworfen. Ein
Programm ist eine Liste von ``Befehlswörtern'', die Events, Delays und Actions
codieren. Letztere werden
jeweils durch die Buchstaben {\tt E}, {\tt D} und {\tt A} codiert.
Parameter werden daran angehängt. Events oder Actions haben als Parameter
einen Kleinbuchstaben gefolgt von einer Ziffernfolge. Delays haben als
Parameter eine Ziffernfolge ohne führende Nullen und maximal fünf
Zeichen. Die einzelnen Befehlswörter dürfen zur besseren Lesbarkeit mit Zeilenumbrüchen
(Newline, Zeichen \verb+\n+) getrennt werden. Auf jeden Event muss mindestens
ein Delay folgen, auf jedes Delay eine Action. Auf ein Event darf keine Action folgen.
\begin{teilaufgaben}
\item Ist diese Sprache regulär?
\item Falls ja: Beschreiben Sie einen endlichen Automaten, der die
Sprache akzeptiert.
\end{teilaufgaben}

\thema{regulär}
\thema{reguläre Ausdrücke}
\thema{DEA}

\begin{loesung}
\begin{teilaufgaben}
\item
Die Sprache ist regulär, denn man kann einen regulären Ausdruck
angeben, der genau die Wörter dieser Sprache akzeptiert. Um diesen
regulären Ausdruck aufzubauen kann man zunächst Ausdrücke für
die einzelnen Events, Delays und  Actions konstruieren:
\begin{center}
\begin{tabular}{|c|c|}
\hline
Event&{\tt E[a-z][0-9]*($|\backslash$n)}\\
Delay&{\tt D[1-9][0-9]*($|\backslash$n)}\\
Action&{\tt A[a-z][0-9]*($|\backslash$n)}\\
\hline
\end{tabular}
\end{center}
Ein Event ist immer gefolgt von mindestens einer Gruppe aus
einem Delay gefolgt von mindestens einer Action:
\begin{center}
Event
        Delay Action Action{\tt *}
        {\tt (} Delay Action Action{\tt *} {\tt )*}
\end{center}
Solche Events kann es jetzt beliebig viele geben, also
\begin{center}
{\tt (}Event
        Delay Action Action{\tt *}
        {\tt (} Delay Action Action{\tt *} {\tt )*}{\tt )*}
\end{center}
Setzt man jetzt die Ausdrücke aus der Tabelle in, erhält man
den endgültigen regulären Ausdruck (ist alles als ein einziger
Ausdruck zu lesen):
\begin{center}
{\tt (}
{\tt E[a-z][0-9]*($|\backslash$n)}
{\tt (D[1-9][0-9]*($|\backslash$n))}
{\tt (A[a-z][0-9]*($|\backslash$n))}
{\tt (A[a-z][0-9]*($|\backslash$n))}
{\tt *}
{\tt (}
{\tt (D[1-9][0-9]*($|\backslash$n))}
{\tt (A[a-z][0-9]*($|\backslash$n))}
{\tt (A[a-z][0-9]*($|\backslash$n))}
{\tt *}
{\tt )*}{\tt )*}
\end{center}
\item
Einen endlichen Automat kann dadurch erhalten, dass man obigen
regulären Ausdruck vom Standardalgorithmus in einen endlichen
Automaten umwandeln lässt. Man kann den endlichen Automaten
aber auch explizit angeben. Zunächst kann man einzelne
Teilautomaten $E$, $D$ und $A$ konstruieren, die jeweils
ein Event-Wort, eine Delay-Wort oder ein Action-Wort akzeptieren:
\[
\entrymodifiers={++[o][F]}
\xymatrix @-1mm {
*+\txt{} \ar[r]
        &z_{10}\ar[r]^{\tt E}
                &z_{11}\ar[r]^{\tt [a-z]}
                        &z_{12}\ar[r]^{\tt [0-9]}
                                &*++[o][F=]{z_{13}} \ar@(ur,dr)^{\tt [0-9]}
\\
*+\txt{} \ar[r]
        &z_{20}\ar[r]^{\tt D}
                &z_{21}\ar[r]^{\tt [1-9]}
                        &*++[o][F=]{z_{22}} \ar@(ur,dr)^{\tt [0-9]}
\\
*+\txt{} \ar[r]
        &z_{30}\ar[r]^{\tt E}
                &z_{31}\ar[r]^{\tt [a-z]}
                        &z_{32}\ar[r]^{\tt [0-9]}
                                &*++[o][F=]{z_{33}} \ar@(ur,dr)^{\tt [0-9]}
}
\]
Diese Teilautomaten können wir jetzt zu einem neuen Automaten
zusammensetzen, der "Ubersichtlichkeit halber schreiben wir jeweils
nur $E$, $D$ und $E$ für die Teilautomaten.
\[
\entrymodifiers={++[o][F]}
\xymatrix @-1mm {
*+\txt{} \ar[r]
        &E \ar[r]^{\tt \backslash n} \ar[dr]^{\tt D}
                &z_1 \ar[d]^{\tt D}
                        &*+\txt{}
\\
*+\txt{}
        &*+\txt{}
                &D \ar[r]^{\tt \backslash n} \ar@/^/[dr]^{\tt A}
                        &z_2 \ar[d]^{\tt A}
\\
*+\txt{}
        &z_3 \ar[uu]^{\tt E} \ar[ur]^{\tt D}
                &*+\txt{}
                        &A      \ar@(ur,dr)^{\tt A}
                                \ar@/^/[lu]^/1em/{\tt D}
                                \ar@/^1.7pc/[lluu]^/2em/{\tt E}
                                \ar[ll]^{\tt \backslash n}
}
\]
Dabei verbinden die Pfeile jeweils die Akzeptierzustände der Teilautomaten
mit den Startzusänden.
\qedhere
\end{teilaufgaben}
\end{loesung}


