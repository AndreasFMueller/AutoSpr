Ist die Sprache der Wörter, die mehr Konsonanten als Vokale enthalten,
regulär?

\themaL{Pumping Lemma fur regulare Sprachen}{Pumping Lemma für reguläre Sprachen}
\themaL{regulär}{regulär}

\begin{loesung}
Nein. Wäre das so, dann müsste es sogar so sein, wenn es nur einen
einzigen Konsonanten $k$ und einen Vokal $v$ gäbe.
Dann müsste also die Sprache $L=\{ w\in\Sigma^*\,|\, |w|_k > |v|_k\}$
regulär sein. Wir zeigen mit dem Pumping-Lemma, dass dies nicht möglich
ist.

Sei $N$ die Pumping Length der als regulär angenommenen Sprache $L$.
Dann bilden wir das Wort $w=v^{N}k^{N+1}$. Es ist offenbar $w\in L$.
Nach dem Pumping Lemma gibt es also eine Zerlegung $w=xyz$ mit
$|xy|\le N$ und $|y|>0$ so, dass auch $xy^nz\in L$ ist.
Aber $y$ kann nur aus $v$ bestehen, zum Beispiel $x=v^a$, $y=v^b$ und
$z=v^ck^{N+1}$. Dann ist
\[
xy^nz=v^{a+bn+c}k^{N+1}
\]
Da aber $a+b+c=N$ ist, gilt
\[
a+bn+c=a+b+c + b(n-1)=N+b(n-1).
\]
Ausserdem ist $b\ge 1$.
Durch Aufpumpen mit $n> 2$ entsteht also ein Wort, welches
$N+b(n-1)> N+1$ erfüllt, d.~h.~ein Wort mit mehr Vokalen als
Konsonanten, $xy^nz\not\in L$. Nach dem Pumping-Lemma müsste das
aufgepumpte Wort aber immer noch in $L$ sein. Dieser Widerspruch
zeigt, dass $L$ nicht regulär sein kann.

Wählt man das Wort $k^Nv^{N-1}$, kann man argumentieren, dass
beim Abpumpen von $y$, welches nur aus Konsonanten besteht, die
Zahl der Konsonanten kleiner wird, so dass die Bedingung, dass ein
Wort mehr Konsonanten als Vokale haben muss, nicht mehr erfüllt ist.

Der Nachweis lässt sich aber auch mit dem Satz von Myhill-Nerode
führen. Dazu sind die Mengen $L(w)$ zu berechnen, man findet:
\begin{center}
\begin{tabular}{|c|l|}
\hline
$w$&$L(w)$\\
\hline
$|w|_v=0$&Mindestens 1 Konsonant mehr als Vokale\\
$|w|_v=1$&Mindestens 2 Konsonanten mehr als Vokale\\
$|w|_v=2$&Mindestens 3 Konsonanten mehr als Vokale\\
$|w|_v=3$&Mindestens 4 Konsonanten mehr als Vokale\\
$|w|_v=4$&Mindestens 5 Konsonanten mehr als Vokale\\
$\dots$&$\dots$\\
$|w|_v=x$&Mindestens $x+1$ Konsonanten mehr als Vokale\\
\hline
\end{tabular}
\end{center}
Da alle Mengen $L(w)$ verschieden sind, würde ein DEA für die Sprache
unendlich viele verschiedene Zustände brauchen, was für einen
DEA nicht zulässig ist.
\end{loesung}

\begin{bewertung}
Pumping Lemma ({\bf P}) 1 Punkt,
Pumping Length ({\bf N}) 1 Punkt,
Konstruktion eines zweckmässigen Wortes ({\bf W}) 1 Punkt,
Zerlegung in $xyz$ ({\bf Z}) 1 Punkt,
Nachweis, dass Aufpumpen oder Abpumpen für dieses Wort aus der Sprache
herausführt
({\bf A}) 1 Punkt,
Schlussfolgerung ({\bf R}) 1 Punkt.
\end{bewertung}
