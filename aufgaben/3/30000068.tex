\begin{teilaufgaben}
\item Was für Strings passen auf den regulären Ausdruck
\begin{verbatim}
([0-9]{1,2})/([0-9]{1,2})/([0-9]{4})
\end{verbatim}
\item
Der reguläre Ausdruck \texttt{(.*\textbackslash.)[\textasciicircum.]+} wird
für das Editieren von Filenamen verwendet, indem der auf den Klammerausdruck
passende Teil des Filenamens behalten wird und dafür die Zeichenkette
\texttt{log} angehängt wird.
Was wird bei diesem Prozess aus dem Filenamen \texttt{pruefung.2020.tex}?
\item
In der Konfiguration eines Apache-Webservers findet man die
folgende Direktive
\begin{verbatim}
RewriteCond %{HTTP_USER_AGENT} Zeus.*?Webster
\end{verbatim}
mit dem der Webmaster die Zugriffsrechte eines bestimmten Useragenten
einschränkten wollte.
Die Direktive bezweckt, Benutzer umzuleiten, deren Browser auf den als
letztes Argument angegebenen regulären Ausdruck passt.
Warum ist das Fragezeichen in diesem regulären Ausdruck unnötig?
\item
Geben Sie einen regulären Ausdruck an, der auf alle Wörter mit genau
10 Zeichen passt und den String \texttt{AutoSpr} enthält.
\end{teilaufgaben}

\begin{loesung}
\begin{teilaufgaben}
\item Die passenden Strings sind Datumsangaben, drei Zahlen getrennt durch
\texttt{/}.
Die letzte Zahl ist die Jahreszahl.
\item Der Klammerausdruck passt auf alles bis zum letzten \texttt{.},
den Punkt eingeschlossen.
Vom vorgegeben Filenamen ist dies der Teil \texttt{pruefung.2020.}.
Daran wird jetzt \texttt{log} angehängt, so dass sich
\texttt{pruefung.2020.log} ergibt.
\item Das Fragezeichen macht den Teil \texttt{.*} optional, aber dieser
Teil kann durchaus das leere Wort sein.
Der Webmaster hat offenbar nicht daran gedacht, dass die *-Operation
auch das leere Wort beinhaltet.
\item Da \texttt{AutoSpr} bereits 7 Zeichen enthält, gibt es genau vier
Möglichkeiten, ein Wort mit 10 Zeichen zu bilden, welches den String
enthält:
\[
\texttt{AutoSpr...|.AutoSpr..|..AutoSpr.|...AutoSpr}
\qedhere
\]
\end{teilaufgaben}
\end{loesung}

\begin{bewertung}
\begin{teilaufgaben}
\item ({\bf A}) 1 Punkt
\item ({\bf B}) 1 Punkt
\item Bedeutung on \texttt{*} 1 Punkt {(\bf S)},
Bedeutung von \texttt{?} ({\bf F}) 1 Punkt.
\item Alternative ({\bf V}) 1 Punkt, verschiedene Verkettungen ({\bf K}) 1 Punkt
\end{teilaufgaben}
\end{bewertung}


