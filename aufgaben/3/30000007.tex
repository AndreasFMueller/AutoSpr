Konstruieren Sie für jede der folgenden Sprachen über dem Alphabet
$\Sigma=\{{\tt 0},{\tt 1}\}$ einen NEA. Reduzieren Sie diesen für die
Teilaufgaben a)-g) auf einen minimalen DEA.
\begin{teilaufgaben}
\item $L_1=\{w\in\Sigma^*\,|\,\text{$w$ enthält mindestens zwei {\tt 1} nacheinander}\}$
\item $L_2=\{w\in\Sigma^*\,|\,\text{$w$ ist eine durch drei teilbare Binärzahl}\}$
\item $L_3=L_1\cap L_2$
\item $L_4=L_1\setminus L_2$
\item $L_5=L_1\cup L_2$
\item $L_6=L_1L_2$
\item $L_7=L_1^*$
\end{teilaufgaben}

\begin{loesung}
\begin{teilaufgaben}
\item Ein einfacher NEA für $L_1$ ist
\[
\entrymodifiers={++[o][F]}
\xymatrix{
*\txt{}\ar[r]
        &{q_0} \ar[r]^{\tt 1} \ar@(dr,dl)^{{\tt 0},{\tt 1}}
                &{q_1} \ar[r]^{\tt 1}
                        &*++[o][F=]{q_2}\ar@(ur,dr)^{{\tt 0},{\tt 1}}
}
\]
Diesen NEA können wir jetzt mit dem Standardalgorithmus in einen
DEA umwandeln:
\[
\entrymodifiers={++[o][F]}
\xymatrix{
*+\txt{}\ar[r]
        &{q_{001}} \ar@(ur,ul)_{\tt 0} \ar@/^/[rdd]^{\tt 1}
                &*++[o][F=]{q_{110}}\ar[ddl]_{{\tt 0},{\tt 1}}
\\
q_{000}\ar@(ul,dl)_{{\tt 0},{\tt 1}}
        &{q_{010}} \ar[l]^{\tt 0} \ar[d]^{\tt 1}
                &*++[o][F=]{q_{101}} \ar@(u,ur)^{\tt 0} \ar@/^/[r]^{\tt 1}
                        &*++[o][F=]{q_{111}} \ar@/^/[l]^{\tt 0}
                                \ar@(ur,dr)^{\tt 1}
\\
*+\txt{}
        &*++[o][F=]{q_{100}} \ar@(dl,dr)_{{\tt 0},{\tt 1}}
                &q_{011} \ar@/^/[uul]^{\tt 0} \ar[ur]_{\tt 1}
}
\]
Eliminieren wir daraus die nicht erreichbaren Zustände, bleibt der
DEA
\[
\entrymodifiers={++[o][F]}
\xymatrix{
*+\txt{}\ar[r]
        &{q_{001}} \ar@(ur,ul)_{\tt 0} \ar@/^/[rdd]^{\tt 1}
\\
*+\txt{}
        &*+\txt{}
                &*++[o][F=]{q_{101}} \ar@(u,ur)^{\tt 0} \ar@/^/[r]^{\tt 1}
                        &*++[o][F=]{q_{111}} \ar@/^/[l]^{\tt 0}
                                \ar@(ur,dr)^{\tt 1}
\\
*+\txt{}
        &*+\txt{}
                &q_{011} \ar@/^/[uul]^{\tt 0} \ar[ur]_{\tt 1}
}
\]
Oder etwas kompakter:
\[
\entrymodifiers={++[o][F]}
\xymatrix{
*+\txt{}\ar[r]
        &{q_{1}} \ar@(ur,ul)_{\tt 0} \ar@/^/[r]^{\tt 1}
                &{q_{3}} \ar@/^/[l]^{\tt 0} \ar[r]^{\tt 1} 
                        &*++[o][F=]{q_{7}} \ar@(ur,ul)_{\tt 1} \ar@/^/[r]^{\tt 0}
                                &*++[o][F=]{q_{5}} \ar@(ur,ul)_{\tt 0} \ar@/^/[l]^{\tt 1}
}
\]
Es ist jedoch noch nicht klar, ob dieser DEA minimal ist. Daher wenden
wir den Algorithmus für den Minimalautomaten an, um möglicherweise
äquivalente Zustände zu finden. Nach der Markierung von Paaren aus
Akzeptierzuständen und anderen Zuständen als nicht äquivalent hat man:
\begin{center}
\begin{tabular}{|c|cccc|}
\hline
     &$q_1$   &$q_3$   &$q_7$   &$q_5$\\
\hline
$q_1$&$\equiv$&        &$\times$&$\times$\\
$q_3$&        &$\equiv$&$\times$&$\times$\\
$q_7$&$\times$&$\times$&$\equiv$&        \\
$q_5$&$\times$&$\times$&        &$\equiv$\\
\hline
\end{tabular}
\end{center}
Im nächsten Schritt bekommt man:
\begin{center}
\begin{tabular}{|c|cccc|}
\hline
     &$q_1$    &$q_3$    &$q_7$    &$q_5$\\
\hline
$q_1$&$\equiv$ &$\times$ &$\otimes$&$\otimes$\\
$q_3$&$\times$ &$\equiv$ &$\otimes$&$\otimes$\\
$q_7$&$\otimes$&$\otimes$&$\equiv$ &        \\
$q_5$&$\otimes$&$\otimes$&         &$\equiv$\\
\hline
\end{tabular}
\end{center}
Da das Paar $(q_5,q_7)$  nicht als nicht-äquivalent
markiert werden kann, müssen die beiden Zustände äquivalent sein,
so dass der minimale Automat $A_1$
\[
\entrymodifiers={++[o][F]}
\xymatrix{
*+\txt{}\ar[r]
        &{q_{1}} \ar@(ur,ul)_{\tt 0} \ar@/^/[r]^{\tt 1}
                &{q_{3}} \ar@/^/[l]^{\tt 0} \ar[r]^{\tt 1} 
                        &*++[o][F=]{q_{7}} \ar@(dr,ur)_{{\tt 0},{\tt 1}}
}
\]
ist.
\item
Der Automat $A_2$ für die durch drei teilbaren Binärzahlen wurde in der
Vorlesung entwickelt:
\[
\entrymodifiers={++[o][F]}
\xymatrix{
*+\txt{}\ar[r]
        &*++[o][F=]{0} \ar@(ur,ul)_{\tt 0} \ar@/^/[r]^{\tt 1}
                &{1} \ar@/^/[r]^{\tt 0} \ar@/^/[l]^{\tt 1}
                        &{2} \ar@/^/[l]^{\tt 0} \ar@(ur,dr)^{\tt 1}
}
\]
Für die folgenden Teilaufgaben ist jeweils der Produktautomat
zu konstruieren. Wir tun dies jeweils so, dass der Automat für
$L_1$ horizontal, derjenige für $L_2$ vertikal dargestellt wird.
Daher hier nochmals der Automat für $L_2$ vertikal dargestellt
\[
\entrymodifiers={++[o][F]}
\xymatrix{
*+\txt{}\ar[r]
        &*++[o][F=]{0} \ar@(ur,dr)_{\tt 0} \ar@/^/[d]^{\tt 1}
\\
*+\txt{}
        &{1} \ar@/^/[d]^{\tt 0} \ar@/^/[u]^{\tt 1}
\\
*+\txt{}
        &{2} \ar@/^/[u]^{\tt 0} \ar@(dr,dl)^{\tt 1}
}
\]
Der Produktautomat ohne Akzeptierzustände ist dann:
\[
\entrymodifiers={++[o][F]}
\xymatrix{
*+\txt{}\ar[dr]
\\
*+\txt{}
        &{q_1/0} \ar@(ul,ur)^{\tt 0} \ar[dr]^{\tt 1}
                &{q_3/0} \ar[l]_{\tt 0} \ar[dr]^{\tt 1}
                        &{q_7/0} \ar@(ul,ur)^{\tt 0} \ar@/^/[d]^{\tt 1}
\\
*+\txt{}
        &{q_1/1} \ar@/^/[d]^{\tt 0} \ar[ur]^{\tt 1}
                &{q_3/1} \ar[dl]^{\tt 0} \ar[ur]^{\tt 1}
                        &{q_7/1} \ar@/^/[d]^{\tt 0} \ar@/^/[u]^{\tt 1}
\\
*+\txt{}
        &{q_1/2} \ar@/^/[u]^{\tt 0} \ar[r]^{\tt 1}
                &{q_3/2} \ar[ul]^{\tt 0} \ar[r]^{\tt 1}
                        &{q_7/2} \ar@/^/[u]^{\tt 0} \ar@(dr,dl)^{\tt 1}
\\
}
\]

\item
Die Sprache $L_1\cap L_2$ wird von folgendem Produktautomaten akzeptiert:
\[
\entrymodifiers={++[o][F]}
\xymatrix{
*+\txt{}\ar[dr]
\\
*+\txt{}
        &{q_1/0} \ar@(ul,ur)^{\tt 0} \ar[dr]^{\tt 1}
                &{q_3/0} \ar[l]_{\tt 0} \ar[dr]^{\tt 1}
                        &*++[o][F=]{q_7/0} \ar@(ul,ur)^{\tt 0} \ar@/^/[d]^{\tt 1}
\\
*+\txt{}
        &{q_1/1} \ar@/^/[d]^{\tt 0} \ar[ur]^{\tt 1}
                &{q_3/1} \ar[dl]^{\tt 0} \ar[ur]^{\tt 1}
                        &{q_7/1} \ar@/^/[d]^{\tt 0} \ar@/^/[u]^{\tt 1}
\\
*+\txt{}
        &{q_1/2} \ar@/^/[u]^{\tt 0} \ar[r]^{\tt 1}
                &{q_3/2} \ar[ul]^{\tt 0} \ar[r]^{\tt 1}
                        &{q_7/2} \ar@/^/[u]^{\tt 0} \ar@(dr,dl)^{\tt 1}
\\
}
\]
Es ist aber wieder nicht klar, ob dieser Automat minimal ist. Daher
wenden wir erneut den Algorithmus für den Minimalautomaten an. Im
ersten Durchgang erhalten wir
\begin{center}
\begin{tabular}{|c|ccccccccc|}
\hline
         &$q_1/0$  &$q_3/0$  &$q_7/0$  &$q_1/1$  &$q_3/1$  &$q_7/1$  &$q_1/2$  &$q_3/2$  &$q_7/2$  \\
\hline
$q_1/0$  &$\equiv$ &         &$\times$ &         &         &         &         &         &         \\
$q_3/0$  &         &$\equiv$ &$\times$ &         &         &         &         &         &         \\
$q_7/0$  &$\times$ &$\times$ &$\equiv$ &$\times$ &$\times$ &$\times$ &$\times$ &$\times$ &$\times$ \\
$q_1/1$  &         &         &$\times$ &$\equiv$ &         &         &         &         &         \\
$q_3/1$  &         &         &$\times$ &         &$\equiv$ &         &         &         &         \\
$q_7/1$  &         &         &$\times$ &         &         &$\equiv$ &         &         &         \\
$q_1/2$  &         &         &$\times$ &         &         &         &$\equiv$ &         &         \\
$q_3/2$  &         &         &$\times$ &         &         &         &         &$\equiv$ &         \\
$q_7/2$  &         &         &$\times$ &         &         &         &         &         &$\equiv$ \\
\hline
\end{tabular}
\end{center}
Zweiter Durchgang:
\begin{center}
\begin{tabular}{|c|ccccccccc|}
\hline
         &$q_1/0$  &$q_3/0$  &$q_7/0$  &$q_1/1$  &$q_3/1$  &$q_7/1$  &$q_1/2$  &$q_3/2$  &$q_7/2$  \\
\hline
$q_1/0$  &$\equiv$ &         &$\otimes$&         &$\times$ &$\times$ &         &         &$\times$ \\
$q_3/0$  &         &$\equiv$ &$\otimes$&         &$\times$ &$\times$ &         &         &$\times$ \\
$q_7/0$  &$\otimes$&$\otimes$&$\equiv$ &$\otimes$&$\otimes$&$\otimes$&$\otimes$&$\otimes$&$\otimes$\\
$q_1/1$  &         &         &$\otimes$&$\equiv$ &$\times$ &$\times$ &         &$\times$ &$\times$ \\
$q_3/1$  &$\times$ &$\times$ &$\otimes$&$\times$ &$\equiv$ &         &$\times$ &$\times$ &$\times$ \\
$q_7/1$  &$\times$ &$\times$ &$\otimes$&$\times$ &         &$\equiv$ &$\times$ &$\times$ &$\times$ \\
$q_1/2$  &         &         &$\otimes$&         &$\times$ &$\times$ &$\equiv$ &         &$\times$ \\
$q_3/2$  &         &         &$\otimes$&$\times$ &$\times$ &$\times$ &         &$\equiv$ &$\times$ \\
$q_7/2$  &$\times$ &$\times$ &$\otimes$&$\times$ &$\times$ &$\times$ &$\times$ &$\times$ &$\equiv$ \\
\hline
\end{tabular}
\end{center}
Dritter Durchgang:
\begin{center}
\begin{tabular}{|c|ccccccccc|}
\hline
         &$q_1/0$  &$q_3/0$  &$q_7/0$  &$q_1/1$  &$q_3/1$  &$q_7/1$  &$q_1/2$  &$q_3/2$  &$q_7/2$  \\
\hline
$q_1/0$  &$\equiv$ &         &$\otimes$&$\times$ &$\otimes$&$\otimes$&$\times$ &$\times$ &$\otimes$\\
$q_3/0$  &         &$\equiv$ &$\otimes$&$\times$ &$\otimes$&$\otimes$&$\times$ &$\times$ &$\otimes$\\
$q_7/0$  &$\otimes$&$\otimes$&$\equiv$ &$\otimes$&$\otimes$&$\otimes$&$\otimes$&$\otimes$&$\otimes$\\
$q_1/1$  &$\times$ &$\times$ &$\otimes$&$\equiv$ &$\otimes$&$\otimes$&$\times$ &$\otimes$&$\otimes$\\
$q_3/1$  &$\otimes$&$\otimes$&$\otimes$&$\otimes$&$\equiv$ &$\times$ &$\otimes$&$\otimes$&$\otimes$\\
$q_7/1$  &$\otimes$&$\otimes$&$\otimes$&$\otimes$&$\times$ &$\equiv$ &$\otimes$&$\otimes$&$\otimes$\\
$q_1/2$  &$\times$ &$\times$ &$\otimes$&$\times$ &$\otimes$&$\otimes$&$\equiv$ &$\times$ &$\otimes$\\
$q_3/2$  &$\times$ &$\times$ &$\otimes$&$\otimes$&$\otimes$&$\otimes$&$\times$ &$\equiv$ &$\otimes$\\
$q_7/2$  &$\otimes$&$\otimes$&$\otimes$&$\otimes$&$\otimes$&$\otimes$&$\otimes$&$\otimes$&$\equiv$ \\
\hline
\end{tabular}
\end{center}
Das letzte verbleiben Paar $(q_1/0,q_3/0)$ kann im nächsten Durchgang ebenfalls als
nicht äquivalent erkannt werden, womit gezeigt ist, dass der Automat
minimal ist.
\item
Die Mengendifferenz $L_1\setminus L_2$ wird akzeptiert von dem Automaten,
der als Akzeptierzustände die Paare von Akzeptierzuständen hat, welche
aus Akzeptierzuständen von $A_1$ und Nicht-Akzeptierzuständen von $A_2$
bestehen:
\[
\entrymodifiers={++[o][F]}
\xymatrix{
*+\txt{}\ar[dr]
\\
*+\txt{}
        &{q_1/0} \ar@(ul,ur)^{\tt 0} \ar[dr]^{\tt 1}
                &{q_3/0} \ar[l]_{\tt 0} \ar[dr]^{\tt 1}
                        &{q_7/0} \ar@(ul,ur)^{\tt 0} \ar@/^/[d]^{\tt 1}
\\
*+\txt{}
        &{q_1/1} \ar@/^/[d]^{\tt 0} \ar[ur]^{\tt 1}
                &{q_3/1} \ar[dl]^{\tt 0} \ar[ur]^{\tt 1}
                        &*++[o][F=]{q_7/1} \ar@/^/[d]^{\tt 0} \ar@/^/[u]^{\tt 1}
\\
*+\txt{}
        &{q_1/2} \ar@/^/[u]^{\tt 0} \ar[r]^{\tt 1}
                &{q_3/2} \ar[ul]^{\tt 0} \ar[r]^{\tt 1}
                        &*++[o][F=]{q_7/2} \ar@/^/[u]^{\tt 0} \ar@(dr,dl)^{\tt 1}
\\
}
\]
Auch hier muss wieder mit Hilfe des Algorithmus für den Minimalautomaten
sichergestellt werden, dass tatsächlich ein minimaler Automat vorliegt:
\begin{center}
\begin{tabular}{|c|ccccccccc|}
\hline
         &$q_1/0$  &$q_3/0$  &$q_7/0$  &$q_1/1$  &$q_3/1$  &$q_7/1$  &$q_1/2$  &$q_3/2$  &$q_7/2$  \\
\hline
$q_1/0$  &$\equiv$ &         &         &         &         &$\times$ &         &         &$\times$ \\
$q_3/0$  &         &$\equiv$ &         &         &         &$\times$ &         &         &$\times$ \\
$q_7/0$  &         &         &$\equiv$ &         &         &$\times$ &         &         &$\times$ \\
$q_1/1$  &         &         &         &$\equiv$ &         &$\times$ &         &         &$\times$ \\
$q_3/1$  &         &         &         &         &$\equiv$ &$\times$ &         &         &$\times$ \\
$q_7/1$  &$\times$ &$\times$ &$\times$ &$\times$ &$\times$ &$\equiv$ &$\times$ &$\times$ &         \\
$q_1/2$  &         &         &         &         &         &$\times$ &$\equiv$ &         &$\times$ \\
$q_3/2$  &         &         &         &         &         &$\times$ &         &$\equiv$ &$\times$ \\
$q_7/2$  &$\times$ &$\times$ &$\times$ &$\times$ &$\times$ &         &$\times$ &$\times$ &$\equiv$ \\
\hline
\end{tabular}
\end{center}
Zweiter Durchgang:
\begin{center}
\begin{tabular}{|c|ccccccccc|}
\hline
         &$q_1/0$  &$q_3/0$  &$q_7/0$  &$q_1/1$  &$q_3/1$  &$q_7/1$  &$q_1/2$  &$q_3/2$  &$q_7/2$  \\
\hline
$q_1/0$  &$\equiv$ &$\times$ &$\times$ &         &         &$\otimes$&         &$\times$ &$\otimes$\\
$q_3/0$  &$\times$ &$\equiv$ &$\times$ &$\times$ &$\times$ &$\otimes$&$\times$ &         &$\otimes$\\
$q_7/0$  &$\times$ &$\times$ &$\equiv$ &$\times$ &$\times$ &$\otimes$&$\times$ &$\times$ &$\otimes$\\
$q_1/1$  &         &$\times$ &$\times$ &$\equiv$ &$\times$ &$\otimes$&         &$\times$ &$\otimes$\\
$q_3/1$  &         &$\times$ &$\times$ &$\times$ &$\equiv$ &$\otimes$&$\times$ &$\times$ &$\otimes$\\
$q_7/1$  &$\otimes$&$\otimes$&$\otimes$&$\otimes$&$\otimes$&$\equiv$ &$\otimes$&$\otimes$&$\times$ \\
$q_1/2$  &         &$\times$ &$\times$ &         & $\times$&$\otimes$&$\equiv$& $\times$&$\otimes$\\
$q_3/2$  &$\times$ &         &$\times$ &$\times$ &$\times$ &$\otimes$&$\times$ &$\equiv$ &$\otimes$\\
$q_7/2$  &$\otimes$&$\otimes$&$\otimes$&$\otimes$&$\otimes$&$\times$ &$\otimes$&$\otimes$&$\equiv$ \\
\hline
\end{tabular}
\end{center}
Dritter Durchgang:
\begin{center}
\begin{tabular}{|c|ccccccccc|}
\hline
         &$q_1/0$  &$q_3/0$  &$q_7/0$  &$q_1/1$  &$q_3/1$  &$q_7/1$  &$q_1/2$  &$q_3/2$  &$q_7/2$  \\
\hline
$q_1/0$  &$\equiv$ &$\otimes$&$\otimes$&$\times$ &$\times$ &$\otimes$&$\times$ &$\otimes$&$\otimes$\\
$q_3/0$  &$\otimes$&$\equiv$ &$\otimes$&$\otimes$&$\otimes$&$\otimes$&$\otimes$&$\times$ &$\otimes$\\
$q_7/0$  &$\otimes$&$\otimes$&$\equiv$ &$\otimes$&$\otimes$&$\otimes$&$\otimes$&$\otimes$&$\otimes$\\
$q_1/1$  &$\times$ &$\otimes$&$\otimes$&$\equiv$ &$\otimes$&$\otimes$&$\times$ &$\otimes$&$\otimes$\\
$q_3/1$  &$\times$ &$\otimes$&$\otimes$&$\otimes$&$\equiv$ &$\otimes$&$\otimes$&$\otimes$&$\otimes$\\
$q_7/1$  &$\otimes$&$\otimes$&$\otimes$&$\otimes$&$\otimes$&$\equiv$ &$\otimes$&$\otimes$&$\otimes$\\
$q_1/2$  &$\times$ &$\otimes$&$\otimes$&$\times$ &$\otimes$&$\otimes$&$\equiv$ &$\otimes$&$\otimes$\\
$q_3/2$  &$\otimes$&$\times$ &$\otimes$&$\otimes$&$\otimes$&$\otimes$&$\otimes$&$\equiv$ &$\otimes$\\
$q_7/2$  &$\otimes$&$\otimes$&$\otimes$&$\otimes$&$\otimes$&$\otimes$&$\otimes$&$\otimes$&$\equiv$ \\
\hline
\end{tabular}
\end{center}
Somit sind alle Zustände nicht äquivalent, der Automat ist also minimal.
\item
Die Vereinigung $L_1\cup L_2$ wird akzeptiert von dem Automaten,
der als Akzeptierzustände die Paare von Akzeptierzuständen hat, welche
aus Akzeptierzuständen von $A_1$ oder von $A_2$ bestehen:
\[
\entrymodifiers={++[o][F]}
\xymatrix{
*+\txt{}\ar[dr]
\\
*+\txt{}
        &*++[o][F=]{q_1/0} \ar@(ul,ur)^{\tt 0} \ar[dr]^{\tt 1}
                &*++[o][F=]{q_3/0} \ar[l]_{\tt 0} \ar[dr]^{\tt 1}
                        &*++[o][F=]{q_7/0} \ar@(ul,ur)^{\tt 0} \ar@/^/[d]^{\tt 1}
\\
*+\txt{}
        &{q_1/1} \ar@/^/[d]^{\tt 0} \ar[ur]^{\tt 1}
                &{q_3/1} \ar[dl]^{\tt 0} \ar[ur]^{\tt 1}
                        &*++[o][F=]{q_7/1} \ar@/^/[d]^{\tt 0} \ar@/^/[u]^{\tt 1}
\\
*+\txt{}
        &{q_1/2} \ar@/^/[u]^{\tt 0} \ar[r]^{\tt 1}
                &{q_3/2} \ar[ul]^{\tt 0} \ar[r]^{\tt 1}
                        &*++[o][F=]{q_7/2} \ar@/^/[u]^{\tt 0} \ar@(dr,dl)^{\tt 1}
\\
}
\]
Auch hier muss wieder mit Hilfe des Algorithmus für den Minimalautomaten
sichergestellt werden, dass tatsächlich ein minimaler Automat vorliegt:
\begin{center}
\begin{tabular}{|c|ccccccccc|}
\hline
         &$q_1/0$  &$q_3/0$  &$q_7/0$  &$q_1/1$  &$q_3/1$  &$q_7/1$  &$q_1/2$  &$q_3/2$  &$q_7/2$  \\
\hline
$q_1/0$  &$\equiv$ &$       $&$       $&$\times $&$\times $&$       $&$\times $&$\times $&$       $\\
$q_3/0$  &$       $&$\equiv$ &$       $&$\times $&$\times $&$       $&$\times $&$\times $&$       $\\
$q_7/0$  &$       $&$       $&$\equiv$ &$\times $&$\times $&$       $&$\times $&$\times $&$       $\\
$q_1/1$  &$\times $&$\times $&$\times $&$\equiv$ &$       $&$\times $&$       $&$       $&$\times $\\
$q_3/1$  &$\times $&$\times $&$\times $&$       $&$\equiv$ &$\times $&$       $&$       $&$\times $\\
$q_7/1$  &$       $&$       $&$       $&$\times $&$\times $&$\equiv$ &$\times $&$\times $&$       $\\
$q_1/2$  &$\times $&$\times $&$\times $&$       $&$       $&$\times $&$\equiv$ &$       $&$\times $\\
$q_3/2$  &$\times $&$\times $&$\times $&$       $&$       $&$\times $&$       $&$\equiv$ &$\times $\\
$q_7/2$  &$       $&$       $&$       $&$\times $&$\times $&$       $&$\times $&$\times $&$\equiv$ \\
\hline
\end{tabular}
\end{center}
Zweiter Durchgang:
\begin{center}
\begin{tabular}{|c|ccccccccc|}
\hline
         &$q_1/0$  &$q_3/0$  &$q_7/0$  &$q_1/1$  &$q_3/1$  &$q_7/1$  &$q_1/2$  &$q_3/2$  &$q_7/2$  \\
\hline
$q_1/0$  &$\equiv$ &$\times $&$\times $&$\otimes$&$\otimes$&$\times $&$\otimes$&$\otimes$&$\times $\\
$q_3/0$  &$\times $&$\equiv$ &$\times $&$\otimes$&$\otimes$&$\times $&$\otimes$&$\otimes$&$\times $\\
$q_7/0$  &$\times $&$\times $&$\equiv$ &$\otimes$&$\otimes$&$       $&$\otimes$&$\otimes$&$       $\\
$q_1/1$  &$\otimes$&$\otimes$&$\otimes$&$\equiv$ &$\times $&$\otimes$&$\times $&$\times $&$\otimes$\\
$q_3/1$  &$\otimes$&$\otimes$&$\otimes$&$\times $&$\equiv$ &$\otimes$&$\times $&$\times $&$\otimes$\\
$q_7/1$  &$\times $&$\times $&$       $&$\otimes$&$\otimes$&$\equiv$ &$\otimes$&$\otimes$&$       $\\
$q_1/2$  &$\otimes$&$\otimes$&$\otimes$&$\times $&$\times $&$\otimes$&$\equiv$ &$\times $&$\otimes$\\
$q_3/2$  &$\otimes$&$\otimes$&$\otimes$&$\times $&$\times $&$\otimes$&$\times $&$\equiv$ &$\otimes$\\
$q_7/2$  &$\times $&$\times $&$       $&$\otimes$&$\otimes$&$       $&$\otimes$&$\otimes$&$\equiv$ \\
\hline
\end{tabular}
\end{center}
Die verbleibenden Felder lassen sich nicht mehr als nicht äquivalent
erkennen, also müss sie äquivalent sein. Man kann also die Zustände
$q_7/0$, $q_7/1$ und $q_7/2$ zusammenlegen, und erhält so einen DEA
mit nur noch 7 Zuständen:
\[
\entrymodifiers={++[o][F]}
\xymatrix{
*+\txt{}\ar[r]
        &*++[o][F=]{10}\ar@(ul,ur)^{\tt 0} \ar[dr]^{\tt 1}
                &*++[o][F=]{30} \ar[l]^{\tt 0} \ar[dr]^{\tt 1}
\\
*+\txt{}
        &{11} \ar[ur]^{\tt 1} \ar@/^/[d]^{\tt 0}
                &{31} \ar[r]^{\tt 1} \ar[dl]^{\tt 0}
                        &*++[o][F=]{7}\ar@(ur,dr)^{{\tt 0},{\tt 1}}
\\
*+\txt{}
        &{12} \ar@/^/[u]^{\tt 0} \ar[r]^{\tt 1}
                &{32} \ar[ur]^{\tt 1} \ar[ul]^{\tt 0}
}
\]
\item $L_6$ ist die Verkettung der zwei Sprachen $L_1$ und $L_2$.
Den zugörigen NEA konstruiert man, indem man die beiden Teilautomaten
mit einem $\varepsilon$-"Ubergang aneinanderhängt:
\[
\entrymodifiers={++[o][F]}
\xymatrix{
*+\txt{}\ar[r]
        &{q_{1}} \ar@(ur,ul)_{\tt 0} \ar@/^/[r]^{\tt 1}
                &{q_{3}} \ar@/^/[l]^{\tt 0} \ar[r]^{\tt 1} 
                        &{q_7} \ar@(ur,ul)_{{\tt 0},{\tt 1}}
				\ar[r]^{\varepsilon}
        &*++[o][F=]{0} \ar@(ur,ul)_{\tt 0} \ar@/^/[r]^{\tt 1}
                &{1} \ar@/^/[r]^{\tt 0} \ar@/^/[l]^{\tt 1}
                        &{2} \ar@/^/[l]^{\tt 0} \ar@(ur,dr)^{\tt 1}
}
\]
\item
Die Stern-Konstruktion liefert den folgenden Automaten für $L_7$
\[
\entrymodifiers={++[o][F]}
\xymatrix{
*+\txt{}\ar[r]
	&*++[o][F=]{*} \ar[r]^{\varepsilon}
        &{q_{1}} \ar@(ur,ul)_{\tt 0} \ar@/^/[r]^{\tt 1}
                &{q_{3}} \ar@/^/[l]^{\tt 0} \ar[r]^{\tt 1} 
                        &{q_{7}} \ar@(dr,ur)_{{\tt 0},{\tt 1}}
				\ar@/^20pt/[lll]^{\varepsilon}
}
\qedhere
\]
\end{teilaufgaben}
\end{loesung}
