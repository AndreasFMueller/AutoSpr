Auf der Website 
\url{https://blog.patricktriest.com/you-should-learn-regex/}
stellt der Autor folgende Aufgaben:
\begin{teilaufgaben}
\item
Finde einen regulären Ausdruck, mit dem man Jahreszahlen zwischen
1800 und 2199 in einem
historischen Dokument finden kann.
Jahreszahlen sind vierstellig und wenn sie im Inneren eines
Strings vorkommen, müssen Sie mit Leerzeichen begrenzt sein.
\item
Kommentare in der Programmiersprache C werden von \texttt{/*} und
\texttt{*/} eingerahmt und können nicht geschachtelt werden.
Zwischen den Kommentarzeichen kann jedes beliebige Zeichen vorkommen.
Finden Sie einen regulären Ausdruck, der einen C-Kommentar
erkennt.
\end{teilaufgaben}

\thema{reguläre Ausdrücke}
\thema{regulär}

\begin{loesung}
\begin{teilaufgaben}
\item
Jahreszahlen zwischen 1800 und 2199 passen auf den regulären Ausdruck
\[
r_1 = \texttt{(18|19|20|21)[0-9]\{2\}}
\]
Nun müssen wir aber noch sicherstellen, dass Zahlen im inneren eines
Strings von Blanks eingerahmt sind.
Wenn links oder rechts von einer Jahreszahl noch etwas steht, dann muss
unmittelbar links oder rechts von der Jahreszahl ein Blank stehen.
Der Ausdruck
\[
r_2
=
\texttt{(.*\blank)?}r_1\texttt{(\blank.*)?}
\]
erfasst dies.
\item
C-Kommentare dürfen nicht geschachtelt werden, nach einem \texttt{*} darf
also nur dann ein \texttt{/} folgen, wenn die Zeichenkette dann auch
fertig ist.
Diese negative Bedingung ist mit einem ad hoc entworfenen regulären Ausdruck
nicht so einfach zu beschreiben.
Wir können aber leicht einen endlichen Automaten angeben, der dies 
bewerkstelligt.
Im inneren der Zeichenkette gibt es zwei mögliche Zustände:
\begin{enumerate}
\item
Es ist unmittelbar ein \texttt{*} vorangegangen, entweder wird der
Kommentar jetzt mit einem \texttt{/} beendet oder es folgt ein von \texttt{/}
verschiedenes Zeichen.
Die ist Zustand $q_3$ im nachfolgenden Zustandsdiagramm.
\item
Es ist unmittelbar ein von \texttt{*} verschiedenes Zeichen
vorangegangen (das \texttt{*}, welches den Kommentar eröffnet hat, zählt nicht).
Dies ist Zustand $q_2$ im nachfolgenden Zustandsdiagramm.
\end{enumerate}
Dies führt auf das folgende Zustandsdiagramm:
\[
\entrymodifiers={++[o][F]}
\xymatrix @+5mm {
*+\txt{} \ar[r]
	&{S} \ar[r]^{\displaystyle\texttt{/}}
		&{q_1} 	\ar[r]^{\displaystyle\texttt{\textbackslash*}}
			&{q_2}	\ar@/_/[d]_{\displaystyle\texttt{\textbackslash*}}
				\ar@(ur,ul)_{\displaystyle\texttt{[\textasciicircum\textbackslash*]}}
				&*+\txt{}
\\
*+\txt{}
	&*+\txt{}
		&*+\txt{}
			&{q_3} \ar[r]^{\displaystyle\texttt{/}}
				\ar@/_/[u]_{\displaystyle\texttt{[\textasciicircum\textbackslash*/]}}
				\ar@(dr,dl)^{\displaystyle\textbackslash*}
				&*++[o][F=]{E}
}
\]
Diesen VNEA müssen wir jetzt in einen regulären Ausdruck umwandeln, indem
wir der Reihe nach die Zustände $q_1$, $q_2$ und $q_3$ entfernen.
Entfernung von $q_1$:
\[
\entrymodifiers={++[o][F]}
\xymatrix @+5mm {
*+\txt{} \ar[r]
	&{S} \ar[rr]^{\displaystyle\texttt{/\textbackslash*}}
		&*+\txt{}
			&{q_2}	\ar@/_/[d]_{\displaystyle\texttt{\textbackslash*}}
				\ar@(ur,ul)_{\displaystyle\texttt{[\textasciicircum\textbackslash*]}}
				&*+\txt{}
\\
*+\txt{}
	&*+\txt{}
		&*+\txt{}
			&{q_3} \ar[r]^{\displaystyle\texttt{/}}
				\ar@/_/[u]_{\displaystyle\texttt{[\textasciicircum/]}}
				\ar@(dr,dl)^{\displaystyle\textbackslash*}
				&*++[o][F=]{E}
}
\]
Entfernung von $q_2$:
\[
\entrymodifiers={++[o][F]}
\xymatrix @+5mm {
*+\txt{} \ar[r]
	&{S} \ar[rr]^{\displaystyle\texttt{/\textbackslash*[\textasciicircum\textbackslash*]*\textbackslash*}}
		&*+\txt{}
			&{q_3} \ar[r]^{\displaystyle\texttt{/}}
				\ar@(dr,dl)^{\displaystyle\texttt{[\textasciicircum\textbackslash*][\textasciicircum\textbackslash*]*\textbackslash*|\textbackslash*}}
				&*++[o][F=]{E}
}
\]
Entfernung von $q_3$:
\[
\entrymodifiers={++[o][F]}
\xymatrix @+5mm {
*+\txt{} \ar[r]
	&{S} \ar[rrrrr]^{\displaystyle\texttt{/\textbackslash*[\textasciicircum\textbackslash*]*\textbackslash*([\textasciicircum\textbackslash*][\textasciicircum\textbackslash*]*\textbackslash*|\textbackslash*)*/}}
	&*+\txt{}
		&*+\txt{}
		&*+\txt{}
			&*+\txt{}
				&*++[o][F=]{E}
}
\]
Die Beschriftung des Übergangs von $S$ nach $A$ ist der gesuchte reguläre
Ausdruck.
\qedhere
\end{teilaufgaben}
\end{loesung}

\begin{diskussion}
Die Lösung für b) von der genannten Website akzeptiert übrigens weit mehr
Zeichenketten, die keine legalen C-Kommentare sind.
\end{diskussion}

\begin{bewertung}
\begin{teilaufgaben}
\item
Richtige Jahrszahlen ({\bf J}) 1 Punkt,
Korrekte Begrenzung mit Whitespace ({\bf W}) 1 Punkt.
\item
Begrenzungszeichen ({\bf B}) 1 Punkt,
Kommentartext ({\bf K}) 1 Punkt,
Erkenntnis, dass im inneren keine Kommentarendzeichen vorkommen dürfen
({\bf E}) 1 Punkt,
Implementation keine Endzeichen im Inhalt ({\bf I}) 1 Punkt.
\end{teilaufgaben}
\end{bewertung}




