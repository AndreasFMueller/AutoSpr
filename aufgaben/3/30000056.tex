IPv6-Adressen bestehen aus acht Gruppen von jeweils 16 Bits.
Eine Gruppe wird als maximal 4 Hex-Ziffern \texttt{0} -- \texttt{9},
\texttt{a} -- \texttt{f} codiert, führende Nullen sind in jeder Gruppe erlaubt.
Die Gruppen werden durch Doppelpunkt \texttt{:} getrennt.
Eine typische IPv6-Adresse ist
\[
\texttt{2001:db8:85a3:0:0:8a2e:370:7334}
\]
Folgen mehr als eine Gruppe aus lauter Nullen aufeinander, können diese 
durch einen doppelten Doppelpunkt \texttt{::} abgekürzt werden, aber
nur einmal, da zum Beispiel in der Adresse \texttt{1::1::1} nicht klar wäre,
wie lange die Null-Gruppen
sind und damit an welcher Position in der Adresse die mittlere \texttt{1}
eigentlich steht.
Die obige Adresse kann also zu
\[
\texttt{2001:db8:85a3::8a2e:370:7334}
\]
abgekürzt werden.

Finden Sie einen regulären Ausdruck, mit dem Eingebefelder für IPv6-Adressen
plausibilisiert werden können.

\begin{loesung}
Die einzelnen Gruppen werden durch den regulären Ausdruck
\[
r_1 = \texttt{[0-9a-f]\char`\{1,4\char`\}}
\]
beschrieben.
Dieser Ausdruck akzeptiert das leere Wort nicht.

Eine typische IPv6-Adresse ohne die Abkürzung von Nullgruppen wird durch
den regulären Ausdruck 
\begin{align*}
r_2
&=
r_1\texttt{(:}r_1\texttt{)\char`\{7\char`\}}
\\
&=
\texttt{[0-9a-f]\char`\{1,4\char`\}}
\texttt{(:}
\texttt{[0-9a-f]\char`\{1,4\char`\}}
\texttt{)\char`\{7\char`\}}
\end{align*}
beschrieben.

Wenn nur mindestens zwei Nullgruppen abgekürzt werden sollen, dann
besteht eine IPv6-Adresse aus zwei Teiladressen, die möglicherweise
leer sind, getrennt von einem doppelten Doppelpunkt \texttt{::}.
\begin{align*}
r_3 &= r_4\texttt{::}r_4
\\
\text{mit}\qquad
r_4
&=
\texttt{(}r_1 \texttt{(:}r_1\texttt{)\char`\{,6\char`\}}\texttt{)?}
\\
&=
\texttt{(}
\texttt{[0-9a-f]\char`\{1,4\char`\}}
\texttt{(:}
\texttt{[0-9a-f]\char`\{1,4\char`\}}
\texttt{)\char`\{,6\char`\}}\texttt{)?}
\\
&=
\texttt{(}
\texttt{[0-9a-f]\char`\{1,4\char`\}}
\texttt{(:}
\texttt{[0-9a-f]\char`\{1,4\char`\}}
\texttt{)\char`\{,6\char`\}}\texttt{)?}
\end{align*}

Eine beliebige IPv6-Adresse passt daher auf den regulären Ausruck
\begin{align*}
r
&=
r_2\texttt{|} r_3,
\end{align*}
der jedoch zu lange ist, ihn auf einer einzigen Zeile darzustellen.

Dieser Ausdruck stellt nicht sicher, dass die Gesamtzahl von Gruppen
auf beiden Seiten des doppelten Doppelpunktes nicht grösser als $6$ ist.
Dazu müssten man zusätzliche Ausrücke ähnlich wie $r_4\texttt{::}r_4$
bauen, jedoch $r_4$ ersetzen durch Ausdrücke, die die Anzahl der Gruppen
limitiert.
Dies scheint jedoch übermässig kompliziert und sollte daher der
Anwendungslogik überlassen werden.
\end{loesung}





