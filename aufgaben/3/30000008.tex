Verwandeln Sie die folgenden regulären Ausdrücke in NEAs über
dem Alphabet $\Sigma =\{{\tt a},{\tt b}\}$:
\begin{teilaufgaben}
\item ${\tt a}({\tt abb})^*| {\tt b}$
\item ${\tt a}^+|({\tt ab})^+$
\end{teilaufgaben}

\begin{hinweis}
Das Zeichen {\tt +} bei regulären Ausdrücken bedeutet
``mindestens einmal''.
\end{hinweis}

\thema{NEA}
\themaL{regulare Ausdrucke}{reguläre Ausdrücke}

\begin{loesung}
Wir wenden die Standardkonstruktionen für die Umwandlung eines regulären
Ausdrucks in einen NEA an. Die regulären Ausdrücke werden dabei
``von innen nach aussen'' übersetzt.
\def\zustand#1{
	\draw #1 circle[radius=0.2];
}
\def\akzeptierzustand#1{
	\zustand{#1}
	\draw #1 circle[radius=0.15];
}
\def\pfeil#1#2#3{
	\draw[->,shorten >= 0.2cm,shorten <= 0.2cm] #1 -- #2;
	\node at ($0.5*#1+0.5*#2$) [above] {#3};
}
\begin{teilaufgaben}
\item
\def\l{1.3}
\def\punkte{
	\coordinate (q0) at (0,0);
	\coordinate (q1) at (\l,0);
	\coordinate (q2) at ({2*\l},0);
	\coordinate (q3) at ({3*\l},0);
	\coordinate (q4) at ({4*\l},0);
	\coordinate (q5) at ({5*\l},0);
	\coordinate (qa) at ({-\l},0);
	\coordinate (qb) at ({-2*\l},0);
	\coordinate (qc) at ({-3*\l},0);
	\coordinate (z0) at ({-4*\l},-\l);
	\coordinate (z1) at ({-3*\l},{-2*\l});
	\coordinate (z2) at ({-2*\l},{-2*\l});
}
Der Teilausdruck {\tt abb} ist eine Verkettung von drei NEAs, die
genau ein ein Zeichen langes Wort akzeptieren:
\begin{center}
\begin{tikzpicture}[>=latex,thick]
\punkte
\coordinate (s) at (-\l,0);
\zustand{(q0)}
\zustand{(q1)}
\zustand{(q2)}
\zustand{(q3)}
\zustand{(q4)}
\pfeil{(s)}{(q0)}{}
\pfeil{(q0)}{(q1)}{\texttt{a}}
\pfeil{(q1)}{(q2)}{$\varepsilon$}
\pfeil{(q2)}{(q3)}{\texttt{b}}
\pfeil{(q3)}{(q4)}{$\varepsilon$}
\pfeil{(q4)}{(q5)}{\texttt{b}}
\akzeptierzustand{(q5)}
\end{tikzpicture}
\end{center}
Für die $*$-Operation braucht man einen zusätzlichen Akzeptierzustand und
``Rückwärts''-Pfeile:
\begin{center}
\begin{tikzpicture}[>=latex,thick]
\punkte
\coordinate (s) at ({-2*\l},0);
\zustand{(q0)}
\zustand{(q1)}
\zustand{(q2)}
\zustand{(q3)}
\zustand{(q4)}
\pfeil{(s)}{(qa)}{}
\pfeil{(q0)}{(q1)}{\texttt{a}}
\pfeil{(q1)}{(q2)}{$\varepsilon$}
\pfeil{(q2)}{(q3)}{\texttt{b}}
\pfeil{(q3)}{(q4)}{$\varepsilon$}
\pfeil{(q4)}{(q5)}{\texttt{b}}
\pfeil{(qa)}{(q0)}{$\varepsilon$}
\zustand{(q5)}
\draw[->,shorten >= 0.2cm,shorten <= 0.2cm]
	(q5) to[out=-90,in=0]
	($(q5)+({-0.5*\l},{-0.5*\l-0.2})$) --
	($(qa)+({0.5*\l},{-0.5*\l-0.2})$) to[out=180,in=-90] (qa);
\node at ($(q2)+(0,{-0.5*\l-0.2})$) [above] {$\varepsilon$};
\akzeptierzustand{(qa)};
\end{tikzpicture}
\end{center}
Dem $({\tt abb})^*$ muss jetzt noch ein {\tt a} vorangehen:
\begin{center}
\begin{tikzpicture}[>=latex,thick]
\punkte
\coordinate (s) at ({-4*\l},0);
\zustand{(q0)}
\zustand{(q1)}
\zustand{(q2)}
\zustand{(q3)}
\zustand{(q4)}
\pfeil{(s)}{(qc)}{}
\pfeil{(qc)}{(qb)}{\texttt{a}}
\pfeil{(qb)}{(qa)}{$\varepsilon$}
\pfeil{(q0)}{(q1)}{\texttt{a}}
\pfeil{(q1)}{(q2)}{$\varepsilon$}
\pfeil{(q2)}{(q3)}{\texttt{b}}
\pfeil{(q3)}{(q4)}{$\varepsilon$}
\pfeil{(q4)}{(q5)}{\texttt{b}}
\pfeil{(qa)}{(q0)}{$\varepsilon$}
\zustand{(q5)}
\draw[->,shorten >= 0.2cm,shorten <= 0.2cm]
	(q5) to[out=-90,in=0]
	($(q5)+({-0.5*\l},{-0.5*\l-0.2})$) --
	($(qa)+({0.5*\l},{-0.5*\l-0.2})$) to[out=180,in=-90] (qa);
\node at ($(q2)+(0,{-0.5*\l-0.2})$) [above] {$\varepsilon$};
\akzeptierzustand{(qa)}
\zustand{(qb)}
\zustand{(qc)}
\end{tikzpicture}
\end{center}
Zum Schluss müssen wir eine Alternative zwischen diesem Ausdruck und dem
Ausdruck {\tt b} bilden:
\begin{center}
\begin{tikzpicture}[>=latex,thick]
\punkte
\coordinate (s) at ({-5*\l},-\l);
\zustand{(q0)}
\zustand{(q1)}
\zustand{(q2)}
\zustand{(q3)}
\zustand{(q4)}
\pfeil{(s)}{(z0)}{}
\pfeil{(qc)}{(qb)}{\texttt{a}}
\pfeil{(qb)}{(qa)}{$\varepsilon$}
\pfeil{(q0)}{(q1)}{\texttt{a}}
\pfeil{(q1)}{(q2)}{$\varepsilon$}
\pfeil{(q2)}{(q3)}{\texttt{b}}
\pfeil{(q3)}{(q4)}{$\varepsilon$}
\pfeil{(q4)}{(q5)}{\texttt{b}}
\pfeil{(qa)}{(q0)}{$\varepsilon$}
\zustand{(q5)}
\draw[->,shorten >= 0.2cm,shorten <= 0.2cm]
	(q5) to[out=-90,in=0]
	($(q5)+({-0.5*\l},{-0.5*\l-0.2})$) --
	($(qa)+({0.5*\l},{-0.5*\l-0.2})$) to[out=180,in=-90] (qa);
\node at ($(q2)+(0,{-0.5*\l-0.2})$) [above] {$\varepsilon$};
\akzeptierzustand{(qa)}
\zustand{(qb)}
\zustand{(qc)}
\zustand{(z0)}
\zustand{(z1)}
\akzeptierzustand{(z2)}
\draw[->,shorten >= 0.2cm,shorten <= 0.2cm] (z0) to[out=90,in=180] (qc);
\draw[->,shorten >= 0.2cm,shorten <= 0.2cm] (z0) to[out=-90,in=180] (z1);
\pfeil{(z1)}{(z2)}{\texttt{b}}
\node at ($0.5*(z0)+0.5*(z1)+(-0.2,0)$) {$\varepsilon$};
\node at ($0.5*(z0)+0.5*(qc)+(-0.2,0)$) {$\varepsilon$};
\end{tikzpicture}
\end{center}
\item
\def\punkte{
\coordinate (s) at (-\l,0);
\coordinate (q) at (0,0);
\coordinate (q1) at ({1*\l},\l);
\coordinate (q2) at ({2*\l},\l);
\coordinate (q3) at ({3*\l},\l);
\coordinate (q4) at ({4*\l},\l);
\coordinate (q5) at ({5*\l},\l);
\coordinate (z1) at ({1*\l},-\l);
\coordinate (z2) at ({2*\l},-\l);
\coordinate (z3) at ({3*\l},-\l);
\coordinate (z4) at ({4*\l},-\l);
\coordinate (z5) at ({5*\l},-\l);
\coordinate (z6) at ({6*\l},-\l);
\coordinate (z7) at ({7*\l},-\l);
\coordinate (z8) at ({8*\l},-\l);
\coordinate (z9) at ({9*\l},-\l);
}
Es stellt sich die Frage, wie man $r^+$ implementieren will. Man könnte
$r^+$ als genau wie $r^*$ implementieren, die Konstruktion aber so
modifizieren, dass aber das leere Wort
doch nicht akzeptiert wird.  Dazu würde man den neuen Start-/Akzeptierzustand,
der in der Konstruktion von $r^*$ vorkommt, nur als Startzustand, nicht
aber als Akzeptierzustand hinzufügen. Dann sähe der Automat wie folgt
aus:
\begin{center}
\begin{tikzpicture}[>=latex,thick]
\punkte
\zustand{(q)}
\pfeil{(s)}{(q)}{}
\draw[->,shorten >= 0.2cm,shorten <= 0.2cm] (q) to[out=90,in=180] (q1);
\draw[->,shorten >= 0.2cm,shorten <= 0.2cm] (q) to[out=-90,in=180] (z1);
\node at ($0.5*(q)+0.5*(q1)+(-0.2,0)$) {$\varepsilon$};
\node at ($0.5*(q)+0.5*(z1)+(-0.2,0)$) {$\varepsilon$};
\zustand{(q1)}
\zustand{(q2)}
\akzeptierzustand{(q3)}
\pfeil{(q1)}{(q2)}{$\varepsilon$}
\pfeil{(q2)}{(q3)}{\texttt{a}}
\draw[->,shorten >= 0.2cm,shorten <= 0.2cm]
	(q3) to[out=-90,in=0]
	($(q3)+0.5*(-\l,{-\l-0.4})$) --
	($(q1)+0.5*(\l,{-\l-0.4})$) to[out=180,in=-90] (q1);
\node at ($(q2)+(0,{-0.5*\l-0.2})$) [above] {$\varepsilon$};
\zustand{(z1)}
\zustand{(z2)}
\zustand{(z3)}
\zustand{(z4)}
\akzeptierzustand{(z5)}
\pfeil{(z1)}{(z2)}{$\varepsilon$}
\pfeil{(z2)}{(z3)}{\texttt{a}}
\pfeil{(z3)}{(z4)}{$\varepsilon$}
\pfeil{(z4)}{(z5)}{\texttt{b}}
\draw[->,shorten >= 0.2cm,shorten <= 0.2cm]
	(z5) to[out=90,in=0]
	($(z5)+0.5*(-\l,{\l+0.4})$) --
	($(z1)+0.5*(\l,{\l+0.4})$) to[out=180,in=90] (z1);
\node at ($(z3)+(0,{0.5*\l+0.2})$) [above] {$\varepsilon$};
\end{tikzpicture}
\end{center}
Man könnte aber auch $r^+=rr^*$ implementieren, also als Verkettung
eines Automaten, der $r$ akzeptiert mit einer $*$-Konstruktion von $r$.
Dann sähe der Automat wie folgt aus:
\begin{center}
\begin{tikzpicture}[>=latex,thick]
\punkte
\zustand{(q)}
\pfeil{(s)}{(q)}{}
\draw[->,shorten >= 0.2cm,shorten <= 0.2cm] (q) to[out=90,in=180] (q1);
\draw[->,shorten >= 0.2cm,shorten <= 0.2cm] (q) to[out=-90,in=180] (z1);
\node at ($0.5*(q)+0.5*(q1)+(-0.2,0)$) {$\varepsilon$};
\node at ($0.5*(q)+0.5*(z1)+(-0.2,0)$) {$\varepsilon$};
\zustand{(q1)}
\zustand{(q2)}
\akzeptierzustand{(q3)}
\zustand{(q4)}
\zustand{(q5)}
\pfeil{(q1)}{(q2)}{\texttt{a}}
\pfeil{(q2)}{(q3)}{$\varepsilon$}
\pfeil{(q3)}{(q4)}{$\varepsilon$}
\pfeil{(q4)}{(q5)}{\texttt{a}}
\draw[->,shorten >= 0.2cm,shorten <= 0.2cm]
	(q5) to[out=-90,in=0]
	($(q5)+0.5*(-\l,{-\l-0.4})$) --
	($(q3)+0.5*(\l,{-\l-0.4})$) to[out=180,in=-90] (q3);
\node at ($(q4)+(0,{-0.5*\l-0.2})$) [above] {$\varepsilon$};
\zustand{(z1)}
\zustand{(z2)}
\zustand{(z3)}
\zustand{(z4)}
\akzeptierzustand{(z5)}
\pfeil{(z1)}{(z2)}{\texttt{a}}
\pfeil{(z2)}{(z3)}{$\varepsilon$}
\pfeil{(z3)}{(z4)}{\texttt{b}}
\pfeil{(z4)}{(z5)}{$\varepsilon$}
\draw[->,shorten >= 0.2cm,shorten <= 0.2cm]
	(z9) to[out=90,in=0]
	($(z9)+0.5*(-\l,{\l+0.4})$) --
	($(z5)+0.5*(\l,{\l+0.4})$) to[out=180,in=90] (z5);
\node at ($(z7)+(0,{0.5*\l+0.2})$) [above] {$\varepsilon$};
\pfeil{(z5)}{(z6)}{$\varepsilon$}
\pfeil{(z6)}{(z7)}{\texttt{a}}
\pfeil{(z7)}{(z8)}{$\varepsilon$}
\pfeil{(z8)}{(z9)}{\texttt{b}}
\zustand{(z6)}
\zustand{(z7)}
\zustand{(z8)}
\zustand{(z9)}
\end{tikzpicture}
\end{center}
\qedhere
\end{teilaufgaben}
\end{loesung}
