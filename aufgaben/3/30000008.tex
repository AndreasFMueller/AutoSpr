Verwandeln Sie die folgenden regulären Ausdrücke in NEAs über
dem Alphabet $\Sigma =\{{\tt a},{\tt b}\}$:
\begin{teilaufgaben}
\item ${\tt a}({\tt abb})^*| {\tt b}$
\item ${\tt a}^+|({\tt ab})^+$
\end{teilaufgaben}

\begin{hinweis}
Das Zeichen {\tt +} bei regulären Ausdrücken bedeutet
``mindestens einmal''.
\end{hinweis}

\thema{NEA}
\thema{reguläre Ausdrücke}

\begin{loesung}
Wir wenden die Standardkonstruktionen für die Umwandlung eines regulären
Ausdrucks in einen NEA an. Die regulären Ausdrücke werden dabei
``von innen nach aussen'' übersetzt.
\begin{teilaufgaben}
\item
Der Teilausdruck {\tt abb} ist eine Verkettung von drei NEAs, die
genau ein ein Zeichen langes Wort akzeptieren:
\[
\entrymodifiers={++[o][F]}
\xymatrix{
*+\txt{}\ar[r]
        &{}\ar[r]^{\tt a}
                &\ar[r]^{\varepsilon}
                        &{}\ar[r]^{\tt b}
                                &\ar[r]^{\varepsilon}
                                        &{}\ar[r]^{\tt b}
                                                &*++[o][F=]{}
}
\]
Für die $*$-Operation braucht man einen zusätzlichen Akzeptierzustand und
``Rückwärts''-Pfeile:
\[
\entrymodifiers={++[o][F]}
\xymatrix{
*+\txt{}\ar[r]
        &*++[o][F=]{} \ar[r]^{\varepsilon}
                &\ar[r]^{\tt a}
                        &\ar[r]^{\varepsilon}
                                &\ar[r]^{\tt b}
                                        &\ar[r]^{\varepsilon}
                                                &{}\ar[r]^{\tt b}
                                                        &{}\ar@/^10pt/[llllll]^{\varepsilon}
}
\]
Dem $({\tt abb})^*$ muss jetzt noch ein {\tt a} vorangehen:
\[
\entrymodifiers={++[o][F]}
\xymatrix{
*+\txt{}\ar[r]
        &\ar[r]^{\tt a}
        &{} \ar[r]^{\varepsilon}
        &*++[o][F=]{} \ar[r]^{\varepsilon}
                &\ar[r]^{\tt a}
                        &\ar[r]^{\varepsilon}
                                &\ar[r]^{\tt b}
                                        &\ar[r]^{\varepsilon}
                                                &{}\ar[r]^{\tt b}
                                                        &{}\ar@/^10pt/[llllll]^{\varepsilon}
}
\]
Zum Schluss müssen wir eine Alternative zwischen diesem Ausdruck und dem
Ausdruck {\tt b} bilden:
\[
\entrymodifiers={++[o][F]}
\xymatrix{
*+\txt{}&*+\txt{}
        &\ar[r]^{\tt a}
        &{} \ar[r]^{\varepsilon}
        &*++[o][F=]{} \ar[r]^{\varepsilon}
                &\ar[r]^{\tt a}
                        &\ar[r]^{\varepsilon}
                                &\ar[r]^{\tt b}
                                        &\ar[r]^{\varepsilon}
                                                &{}\ar[r]^{\tt b}
                                                        &{}\ar@/^10pt/[llllll]^{\varepsilon}
\\
*+\txt{} \ar[r]
        &\ar[ur]^{\varepsilon} \ar[dr]^{\varepsilon}
        &*+\txt{}
        &*+\txt{}
        &*+\txt{}
        &*+\txt{}
        &*+\txt{}
        &*+\txt{}
        &*+\txt{}
\\
*+\txt{}&*+\txt{}
                &\ar[r]^{\tt b}
                        &*++[o][F=]{}
                                &*+\txt{}
}
\]
\item
Es stellt sich die Frage, wie man $r^+$ implementieren will. Man könnte
$r^+$ als genau wie $r^*$ implementieren, die Konstruktion aber so
modifizieren, dass aber das leere Wort
doch nicht akzeptiert wird.  Dazu würde man den neuen Start-/Akzeptierzustand,
der in der Konstruktion von $r^*$ vorkommt, nur als Startzustand, nicht
aber als Akzeptierzustand hinzufügen. Dann sähe der Automat wie folgt
aus:
\[
\entrymodifiers={++[o][F]}
\xymatrix{
*+\txt{}
        &*+\txt{}
%               &*++[o][F=]{} \ar[r]^{\varepsilon}
                &{} \ar[r]^{\varepsilon}
                        &{} \ar[r]^{\tt a}
                                &*++[o][F=]{} \ar@/^10pt/[ll]^{\varepsilon}
\\
*+\txt{}\ar[r]
        &\ar[ur]^{\varepsilon} \ar[dr]^{\varepsilon}
\\
*+\txt{}
        &*+\txt{}
%               &*++[o][F=]{} \ar[r]^{\varepsilon}
                &{} \ar[r]^{\varepsilon}
                        &\ar[r]^{\tt a}
                                &\ar[r]^{\varepsilon}
                                        &\ar[r]^{\tt b}
                                                &*++[o][F=]{}\ar@/^10pt/[llll]^{\varepsilon}
}
\]
Man könnte aber auch $r^+=rr^*$ implementieren, also als Verkettung
eines Automaten, der $r$ akzeptiert mit einer $*$-Konstruktion von $r$.
Dann sähe der Automat wie folgt aus:
\[
\entrymodifiers={++[o][F]}
\xymatrix{
*+\txt{}
        &*+\txt{}
%               &*++[o][F=]{} \ar[r]^{\varepsilon}
                &{}\ar[r]^{\tt a}
                        &{}\ar[r]^{\varepsilon}
                &*++[o][F=]{} \ar[r]^{\varepsilon}
                        &{} \ar[r]^{\tt a}
                                &{} \ar@/^10pt/[ll]^{\varepsilon}
\\
*+\txt{}\ar[r]
        &\ar[ur]^{\varepsilon} \ar[dr]^{\varepsilon}
\\
*+\txt{}
        &*+\txt{}
%               &*++[o][F=]{} \ar[r]^{\varepsilon}
                        &\ar[r]^{\tt a}
                                &\ar[r]^{\varepsilon}
                                        &\ar[r]^{\tt b}
                &{} \ar[r]^{\varepsilon}
                        &*++[o][F=]{} \ar[r]^{\varepsilon}
                        &\ar[r]^{\tt a}
                                &\ar[r]^{\varepsilon}
                                        &\ar[r]^{\tt b}
                                                &{}\ar@/^10pt/[llll]^{\varepsilon}
}
\]
\qedhere
\end{teilaufgaben}
\end{loesung}
