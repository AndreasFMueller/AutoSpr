1905 beschlossen die Kantone, dass alle Autos in der Schweiz mit einem
Kennzeichen versehen werden müssen, und teilten jedem Kanton einen
Nummernblock zu. Nach dem Motto ``Mehr wird niemand brauchen'', welches
ja auch schon in der Informatik spektakulär gescheitert ist, erhielt
der Kanton Zürich nur gerade 1000 Nummern, der
Kanton Appenzell Innerrhoden sogar nur 100.
Daher wurde 1933 das heute noch gültige System der Autokennzeichen
eingeführt. Die Kennzeichen bestehen aus dem Kantonskürzel
oder den Buchstaben ``A'' (Administration), ``P'' (Post) oder
``M'' für Militärfahrzeuge gefolgt von einer maximal sechsstelligen Zahl
(keine führende Nullen).
In Sonderfällen kann ein weiterer Buchstabe angehängt sein, welcher
Sondernutzungen anzeigt: ``U'' für ``Garagennummern'',
``V'' für Mietfahrzeuge und ``Z'' für Zollschilder (befristet).
Formulieren Sie einen regulären Ausdruck für schweizerische Autokennzeichen,
der alle eben beschriebenen Kriterien abbildet.

\begin{loesung}
Wir brauchen zunächst einen regulären Ausdruck für die Kantonskürzel
und die anderen drei Präfixe:
\[
{\tt AG}|
{\tt AR}|
{\tt AI}|
{\tt BL}|
{\tt BS}|
{\tt BE}|
{\tt FR}|
{\tt GE}|
{\tt GL}|
{\tt GR}|
{\tt JU}|
{\tt LU}|
{\tt NE}|
{\tt NW}|
{\tt OW}|
{\tt SH}|
{\tt SZ}|
{\tt SO}|
{\tt SG}|
{\tt TI}|
{\tt TG}|
{\tt UR}|
{\tt VD}|
{\tt VS}|
{\tt ZG}|
{\tt ZH}|
{\tt A}|
{\tt P}|
{\tt M}
\]
Diesem folgt eine maximal sechsstellige Zahl, die aber nicht mit einer
führenden Null beginnen darf. Wir verwenden die Notationen
\begin{align*}
\text{\tt [1-9]}&=\text{Ziffern von {\tt 1} bis {\tt 9}}
&&=
({\tt 1}|
{\tt 2}|
{\tt 3}|
{\tt 4}|
{\tt 5}|
{\tt 6}|
{\tt 7}|
{\tt 8}|
{\tt 9})
\\
\text{\tt [0-9]}&=\text{Ziffern von {\tt 0} bis {\tt 9}}
&&=
({\tt 0}|
{\tt 1}|
{\tt 2}|
{\tt 3}|
{\tt 4}|
{\tt 5}|
{\tt 6}|
{\tt 7}|
{\tt 8}|
{\tt 9})
\\
r{\text{\tt \{0,5\}}}&=\text{Ziffern 0 bis 5 Kopien von r}
&&=(|r|rr|rrr|rrrr|rrrrr)
\end{align*}
Damit kann man eine maximal sechsstellige Zahl ohne führende Nullen als
\[
\text{\tt [1-9][0-9]\{0,5\}}
\]
schreiben.
Dann wird noch maximal einer der Zusatzbuchstaben angehängt:
\[
(\varepsilon|{\tt U}|{\tt V}|{\tt Z})
\]
Alles zusammen begkommen wir den regulären Ausdruck
\[
\small
(
{\tt AG}|
{\tt AR}|
{\tt AI}|
{\tt BL}|
{\tt BS}|
{\tt BE}|
{\tt FR}|
{\tt GE}|
{\tt GL}|
{\tt GR}|
{\tt JU}|
{\tt LU}|
{\tt NE}|
{\tt NW}|
{\tt OW}|
{\tt SH}|
{\tt SZ}|
{\tt SO}|
{\tt SG}|
{\tt TI}|
{\tt TG}|
{\tt UR}|
{\tt VD}|
{\tt VS}|
{\tt ZG}|
{\tt ZH}|
{\tt A}|
{\tt P}|
{\tt M})
\text{\tt [1-9][0-9]\{0,5\}}
(\varepsilon|{\tt U}|{\tt V}|{\tt Z})
\]
Man könnte einwenden, dass es nach diesem regulären Ausdruck auch
Militärfahrzeuge gibt, die man mieten kann, doch in der Aufgabenstellung
wird dies nicht verboten.
\end{loesung}

\begin{bewertung}
Bewertet wird, ob der reguläre Ausdruck alle genannten Kriterien 
erfüllt, als da sind:
zweistellige Kantonskürzel ({\bf K}) 1 Punkt,
alternative Kürzel A/P/M ({\bf A}) 1 Punkt,
Zahl ein- bis sechsstellig ({\bf L}) 1 Punkt,
keine führende Nullen ({\bf N}) 1 Punkt,
Sondernutzung nur U/V/Z ({\bf S}) 1 Punkt,
Sondernutzugn optional ({\bf O}) 1 Punkt.
\end{bewertung}

