Gegeben ist das Alphabet $\Sigma=\{\texttt{a},\texttt{b},\texttt{c}\}$.
\begin{teilaufgaben}
\item
Finden Sie das Zustandsdiagramm eines Stackautomaten, der die Sprache
\[
L
=
\{ \texttt{a}^n \texttt{b}^l \texttt{c}^{n+l}\mid  n,l\ge 0\}
\]
akzeptiert.
\item
Finden Sie eine Grammatik für die Sprache $L$.
\item
Hat die von Ihnen in b) gefundene Grammatik Chomsky-Normalform?
\item 
Bestimmen Sie die Chomsky-Normalform der in b) gefundenen Grammatik.
\end{teilaufgaben}

\thema{Stackautomat}
\thema{Grammatik}
\thema{Chomsky-Noramlform}

\begin{hinweis}
Verwenden Sie, dass die Wörter der Sprache als
\[
w
=
\texttt{a}^n \texttt{b}^l \texttt{c}^{l+n}
=
\texttt{a}^n \texttt{b}^l \texttt{c}^{l} \texttt{c}^{n}
\]
geschrieben werden können.
\end{hinweis}

\begin{loesung}
\begin{teilaufgaben}
\item
Wir verwenden den Stack um die Zeichen \texttt{a} und \texttt{b}
im Input zu zählen und deren Summe mit der Anzahl der Zeichen \texttt{c}
zu vergleichen.
\[
\entrymodifiers={++[o][F]}
\xymatrix {
*+\txt{}\ar[r]
	&{q_0} \ar[r]^{\varepsilon,\varepsilon\to\texttt{\$}}
		&{q_1} \ar@(ur,ul)_{\texttt{a},\varepsilon\to\texttt{x}}
			\ar[r]^{\varepsilon,\varepsilon\to\varepsilon}
			&{q_2} \ar@(ur,ul)_{\texttt{b},\varepsilon\to\texttt{x}}
				\ar[r]^{\varepsilon,\varepsilon\to\varepsilon}
			&{q_3} \ar@(ur,ul)_{\texttt{c},\texttt{x}\to\varepsilon}
				\ar[r]^{\varepsilon,\texttt{\$}\to\varepsilon}
			&*++[o][F=]{q_4}
}
\]
\item
Wörter in dieser Sprache haben die Form
\[
w
=
\texttt{a}^n \texttt{b}^l \texttt{c}^{l+n}
=
\texttt{a}^n \texttt{b}^l \texttt{c}^{l} \texttt{c}^{n}
\]
Daraus kann man bereits ablesen, wie man die Grammatik aufbauen kann.
Man baut erst ein Wort der Form $\texttt{b}^l \texttt{c}^l$ und
``klammert'' es dann mit Paaren von \texttt{a} und \texttt{c} ein.
Für ein Wort der Form $\texttt{b}^l \texttt{c}^l$ kann man die von der
Sprache $\{\texttt{0}^n \texttt{1}^n\mid  n\le 0\}$ inspirierte Grammatik
\begin{align*}
B&\rightarrow \varepsilon \\
 &\rightarrow \texttt{b} B \texttt{c}
\end{align*}
verwenden.
Das Einklammern besorgen die Regeln
\begin{align*}
A&\rightarrow B \\
 &\rightarrow \texttt{a} A \texttt{c}
\end{align*}
Die vollständige Grammatik ist daher
\begin{align*}
A&\rightarrow B \\
 &\rightarrow \texttt{a} A \texttt{c} \\
B&\rightarrow \varepsilon \\
 &\rightarrow \texttt{b} B \texttt{c}
\end{align*}
\item
Die Grammatik hat nicht Chomsky-Normalform.
Die Startvariable kommt auf der rechten Seite vor, die $\varepsilon$-Regel
$B\to\varepsilon$ ist nicht zulässig, es gibt eine unit rule $A\to B$
und die verbleibenden Regeln haben drei statt höchstens zwei Zeichen
auf der rechten Seite.
\item
Um die Grammatik in Chomsky-Normalform zu bringen, ist erst die Startvariable
zu ersetzen, da die aktuelle Startvariable $A$ auf der rechten Seite vorkommt:
\begin{align*}
S&\rightarrow A\\
A&\rightarrow B \\
 &\rightarrow \texttt{a} A \texttt{c} \\
B&\rightarrow \varepsilon \\
 &\rightarrow \texttt{b} B \texttt{c}
\end{align*}
Im zweiten Schritt wird die $\varepsilon$-Regel $B\to\varepsilon$ entfernt:
\begin{align*}
S&\rightarrow A\\
A&\rightarrow B \mid  \varepsilon\\
 &\rightarrow \texttt{a} A \texttt{c} \\
B&\rightarrow \texttt{b} B \texttt{c} \mid  \texttt{b} \texttt{c}
\end{align*}
Da die neue $\varepsilon$-Regel $A\to\varepsilon$ entstanden ist, muss
dieser Schritt wiederholt werden:
\begin{align*}
S&\rightarrow A \mid  \varepsilon\\
A&\rightarrow B \\
 &\rightarrow \texttt{a} A \texttt{c} \mid  \texttt{ac}\\
B&\rightarrow \texttt{b} B \texttt{c} \mid  \texttt{b} \texttt{c}
\end{align*}
Die verbleibende $\varepsilon$-Regel $S\to\varepsilon$ ist erlaubt.

Im dritten Schritt werden Unit-Rules eliminiert.
Wir beginnen mit $S\to A$:
\begin{align*}
S&\rightarrow \varepsilon \mid  B \mid  \texttt{a}A\texttt{c} \mid \texttt{ac}\\
A&\rightarrow B \mid  \texttt{a} A \texttt{c} \mid  \texttt{ac}\\
B&\rightarrow \texttt{b} B \texttt{c} \mid  \texttt{b} \texttt{c}
\end{align*}
Es ist eine neue Unit-rule $S\to B$ enstanden, welche als nächste eliminiert
werden muss:
\begin{align*}
S&\rightarrow \varepsilon \mid  \texttt{a}A\texttt{c} \mid \texttt{ac}
\mid  \texttt{b} B \texttt{c} \mid  \texttt{b} \texttt{c}
\\
A&\rightarrow B \mid  \texttt{a} A \texttt{c} \mid  \texttt{ac}\\
B&\rightarrow \texttt{b} B \texttt{c} \mid  \texttt{b} \texttt{c}
\end{align*}
Es bleibt die Regel $A\to B$, welche wie folgt eliminiert werden kann:
\begin{align*}
S&\rightarrow \varepsilon \mid  \texttt{a}A\texttt{c} \mid \texttt{ac}
\mid  \texttt{b} B \texttt{c} \mid  \texttt{b} \texttt{c}
\\
A&\rightarrow \texttt{a} A \texttt{c} \mid  \texttt{ac} \mid 
 \texttt{b} B \texttt{c} \mid  \texttt{b} \texttt{c} \\
B&\rightarrow \texttt{b} B \texttt{c} \mid  \texttt{b} \texttt{c}
\end{align*}

Im vierten Schritt müssen jetzt noch die Terminalsymbole und die 
Dreier-Regeln aufgelöst werden:
\begin{align*}
S&\rightarrow \varepsilon \mid  XW \mid  UW \mid  YW \mid  VW \\
A&\rightarrow X W \mid  UW \mid  YW \mid  VW \\
B&\rightarrow YW \mid  VW \\
X&\rightarrow UA \\
Y&\rightarrow VB \\
U&\rightarrow \texttt{a} \\
V&\rightarrow \texttt{b} \\
W&\rightarrow \texttt{c}
\qedhere
\end{align*}
\end{teilaufgaben}
\end{loesung}

\begin{diskussion}
Teilaufgabe d) war nicht Teil der Prüfung.
\end{diskussion}

\begin{bewertung}
\begin{teilaufgaben}
\item Prinzip ``Zählen mit Hilfe des Stacks'' ({\bf P}) 1 Punkt,
korrekte Implementation ({\bf I}) 1 Punkt.
\item 
Schachtelung mit \texttt{a} bzw.~\texttt{b} und \texttt{c} ({\bf S}) 1 Punkt,
Schachtelung von $\texttt{b}^l\texttt{c}^l$ zwischen \texttt{a} und \texttt{c}
({\bf K}) 1 Punkt,
Terminierung mit $\varepsilon$ ({\bf L}) 1 Punkt.
\item
Begründung dafür, dass keine CNF vorliegt ({\bf C}) 1 Punkt.
\end{teilaufgaben}
\end{bewertung}
