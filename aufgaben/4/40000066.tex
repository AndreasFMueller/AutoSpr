Können Sie eine Grammatik für die Sprache
\[
L = \{
w\in\{\texttt{0},\texttt{1}\}^*
\;|\;
w =
\texttt{0}^k
\texttt{1}
\texttt{0}^l
\texttt{1}
\texttt{0}^k
\texttt{1}
\texttt{0}^l
\}
\]
angeben?

\begin{loesung}
Nein, dies ist nicht möglich, weil die Sprache nicht kontextfrei ist, wie
man mit dem Pumping-Lemma für kontextfreie Sprachen zeigen kann.
\begin{enumerate}
\item
Annahme: $L$ ist kontextfrei
\item
Nach dem Pumping Lemma für kontextfreie Sprachen gibt es die Pumping Length $N$.
\item
Wähle das Wort $w=
\texttt{0}^N
\texttt{1}
\texttt{0}^N
\texttt{1}
\texttt{0}^N
\texttt{1}
\texttt{0}^N$
\item
Nach dem Pumping Lemma gibt es eine Aufteilung von $w$ in fünf Teile
$w=uvxyz$ derart, dass $|vxy| \le N$ und $|vy|\ge 1$.
Ausserdem ist jedes gepumpte Wort $uv^kxy^kz\in L$.
\item
Da sich die Anzahl der Einsen beim Pumpen nicht ändern darf, müssen
$v$ und $y$ vollständig in einem Nullen-Block enthalten sein.
Daher kann sich nur die Anzahl der Nullen in höchstens zwei Nullen-Blöcken
ändern.
Die beiden Blöcke müssen wegen $|vxy|\le N$ ausserdem benachbart sein.

Zum ersten Nullen-Block gehört der dritte, der gleich viele Nullen
enthalten muss, zum zweiten gehört der vierte.
Wie auch immer die beiden Blöcke gewählt werden, ändert sich die Anzahl
Nullen in den Blöcken, aber nicht in den zugehörigen Blöcken.
Das gepumpte Wor tkann also nicht mehr in $L$ sein.
\item
Dieser Widerspruch zeigt, dass die Annahme, $L$ sei kontextfrei, nicht
haltbar ist.
Also ist $L$ nicht kontextfrei.
\qedhere
\end{enumerate}
\end{loesung}

\begin{bewertung}
Pumping Lemma und Annahme $L$ kontextfrei ({\bf PL}) 1 Punkt,
Pumping Length ({\bf N}) 1 Punkt,
Wahl eines Wortes ({\bf W}) 1 Punkt,
Unterteilung ({\bf U}) 1 Punkt,
Widerspruch beim Pumpen ({\bf P}) 1 Punkt,
Schlussfolgerung ({\bf S}) 1 Punkt.
\end{bewertung}

