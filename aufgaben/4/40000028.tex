Die Sprache Lisp hat eine etwas eigene Syntax, Larry Wall hat sie einst
mit ``Haferbrei vermischt mit abgeschnittenen Fingernägeln'' verglichen.
In Lisp gibt es Atome:
\begin{itemize}
\item numerische Atome, d.~h.~Zahlen: {\tt 1 2 3
-4  3.14  -7.5  6.02E+23}
\item Strings, also Zeichenfolgen in Anführungszeichen:
\begin{center}
{\tt \textquotedblright ein String\textquotedblright}
{\tt \textquotedblright \&]\$787?\textquotedblright}
{\tt \textquotedblright setq\textquotedblright}
{\tt \textquotedblright 12\textquotedblright}
\end{center}
\item Symbole, alle anderen Zeichenketten, die von Zahlen und Strings
untersscheidbar sind, und kein Klammern und Anführungszeichen enthalten.
\end{itemize}
Aus den Atomen können jetzt Listen kombiniert werden. Eine Liste besteht
aus einer öffnenden Klammer, einer Folge von Listenelementen und
einer schliessenden Klammer. Listenelemente sind entweder Atome oder Listen.
Listen können auch leer sein, ebenso Strings.
Ein typisches Lisp-Programm ist
\begin{verbatim}
(defun print-quadratzahlen (x)
  (cond ((plusp x)
         (print (* x x))
         (print-quadratzahlen (- x 1)))))
\end{verbatim}
\begin{teilaufgaben}
\item
Formulieren Sie eine kontextfreie Grammatik für Lisp.
\item
Falls Ihre Grammatik nicht Chomsky-Normalform hat: markieren Sie
alle Regeln, die der Normalform widersprechen.
\end{teilaufgaben}

\begin{hinweis}
Zur Vereinfachung gehen Sie davon aus, dass Zahlen positive natürliche
Zahlen sind, dass Strings nur Kleinbuchstaben und Ziffern enthalten, und
dass ein Symbol eine Folge von Kleinbuchstaben (keine Ziffern) ist.
\end{hinweis}

\begin{loesung}
\begin{teilaufgaben}
\item Die folgende Grammatik folgt ziemlich genau der obigen Beschreibung
der Sprache. Man beachte, dass die beiden $\varepsilon$-Regeln
erlauben, dass Listen und Strings leer sind.
\begin{align*}
\text{liste}&\rightarrow {\color{blue}
{\text{'{\tt (}'}}\;
\text{listenelemente}\;
{\text{'{\tt )}'}}}
\\
\text{listenelemente}&\rightarrow {\color{green}\varepsilon} \;|\; \text{listenelemente}\;\text{element}
\\
\text{element}&\rightarrow {\color{red}\text{liste}} \;|\; {\color{red}\text{atom}}
\\
\text{atom}&\rightarrow {\color{red}\text{zahl}}\;|\;{\color{red}\text{string}}\;|\;{\color{red}\text{symbol}}
\\
\text{zahl}&\rightarrow {\color{red}\text{ziffer}}\;|\;\text{zahl}\;\text{ziffer}
\\
\text{symbol}&\rightarrow {\color{red}\text{buchstabe}} \;|\; \text{symbol}\;\text{buchstabe}
\\
\text{string}&\rightarrow {\color{blue}
\text{'{\tt\textquotedblright}'}\; \text{stringinhalt}\;
\text{'{\tt\textquotedblright}'}}
\\
\text{stringinhalt}&\rightarrow {\color{green}\varepsilon} \;|\; \text{stringinhalt}\;\text{zeichen}
\\
\text{ziffer}&\rightarrow
{\text{'{\tt 0}'}} \;|\;
{\text{'{\tt 1}'}} \;|\;
{\text{'{\tt 2}'}} \;|\;
\dots \;|\;
{\text{'{\tt 9}'}}
\\
\text{buchstabe}&\rightarrow
{\text{'{\tt a}'}} \;|\;
{\text{'{\tt b}'}} \;|\;
\dots \;|\;
{\text{'{\tt z}'}}
\\
\text{zeichen}&\rightarrow {\color{red}\text{ziffer}}\; |\; {\color{red}\text{buchstabe}}
\end{align*}
\item
In der Grammatik oben sind
Unit rules
{\color{red}
rot},
Regeln mit mehr als zwei Symbolen auf der rechten Seite
{\color{blue}
blau},
und
$\varepsilon$-Regeln, die nicht von der Startvariablen ausgehen,
{\color{green}
grün} markiert.
\qedhere
\end{teilaufgaben}
\end{loesung}

