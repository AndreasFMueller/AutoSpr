Sei $\Sigma=\{\texttt{a},\texttt{b},\dots,\texttt{z}\}$ das Alphabet
bestehend aus den Kleinbuchstaben und 
\[
L=\{ w\;|\; \text{$w=uu$ mit $u\in\Sigma^*$}\}
\]
die Sprache aus Wörtern, die als die Verdoppelung eines beliebigen
Wortes entstehen, also zum Beispiel
\[
\texttt{mama},\quad
\texttt{papa},\quad
\texttt{blubblub},\quad
\texttt{poipoi}
\]
Kann ein Stackautomat diese Sprache erkennen?

\thema{Stackautomat}
\themaL{Pumping Lemma fur kontextfreie Sprachen}{Pumping Lemma für kontextfreie Sprachen}

\begin{loesung}
Ein Stackautomat kann kontextfreie Sprachen, diese Sprache ist aber nicht
kontextfrei, wie man mit dem Pumping-Lemma für kontextfreie
Sprachen zeigen kann.
\ding{182}~%
Dazu nehmen wir an, $L$ sei kontextfrei.
\ding{183}~%
Dann müsste es eine Zahl $N$, die Pumping Length, geben, mit der Eigenschaft,
dass Wörter mit mindestens Länge $N$ die Pumpeigenschaft haben.
\ding{184}~%
Wir wählen das Wort
\[
w=
\texttt{a}^{2N}
\texttt{b}^{2N}
\texttt{a}^{2N}
\texttt{b}^{2N}
\in L.
\]
\ding{185}~%
Dann gibt es eine Zerlegung des Wortes $w$ in fünf Teile $w=uvxyz$ mit
$|vxy|\le N$ und $|vy|>0$, so dass $uv^kxy^kz\in L$ für jedes $k\ge 0$.

\ding{186}~%
Pumpt man ab, d.~h.~setzt man $k=0$, dann werden zwar einige \texttt{a}
und \texttt{b} entfernt, aber weil wir die \texttt{a}- und \texttt{b}-Blöcke
so lange gewählt haben, besteht das Wort immer noch nur aus
vier Blöcken von \texttt{a} bzw.~\texttt{b}, ein Block kann beim
Abpumpen nicht ganz verschwinden (wegen dem Exponenenten $2N$ ist er
länger als die Teile $v$ und $y$ sein können).
Wenn das Wort immer noch in $L$ sein soll, dann müssen die \texttt{a}-
bzw.~\texttt{b}-Teile gleich lang sein, beim Abpumpen müssen also
gleich viele Buchstaben aus jedem Teil entfernt worden sein.
Die beiden \texttt{a}-Teile sind aber zu weit auseinander, Abpumpen
kann nur aus einem Buchstaben entfernen, das Gleiche gilt für \texttt{b}.
\ding{187}~%
Dieser Widerspruch zeigt, dass die Annahme, $L$ sei kontextfrei, nicht
zu halten ist.
\end{loesung}

\begin{bewertung}
Pumping-Lemma, Annahme Sprache ist kontextfrei ({\bf K}) 1 Punkt,
Pumping-Length ({\bf N}) 1 Punkt,
Konstruktion eines Wortes in $L$ unter Verwendung von $N$ ({\bf W}) 1 Punkt,
Unterteilung des Wortes ({\bf U}) 1 Punkt,
Nachweis, dass Pumpeigenschaft verletzt sein muss ({\bf P}) 1 Punkt,
Schlussfolgerung, dass $L$ nicht regulär sein kann ({\bf S}) 1 Punkt.
\end{bewertung}




