Auf der Website des {\em Deutschen Forschungszentrums für Künstliche
Intelligenz} wurde eine Präsentation über kontextfreie 
Grammatiken gefunden, die die folgende Grammatik enthält
\begin{align*}
E&\to S\,\texttt{+}\,S\\
 &\to S\,\texttt{*}\,S\\
S&\to \texttt{(}\, E \, \texttt{)} \\
 &\to I\\
I&\to D\\ 
 &\to I D\\
D&\to
\texttt{0}
\,|\,
\texttt{1}
\,|\,
\texttt{2}
\,|\,
\texttt{3}
\,|\,
\texttt{4}
\,|\,
\texttt{5}
\,|\,
\texttt{6}
\,|\,
\texttt{7}
\,|\,
\texttt{8}
\,|\,
\texttt{9}
\end{align*}
mit den Ziffern, Klammern und Operationszeichen als Terminalsymbolen,
$\Sigma = \{\texttt{0},\texttt{1},\dots,\texttt{9},\texttt{(},\texttt{)},\texttt{+},\texttt{*}\}$.
\begin{teilaufgaben}
\item
Bringen Sie die Grammatik in Chomsky-Normalform.
\item
Welche der folgenden Zeichenketten können von dieser Grammatik
erzeugt werden:
\begin{center}
\texttt{1}
\qquad
\texttt{1+1}
\qquad
\texttt{1+1*1}
\qquad
\texttt{1+(1*1)}
\qquad
\texttt{1+1+1}
\qquad
\texttt{1+(1)+1}
\qquad
\texttt{(1+1+1)}
\end{center}
\item 
Ist die Grammatik eindeutig?
\end{teilaufgaben}

\begin{loesung}
\begin{teilaufgaben}
\item 
Zunächst muss sichergestellt werden, dass die Startvariable auf der rechten
Seite nicht vorkommt, dazu wird die Regel $S_0\to E$ hinzugefügt:
\begin{align*}
S_0&\to E \\
E&\to S\texttt{+}S \;|\; S\texttt{*}S \\
S&\to \texttt{(}E\texttt{)} \;|\; I \\
I&\to D \;|\; I D \\
D&\to
\texttt{0}
\;|\;
\texttt{1}
\;|\;
\texttt{2}
\;|\;
\texttt{3}
\;|\;
\texttt{4}
\;|\;
\texttt{5}
\;|\;
\texttt{6}
\;|\;
\texttt{7}
\;|\;
\texttt{8}
\;|\;
\texttt{9}
\end{align*}
Das leere Wort $\varepsilon$ kommt nirgends vor, also ist diesbezüglich
nichts zu unternehmen.

In der Grammatik kommen drei Unit-Rules vor, wir entfernen zunächst die
Regel $S_0\to E$ und erhalten
\begin{align*}
S_0&\to S\texttt{+}S \;|\; S\texttt{*}S \\
E&\to S\texttt{+}S \;|\; S\texttt{*}S \\
S&\to \texttt{(}E\texttt{)} \;|\; I \\
I&\to D \;|\; I D \\
D&\to
\texttt{0}
\;|\;
\texttt{1}
\;|\;
\texttt{2}
\;|\;
\texttt{3}
\;|\;
\texttt{4}
\;|\;
\texttt{5}
\;|\;
\texttt{6}
\;|\;
\texttt{7}
\;|\;
\texttt{8}
\;|\;
\texttt{9}
\end{align*}
Jetzt wird die Regel $S\to I$ entfernt, was
\begin{align*}
S_0&\to S\texttt{+}S \;|\; S\texttt{*}S \\
E&\to S\texttt{+}S \;|\; S\texttt{*}S \\
S&\to \texttt{(}E\texttt{)} \;|\; I \\
I&\to 
\texttt{0}
\;|\;
\texttt{1}
\;|\;
\texttt{2}
\;|\;
\texttt{3}
\;|\;
\texttt{4}
\;|\;
\texttt{5}
\;|\;
\texttt{6}
\;|\;
\texttt{7}
\;|\;
\texttt{8}
\;|\;
\texttt{9}
\;|\; I D \\
D&\to
\texttt{0}
\;|\;
\texttt{1}
\;|\;
\texttt{2}
\;|\;
\texttt{3}
\;|\;
\texttt{4}
\;|\;
\texttt{5}
\;|\;
\texttt{6}
\;|\;
\texttt{7}
\;|\;
\texttt{8}
\;|\;
\texttt{9}
\end{align*}
ergibt.
Schliesslich muss noch die Regel $S\to I$ entfernt werden, so dass noch
\begin{align*}
S_0&\to S\texttt{+}S \;|\; S\texttt{*}S \\
E&\to S\texttt{+}S \;|\; S\texttt{*}S \\
S&\to \texttt{(}E\texttt{)} \;|\;
\texttt{0}
\;|\;
\texttt{1}
\;|\;
\texttt{2}
\;|\;
\texttt{3}
\;|\;
\texttt{4}
\;|\;
\texttt{5}
\;|\;
\texttt{6}
\;|\;
\texttt{7}
\;|\;
\texttt{8}
\;|\;
\texttt{9}
\;|\; I D \\
I&\to 
\texttt{0}
\;|\;
\texttt{1}
\;|\;
\texttt{2}
\;|\;
\texttt{3}
\;|\;
\texttt{4}
\;|\;
\texttt{5}
\;|\;
\texttt{6}
\;|\;
\texttt{7}
\;|\;
\texttt{8}
\;|\;
\texttt{9}
\;|\; I D \\
D&\to
\texttt{0}
\;|\;
\texttt{1}
\;|\;
\texttt{2}
\;|\;
\texttt{3}
\;|\;
\texttt{4}
\;|\;
\texttt{5}
\;|\;
\texttt{6}
\;|\;
\texttt{7}
\;|\;
\texttt{8}
\;|\;
\texttt{9}
\end{align*}
bleibt.

Im letzten Schritt müssen die ``Dreier''-Regeln entfernt werden
und die darin enthaltenen Terminalsymbole in separate Regeln ausgelagert
werden.
Die Ersetzung der Terminalsymbole ergibt:
\begin{align*}
S_0&\to SPS \;|\; SMS \\
E&\to SPS \;|\; SMS \\
S&\to AEZ \;|\;
\texttt{0}
\;|\;
\texttt{1}
\;|\;
\texttt{2}
\;|\;
\texttt{3}
\;|\;
\texttt{4}
\;|\;
\texttt{5}
\;|\;
\texttt{6}
\;|\;
\texttt{7}
\;|\;
\texttt{8}
\;|\;
\texttt{9}
\;|\; I D \\
I&\to 
\texttt{0}
\;|\;
\texttt{1}
\;|\;
\texttt{2}
\;|\;
\texttt{3}
\;|\;
\texttt{4}
\;|\;
\texttt{5}
\;|\;
\texttt{6}
\;|\;
\texttt{7}
\;|\;
\texttt{8}
\;|\;
\texttt{9}
\;|\; I D \\
D&\to
\texttt{0}
\;|\;
\texttt{1}
\;|\;
\texttt{2}
\;|\;
\texttt{3}
\;|\;
\texttt{4}
\;|\;
\texttt{5}
\;|\;
\texttt{6}
\;|\;
\texttt{7}
\;|\;
\texttt{8}
\;|\;
\texttt{9}
\\
M&\to \texttt{*}\\
P&\to \texttt{+}\\
A&\to \texttt{(}\\
Z&\to \texttt{)}
\end{align*}
Die Aufspaltung der Dreier-Regeln schliesslich ergibt
\begin{align*}
S_0&\to SU_1 \;|\; SU_2 \\
U_1&\to PS \\
U_2&\to MS \\
E&\to SU_1 \;|\; SU_2 \\
S&\to AU_3 \;|\; 
\texttt{0}
\;|\;
\texttt{1}
\;|\;
\texttt{2}
\;|\;
\texttt{3}
\;|\;
\texttt{4}
\;|\;
\texttt{5}
\;|\;
\texttt{6}
\;|\;
\texttt{7}
\;|\;
\texttt{8}
\;|\;
\texttt{9}
\;|\; I D \\
U_3&\to EZ\\
I&\to 
\texttt{0}
\;|\;
\texttt{1}
\;|\;
\texttt{2}
\;|\;
\texttt{3}
\;|\;
\texttt{4}
\;|\;
\texttt{5}
\;|\;
\texttt{6}
\;|\;
\texttt{7}
\;|\;
\texttt{8}
\;|\;
\texttt{9}
\;|\; I D \\
D&\to
\texttt{0}
\;|\;
\texttt{1}
\;|\;
\texttt{2}
\;|\;
\texttt{3}
\;|\;
\texttt{4}
\;|\;
\texttt{5}
\;|\;
\texttt{6}
\;|\;
\texttt{7}
\;|\;
\texttt{8}
\;|\;
\texttt{9}
\\
M&\to \texttt{*}\\
P&\to \texttt{+}\\
A&\to \texttt{(}\\
Z&\to \texttt{)}
\end{align*}
Damit ist Chomsky-Normalform erreicht.
\item
Innerhalb einer Klammer muss sich immer ein Ausdruck mit genau einem
nicht eingeklammerten Operationszeichen befinden, was die letzten zwei
Ausdrücke ausschliesst.
Schachtelung ist nur möglich, wenn Ausdrücke auch eingeklammert werden,
was den drittletzten und den dritten Ausdruck auschliesst.
Es bleiben also nur der zweite und vierte Ausdruck, die von der
Grammatik erzeugt werden können.
\item
Die Grammatik ist eindeutig, weil Schachtelung immer nur mit Klammern
möglich ist.
\qedhere
\end{teilaufgaben}
\end{loesung}

\begin{diskussion}
Möglicherweise hat das {\em Deutsche Forschungszentrumg für künstliche
Intelligenz} gemeint, $S$ sei die Startvariable.
Dann entsteht etwas, leicht weniger unsinniges.
Auch wenn man $S$ als Startvariable nimmt, ist zum Beispiel der
Operator-Vorrang nicht geregelt, Punkt vor Strich wird nicht
eingehalten.
Keine wirklich intelligente Grammatik, die das Forschungszentrum
für künstliche Intelligenz da produziert hat.
Vielleicht von einer KI produziert?
\end{diskussion}

\begin{bewertung}
\begin{teilaufgaben}
\item Startvariable ({\bf S}) 1 Punkt,
Unit Rules ({\bf U}) 1 Punkt,
Ersetzung der Terminalsymbole ({\bf T}) 1 Punkt,
lange Regeln ({\bf L}) 1 Punkt.
\item
Akzeptable Ausdrücke ({\bf A}) 1 Punkt.
\item
Eindeutigkeit ({\bf E}) 1 Punkt.
\end{teilaufgaben}
\end{bewertung}




