Ist die Sprache
\[
L=\{\texttt{0}^i\texttt{1}^j\texttt{0}^i\;|\; j>0\wedge i\ge 0\}
\]
kontextfrei?
Wenn ja, geben Sie eine Grammatik in Chomsky-Normalform dafür an.

\thema{kontextfrei}
\thema{kontextfreie Grammatik}
\thema{Chomsky-Normalform}

\begin{loesung}
Die Sprache ist kontextfrei, denn die die Anzahl der Einsen ist unabhängig
von der Anzahl der Nullen. 
Man kann zum Beispiel den folgenden fast\footnote{Der Stackautomat wird
deterministisch, wenn man fehlende Übergängen zu einem ``Fehlerzustand''
ergänzt.} deterministischen Stackautomaten
verwenden, um die Sprache zu akzeptieren:
\begin{center}
\begin{tikzpicture}[>=latex,thick]

\coordinate (start) at (-2,2);
\coordinate (q0) at (0,2);
\coordinate (q1) at (3,2);
\coordinate (q2) at (3,0);
\coordinate (q3) at (3,-2);
\coordinate (q4) at (0,-2);

\draw[->,shorten <= 0.3cm, shorten >= 0.3cm] (start) -- (q0);

\draw[->,shorten <= 0.3cm, shorten >= 0.3cm] (q0) -- (q1);
\draw[->,shorten <= 0.3cm, shorten >= 0.3cm] (q1) -- (q2);
\draw[->,shorten <= 0.3cm, shorten >= 0.3cm] (q2) -- (q3);
\draw[->,shorten <= 0.3cm, shorten >= 0.3cm] (q3) -- (q4);

\draw (q0) circle[radius=0.3];
\node at (q0) {$q_0$};

\node at ($0.5*(q0)+0.5*(q1)$) [above] {$\varepsilon,\varepsilon\to\texttt{\$}$};

\draw (q1) circle[radius=0.3];
\node at (q1) {$q_1$};

\draw[->,shorten >= 0.3cm,shorten <= 0.3cm]
	(q1) to[out=0,in=90,distance=1.3cm] (q1);
\node at ($(q1)+(0.5,0.6)$) [right] {$\texttt{0},\varepsilon\to\texttt{0}$};

\node at ($0.5*(q1)+0.5*(q2)$) [right] {$\texttt{1},\varepsilon\to\varepsilon$};

\draw (q2) circle[radius=0.3];
\node at (q2) {$q_2$};

\draw[->,shorten >= 0.3cm,shorten <= 0.3cm]
	(q2) to[out=45,in=-45,distance=1.3cm] (q2);
\node at ($(q2)+(0.7,0)$) [right] {$\texttt{1},\varepsilon\to\varepsilon$};

\node at ($0.5*(q2)+0.5*(q3)$) [right] {$\texttt{0},\texttt{0}\to\varepsilon$};

\draw (q3) circle[radius=0.3];
\node at (q3) {$q_3$};
\draw[->,shorten >= 0.3cm,shorten <= 0.3cm]
	(q3) to[out=0,in=-90,distance=1.3cm] (q3);
\node at ($(q3)+(0.5,-0.6)$) [right] {$\varepsilon,\texttt{0}\to\varepsilon$};

\draw[->,shorten >= 0.3cm,shorten <= 0.3cm] (q2) -- (q4);
\node at ($0.5*(q3)+0.5*(q4)$) [below] {$\varepsilon,\texttt{\$}\to\varepsilon$};
\node at ($0.7*(q2)+0.3*(q4)+(0,0.1)$) [left] {$\varepsilon,\texttt{\$}\to\varepsilon$};

\draw (q4) circle[radius=0.3];
\draw (q4) circle[radius=0.25];
\node at (q4) {$q_4$};


\end{tikzpicture}
\end{center}
Es ist aber gar nicht nötig, diesen Stackautomaten zu entwickeln, denn 
die Existenz einer Grammatik zeigt auch, dass die Sprache kontextfrei ist.

Die Grammatik muss Folgen von Einsen beliebiger Länge zwischen Paaren
von Null einschachteln.
Die Grammatik
\begin{align*}
S&\to \texttt{0}S\texttt{0}
\\
 &\to E
\\
E&\to E\texttt{1}\\
 &\to \texttt{1}
\end{align*}
schafft das, ist aber nicht in Chomsky-Normalform.
Wir wandeln Sie daher in Chomsky-Normalform um:

Neue Startvariable:
\begin{align*}
S_0&\to S
\\
S&\to \texttt{0}S\texttt{0} \;|\; E
\\
E&\to E\texttt{1} \;|\; \texttt{1}
\end{align*}
Es kommen keine $\varepsilon$-Regeln vor, um die wir uns kümmern müssten,
wir können daher gleich zur Entfernung der Unit-Rules übergehen.
Entfernung der Regel $S\to E$:
\begin{align*}
S_0&\to S
\\
S&\to \texttt{0}S\texttt{0} \;|\; E\texttt{1} \;|\; \texttt{1}
\\
E&\to E\texttt{1} \;|\;  \texttt{1}
\end{align*}
Entfernung der Unit-Rue $S_0\to S$:
\begin{align*}
S_0&\to \texttt{0}S\texttt{0} \;|\; E\texttt{1} \;|\; \texttt{1}
\\
S&\to \texttt{0}S\texttt{0} \;|\; E\texttt{1} \;|\; \texttt{1}
\\
E&\to E\texttt{1} \;|\; \texttt{1}
\end{align*}
Jetzt müssen die Dreierregeln und die Terminalsymbole noch aufgeräumt werden:
\begin{align*}
S_0&\to ZA \;|\; EU \;|\; \texttt{1}
\\
S&\to ZA \;|\; EU \;|\; \texttt{1}
\\
A&\to SZ
\\
E&\to EU \;|\; \texttt{1}
\\
U&\to \texttt{1}
\\
Z&\to \texttt{0}
\end{align*}
Damit ist Chomsky-Normalform erreicht.
\end{loesung}

\begin{bewertung}
Grammatik (3 Punkte) korrekt formuliert ({\bf G}) 1 Punkt,
Schachtelung von \texttt{0} ({\bf A}) 1 Punkt,
Mindestens eine \texttt{1} ({\bf E}) 1 Punkt.
CNF Umwandlung (3 Punkte): neue Startvariable ({\bf S}) 1 Punkt,
unit rules ({\bf U}) 1 Punkt,
Dreierregeln und Terminalsymbole ({\bf T}) 1 Punkt.
\end{bewertung}
