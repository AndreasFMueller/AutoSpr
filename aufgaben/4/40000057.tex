Betrachten Sie folgende zwei Sprachen über dem Alphabet $\Sigma = \{
\texttt{a},
\texttt{b},
\texttt{c},
\texttt{d}\}$:
\begin{align*}
L_1&=\{\;\texttt{a}^n\texttt{b}^m\texttt{c}^m\texttt{d}^n
\;|\;
m,n\ge 0\}
\\
\text{und}
\qquad
L_2&=\{w\in\Sigma^*
\;|\;
|w|_{\texttt{a}} = |w|_{\texttt{d}}
\wedge
|w|_{\texttt{b}} = |w|_{\texttt{c}}
\}
\end{align*}
Es ist klar, dass $L_1\subset L_2$.
Sind diese Sprachen kontextfrei?

\thema{kontextfrei}
\themaL{Pumping Lemma fur kontextfreie Sprachen}{Pumping Lemma für kontextfreie Sprachen}
\thema{Grammatik}

\begin{loesung}
Die Sprache $L_1$ hat die kontextfreie Grammatik
\begin{align*}
S&\to R                             \\
 &\to \texttt{a}\;S\;\texttt{d}     \\
R&\to\varepsilon                    \\
 &\to\texttt{b}\;R\;\texttt{c}
\end{align*}
somit ist $L_1$ kontextfrei.

Die Sprache $L_2$ ist nicht kontextfrei, wie man mit dem Pumping-Lemma
beweisen kann.
\begin{enumerate}
\item
Annahme: $L_2$ ist kontextfrei.
\item
Nach dem Pumping-Lemma gibt es die Pumping Length $N$.
\item
Wir konstruieren das Wort
$w=
\texttt{a}^N
\texttt{b}^N
\texttt{d}^N
\texttt{c}^N$.
Wegen $
|w|_{\texttt{a}}
=
|w|_{\texttt{b}}
=
|w|_{\texttt{d}}
=
|w|_{\texttt{c}}
=N$ ist $w\in L_2$.
\item
Wegen $|w|=4N$ ist es das Wort lang genug, dass das Pumping Lemma darauf
anwendbar ist.
Es muss also eine Unterteilung $w=
{\color{blue}u}
{\color{red}v}
{\color{green}x}
{\color{red}y}
{\color{blue}z}
$ geben derart, dass $|vxy|\le N$ ist und derart, dass alle aufgepumpten
Wörter $uv^kxy^kz\in L_2$ sind für $k\ge 0$.
\item
Der Teil $vxy$ kann höchstens zwei verschiedene Arten von Buchstaben umfassen,
und zwar nur
\begin{enumerate}
\item
\texttt{a} und \texttt{b}
\item
\texttt{b} und \texttt{d}
\item
\texttt{d} und \texttt{c}
\end{enumerate}
In jedem dieser Fälle wird beim Pumpen die Anzahl mindestens eines Buchstabens
verändert,
aber die Anzahl des anderen Buchstabens, von dem es im Wort gleich
viele geben sollte, wird nicht verändert.
Damit ist das Wort $w$ nach dem Pumpen mit $k\ne 1$ nicht mehr in $L_2$
\item
Dieser Widerspruch zeigt, dass die Sprache $L_2$ nicht kontextfrei sein kann.
\qedhere
\end{enumerate}
\end{loesung}

\begin{bewertung}
Sprache $L_1$ kontextfrei ({\bf L}$\mathstrut_1$) 1 Punkt,
Grammatik für Sprache $L_1$ ({\bf G}) 1 Punkt,
Beispielwort ({\bf W}) 1 Punkt,
Aufteilung ({\bf A}) 1 Punkt,
Widerspruch beim Pumpen ({\bf P}) 1 Punkt,
Schlussfolgerung ({\bf S}) 1 Punkt.
\end{bewertung}


