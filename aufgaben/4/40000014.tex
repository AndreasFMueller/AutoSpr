Stellen Sie eine kontextfreie Gramatik auf, die die Sprache der ``wachsenden''
Wörter  (siehe Aufgabe~\ref{30000016}) erzeugt.
% Geben Sie auch eine Grammatik in Chomsky-Normal-Form an.

\begin{loesung}
Wachsende Wörter bestehen aus einzelnen Wörtern der Form
$w={\tt 0}^k{\tt 1}^{k+s}$.  Das Wort $w$ kann man erzeugen,
indem man zuerst das leere Wort in {\tt 0} und {\tt 1}
``einschachtelt'', und dann noch eine Anzahl von Einsen anhängt.
Damit haben wir aber in informeller Form alle Produktionsregeln
für Wörter der Sprache $L$ zusammengefasst.

Das Symbol $W$ stehe für Wörter wie $w$, dann kann man beliebige Wörter der
Sprache zusammensetzen, in dem man solchen Wörter weitere anhängt:
\begin{align*}
S_0&\to \varepsilon\\
   &\to S\\
S&\to A\\
 &\to SA
\end{align*}
Die Teile $A$ entstehen wir folgt:
\begin{align*}
A&\to W\\
 &\to A{\tt 1}\\
W&\to {\tt01}\\
 &\to {\tt 0}W{\tt 1}\\
\end{align*}
Diese Grammatik hat übrigens nicht Chomsky-Normalform, es gibt Unit rules,
Regeln mit drei Symbolen auf der rechten Seite. Wir eliminieren zunächst
die Unit rules.

Dies war nicht verlangt, aber man kann Chomsky-Normalform mit dem
üblichen Algorithmus erreichen: Elimination von $S_0\to S$:
\begin{align*}
S_0&\to \varepsilon\;|\;A\;|\;SA\\
S  &\to A\;|\;SA\\
A  &\to W\;|\;A{\tt 1}\\
W  &\to {\tt 01}\;|\; {\tt 0}W{\tt 1}
\end{align*}
Elimination von $S_0\to A$:
\begin{align*}
S_0&\to \varepsilon\;|\;SA\;|\;W\;|\;A{\tt 1}\\
S  &\to A\;|\;SA\\
A  &\to W\;|\;A{\tt 1}\\
W  &\to {\tt 01}\;|\; {\tt 0}W{\tt 1}
\end{align*}
Elimination von $S_0\to W$:
\begin{align*}
S_0&\to \varepsilon\;|\;SA\;|\;A{\tt 1}\;|\;{\tt 01}\;|\;{\tt 0}W{\tt 1}\\
S  &\to A\;|\;SA\\
A  &\to W\;|\;A{\tt 1}\\
W  &\to {\tt 01}\;|\; {\tt 0}W{\tt 1}
\end{align*}
Elimination von $S\to A$:
\begin{align*}
S_0&\to \varepsilon\;|\;SA\;|\;A{\tt 1}\;|\;{\tt 01}\;|\;{\tt 0}W{\tt 1}\\
S  &\to SA\;|\;W\;|\;A{\tt 1}\\
A  &\to W\;|\;A{\tt 1}\\
W  &\to {\tt 01}\;|\; {\tt 0}W{\tt 1}
\end{align*}
Elimination von $S\to W$:
\begin{align*}
S_0&\to \varepsilon\;|\;SA\;|\;A{\tt 1}\;|\;{\tt 01}\;|\;{\tt 0}W{\tt 1}\\
S  &\to SA\;|\;A{\tt 1}\;|\;{\tt 01}\;|\;{\tt 0}W{\tt 1}\\
A  &\to W\;|\;A{\tt 1}\\
W  &\to {\tt 01}\;|\; {\tt 0}W{\tt 1}
\end{align*}
Elimination von $A\to W$:
\begin{align*}
S_0&\to \varepsilon\;|\;SA\;|\;A{\tt 1}\;|\;{\tt 01}\;|\;{\tt 0}W{\tt 1}\\
S  &\to SA\;|\;A{\tt 1}\;|\;{\tt 01}\;|\;{\tt 0}W{\tt 1}\\
A  &\to A{\tt 1}\;|\;{\tt 01}\;|\;{\tt 0}W{\tt 1}\\
W  &\to {\tt 01}\;|\; {\tt 0}W{\tt 1}
\end{align*}
Auflösung der Regeln mit rechter Seite ${\tt 0}W{\tt 1}$:
\begin{align*}
S_0&\to \varepsilon\;|\;SA\;|\;A{\tt 1}\;|\;{\tt 01}\;|\;{\tt 0}B\\
B  &\to W{\tt 1}\\
S  &\to SA\;|\;A{\tt 1}\;|\;{\tt 01}\;|\;{\tt 0}B\\
A  &\to A{\tt 1}\;|\;{\tt 01}\;|\;{\tt 0}B\\
W  &\to {\tt 01}\;|\; {\tt 0}B
\end{align*}
Entfernung der Terminale {\tt 0} und {\tt 1} aus Regeln mit zwei Symbolen
auf der rechten Seite:
\begin{align*}
S_0&\to \varepsilon\;|\;SA\;|\;AE\;|\;NE\;|\;NB\\
B  &\to BE\\
S  &\to SA\;|\;AE\;|\;NE\;|\;NB\\
A  &\to AE\;|\;NE\;|\;NB\\
W  &\to NE\;|\; NB\\
N  &\to {\tt 0}\\
E  &\to {\tt 1}
\qedhere
\end{align*}
\end{loesung}
