In der Vorlesung wurde die Sprache LOOP besprochen. Zeigen Sie, dass
die Sprache kontextfrei ist, indem Sie eine
kontextfreie Grammatik für diese Sprache formulieren.
Hat Ihre Grammatik Chomsky-Normalform?

\thema{Grammatik}
\thema{kontextfrei}
\thema{Chomsky Normalform}

\begin{loesung}
\begin{align*}
\text{\textsl{loop-programm}}&\to\text{\textsl{anweisungsfolge}}\\
&\to\varepsilon\\
\text{\textsl{anweisungsfolge}}
&\to\text{\textsl{anweisungsfolge}}\;\text{'{\tt ;}'}\;
\text{\textsl{anweisung}}
\\
\text{\textsl{anweisung}}
&\to\text{\textsl{zuweisung}}
\\
&\to\text{\textsl{loop-anweisung}}
\\
\text{\textsl{zuweisung}}
&\to
\text{\textsl{variable}}\; \text{'{\tt :}'}\, \text{'{\tt =}'}\;
\textsl{variable}\;\text{\textsl{operator}}\;\text{\textsl{konstante}}
\\
\text{\textsl{operator}}
&\to\text{'{\tt +}'}\;|\;\text{'{\tt -}'}
\\
\text{\textsl{konstante}}&\to\text{\textsl{ziffer}}\\
&\to\text{\textsl{konstante}}\;\text{\textsl{ziffer}}
\\
\text{\textsl{ziffer}}&\to 
\text{'{\tt 0}'}\;|\;
\text{'{\tt 1}'}\;|\;
\text{'{\tt 2}'}\;|\;
\text{'{\tt 3}'}\;|\;
\text{'{\tt 4}'}\;|\;
\text{'{\tt 5}'}\;|\;
\text{'{\tt 6}'}\;|\;
\text{'{\tt 7}'}\;|\;
\text{'{\tt 8}'}\;|\;
\text{'{\tt 9}'}\\
\text{\textsl{variable}}&\to \text{'{\tt x}'}\;\text{\textsl{konstante}}\\
\text{\textsl{loop-anweisung}}&\to
\text{\textsl{loop}}\;
\text{\textsl{variable}}\;
\text{\textsl{do}}\;
\text{\textsl{anweisungsfolge}}\;
\text{\textsl{end}}\\
\text{\textsl{loop}}&\to \text{'{\tt L}'}\; \text{'{\tt O}'}\; \text{'{\tt O}'}\; \text{'{\tt P}'}\\
\text{\textsl{end}}&\to \text{'{\tt E}'}\; \text{'{\tt N}'}\; \text{'{\tt D}'}\\
\text{\textsl{do}}&\to \text{'{\tt D}'}\; \text{'{\tt O}'} \\
\end{align*}
Man liest aus der Grammatik auch ab, dass die in anderen Sprachen als
Trennzeichen nötigen Leerzeichen überflüssig sind, obwohl sie natürlich
helfen, ein Programm lesbar zu machen.

Diese Grammatik hat natürlich nicht Chomsky-Normalform
\begin{itemize}
\item Es gibt mehrere Unit-Rules
\item In mehreren Regeln kommen auf der rechten Seite mehr als
zwei Elemente vor
\item Es gibt Regeln, die mehr als ein Terminalsymbol auf der rechten
Seite enthalten.
\item Die Regel
$\text{\textsl{variable}}\to \text{'{\tt x}'}\;\text{\textsl{konstante}}$
enthält zwar zwei Elemente auf der rechten Seite, aber eines davon
ist ein Terminalsymbol, in Chomsky-Normalform müssten beide Variable
sein.
\qedhere
\end{itemize}
\end{loesung}
