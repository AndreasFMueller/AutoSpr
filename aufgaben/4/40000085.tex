Verwenden Sie die Tabellendurchführung des CYK-Algorithmus, und
das Wort \texttt{000111} mit der Grammatik
\begin{align*}
S_0& \to NE \mid NA \mid \varepsilon \\
S  & \to NE \mid NA \\
A  &\to SE \\
N&\to \texttt{0} \\
E&\to \texttt{1} 
\end{align*}
zu produzieren.

\begin{loesung}
\def\h{1.0}
\def\punkt#1#2{({(#1)*\h},{(#2)*\h})}
\def\tabelle{
	\draw \punkt{0}{0} -- \punkt{6}{0};
	\foreach \x in {1,...,6}{
		\draw \punkt{0}{\x} -- \punkt{7-\x}{\x};
	}
	\draw \punkt{0}{0} -- \punkt{0}{6};
	\foreach \x in {1,...,6}{
		\draw \punkt{\x}{0} -- \punkt{\x}{7-\x};
	}
	\node at \punkt{0.5}{-0.5} {\texttt{0}};
	\node at \punkt{1.5}{-0.5} {\texttt{0}};
	\node at \punkt{2.5}{-0.5} {\texttt{0}};
	\node at \punkt{3.5}{-0.5} {\texttt{1}};
	\node at \punkt{4.5}{-0.5} {\texttt{1}};
	\node at \punkt{5.5}{-0.5} {\texttt{1}};
	\draw[decorate,decoration={calligraphic brace},thick]
		\punkt{6}{-1} -- \punkt{0}{-1};
	\node at \punkt{3}{-1.5} {$w$};
}
\def\terminal#1#2#3{
	\draw[->,color=red,line width=1.4pt,shorten >= 0.3cm,shorten <= 0.3cm]
		\punkt{#1-0.5}{#2-0.5} -- ++\punkt{0}{-1};
	\node at \punkt{#1-0.5}{#2-0.5}{#3};
}
\def\zeichen#1#2#3#4{
	\draw[->,color=red,line width=1.4pt,shorten >= 0.2cm,shorten <= 0.2cm]
		\punkt{#1-0.5}{#2-0.5} -- ++\punkt{0}{-#4};
	\draw[->,color=red,line width=1.4pt,shorten >= 0.28cm,shorten <= 0.28cm]
		\punkt{#1-0.5}{#2-0.5} -- ++\punkt{#2-#4}{#4-#2};
	\node at \punkt{#1-0.5}{#2-0.5}{#3};
}
\def\keinzeichen#1#2{
	\fill[color=gray!40] \punkt{#1}{#2}
		-- ++\punkt{-1}{0}
		-- ++\punkt{0}{-1}
		-- ++\punkt{1}{0} -- cycle;
}
\def\lineone{
	\terminal{1}{1}{$N$}
	\terminal{2}{1}{$N$}
	\terminal{3}{1}{$N$}
	\terminal{4}{1}{$E$}
	\terminal{5}{1}{$E$}
	\terminal{6}{1}{$E$}
}
\def\linetwo{
	\keinzeichen{1}{2}
	\keinzeichen{2}{2}
	\zeichen{3}{2}{$S_0,S$}{1}
	\keinzeichen{4}{2}
	\keinzeichen{5}{2}
}
\def\linethree{
	\keinzeichen{1}{3}
	\keinzeichen{2}{3}
	\zeichen{3}{3}{$A$}{1}
	\keinzeichen{4}{3}
}
\def\linefour{
	\keinzeichen{1}{4}
	\zeichen{2}{4}{$S_0,S$}{3}
	\keinzeichen{3}{4}
}
\def\linefive{
	\keinzeichen{1}{5}
	\zeichen{2}{5}{$A$}{1}
}
\def\linesix{
	\zeichen{1}{6}{$S_0,S$}{5}
}
Im ersten Schritt wird die unterste Tabelle mit den Variablen
gefüllt, die in Terminalsymbole umgewandelt werden können:
\begin{center}
\begin{tikzpicture}[>=latex,thick]
\lineone
%\linetwo
%\linethree
%\linefour
%\linefive
%\linesix
\tabelle
\end{tikzpicture}
\end{center}
In den folgenden Schritten werden die Zeilen weiter oben mit den
linken Seiten von Zweierregeln gefüllt:
\begin{center}
\begin{tikzpicture}[>=latex,thick]
\begin{scope}
	\lineone
	\linetwo
	\tabelle
\end{scope}
\begin{scope}[xshift=7.0cm]
	\lineone
	\linetwo
	\linethree
	\tabelle
\end{scope}
\end{tikzpicture}
\end{center}
\begin{center}
\begin{tikzpicture}[>=latex,thick]
\begin{scope}
	\lineone
	\linetwo
	\linethree
	\linefour
	\tabelle
\end{scope}
\begin{scope}[xshift=7.0cm]
	\lineone
	\linetwo
	\linethree
	\linefour
	\linefive
	\tabelle
\end{scope}
\end{tikzpicture}
\end{center}
\begin{center}
\begin{tikzpicture}[>=latex,thick]
\begin{scope}
	\lineone
	\linetwo
	\linethree
	\linefour
	\linefive
	\linesix
	\tabelle
\end{scope}
\end{tikzpicture}
\end{center}
Es ergibt sich, dass das Wort tatsächlich aus $S_0$ ableitbar ist und
können auch den Parse-Tree ablesen.
\end{loesung}
