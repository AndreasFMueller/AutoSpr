Ist die Sprache 
\[
L=\{\texttt{a}^n\texttt{b}^l \texttt{c}^{n\cdot l}\;|\; n,l\ge 0\}
\]
kontextfrei?

\begin{loesung}
Nein, dies kann man mit dem Pumping Lemma für kontextfreie Sprachen
nachweisen.
\begin{enumerate}
\item
Wer nehmen an, $L$ sei kontextfrei.
\item
Dann gibt es insbesondere die Pumping Length $N$
\item
Wir konstruieren das Wort 
$w=\texttt{a}^N \texttt{b}^N \texttt{c}^{N^2}\in L$
\item 
Nach dem Pumping Lemma gibt es eine Aufteilung des Wortes in 
fünf Teile $w=uvxyz$, wobei $|vxy|\le N$ sein muss.
\item
Dies bedeutet, dass vom Pumpen höchstens zwei der Buchstaben
\texttt{a}, \texttt{b} und \texttt{c} betroffen sein können.
Es ist aber nicht möglich, dass nur die Zahl der Buchstaben
\texttt{a} und \texttt{b} zunehmen kann, denn dann muss auch
die Zahl der Buchstaben \texttt{c} zunehmen.

Es ist aber auch nicht möglich, dass nur die Anzahl der Buchstaben
\texttt{b} und \texttt{c} zunimmt.
Mit jedem Zeichen \texttt{b} müssen nämlich genau $N$ Zeichen
\texttt{c} hinzugefügt werden.
Das bedeutet, dass $|vy|\le N+1$ sein müsste, was aber ausgeschlossen
ist, weil ja schon $|vxy|\le N$ ist.
\item
Der Widerspruch im letzten Schritt zeigt, dass die Annahme,
$L$ sei kontextfrei, nicht haltbar ist, und dass daher $L$ nicht
kontextfrei sein kann.
\qedhere
\end{enumerate}
\end{loesung}

\begin{bewertung}
Standard-Pumping-Lemma-Beweis.
Annahme ({\bf A}) 1 Punkt,
Pumping Length ({\bf N}) 1 Punkt,
Wort wählen unter Verwendung von $N$ ({\bf W}) 1 Punkt,
Unterteilung des Wortes ({\bf U}) 1 Punkt,
Widerspruch beim Pumpen mit Begründung ({\bf P}) 1 Punkt,
Folgerung ({\bf F}) 1 Punkt.
\end{bewertung}
