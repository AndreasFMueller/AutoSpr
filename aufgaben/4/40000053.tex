Sei $\Sigma=\{ \texttt{a}, \texttt{b}, \texttt{c}\}$.
Kann ein Stackautomat die Sprache
\[
L=
\{
w\in\Sigma^*
\;|\;
|w|_{\texttt{a}}
\cdot
|w|_{\texttt{b}}
=
|w|_{\texttt{c}}
\}
\]
erkennen?

\begin{hinweis}
Verwenden Sie ein Wort der Form
$ \texttt{a}\mathstrut^N \texttt{b}\mathstrut^N \texttt{c}\mathstrut^{N^2}$
\end{hinweis}

\thema{kontextfrei}
\themaL{Pumping Lemma fur kontextfreie Sprachen}{Pumping Lemma für kontextfreie Sprachen}

\begin{loesung}
\definecolor{darkred}{rgb}{0.8,0,0}
\definecolor{darkgreen}{rgb}{0,0.6,0}
Dies ist nicht möglich, wie man mit dem Pumping Lemma für kontextfreie
Sprachen zeigen kann.
Dazu nehmen wir an, dass die Sprache $L$ kontextfrei ist.
Die Pumping length sei $N$.
Wir verwenden das Wort
\(
w = \texttt{a}^N\texttt{b}^N\texttt{c}^{N^2}
\),
\begin{center}
\begin{tikzpicture}[>=latex,thick]
\pgfmathparse{sqrt(13)-1}
\xdef\l{\pgfmathresult}
\draw (0,0) rectangle (12,0.6);
\draw (\l,0) -- (\l,0.6);
\draw ({2*\l},0) -- ({2*\l},0.6);
\node at ({0.5*\l},0.25) {$\texttt{a}\mathstrut^N$};
\node at ({1.5*\l},0.25) {$\texttt{b}\mathstrut^N$};
\node at ({0.5*(2*\l+12)},0.25) {$\texttt{c}\mathstrut^{N^2}$};
\node at (\l,0.6) [above] {$N\mathstrut$};
\node at ({2*\l},0.6) [above] {$2N\mathstrut$};
\node at (12,0.6) [above] {$2N+N^2\mathstrut$};
\node at (0,0.3) [left] {$w=\mathstrut$};
\node at (0,0.3) [left] {\phantom{$
{\color{blue}u}
{\color{darkred}v}
{\color{darkgreen}x}
{\color{darkred}y}
{\color{blue}z}=\mathstrut$}};
\node at (12,0.3) [right] {.\strut};
\end{tikzpicture}
\end{center}
Nach dem Pumping-Lemma gibt es eine Aufteilung des Wortes $w$ in fünf
Teile
\begin{center}
\begin{tikzpicture}[>=latex,thick]
\pgfmathparse{sqrt(13)-1}
\xdef\l{\pgfmathresult}
\draw (0,0) rectangle (12,0.6);
\draw (\l,0) -- (\l,0.6);
\draw ({2*\l},0) -- ({2*\l},0.6);
\node at ({0.5*\l},0.25) {$\texttt{a}\mathstrut^N$};
\node at ({1.5*\l},0.25) {$\texttt{b}\mathstrut^N$};
\node at ({0.5*(2*\l+12)},0.25) {$\texttt{c}\mathstrut^{N^2}$};
\node at (\l,0.6) [above] {$N\mathstrut$};
\node at ({2*\l},0.6) [above] {$2N\mathstrut$};
\node at (12,0.6) [above] {$2N+N^2\mathstrut$};
\node at (0,0.3) [left] {$
{\color{blue}u}
{\color{darkred}v}
{\color{darkgreen}x}
{\color{darkred}y}
{\color{blue}z}=\mathstrut$};
\node at (12,0.3) [right] {.\strut};
\def\v{1.6}
\def\x{2.1}
\def\y{2.5}
\def\z{3.5}

\fill[color=blue!20,opacity=0.5]
	(0.05,0.05) rectangle ({\v-0.025},0.55);
\draw[color=blue]
	(0.05,0.05) rectangle ({\v-0.025},0.55);
\node[color=blue] at ({0.5*\v},0.3) {$u\mathstrut$};

\fill[color=darkred!20,opacity=0.5]
	({\v+0.025},0.05) rectangle ({\x-0.025},0.55);
\draw[color=darkred]
	({\v+0.025},0.05) rectangle ({\x-0.025},0.55);
\node[color=darkred] at ({0.5*(\v+\x)},0.3) {$v\mathstrut$};

\fill[color=darkgreen!20,opacity=0.5]
	({\x+0.025},0.05) rectangle ({\y-0.025},0.55);
\draw[color=darkgreen]
	({\x+0.025},0.05) rectangle ({\y-0.025},0.55);
\node[color=darkgreen] at ({0.5*(\x+\y)},0.3) {$x\mathstrut$};

\fill[color=darkred!20,opacity=0.5]
	({\y+0.025},0.05) rectangle ({\z-0.025},0.55);
\draw[color=darkred]
	({\y+0.025},0.05) rectangle ({\z-0.025},0.55);
\node[color=darkred] at ({0.5*(\y+\z)},0.3) {$y\mathstrut$};

\fill[color=blue!20,opacity=0.5]
	({\z+0.025},0.05) rectangle (11.95,0.55);
\draw[color=blue]
	({\z+0.025},0.05) rectangle (11.95,0.55);
\node[color=blue] at ({0.5*(\z+12)},0.3) {$z\mathstrut$};

\draw[line width=0.3pt] (\v,0) -- (\v,-0.4);
\draw[line width=0.3pt] (\z,0) -- (\z,-0.4);
\draw[<->] (\v,-0.3) -- (\z,-0.3);
\node at ({0.5*(\v+\z)},-0.3) [below] {$\le N$};

\end{tikzpicture}
\end{center}
$w=
{\color{blue}u}
{\color{darkred}v}
{\color{darkgreen}x}
{\color{darkred}y}
{\color{blue}z}$
mit
$|{\color{darkred}v}
{\color{darkgreen}x}
{\color{darkred}y}| \le N$ derart, dass alle gepumptent
Wörter
${\color{blue}u}{\color{darkred}v}^k
{\color{darkgreen}x}{\color{darkred}y}^k{\color{blue}z}$
ebenfalls in $L$ sind.
Die Blöcke 
${\color{darkred}v}{\color{darkgreen}x}{\color{darkred}y}$
können auch anderswo liegen, dürfen aber zusammen nicht länger
als $N$ sein.
Wegen
$|{\color{darkred}v}{\color{darkgreen}x}{\color{darkred}y}|\le N$
kann ${\color{darkred}v}{\color{darkgreen}x}{\color{darkred}y}$
höchstens zwei verschiedene Arten von Zeichen enthalten.
Somit gibt es zwei mögliche Fälle:
\begin{enumerate}
\item
$vwx$ enthält nur die Zeichen \texttt{a} und \texttt{b}.
Dann ändert beim Pumpen die Anzahl der \texttt{a} und \texttt{b} ohne
dass sich die Anzahl der \texttt{c} ändert, das gepumpte Wort wird daher
nicht mehr in $L$ sein.
\item
$vwx$ enthält nur die Zeichen \texttt{b} und \texttt{c}.
Beim Pumpen ändert die Zahl der \texttt{b} und \texttt{c}, das allein
steht aber noch nicht im Widerspruch zur Definition von $L$.
Ändert die Anzahl der \texttt{b} um eins, dann müssen im \texttt{c}-Teil
genau $N$ neue \texttt{c} hinzukommen, damit die Regel
$
|w|_{\texttt{a}}
\cdot
|w|_{\texttt{b}}
=
|w|_{\texttt{c}}
$
immer noch erfüllt ist.
Dazu müsste der Teil $x$ aber mindestens $N$ Zeichen enthalten, 
also wäre $|{\color{darkred}v}{\color{darkgreen}x}{\color{darkred}z}|\ge N+1$,
was nicht sein darf.
\end{enumerate}
Es ist also nicht möglich, das Wort $w$ aufzupumpen, im Widerspruch
zur Aussage des Pumping Lemma.
Dieser Widerspruch zeigt, dass die Anahme, $L$ sei kontextfrei, zu verwerfen
ist.
\end{loesung}

\begin{bewertung}
Pumping-Lemma ({\bf PL}) 1 Punkt,
Pumping length ({\bf N}) 1 Punkt,
Beispielwort ({\bf W}) 1 Punkt,
Zerlegung ({\bf Z}) 1 Punkt,
Pumpeigenschaft ({\bf P}) 1 Punkt,
Widerspruch ({\bf X}) 1 Punkt.
\end{bewertung}

