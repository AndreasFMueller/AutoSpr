Sei $\Sigma=\{ \texttt{a}, \texttt{b}, \texttt{c}\}$.
Kann ein Stackautomat die Sprache
\[
L=
\{
w\in\Sigma^*
\;|\;
|w|_{\texttt{a}}
\cdot
|w|_{\texttt{b}}
=
|w|_{\texttt{c}}
\}
\]
erkennen?

\begin{hinweis}
Verwenden Sie ein Wort der Form
$ \texttt{a}^N \texttt{b}^N \texttt{c}^{N^2}$
\end{hinweis}

\thema{kontextfrei}
\themaL{Pumping Lemma fur kontextfreie Sprachen}{Pumping Lemma für kontextfreie Sprachen}

\begin{loesung}
Dies ist nicht möglich, wie man mit dem Pumping Lemma für kontextfreie
Sprachen zeigen kann.
Dazu nehmen wir an, dass die Sprache $L$ kontextfrei ist.
Die Pumping length sei $N$.
Wir verwenden das Wort
\[
w = \texttt{a}^N\texttt{b}^N\texttt{c}^{N^2}.
\]
Nach dem Pumping-Lemma gibt es eine Aufteilung des Wortes $w$ in fünf
Teile $w=uvwxy$ mit $|vwx| \le N$ derart, dass alle gepumptent
Wörter $uv^kwx^ky$ ebenfalls in $L$ sind.
Wegen der Bedingung $|vwx|\le N$ kann $vwx$ höchstens zwei verschiedene
Arten von Zeichen enthalten.
Somit gibt es zwei mögliche Fälle:
\begin{enumerate}
\item
$vwx$ enthält nur die Zeichen \texttt{a} und \texttt{b}.
Dann ändert beim Pumpen die Anzahl der \texttt{a} und \texttt{b} ohne
dass sich die Anzahl der \texttt{c} ändert, das gepumpte Wort wird daher
nicht mehr in $L$ sein.
\item
$vwx$ enthält nur die Zeichen \texttt{b} und \texttt{c}.
Beim Pumpen ändert die Zahl der \texttt{b} und \texttt{c}, das allein
steht aber noch nicht im Widerspruch zur Definition von $L$.
Ändert die Anzahl der \texttt{b} um eins, dann müssen im \texttt{c}-Teil
genau $N$ neue \texttt{c} hinzukommen, damit die Regel
$
|w|_{\texttt{a}}
\cdot
|w|_{\texttt{b}}
=
|w|_{\texttt{c}}
$
immer noch erfüllt ist.
Dazu müsste der Teil $x$ aber mindestens $N$ Zeichen enthalten, 
also wäre $|vwx|\ge N+1$, was nicht sein darf.
\end{enumerate}
Es ist also nicht möglich, das Wort $w$ aufzupumpen, im Widerspruch
zur Aussage des Pumping Lemma.
Dieser Widerspruch zeigt, dass die Anahme, $L$ sei kontextfrei, zu verwerfen
ist.
\end{loesung}

\begin{bewertung}
Pumping-Lemma ({\bf PL}) 1 Punkt,
Pumping length ({\bf N}) 1 Punkt,
Beispielwort ({\bf W}) 1 Punkt,
Zerlegung ({\bf Z}) 1 Punkt,
Pumpeigenschaft ({\bf P}) 1 Punkt,
Widerspruch ({\bf X}) 1 Punkt.
\end{bewertung}

