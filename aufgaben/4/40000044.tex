Sei $\Sigma=\{\texttt{a},\texttt{b},\dots,\texttt{z}\}$ das Alphabet
bestehend aus den Kleinbuchstaben.
Die Sprache
\[
L_{ab}=\{w\in\Sigma^*\;|\; |w|_a=|w|_b>1\}
\]
besteht aus den W"ortern, die die Buchstaben $a\in \Sigma$ und $b\in\Sigma$
mehr als einmal enthalten, aber beide gleich oft.
Es ist zum Beispiel
\begin{align*}
\texttt{essen}&\in L_{\texttt{es}}\\
\texttt{rapperswil}&\in L_{\texttt{pr}}\\
\texttt{seenachtfest}&\in L_{\texttt{st}}
\end{align*}
\begin{teilaufgaben}
\item
Konstruieren Sie eine kontextfreie Grammatik f"ur $L_{ab}$.
\item
Bestimmen Sie f"ur jede Regel der Grammatik aus a), ob sie der
Chomsky-Normalform entspricht, und wenn nicht, warum nicht.
\item
Bringen Sie die Grammatik von a) in Chomsky-Normalform.
\item
K"onnen Sie daraus eine Aussage dar"uber ableiten, ob die Sprache 
\[
L=\{w\in\Sigma^*\;|\;\text{es gibt zwei verschiedene Buchstaben
$a,b\in\Sigma$ mit $|w|_a=|w|_b > 1$}\}
\]
(siehe auch Aufgabe~\ref{30000047}) kontextfrei ist?
\end{teilaufgaben}

\begin{loesung}
\begin{teilaufgaben}
\item
Die Sprache $L_{ab}$ ist kontextfrei, weil sich daf"ur eine kontextfreie
Grammatik angeben l"asst.
Die Grammatik f"ugt die Buchstaben $a$ und $b$ jeweils paarweise
hinzu, w"ahrend sie andere Buchstaben einzeln hinzuf"ugt:
\begin{align}
S&\rightarrow aSb\;|\; bSa \label{40000044:1}\\
 &\rightarrow SU \;|\; US  \label{40000044:2}\\
 &\rightarrow \varepsilon  \label{40000044:3}\\
U&\rightarrow \text{alle Terminalsymbole ausser $a$ und $b$} \label{40000044:4}
\end{align}
\item
Die Startvariable kommt auf der rechten Seite vor in (\ref{40000044:1})
und (\ref{40000044:2}).
In (\ref{40000044:1}) kommen auf der rechten Seite mehr als zwei Symbole vor.
\item
Die Grammatik ist noch nicht in Chomsky-Normalform.
Zun"achst kommt die Startvariable auch rechts vor, was wir durch eine
zus"atzliche Startvariable erreichen k"onnen:
\begin{align*}
S_0&\rightarrow S\\
S  &\rightarrow aSb\;|\; bSa\\
   &\rightarrow SU \;|\; US\\
   &\rightarrow \varepsilon\\
U  &\rightarrow \text{alle Terminalsymbole ausser $a$ und $b$}
\end{align*}
Jetzt ist aber eine unzul"assige $\varepsilon$-Regel vorhanden, die entfernt
werden muss:
\begin{align*}
S_0&\rightarrow S\;|\;\varepsilon\\
S  &\rightarrow aSb\;|\; bSa\;|\;ab\;|\;ba\\
   &\rightarrow SU \;|\; US\;|\;U\;\\
U  &\rightarrow \text{alle Terminalsymbole ausser $a$ und $b$}
\end{align*}
Jetzt hat man aber zwei Unit-Rules, die entfernt werden m"ussen.
Zuerst $S\rightarrow U$:
\begin{align*}
S_0&\rightarrow S\;|\;\varepsilon\\
S  &\rightarrow aSb\;|\; bSa\;|\;ab\;|\;ba\\
   &\rightarrow SU \;|\; US\\
   &\rightarrow \text{alle Terminalsymbole ausser $a$ und $b$}\\
U  &\rightarrow \text{alle Terminalsymbole ausser $a$ und $b$}
\end{align*}
Nun $S_0\rightarrow S$:
\begin{align*}
S_0&\rightarrow aSb\;|\; bSa\;|\;ab\;|\;ba\\
   &\rightarrow SU \;|\; US\\
   &\rightarrow \text{alle Terminalsymbole ausser $a$ und $b$}\\
   &\rightarrow \varepsilon\\
S  &\rightarrow aSb\;|\; bSa\;|\;ab\;|\;ba\\
   &\rightarrow SU \;|\; US\\
   &\rightarrow \text{alle Terminalsymbole ausser $a$ und $b$}\\
U  &\rightarrow \text{alle Terminalsymbole ausser $a$ und $b$}
\end{align*}
Jetzt sind nur noch die Regeln, die $a$ oder $b$ auf der rechten
Seite enthalten nicht gem"ass der Chomsky-Normalform.
Dazu brauchen wir zus"atzliche Variablen:
\begin{align*}
S_0&\rightarrow AC\;|\; DA \\
   &\rightarrow AB\;|\;BA\\
   &\rightarrow SU \;|\; US\\
   &\rightarrow \text{alle Terminalsymbole ausser $a$ und $b$}\\
   &\rightarrow \varepsilon\\
S  &\rightarrow AC\;|\; DA \\
   &\rightarrow AB\;|\;BA\\
   &\rightarrow SU \;|\; US\\
   &\rightarrow \text{alle Terminalsymbole ausser $a$ und $b$}\\
U  &\rightarrow \text{alle Terminalsymbole ausser $a$ und $b$}\\
A  &\rightarrow a \\
B  &\rightarrow b \\
C  &\rightarrow SB \\
D  &\rightarrow SA
\end{align*}
Damit ist die Grammatik in Chomsky-Normalform.
\item
Die Sprache $L$ ist die Vereinigung aller Sprachen $L_{ab}$,
\[
L=\bigcup_{a,b\in\Sigma\atop a\ne b}L_{ab},
\]
und da Vereinigung eine regul"are Operation ist, und kontextfreie Sprachen
unter Vereinigung abgeschlossen sind, muss auch $L$ kontextfrei sein.
\qedhere
\end{teilaufgaben}
\end{loesung}

\begin{diskussion}
Die Aufgabe ist in der Pr"ufung ohne Teilaufgabe c) gestellt worden.
\end{diskussion}

\begin{bewertung}
Buchstaben $a$ und $b$ werden paarweise ({\bf P}) und in beliebiger
Reihenfolge ({\bf R}) hinzugef"ugt, je 1 Punkt,
Leeres Wort wird erzeugt ({\bf E}) 1 Punkt,
beliebige andere Buchstaben k"onnen vorne oder hinten angef"ugt werden
({\bf A}) 1 Punkt,
Paarregel ({\bf CP}) 1 Punkt,
Vereinigung f"ur Teilaufgabe d) ({\bf U}) 1 Punkt.
\end{bewertung}



