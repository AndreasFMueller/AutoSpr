Konstruieren Sie das Zustandsdiagramm eines Stackautomaten aus der
folgenden Grammatik für Klammerausdrücke
\begin{align*}
K&\to \varepsilon \\
 &\to \texttt{(}K\texttt{)}K
\end{align*}

\begin{loesung}
Wie verwenden die Konstruktion des ``Blümchenautomaten'':
\begin{center}
\def\r{0.3}
\def\l{3.6}
\def\zustand#1#2{
	\draw #1 circle[radius=\r];
	\node at #1 {$#2$};
}
\def\akzeptierzustand#1#2{
	\zustand{#1}{#2}
	\draw #1 circle[radius={\r-0.05}];
}
\def\pfeil#1#2{
	\draw[->,shorten <= 0.3cm,shorten >= 0.3cm] #1 -- #2;
}
\begin{tikzpicture}[>=latex,thick]
\coordinate (q0) at (0,0);
\coordinate (q1) at (\l,0);
\coordinate (q2) at ({2*\l},0);
\coordinate (q3) at ({3*\l},0);
\coordinate (a1) at ({1.5*\l},{-0.5*\l});
\coordinate (a2) at ({2*\l},{-\l});
\coordinate (a3) at ({2.5*\l},{-0.5*\l});
\zustand{(q0)}{q_0}
\zustand{(q1)}{S}
\zustand{(q2)}{R}
\zustand{(a1)}{}
\zustand{(a2)}{}
\zustand{(a3)}{}
\pfeil{(q2)}{(a1)}
\pfeil{(a1)}{(a2)}
\pfeil{(a2)}{(a3)}
\pfeil{(a3)}{(q2)}
\akzeptierzustand{(q3)}{A}
\pfeil{(-1.5,0)}{(q0)}
\pfeil{(q0)}{(q1)}
\pfeil{(q1)}{(q2)}
\pfeil{(q2)}{(q3)}

\draw[->,shorten >= 0.3cm,shorten <= 0.3cm]
	(q2) to[out=30,in=75,distance=2cm] (q2);

\draw[<-,shorten >= 0.3cm,shorten <= 0.3cm]
	(q2) to[out=105,in=150,distance=2cm] (q2);

\draw[->,shorten >= 0.3cm,shorten <= 0.3cm]
	(q2) to[out=-75,in=-105,distance=2cm] (q2);

\node at ($0.5*(q0)+0.5*(q1)$) [above]
	{$\varepsilon,\varepsilon\to\texttt{\$}$};

\node at ($0.5*(q1)+0.5*(q2)$) [above]
	{$\varepsilon,\varepsilon\to K$};

\node at ($0.5*(q2)+0.5*(q3)$) [above]
	{$\varepsilon,\texttt{\$}\to \varepsilon$};

\node at ($0.7*(a1)+0.3*(q2)$) [above left]
	{$\varepsilon,K\to K$};

\node at ($0.5*(a1)+0.5*(a2)$) [below left]
	{$\varepsilon,\varepsilon\to\texttt{)}$};

\node at ($0.5*(a2)+0.5*(a3)$) [below right]
	{$\varepsilon,\varepsilon\to K$};

\node at ($0.3*(q2)+0.7*(a3)$) [above right]
	{$\varepsilon,\varepsilon\to \texttt{(}$};

\node at ($(q2)+(0,{-0.5*\l})$) {$\varepsilon,K\to\varepsilon$};

\node at ($(q2)+(-0.2,{0.35*\l})$) [above left]
	{$\texttt{(},\texttt{(}\to\varepsilon$};

\node at ($(q2)+(0.2,{0.35*\l})$) [above right]
	{$\texttt{)},\texttt{)}\to\varepsilon$};

\end{tikzpicture}
\end{center}
\end{loesung}
