Ist die Sprache
\[
L=
\{
\texttt{0}^i \texttt{1}^j\texttt{0}^k\mid  j> i>0\wedge k>0\}
\}
\]
kontextfrei?
Wenn ja, geben Sie eine kontextfreie Grammatik in Chomsky-Normalform dafür an.

\thema{kontextfrei}
\thema{kontextfreie Grammatik}
\thema{Chomsky-Normalform}

\begin{loesung}
Die Sprache ist kontextfrei.
Die Grammatik muss Wörter der Form $\texttt{0}^i\texttt{1}^i$, $i>0$, mit
Wörtern der Form $\texttt{1}^l\texttt{0}^k$, $l>0\wedge k>0$, verketten.
Für Wörter der ersten Art haben wir die Grammatik
\begin{align*}
A&\to \texttt{0}A\texttt{1} \mid  \texttt{01},
\end{align*}
für die Wörter der zweiten Art können wir
\begin{align*}
B&\to \texttt{1} B \mid  B\texttt{0} \mid  \texttt{10}
\end{align*}
verwenden.
Damit finden wir die Grammatik
\begin{align*}
S&\to AB
\\
A&\to \texttt{0}A\texttt{1} \mid  \texttt{01}
\\ 
B&\to \texttt{1}B \mid  B\texttt{0} \mid  \texttt{10}
\end{align*}
Diese Grammatik hat nicht Chomsky-Normalform.
Aber ihre Startvariable kommt auf der rechten Seite nicht vor und
sie hat weder $\varepsilon$-Regeln noch Unit Rules.
Es bleibt daher nur noch, die Terminalsymbole und die Dreierregeln
zu ``flicken'':
\begin{align*}
S&\to AB
\\
A&\to NU \mid  NE
\\
U&\to AE
\\ 
B&\to EB \mid  BN \mid  EN
\\
N&\to \texttt{0}
\\
E&\to \texttt{1}
\end{align*}
Damit hat die Grammatik Chomsky-Normalform.
\end{loesung}

\begin{bewertung}
Grammatik (3 Punkte) korrekt formuliert ({\bf G}) 1 Punkt,
Mehr \texttt{1} wie \texttt{0} ({\bf A}) 1 Punkt,
Mindestens eine \texttt{0} im zweiten Teil ({\bf N}) 1 Punkt.
CNF Umwandlung (3 Punkte): neue Startvariable ({\bf S}) 1 Punkt,
unit rules ({\bf U}) 1 Punkt,
Dreierregeln und Terminalsymbole ({\bf T}) 1 Punkt.
\end{bewertung}
