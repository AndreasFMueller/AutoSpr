Die Grammatiken
\begin{align*}
S_1 & \to S_1P \mid Z               & S_2 & \to \texttt{0}S_2\texttt{1}    \\
P   & \to ZZ                        &     & \to \varepsilon                \\
Z   & \to \texttt{0}\mid\texttt{1}  &     &                               
\end{align*}
erzeugen die Sprachen $L_1=\{w\in\Sigma^*\mid  \text{$|w|$ ungerade}\}$
und $L_2=\{\texttt{0}^n\texttt{1}^n \mid k\ge 0\}$.
Konstruieren Sie eine Grammatik für
\begin{teilaufgaben}
\item Konstruieren Sie eine Grammatik für die Sprache $L=\{
\texttt{0}^{n_1}\texttt{1}^{n_1}
\texttt{0}^{n_2}\texttt{1}^{n_2}
\dots
\texttt{0}^{n_l}\texttt{1}^{n_l}
\mid
n_i\ge 0\forall i\le l
\}$
\item Konstruieren Sie eine Grammatik für die Sprache $L=\{
w\in\Sigma^* \mid
\text{$|w|$ ungerade und $w$ endet mit $\texttt{0}n\texttt{1}^n$, $n\ge 0$}
\}$
\end{teilaufgaben}

\begin{loesung}
\begin{teilaufgaben}
\item
Die gesuchte Sprache ist $L=L_2^*$, wir müssen daher die *-Konstruktion
auf die Grammatik von $L_2$ anwenden.
Dies ergibt 
\begin{align*}
S   & \to SS_2 \mid \varepsilon  \\
S_2 & \to \texttt{0}S_2\texttt{1} \mid \varepsilon
\end{align*}
\item
Da ein Wort der Form $\texttt{0}^n\texttt{1}^n$ gerade Länge hat, besteht
ein Wort von $L$ aus einem Stück underader Länge gefolgt von einem
Wort der Form $\texttt{0}^n\texttt{1}^n$.
Somit ist die Sprache $L=L_1L_2$ eine Verkettung.
Wir müssen also eine Grammatik für die Verkettung produzieren:
\begin{align*}
S   & \to S_1S_2                    &     &                                \\
S_1 & \to S_1P \mid Z               & S_2 & \to \texttt{0}S_2\texttt{1}    \\
P   & \to ZZ                        &     & \to \varepsilon                \\
Z   & \to \texttt{0}\mid\texttt{1}  &     &                               
\qedhere
\end{align*}
\end{teilaufgaben}
\end{loesung}




