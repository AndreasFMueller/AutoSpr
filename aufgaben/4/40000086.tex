Standardisieren Sie den folgenden Stackautomaten für die Sprache
$L=\{\texttt{a}^n\texttt{b}^m\mid n\ge m\}$ als Vorbereitung
darauf, die Grammatik abzulesen.
\begin{center}
\def\h{3}
\def\r{0.35}
\def\punkt#1#2{({(#1)*\h},{(#2)*\h})}
\def\zustand#1#2{
	\draw #1 circle[radius=\r];
	\node at #1 {#2};
}
\def\akzeptierzustand#1#2{
	\zustand{#1}{#2}
	\draw #1 circle[radius={\r-0.05}];
}
\def\pfeil#1#2{
	\draw[->,shorten >= 0.35cm,shorten <= 0.35cm] #1 -- #2;
}
\begin{tikzpicture}[>=latex,thick]
\coordinate (q0) at (0,0);
\coordinate (q1) at (\h,0);
\zustand{(q0)}{$q_0$}
\akzeptierzustand{(q1)}{$q_1$}
\pfeil{(-2,0)}{(q0)}
\pfeil{(q0)}{(q1)}
\node at ($0.5*(q0)+0.5*(q1)$) [above]
	{$\varepsilon,\varepsilon\to\varepsilon$};
\draw[->,shorten <= 0.35cm,shorten >= 0.35cm,distance=2cm]
	(q0) to[out=60,in=120] (q0);
\draw[->,shorten <= 0.35cm,shorten >= 0.35cm,distance=2cm]
	(q1) to[out=60,in=120] (q1);
\node at ($(q0)+(0,1.3)$) [above] {$\texttt{a},\varepsilon\to\texttt{a}$};
\node at ($(q1)+(0,1.3)$) [above] {$\texttt{b},\texttt{a}\to\varepsilon$};
\end{tikzpicture}
\end{center}

\thema{Grammatik}
\thema{Stackautomat}
\thema{Grammatik aus Stackautomat}

\begin{loesung}
\def\h{3}
\def\r{0.35}
\def\punkt#1#2{({(#1)*\h},{(#2)*\h})}
\def\zustand#1#2{
	\draw #1 circle[radius=\r];
	\node at #1 {#2};
}
\def\akzeptierzustand#1#2{
	\zustand{#1}{#2}
	\draw #1 circle[radius={\r-0.05}];
}
\def\pfeil#1#2{
	\draw[->,shorten >= 0.35cm,shorten <= 0.35cm] #1 -- #2;
}
\def\punkte{
	\coordinate (q00) at (-\h,0);
	\coordinate (q0) at (0,0);
	\coordinate (q1) at (\h,0);
	\coordinate (q2) at ({2*\h},0);
	\coordinate (q3) at ({3*\h},0);
	\coordinate (qi) at ({0.5*\h},{-sqrt(3)*\h/2});
	\coordinate (qii) at ({1.5*\h},{-sqrt(3)*\h/2});
}
Es muss ein neuer Zustand $q_a$ hinzugefügt werden, bei dem
die noch verbleibenden Zeichen auf dem Stack entfernt werden können.
\begin{center}
\begin{tikzpicture}[>=latex,thick]
\punkte
\zustand{(q0)}{$q_0$}
\zustand{(q1)}{$q_1$}
\akzeptierzustand{(q2)}{$q_a$}
\pfeil{(-2,0)}{(q0)}
\pfeil{(q0)}{(q1)}
\pfeil{(q1)}{(q2)}
\node at ($0.5*(q0)+0.5*(q1)$) [above]
	{$\varepsilon,\varepsilon\to\varepsilon$};
\node at ($0.5*(q1)+0.5*(q2)$) [above]
	{$\varepsilon,\varepsilon\to\varepsilon$};
\draw[->,shorten <= 0.35cm,shorten >= 0.35cm,distance=2cm]
	(q0) to[out=60,in=120] (q0);
\draw[->,shorten <= 0.35cm,shorten >= 0.35cm,distance=2cm]
	(q1) to[out=60,in=120] (q1);
\draw[->,shorten <= 0.35cm,shorten >= 0.35cm,distance=2cm]
	(q2) to[out=60,in=120] (q2);
\node at ($(q0)+(0,1.3)$) [above] {$\texttt{a},\varepsilon\to\texttt{a}$};
\node at ($(q1)+(0,1.3)$) [above] {$\texttt{b},\texttt{a}\to\varepsilon$};
\node at ($(q2)+(0,1.3)$) [above] {$\varepsilon,\texttt{a}\to\varepsilon$};
\end{tikzpicture}
\end{center}
Damit das Stackende überhaupt erkannt werden kann, müssen die Zustände
$q_0'$ und $q_a'$ hinzugefügt und ein Übergang von $q_0'$ nach $q_0$,
der das Zeichen \texttt{\$} auf den Stack legt, und einer von $q_a$ nach
$q_a'$, der das Zeichen \texttt{\$} wieder entfernt:
\begin{center}
\begin{tikzpicture}[>=latex,thick]
\punkte
\zustand{(q00)}{$q_0'$}
\zustand{(q0)}{$q_0$}
\zustand{(q1)}{$q_1$}
\zustand{(q2)}{$q_a$}
\akzeptierzustand{(q3)}{$q_a'$}
\pfeil{(-5,0)}{(q00)}
\pfeil{(q0)}{(q1)}
\pfeil{(q00)}{(q0)}
\pfeil{(q1)}{(q2)}
\pfeil{(q2)}{(q3)}
\node at ($0.5*(q00)+0.5*(q0)$) [above]
	{$\varepsilon,\varepsilon\to\texttt{\$}$};
\node at ($0.5*(q0)+0.5*(q1)$) [above]
	{$\varepsilon,\varepsilon\to\varepsilon$};
\node at ($0.5*(q1)+0.5*(q2)$) [above]
	{$\varepsilon,\varepsilon\to\varepsilon$};
\node at ($0.5*(q2)+0.5*(q3)$) [above]
	{$\varepsilon,\texttt{\$}\to\varepsilon$};
\draw[->,shorten <= 0.35cm,shorten >= 0.35cm,distance=2cm]
	(q0) to[out=60,in=120] (q0);
\draw[->,shorten <= 0.35cm,shorten >= 0.35cm,distance=2cm]
	(q1) to[out=60,in=120] (q1);
\draw[->,shorten <= 0.35cm,shorten >= 0.35cm,distance=2cm]
	(q2) to[out=60,in=120] (q2);
\node at ($(q0)+(0,1.3)$) [above] {$\texttt{a},\varepsilon\to\texttt{a}$};
\node at ($(q1)+(0,1.3)$) [above] {$\texttt{b},\texttt{a}\to\varepsilon$};
\node at ($(q2)+(0,1.3)$) [above] {$\varepsilon,\texttt{a}\to\varepsilon$};
\end{tikzpicture}
\end{center}
Schliesslich müssen die $\varepsilon$-Umgänge durch Übergänge ersetzt
werden, die ein Zeichen auf den Stack legen und gleich wieder entfernen:
\begin{center}
\begin{tikzpicture}[>=latex,thick]
\punkte
\zustand{(q00)}{$q_0'$}
\zustand{(q0)}{$q_0$}
\zustand{(q1)}{$q_1$}
\zustand{(q2)}{$q_a$}
\zustand{(qi)}{$q_i$}
\zustand{(qii)}{$q_i'$}
\akzeptierzustand{(q3)}{$q_a'$}
\pfeil{(-5,0)}{(q00)}
\pfeil{(q0)}{(qi)}
\pfeil{(qi)}{(q1)}
\pfeil{(q00)}{(q0)}
\pfeil{(q1)}{(qii)}
\pfeil{(qii)}{(q2)}
\pfeil{(q2)}{(q3)}
\node at ($0.5*(q00)+0.5*(q0)$) [above]
	{$\varepsilon,\varepsilon\to\texttt{\$}$};
\node at ($0.5*(q2)+0.5*(q3)$) [above]
	{$\varepsilon,\texttt{\$}\to\varepsilon$};
\draw[->,shorten <= 0.35cm,shorten >= 0.35cm,distance=2cm]
	(q0) to[out=60,in=120] (q0);
\draw[->,shorten <= 0.35cm,shorten >= 0.35cm,distance=2cm]
	(q1) to[out=60,in=120] (q1);
\draw[->,shorten <= 0.35cm,shorten >= 0.35cm,distance=2cm]
	(q2) to[out=60,in=120] (q2);
\node at ($(q0)+(0,1.3)$) [above] {$\texttt{a},\varepsilon\to\texttt{a}$};
\node at ($(q1)+(0,1.3)$) [above] {$\texttt{b},\texttt{a}\to\varepsilon$};
\node at ($(q2)+(0,1.3)$) [above] {$\varepsilon,\texttt{a}\to\varepsilon$};
\node at ($0.5*(q0)+0.5*(qi)$) [below,rotate=-60]
	{$\varepsilon,\varepsilon\to\texttt{x}$};
\node at ($0.5*(qi)+0.5*(q1)$) [below,rotate=60]
	{$\varepsilon,\texttt{x}\to\varepsilon$};
\node at ($0.5*(q1)+0.5*(qii)$) [below,rotate=-60]
	{$\varepsilon,\varepsilon\to\texttt{x}$};
\node at ($0.5*(qii)+0.5*(q2)$) [below,rotate=60]
	{$\varepsilon,\texttt{x}\to\varepsilon$};
\end{tikzpicture}
\end{center}
Damit ist die Standardform erreicht, aus der sich später die Grammatik
wird  ablesen lassen.
\end{loesung}
