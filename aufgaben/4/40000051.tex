Betrachten Sie die folgenden zwei Sprachen über dem Alphabet
$\Sigma=\{\texttt{a},\texttt{b},\texttt{c}, \texttt{d}\}$:
\begin{align*}
L_1
&=
\{ \texttt{a}^n\texttt{b}^m\texttt{c}^n\texttt{d}^m
\;|\; m,n\ge 0
\}
\qquad\text{und}
\\
L_2
&=
\{ \texttt{a}^n\texttt{b}^m\texttt{c}^m\texttt{d}^n
\;|\; m,n\ge 0
\}.
\end{align*}
Sind diese Sprachen kontextfrei?

\begin{loesung}
Die Sprache $L_2$ hat die Grammatik
\begin{align*}
S&\rightarrow R
\\
 &\rightarrow \texttt{a}\; S \;\texttt{d}
\\
R&\rightarrow \varepsilon
\\
 &\rightarrow \texttt{b}\; R \;\texttt{c}
\end{align*}
Die Sprache $L_1$ dagegen ist nicht kontextfrei, wie wir mit dem Pumping-Lemma
für kontextfreie Sprachen beweisen können.
Dazu nehmen wir an, dass $L_1$ kontextfrei ist, dann gibt es die Zahl $N$,
die Pumping length von $L_1$.
Damit konstruieren wir jetzt das Wort
$w=
\texttt{a}^N
\texttt{b}^N
\texttt{c}^N
\texttt{d}^N
$.
Gemäss dem Pumping Lemma gibt es eine Unterteilung $w=uvxyz$ derart,
dass $|vxy|\le N$ und $|vy|>0$.
Die beiden Teilwörter $v$ und $y$ befinden sich in benachbarten Blöcken,
die die Längen $m$ und $n$ haben.
Beim Pumpen werden diese Blöcke länger, nicht aber die Blöcke, ebenfalls
mit Länge $m$ und $n$, die nicht von $v$ und $y$ berührt werden.
Damit fällt das Wort beim Aufpumpen aus der Sprache $L_1$ heraus, im
Widerspruch zur Aussage des Pumping Lemmas.
Dieser Widerspruch zeigt, dass $L_1$  nicht kontextfrei sein kann.
\end{loesung}




