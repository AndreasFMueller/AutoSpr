Lesen Sie die Grammatik aus dem folgenden Stackautomaten für die
Sprache $L=\{ \texttt{a}^n\texttt{b}^m \mid n\ge m\}$ ab:
\begin{center}
\def\h{2.5}
\def\r{0.35}
\def\punkt#1#2{({(#1)*\h},{(#2)*\h})}
\def\zustand#1#2{
	\draw #1 circle[radius=\r];
	\node at #1 {#2};
}
\def\akzeptierzustand#1#2{
	\zustand{#1}{#2}
	\draw #1 circle[radius={\r-0.05}];
}
\def\pfeil#1#2{
	\draw[->,shorten >= 0.35cm,shorten <= 0.35cm] #1 -- #2;
}
\def\punkte{
	\coordinate (q00) at (-\h,0);
	\coordinate (q0) at (0,0);
	\coordinate (q1) at (\h,0);
	\coordinate (q2) at ({2*\h},0);
	\coordinate (q3) at ({3*\h},0);
	\coordinate (qi) at ({0.5*\h},{-sqrt(3)*\h/2});
	\coordinate (qii) at ({1.5*\h},{-sqrt(3)*\h/2});
}
\begin{tikzpicture}[>=latex,thick]
\punkte
\zustand{(q00)}{$q_0'$}
\zustand{(q0)}{$q_0$}
\zustand{(q1)}{$q_1$}
\zustand{(q2)}{$q_a$}
\zustand{(qi)}{$q_i$}
\zustand{(qii)}{$q_i'$}
\akzeptierzustand{(q3)}{$q_a'$}
\pfeil{(-5,0)}{(q00)}
\pfeil{(q0)}{(qi)}
\pfeil{(qi)}{(q1)}
\pfeil{(q00)}{(q0)}
\pfeil{(q1)}{(qii)}
\pfeil{(qii)}{(q2)}
\pfeil{(q2)}{(q3)}
\node at ($0.5*(q00)+0.5*(q0)$) [above]
	{$\varepsilon,\varepsilon\to\texttt{\$}$};
\node at ($0.5*(q2)+0.5*(q3)$) [above]
	{$\varepsilon,\texttt{\$}\to\varepsilon$};
\draw[->,shorten <= 0.35cm,shorten >= 0.35cm,distance=2cm]
	(q0) to[out=60,in=120] (q0);
\draw[->,shorten <= 0.35cm,shorten >= 0.35cm,distance=2cm]
	(q1) to[out=60,in=120] (q1);
\draw[->,shorten <= 0.35cm,shorten >= 0.35cm,distance=2cm]
	(q2) to[out=60,in=120] (q2);
\node at ($(q0)+(0,1.3)$) [above] {$\texttt{a},\varepsilon\to\texttt{a}$};
\node at ($(q1)+(0,1.3)$) [above] {$\texttt{b},\texttt{a}\to\varepsilon$};
\node at ($(q2)+(0,1.3)$) [above] {$\varepsilon,\texttt{a}\to\varepsilon$};
\node at ($0.5*(q0)+0.5*(qi)$) [below,rotate=-60]
	{$\varepsilon,\varepsilon\to\texttt{x}$};
\node at ($0.5*(qi)+0.5*(q1)$) [above,rotate=60]
	{$\varepsilon,\texttt{x}\to\varepsilon$};
\node at ($0.5*(q1)+0.5*(qii)$) [below,rotate=-60]
	{$\varepsilon,\varepsilon\to\texttt{x}$};
\node at ($0.5*(qii)+0.5*(q2)$) [above,rotate=60]
	{$\varepsilon,\texttt{x}\to\varepsilon$};
\end{tikzpicture}
\end{center}

\begin{loesung}
Startvariable ist $A_{q_0'q_a'}$.
\[
\left.
\begin{aligned}
A_{q_0'q_a'} & \to \varepsilon A_{q_0q_a}\varepsilon  \\
A_{q_0q_1}   & \to \varepsilon A_{q_iq_i}\varepsilon  \\
             & \to \texttt{a} A_{q_0q_1} \texttt{b} \\
A_{q_0q_a}   & \to A_{q_0q_1}A_{q_1q_a} \\
             & \to \texttt{a} A_{q_0q_a} \\
             & \to \texttt{a} A_{q_0q_1} \texttt{b} A_{q_1q_a} \\
A_{q_iq_i}   & \to \varepsilon \\
A_{q_1a_a}   & \to \varepsilon A_{q_i'q_i'}\varepsilon  \\
A_{q_i'q_i'} & \to \varepsilon \\
\end{aligned}
\quad
\right\}
\qquad\Rightarrow\qquad
\left\{
\quad
\begin{aligned}
S_0 & \to S \\
A   & \to \varepsilon \\
    & \to \texttt{a} A \texttt{b} \\
S   & \to A \\
    & \to \texttt{a} S  \\
    & \to \texttt{a} A \texttt{b} \\
\end{aligned}
\right.
\]
mit den Abkürzungen $S_0=A_{q_0'q_a'}$, $S=A_{q_0q_a}$ und $A=A_{q_0q_1}$.
\end{loesung}



