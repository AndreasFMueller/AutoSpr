Verwenden Sie das Pumping Lemma um zu zeigen, dass die Sprache
\(
L
=
\{
\texttt{(}^n
\texttt{+}^m
\texttt{)}^n
\mid
m < n
\}
\)
nicht kontextfrei ist.

\begin{hinweis}
Beachten Sie, dass es zwei Arten von Unterteilungen gibt, die auf
unterschiedliche Art beim Pumpen zu einem Widerspruch führen.
\end{hinweis}

\begin{loesung}
\def\h{0.6}
\def\o{0.04}
\def\punkt#1#2{({(#1)*\h},{(#2)*\h})}
\def\kasten#1#2#3#4{
	\ifnum0=#1
	\def\s{\o}
	\else
	\def\s{0}
	\fi
	\ifnum 22=#2
	\def\t{\o}
	\else
	\def\t{0}
	\fi
	\fill[color=#3!20,opacity=0.5]
		\punkt{#1+\o+\s}{2*\o} rectangle \punkt{#2-\o-\t}{1-2*\o};
	\draw[color=#3]
		\punkt{#1+\o+\s}{2*\o} rectangle \punkt{#2-\o-\t}{1-2*\o};
	\node[color=#3] at \punkt{0.5*(#1+#2)}{0.5} {$#4\mathstrut$};
}
Wir verwenden das Pumping-Lemma für kontextfreie Sprachen.
\begin{enumerate}
\item
Wir nehmen an, $L$ sei kontextfrei.
\item
Nach dem Pumping-Lemma gibt es die Zahl $N\in\mathbb{N}$ mit der Eigenschaft,
dass Wörter $w\in L$ mit mindestens der Länge $N$ die Pump-Eigenschaft haben.
\item
Wir verwenden das Wort 
$
w
=
\texttt{(}^{N+1}
\texttt{+}^N
\texttt{)}^{N+1}
$
\begin{center}
\begin{tikzpicture}[>=latex,thick]
\draw \punkt{0}{0} rectangle \punkt{22}{1};
\draw \punkt{8}{0} -- \punkt{8}{1};
\draw \punkt{15}{0} -- \punkt{15}{1};
\node at \punkt{8}{1} [above] {$N+1$};
\node at \punkt{14}{1} [above] {$2N+1$};
\node at \punkt{22}{1} [above] {$3N+2$};
\foreach \x in {0.5,1.5,...,8}{
	\node at \punkt{\x}{0.5} {$\texttt{(}\mathstrut$};
}
\foreach \x in {8.5,9.5,...,15}{
	\node at \punkt{\x}{0.5} {$\texttt{+}\mathstrut$};
}
\foreach \x in {15.5,16.5,...,22}{
	\node at \punkt{\x}{0.5} {$\texttt{)}\mathstrut$};
}
\end{tikzpicture}
\end{center}
\item
Nach dem Pumping Lemma gibt es eine Unterteilung
$w
=
{\color{blue}x}
{\color{red}y}
{\color{darkgreen}z}
{\color{red}u}
{\color{blue}v}
$
mit
$
|
{\color{red}y}
{\color{darkgreen}z}
{\color{red}u}
| \le N
$
und
$
|{\color{red}yu}|>0
$.
Wegen der ersten Bedingung kann der Block 
$
{\color{red}y}
{\color{darkgreen}z}
{\color{red}u}
$
höchstens eine Art von Klammern enthalten oder eventuell gar keine.
Die Unterteilungen sehen also ungefähr so aus:
\begin{center}
\def\rechteck{
	\fill[color=gray!40] \punkt{0}{0} rectangle \punkt{22}{1};
	\foreach \x in {0.5,1.5,...,8}{
		\node[color=white] at \punkt{\x}{0.5} {$\texttt{(}\mathstrut$};
	}
	\foreach \x in {8.5,9.5,...,15}{
		\node[color=white] at \punkt{\x}{0.5} {$\texttt{+}\mathstrut$};
	}
	\foreach \x in {15.5,16.5,...,22}{
		\node[color=white] at \punkt{\x}{0.5} {$\texttt{)}\mathstrut$};
	}
	\draw \punkt{0}{0} rectangle \punkt{22}{1};
	\draw \punkt{8}{0} -- \punkt{8}{1};
	\draw \punkt{15}{0} -- \punkt{15}{1};
	\node at \punkt{8}{1} [above] {$N+1$};
	\node at \punkt{15}{1} [above] {$2N+1$};
	\node at \punkt{22}{1} [above] {$3N+2$};
}
\def\unterteilung#1{
	\rechteck
	\kasten{0}{#1}{blue}{x}
	\kasten{#1}{#1+1}{red}{y}
	\kasten{#1+1}{#1+2.5}{darkgreen}{z}
	\kasten{#1+2.5}{#1+4}{red}{u}
	\kasten{#1+4}{22}{blue}{v}
	\draw[line width=0.4pt] \punkt{#1}{-0.6} -- \punkt{#1}{1.1};
	\draw[line width=0.4pt] \punkt{#1+4}{-0.6} -- \punkt{#1+4}{1.1};
	\draw[<->] \punkt{#1}{-0.3} -- \punkt{#1+4}{-0.3};
	\node at \punkt{#1+2}{-0.2} [below] {$\le N$};
}
\begin{tikzpicture}[>=latex,thick]
	\unterteilung{5}
	\begin{scope}[yshift=-1.8cm]
		\unterteilung{9}
	\end{scope}
	\begin{scope}[yshift=-3.6cm]
		\unterteilung{13.5}
	\end{scope}
	\end{tikzpicture}
\end{center}
\item
Im ersten Fall ändert sich beim Pumpen die Anzahl der
öffnenden Klammern, nicht aber die der schliessenden Klammern, somit
kann ein gepumptes Wort nicht mehr in $L$ sein.
Ebenso ändert sich im letzten Fall beim Pumpen die Anzahl der schliessenden
Klammern, wieder kann das gepumpte Wort nicht mehr in $L$ sein.

Im mittleren Fall enthalten die Teil $y$ und $u$ nur Pluszeichen, die Anzahl
der Pluszeichen nimmt also beim Pumpen zu.
Die Anzahl der Pluszeichen ist nach dem Pumpen somit grösser als $N$, also
mindestens $N+1$, mindestens so gross wie die Anzahl der öffnenden oder
schliessenden Klammern.
In der Sprache sind aber nur Wörter, die weniger Pluszeichen als öffnende oder
schliessende Klammern erhalten.
Auch in diesem Fall ist also das gepumpte Wort nicht mehr in der Sprache.
\item
Der Widerspruch zeigt, dass die Annahme, die Sprache $L$ sei kontextfrei,
nicht haltbar ist,
die Sprache $L$ ist nicht kontext.
\qedhere
\end{enumerate}
\end{loesung}
