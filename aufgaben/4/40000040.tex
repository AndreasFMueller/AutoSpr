Homogene Polynome sind Summen von Monomen, die alle den selben Grad haben.
So sind zum Beispiel 
\[
x+y, xy+yz+zx, xyz+yzt+ztx
\]
alle homogen, während
\[
x+xy+xyz, abc+d
\]
nicht homogen sind. Gibt es eine Grammatik, die genau die homogenen Polynome
erzeugt?

\thema{Grammatik}
\thema{Pumping Lemma für kontextfreie Sprachen}

\begin{hinweis}
Das Polynom $x^2+2y^2$ ist ebenfalls homogen, es kann aber auch als $xx+2yy$
geschrieben werden. 
Ignorieren Sie zur Vereinfachung die Möglichkeit von Exponenten,
und nehmen sie an, dass im Input Potenzen wie $x^n$ bereits als
$n$ Faktoren $x\cdots x$ ausgeschrieben wurden.
Ignorieren Sie ausserdem Koeffizienten vor den Monomen.
\end{hinweis}

\begin{loesung}
Polynome (mit den Vereinfachungen im Hinweis) bilden eine Sprache
über dem Alphabet bestehend aus
den Variablensymbolen und $\{\texttt{+},\texttt{-}\}$.
Homogene Polynome sind
dann gleich lange Wörter aus Variablensymbolen, die durch
$\texttt{+}$ oder $\texttt{-}$ verbunden werden. Wir bezeichnen die
Sprache der homogenen Polynome mit $L$.

Diese Sprache $L$ ist nicht kontextfrei, wie man mit dem Pumping Lemma
für kontextfreie Sprachen zeigen kann.

Wir nehmen an, $L$ sei kontextfrei. Nach dem Pumping Lemma
gibt es eine Zahl $N$, so dass Wörter, die länger als $N$ sind,
``aufpumpbar'' sind. Wir konstruieren das Wort
\[
w=\texttt{x}^N\texttt{+y}^N\texttt{+z}^N
\]
welches sicher in $L$ ist.

Nach dem Pumping Lemma hat dieses Wort eine Aufteilung
$w=uvxyz$ mit $|vxy|\le N$, das Teilwort $vxy$ kann sich also 
höchstens über zwei der Monome in $w$ erstrecken.
Beim Aufpumpen der Teile $v$ und $y$ gemäss der Formel
$w_k=uv^kxy^kz$ wird sich also die Zahl der Variablen in diesen
beiden Monomen ändern, nicht aber die Zahl der Variablen im 
verbleibenden Monom. Insbesondere sind die aufgepumpten Wörter $w_k$
nicht mehr in $L$, im Widerspruch zur Aussage des Pumping Lemma.
Dieser Widerspruch zeigt, dass $L$ nicht kontextfrei sein kann.
\end{loesung}

\begin{diskussion}
Es gibt in diesem Beweis noch einen etwas subtilen Schritt, der einer
besonderen Diskussion würdig ist. Es ist nämlich eine Aufteilung
$w={\color{blue}u}{\color{red}v}{\color{green}x}yz$ denkbar,
die möglicherweise doch aufpumpbar ist:
\begin{center}
\includeagraphics[width=315pt]{p-1.pdf}
\end{center}
In dieser Aufteilung besteht $\color{red}v$ aus je einem Teilstück
aus dem zweiten Monom und einem aus dem dritten Monom. Beim
Aufpumpen ändert sich also nicht nur die Anzahl der
Variablen, sondern auch die Anzahl der Monome:
\begin{center}
\includeagraphics[width=405pt]{p-2.pdf}
\end{center}
Dies funktioniert
aber nur, wenn das neu entstehende Monom wieder die richtige
Länge hat, nämlich $N$. Das neu entstehende Monom besteht
aus dem Teil von $\color{red}v$ rechts vom Zeichen \texttt{+},
gefolgt vom Teil von $\color{red}v$ links vom Zeichen \texttt{+}.
Die Länge des Monoms ist also $|v|-1$. Aber $|v|\le |vxy|\le N$,
d.~h.~ist das Monom echt kürzer als $N$. Das aufgepumpte Polynom ist
also nicht mehr homogen.
\end{diskussion}

\begin{bewertung}
Verwendung des Pumping Lemma ({\bf PL}) 1 Punkt,
Pumping Length ({\bf N}) 1 Punkt,
Konstruktion eines geeigneten Wortes ({\bf W}) 1 Punkt,
Aufteilung des Wortes ({\bf A}) 1 Punkt,
Wort fällt beim Aufpumpen aus der Sprache heraus ({\bf L}) 1 Punkt,
Schlussfolgerung, Sprache nicht kontextfrei ({\bf S}) 1 Punkt.
\end{bewertung}

