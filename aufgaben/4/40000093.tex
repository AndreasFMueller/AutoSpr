Ist die Sprache
\[
L
=
\{
wuw^t
\mid
w,u\in\Sigma^*,
|u| = |w| +1
\}
\]
über dem Alphabet $\Sigma=\{\texttt{a},\texttt{b}\}$ kontextfrei?
Dabei bedeutet $w^t$ das gespiegelte Wort von $w$.

\thema{nicht kontextfrei}
\themaL{Pumping Lemma fur kontextfreie Sprachen}{pumping Lemma für kontextfreie Sprachen}

\begin{loesung}
\definecolor{darkred}{rgb}{0.8,0,0}
\definecolor{darkgreen}{rgb}{0,0.6,0}
\def\h{0.6}
\def\laenge{
	\node at ({5*\h},{\h}) [above] {$N\mathstrut$};
	\node at ({11*\h},{\h}) [above] {$2N+1\mathstrut$};
	\node at ({16*\h},{\h}) [above] {$3N+1\mathstrut$};
}
\def\wort{
	\draw (0,0) rectangle ({16*\h},\h);
	\draw ({5*\h},0) -- ++(0,\h);
	\draw ({11*\h},0) -- ++(0,\h);
	\node at ({0.5*\h},{0.5*\h}) {\texttt{a}\strut};
	\node at ({4.5*\h},{0.5*\h}) {\texttt{a}\strut};
	\node at ({5.5*\h},{0.5*\h}) {\texttt{b}\strut};
	\node at ({10.5*\h},{0.5*\h}) {\texttt{b}\strut};
	\node at ({11.5*\h},{0.5*\h}) {\texttt{a}\strut};
	\node at ({15.5*\h},{0.5*\h}) {\texttt{a}\strut};
}
\def\aufteilung#1#2#3#4{
	\fill[color=blue!30,opacity=0.7]
		({0*\h+0.05},0.05) rectangle ({(#1)*\h-0.025},{\h-0.05});
	\draw[color=blue]
		({0*\h+0.05},0.05) rectangle ({(#1)*\h-0.025},{\h-0.05});
	\fill[color=darkred!30,opacity=0.7]
		({(#1)*\h+0.025},0.05) rectangle ({(#2)*\h-0.025},{\h-0.05});
	\draw[color=darkred]
		({(#1)*\h+0.025},0.05) rectangle ({(#2)*\h-0.025},{\h-0.05});
	\fill[color=darkgreen!30,opacity=0.7]
		({(#2)*\h+0.025},0.05) rectangle ({(#3)*\h-0.025},{\h-0.05});
	\draw[color=darkgreen]
		({(#2)*\h+0.025},0.05) rectangle ({(#3)*\h-0.025},{\h-0.05});
	\fill[color=darkred!30,opacity=0.7]
		({(#3)*\h+0.025},0.05) rectangle ({(#4)*\h-0.025},{\h-0.05});
	\draw[color=darkred]
		({(#3)*\h+0.025},0.05) rectangle ({(#4)*\h-0.025},{\h-0.05});
	\fill[color=blue!30,opacity=0.7]
		({(#4)*\h+0.025},0.05) rectangle ({16*\h-0.05},{\h-0.05});
	\draw[color=blue]
		({(#4)*\h+0.025},0.05) rectangle ({16*\h-0.05},{\h-0.05});
	\node[color=blue] at ({0.5*(#1)*\h},{0.5*\h}) {$x$};
	\node[color=darkred] at ({0.5*((#1)+(#2))*\h},{0.5*\h}) {$y$};
	\node[color=darkgreen] at ({0.5*((#2)+(#3))*\h},{0.5*\h}) {$z$};
	\node[color=darkred] at ({0.5*((#3)+(#4))*\h},{0.5*\h}) {$u$};
	\node[color=blue] at ({0.5*(16+(#4))*\h},{0.5*\h}) {$v$};
}
\def\nummer#1{
\node at (-0.5,{0.5*\h}) [left] {#1.\strut};
}
Die Sprache ist nicht kontextfrei, wie man mit dem Pumpinglemma für
kontextfreie Sprachen beweisen kann.
Dazu versuchen wir den Beweis wie folgt:
\begin{enumerate}
\item
Annahme: $L$ ist kontextfrei
\item
Nach dem Pumping Lemma  gibt es die Pumping Length $N$.
\item
Wähle das Wort $w=\texttt{a}^N\texttt{b}^{N+1}\texttt{a}^N$
\begin{center}
\begin{tikzpicture}[>=latex,thick]
\wort
\laenge
\end{tikzpicture}
\end{center}
\item
Nach dem Pumping-Lemma gibt es eine Aufteilung
$
w=
{\color{blue}u}
{\color{darkred}v}
{\color{darkgreen}x}
{\color{darkred}y}
{\color{blue}z}
$
mit $
|{\color{darkred}v}
{\color{darkgreen}x}
{\color{darkred}y}|
\le N$
und
$|{\color{darkred}vy}| > 0$:
\begin{center}
\begin{tikzpicture}[>=latex,thick]
\begin{scope}
\wort
\laenge
\aufteilung{0.8}{1.8}{3}{3.8}
\nummer{1}
\end{scope}
\begin{scope}[yshift=-1cm]
\wort
\aufteilung{3.8}{4.8}{5.4}{6.8}
\nummer{2}
\end{scope}
\begin{scope}[yshift=-2cm]
\wort
\aufteilung{6.8}{7.8}{8.4}{9.8}
\nummer{3}
\end{scope}
\begin{scope}[yshift=-3cm]
\wort
\aufteilung{8.8}{9.8}{11.4}{12.8}
\nummer{4}
\end{scope}
\begin{scope}[yshift=-4cm]
\wort
\aufteilung{11.8}{12.8}{13.4}{14.8}
\nummer{5}
\end{scope}
\end{tikzpicture}
\end{center}
\item
Die roten Teile können in höchstens zwei der Blöcke liegen, nicht in
dreien.
Wenn beim Pumpen die Anzahl der \texttt{b} ändert, passt die Länge nicht
mehr zum Block von \texttt{a}s, der vom Pumpen unberührt ist.
Es bleibt daher nur die Möglichkeit, dass nur einer der Blöcke von
\texttt{a}s gepumpt wird, dann passt aber die Länge der beiden Blöcke von
\texttt{a}s nicht mehr zusammen.
Das gepumpte Wort ist also nicht mehr in $L$.
\item
Dieser Widerspruch zeigt, dass die Sprache $L$ nicht kontextfrei sein kann.
\end{enumerate}
Mit diesem Beweis gibt es aber ein Problem.
Für das Wort gibt es nämlich eine Aufteilung, die pumpbar ist.
Der Grund ist, dass die \texttt{b} auch Teil von $w$ sein könnten.
Man muss nur sicherstellen, dass nur der Mittelteil $u$ wachsen kann.
Dies ist möglich, wenn ${\color{darkred}y}$ und  ${\color{darkred}v}$
beide im Mittelteil drin liegen und ausserdem zusammen eine durch drei
teilbare Länge haben.
Wächst der Mittelteil um 3 Zeichen, dann müssen die äussersten
beiden Zeichen des Mittelteils jetzt neu zu $w$ bzw.~$w^t$ gezählt
werden.

Aus diesem Gegenbeispiel zum Beweis kann man jetzt aber ableiten, wie
man das Wort verbessern kann, damit es auch das Gegenbeispiel abdeckt.
Die beiden Zeichen am Ende des Mittelteils müssen verschieden sein.
Wenn der Mittelteil grösser wird, dann geht die Symmetrie der Teile
$w$ und $w^t$ verloren.
Im Wort
\begin{center}
\begin{tikzpicture}[>=latex,thick]
\fill[color=gray] (0,0) rectangle ({25*\h},\h);
\fill[color=gray!20] (0,0) rectangle ({4*\h},\h);
\fill[color=gray!20] ({8*\h},0) rectangle ({(8+1)*\h},\h);
\fill[color=gray!20] ({21*\h},0) rectangle ({25*\h},\h);
\fill[color=gray!20] ({15*\h},0) rectangle ({16*\h},\h);
\draw (0,0) rectangle ({25*\h},\h);
\draw ({4*\h},0) -- ++(0,\h);
\draw ({8*\h},0) -- ++(0,\h);
\draw ({9*\h},0) -- ++(0,\h);
\draw ({10*\h},0) -- ++(0,\h);
\draw ({15*\h},0) -- ++(0,\h);
\draw ({16*\h},0) -- ++(0,\h);
\draw ({17*\h},0) -- ++(0,\h);
\draw ({21*\h},0) -- ++(0,\h);
\draw[line width=0.3pt] ({8\h},\h) ++ (0,0.2);
\draw[line width=0.3pt] ({17\h},\h) ++ (0,0.2);
\node at ({8*\h},\h) [above] {$n\mathstrut$};
\node at ({17*\h},\h) [above] {$2n+1\mathstrut$};
\node at ({25*\h},\h) [above] {$3n+1\mathstrut$};
\draw[->,color=orange,line width=3pt] ({8*\h},-\h) -- ++({-8*\h},0);
\draw[->,color=orange,line width=3pt] ({17*\h},-\h) -- ++({8*\h},0);
\draw[->,color=magenta,line width=3pt] ({8*\h},-\h) -- ++({9*\h},0);
\end{tikzpicture}
\end{center}
sind die orangen Teile spiegelsymmetrisch, der magenta Teil aber nicht.
Wenn die Pumpoperation den Mittelteil vergrössert, dann müssen auch
die Teile $w$ und $w^t$ vergrössert werden, wodurch sie in den
nichtsymmetrischen Teil hineinragen.
Und wenn die anderen Teile gepumpt werden, geht dies Symmetrie ebenfalls
verloren.
\qedhere
\end{loesung}

\begin{bewertung}
Annahme $L$ kontextfrei ({\bf PL}) 1 Punkt,
Pumping Length ({\bf N}) 1 Punkt,
geeignetes Wort wählen ({\bf W}) 1 Punkt,
Unterteilung des Wortes ({\bf U}) 1 Punkt,
Widerspruch beim Pumpen ({\bf P}) 1 Punkt,
Schlussfolgerung ({\bf S}) 1 Punkt.
\end{bewertung}

