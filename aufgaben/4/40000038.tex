In der Vorlesung wurde gezeigt, dass die Sprache
$\{\texttt{a}\mathstrut^n\texttt{b}\mathstrut^n\texttt{c}\mathstrut^n\;|\;n\ge 0\}$
über dem Alphabet $\Sigma=\{\texttt{a},\texttt{b},\texttt{c}\}$
nicht kontextfrei ist.
Wenn man die Bedingungen etwas lockert, und nur noch verlangt, dass die
Anzahl der verschiedenen Zeichen übereinstimmt, die Reihenfolge aber
beliebig sein darf, erhält man die Sprache
\[
L=\{w\in\Sigma^*\;|\;|w|_\texttt{a}=|w|_\texttt{b}=|w|_\texttt{c}\}.
\]
Ist $L$ kontextfrei?

\thema{kontextfrei}
\themaL{Pumping Lemma fur kontextfreie Sprachen}{Pumping Lemma für kontextfreie Sprachen}

\begin{loesung}
\definecolor{darkred}{rgb}{0.8,0,0}
\definecolor{darkgreen}{rgb}{0,0.6,0}
Nein, auch $L$ ist nicht kontextfrei.
Man kann dies mit dem Pumping-Lemma für kontextfreie Sprachen nachweisen,
wobei man sogar das gleiche Beispielwort verwenden kann wie im Falle der
Sprache $\{ \texttt{a}\mathstrut^n \texttt{b}\mathstrut^n \texttt{c}\mathstrut^n\;|\; n \ge 0\}$ .

\ding{182}~%
Man nimmt dazu an, dass $L$ kontextfrei sei.
\ding{183}~%
Gemäss Pumping-Lemma gibt es daher die Pumping-Length $N$.
\ding{184}~%
Wir konstruieren jetzt das Wort
\begin{center}
\begin{tikzpicture}[>=latex,thick]
\draw (0,0) rectangle (12,0.6);
\draw (4,0) -- (4,0.6);
\draw (8,0) -- (8,0.6);
\node at (2,0.25) {$\texttt{a}\mathstrut^N$};
\node at (6,0.25) {$\texttt{b}\mathstrut^N$};
\node at (10,0.25) {$\texttt{c}\mathstrut^N$};
\node at (0,0.25) [left] {$w=\mathstrut$};
\node at (12,0.25) [right] {$,\mathstrut$};
\node at (4,0.6) [above] {$N$};
\node at (8,0.6) [above] {$2N$};
\node at (12,0.6) [above] {$3N$};
\node at (0,0.25) [left] {\phantom{$
{\color{blue}u}
{\color{darkred}v}
{\color{darkgreen}x}
{\color{darkred}y}
{\color{blue}z}
=\mathstrut$}};
\end{tikzpicture}
\end{center}
welches die Voraussetzungen
des Pumping-Lemma sicher erfüllt.
\ding{185}~%
Es muss also eine Unterteilung $w=uvxyz$ geben, so dass $|vxy|\le N$ ist.
\begin{center}
\begin{tikzpicture}[>=latex,thick]
\draw (0,0) rectangle (12,0.6);
\draw (4,0) -- (4,0.6);
\draw (8,0) -- (8,0.6);
\node at (2,0.25) {$\texttt{a}^N\mathstrut$};
\node at (6,0.25) {$\texttt{b}^N\mathstrut$};
\node at (10,0.25) {$\texttt{c}^N\mathstrut$};
\node at (12,0.25) [right] {$.\mathstrut$};
\node at (0,0.25) [left] {$
{\color{blue}u}
{\color{darkred}v}
{\color{darkgreen}x}
{\color{darkred}y}
{\color{blue}z}
=\mathstrut$};
\node at (4,0.6) [above] {$N$};
\node at (8,0.6) [above] {$2N$};
\node at (12,0.6) [above] {$3N$};
\def\v{2.4}
\def\x{3.8}
\def\y{4.5}
\def\z{5.2}

\fill[color=blue!20,opacity=0.5]
	(0.05,0.05) rectangle ({\v-0.025},0.55);
\draw[color=blue]
	(0.05,0.05) rectangle ({\v-0.025},0.55);
\node[color=blue] at ({0.5*\v},0.3) {$u\mathstrut$};

\fill[color=darkred!20,opacity=0.5]
	({\v+0.025},0.05) rectangle ({\x-0.025},0.55);
\draw[color=darkred]
	({\v+0.025},0.05) rectangle ({\x-0.025},0.55);
\node[color=darkred] at ({0.5*(\v+\x)},0.3) {$v\mathstrut$};

\fill[color=darkgreen!20,opacity=0.5]
	({\x+0.025},0.05) rectangle ({\y-0.025},0.55);
\draw[color=darkgreen]
	({\x+0.025},0.05) rectangle ({\y-0.025},0.55);
\node[color=darkgreen] at ({0.5*(\x+\y)},0.3) {$x\mathstrut$};

\fill[color=darkred!20,opacity=0.5]
	({\y+0.025},0.05) rectangle ({\z-0.025},0.55);
\draw[color=darkred]
	({\y+0.025},0.05) rectangle ({\z-0.025},0.55);
\node[color=darkred] at ({0.5*(\y+\z)},0.3) {$y\mathstrut$};

\fill[color=blue!20,opacity=0.5]
	({\z+0.025},0.05) rectangle (11.95,0.55);
\draw[color=blue]
	({\z+0.025},0.05) rectangle (11.95,0.55);
\node[color=blue] at ({0.5*(\z+12)},0.3) {$z\mathstrut$};

\draw[line width=0.2pt] (\v,0) -- (\v,-0.4);
\draw[line width=0.2pt] (\z,0) -- (\z,-0.4);
\draw[<->] (\v,-0.3) -- (\z,-0.3);
\node at ({0.5*(\v+\z)},-0.3) [below] {$\le N$};
\end{tikzpicture}
\end{center}
Die Blöcke ${\color{darkred}v}{\color{darkgreen}x}{\color{darkred}y}$
können auch anderswo liegen, dürfen aber zusammen nicht länger als $N$ sein.
\ding{186}~%
Diese letzte Bedingung hat zur Folge, dass $v$ und $y$ höchstens in
zwei der drei Blöcke $\texttt{a}^N$, $\texttt{b}^N$ oder $\texttt{c}^N$
liegen können.
Insbesondere ändert sich beim Pumpen nur die Anzahl von höchstens zwei der drei
Buchstaben, nach auf- oder abpumpen erhält man also ein Wort, bei dem 
die Anzahl jedes der drei Buchstaben nicht mehr gleich ist, also kein
Wort mehr aus $L$, im Widerspruch zur Aussage des Pumping-Lemma.
\ding{187}~%
Daher kann $L$ nicht kontextfrei sein.
\end{loesung}

\begin{diskussion}
Es genügt nicht, darauf zu verweisen, dass $L$ die Sprache
$L'
=
\{
\texttt{a}\mathstrut^n \texttt{b}\mathstrut^n \texttt{c}\mathstrut^n
\mid n\ge 0
\}$
enthalte, und dass daher auch $L'$ nicht kontextfrei sein.
Dann könnte man nämlich auch argumentieren, dass $L$ die kontextfreie Sprache
$\emptyset$ enthalte, und daher kontextfreie sei!

Wenn man sich den Pumping-Lemma-Beweis genauer anschaut, kann man auch verstehen,
warum es nicht reicht. Der Beweis beruht ja darauf, dass aufgepumpte Wörter nicht
mehr in der Sprache sind. Wenn die Sprache aber mehr Wörter umfasst, dann ist
es auch schwieriger ``aus der Sprache herauszufallen'', es ist ja jetzt leichter,
die Bedingung der Sprache zu erfüllen. Damit ist es möglich, dass ein Wort, welches
in $L'$ nicht aufgepumpt werden kann, in $L$ aufpumpbar wird.
\end{diskussion}

\begin{bewertung}
Pumping Lemma für kontextfreie Sprachen ({\bf PL}) 1 Punkt,
Pumping Length ({\bf N}) 1 Punkt,
Beispielwort unter Verwendung von $N$ ({\bf W}) 1 Punkt,
Zerlegung des Wortes ({\bf Z}) 1 Punkt,
Konsequenzen beim Auf- bzw.~Abpumpen ({\bf A}) 1 Punkt,
Schlussfolgerung ({\bf S}) 1 Punkt.
\end{bewertung}

