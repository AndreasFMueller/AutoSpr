In einer HTML-Tabelle sollte man in jedem {\tt <tr>}-Element gleich
viele {\tt <td>}-Element angeben (sofern man nicht {\tt <td>}-Elemente
verwendet, die sich über mehrer Spalten oder Zeilen erstrecken).
Es stellt sich die Frage, ob man dies durch eine kontextfreie Grammatik
sicherstellen könnte.
Betrachten Sie dazu die folgende vereinfachte Sprache $L$ über dem Alphabet
$\Sigma=\{\text{\tt r}, \text{\tt d}\}$. Die Buchstaben sollen
dabei für {\tt <tr>}-Element und {\tt <td>}-Element stehen.
Die Wörter von $L$ sollen die Form
\[
(\text{\tt rd}^n)^k
\]
haben. Damit ist gemeint, dass jedes Wort aus $k$ Teilstücken besteht,
die alle identisch sind und jeweils aus einem {\tt r} und $n$ Zeichen
{\tt d} bestehen. Die Wörter
\[
\text{\tt rrr}, \quad
\text{\tt rddrddrdd},\quad
\text{\tt rdddddddrddddddd},\quad
\text{\tt rddddddddddddddd}
\]
gehören also zu $L$, die Wörter
\[
\text{\tt rdr}, \quad
\text{\tt rdrddrddd},\quad
\text{\tt rdddddddr},\quad
\text{\tt dddddddddddddddr}
\]
aber nicht.
Gibt es für diese Sprache eine kontextfreie Grammatik?

\thema{Grammatik}

\begin{loesung}
Nein, wie man mit dem Pumping Lemma für kontextfreie Sprachen zeigen kann.

Angenommen $L$ sei kontextfrei, dann gibt es eine Pumping Length $N$, so
dass Wörter mit mindestens Länge $N$ aufgepumpt werden können. 
Wir wählen als Wort
\[
w=\text{\tt r}
\underbrace{\text{\tt d}\dots\text{\tt d}}_{N}
\text{\tt r}
\underbrace{\text{\tt d}\dots\text{\tt d}}_{N}
\text{\tt r}
\underbrace{\text{\tt d}\dots\text{\tt d}}_{N}
\]
Es hat Länge $3N+3$, es muss also eine Zerlegung $w=uvxyz$ geben
so, dass $|vxy|\le N$ ist. Wie auch immer diese Zerlegung aussieht,
$v$ oder $y$, vielleicht sogar beide, werden ausschliesslich aus {\tt d}
bestehen. Ausserdem können $v$ und $y$ höchstens zwei der Blöcke
aus Zeichen $d$ im Wort $w$ überlappen. Der erste oder der letzte Block
ist vollständig in $u$ bzw.~$z$ enthalten. Insbesondere müssen in
einem aufgepumpten Wort immer noch alle Blöcke aus Zeichen {\tt d}
gleich lang sein, damit es immer noch in $l$ ist.

Nehmen wir an, $v$ bestehe aus lauter {\tt d}. Dann wird sich beim
Aufpumpen ``sein'' Block verlängern, im Widerspruch zum eben gesagten.
"Ahnlich ergibt sich auch ein Widerspruch, wenn  $y$ aus lauter {\tt d}
besteht.

Diese Widersprüche zeigen, dass die Sprache $L$ nicht kontextfrei sein
kann.
\end{loesung}

\begin{bewertung}
Pumping Lemma für kontextfreie Sprachen ({\bf PL}) 1 Punkt,
Pumping Length ({\bf N}) 1 Punkt,
Beispielwort unter Verwendung von $N$ ({\bf W}) 1 Punkt,
Zerlegung des Wortes ({\bf Z}) 1 Punkt,
Konsequenzen beim Auf- bzw.~Abpumpen ({\bf A}) 1 Punkt,
Schlussfolgerung ({\bf S}) 1 Punkt.
\end{bewertung}
