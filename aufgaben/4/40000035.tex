In einer HTML-Tabelle sollte man in jedem {\tt <tr>}-Element gleich
viele {\tt <td>}-Element angeben (sofern man nicht {\tt <td>}-Elemente
verwendet, die sich "uber mehrer Spalten oder Zeilen erstrecken).
Es stellt sich die Frage, ob man dies durch eine kontextfreie Grammatik
sicherstellen k"onnte.
Betrachten Sie dazu die folgende vereinfachte Sprache $L$ "uber dem Alphabet
$\Sigma=\{\text{\tt r}, \text{\tt d}\}$. Die Buchstaben sollen
dabei f"ur {\tt <tr>}-Element und {\tt <td>}-Element stehen.
Die W"orter von $L$ sollen die Form
\[
(\text{\tt rd}^n)^k
\]
haben. Damit ist gemeint, dass jedes Wort aus $k$ Teilst"ucken besteht,
die alle identisch sind und jeweils aus einem {\tt r} und $n$ Zeichen
{\tt d} bestehen. Die W"orter
\[
\text{\tt rrr}, \quad
\text{\tt rddrddrdd},\quad
\text{\tt rdddddddrddddddd},\quad
\text{\tt rddddddddddddddd}
\]
geh"oren also zu $L$, die W"orter
\[
\text{\tt rdr}, \quad
\text{\tt rdrddrddd},\quad
\text{\tt rdddddddr},\quad
\text{\tt dddddddddddddddr}
\]
aber nicht.
Gibt es f"ur diese Sprache eine kontextfreie Grammatik?

\begin{loesung}
Nein, wie man mit dem Pumping Lemma f"ur kontextfreie Sprachen zeigen kann.

Angenommen $L$ sei kontextfrei, dann gibt es eine Pumping Length $N$, so
dass W"orter mit mindestens L"ange $N$ aufgepumpt werden k"onnen. 
Wir w"ahlen als Wort
\[
w=\text{\tt r}
\underbrace{\text{\tt d}\dots\text{\tt d}}_{N}
\text{\tt r}
\underbrace{\text{\tt d}\dots\text{\tt d}}_{N}
\text{\tt r}
\underbrace{\text{\tt d}\dots\text{\tt d}}_{N}
\]
Es hat L"ange $3N+3$, es muss also eine Zerlegung $w=uvxyz$ geben
so, dass $|vxy|\le N$ ist. Wie auch immer diese Zerlegung aussieht,
$v$ oder $y$, vielleicht sogar beide, werden ausschliesslich aus {\tt d}
bestehen. Ausserdem k"onnen $v$ und $y$ h"ochstens zwei der Bl"ocke
aus Zeichen $d$ im Wort $w$ "uberlappen. Der erste oder der letzte Block
ist vollst"andig in $u$ bzw.~$z$ enthalten. Insbesondere m"ussen in
einem aufgepumpten Wort immer noch alle Bl"ocke aus Zeichen {\tt d}
gleich lang sein, damit es immer noch in $l$ ist.

Nehmen wir an, $v$ bestehe aus lauter {\tt d}. Dann wird sich beim
Aufpumpen ``sein'' Block verl"angern, im Widerspruch zum eben gesagten.
"Ahnlich ergibt sich auch ein Widerspruch, wenn  $y$ aus lauter {\tt d}
besteht.

Diese Widerspr"uche zeigen, dass die Sprache $L$ nicht kontextfrei sein
kann.
\end{loesung}

\begin{bewertung}
Pumping Lemma f"ur kontextfreie Sprachen ({\bf PL}) 1 Punkt,
Pumping Length ({\bf N}) 1 Punkt,
Beispielwort unter Verwendung von $N$ ({\bf W}) 1 Punkt,
Zerlegung des Wortes ({\bf Z}) 1 Punkt,
Konsequenzen beim Auf- bzw.~Abpumpen ({\bf A}) 1 Punkt,
Schlussfolgerung ({\bf S}) 1 Punkt.
\end{bewertung}
