Finden Sie einen Stackautomaten, der korrekte Klammerausdrücke erkennt.

\begin{loesung}
Der Stackautomat muss überprüfen, ob es zu jeder öffnenden Klammer auch
wieder eine schliessende Klammer gibt.
Der folgende Automat schafft dies:
\begin{center}
\def\l{3}
\def\r{0.4}
\def\zustand#1#2{
	\draw #1 circle[radius=\r];
	\node at #1 {$#2\mathstrut$};
}
\def\akzeptierzustand#1#2{
	\zustand{#1}{#2}
	\draw #1 circle[radius={\r-0.05}];
}
\def\pfeil#1#2{\draw[->,shorten >= 0.4cm,shorten <= 0.4cm] #1 -- #2;}
\begin{tikzpicture}[>=latex,thick]
\coordinate (q0) at (0,0);
\coordinate (q1) at (\l,{-0.5*\l});
\coordinate (q2) at (0,-\l);
\zustand{(q0)}{q_0}
\zustand{(q1)}{q_1}
\akzeptierzustand{(q2)}{q_2}
\pfeil{(-2,0)}{(q0)}
\pfeil{(q0)}{(q1)}
\pfeil{(q1)}{(q2)}

\node at ($0.3*(q1)+0.7*(q2)$) [below right]
	{$\varepsilon,\texttt{\$}\to\varepsilon$};

\node at ($0.7*(q0)+0.3*(q1)$) [above right]
	{$\varepsilon,\varepsilon\to\texttt{\$}$};

\draw[->,shorten >= 0.4cm,shorten <= 0.4cm]
	(q1) to[out=20,in=60,distance=2cm] (q1);
\node at ($(q1)+(1,1)$) [above right]
	{$\texttt{(},\varepsilon\to\texttt{(}$};

\draw[->,shorten >= 0.4cm,shorten <= 0.4cm]
	(q1) to[out=-20,in=-60,distance=2cm] (q1);
\node at ($(q1)+(1,-1)$) [below right]
	{$\texttt{)},\texttt{(}\to\varepsilon$};
\end{tikzpicture}
\end{center}
\end{loesung}
