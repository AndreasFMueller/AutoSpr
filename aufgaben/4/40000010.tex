Sei $\Sigma=\{0,1\}$.
Wir sagen, ein Wort in $\Sigma^*$ sei ``wachsend'',
wenn auf jede Folge von Nullen eine mindestens
so lange Folge von Einsen folgt.
Betrachten Sie die Sprache
\[
L=\{
w\in \Sigma^*\,|\, \text{$w$ ist wachsend}
\}
\]
\begin{teilaufgaben}
\item Ist $L$ regul"ar?
\item Ist $L$ kontextfrei?
\end{teilaufgaben}

\begin{loesung}
\begin{teilaufgaben}
\item
$L$ ist nicht regul"ar, wie man mit dem Pumping-Lemma beweisen kann.
Dazu nimmt man an, $L$ sei regul"ar. Das Pumping-Lemma garantiert, dass
es eine Zahl $N$ gibt, die Pumping-Length, so dass W"orter mit gr"osserer
L"ange aufgepumpt werden können. Das Wort $0^N1^N$ ist wachsend, denn
auf die Folge von $N$ Nullen folgen $N\ge N$ Einsen. Nach dem Pumping-Lemma
kann $w=xyz$ geschrieben werden, mit $|xy|\le N$ und $|y|>0$. Insbesondere bestehen
$x$ und $y$ ausschliesslich aus Nullen. Das aufgepumpte Wort
$xy^kz$ besteht also aus $N+|y|(k-1)$ Nullen, gefolgt von $N$ Einsen.
Da $N+|y|(k-1) > N$ f"ur $k>1$, ist das aufgepumpte Wort nicht nicht
mehr wachsend, im Widerspruch zur Behauptung des Pumping-Lemmas. Der
Widersprucht zeigt, dass die Voraussetzung nicht zutreffen kann, $L$
ist also nicht regul"ar.
\item
Wachsende W"orter bestehen aus einzelnen W"ortern der Form
$w={\tt 0}^k{\tt 1}^{k+s}$.  Das Wort $w$ kann man erzeugen,
indem man zuerst das leere Wort in {\tt 0} und {\tt 1}
``einschachtelt'', und dann noch eine Anzahl von Einsen anh"angt.
Damit haben wir aber in informeller Form alle Produktionsregeln
f"ur W"orter der Sprache $L$ zusammengefasst.

Das Symbol $W$ stehe f"ur W"orter wie $w$, dann kann man beliebige W"orter der
Sprache zusammensetzen, in dem man solchen W"orter weitere anh"angt:
\begin{align*}
S_0&\to \varepsilon\\
   &\to S\\
S&\to A\\
 &\to SA
\end{align*}
Die Teile $A$ entstehen wir folgt:
\begin{align*}
A&\to W\\
 &\to A{\tt 1}\\
W&\to {\tt01}\\
 &\to {\tt 0}W{\tt 1}\\
\qedhere
\end{align*}
\end{teilaufgaben}
\end{loesung}
