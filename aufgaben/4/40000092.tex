Ist die Sprache
\[
L
=
\{
wuw^t
\mid
w,u\in\Sigma^*\wedge |u|\le 3
\}
\]
über dem Alphabet $\Sigma=\{\texttt{a},\texttt{b}\}$ kontextfrei?
Dabei bezeichnet $w^t$ das gespiegelte Wort von $w$.
Die Sprache $L$ besteht aus Wörtern, die bis auf einen Block der Länge
$\le 3$ genau in der Mitte spiegelsymmetrisch ist.

\begin{loesung}
Die Sprache ist kontextfrei, weil die folgende kontextfreie Grammatik
die Sprache erzeugen kann:
\begin{align*}
S&\to \texttt{a} S \texttt{a} \mid \texttt{b} S \texttt{b} \mid U\\
U&\to \varepsilon \mid A \mid AA \mid AAA \\
A&\to \texttt{a} \mid \texttt{b}
\end{align*}
Die Variable $U$ steht für den Teil $u$ des Wortes.

Alternativ kann man auch einen Stackautomaten angeben, der die Sprache
akzeptiert.
Im Zustandsdiagramm
\begin{center}
\def\l{2.5}
\begin{tikzpicture}[>=latex,thick]
\coordinate (q0) at ({0*\l},{1*\l});
\coordinate (q1) at ({1*\l},{1*\l});
\coordinate (q2) at ({1*\l},{-1*\l});
\coordinate (q3) at ({0*\l},{-1*\l});
\coordinate (u0) at ({1*\l},{0*\l});
\coordinate (u1) at ({2*\l},{0*\l});
\coordinate (u2) at ({3*\l},{0*\l});
\coordinate (u3) at ({4*\l},{0*\l});
\draw (q0) circle[radius=0.35];
\draw (q1) circle[radius=0.35];
\draw (q2) circle[radius=0.35];
\draw (q3) circle[radius=0.35];
\draw (q3) circle[radius=0.30];
\draw (u0) circle[radius=0.35];
\draw (u1) circle[radius=0.35];
\draw (u2) circle[radius=0.35];
\draw (u3) circle[radius=0.35];
\node at (q0) {$q_0$};
\node at (q1) {$q_1$};
\node at (q2) {$q_2$};
\node at (q3) {$q_3$};
\node at (u0) {$u_0$};
\node at (u1) {$u_1$};
\node at (u2) {$u_2$};
\node at (u3) {$u_3$};
\draw[->,shorten >= 0.35cm] ($(q0)+(-1.5,0)$) -- (q0);
\draw[->,shorten >= 0.35cm,shorten <= 0.35cm] (q0) -- (q1);
\draw[->,shorten >= 0.35cm,shorten <= 0.35cm] (q1) -- (u0);
\draw[->,shorten >= 0.35cm,shorten <= 0.35cm] (q1) -- (u1);
\draw[->,shorten >= 0.35cm,shorten <= 0.35cm] (q1) -- (u2);
\draw[->,shorten >= 0.35cm,shorten <= 0.35cm] (q1) -- (u3);
\draw[->,shorten >= 0.35cm,shorten <= 0.35cm] (u1) -- (u0);
\draw[->,shorten >= 0.35cm,shorten <= 0.35cm] (u2) -- (u1);
\draw[->,shorten >= 0.35cm,shorten <= 0.35cm] (u3) -- (u2);
\draw[->,shorten >= 0.35cm,shorten <= 0.35cm] (u0) -- (q2);
\draw[->,shorten >= 0.35cm,shorten <= 0.35cm] (q2) -- (q3);
\draw[->,shorten >= 0.35cm,shorten <= 0.35cm]
	(q1) to[out=0,in=60,distance=1.5cm] (q1);
\node at ($(q1)+(1,0.4)$) [right] {$
\displaystyle
\texttt{a},\varepsilon\to\texttt{a}
\atop
\displaystyle
\texttt{b},\varepsilon\to\texttt{b}
$};
\draw[->,shorten >= 0.35cm,shorten <= 0.35cm]
	(q2) to[out=0,in=60,distance=1.5cm] (q2);
\node at ($(q2)+(1,0.4)$) [right] {$
\displaystyle
\texttt{a},\texttt{a}\to\varepsilon
\atop
\displaystyle
\texttt{b},\texttt{b}\to\varepsilon
$};

\node at ($0.5*(q0)+0.5*(q1)$)
	[above] {$\varepsilon,\varepsilon\to\texttt{\$}$};
\node at ($0.5*(q2)+0.5*(q3)$)
	[above] {$\varepsilon,\texttt{\$}\to\varepsilon$};
\node at ($0.5*(u0)+0.5*(u1)$) [below] {$
\displaystyle
\texttt{a},\varepsilon\to\varepsilon
\atop
\displaystyle
\texttt{b},\varepsilon\to\varepsilon
$};
\node at ($0.5*(u1)+0.5*(u2)$) [below] {$
\displaystyle
\texttt{a},\varepsilon\to\varepsilon
\atop
\displaystyle
\texttt{b},\varepsilon\to\varepsilon
$};
\node at ($0.5*(u2)+0.5*(u3)$) [below] {$
\displaystyle
\texttt{a},\varepsilon\to\varepsilon
\atop
\displaystyle
\texttt{b},\varepsilon\to\varepsilon
$};
\end{tikzpicture}
\end{center}
nicht beschriftete Zustandsübergänge als $\varepsilon$-Übergänge der
Form $\varepsilon,\varepsilon\to\varepsilon$ zu lesen.
\end{loesung}

\begin{bewertung}
Grammatik für $A$ ({\bf A}) 1 Punkt,
Schachtelungsidee ({\bf S}) 2 Punkt,
der mittlere Teil kann leer sein ({\bf U}) 1 Punkt,
der mittlere Teil kann aus \texttt{a} und \texttt{b} bestehen ({\bf B}) 1 Punkt,
Schlussfolgerung kontextfrei ({\bf K}) 1 Punkt.
\end{bewertung}

