Nach dem Erfolg im Erkennen von Tupel-Additionen in
Aufgabe~\ref{40000023}
will Heiri Muster nun den Parser erweitern, um Tupel-Gleichungen
zu erkennen, also Gleichungen der Form
\[
(3,47,1291,4711)+(4,1848,2010,-1)=(7,1895,3301,4710)
\]
Ist dies möglich?

\themaL{Pumping Lemma fur kontextfreie Sprachen}{Pumping Lemma für kontextfreie Sprachen}

\begin{loesung}
Schreiben wir $m$ für jede Zahl, müssen wir also in der Lage
sein, einen Ausdruck der Form
\begin{equation}
(m,m,m,m)+(m,m,m,m)=(m,m,m,m)
\label{tupelgleichungen}
\end{equation}
zu erkennen, wobei jede Klammer gleich viele $m$ enthalten muss.

Nehmen wir an, die Sprache $L$ dieser Ausdrücke sei kontextfrei.
Dann gibt
es eine Länge $N$, so dass in Tuppelausdruck $w$ mit mindestens
$N$ Zeichen zerlegt werden kann in $w=uvxyz$ so, dass $|vxy|<N$,
$|vy|>0$ und $uv^ixy^iz\in L$ für beliebige $i$. Wählen wir jetzt
einen Ausdruck mit mehr als $N$ Komponenten in jeder Klammer,
dann muss das Teilwort $vxy$ ganz in dem Teil vor dem $=$ oder ganz
in dem Teil nach dem $+$ enthalten sein. Da ausserdem $|vy|>0$
muss mindestens $v$ oder $y$ eine $m$ enthalten, ein Komma alleine
können beide nicht enthalten, denn dadurch würde für $i>0$
ein illegaler Ausdruck entstehen. Ist $vxy$ im Teil vor dem $=$
enthalten, dann ändert sich durch das Aufpumpen mit $i>1$ also
die Zahl der $m$ in den Klammern auf der linken Seite, auf
der rechten Seite ändert sie sich jedoch nicht. Andernfalls,
wenn $vxy$ in dem Teil nach dem $+$ enthalten ist, ändert
sich beim Aufpumpen mit $i>1$ die Zahl der $m$ in der zweiten
oder dritten Klammer, nicht aber in der ersten. In beiden Fällen
entsteht ein inkorrekter Ausdruck. Dieser Widerspruch zeigt, dass
die Sprache nicht kontextfrei ist. Es ist also nicht möglich
eine kontextfreie Grammatik zu konstruieren, welche nur korrekte
Tupelgleichungen (\ref{tupelgleichungen}) akzeptiert.
\end{loesung}
