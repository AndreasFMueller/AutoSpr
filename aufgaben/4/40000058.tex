In Aufgabe~\ref{40000030} wurde der Stackautomat
\[
\entrymodifiers={++[o][F]}
\xymatrix {
*+\txt{}\ar[r]
        &{0} \ar[r]^{\varepsilon,\varepsilon\to{\tt \$}}
                &{1}	\ar@(u,ur)^{\texttt{a},\varepsilon\to\texttt{a}}
                	\ar@(ur,r)^{\texttt{b},\varepsilon\to\texttt{b}}
			\ar@/_30pt/[dd]_{\texttt{a},\varepsilon\to\varepsilon}
			\ar@/^30pt/[dd]^{\texttt{b},\varepsilon\to\varepsilon}
			\ar[dd]^{\varepsilon,\varepsilon\to\varepsilon}
\\
*+\txt{}
	&*+\txt{}
		&*+\txt{}
\\
*+\txt{}
        &*++[o][F=]{4}
                &{3}	\ar@(r,dr)^{\texttt{a},\texttt{a}\to\varepsilon}
                	\ar@(dr,d)^{\texttt{b},\texttt{b}\to\varepsilon}
			\ar[l]^{\varepsilon,{\tt \$}\to\varepsilon}
}
\]
für Palindrome über dem Alphabet $\Sigma=\{\texttt{a},\texttt{b}\}$
konstruiert.
Leiten Sie daraus eine Grammatik für Palindrome über $\Sigma$ ab.

\thema{Grammatik aus Stackautomat}

\begin{loesung}
Gemäss dem im Skript beschriebenen Algorithmus muss der Automat zunächst
standardisiert werden.
Die einzige nötige Änderung ist, dass die vertikalen Übergange von
$1$ nach $3$ ein Zeichen auf den Stack legen müssen:
\[
\entrymodifiers={++[o][F]}
\xymatrix {
*+\txt{}\ar[r]
        &{0} \ar[r]^{\varepsilon,\varepsilon\to{\tt \$}}
                &{1}	\ar@(u,ur)^{\texttt{a},\varepsilon\to\texttt{a}}
                	\ar@(ur,r)^{\texttt{b},\varepsilon\to\texttt{b}}
			\ar@/_30pt/[d]_{\texttt{a},\varepsilon\to\texttt{x}}
			\ar@/^30pt/[d]^{\texttt{b},\varepsilon\to\texttt{x}}
			\ar[d]^{\varepsilon,\varepsilon\to\texttt{x}}
\\
*+\txt{}
	&*+\txt{}
		&{2} \ar[d]_{\varepsilon,\texttt{x}\to\varepsilon}
\\
*+\txt{}
        &*++[o][F=]{4}
                &{3}	\ar@(r,dr)^{\texttt{a},\texttt{a}\to\varepsilon}
                	\ar@(dr,d)^{\texttt{b},\texttt{b}\to\varepsilon}
			\ar[l]^{\varepsilon,{\tt \$}\to\varepsilon}
}
\]
Daraus kann man jetzt die Regeln ableiten
\begin{align*}
A_{04}&\to A_{13}
\\
A_{13}&\to \texttt{a} A_{13} \texttt{a} \\
      &\to \texttt{b} A_{13} \texttt{b} \\
      &\to \texttt{a}A_{22}\\
      &\to \texttt{b}A_{22}\\
      &\to \varepsilon A_{22}\\
A_{00}&\to\varepsilon\\
A_{11}&\to\varepsilon\\
A_{22}&\to\varepsilon\\
A_{33}&\to\varepsilon\\
A_{44}&\to\varepsilon\\
\end{align*}
Da man $A_{22}$ nur in $\varepsilon$ verwandeln kann, kann man die Variable
$A_{22}$ in den Regeln ganz weglassen.
Da man ausserdem $A_{04}$ nur in $A_{13}$ verwandeln kann, kann man auch
direkt von $A_{13}$ als Startvariable ausgehen.
Schreibt man $S=A_{13}$ wird die Grammatik zu
\begin{align*}
S&\to\texttt{a} S\texttt{a}\\
 &\to\texttt{b} S\texttt{b}\\
 &\to\texttt{a}\\
 &\to\texttt{b}\\
 &\to\varepsilon
\qedhere
\end{align*}
\end{loesung}



