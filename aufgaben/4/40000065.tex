Betrachten Sie die Grammatik
\begin{align*}
S & \to X \texttt{1} X \texttt{1} X \\
X & \to \texttt{0} X  \;|\; \texttt{1} X \;|\; \varepsilon
\end{align*}
\begin{teilaufgaben}
\item
Beschreiben Sie die Wörter, die von dieser Grammatik produziert werden können.
Schreiben Sie die Sprache in Mengennotation
$L=\{w\in\{\texttt{0},\texttt{1}\}^*\;|\dots\}$.
\item
Hat die Grammatik Chomsky-Normalform?
Wenn nicht, bringen Sie sie in Chomsky-Normalform.
\end{teilaufgaben}

\begin{loesung}
\begin{teilaufgaben}
\item
Die Regel für $X$ erzeugt beliebige Folgen von \texttt{0} und \texttt{1}.
Die Regel für S stellt sicher, dass das erzeugte Wort mindestens zwei
\texttt{1} enthält.
Die erzeugte Sprache ist also
\[
L=\{
w\in\{\texttt{0},\texttt{1}\}^*
\;|\;
|w|_{\texttt{1}} \ge 2\}.
\]
\item
Die Grammatik hat offenbar nicht Chomsky-Normalform.
Immerhin kommt die Startvariable auf der rechten Seite nicht vor,
so dass der erste Schritt des Algorithmus entfällt.

2. Schritt: Entfernung der $\varepsilon$-Regel:
\begin{align*}
S & \to
X \texttt{1} X \texttt{1} X
\;|\;
X \texttt{1} X \texttt{1} 
\;|\;
X \texttt{1}   \texttt{1} X
\;|\;
X \texttt{1}   \texttt{1} 
\;|\;
  \texttt{1} X \texttt{1} X
\;|\;
  \texttt{1} X \texttt{1}  
\;|\;
  \texttt{1}   \texttt{1} X
\;|\;
  \texttt{1}   \texttt{1}  
\\
X & \to
\texttt{0} X
\;|\;
\texttt{0}
\;|\;
\texttt{1} X
\;|\;
\texttt{1}
\end{align*}
Es gibt keine Unit-Rules, so dass der 3.~Schritt entfallen kann.

4. Schritt: Die Einsen in der ersten Regeln müssen durch Variablen
ersetzt werden, ebenso die \texttt{0} und \texttt{1} in den Zweierregeln
$X\to\texttt{0}X$ und $X\to\texttt{1}X$.
\begin{align*}
S & \to XAXAX \;|\; XAXA \;|\; XAAX \;|\; XAA \;|\; AXAX \;|\; AXA  \;|\; AAX \;|\; AA  
\\
X & \to
BX
\;|\;
\texttt{0}
\;|\;
BX
\;|\;
\texttt{1}
\\
A&\to \texttt{1}
\\
B&\to \texttt{0} \;|\; \texttt{1}
\end{align*}
Die langen Regeln in der ersten Regel müssen schrittweise aufgebaut werden:
\begin{align*}
S & \to WU \;|\; WA \;|\; VU \;|\; VA \;|\; UU \;|\; UA  \;|\; AU \;|\; AA  
\\
U & \to AX
\\
V & \to XA
\\
W & \to XU
\\
X & \to
BX
\;|\;
\texttt{0}
\;|\;
BX
\;|\;
\texttt{1}
\\
A&\to \texttt{1}
\\
B&\to \texttt{0} \;|\; \texttt{1}
\end{align*}
Damit ist Chomsky-Normalform erreicht.
\qedhere
\end{teilaufgaben}
\end{loesung}

\begin{bewertung}
\begin{teilaufgaben}
\item Sprache ({\bf L}) 2 Punkte,
\item keine Chomsky-Normalform ({\bf C}) 1 Punkt,
\item $\varepsilon$-Regeln ({$\bm \varepsilon$}) 1 Punkt,
\item Terminalsymbole ({\bf T}) 1 Punkt,
\item Verkettungen ({\bf V}) 1 Punkt.
\end{teilaufgaben}
\end{bewertung}
