In einer Programmiersprache wird unterschieden zwischen Deklaration
und Definition einer Funktion.
Definitionen und Deklarationen k"onnen beliebig gemischt werden,
solange Deklarationen immer vor Definitionen kommen.
Eine Deklaration besteht aus dem Funktionsnamen $f$ und einer Argumentliste,
wir schreiben symbolisch $a^n$  f"ur die aus $n$ Argumenten bestehende
Argumentliste.
Definitionen enthalten zus"atzlich einen Block mit ausf"uhrbarem Code.
Die Programmiersprache l"asst also Konstruktionen der Form
\[
\underbrace{f_1a^n}_{\text{Deklaration von $f_1$}}
\underbrace{f_2a^m}_{\text{Deklaration von $f_2$}}
\underbrace{f_1a^nb_1}_{\text{Definition von $f_1$}}
\underbrace{f_2a^m b_2}_{\text{Definition von $f_2$}}
\]
zu.
Kann man durch Wahl einer geeigneten Grammatik erzwingen, dass die
Anzahl der Argument Argumentliste in Deklaration und Definition jeweils 
"ubereinstimmt?




\begin{loesung}
Dies w"urde bedeuten, dass die Sprache kontextfrei ist.
Diese Sprache ist aber nicht kontextfrei, wie man mit dem Pumping Lemma
f"ur kontextfreie Sprachen beweisen kann.
\begin{enumerate}
\item
Wir nehmen an, die Sprache sei kontextfrei.
\item
Nach dem Pumping-Lemma gibt es dann die Pumping Length $N$ so,
dass W"orter mit L"ange $\ge N$ aufgepumpt werden k"onnen.
\item
Wir konstruieren das Wort
\[
w=f_1a^N f_2a^N f_1a^Nb_1f_2a^Nb_2
\]
welches ganz offensichtlich in der Sprache drin ist.
\item
Nach dem Pumping-Lemma gibt es dann eine Unterteilung des Wortes $w$ in
f"unf Teile $w=xyzuv$ derart, dass $|yzu|\le N$.
Dies bedeutet, dass $y$ und $u$ Zeichen $a$ aus h"ochstens zwei verschiedenen
$a$-Bl"ocken enthalten k"onnen.
\item
Beim Aufpumpen wird also notwendigerweise die Anzahl der $a$ einer
Deklaration und einer nicht dazu passenden Definition ver"andert,
so dass nach dem Aufpumpen die Definition nicht mehr zur Deklaration 
passt.
Das aufgepumpte Wort ist also nicht mehr in der Sprache.
\item
Dieser Widerspruch zeigt, dass die Sprache nicht kontextfrei sein kann.
\qedhere
\end{enumerate}
\end{loesung}

\begin{bewertung}
\end{bewertung}

