In einer Programmiersprache wird unterschieden zwischen Deklaration
und Definition einer Funktion.
Definitionen und Deklarationen können beliebig gemischt werden,
solange Deklarationen immer vor Definitionen kommen.
Eine Deklaration besteht aus dem Funktionsnamen $f$ und einer Argumentliste,
wir schreiben symbolisch $a^n$  für die aus $n$ Argumenten bestehende
Argumentliste.
Definitionen enthalten zusätzlich einen Block mit ausführbarem Code.
Die Programmiersprache lässt also Konstruktionen der Form
\[
\underbrace{f_1a^n}_{\text{Deklaration von $f_1$}}
\underbrace{f_2a^m}_{\text{Deklaration von $f_2$}}
\underbrace{f_1a^nb_1}_{\text{Definition von $f_1$}}
\underbrace{f_2a^m b_2}_{\text{Definition von $f_2$}}
\]
zu.
Kann man durch Wahl einer geeigneten Grammatik erzwingen, dass die
Anzahl der Argumente der Argumentliste in Deklaration und Definition
jeweils übereinstimmt?

\thema{Grammatik}
\themaL{Pumping Lemma fur kontextfreie Sprachen}{Pumping Lemma für kontextfreie Sprachen}

\begin{loesung}
Dies würde bedeuten, dass die Sprache kontextfrei ist.
Diese Sprache ist aber nicht kontextfrei, wie man mit dem Pumping Lemma
für kontextfreie Sprachen beweisen kann.
\begin{enumerate}
\item
Wir nehmen an, die Sprache sei kontextfrei.
\item
Nach dem Pumping-Lemma gibt es dann die Pumping Length $N$ so,
dass Wörter mit Länge $\ge N$ aufgepumpt werden können.
\item
Wir konstruieren das Wort
\[
w=f_1a^N f_2a^N f_1a^Nb_1f_2a^Nb_2
\]
welches ganz offensichtlich in der Sprache drin ist.
\item
Nach dem Pumping-Lemma gibt es dann eine Unterteilung des Wortes $w$ in
fünf Teile $w=xyzuv$ derart, dass $|yzu|\le N$.
Dies bedeutet, dass $y$ und $u$ Zeichen $a$ aus höchstens zwei verschiedenen
$a$-Blöcken enthalten können.
\item
Beim Aufpumpen wird also notwendigerweise die Anzahl der $a$ einer
Deklaration und einer nicht dazu passenden Definition verändert,
so dass nach dem Aufpumpen die Definition nicht mehr zur Deklaration 
passt.
Das aufgepumpte Wort ist also nicht mehr in der Sprache.
\item
Dieser Widerspruch zeigt, dass die Sprache nicht kontextfrei sein kann.
\qedhere
\end{enumerate}
\end{loesung}

\begin{bewertung}
Pumping Lemma ({\bf PL}) 1 Punkt,
Pumping Length ({\bf N }) 1 Punkt,
Beispielwort ({\bf W}) 1 Punkt,
Aufteilung ({\bf A}) 1 Punkt,
Widerspruch beim Pumpen ({\bf P}) 1 Punkt,
Schlussfolgerung ({\bf S}) 1 Punkt.
\end{bewertung}

