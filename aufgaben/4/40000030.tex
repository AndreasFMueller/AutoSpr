Zeichnen Sie das Zustandsdiagramm eines Stackautomaten, der
Palindrome erkennt.

\thema{Stackautomat}

\begin{loesung}
Palindrome kann man mit einem Stackautomaten erkennen, indem man 
die erste Hälfte des Wortes auf den Stack schreibt, und dann mit
der zweiten Hälfte vergleicht. Allerdings ist zwischen ungeraden
und geraden Wörtern zu unterscheiden. Ausserdem ist natürlich
nicht bekannt, wie lange ein Wort sein wird, der Wechsel zur zweiten
Phase, wo der Stack wieder abgebaut wird, erfolgt also nicht deterministisch.

Der folgende Stackautomat implementiert diese Idee. Zur Vereinfachung der
Notation verwenden wir folgende abgekürzten Schreibweisen:
\begin{align*}
.,&\varepsilon\to.&&\text{bedeutet}&&\text{Lege ein Zeichen vom Input auf den Stack}
\\
.,&.\to\varepsilon&&\text{bedeutet}&&\text{Nehme das gleiche Zeichen vom Input und vom Stack}
\end{align*}
Der Punkt steht also für ein beliebiges Zeichen, wenn aber zwei Punkte in einem
Übergang verwendet werden, ist der gleiche Buchstabe gemeint.
\begin{center}
\begin{tikzpicture}[>=latex,thick]
\coordinate (s) at (-2,0);
\coordinate (q0) at (0,0);
\coordinate (q1) at (3,0);
\coordinate (q3) at (0,-3);
\coordinate (q2) at (3,-3);
\def\zustand#1#2{
	\node at #1 {#2};
	\draw #1 circle[radius=0.4];
}
\def\akzeptierzustand#1#2{
	\zustand{#1}{#2}
	\draw #1 circle[radius=0.35];
}
\zustand{(q0)}{$q_0$}
\zustand{(q1)}{$q_1$}
\zustand{(q2)}{$q_2$}
\akzeptierzustand{(q3)}{$q_3$}
\draw[->,shorten >= 0.4cm,shorten <= 0.4cm] (s) -- (q0);
\draw[->,shorten >= 0.4cm,shorten <= 0.4cm] (q0) -- (q1);
\draw[->,shorten >= 0.4cm,shorten <= 0.4cm] (q2) -- (q3);
\draw[->,shorten >= 0.4cm,shorten <= 0.4cm]
	(q1) to[out=-70,in=70] (q2);
\draw[->,shorten >= 0.4cm,shorten <= 0.4cm]
	(q1) to[out=-110,in=110] (q2);
\draw[->,shorten >= 0.4cm,shorten <= 0.4cm]
	(q1) to[out=30,in=-30,distance=1.4cm] (q1);
\draw[->,shorten >= 0.4cm,shorten <= 0.4cm]
	(q2) to[out=-30,in=30,distance=1.4cm] (q2);
\node at ($0.5*(q0)+0.5*(q1)$)
	[above] {$\varepsilon,\varepsilon\to\texttt{\$}$\strut};
\node at ($0.5*(q2)+0.5*(q3)$)
	[below] {$\varepsilon,\texttt{\$}\to\varepsilon$\strut};
\node at ($(q1)+(1.0,0)$)
	[right] {$\texttt{.},\varepsilon\to\texttt{.}$\strut};
\node at ($(q2)+(1.0,0)$)
	[right] {$\texttt{.},\texttt{.}\to\varepsilon$\strut};
\node at ($0.5*(q1)+0.5*(q2)+(0.3,0)$)
	[right] {$\varepsilon,\varepsilon\to\varepsilon$\strut};
\node at ($0.5*(q1)+0.5*(q2)+(-0.3,0)$)
	[left] {$\texttt{.},\varepsilon\to\varepsilon$\strut};
\end{tikzpicture}
\end{center}
Der Stackautomat legt beim Zustand $q_1$ jedes Inputzeichen auf
den Stack.
Beim Zustand $q_2$ werden dieses später wieder abgebaut, wenn die
gleichen Zeichen im Input gefunden werden.
So werden Palindrome gerader Länge akzeptiert.
Der Übergang $\texttt{.},\varepsilon\to\varepsilon$ von $q_1$
zu $q_2$ wird benötigt, um bei Palindromen ungerader Länge das
alleinstehende mittlere Zeichen zu verarbeiten.
\end{loesung}
