Zeichnen Sie das Zustandsdiagramm eines Stackautomaten, der
Palindrome erkenne.

\begin{loesung}
Palindrome kann man mit einem Stackautomaten erkennen, indem man 
die erste Hälfte des Wortes auf den Stack schreibt, und dann mit
der zweiten Hälfte vergleicht. Allerdings ist zwischen ungeraden
und geraden Wörtern zu unterscheiden. Ausserdem ist natürlich
nicht bekannt, wie lange ein Wort sein wird, der Wechsel zur zweiten
Phase, wo der Stack wieder abgebaut wird, erfolgt also nicht deterministisch.

Der folgende Stackautomat implementiert diese Idee. Zur Vereinfachung der
Notation verwenden wir folgende Abgekürzten Schreibweisen:
\begin{align*}
.,&\varepsilon\to.&&\text{bedeutet}&&\text{Lege ein Zeichen vom Input auf den Stack}
\\
.,&.\to\varepsilon&&\text{bedeutet}&&\text{Nehme ein Zeichen vom Input vom Stack}
\end{align*}
\[
\entrymodifiers={++[o][F]}
\xymatrix {
*+\txt{}\ar[r]
	&{q_0} \ar[r]^{\varepsilon,\varepsilon\to{\tt \$}}
		&{q_1} \ar@(u,r)^{.,\varepsilon\to .}
		   \ar@/_/[d]_{.,\varepsilon\to\varepsilon}
		   \ar@/^/[d]^{\varepsilon,\varepsilon\to\varepsilon}
\\
*+\txt{}
	&*++[o][F=]{q_3}
		&{q_2} \ar@(r,d)^{.,.\to\varepsilon}
		    \ar[l]^{\varepsilon,{\tt \$}\to\varepsilon}
}
\]
\end{loesung}
