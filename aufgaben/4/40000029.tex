Die Sprache $L$ besteht aus den Zahlen, die in Dezimaldarstellung
(führende Nullen erlaubt)
aus zwei gleich langen Teilen bestehen. Die Ziffern im zweiten
Teil entstehen dabei aus den Ziffern im ersten Teil durch die
Operation $z\mapsto 9-z$ (Neunerkomplement), ausserdem ist die Reihenfolge
umgekehrt. Beispiele:
\[
L=\{ 09, 18, 27, 36,\dots 0099, 1188, 1278,\dots \}.
\]
\begin{teilaufgaben}
\item
Können Sie eine kontextfreie Grammatik für $L$  angeben?
\item
Können Sie einen Stackautomaten angeben, der die Sprache $L$ 
akzeptiert?
\end{teilaufgaben}

\begin{loesung}
\begin{teilaufgaben}
\item
Eine mögliche kontextfreie
Grammatik für $L$ ist:
\begin{align*}
\text{zahl}
&\rightarrow \varepsilon\\
&\rightarrow \text{'{\tt 0}'}\;\text{zahl}\;\text{'{\tt 9}'}\\
&\rightarrow \text{'{\tt 1}'}\;\text{zahl}\;\text{'{\tt 8}'}\\
&\rightarrow \text{'{\tt 2}'}\;\text{zahl}\;\text{'{\tt 7}'}\\
&\rightarrow \text{'{\tt 3}'}\;\text{zahl}\;\text{'{\tt 6}'}\\
&\rightarrow \text{'{\tt 4}'}\;\text{zahl}\;\text{'{\tt 5}'}\\
&\rightarrow \text{'{\tt 5}'}\;\text{zahl}\;\text{'{\tt 4}'}\\
&\rightarrow \text{'{\tt 6}'}\;\text{zahl}\;\text{'{\tt 3}'}\\
&\rightarrow \text{'{\tt 7}'}\;\text{zahl}\;\text{'{\tt 2}'}\\
&\rightarrow \text{'{\tt 8}'}\;\text{zahl}\;\text{'{\tt 1}'}\\
&\rightarrow \text{'{\tt 9}'}\;\text{zahl}\;\text{'{\tt 0}'}\\
\end{align*}
\item
Zu dieser Grammatik gehört folgender Stackautomat:
\[
\entrymodifiers={++[o][F]}
\xymatrix{
*+\txt{}\ar[r]
	&q_0\ar[r]^{\varepsilon,\varepsilon\to\text{\tt\$}}
		&q_1 \ar@(ur,ul)_{z,\varepsilon\to 9-z}
			\ar[r]^{\varepsilon,\varepsilon\to\varepsilon}
			&q_2 \ar@(ur,ul)_{z,z\to\varepsilon}
				\ar[r]^{\varepsilon,\text{\tt\$}\to\varepsilon}
				&*++[o][F=]{q_3}
}
\]
Darin bedeutet der "Ubergang
$z,\varepsilon\to9-z$, dass für jede Ziffer $z$ des Inputs die Ziffer
$9-z$ auf den Stack gelegt wird.
\qedhere
\end{teilaufgaben}
\end{loesung}
