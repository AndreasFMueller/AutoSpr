In Aufgabe~\ref{70000032} wird ein Spielfeld für ein Heyawake-Rätsel vorgeben.
Eine maschinenlesbare Version des Spielfeldes könnte wie 
folgt aussehen:
\begin{center}
\small
\verbatimainput{feld.txt}
\end{center}
Alle Zeilen sind gleich lang, und es folgen sich abwechselnd Zeilen,
die die Felder beschreiben und Zeilen, die die horizontalen Linien
beschreiben.
Punkte (\texttt{.}) beschreiben leere Felder.
Jeder zweite Platz in einer Zeile, die Felder beschreibt,
enthält entweder \texttt{|} oder ein Leerzeichen, so werden die
vertikalen Wände beschrieben.
Horizontale Wände werden durch \texttt{-} codiert.
Können Sie eine (kontextfreie) Grammatik für solche Feldbeschreibungen
angeben?

\begin{loesung}
Nein, diese Sprache ist nicht kontextfrei, wie man mit dem 
Pumping Lemma für kontextfreie Sprachen beweisen kann.
Dazu nehmen wir zunächst an, dass die Sprache der Feldbeschreibungen
kontextfrei sei. Dann gibt es nach dem Pumping Lemma eine Zahl $N$,
die Pumping Length, so dass alle Wörter der Sprache mit mindestens
Länge $N$ ``aufpumpbar'' sind.

Wir beschreiben jetzt ein völlig leeres $N\times N$-Feld. Es besteht
abwechseln aus Zeilen aus lauter Punkten und Leerzeichen, und Zeilen
aus lauter Leerzeichen. Alle Zeilen haben die gleiche Länge $2N-1$.
Dies ist das Wort $w$.

Das Pumping Lemma besagt, dass man das Wort  $w$ in fünf Teile
$w=xyzuv$ zerlegt werden kann, wobei $|yzv|\le N$ sein muss.
Ausserdem müssen die aufgepumpten Wörter $w_k=xy^kzu^kv$ alle wieder
korrekte Feldbeschreibungen sein.
Die Teile $y$ und $u$ sind in höchstens vier von den $2N-1$ Zeilen.
Ausserdem enthalten sie so wenige Zeichen, durch entfernen von
$y$ und $u$ nicht wieder vollständige Zeilen der Länge $2N-1$ entstehen
können.
Beim Abpumpen wird also die Länge von maximal vier Zeilen verändert,
wänrend die übrigen $2N-5$ Zeilen unverändert bleiben\footnote{Wenn man
es ganz genau nimmt, dann müsste man hier voraussetzen, dass $N \ge 3$ ist,
da man sonst nicht sicher sein kann, dass eine unveränderte Zeile
übrig bleibt. Diese Voraussetzung darf man aber immer machen, denn
wenn $N$ Pumping Length ist, dann ist auch $N+1$ Pumping Length.
Die Pumping Length sagt ja nur, wie lange ein Wort {\em mindestens} sein
muss, damit es sicher aufpumpbar ist. Und wenn $N$ lang genug ist, dann ist
$N+1$ sicher auch lang genug, und jede grössere Länge erst recht.
Man darf also immer annehmen, dass die Pumping Length $N$ ``gross genug'' ist.}.
Durch das Abpumpen ist das Wort $w_0$ nicht mehr in der Sprache.
Dieser Widerspruch zeigt, dass die Sprache der Heyawake-Rätsel nicht
kontextfrei ist.
\end{loesung}

\begin{diskussion}
Der Schlüssel zu dieser Aufgabe ist bereits in der Aussage enthalten,
alle Zeilen müssten gleich lang sein.
Wenn es eine Grammatik für diese Sprache gäbe, dann gäbe es auch
eine Grammatik für die Sprache
\[
L=\{ (\texttt{x}^n\texttt{\\n})^m\,|\, n,m>0\}
=
\{
\underbrace{
\texttt{x}^n\texttt{\\n}
\texttt{x}^n\texttt{\\n}
\texttt{x}^n\texttt{\\n}
\dots
\texttt{x}^n}_{\text{$m$ Faktoren}}
\,|\,n>0
\},
\]
welche man erhält, indem man in den Heyawake jedes Zeichen ausser 
\texttt{\textbackslash n} durch \texttt{x}
ersetzt wurde, und \texttt{\textbackslash n} durch \texttt{n}.
Diese Sprache kann man für $m>3$ ganz analog zu der in der Vorlesung
diskutierten Sprache $\{
\texttt{a}^n
\texttt{b}^n
\texttt{c}^n\,|\,n\ge 0\}$
behandeln.
\end{diskussion}

\begin{bewertung}
Erkenntnis, dass dies Sprache nicht kontextfrei ist ({\bf C}) 1 Punkt,
Pumping Lemma Annahme ({\bf A}) 1 Punkt,
Pumping Length ({\bf N}) 1 Punkt,
Beispielwort ({\bf W}) 1 Punkt,
Unterteilung in fünf Teile ({\bf U}) 1 Punkt,
Wort nach Pumpen nicht mehr in der Sprache ({\bf P}).
\end{bewertung}

