Für die Sprache der korrekt geschachtelten Klammerausdrücke wurde in der
Vorlesung die Grammatik
\begin{equation}
\begin{aligned}
S\rightarrow
\texttt{(}S\texttt{)}
\;|\;
SS
\;|\;
\varepsilon
\label{40000052:grammar1}
\end{aligned}
\end{equation}
gefunden, die sich jedoch als nicht
eindeutig herausgestellt hat, das Wort \texttt{()} hat zum Beispiel die
beiden völlig verschiedenen Parse-Trees
\[
\xymatrix{
	&S \ar[dl] \ar[dr]
		&
			&
				&
					&
						&
							&S\ar[dl] \ar[dr]
\\
S\ar[d]
	&
		&S\ar[dl] \ar[d] \ar[dr]
			&
				&
					&
						&S\ar[dl] \ar[d] \ar[dr]
							&
								&S\ar[d]
\\
\varepsilon
	&\texttt{(}\ar[d]
		&S\ar[d]
			&\texttt{)}\ar[d]
				&
					&\texttt{(}\ar[d]
						&S\ar[d]
							&\texttt{)}\ar[d]
								&\varepsilon
\\
	&\texttt{(}
		&
			&\texttt{)}
				&
					&\texttt{(}
						&
							&\texttt{)}
								&
}
\]


Betrachten Sie die Grammatik
\begin{equation}
\begin{aligned}
S&\rightarrow \texttt{(} S \texttt{)} S
\\
&\rightarrow \varepsilon
\end{aligned}
\label{40000052:grammar2}
\end{equation}
\begin{teilaufgaben}
\item Finden Sie den Parse Tree von \texttt{()} in der Grammatik
\eqref{40000052:grammar2}.
\item Ist die Grammatik \eqref{40000052:grammar2} eindeutig?
\item Finden Sie die Chomsky-Normalform dieser Grammatik.
\end{teilaufgaben}

\thema{Grammatik}
\thema{Parse Tree}
\thema{eindeutiger Parse Tree}
\thema{Chomsky Normalform}

\begin{loesung}
\begin{teilaufgaben}
\item Der Parse Tree ist
\[
\xymatrix{
	&S \ar[dl] \ar[d] \ar[dr] \ar[drr]
		&
			&
\\
\texttt{(}\ar[d]
	&S\ar[d]
		&\texttt{)}\ar[d]
			&S\ar[d]
\\
\texttt{(}
	&\varepsilon
		&\texttt{)}
			&\varepsilon
}
\]
\item Die Grammatik ist eindeutig, weil für jede öffnende Klammer die erste 
Regel genau einmal angewendet werden muss. 
Die Grammatik
\eqref{40000052:grammar1} dagegen lässt immer die Option offen, die
Regel $S\to SS$ anzuwenden, was zur Zweideutigkeit führt.
\item Zunächst muss sichergestellt werden, dass die Startvariable auf
der rechten Seite nicht vorkommt:
\begin{align*}
S_0&\rightarrow S
\\
S&\rightarrow \texttt{(} S \texttt{)} S \;|\; \varepsilon
\end{align*}
Jetzt müssen $\varepsilon$-Regeln entfernt werden:
\begin{align*}
S_0&\rightarrow
S
\;|\;
\varepsilon
\\
S&\rightarrow
\texttt{(} S \texttt{)} S \;|\;
\texttt{(}  \texttt{)} S \;|\;
\texttt{(} S \texttt{)}  \;|\;
\texttt{(}  \texttt{)} 
\end{align*}
Jetzt müssen Unit-Rules entfernt werden:
\begin{align*}
S_0&\rightarrow
\texttt{(} S \texttt{)} S \;|\;
\texttt{(}  \texttt{)} S \;|\;
\texttt{(} S \texttt{)}  \;|\;
\texttt{(}  \texttt{)} 
\;|\;
\varepsilon
\\
S&\rightarrow
\texttt{(} S \texttt{)} S \;|\;
\texttt{(}  \texttt{)} S \;|\;
\texttt{(} S \texttt{)}  \;|\;
\texttt{(}  \texttt{)} 
\end{align*}
Jetzt müssen nur noch die verbleibenden Sequenzen aufgeteilt werden:
\begin{align*}
S_0&\rightarrow
AX \;|\;
AY \;|\;
AZ  \;|\;
AB 
\;|\;
\varepsilon
\\
S&\rightarrow
AX \;|\;
AY \;|\;
AZ  \;|\;
AB 
\\
A&\rightarrow \texttt{(}\\
B&\rightarrow \texttt{)}\\
X&\rightarrow SY \\
Y&\rightarrow BS \\
Z&\rightarrow SB
\end{align*}
Damit ist Chomsky-Normalform erreicht.
\qedhere
\end{teilaufgaben}
\end{loesung}

\begin{bewertung}
\begin{teilaufgaben}
\item
Parse Tree ({\bf P}) 1 Punkt.
\item
Begründung für Eindeutigkeit ({\bf B})  1 Punkt.
\item
Startvariable nur auf der linken Seite ({\bf S}) 1 Punkt,
Epsilon-Regel ({\bf E}) 1 Punkt,
Unit-Rules ({\bf U}) 1 Punkt,
Mehrfach-Regeln und Terminalsymbole ({\bf T}) 1 Punkt.
\end{teilaufgaben}
\end{bewertung}




