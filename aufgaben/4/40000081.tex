Üben Sie die Regeln zur Reduktion auf CNF an den folgenden Beispielen:\\
\hbox to\textwidth{%
\begin{minipage}[t]{0.3\textwidth}%
\noindent
a)\;
Wenden Sie die $\varepsilon$-Regel {\color{red}$A\to\varepsilon$}
auf die folgenden Regeln an:
\begin{align*}
B&\to AC      \\
 &\to CA      \\
 &\to ACA     \\
D&\to AAAA    \\
\color{red}
A&\color{red}\to \varepsilon
\end{align*}
\end{minipage}
\hfill
\begin{minipage}[t]{0.3\textwidth}%
\noindent
b)\;
Eliminieren Sie die Unit-Rule {\color{red}$A\to B$} aus den folgenden Regeln
\begin{align*}
C &\to AC     \\
B &\to AB     \\
  &\to BB \mid C \\
\color{red}A &\color{red}\to B
\end{align*}
\end{minipage}
\hfill
\begin{minipage}[t]{0.3\textwidth}
\noindent
c)\;
Reduzieren Sie die folgenden Regeln auf CNF
\begin{align*}
A&\to A\texttt{b}C \mid \texttt{e} \\
C&\to A\texttt{d}
\end{align*}
\end{minipage}}

\thema{Grammatik}
\thema{Chomsky-Normalform}

\begin{loesung}
\begin{teilaufgaben}
\item
Die Regel $A\to\varepsilon$ bedeutet, dass $A$ auf der rechten Seite
einer Regel optional ist, also weggelassen werden kann:
\begin{align*}
B&\to AC \mid C     \\
 &\to CA \mid C     \\
 &\to ACA \mid CA \mid AC \mid C    \\
D&\to AAAA \mid AAA \mid AA \mid A \mid \varepsilon
\end{align*}
Darin sind einige Regeln redundant. 
Lässt man duplizierte Regeln weg, bleibt
\begin{align*}
B&\to ACA \mid AC \mid CA \mid C \\
D&\to AAAA \mid AAA \mid AA \mid A \mid \varepsilon
\end{align*}
\item
Die Regel $A\to B$ bedeutet, dass man alles, was man aus $B$ machen kann,
auch aus $A$ machen kann:
\begin{align*}
C &\to AC     \\
B &\to AB     \\
  &\to BB \mid C  \\
A &\to AB     \\
  &\to BB \mid C 
\end{align*}
\item
Es gibt zwei Terminalsymbole, die über Zwischenvariablen erzeugt
werden müssen:
\begin{align*}
B & \to \texttt{b} \\
D & \to \texttt{d} \\
A&\to ABC \mid \texttt{e} \\
C&\to AD
\end{align*}
Es bleibt die Dreier-Regel, die in Zweier-Schritte aufgeteilt werden
muss:
\[
A\to ABC
\qquad\Rightarrow\qquad
\left\{\quad
\begin{aligned}
A&\to AA_1 \\
A_1&\to BC
\end{aligned}
\right.
\]
Alle Regeln zusammen sind
\begin{align*}
B & \to \texttt{b} \\
D & \to \texttt{d} \\
A&\to AA_1 \mid \texttt{e} \\
A_1&\to BC \\
C&\to AD
\end{align*}

\end{teilaufgaben}
\end{loesung}
