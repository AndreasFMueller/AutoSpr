Formulieren Sie eine Grammatik für die folgenden Sprachen über
den Terminalsymbolen $\Sigma=\{{\tt 0},{\tt 1}\}$:
\begin{teilaufgaben}
\item $L=\Sigma^*$
\item Wörter, die mit dem gleichen Symbol enden, mit dem sie beginnen.
\item $L$ enthält alle Wörter gerader Länge.
\item $L=\{0^n1^m|n>m\}$
\end{teilaufgaben}

\thema{Grammatik}

\begin{loesung}
\begin{teilaufgaben}
\item
Jedes Wort kann man erhalten, indem man dem leeren Wort $\varepsilon$
wiederholt {\tt 0} oder {\tt 1} anhängt.
\begin{align*}
W&\rightarrow W{\tt 0}\\
 &\rightarrow W{\tt 1}\\
 &\rightarrow\varepsilon
\end{align*}
\item Sei $E$ eine Variable, die für Wörter mit gleichem Anfangs-
und Endzeichen steht. Dazu gehören $\varepsilon$, {\tt 0} und {\tt 1}.
Längere
Wörter kann man aus beliebigen Wörtern $W$ in $\Sigma^*$ erzeugen,
indem man das gleiche Symbol voranstellt und anhängt. Für beliebige
Wörter kann
man die Produktionsregeln aus a) verwenden.
\begin{align*}
E&\rightarrow \varepsilon\\
 &\rightarrow {\tt 0}\\
 &\rightarrow {\tt 1}\\
 &\rightarrow {\tt 0}W{\tt 0}\\
 &\rightarrow {\tt 1}W{\tt 1}\\
W&\rightarrow W{\tt 0}\\
 &\rightarrow W{\tt 1}\\
 &\rightarrow\varepsilon
\end{align*}
\item Die Variable $G$ steht für Wörter gerader Länge. Ein Wort
gerader Länge kann man bilden, indem man einem Wort gerader Länge
ein Zeichenpaar anhängt. Sei $P$ eine Variable, die für ein Zeichenpaar
steht. Die Variable $Z$ soll für ein einzelnes Zeichen stehen, die
Produktionsregeln der Grammatik sind damit:
\begin{align*}
G&\rightarrow GP\\
 &\rightarrow \varepsilon\\
P&\rightarrow ZZ\\
Z&\rightarrow {\tt 0}\\
 &\rightarrow {\tt 1}
\end{align*}
\item
Das kürzeste Wort der gesuchten Art ist {\tt 0}. Sei $S$ eine
Variable, die für Wörter der Sprache steht. Wörter in $L$
kann man aus bereits produzierten Wörtern in $L$ produzieren,
indem man  eine {\tt 0} voranstellt, und optional eine {\tt 1}
anhängt. Dies führt auf die Produktionsregeln
\begin{align*}
S&\rightarrow {\tt 0}S\\
 &\rightarrow {\tt 0}S{\tt 1}\\
 &\rightarrow {\tt 0}
\qedhere
\end{align*}
\end{teilaufgaben}
\end{loesung}
