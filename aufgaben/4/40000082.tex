Die Grammatik
\begin{align*}
S_0& \to NE \mid NA \mid \varepsilon \\
S  & \to NE \mid NA \\
A  &\to SE \\
N&\to \texttt{0} \\
E&\to \texttt{1} 
\end{align*}
für Wörter der Form
$\texttt{0}^n\texttt{1}^n$
hat Chomsky-Normalform.
Wenden Sie den CYK-Algorithmus mit dieser Grammatik auf das Wort
\texttt{0011} an.

\def\ableitbar#1#2{
	\texttt{ableitbar(}#1\texttt{,}#2\texttt{)}
}

\begin{loesung}
Der CYK-Algorithmus verwendet rekursive Aufrufe der Funktion
\texttt{ableitbar}, um herauszufinden, ob ein Teilwort aus einer 
Variablen ableitbar ist.
Um die spätere Diskussion zu vereinfachen, überlegen wir uns zunächst
ein paar Spezialfälle.

Das leere Wort ist nur aus der Startvariablen ableitbar, daher ist
$\ableitbar{A}{\varepsilon}$ nur dann wahr, wenn $A=S_0$ ist.

Hat das Wort $w$ die Länge $1$, dann ist 
$\ableitbar{A}{w}$ nur wahr
für $A=N$ und $w=\texttt{0}$
bzw.~für $A=E$ und $w=\texttt{1}$.

Der erste Aufruf der Funktion \texttt{ableitbar} fragt, ob das Wort
\texttt{0011} aus $S_0$ ableitbar ist.
Dazu muss zunächst kontrolliert werden, ob das leere Wort vorliegt,
dies ist hier nicht der Fall.
Dann kommen die Zweierregeln dran, die Regeln
$S_0\to NE$
und
$S_0\to NA$.
Für beide müssen die Unterteilungen
\texttt{0|011},
\texttt{00|11} oder
\texttt{001|1}
daraufhin getestet werden, ob die Teile aus $N$ bzw.~$E$ oder $A$
ableitbar sind:
\begin{itemize}
\item
\texttt{0|011}:
Es muss $\ableitbar{N}{\texttt{0}}$ und $\ableitbar{A}{\texttt{011}}$ 
aufgerufen werden.
Der erste Aufruf ist erfolgreich, so dass es jetzt nur noch auf den
zweiten ankommt.
Ausgehend von $A$ gibt es nur die Regel $A\to SE$.
Innerhalb des Aufrufs werden die Unterteilungen
\texttt{0|11} 
und
\texttt{01|1} 
getestet.
\begin{itemize}
\item
$\ableitbar{S}{\texttt{01}}$ und $\ableitbar{E}{\texttt{1}}$:
Der zweite Aufruf wird {\em true} zurückgeben, so dass es vor allem
auf den ersten ankommt.
\begin{itemize}
\item
Da es nur die Regel $S\to NA$ gibt, muss 
\end{itemize}
\item
$\ableitbar{S}{\texttt{0}}$ und $\ableitbar{E}{\texttt{11}}$:
muss nicht mehr aufgerufen werden.
\end{itemize}
\item
\texttt{00|11}: muss nicht mehr aufgerufen werden.
\item
\texttt{001|1}: muss nicht mehr aufgerufen werden.
\end{itemize}



\end{loesung}
