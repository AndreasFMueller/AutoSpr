Zeigen Sie, dass die folgenden Sprachen nicht kontextfrei sind:
\begin{teilaufgaben}
\item $L=\{\texttt{0}^n\texttt{1}^n\texttt{0}^n\texttt{1}^n\mid n\ge 0\}$
\item $L=\{w{\texttt{\#}}w\mid w\in\{{\tt 0},{\tt 1}\}^*\}$ über dem Alphabet
$\Sigma=\{\texttt{0},\texttt{1},\texttt{\#}\}$
\end{teilaufgaben}

\thema{kontextfrei}
\themaL{Pumping Lemma fur kontextfreie Sprachen}{Pumping Lemma für kontextfreie Sprachen}

\begin{loesung}
\begin{teilaufgaben}
\item Sei $N$ die Pumping length.
Bilde das Wort $w=\texttt{0}^N\texttt{1}^N\texttt{0}^N\texttt{1}^N$.
Das Pumping Lemma
behauptet, dass es eine Unterteilung des Wortes $w=uvxyz$ gibt, wir
müssen zeigen, dass jede möglich Unterteilung, die allen vom
Pumping Lemma garantierten Rahmenbedingungen genügt, bei Aufpumpen
auf ein nicht zu $L$ gehörendes Wort führt.
Im vorliegenden Fall
gibt es zwei wesentlich verschieden Fälle.

Im ersten Fall sind die
Teilwörter $v$ und $y$ jeweils vollständig in einem der aus Nullen
oder Einsen bestehenden Blöcke enthalten. Durch Aufpumpen nimmt die
Anzahl der gleichartigen Stellen in diesem Block zu, der andere Block
mit gleichen Ziffern wird jedoch nicht verändert.
Die Eigenschaft, dass die Nuller- und
Einer-Blöcke jeweils gleich viele Stellen enthalten müssen, wir
beim Aufpumpen zerstört.

Im zweiten Fall enthält einer der Teile $v$ oder $y$ eine
Grenze zwischen Nullen und Einsen, also einen Wechsel von
\texttt{0} auf \texttt{1} oder umgekehrt.
Ein Wort in $L$ kann
genau zwei Wechsel von \texttt{0} auf \texttt{1} und genau einen
von \texttt{1} auf \texttt{0} enthalten.
Durch Aufpumpen des Teils,
der einen Wechsel enthält, wird die Anzahl der Wechsel erhöht,
so dass das aufgepumpte Wort nicht mehr zu $L$ gehört.

Das Pumping Lemma behauptet aber, dass auch die aufgepumpten Wörter
zur Sprache gehören muss. Dieser Widerspruch zeigt, dass $L$
nicht kontextfrei sein kann.
\item Sei $N$ die Pumping length. Wie in a) verwenden wir als
Wort $\texttt{0}^N\texttt{1}^N\texttt{\#}\texttt{0}^N\texttt{1}^N$.
Bei der Unterteilung
gemäss den Restriktionen des Pumping Lemma gibt es genau einen zusätzlichen
möglichen Fall, nämlich den, dass $v$ oder $y$ das Zeichen {\tt\#}
enthält.
In diesem Fall wird durch Aufpumpen die Zahl der {\tt\#}
erhöht, ein Wort in $L$ darf jedoch nur ein einziges {\tt\#}-Zeichen
enthalten.
Diese Widerspruch zeigt, dass $L$ nicht kontextfrei sein kann.
\qedhere
\end{teilaufgaben}
\end{loesung}
