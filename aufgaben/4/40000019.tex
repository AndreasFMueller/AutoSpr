Konstruieren Sie einen Pushdown-Automaten, der die durch folgende
Grammatik definierte Sprache akzeptiert:
\begin{align*}
S&\rightarrow \varepsilon\\
 &\rightarrow SS\\
 &\rightarrow KSV\\
 &\rightarrow VSK\\
K&\rightarrow 
\text{\tt b}\,|\,
\text{\tt c}\,|\,
\text{\tt d}\,|\,
\text{\tt f}\,|\,
\text{\tt g}\,|\,
\text{\tt h}\,|\,
\text{\tt j}\,|\,
\text{\tt k}\,|\,
\text{\tt l}\,|\,
\text{\tt m}\,|\,
\text{\tt n}\,|\,
\text{\tt p}\,|\,
\text{\tt q}\,|\,
\text{\tt r}\,|\,
\text{\tt s}\,|\,
\text{\tt t}\,|\,
\text{\tt v}\,|\,
\text{\tt w}\,|\,
\text{\tt x}\,|\,
\text{\tt y}\,|\,
\text{\tt z}
\\
V&\rightarrow
\text{\tt a}\,|\,
\text{\tt e}\,|\,
\text{\tt i}\,|\,
\text{\tt o}\,|\,
\text{\tt u}
\end{align*}
Zeigen Sie, dass das Wort {\tt freude} zwei verschiedene
parse trees hat.

\thema{Stackautomat}
\thema{eindeutiger Parse Tree}

\begin{loesung}
Die Sprache enthält die Wörter
\[
\varepsilon,
\text{\it KV},
\text{\it VK},
\text{\it VVKK},
\text{\it VKVK},
\text{\it KVKV},
\text{\it KKVV},\dots
\]
also alle Wörter, die gleich viele Konsonanten wie Vokale enthalten.
Wir brauchen also nur einen Pushdown-Automaten, der feststellen kann,
ob die gleiche Anzahl Vokale wie Konsonanten gelesen worden sind. Dazu
verwenden wir als Stackalphabet $\Gamma = \{{\tt \$}, {\tt +}\}$,
wobei die {\tt +} einen "Uberschuss von Konsontanten oder Vokalen
anzeigen. Welches Zeichen gerade im "Uberfluss vorhanden ist,
zeigt der aktuelle Zustand, der entweder $V$ ("Uberschuss an
Vokalen) oder $K$ ("Uberschuss an Konsonanten) sein kann.
Der Automat ist im Zustand $Z$, wenn gleich viele Vokale wie
Konsonanten gelesen wurden:
\[
\entrymodifiers={++[o][F]}
\xymatrix @+10mm {
*+\txt{}\ar[r]
        &{s}\ar[d]^{\varepsilon,\varepsilon\to{\tt \$}}
\\
{V} \ar@(ul,dl)_{\scriptsize    \begin{gathered}
                                        V,\varepsilon\to {\tt +}\\
                                        K,{\tt +}\to\varepsilon
                                \end{gathered}}
    \ar@/_/[r]_{\varepsilon,{\tt \$}\to{\tt \$}}
        &{Z} \ar@/^/[r]^{K,\varepsilon\to {\tt +}}
             \ar@/_/[l]_{V,\varepsilon\to {\tt +}}
             \ar[d]^{\varepsilon,{\tt\$}\to\varepsilon}
                &{K} \ar@(ur,dr)^{\scriptsize   \begin{gathered}
                                                        K,\varepsilon\to {\tt +}\\
                                                        V,{\tt +}\to\varepsilon
                                                \end{gathered}}
                     \ar@/^/[l]^{\varepsilon,{\tt \$}\to{\tt\$}}
\\
*+\txt{}
        &*++[o][F=]{E}
}
\]
Dabei stehen die Variablen $V$ und $K$ in den Übergängen für 
die Terminalsymbole, in die sie gemäss der Regeln der Grammatik
verwandelt werden können.

Die möglichen Parse Trees sind:
\[
\xymatrix{
        &
                &S\ar[dl] \ar[drr]
\\
        &S\ar[dl] \ar[d] \ar[drr]
                &
                        &
                                &S\ar[d] \ar[dr]
\\
K\ar[dd]
        &S\ar[d] \ar[dr]
                &
                        &V\ar[dd]
                                &K\ar[dd]
					&V\ar[dd]
\\
	&K\ar[d]
		&V\ar[d]
			&
				&
					&
\\
{\tt f}
        &{\tt r}
                &{\tt e}
                        &{\tt u}
                                &{\tt d}
                                        &{\tt e}
}
\]

\[
\xymatrix{
        &
                &S\ar[dll] \ar[d] \ar[drrr]
\\
K\ar[ddd]
        &
                &S\ar[dl] \ar[dr]
                        &
                                &
                                        &V\ar[ddd]
\\
        &S\ar[d] \ar[dr]
                &
                        &S\ar[d] \ar[dr]
                                &
\\
	&K\ar[d]
		&V\ar[d]
			&K\ar[d]
				&V\ar[d]
					&
\\
{\tt f}
        &{\tt r}
                &{\tt e}
                        &{\tt u}
                                &{\tt d}
                                        &{\tt e}
}
\]
Darin haben wir zur Abkürzung die Produktionsregeln
$S\to KV$ und $S\to VK$ verwendet, welche sich durch
zusammensetzen von $S\to KSV$ bzw.~$S\to VSK$ mit $S\to\varepsilon$
ergeben.
\end{loesung}
