Die Schweizer Autokennzeichen
(siehe Aufgabe \ref{30000036})
lassen sich auch durch eine
kontextfreie Grammatik spezifizieren.
\begin{teilaufgaben}
\item Geben Sie eine Grammatik an, die genau die Schweizer Autokennzeichen
erzeugt.
\item Markieren Sie in Ihrer Grammatik genau diejenigen Regeln, welche
nicht der Chomsky Normalform entsprechen, und begründen Sie ihre
Markierungen.
\end{teilaufgaben}

\thema{Grammatik}
\thema{Chomsky Normalform}

\begin{loesung}
\begin{teilaufgaben}
\item Wir brauchen zunächst Regeln für die Kantonskürzel:
\begin{align*}
\text{\textsl{praefix}}
&\rightarrow
\text{\color{red}\textsl{kantonskuerzel}}\;|\;
\text{\color{red}\textsl{adminpraefix}}\\
\text{\textsl{kantonskuerzel}}
&\rightarrow
{\color{blue}\tt AG}\;|\;
{\color{blue}\tt AR}\;|\;
{\color{blue}\tt AI}\;|\;
{\color{blue}\tt BL}\;|\;
{\color{blue}\tt BS}\;|\;
{\color{blue}\tt BE}\;|\;
{\color{blue}\tt FR}\;|\;
{\color{blue}\tt GE}\;|\;
{\color{blue}\tt GL}\;|\;
{\color{blue}\tt GR}\;|\;
{\color{blue}\tt JU}\;|\;
{\color{blue}\tt LU}\;|\;
{\color{blue}\tt NE}
\\
&\rightarrow
{\color{blue}\tt NW}\;|\;
{\color{blue}\tt OW}\;|\;
{\color{blue}\tt SH}\;|\;
{\color{blue}\tt SZ}\;|\;
{\color{blue}\tt SO}\;|\;
{\color{blue}\tt SG}\;|\;
{\color{blue}\tt TI}\;|\;
{\color{blue}\tt TG}\;|\;
{\color{blue}\tt UR}\;|\;
{\color{blue}\tt VD}\;|\;
{\color{blue}\tt VS}\;|\;
{\color{blue}\tt ZG}\;|\;
{\color{blue}\tt ZH}\\
\text{\textsl{adminpraefix}}
&\rightarrow {\tt A}\;|\;
{\tt P}\;|\;
{\tt M}
\end{align*}
Dann brauchen wir Regeln für eine maximal sechsstellige Zahl
\begin{align*}
\text{\textsl{zahl}}
&\rightarrow
\text{\textsl{positiveziffer}}\;\text{\textsl{rest}}
\\
\text{\textsl{rest}}
&\rightarrow{\color{green}\varepsilon}
\\
&\rightarrow
\text{\color{red}\textsl{ziffer}}
\\
&\rightarrow
\text{\textsl{ziffer}}\;
\text{\textsl{ziffer}}
\\
&\rightarrow
{\color{blue}
\text{\textsl{ziffer}}\;
\text{\textsl{ziffer}}\;
\text{\textsl{ziffer}}}
\\
&\rightarrow
{\color{blue}
\text{\textsl{ziffer}}\;
\text{\textsl{ziffer}}\;
\text{\textsl{ziffer}}\;
\text{\textsl{ziffer}}}
\\
&\rightarrow
{\color{blue}
\text{\textsl{ziffer}}\;
\text{\textsl{ziffer}}\;
\text{\textsl{ziffer}}\;
\text{\textsl{ziffer}}\;
\text{\textsl{ziffer}}}
\\
\text{\textsl{ziffer}}
&\rightarrow
{\tt 0}\;|\;
{\tt 1}\;|\;
{\tt 2}\;|\;
{\tt 3}\;|\;
{\tt 4}\;|\;
{\tt 5}\;|\;
{\tt 6}\;|\;
{\tt 7}\;|\;
{\tt 8}\;|\;
{\tt 9}
\\
\text{\textsl{positiveziffer}}
&\rightarrow
{\tt 1}\;|\;
{\tt 2}\;|\;
{\tt 3}\;|\;
{\tt 4}\;|\;
{\tt 5}\;|\;
{\tt 6}\;|\;
{\tt 7}\;|\;
{\tt 8}\;|\;
{\tt 9}
\\
\end{align*}
Schliesslich brauchen wir noch Regeln für das optionale Suffix:
\begin{align*}
\text{\textsl{suffix}}
&\rightarrow{\color{green}\varepsilon} \;|\;
{\tt U} \;|\;
{\tt V} \;|\;
{\tt Z}
\end{align*}
Diese Regeln können wir jetzt mit Hilfe einer Startregel zu einer
Grammatik für Autokennzeichen zusammenfügen:
\begin{align*}
\text{\textsl{kennzeichen}}&\rightarrow
\text{\textsl{praefix}}\;\text{\textsl{zahl}}\;\text{\textsl{suffix}}
\end{align*}
\item In den obigen Regeln sind Regeln, die nicht der Chomsky Normalform
entsprechen, nach folgendem Farbcode markiert:
\begin{itemize}
\item {\color{red} rot}: Unit-Rules
\item {\color{blue} blau}: auf der rechten Seite kommen mehr als
zwei Elemente vor, oder zwei Elemente, aber mindestens eines davon
ist ein Terminalsymbol.
\item {\color{green} grün}: leeres Wort auf der rechten Seite der Regel
\qedhere
\end{itemize}
\end{teilaufgaben}
\end{loesung}

\begin{bewertung}
\begin{teilaufgaben}
\item Fehlerfreie Grammatik 4 Punkte, Abzüge für Fehler wie
nicht spezifizierte administrative Präfixe ({\bf A}), nicht spezifizierte
Sondernutzungszeichen ({\bf S}), führende Nullen ({\bf N}).
\item 2 Punkte, Abzüge für nicht erkannte Regeln oder für falsch
markierte Regeln.
\end{teilaufgaben}
\end{bewertung}

