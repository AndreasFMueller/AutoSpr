Die Schweizer Autokennzeichen
(siehe Aufgabe \ref{30000036})
lassen sich auch durch eine
kontextfreie Grammatik spezifizieren.
\begin{teilaufgaben}
\item Geben Sie eine Grammatik an, die genau die Schweizer Autokennzeichen
erzeugt.
\item Markieren Sie in Ihrer Grammatik genau diejenigen Regeln, welche
nicht der Chomsky-Normalform entsprechen, und begründen Sie ihre
Markierungen.
\end{teilaufgaben}

\thema{Grammatik}
\thema{Chomsky-Normalform}

\begin{loesung}
\begin{teilaufgaben}
\item Wir brauchen zunächst Regeln für die Kantonskürzel:
\begin{align*}
\text{\textsl{praefix}}
&\rightarrow
\text{\color{red}\textsl{kantonskuerzel}}\mid 
\text{\color{red}\textsl{adminpraefix}}\\
\text{\textsl{kantonskuerzel}}
&\rightarrow
{\color{blue}\tt AG}\mid 
{\color{blue}\tt AR}\mid 
{\color{blue}\tt AI}\mid 
{\color{blue}\tt BL}\mid 
{\color{blue}\tt BS}\mid 
{\color{blue}\tt BE}\mid 
{\color{blue}\tt FR}\mid 
{\color{blue}\tt GE}\mid 
{\color{blue}\tt GL}\mid 
{\color{blue}\tt GR}\mid 
{\color{blue}\tt JU}\mid 
{\color{blue}\tt LU}\mid 
{\color{blue}\tt NE}
\\
&\rightarrow
{\color{blue}\tt NW}\mid 
{\color{blue}\tt OW}\mid 
{\color{blue}\tt SH}\mid 
{\color{blue}\tt SZ}\mid 
{\color{blue}\tt SO}\mid 
{\color{blue}\tt SG}\mid 
{\color{blue}\tt TI}\mid 
{\color{blue}\tt TG}\mid 
{\color{blue}\tt UR}\mid 
{\color{blue}\tt VD}\mid 
{\color{blue}\tt VS}\mid 
{\color{blue}\tt ZG}\mid 
{\color{blue}\tt ZH}\\
\text{\textsl{adminpraefix}}
&\rightarrow {\tt A}\mid 
{\tt P}\mid 
{\tt M}
\end{align*}
Dann brauchen wir Regeln für eine maximal sechsstellige Zahl
\begin{align*}
\text{\textsl{zahl}}
&\rightarrow
\text{\textsl{positiveziffer}}\;\text{\textsl{rest}}
\\
\text{\textsl{rest}}
&\rightarrow{\color{green}\varepsilon}
\\
&\rightarrow
\text{\color{red}\textsl{ziffer}}
\\
&\rightarrow
\text{\textsl{ziffer}}\;
\text{\textsl{ziffer}}
\\
&\rightarrow
{\color{blue}
\text{\textsl{ziffer}}\;
\text{\textsl{ziffer}}\;
\text{\textsl{ziffer}}}
\\
&\rightarrow
{\color{blue}
\text{\textsl{ziffer}}\;
\text{\textsl{ziffer}}\;
\text{\textsl{ziffer}}\;
\text{\textsl{ziffer}}}
\\
&\rightarrow
{\color{blue}
\text{\textsl{ziffer}}\;
\text{\textsl{ziffer}}\;
\text{\textsl{ziffer}}\;
\text{\textsl{ziffer}}\;
\text{\textsl{ziffer}}}
\\
\text{\textsl{ziffer}}
&\rightarrow
{\tt 0}\mid 
{\tt 1}\mid 
{\tt 2}\mid 
{\tt 3}\mid 
{\tt 4}\mid 
{\tt 5}\mid 
{\tt 6}\mid 
{\tt 7}\mid 
{\tt 8}\mid 
{\tt 9}
\\
\text{\textsl{positiveziffer}}
&\rightarrow
{\tt 1}\mid 
{\tt 2}\mid 
{\tt 3}\mid 
{\tt 4}\mid 
{\tt 5}\mid 
{\tt 6}\mid 
{\tt 7}\mid 
{\tt 8}\mid 
{\tt 9}
\\
\end{align*}
Schliesslich brauchen wir noch Regeln für das optionale Suffix:
\begin{align*}
\text{\textsl{suffix}}
&\rightarrow{\color{green}\varepsilon} \mid 
{\tt U} \mid 
{\tt V} \mid 
{\tt Z}
\end{align*}
Diese Regeln können wir jetzt mit Hilfe einer Startregel zu einer
Grammatik für Autokennzeichen zusammenfügen:
\begin{align*}
\text{\textsl{kennzeichen}}&\rightarrow
\text{\textsl{praefix}}\;\text{\textsl{zahl}}\;\text{\textsl{suffix}}
\end{align*}
\item In den obigen Regeln sind Regeln, die nicht der Chomsky-Normalform
entsprechen, nach folgendem Farbcode markiert:
\begin{itemize}
\item {\color{red} rot}: Unit-Rules
\item {\color{blue} blau}: auf der rechten Seite kommen mehr als
zwei Elemente vor, oder zwei Elemente, aber mindestens eines davon
ist ein Terminalsymbol.
\item {\color{green} grün}: leeres Wort auf der rechten Seite der Regel
\qedhere
\end{itemize}
\end{teilaufgaben}
\end{loesung}

\begin{bewertung}
\begin{teilaufgaben}
\item Fehlerfreie Grammatik 4 Punkte, Abzüge für Fehler wie
nicht spezifizierte administrative Präfixe ({\bf A}), nicht spezifizierte
Sondernutzungszeichen ({\bf S}), führende Nullen ({\bf N}).
\item 2 Punkte, Abzüge für nicht erkannte Regeln oder für falsch
markierte Regeln.
\end{teilaufgaben}
\end{bewertung}

