Konstruieren Sie einen Pushdown-Automaten, der die durch folgende
Grammatik definierte Sprache akzeptiert:
\begin{align*}
S&\rightarrow \varepsilon\\
 &\rightarrow SS\\
 &\rightarrow GSU\\
G&\rightarrow {\tt 0}\mid {\tt 2}\mid {\tt 4}\mid {\tt 6}\mid {\tt 8}
\\
U&\rightarrow {\tt 1}\mid {\tt 3}\mid {\tt 5}\mid {\tt 7}\mid {\tt 9}
\end{align*}

\begin{hinweis}
Versuchen sie zuerst die Sprache zu verstehen, die von
dieser Grammatik erzeugt wird und erstellen Sie dann einen ``passenden''
Pushdown-Automaten.
\end{hinweis}

\thema{Stackautomat}

\begin{loesung}
Die Wörter der Sprache entstehen aus korrekt geschachtelten
Klammer-Ausdrücken, indem man die öffenenden Klammern durch gerade
Ziffern, die schliessenden Klammern durch ungerade Ziffern ersetzt.
Da zwischen den öffnenden und schliessenden Ziffern kein weiterer
Zusammenhang besteht, genügt es auf dem Stack über die Anzahl
der geraden und ungeraden Ziffern Buch zu führen. Der Automat legt
also ein Symbol $G$ auf den Stack, wenn eine gerade Ziffer gelesen
wird, und entfernt ein Symbol $G$ vom Stack, wenn eine ungerade
Ziffer gelesen wird. Ein Wort wird akzeptiert, wenn am Ende des
Wortes der Stack leer ist.

Als Zustandsdiagramm kann man
\[
\entrymodifiers={++[o][F]}
\xymatrix{
*+\txt{}\ar[r]
        &{z_0} \ar[dr]^{\varepsilon,\varepsilon\rightarrow \$}
\\
*+\txt{}
        &*+\txt{}
                &{z_1}\ar@(ur,dr)^{{G,\varepsilon \rightarrow G}\atop{U,G\rightarrow\varepsilon}} \ar[dl]^{\varepsilon,\$\rightarrow\varepsilon}
\\
*+\txt{}
        &*++[o][F=]{z_2}
}
\]
verwenden.

Alternativ, jedoch einiges aufwendiger, kann auch der Algorithmus
zur Konstruktion eines PDA aus einer
Grammatik verwendet werden, der in der Vorlesung skizziert worden ist.
\end{loesung}
