Das Wort \texttt{(1+2)-((3)+4)} kann mit untenstehendem Parse Tree
aus der Grammatik erzeugt werden.
Ergänzen Sie die Variablen und finden Sie die Unterteilungen des
Wortes gemäss dem Beweis des Pumping-Lemma, mit der sich Klammern
aufpumpen lassen.
\begin{center}
\def\dx{0.5}
\def\dy{0.3}
\def\punkt#1#2{
	({((#1)-0.5*(#2))*\dx},{(-(#2))*\dy})
}
\def\gabel#1#2#3#4{
	\draw \punkt{#1}{#2} -- \punkt{#1}{#2+#3};
	\draw \punkt{#1}{#2} -- \punkt{#1+#4}{#2+#4};
}
\def\wort{
	\node at \punkt{0.25}{12.5}  {\texttt{(}\strut};
	\node at \punkt{1.25}{12.5}  {\texttt{1}\strut};
	\node at \punkt{2.25}{12.5}  {\texttt{+}\strut};
	\node at \punkt{3.25}{12.5}  {\texttt{2}\strut};
	\node at \punkt{4.25}{12.5}  {\texttt{)}\strut};
	\node at \punkt{5.25}{12.5}  {\texttt{+}\strut};
	\node at \punkt{6.25}{12.5}  {\texttt{(}\strut};
	\node at \punkt{7.25}{12.5}  {\texttt{(}\strut};
	\node at \punkt{8.25}{12.5}  {\texttt{3}\strut};
	\node at \punkt{9.25}{12.5}  {\texttt{)}\strut};
	\node at \punkt{10.25}{12.5} {\texttt{+}\strut};
	\node at \punkt{11.25}{12.5} {\texttt{4}\strut};
	\node at \punkt{12.25}{12.5} {\texttt{)}\strut};
}
\begin{tikzpicture}[>=latex,thick]
\begin{scope}[xshift=-0.5*\textwidth]
\node at \punkt{6.5}{6.5} {$\begin{aligned}
A & \to AB \mid OC \\
  & \to \texttt{0}\mid\texttt{1}\mid \ldots\mid\texttt{9} \\
B & \to ZA \\
C & \to AS \\
Z & \to \texttt{+}  \mid \texttt{-} \\
O & \to \texttt{(} \\
S & \to \texttt{)} \\
\end{aligned}$};
\end{scope}
\gabel{0}{0}{8}{5}
	\gabel{0}{8}{4}{1}
		\gabel{1}{9}{1}{3}
			\gabel{1}{10}{2}{1}
				\gabel{2}{11}{1}{1}
	\gabel{5}{5}{7}{1}
		\gabel{6}{6}{6}{1}
			\gabel{7}{7}{1}{5}
				\gabel{7}{8}{2}{3}
					\gabel{7}{10}{2}{1}
						\gabel{8}{11}{1}{1}
					\gabel{10}{11}{1}{1}
\begin{scope}[yshift=-0.1cm]
\wort
\end{scope}
\end{tikzpicture}
\end{center}

\begin{loesung}
Die Grammatik hat zwar nicht ganz Chomsky-Normalform, weil die Startvariable
auch auf der rechten Seite vorkommt, für die Zwecke des Pumping Lemmas
reicht diese Form jedoch.
\begin{center}
\def\dx{0.6}
\def\dy{0.8}
\def\punkt#1#2{
	({((#1)-0.5*(#2))*\dx},{(-(#2))*\dy})
}
\def\terminal#1#2#3{
	\fill[color=white,opacity=0.8] \punkt{#1}{#2} circle[radius=0.20];
	\node at \punkt{#1}{#2} {$#3\mathstrut$};
}
\def\gabel#1#2#3#4#5{
	\draw \punkt{#1}{#2} -- \punkt{#1}{#2+#3};
	\draw \punkt{#1}{#2} -- \punkt{#1+#4}{#2+#4};
	\fill[color=white,opacity=0.8] \punkt{#1}{#2} circle[radius=0.20];
	\node at \punkt{#1}{#2} {$#5\mathstrut$};
}
\def\wort{
	\node at \punkt{0.25}{12.5}  {\texttt{(}\strut};
	\node at \punkt{1.25}{12.5}  {\texttt{1}\strut};
	\node at \punkt{2.25}{12.5}  {\texttt{+}\strut};
	\node at \punkt{3.25}{12.5}  {\texttt{2}\strut};
	\node at \punkt{4.25}{12.5}  {\texttt{)}\strut};
	\node at \punkt{5.25}{12.5}  {\texttt{+}\strut};
	\node at \punkt{6.25}{12.5}  {\texttt{(}\strut};
	\node at \punkt{7.25}{12.5}  {\texttt{(}\strut};
	\node at \punkt{8.25}{12.5}  {\texttt{3}\strut};
	\node at \punkt{9.25}{12.5}  {\texttt{)}\strut};
	\node at \punkt{10.25}{12.5} {\texttt{+}\strut};
	\node at \punkt{11.25}{12.5} {\texttt{4}\strut};
	\node at \punkt{12.25}{12.5} {\texttt{)}\strut};
}
\def\dreieck#1#2#3{
	\fill[color=#3!10]
		\punkt{#1-0.5}{12.3} rectangle ++({(13.3-#2)*\dx},-0.5);
	\fill[color=#3!20] \punkt{#1-0.5}{12.3}
		-- \punkt{#1-0.5}{#2-1}
		-- \punkt{#1+0.5+(12.3-#2)}{12.3};
	\draw[color=#3] \punkt{#1-0.5}{12.3}
		-- \punkt{#1-0.5}{#2-1}
		-- \punkt{#1+0.5+(12.3-#2)}{12.3};
}
\begin{tikzpicture}[>=latex,thick]

\dreieck{0}{0}{blue}
\dreieck{7}{10}{red}
\dreieck{8}{12}{darkgreen}
\gabel{0}{0}{8}{5}{A}
	\gabel{0}{8}{4}{1}{A}
		\gabel{1}{9}{1}{3}{C}
			\gabel{1}{10}{2}{1}{A}
				\gabel{2}{11}{1}{1}{B}
	\gabel{5}{5}{7}{1}{B}
		\gabel{6}{6}{6}{1}{A}
			\gabel{7}{7}{1}{5}{C}
				\gabel{7}{8}{2}{3}{A}
					\gabel{7}{10}{2}{1}{A}
						\gabel{8}{11}{1}{1}{C}
					\gabel{10}{11}{1}{1}{B}
\terminal{0}{12}{O}
\terminal{1}{12}{A}
\terminal{2}{12}{Z}
\terminal{3}{12}{A}
\terminal{4}{12}{S}
\terminal{5}{12}{Z}
\terminal{6}{12}{O}
\terminal{7}{12}{O}
\terminal{8}{12}{A}
\terminal{9}{12}{S}
\terminal{10}{12}{Z}
\terminal{11}{12}{A}
\terminal{12}{12}{S}
\begin{scope}[yshift=-0.1cm]
	\wort
\end{scope}


\end{tikzpicture}
\end{center}
Daraus lassen sich die folgende Aufteilung des Wortes ablesen,
welche die innersten Klammern aufzupumpen ermöglicht:
\[
\bgroup
\color{blue}
\texttt{(1+2)+(}
\color{red}
\texttt{(}
\color{darkgreen}
\texttt{4}
\color{red}
\texttt{)}
\color{blue}
\texttt{+4)}
\egroup
\]
Die roten Teile können beliebig aufgepumpt werden, wobei immer
Wörter der Sprache entstehen.
\end{loesung}



