Im Spiel {\em Corral} (vergleiche auch Aufgabe~\ref{70000060})
muss ein geschlossener Pfad beschrieben werden.
Eine Möglichkeit, einen solchen Pfad zu spezifizieren, ist, zunächst
die Koordinaten des Startpunkts anzugeben und anschliessend eine
Folge von Buchstaben aus der Mengen
$\Sigma=\{\texttt{u},\texttt{d},\texttt{l},\texttt{r}\}$,
wobei jeder Buchstabe für ein Segment des Pfades steht, welches 
nach oben (\texttt{u}),
unten (\texttt{d}),
links (\texttt{l})
bzw.~rechts (\texttt{r})
zeigt.
Sie $L$ die Sprache
\[
L
=
\{
w\in \Sigma^*
\;|\;
\text{$w$ beschreibt einen geschlossenen Pfad}
\}
\]
der Wörter, die geschlossene Pfade beschreiben.
Können Sie eine kontextfreie Grammatik für $L$ angeben?

\begin{loesung}
\definecolor{darkgreen}{rgb}{0,0.8,0}
Damit ein Pfad geschlossen ist, müssen gleich viele 
\texttt{u} wie
\texttt{d} und
gleichviele
\texttt{l} wie \texttt{r} 
in $w$ vorkommen, die Sprache $L$ ist also auch
\begin{equation}
L=\{
w\in\Sigma^*
\;|\;
|w|_{\texttt{u}} = |w|_{\texttt{d}}
\wedge
|w|_{\texttt{l}} = |w|_{\texttt{r}}
\}
\label{40000073:cond}
\end{equation}
Eine kontextfreie Grammatik für $L$ gibt es aber nicht, denn
$L$ ist nicht kontextfrei, wie man mit dem Pumping Lemma für kontextfreie
Sprachen zeigen kann.

\begin{enumerate}
\item
Wir nehmen an, dass $L$ kontextfrei ist.
\item
Nach dem Pumping Lemma für kontextfreie Sprachen gibt es die
Pumping Length $N$
\item
Wir nehmen das Wort
$w= \texttt{u}^N \texttt{l}^N \texttt{d}^N \texttt{r}^N $,
es beschreibt ein im Gegenuhrzeigersinn durchlaufenes Quadrat mit
Seitenlänge $N$.
\item
Nach dem Pumping-Lemma gibt es eine Aufteilung des Wortes in
fünf Teile
$w=
{\color{blue}u}
{\color{red}v}
{\color{darkgreen}x}
{\color{red}y}
{\color{blue}z}$
\begin{center}
\begin{tikzpicture}[>=latex,thick]
\def\xa{0}
\def\xb{3.5}
\def\xc{7}
\def\xd{10.5}
\def\xe{14}
\def\y{0.7}
\draw (0,0) rectangle (\xe,\y);
\draw (\xb,0) -- (\xb,\y);
\draw (\xc,0) -- (\xc,\y);
\draw (\xd,0) -- (\xd,\y);
\node[color=gray] at ({0.5*(\xa+\xb)},{0.5*\y}) {$\texttt{u}\mathstrut^N$};
\node[color=gray] at ({0.5*(\xb+\xc)},{0.5*\y}) {$\texttt{l}\mathstrut^N$};
\node[color=gray] at ({0.5*(\xc+\xd)},{0.5*\y}) {$\texttt{d}\mathstrut^N$};
\node[color=gray] at ({0.5*(\xd+\xe)},{0.5*\y}) {$\texttt{r}\mathstrut^N$};
\def\s{0.05}
\def\Xb{5}
\pgfmathparse{\Xb+1.1}
\xdef\Xc{\pgfmathresult}
\pgfmathparse{\Xc+0.6}
\xdef\Xd{\pgfmathresult}
\pgfmathparse{\Xd+1.3}
\xdef\Xe{\pgfmathresult}


\def\rechteck#1#2#3#4{
	\fill[color=#3!20,opacity=0.5]
		({#1+\s},\s) rectangle ({#2-\s},{\y-\s});
	\draw[color=#3] ({#1+\s},\s) rectangle ({#2-\s},{\y-\s});
	\node[color=#3] at ({0.5*(#1+#2)},{0.5*\y}) {$#4\mathstrut$};
}

\rechteck{\xa}{\Xb}{blue}{u}
\rechteck{\Xb}{\Xc}{red}{v}
\rechteck{\Xc}{\Xd}{darkgreen}{x}
\rechteck{\Xd}{\Xe}{red}{y}
\rechteck{\Xe}{\xe}{blue}{z}

\draw[line width=0.3pt] (\Xb,\y) -- (\Xb,-1);
\draw[line width=0.3pt] (\Xe,\y) -- (\Xe,-1);
\draw[<->] (\Xb,-0.8) -- (\Xe,-0.8);
\node at ({0.5*(\Xb+\Xe)},-0.8) [above] {$\le N$};

\end{tikzpicture}
\end{center}
mit $|x|>0$ und $|vxy|\le N$ derart,
dass auch die aufgepumpten Wörter $uv^kxy^kz\in L$ sind.
\item
Wegen $|vxy|\le N$ kann sich beim Pumpen höchstens die Anzahl von
zwei der Buchstaben ändern, niemals aber die Anzahl beider Buchstaben
gleichzeitig, die jeweils gleich sein sollte.
\item
Dieser Widerspruch zeigt, dass $L$ nicht kontextfrei sein kann.
\qedhere
\end{enumerate}
\end{loesung}

\begin{bewertung}
Bedingung \eqref{40000073:cond} ({\bf B}) 1 Punkt,
Pumping Lemma und Pumping Length ({\bf N}) 1 Punkt,
Wahl des Wortes ({\bf W}) 1 Punkt,
Unterteilung ({\bf U}) 1 Punkt,
Widerspruch beim Pumpen ({\bf P}) 1 Punkt,
Schlussfolgerung $L$ nicht kontextfrei ({\bf S}) 1 Punkt.
\end{bewertung}
