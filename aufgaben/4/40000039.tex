Taschenrechner mit umgekehrter polnischer Notation verwenden einen Stack.
Mit einer speziellen Taste \keys{ENTER} werden Zahlen auf den Stack
geschoben. Dies geschieht auch implizit bei Verwendung einer
Operationstaste, also einer der Tasten \keys{{$+$}}, \keys{{$-$}},
\keys{{$\times$}}, \keys{{$\div$}}, die ausserdem die obersten
beiden Elemente auf dem Stack durch das Resultat der Rechenoperation 
ersetzt. Die Tastenfolgen, die dazu führen, dass alle Zahlen auf dem 
Stack verarbeitet sind, bilden eine Sprache.

\begin{teilaufgaben}
\item Warum ist die von so einem Taschenrechner akzeptiert Sprache
kontextfrei?
\item Stellen Sie eine Grammatik dieser Sprache auf.
\item Ist Ihre Grammatik in Chomsky-Normalform? Wenn nein, bringen Sie sie
in Chomsky-Normalform.
\end{teilaufgaben}

\begin{hinweis}
Nehmen Sie zur Vereinfachung an, dass der Taschenrechner nur für ganze
Zahlen gebaut ist. Kürzen Sie ganze Zahlen in der Grammatik durch
die Variable \textsl{zahl} ab, und verzichten sie darauf, die Auflösung
dieser Variable in der Grammatik durchzuführen.
\end{hinweis}

\thema{Grammatik}
\thema{Chomsky-Normalform}
\thema{kontextfrei}

\begin{loesung}
\begin{teilaufgaben}
\item Die Sprache kann offenbar von einem Stackautomaten verarbeitet werden,
also ist sie kontextfrei.
\item Die folgende Grammatik erzeugt korrekte Tastenfolgen:
\begin{align*}
S&\rightarrow S \; S \; \textsl{operationstaste}
\\
 &\rightarrow S \; \textsl{zahl} \; \textsl{operationstaste}
\\
 &\rightarrow \textsl{zahl}\;\keys{ENTER}\; S\;\textsl{operationstaste}
\\
 &\rightarrow \textsl{zahl}\;\keys{ENTER}\; \textsl{zahl}\;\textsl{operationstaste}
\\
\textsl{operationstaste}&\rightarrow
\keys{{$+$}}
\mid 
\keys{{$-$}}
\mid 
\keys{{$\times$}}
\mid 
\keys{{$\div$}}
\\
\end{align*}
Darin steht die Variable $S$ für eine Folge von Tastendrücken, die
alle Zahlen auf dem Stack verarbeitet.
\item
Die Grammatik hat nicht Chomsky-Normalform, aber sie ist leicht in
CNF zu bringen. Die Bedingung, dass die Startvariable auf der rechten
Seite nicht vorkommen darf, wird durch die Grammatik
\begin{align*}
\color{red}S_0&\color{red}\rightarrow S
\\
S&\rightarrow S \; S \; \textsl{operationstaste}
\\
 &\rightarrow S \; \textsl{zahl} \; \textsl{operationstaste}
\\
 &\rightarrow \textsl{zahl}\;\keys{ENTER}\; S\;\textsl{operationstaste}
\\
 &\rightarrow \textsl{zahl}\;\keys{ENTER}\; \textsl{zahl}\;\textsl{operationstaste}
\\
\textsl{operationstaste}&\rightarrow
\keys{{$+$}}
\mid 
\keys{{$-$}}
\mid 
\keys{{$\times$}}
\mid 
\keys{{$\div$}}
\end{align*}
erfüllt. Diese Grammatik hat aber eine Unit-Rule $S_0\to S$, die als nächstes
eliminiert werden muss:
\begin{align*}
\color{red}S_0&\color{red}\rightarrow S \; S \; \textsl{operationstaste}
\\
 &\color{red}\rightarrow S \; \textsl{zahl} \; \textsl{operationstaste}
\\
 &\color{red}\rightarrow \textsl{zahl}\;\keys{ENTER}\; S\;\textsl{operationstaste}
\\
 &\color{red}\rightarrow \textsl{zahl}\;\keys{ENTER}\; \textsl{zahl}\;\textsl{operationstaste}
\\
S&\rightarrow S \; S \; \textsl{operationstaste}
\\
 &\rightarrow S \; \textsl{zahl} \; \textsl{operationstaste}
\\
 &\rightarrow \textsl{zahl}\;\keys{ENTER}\; S\;\textsl{operationstaste}
\\
 &\rightarrow \textsl{zahl}\;\keys{ENTER}\; \textsl{zahl}\;\textsl{operationstaste}
\\
\textsl{operationstaste}&\rightarrow
\keys{{$+$}}
\mid 
\keys{{$-$}}
\mid 
\keys{{$\times$}}
\mid 
\keys{{$\div$}}
\end{align*}
Einige Regeln beinhalten mehr als zwei Symbole auf der rechten Seite, 
wobei eines auch ein Terminalsymbol sein kann. Wir eliminieren die
Terminalsymbole durch eine neue Regel:
\begin{align*}
S_0&\rightarrow S \; S \; \textsl{operationstaste}
\\
 &\rightarrow S \; \textsl{zahl} \; \textsl{operationstaste}
\\
 &\rightarrow \textsl{zahl}\; {\color{red}E} \; S\;\textsl{operationstaste}
\\
 &\rightarrow \textsl{zahl}\; {\color{red}E} \; \textsl{zahl}\;\textsl{operationstaste}
\\
S&\rightarrow S \; S \; \textsl{operationstaste}
\\
 &\rightarrow S \; \textsl{zahl} \; \textsl{operationstaste}
\\
 &\rightarrow \textsl{zahl} \; {\color{red}E} \; S\;\textsl{operationstaste}
\\
 &\rightarrow \textsl{zahl}\; {\color{red}E} \; \textsl{zahl}\;\textsl{operationstaste}
\\
\textsl{operationstaste}&\rightarrow
\keys{{$+$}}
\mid 
\keys{{$-$}}
\mid 
\keys{{$\times$}}
\mid 
\keys{{$\div$}}
\\
\color{red}E&\color{red}\rightarrow\keys{ENTER}
\end{align*}
Die Folgen von mehr als zwei Variablen auf der rechten Seite können jetzt
ebenfalls eliminiert werden:
\begin{align*}
S_0&\rightarrow {\color{red}A_1} \; \textsl{operationstaste}
\\
\color{red}A_1&\color{red}\rightarrow S \; S 
\\
S_0&\rightarrow {\color{red}A_2} \; \textsl{operationstaste}
\\
\color{red}A_2&\color{red}\rightarrow S \; \textsl{zahl}
\\
S_0&\rightarrow {\color{red}A_3} \; \textsl{operationstaste}
\\
\color{red}A_3&\color{red}\rightarrow A_4 \; S
\\
\color{red}A_4&\color{red}\rightarrow \textsl{zahl} \; E
\\
S_0&\rightarrow {\color{red}A_5} \; \textsl{operationstaste}
\\
\color{red}A_5&\color{red}\rightarrow A_4 \; \textsl{zahl}
\\
S&\rightarrow {\color{red}A_1} \; \textsl{operationstaste}
\\
S&\rightarrow {\color{red}A_2} \; \textsl{operationstaste}
\\
S&\rightarrow {\color{red}A_3} \;\textsl{operationstaste}
\\
S&\rightarrow {\color{red}A_5} \; \textsl{operationstaste}
\\
\textsl{operationstaste}&\rightarrow
\keys{{$+$}}
\mid 
\keys{{$-$}}
\mid 
\keys{{$\times$}}
\mid 
\keys{{$\div$}}
\\
E&\rightarrow\keys{ENTER}
\end{align*}
Damit hat die Grammatik Chomsky-Normalform.
\qedhere
\end{teilaufgaben}
\end{loesung}

\begin{bewertung}
\begin{teilaufgaben}
\item
Sprachen, die mit einem Stackautomaten verarbeitet werden können, sind
kontextfrei ({\bf A}) 1 Punkt,
\item
Grammatik erlaubt beliebige Berechnungen zu verknüpfen wie in der
Regel $S\to S\;S\;\textsl{operationstaste}$ ({\bf R}) 1 Punkt,
Regeln für Verknüpfung von Zahlen mit Ausdrücken ({\bf ZS/SZ}) 1 Punkt,
Regel für Verknüpfung zweier Zahlen ({\bf 2}) 1 Punkt,
\item
Unit Rule entfernt ({\bf U}) 1 Punkt, Auflösung von Folgen mit mehr
als 2 Variablen ({\bf F}) 1 Punkt.
\end{teilaufgaben}
\end{bewertung}

