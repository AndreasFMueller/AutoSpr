Der Sprachforscher Stuart M.~Shieber hat aus Beispielen wie dem Satz
\begin{center}
De Jan säit das mer d'Chind em Hans es Hus händ wele laa hälfe aastriiche.
\end{center}
abgeleitet, dass das Schweizderdeutsch grammatische Konstruktionen
zulässt, die auf Wörter der Form
\[
wa^mb^nxc^md^ny
\]
hinaus laufen\footnote{Stuart M.~Shieber,
{\em Evidence against the context-freeness of natural language},
Linguistics and Philosophy, {\bf 8} (1985) 333--343}.
Zeigen Sie, dass diese Sprache nicht kontextfrei ist.

\thema{Pumping Lemma für kontextfreie Sprachen}
\thema{kontextfrei}

\begin{loesung}
Wir wenden das Pumping Lemma für kontextfreie Sprachen auf die Sprache
\[
L=\{
wa^mb^nxc^md^ny
\;|\; m,n\ge 0
\}
\]
an.
\begin{enumerate}
\item Annahme: $L$ ist kontextfrei.
\item Nach dem Pumping-Lemma gibt es die Pumping-Length $N$.
\item Wir konstruieren das Beispielwort
\[
z=wa^Nb^Nxc^Nd^Ny,
\]
es ist offensichtlich in der Sprache $L$
und es ist ausreichend lang, dass die Schlussfolgerungen des Pumping-Lemma
darauf anwendbar sind.
\item
Gemäss Pumping Lemma gibt es eine Unterteilung
\[
z=pqrst,
\]
wobei $|qrs|\le N$ gelten muss.
Es folgt, dass $q$ und $s$ nur jeweils zwei benachbarte der
vier ``langen'' Blöcke $a^N$, $b^N$, $c^N$ oder $d^N$ berühren können.
\item 
Beim Aufpumpen werden zwei benachbarte langen Blöcke verändert.
Die Bedingung, dass alternierende Blöcke, also $a^N$ und $c^N$
bzw.~$b^N$ und $d^N$ gleich lang sein müssen, wird nach dem Pumpen
daher nicht mehr erfüllt sein.
Ein aufgepumptes Wort wird daher nicht mehr in $L$ sein.
\item Dieser Widerspruch zeigt, dass die Annahme, $L$ sei kontextfrei,
nicht haltbar ist.
\qedhere
\end{enumerate}
\end{loesung}

\begin{diskussion}
Dies bedeutet natürlich nicht, dass Schweizerdeutsch eine nicht
kontextfreie Sprache ist.
In der Wirklichkeit sind die Sätze nämlich immer von beschränkter Länge,
die Exponenten $n$ und $m$ können daher nicht beliebig gross sein.
Das müssen Sie aber, denn der Pumping-Lemma-Beweis verlangt, dass man
$n=m=N$ setze können muss, wobei die Pumping-Length $N$ eben sehr gross
sein kann.
\end{diskussion}

\begin{bewertung}
Jeder Punkt des Pumping-Lemma-Beweises ein Punkt:
Annahme ({\bf A}), Pumping Length ({\bf N}), Beispielwort ({\bf W}),
Zerlegung ({\bf Z}), Widerspruch beim Pumpen ({\bf P}), 
Folgerung ({\bf F}).
Der Punkt ({\bf Z}) wird nur gegeben, wenn zum Beispiel aus dem nachfolgenden
Pumpschritt klar wird, dass es mehrere mögliche Unterteilungen gibt.
\end{bewertung}

