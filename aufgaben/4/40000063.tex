Betrachten Sie die kontextfreie Grammatik
\begin{align*}
S&\to \texttt{a} S \texttt{c} \;|\; X
\\
X&\to \texttt{b}X\texttt{c} \;|\; \varepsilon
\end{align*}
\begin{teilaufgaben}
\item
Welche Sprache wird von dieser Grammatik beschrieben?
Schreiben Sie die Sprache in Mengennotation 
$L=\{
w\in\{\texttt{a},\texttt{b},\texttt{c}\}^*
\;|\dots\}$.
\item
Hat die Grammatik Chomsky-Normalform?
Wenn nicht, bringen Sie sie in Chomsky Normalform.
\end{teilaufgaben}


\begin{loesung}
\begin{teilaufgaben}
\item
Die zweite Zeile erzeugt Wörter der Form $\texttt{b}^n\texttt{c}^n$.
Die erste Regel ermöglicht, ein solches Wort zwischen \texttt{a} und
\texttt{c} einzuschachteln, so dass ein Wort der Form
$\texttt{a}^m\texttt{b}^n\texttt{c}^{n+m}$ entsteht.
Die Sprache ist also
\[
L=\{
w\in\{\texttt{a},\texttt{b},\texttt{c}\}^*
\;|\;
\text{$
w=
\texttt{a}^m \texttt{b}^n \texttt{c}^{k}$
mit
$n+m=k$
}
\}.
\]
\item
1. Schritt: Startvariable darf auf der rechten Seite nicht vorkommen:
\begin{align*}
S_0 & \to S
\\
S & \to \texttt{a} S \texttt{c} \;|\; X
\\
X & \to \texttt{b}X\texttt{c} \;|\; \varepsilon
\end{align*}
2. Schritt: $\varepsilon$-Regel entfernen.
Die $\varepsilon$-Regel $X\to\varepsilon$ bedeuetet, dass $X$ weggelassen
werden kann:
\begin{align*}
S_0 & \to S
\\
S & \to \texttt{a} S \texttt{c} \;|\; X \;|\; \varepsilon
\\
X & \to \texttt{b}X\texttt{c} \;|\; \texttt{bc}
\end{align*}
Dadurch ist eine neue $\varepsilon$-Regel $S\to\varepsilon$ entstanden,
welche bedeutet, dass $S$ ebenfalls weggelassen werden kann:
\begin{align*}
S_0 & \to S \;|\; \varepsilon
\\
S & \to
\texttt{a} S \texttt{c} \;|\; \texttt{ac} \;|\; X
\\
X & \to \texttt{b}X\texttt{c} \;|\; \texttt{bc}
\end{align*}
3. Schritt: Unit rules entfernen.
Zunächst die Regel $S_0\to S$:
\begin{align*}
S_0 & \to
\texttt{a} S \texttt{c} \;|\; \texttt{ac} \;|\; X \;|\; \varepsilon
\\
S & \to
\texttt{a} S \texttt{c} \;|\; \texttt{ac} \;|\; X
\\
X & \to \texttt{b}X\texttt{c} \;|\; \texttt{bc}
\end{align*}
Es ist eine neue Regel $S_0\to X$ entstanden:
\begin{align*}
S_0 & \to
\texttt{a} S \texttt{c} \;|\; \texttt{ac} \;|\;
\texttt{b}X\texttt{c} \;|\; \texttt{bc}
\;|\; \varepsilon
\\
S & \to
\texttt{a} S \texttt{c} \;|\; \texttt{ac} \;|\; X
\\
X & \to \texttt{b}X\texttt{c} \;|\; \texttt{bc}
\end{align*}
Schliesslich bleibt $S\to X$:
\begin{align*}
S_0 & \to
\texttt{a} S \texttt{c} \;|\; \texttt{ac} \;|\;
\texttt{b}X\texttt{c} \;|\; \texttt{bc}
\;|\; \varepsilon
\\
S & \to
\texttt{a} S \texttt{c} \;|\; \texttt{ac} \;|\;
\texttt{b}X\texttt{c} \;|\; \texttt{bc}
\\
X & \to \texttt{b}X\texttt{c} \;|\; \texttt{bc}
\end{align*}
4. Schritt: Terminalsymbole müssen separat umgesetzt werden:
\begin{align*}
S_0 & \to
ASC \;|\; AC \;|\;
BXC \;|\; BC
\;|\; \varepsilon
\\
S & \to
A S C \;|\; AC \;|\;
BXC \;|\; BC
\\
X & \to BXC \;|\; BC
\\
A&\to \texttt{a} \\
B&\to \texttt{b} \\
C&\to \texttt{c}
\end{align*}
und Tripel müssen in zwei Schritten aufgebaut werden:
\begin{align*}
S_0 & \to AU \;|\; AC \;|\; BV \;|\; BC \;|\; \varepsilon
\\
S & \to AU \;|\; AC \;|\; BV \;|\; BC
\\
X & \to BV \;|\; BC
\\
U&\to SC
\\
V & \to XC
\\
A&\to \texttt{a} \\
B&\to \texttt{b} \\
C&\to \texttt{c}
\end{align*}
Damit ist Chomsky-Normalform erreicht.
\qedhere
\end{teilaufgaben}
\end{loesung}

\begin{bewertung}
\begin{teilaufgaben}
\item Gleiche viele \texttt{c} wie \texttt{a} und \texttt{b} zusammen
({\bf G}) 1 Punkt,
Reihenfolge ({\bf R}) 1 Punkt,
\item Startvariable ({\bf S}) 1 Punkt,
$\varepsilon$-Regel ({\bf E}) 1 Punkt,
Unit rules ({\bf U}) 1 Punkt,
Terminalsymbole und 3er-Regeln ({\bf T}) 1 Punkt.
\end{teilaufgaben}
\end{bewertung}
