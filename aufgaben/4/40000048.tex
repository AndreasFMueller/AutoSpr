Bringen Sie die Grammatik mit den Regeln
\begin{align*}
S&\rightarrow (S)S
\\
 &\rightarrow \varepsilon
\end{align*}
in Chomsky-Normalform.

\begin{loesung}
Die Reduktion auf Chomsky-Normalform vollzieht sich in den folgenden Schritten:
\begin{enumerate}
\item
Neue Startvariable:
\begin{align*}
S_0&\rightarrow S
\\
S&\rightarrow \texttt{(}S\texttt{)}S
\\
 &\rightarrow \varepsilon
\end{align*}
\item
$\varepsilon$-Regeln entfernen:
\begin{align*}
S_0&\rightarrow S \;|\; \varepsilon
\\
S&\rightarrow \texttt{(}S\texttt{)}S \;|\; \texttt{(}\texttt{)}S \;|\;
  \texttt{(}S\texttt{)} \texttt{(}\texttt{)}
\end{align*}
\item
Unit-Rules entfernen: Es muss nur die Unit-Rule $S_0\to S$ entfernt
werden:
\begin{align*}
S_0& \rightarrow \texttt{(}S)S \;|\; \texttt{(})S \;|\; \texttt{(}S) \;|\;
	\texttt{(})
\\
   &\rightarrow \varepsilon
\\
S&\rightarrow \texttt{(}S\texttt{)}S \;|\; \texttt{(}\texttt{)}S \;|\;
	\texttt{(}S\texttt{)} \;|\; \texttt{(}\texttt{)}
\end{align*}
\item Terminalsymbole in eigene Regeln auslagern:
\begin{align*}
S_0& \rightarrow ASBS \;|\; ABS \;|\; ASB \;|\; AB
\\
   &\rightarrow \varepsilon
\\
S&\rightarrow ASBS \;|\; ABS \;|\; ASB \;|\; AB
\\
A&\rightarrow \texttt{(}
\\
B&\rightarrow \texttt{)}
\end{align*}
Die rechten Seiten mit mehr als zwei Symbolen müssen nun noch Schrittweise
aufgebaut werden:
\begin{align*}
S_0& \rightarrow AU_1 \;|\; AU_2 \;|\; AV \;|\; AB
\\
   &\rightarrow \varepsilon
\\
S&\rightarrow AU_1 \;|\; AU_2 \;|\; AV \;|\; AB
\\
A&\rightarrow \texttt{(}
\\
B&\rightarrow \texttt{)}
\\
U_1&\rightarrow SU_2\\
U_2&\rightarrow BS\\
V  &\rightarrow SB
\end{align*}
\end{enumerate}
Damit ist die Grammatik vollständig in Chomsky-Normalform umgewandelt.
\end{loesung}

\begin{bewertung}
Startvariable ({\bf S}) 1 Punkt,
$\varepsilon$-Regel ({\bf E}) 1 Punkt,
Unit rule ({\bf U}) 1 Punkt,
Terminalsymbole in eigene Regeln ({\bf T}) 1 Punkt,
Schrittweiser Aufbau langer Regeln ({\bf S}) 2 Punkt.
\end{bewertung}

