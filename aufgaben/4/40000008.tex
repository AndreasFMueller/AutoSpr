In der Vorlesung wurde die folgende Grammatik $G$ für arithmetische
Ausdrücke vorgestellt
\begin{align*}
E&\to E{\tt +}T\;|\; T\\
T&\to T{\tt \times}F\;|\;F\\
F&\to {\tt (}\,E\,{\tt )}\;|\; {\tt a}
\end{align*}
Verwandeln Sie diese Grammatik in einen Stackautomaten, der die gleiche
Sprache akzeptiert.

\thema{Stackautomat}

\begin{loesung}
Das Gerüst des Automaten ist
\[
\entrymodifiers={++[o][F]}
\xymatrix{
*+\txt{}\ar[r]
        &{}\ar[r]^{\varepsilon,\varepsilon\to{\tt \$}}
                &{}\ar[r]^{\varepsilon,\varepsilon\to S}
                        &R\ar[r]^{\varepsilon,{\tt\$}\to\varepsilon}
                                &*++[o][F=]{}
}
\]
Dem Zustand fügen wir jetzt Schrittweise die den Regeln
entsprechenden "Ubergänge hinzu. Zuerst $E\to E{\tt +}T\;|\; T$:
\[
\entrymodifiers={++[o][F]}
\xymatrix{
R       \ar[r]^{\varepsilon,E\to T}
        \ar@(ul,dl)_{\varepsilon,E\to T}
        &{}     \ar[r]^{\varepsilon,\varepsilon\to {\tt +}}
                &{}     \ar@/^10pt/[ll]^{\varepsilon,\varepsilon\to E}
}
\]
Dann $T\to T{\tt \times}F\;|\;F$:
\[
\entrymodifiers={++[o][F]}
\xymatrix{
R       \ar[r]^{\varepsilon,T\to F}
        \ar@(ul,dl)_{\varepsilon,T\to F}
        &{}     \ar[r]^{\varepsilon,\varepsilon\to {\tt \times}}
                &{}     \ar@/^10pt/[ll]^{\varepsilon,\varepsilon\to T}
}
\]
Und zum Schluss $F\to {\tt (}\,E\,{\tt )}\;|\; {\tt a}$:
\[
\entrymodifiers={++[o][F]}
\xymatrix{
R       \ar[r]^{\varepsilon,F\to {\tt )}}
        \ar@(ul,dl)_{\varepsilon,F\to {\tt a}}
        &{}     \ar[r]^{\varepsilon,\varepsilon\to E}
                &{}     \ar@/^10pt/[ll]^{\varepsilon,\varepsilon\to {\tt (}}
}
\]
Jetzt brauchen wir nur noch die Regeln, die die Terminalsymbole mit dem
Input verarbeiten:
\[
\entrymodifiers={++[o][F]}
\xymatrix{
R       \ar@(u,ur)^{{\tt a},{\tt a}\to\varepsilon}
        \ar@(r,dr)^{{\tt (},{\tt (}\to\varepsilon}
        \ar@(dr,d)^{{\tt )},{\tt )}\to\varepsilon}
        \ar@(ul,l)_{{\tt +},{\tt +}\to\varepsilon}
        \ar@(l,dl)_{{\tt \times},{\tt \times}\to\varepsilon}
}
\]
Damit ist der Automat vollständig konstruiert.
\end{loesung}
