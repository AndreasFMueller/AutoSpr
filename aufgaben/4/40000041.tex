Man finde eine Grammatik für die Sprache
$L=\{ w\in\Sigma^*\,|\, |w|_{\texttt{1}}=|w|_{\texttt{0}}\}$
über dem Alphabet $\Sigma=\{\texttt{0},\texttt{1}\}$, indem
man sie aus dem folgenden Stackautomaten ableitet:
\[
\entrymodifiers={++[o][F]}
\xymatrix {
*+\txt{}\ar[r]
	&{0}\ar[dr]_{\varepsilon,\varepsilon\to\texttt{\$}}
		&*+\txt{}
			&{2} \ar[dr]^{\varepsilon,\texttt{\$}\to\varepsilon}
			    \ar@(u,ul)_{\texttt{0},\varepsilon\to\texttt{0}}
			    \ar@(u,ur)^{\texttt{1},\texttt{0}\to\varepsilon}
\\
*+\txt{}
	&*+\txt{}
		&{1}\ar[dl]_{\varepsilon,\texttt{\$}\to\varepsilon}
		    \ar[ur]^{\texttt{0},\varepsilon\to\texttt{0}}
		    \ar[dr]_{\texttt{1},\varepsilon\to\texttt{1}}
			&*+\txt{}
				&{4}\ar[ll]_{\varepsilon,\varepsilon\to\texttt{\$}}
\\
*+\txt{}
	&*++[o][F=]{5}
		&*+\txt{}
			&{3} \ar[ur]_{\varepsilon,\texttt{\$}\to\varepsilon}
			    \ar@(d,dl)^{\texttt{1},\varepsilon\to\texttt{1}}
			    \ar@(d,dr)_{\texttt{0},\texttt{1}\to\varepsilon}
}
\]

\begin{hinweis}
Dieser Automat wurde im Wesentlichen in der Vorlesung als Bespiel
hergeleitet. Die Idee des Automaten ist, dass der obere Zweig mit
dem Zustand $2$ verwendet wird, solange mehr \texttt{0} als \texttt{1}
gelesen wurden, der untere Zweig mit dem Zustand $3$ wird
dagegen verwendet, wenn mehr \texttt{1} als \texttt{0} gelesen wurden.
Im oberen Zweig werden die Nullen auf den Stack gelegt, und der Stack wird
mit Einsen wieder abgebaut.
Im unteren Zweig werden die Einsen auf den Stack gelegt, und der Stack
wird mit Nullen wieder abgebaut.
Es ist also möglich, dass während der Verarbeitung eines Wortes
mehrmals zwischen dem oberen und untern Teilautomaten hin- und hergewechselt
wird.
Akzeptabel ist ein Wert, welches den Stack leer lässt, im Zustand $1$
sind immer gleich viele Einsen wie Nullen gelesen worden.
\end{hinweis}

\thema{Stackautomat}
\thema{Grammatik aus Stackautomat}

\begin{loesung}
Dieser Stackautomat erfüllt bereits alle Voraussetzungen, damit
der Algorithmus aus dem Skript angewendet werden kann. Es gibt nur
einen Akzeptierzustand, und der Stack wird vor dem Akzeptieren
eines Wortes geleert.

Der im Skript besprochene Algorithmus verwendet als Variablen der
Grammatik die Symbole $A_{pq}$, die für Wörter stehen, die den
Stackautomaten vom Zustand $p$ in den Zustand $q$ überführen, ohne
den Stack zu verändern. Die Startvariable der so konstruierten Grammatik
ist also $A_{05}$.

Zwischen $0$ und $1$ wird das Zeichen $\texttt{\$}$ auf den Stack gelegt,
und es kann im Schritt zwischen $1$ und $5$ oder im Schritt zwischen $2$
und $4$ oder zwischen $3$ und $4$ wieder entfernt werden. Ausserdem
kann zwischen $4$ und $1$ wieder ein $\texttt{\$}$ auf den Stack gelegt
werden. Daher heissen die ersten Regeln
\begin{align*}
A_{05}&\to \varepsilon A_{11}\varepsilon \\
      &\to A_{04}A_{45}
\end{align*}
Ausgehend vom Zustand $1$ gibt es gar keine Wege, die ohne den Stack zu
verändern wieder bei $1$ ankommen, denn jeder solche Weg müsste über
Zustand $4$ führen, wo ein Zeichen entfernt wird, welches nicht zwischen
$1$ und $4$ auf den Stack gelegt wurde. Daher ist die einzige mögliche
Regel für $A_{11}$
\begin{align*}
A_{11}&\to\varepsilon.
\end{align*}
Für die Variable $A_{04}$ gibt es zwei Regeln, eine, die über den
Zustand $2$ führt, die andere, die über den Zustand $3$ führt:
\begin{align*}
A_{04}&\to \varepsilon A_{12}\varepsilon\\
      &\to \varepsilon A_{13}\varepsilon
\end{align*}
Für $A_{12}$ und $A_{13}$ wiederum bekommt man die Regeln
\begin{align*}
A_{12}&\to\texttt{0} A_{22}\texttt{1}\\
A_{13}&\to\texttt{1} A_{33}\texttt{0}
\end{align*}
Zusätzlich zu den $\varepsilon$-Regeln für die Variablen $A_{22}$ und $A_{33}$
findet man die Regeln, die ein Zeichen auf den Stack legen und am
Schluss wieder abbauen:
\begin{align*}
A_{22}&\to \texttt{0}A_{22}\texttt{1}\;|\;\varepsilon\\
A_{33}&\to \texttt{1}A_{33}\texttt{0}\;|\;\varepsilon
\end{align*}
Diese zwei Regeln besagen, dass $A_{22}$ für Wörter der Form
$\texttt{0}^n\texttt{1}^n$ steht und $A_{33}$ für Wörter der Form
$\texttt{1}^n\texttt{0}^n$ mit $n\ge 0$.
Entsprechend steht die Variablen $A_{12}$  für Wörter der Form
$\texttt{0}^n\texttt{1}^n$ und $A_{13}$ für Wörter der Form
$\texttt{1}^n\texttt{0}^n$ mit $n\ge 1$.

Wir müssen noch die Variable $A_{45}$ weiter entwickeln. Es gibt
drei Wege, wie man von $4$ nach $5$ gelangen kann ohne den
Stack zu verändern. Entweder auf dem direkten Weg oder mit
einer Schleife über den Zustand $2$ oder $3$.
Diesen Wegen entsprechen die Regeln
\begin{align*}
A_{45}&\to\varepsilon A_{11}\varepsilon\to\varepsilon\\
      &\to A_{44}A_{45}\\
A_{44}&\to\varepsilon A_{12}\varepsilon\\
      &\to\varepsilon A_{13}\varepsilon
\end{align*}
Diese Regeln drücken aus, dass $A_{45}$ eine Verkettung von Wörtern
ist, die jeweils aus $A_{44}$ entwickelt werden können.
Letztere Variable kann, wie bereits oben dargelegt,
in Wörter der Form $\texttt{0}^n\texttt{1}^n$ bzw.~$\texttt{1}^n\texttt{0}^n$
mit $n>0$ umgewandelt werden.

Schreiben wir der besseren Lesbarkeit halber $N$ für Wörter, die mit
einer Folgen von Nullen beginnen, also von der
Form $\texttt{0}^n\texttt{1}^n$ mit $n\ge 1$ sind, und $E$ für Wörter,
die mit einer Folge von Einsen beginnen, also von der
Form $\texttt{1}^n\texttt{0}^n$ sind, sieht die Grammatik jetzt so aus
\begin{align*}
A_{05}&\to\varepsilon \\
      &\to A_{04}A_{45} \\
A_{04}&\to A_{04}A_{44} \\
      &\to N \\
      &\to E \\
A_{44}&\to N \\
      &\to E \\
      &\to \varepsilon
\end{align*}
Dies kann man noch etwas vereinfachen zu
\begin{align*}
S&\to \varepsilon\;|\; B\\
B&\to BN\;|\;BE\;|\;N\;|\;E\\
N&\to \texttt{0}N\texttt{1}\;|\; \texttt{01}\\
E&\to \texttt{1}E\texttt{0}\;|\; \texttt{10}
\end{align*}
Darin steht $B$ für eine nicht leere Folge von Wörtern der Art $N$
oder $E$.
\end{loesung}


