Ersetzt man in einem XML-Dokument alle öffnenden Elemente durch {\tt <},
alle schliessenden Elemente durch {\tt >} und alle leeren Elemente
durch {\tt /}, und lässt alles andere weg, bleibt ein Wort über
dem Alphabet $\Sigma =\{{\tt <},{\tt >},{\tt /}\}$ übrig.
Bilden die so konstruierten Wörter eine reguläre Sprache?

\themaL{Pumping Lemma fur regulare Sprachen}{Pumping Lemma für reguläre Sprachen}

\begin{loesung}
Nein, wie wir mit dem Pumping Lemma zeigen können.

Sei $L$ die Sprache der Wörter, die durch den genannten Prozess
entstehen können. In so einem Wort muss die Anzahl `{\tt <}' und die
Anzahl `{\tt >}' gleich gross sein. Wir zeigen, dass dies im Widerspruch
steht zu der Annahme, $L$ sei regulär.

Wir nehmen also an, $L$ sei regulär. Nach dem Pumping Lemma gibt
es also eine Zahl $N$, die pumping length, so dass sich Wörter mit
Länge mindestens $N$ innerhalb der ersten $N$ Zeichen aufpumpen lassen.
Wir wählen als Wort $w=\text{\tt <}^N\text{\tt >}^N$. Dieses Wort
kann sicher aus einem XML-File erzeugt werden. Nach dem Pumping
Lemma kann man $w=xyz$ schreiben, mit $|y|>0$ und $|xy|\le N$, so dass
alle Wörter $xy^kz\in L$ sind. Wegen $|xy|\le N$ besteht aber
$y$ aus lauter `{\tt <}'. Das Wort $xy^kz$ entält also mehr
`{\tt <}' als `{\tt >}' und kann damit nicht in $L$ sein. Dieser
Widerspruch zeigt, dass die Annahme, $L$ sei regulär, falsch gewesen
sein muss. $L$ ist also nicht regulär.
\end{loesung}
