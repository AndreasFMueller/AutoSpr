Sei $\Sigma=\{
\texttt{(},
\texttt{)},
\texttt{[},
\texttt{]},
\verb+{+,
\verb+}+
\}$
die Menge aller Klammern.
\begin{teilaufgaben}
\item
Geben Sie eine kontextfreie Grammatik für die Sprache aller korrekten
Klammerausdrücke über diesem Alphabet.
\item
Führt Ihre Grammatik auf eindeutige Parse-Trees?
\item
Können Sie ihre Grammatik so modifizieren, dass sie eineutige Parse-Trees hat?
\item
Hat Ihre Grammatik Chomsky Normalform?
\end{teilaufgaben}

\begin{loesung}
\begin{teilaufgaben}
\item
Eine mögliche Grammatik ist
\begin{align}
S&\rightarrow \varepsilon 
\\
 &\rightarrow \texttt{(} \; L \; \texttt{)}
\\
 &\rightarrow \texttt{[} \; L \; \texttt{]}
\\
 &\rightarrow \texttt{\char`\{} \; L \; \texttt{\char`\}}
\\
L&\rightarrow LS
\\
 &\rightarrow S
\end{align}
Sie besagt, dass Klammerausdrücke gebildet werden, indem man Listen
von Klammerausdrücken ($L$) einklammert.
$L$ steht für eine Verkettung von Klammerausdrücken.
\item
Da Regel $L\rightarrow LS$ stellt sicher, dass Klammerausdrücke von links
nach rechts aufgebaut werden, dass also der Parse-Tree eindeutig ist.
\item
Eine Regel der Form $S\rightarrow SS$, wie wir sie im Skript und den
Übungen getroffen haben, führt auf nicht eindeutige Parse-Trees.
Die vorgeschlagene Lösung vermeidet diese Bildung.
\item
Die Regeln 
(2), (3) und (4) sind in der CNF verboten, weil nicht mehr als zwei
Zeichen auf der rechten Seite einer Regel stehen dürfen.
Regel (6) ist eine Unit-Rule, die ebenfalls verboten ist.
\qedhere
\end{teilaufgaben}
\end{loesung}




