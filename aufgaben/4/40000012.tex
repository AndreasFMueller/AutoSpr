LDAP-Filter werden verwendet, um aus einem LDAP-Verzeichnis eine Menge
von Knoten auf Grund von Werten einzelner Attribute auszuwählen.
Elementare LDAP-Suchfilter haben die Form
\begin{verbatim}
attribut=wert
\end{verbatim}
sie selektieren genau die Knoten, deren Attribut {\tt attribut}
den Wert {\tt wert} hat.
Für die Zwecke dieser Aufgabe sind Attribute und Werte nichtleere
Strings aus Buchstaben und Ziffern. 
Daraus lassen sich
mit Hilfe von Klammern und logischen Verknüpfungen
beliebige LDAP-Filter aufbauen.
Die logischen Verknüpfungen verwenden eine Präfix-Notation: 
\begin{center}
\begin{tabular}{rl}
UND:&\tt (\&(attr1=wert1)(attr2=wert2))\\
ODER:&\tt (|(attr1=wert1)(attr2=wert2))\\
NICHT:&\tt (!(attr=wert))
\end{tabular}
\end{center}
Bei der UND- und der ODER-Verknüpfung dürfen beliebig viele 
Filter miteinander verknüpft werden.
\begin{teilaufgaben}
\item
Gibt es einen regulären Ausdruck, der LDAP-Filter akzeptiert?
\item
Bilden die syntaktisch korrekten LDAP-Filter eine kontextfreie Sprache?
\end{teilaufgaben}

\thema{Grammatik}
\thema{kontextfrei}

\begin{loesung}
\begin{teilaufgaben}
\item
Dazu müsste die Sprache der LDAP-Filter regulär sein.
Da LDAP-Filter beliebig tief geschachtelte Klammern verwenden dürfen,
kann man mit dem Pumping-Lemma wie in Aufgabe \ref{40000013} zeigen,
dass die Sprache nicht regulär ist.
\item
Die Sprache der LDAP-Filter ist kontextfrei, denn man kann dafür
eine Grammatik aufstellen.
\begin{align*}
\text{filter}
        &\rightarrow\text{elementaryfilter}\\
        &\rightarrow\text{pfilter}\\
\text{elementaryfilter}
        &\rightarrow \text{string} \;\text{'{\tt =}'}\; \text{string}\\
\text{pfilter}
        &\rightarrow \text{'{\tt (}'}\;\text{elementaryfilter}\;\text{'{\tt )}'} \\
        &\rightarrow \text{'{\tt (}'}\;\text{pfilter}\;\text{'{\tt )}'} \\
        &\rightarrow \text{'{\tt (}'}\; \text{'{\tt \&}'}\;
                \text{pfiltersequence}\;\text{'{\tt )}'} \\
        &\rightarrow \text{'{\tt (}'}\; \text{'{\tt |}'}\;
                \text{pfiltersequence}\;\text{'{\tt )}'} \\
        &\rightarrow \text{'{\tt (}'}\; \text{'{\tt !}'}\;
                \text{pfilter}\;\text{'{\tt )}'} \\
\text{pfiltersequence}
        &\rightarrow \text{pfilter}\\
        &\rightarrow \text{pfiltersequence}\;\text{pfilter}\\
\text{string}
        &\rightarrow \text{char}\\
        &\rightarrow \text{string} \; \text{char}\\
\text{char}
        &\rightarrow
\text{'{\tt a}'}\;|\;
\text{'{\tt b}'}\;|\;
\text{'{\tt c}'}\;|\;\dots\;|\;
\text{'{\tt z}'}\;|\;
\text{'{\tt 0}'}\;|\;\dots\;|\;
\text{'{\tt 9}'}
\qedhere
\end{align*}
\end{teilaufgaben}
\end{loesung}

