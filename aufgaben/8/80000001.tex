In der Sprache LOOP kann jede Variable direkt adressiert werden, und
in der gleichen Instruktion k"onnen auch noch arithmetische Operationen
ausgef"uhrt werden. `Reinrassige' RISC-CPUs arbeiten anders.
Arithmetische Operationen sind nur zwischen Registern m"oglich,
und Laden eines Registers mit einem Wert aus dem Hauptspeicher ist
immer getrennt von Rechenoperationen (Load-/Store-Architektur).

Betrachten Sie folgende RISC-LOOP genannte Modifikation von LOOP.
Statt direkt adressierbarer Variablen
$x_0,x_1,x_2,\dots$ gibt es einen Index $i$, der auf jede der Variablen zeigen
kann. Der Zeiger ist zu Beginn des Programms auf $0$ initialisiert, und kann
inkrementiert oder dekrementiert werden, dazu gibt
es zwei neue Befehle {\tt INCR} und {\tt DECR}, die immer auf $i$
wirken.
Der Index $i$ kann jedoch nicht direkt auf einen bestimmten Wert gesetzt werden.

Mit $x_i$
kann jeweils auf diejenige Variable zugegriffen werden, auf die $i$ zeigt.
$x_i$ kann aber nicht direkt in einem arithmetischen Ausdruck verwendet werden,
vielmehr muss ihr Wert zuerst in das Register $r$ geladen werden, welches
als einziges f"ur arithmetische Operationen zur Verf"ugung steht. Das folgende
Programm berechnet, was in LOOP die Anweisung $x_3 := x_2 + 4$
geschafft hat:
\begin{algorithmic}[1]
\STATE {\tt INCR}
\STATE {\tt INCR}
\STATE $r$ := $x_i$
\STATE $r$ := $r + 4$
\STATE {\tt INCR}
\STATE $x_i$ := $r$
\end{algorithmic}
Auch die LOOP Anweisung kann nicht mehr jede beliebige Variable als
Argument verwenden, sondern nur den aktuellen Wert von $r$.
Zeigen Sie, dass RISC-LOOP trotzdem nicht weniger leistungsf"ahig als
LOOP ist.

\begin{loesung}
Der Zugriff auf Variablen kommt nur in Anweisungen der Form
\[
x_j:=x_i\pm c
\]
und in der LOOP Anweisung
\[
\text{\tt LOOP $x_i$ DO \dots END}
\]
vor. Um zu zeigen, dass RISC-LOOP nicht weniger leistungsf"ahig ist, m"ussen
wir nur zeigen, dass wir jedes LOOP-Programm in ein RISC-LOOP Programm
umformen k"onnen. Dazu m"ussen wir offenbar st"andig den Index $i$ modifizieren,
um auf die richtige Variable zugreifen zu k"onnen. Wir organisieren das
so, dass wir daf"ur sorgen, dass nach einem Zugriff $i$ immer wieder auf
$0$ zur"uckgesetzt wird. Die Zuweisung $x_3:=x_2+4$ ersetzen wir daher durch das Programm
\begin{center}
\begin{tabular}{r<{:}l}
1&{\tt INCR}\\
2&{\tt INCR}\\
3&$r$ := $x_i$\\
4&{\tt DECR}\\
5&{\tt DECR}\\
6&$r$ := $r + 4$\\
7&{\tt INCR}\\
8&{\tt INCR}\\
9&{\tt INCR}\\
10&$x_i$ := $r$\\
11&{\tt DECR}\\
12&{\tt DECR}\\
13&{\tt DECR}
\end{tabular}
\end{center}
Allgemein "ubersetzen wir $x_j:=x_i\pm c$ durch
\begin{itemize}
\item $i$ {\tt INCR} Befehle
\item Zugriff $r:=x_i$
\item $i$ {\tt DECR} Befehle
\item Rechnung $r:=r\pm c$
\item $j$ {\tt INCR} Befehle

\item Speicherung $x_j:=r$
\item $j$ {\tt DECR} Befehle
\end{itemize}
Analog "ubersetzen wir $\text{\tt LOOP} x_i$ durch
\begin{itemize}
\item $i$ {\tt INCR} Befehle
\item Zugriff $r:=x_i$
\item $i$ {\tt DECR} Befehle
\item Schleifenbefehl $\text{\tt LOOP }r$.
\end{itemize}
Auf diese Weise l"asst sich jedes LOOP-Programm in eine RISC-LOOP Programm
"ubersetzen, RISC-LOOP ist also mindestens so leistungsf"ahig wie LOOP.
\end{loesung}
