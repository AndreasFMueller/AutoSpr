In der Vorlesung wurde gezeigt, wie in der Sprache \texttt{LOOP}
verschiedene einfache Operationen realisiert werden können.
Zum Beispiel wurde eine Subtraktion $x_a:=x_b-x_c$ realisiert,
welche allerdings negative Resultate immer in $0$ umwandelt.
Lösen Sie im gleichen Sinne die folgenden Aufgaben:
\begin{teilaufgaben}
\item
Schreiben Sie ein \texttt{LOOP}-Programm, welches die Werte der
Variablen $x_i$ und $x_k$ vergleicht und ein Programm $P$ ausführt,
wenn $x_i \le x_k$ ist.
\item
Schreiben Sie ein \texttt{LOOP}-Programm, welches den Rest der Division
der Zahlenwerte in den Registern $x_i$ und $x_k$ berechnet.
\end{teilaufgaben}

\begin{loesung}
\begin{teilaufgaben}
\item
Das folgende Programm löst das Problem.
\begin{center}
\begin{tabular}{>{\bgroup\color{gray}\tiny}r<{:\egroup}>{$}l<{$}l}
 1&x_0 := x_i - x_k  &// $x_0=0$ für $x_i\le x_k$\\
 2&x_1 := 1          &\\
 3&\texttt{LOOP }x_0 &// für $x_i > x_k$ wird $x_1 := 0$ gesetzt\\
 4&\quad x_1 := 0    &\\
 5&\texttt{END}      &\\
 6&\texttt{LOOP }x_1 &// wird ausgeführt für $x_i\le x_k$\\
 7&\quad P           &\\
 8&\texttt{END}      &\\
\end{tabular}
\end{center}
\item
Der Algorithmus arbeitet mit einer Kopie des Wertes von $x_i$ in $x_2$.
Er subtrahiert den Wert von $x_k$ so lange von $x_2$ wie $x_2 \ge x_k$ 
ist, dazu wird das Programm von Teilaufgabe a) verwendet.
Da in \texttt{LOOP} die Anzahl Iterationen zu Beginn festgelegt werden
muss, wird dafür $x_i$ verwendet, da der Quotient nicht grösser als $x_i$
sein kann.
\begin{center}
\begin{tabular}{>{\bgroup\color{gray}\tiny}r<{:\egroup}>{$}l<{$}l}
 1&x_2 := x_i                    &\\
 2&\texttt{LOOP } x_i            &\\
 3&\quad\texttt{IF } x_k \le x_2 &// Bedingtes \texttt{IF} von Teilaufgabe a)\\
 4&\quad\quad x_2 := x_2 - x_k   &\\
 6&\quad\texttt{END}             &\\
 7&\texttt{END}                  &\\
\end{tabular}
\end{center}
Der gesuchte Rest steht in der Variablen $x_2$.
\end{teilaufgaben}
\end{loesung}
