%
% notation.tex
%
% (c) 2019 Prof Dr Andreas Müller, Hochschule Rapperswil
%
\rhead{Notation}
\section{Notation}
In diesem Abschnitt stellen wir eine Reihe bereits bekannter Notationen
zusammen.
\subsection{Logik}
\index{Logik}%
\index{Prädikatenlogik}%
\subsubsection{Prädikate}
Wir verwenden die Prädikatenlogik.
\index{Prädikat}%
{\em Prädikate} sind formale Aussagen über mathematische Objekte, die auch
variable Teile enthalten können. Prädikate sind also ``Funktionen'' mit
Wahrheitswerten als Rückgabewerten.
\begin{align*}
P(x,y)&=x < y&&\text{wahr falls $x$ kleiner als $y$}\\
P(n)&=n \equiv 0\mod 2&&\text{wahr falls $n$ gerade}
\end{align*}
Prädikate können durch logische Verknüpfungen zu neuen Aussagen kombiniert
werden:
\begin{center}
\begin{tabular}{|l|c|c|l|}
\hline
Name&Verknüpfung&Zeichen&Bedeutung\\
\hline
\index{Konjunktion}%
Konjunktion&UND&$P\wedge Q$&wahr falls sowohl $P$ als auch $Q$ wahr sind\\
\index{Disjunktion}%
Disjunktion&ODER&$P\vee Q$&wahr falls $P$ oder $Q$ (oder beide) wahr sind\\
\index{Negation}%
Negation&NICHT&$\neg P$&wahr falls $P$ nicht wahr ist\\
\hline
\end{tabular}
\end{center}
Die Kombination $\neg P\vee Q$ ist also wahr, wenn $P$ nicht
wahr ist, oder falls $P$ wahr ist, auf jeden Fall auch $Q$
wahr ist. Die Aussage, dass $\neg P\vee Q$ wahr sei, heisst also
nichts anderes, als dass $Q$ folgt, wenn $P$ wahr ist. Daher schreibt
man
\[
\neg P\vee Q\qquad =\qquad P\Rightarrow Q.
\]

Hat man mit einer grossen Zahl von Prädikaten $P_1,P_2,\dots$ zu tun,
dann kann man deren Konjunktion oder Disjunktion ähnlich wie beim Summenzeichen
mit einem ``grossen'' Verknüpfungszeichen schreiben:
\begin{align*}
\bigwedge_{i=1}^n P_i&=P_1\wedge P_2\wedge P_3\wedge\dots\wedge P_n\\
\bigvee_{i=1}^n P_i&=P_1\vee P_2\vee P_3\vee\dots\vee P_n
\end{align*}
Die Indizes müssen nicht eine fortlaufende Folge bilden, sie können
auch aus einer beliebigen Indexmenge $I$ stammen, wofür man dann
schreibt
\[
\bigwedge_{i\in I}P_i
\qquad
\text{bzw.}
\qquad
\bigvee_{i\in I}P_i
\]

Die beiden Operationen $\wedge$ und $\vee$ sind miteinander verträglich,
es gelten die {\em Distributiv-Gesetze}
\begin{align*}
P\wedge(Q\vee R)&=(P\wedge Q)\vee (P\wedge R)\\
P\vee(Q\wedge R)&=(P\vee Q)\wedge (P\vee R).
\end{align*}
\index{Distributivgesetz}%

\subsubsection{Normalformen}
\index{Normalform}%
\index{Normalform!konjunktive}%
Durch wiederholte Anwendung der Distributivgesetze kann man 
eine Formel in eine von zwei Normalformen bringen, je nachdem
ob man die ``$\wedge$'' oder die ``$\vee$'' nach aussen bringt.
Sind die ``äusseren'' Verknüpfungen die ``$\wedge$'', spricht
man von {\em konjunktiver Normalform}, eine solche Formel sieht so aus:
\[
(x_1\vee x_3)\wedge(x_2\vee \bar x_4\vee x_5)\wedge\dots
\]
Die Klammerausdrücke heissen {\em Klauseln}.
\index{Klausel}%

Bringt man stattdessen die ``$\vee$'' nach aussen, spricht man von
{\em disjunktiver Normalform}, also eine Formel der Form
\index{Normalform!diskjunktive}%
\[
(x_1\wedge x_3)\vee(x_2\wedge \bar x_4\wedge x_5)\vee\dots
\]

Die Umwandlung zwischen den Normalformen kann sehr verschwenderisch sein.
Die Formel
\[
(x_1\wedge y_1)\vee(x_2\wedge y_2)\vee\dots\vee (x_n\wedge y_n)
\]
in disjunktiver Normalform wird bei dieser Art von Umwandlung zu
\begin{equation}
(x_1\vee x_2\vee\dots\vee x_{n-1}\vee x_n)
\wedge
(y_1\vee y_2\vee\dots\vee y_{n-1}\vee y_n)
\wedge
\dots 
=\bigwedge (z_1\vee z_2\vee\dots \vee z_{n-1}\vee z_n),
\label{bigcnf}
\end{equation}
wobei $(z_1,\dots,z_n)$ eine beliebige Kombination $z_i=x_i$ oder $z_i=y_i$
ist. Die Formel (\ref{bigcnf}) hat also $2^n$ Klauseln.

\subsubsection{de Morgansche Regeln}
\index{de Morgansche Regeln}%
Die de Morganschen Regeln erlauben, ODER in UND zu verwandeln, indem
man die einzelnen Terme negiert:
\begin{align}
P\wedge Q&= \neg(\neg P\vee\neg Q)\notag\\
P\vee Q&= \neg(\neg P\wedge\neg Q)\notag\\
P\Rightarrow Q&=\neg P\vee Q=\neg\neg(\neg P\vee Q)=\neg(P\wedge\neg Q)=\neg\neg Q\vee \neg P\notag\\
&=\neg Q\Rightarrow \neg P\label{kontraposition}
\end{align}
\index{Kontraposition}%
Die letzte Formel heisst auch die Kontraposition, sie ist der Kern
eines Beweises mit Widerspruch (siehe Abschnitt \ref{widerspruchsbeweis})


\subsubsection{Quantoren}
\index{Quantor}%
Prädikate mit mindestens einer freien Variablen können mit Hilfe
von Quantoren zu neuen Präkdikaten zusammengebaut werden. 
Die Aussage $n^2\ge 0$ ist zum Beispiel für alle natürlichen
Zahlen richtig, man schreibt dafür
\[
\forall n\in\mathbb N(n^2\ge 0).
\]
\index{All-Quantor}%
Das Zeichen $\forall$ wird ``für alle'' gelesen, es heisst
{\em All-Quantor}.
\index{Existenz-Quantor}%
Jede positive reelle Zahl $x$ hat eine Wurzel, es gibt also eine
Zahl $w$ mit der Eigenschaft $x=w^2$. Formal drückt man dies mit
dem {\em Existenz-Quantor} $\exists$ aus:
\[
\exists w\in\mathbb R(x =w^2)
\]
Dies liest man ``es gibt eine reelle Zahl $w$, deren Quadrat $x$ ist.
Will man obigen Satz, dass jede positive relle Zahl eine Wurzel hat,
formal ausdrücken, schreibt man
\[
\forall x>0\exists w\in\mathbb R(x=w^2)
\]

Zwei Aussagen $P$ und $Q$ sind {\em äquivalent}, wenn sowohl
$P\Rightarrow Q$ als auch $Q\Rightarrow P$, kurz $P\Leftrightarrow Q$.

\subsection{Mengenlehre}
\index{Mengenlehre}%
Wir gehen von einem intuitiven Mengenverständnis aus, und bezeichnen
Mengen meistens mit grossen Buchstaben $A$, $B$, $C$ usw.
Aus einer Menge $A$ kann eine neue Menge gebildet werden, indem
nur die Elemente ausgewählt werden, die eine bestimmte Eigenschaft
$E$ haben:
\[
\{a\in A\;|\;E(a)\}
\]
Die Menge der $Q$ der Quadratzahlen besteht zum Beispiel aus den
Zahlen $n$, für
die es eine andere Zahl $w$ gibt (die Wurzel), mit der Eigenschaft
$n=w^2$.  In Formeln:
\[
Q=\{n\; |\; \exists m\in\mathbb N(n=m^2)\}
\]

Sind $A$ und $B$ zwei Mengen, dann können wir daraus neue Mengen
bilden:
\begin{center}
\begin{tabular}{|l|c|l|}
\hline
Operation&Zeichen&Bedeutung\\
\hline
\index{Durchschnitt}%
Durchschnitt&$A\cap B$&Enthält die Elemente, die in $A$ {\bf und} $B$ sind\\
\index{Vereinigung}%
Vereinigung&$A\cup B$&Enthält die Elemente, die in $A$ {\bf oder} $B$ sind\\
\index{Komplement}%
Komplement&$\bar A$&Enthält die Elemente, die {\bf nicht} in $A$ sind\\
\index{Differenz}%
Differenz&$A\setminus B$&Enthält die Elemente aus A, die nicht in $B$ sind\\
\hline
\end{tabular}
\end{center}
\index{symmetrische Differenz}%
Für die Anwendungen wichtig ist ausserdem die {\em symmetrische Differenz}
zweier Mengen: 
\[
A{\;\Delta\;}B = (A\setminus B)\cup (B\setminus A).
\]
Die symmetrische Differenz verschwindet genau dann, wenn die beiden
Mengen gleich sind:
\[
A{\;\Delta\;}B = \emptyset
\quad\Leftrightarrow\quad
A=B.
\]
\begin{figure}
\begin{center}
\includegraphics[width=.4\hsize]{images/turing-4}
\end{center}
\caption{Symmetrische Differenz der Mengen $A$ und $B$ (schraffiert)
\label{symdiff}}
\end{figure}

Die folgenden Zahlmengen sind wohlbekannt und wir werden sie ebenfalls
oft verwenden:
\begin{center}
\begin{tabular}{|c|l|}
\hline
Symbol&Beschreibung\\
\hline
\index{natürliche Zahlen}%
$\mathbb N$&Menge der natürlichen Zahlen $\{0,1,2,\dots\}$\\
$\mathbb N^*$&Menge der positiven natürlichen Zahlen $\{1,2,3,\dots\}$\\
\index{ganze Zahlen}%
$\mathbb Z$&Menge der ganzen Zahlen $\{\dots,-2,-1,0,1,2,\dots\}$\\
\index{rationale Zahlen}%
$\mathbb Q$&Menge der rationalen Zahlen $\{\frac{p}{q}|p\in\mathbb Z,q\in\mathbb N^*\}$\\
\index{reelle Zahlen}%
$\mathbb R$&Menge der reellen Zahlen\\
\index{leere Menge}%
$\emptyset=\{\}$&leere Menge\\
\hline
\end{tabular}
\end{center}
Daraus lassen sich weitere Mengen konstruieren:
\begin{align*}
\mathbb Z^*&=\mathbb Z\setminus \{0\}\\
\mathbb Q^*&=\mathbb Q\setminus \{0\}\\
\mathbb R^*&=\mathbb R\setminus \{0\}\\
\mathbb R_{> 0}&=\mathbb R^+=\{x\in\mathbb R\,|\,x>0\}\\
[n]&=\{x\in \mathbb N^*\,|\,x\le n\}
\end{align*}
\index{Machtigkeit@Mächtigkeit}%
Enthält eine Menge $A$ nur endlich viele Elemente, schreiben wir die Anzahl
ihrer Elemente $|A|$, sie heisst auch die {\em Mächtigkeit} von $A$.
\[
|\{n\in\mathbb N\,|\,\text{$n$ prim}\wedge n<10\}|=|\{2,3,5,7\}|=4.
\]
Die Mengen $[n]$ enthalten genau $n$ Elemente, $|[n]|=n$.

Enthält eine Menge $B$ alle Elemente von $A$, sagt man, $A$ sei in $B$
enthalten, schreibt dafür $A\subset B$ und sagt, $A$ sei eine
{\em Teilmenge} von $B$. Ist $A\subset B$ und auch
$B\subset A$, dann enthalten $A$ und $B$ die gleichen Elemente, also
$A=B$.

\index{Potenzmenge}%
Die Menge aller Teilmengen von $A$ heisst die {\em Potenzmenge} $P(A)$ von $A$:
\[
P(A)=\{ T\,|\,T\subset A\}
\]
Die Potenzmenge einer endlichen Anzahl Elemente $A$ kann man wie folgt bilden.
Um eine Teilmenge von $A$ zu bilden, muss man für jedes Element
von $A$ entscheiden, ob es in die Teilmenge kommen soll oder nicht.
Man hat also für jedes der $|A|$ Elemente eine Entscheidung mit zwei
möglichen Ausgängen zu fällen, was auf
\[
\underbrace{2\cdot\dots\cdot 2}_{\text{$|A|$ Faktoren}}=2^{|A|}
\]
möglich ist, also $|P(A)|=2^{|A|}$.

\subsection{Paare und Tupel}
\index{Paar}%
\index{Tupel@$n$-Tupel}%
\index{kartesisches Produkt}%
Die Beschreibung eines Punktes in der Ebene erfordert die Angabe zweier
Koordinaten, die man üblicherweise als Paar $(x,y)$ schreibt. Allgemein
kann man aus den Elementen $a\in A$ und $b\in B$ zweier Mengen $A$ und $B$
die Menge der Paare (2-Tupel) bilden, diese Menge heisst das
{\em kartesische Produkt} der beiden Mengen:
\[
A\times B = \{(a,b)\,|\,a\in A\wedge b\in B\}.
\]
\index{Tripel}%
Analog kann man aus drei Mengen die Menge aller Tripel
\[
A\times B\times C=\{(a,b,c)\,|\,a\in A\wedge b\in B\wedge c\in C\}
\]
bilden, oder für $n$ Mengen die Menge der $n$-Tupel
\[
A_1\times A_2\times\dots\times A_n
=\{(a_1,a_2,\dots,a_n)\,|a_1\in A_1\wedge a_2\in A_2\wedge\dots\wedge a_n\in A_n\}.
\]
Falls die Faktormengen alle identisch sind, also $A_1=A_2=\dots=A_n=A$
schreiben wir auch
\[
A^n= \underbrace{A\times \dots \times A}_{\text{$n$ Faktoren}}
\]
für das kartesische Produkt von $n$ Faktoren $A$.

\subsection{Relationen}
\index{Relation}%
Eine zweistellige {\em Relation} $R$ ist ein Spezialfall eines Prädikates.
Realisieren könnte man eine Relation dadurch, dass man die Menge
aller Paare bildet, für die die Relation erfüllt ist. Um zu entscheiden,
ob $xRy$ gilt, muss man also nur noch in der Menge nachschauen,
ob $(x,y)$ dort drin ist. Man kann also die Relation mit der
Menge
\[
\{(x,y)\in A\times B\,| xRy\}
\]
identifizieren.
\index{Graph!einer Relation}%
Diese Menge heisst auch der {\em Graph} der Relation.

Besonders wichtig sind {\em Äquivalenzrelationen}, die sich durch folgende
Eigenschaften auszeichnen.
\begin{compactenum}
\item $R$ ist {\em reflexiv}, falls für jedes $x$ gilt $xRx$: $\forall x(xRx)$
\item $R$ heisst {\em symmetrisch}, falls mit $xRy$ auch $yRx$ gilt: $\forall x\forall y(xRy\Rightarrow yRx)$
\item $R$ heisst {\em transitiv}, falls mit $xRy$ und $yRz$ auch $xRz$ gilt.
\end{compactenum}
Beispiele von Äquivalenzrelationen sind:
\begin{itemize}
\item Gleichheit
\item $xRy$ für $x,y\in\mathbb N$ falls $x$ und $y$ den gleichen
Rest bei Teilung durch eine feste Primzahl $p$ haben.
\item $xRy$ für Dreiecke $x$ und $y$, falls $x$ und $y$ kongruent sind.
\end{itemize}

\subsection{Funktionen, Abbildungen}
\index{Funktion}%
\index{Abbildung}%
Eine {\em Funktion} oder {\em Abbildung} $f\colon A\to B$ ist
eine Zuordnung, die
jedem Element von $A$ genau ein Element von $B$ zuordnet.
\index{Bild}%
\index{Urbild}%
Das Element
$b=f(a)$ heisst das {\em Bild} von $a$, $a$ heisst ein {\em Urbild} von $b$. Ein
Element in $b$ kann mehrere Urbilder haben.

Eine Funktion ist eine spezielle Relation: 
$a$  und $b$ stehen in der Relation zueinander, wenn $f(a)=b$. 
\index{Graph!einer Abbildung}%
Die zugehörige Menge von Paaren ist
\[
G(f)=\{(a,b)\,|\,b=f(a)\},
\]
und heisst der {\em Graph}. Ist $A=\mathbb R$, $B=\mathbb R$ und
$f\colon\mathbb R\to\mathbb R$, dann ist $G(f)$ die Menge
der Punkte $(x,y)$ in der Ebene, die $y=f(x)$ erfüllen, also
genau der Graph im üblichen Sinne.

\index{Menge aller Abbildungen}%
Die Menge aller Abbildungen von $A$ nach $B$ schreibt man auch
$B^A$. Die Notation wird verständlich, wenn man für endliche
Mengen $A$ und $B$ zählt, wieviel
Abbildungen zwischen $A$ und $B$ es gibt. Um eine Abbildung von
$A$ nach $B$ zu konstruieren, muss man für jedes der $|A|$ Elemente von $A$
eines der $|B|$ Elemente von $B$ auswählen. Man muss also $|A|$ mal
eine Auswahl mit $|B|$ Möglichkeiten treffen, man hat also
\[
\underbrace{|B|\cdot\dots\cdot|B|}_{\text{$|A|$ Faktoren}}=|B|^{|A|}
\]
Abbildungen.

Die Elemente der Potenzmenge von $A$ entstehen dadurch, dass man für jedes
Element von $A$ einen Wert $0$ oder $1$ auswählen muss, $0$ gibt an,
dass das Element nicht in die Teilmenge kommt, $1$ gibt an, dass es dazugehört.
Eine Teilmenge von $A$ entspricht also genau einer Abbildung $A\to\{0,1\}$,
man kann die Potenzmenge mit der Menge der Abbildungen $A\to\{0,1\}$
identifizieren:
\[
P(A) = \{0,1\}^{A}.
\]

Ein $n$-Tupel von Elementen aus $A$ ordnet jedem der Plätze
$1,\dots,n$ genau ein Element aus $A$ zu, ein Tupel ist also eigentlich
eine Abbildung $[n]=\{1,\dots,n\}\to A$, die Menge der Tupel ist somit
\[
A^n=A^{\{1,\dots,n\}}=A^{[n]}.
\]

\subsection{Graphen}
\index{Kante}%
\index{Ecke}%
\index{Vertex}%
Ein Graph besteht aus Ecken (Vertizes) und Kanten,
die die Ecken verbinden.
Dabei
soll es aber anders als zum Beispiel bei einem Verkehrsnetz zwischen zwei
Ecken immer nur eine Verbindung geben können. Die Kante ist also durch
die beiden Endpunkte vollständig bestimmt, ausserdem spielt deren
Reihenfolge keine Rolle. Um eine Kante zu beschreiben, braucht man
also nur die Menge $\{a,b\}$ der Endpunkte zu kennen.
Wir fassen das in folgende Definition zusammen.

\begin{definition}
\index{Graph}%
\label{def_graph}
Ein Graph ist ein Paar $(V,E)$ bestehend aus einer Menge $V$ von Ecken
(Vertices),
und einer Menge $E$ von zweielementigen Teilmengen von $V$, die Kanten
genannt werden.
\end{definition}

\begin{figure}
\begin{center}
\includegraphics{images/graph-1.pdf}
\end{center}
\caption{Graph\label{grundlagen:graph}}
\end{figure}
Die Abbildung\ref{grundlagen:graph} zeigt einen Graphen mit Vertex-Menge
\[
V=\{v_0,
v_1,
v_2,
v_3,
v_4,
v_5,
v_6\},
\]
und Kanten
\begin{align*}
e_0&=\{v_0,v_4\},&
e_1&=\{v_0,v_3\},&
e_2&=\{v_0,v_1\},&
e_3&=\{v_2,v_4\},
\\
e_4&=\{v_1,v_3\},&
e_5&=\{v_2,v_3\},&
e_6&=\{v_1,v_2\},&
e_7&=\{v_1,v_5\},
\\
e_8&=\{v_1,v_6\},&
e_9&=\{v_2,v_5\},&
e_{10}&=\{v_5,v_6\}.
\end{align*}
Die Menge $E$ besteht also aus den Elementen $e_1,\dots,e_{10}$.

\begin{figure}
\begin{center}
\includegraphics{images/graph-2.pdf}
\end{center}
\caption{Kein Graph, jede Kante eines Graphen hat zwei verschiedene
Ecken.\label{grundlagen:keingraph}}
\end{figure}
Man beachte, dass ein Graph keine Kante haben kann, die zurück zum 
selben Vertex führen kann.
Das Gebilde in Abbildung~\ref{grundlagen:keingraph}
ist also kein Graph, weil jede Kante eines Graphen zwei verschiedene
Ecken hat.

\index{Grad}%
Der {\em Grad} einer Ecke in einem Graphen ist die Anzahl der Kanten,
die von dieser Ecke ausgehen.

\index{Pfad}%
Ein {\em Pfad} in einem Graphen ist eine Folge von Ecken, die durch Kanten
verbunden sind.
\index{Pfad!einfacher}%
Ein {\em einfacher Pfad} ist ein Pfad, der keine Ecke mehr
als einmal trifft.
\index{Zyklus}%
Ein {\em Zyklus} ist ein Pfad, der mit der gleichen Ecke
endet, mit der er begonnen hat.
\index{Zyklus!einfacher}%
Ein einfacher Zyklus ist ein Zyklus,
der keine Ecke zweimal besucht.

\index{Baum}%
Ein {\em Baum} ist ein Graph ohne Zyklen. Ein Baum kann einen ausgezeichneten
Punkt, die {\em Wurzel} des Baumes, enthalten. Alle anderen Ecken vom Grad 1
\index{Blatt}%
heissen Blätter des Baumes.

\begin{figure}
\begin{center}
\includegraphics{images/graph-3.pdf}
\end{center}
\caption{Gerichteter Graph\label{grundlagen:gerichtetergraph}}
\end{figure}
Möchte man die Richtung der Kanten berücksichtigen,
hat man mit einem gerichteten
Graphen zu tun (Abbildung~\ref{grundlagen:gerichtetergraph}).
Da es jetzt auf die Reihenfolge der Endpunkte an kommt,
sich insbesondere die Kante von $a$ nach $b$ von der Kante von $b$ nach
$a$ unterscheidet, können wir die Kanten nicht mehr durch die ungeordneten
Mengen beschreiben, sondern müssen geordnete Tupel verwenden.

\begin{definition}
\index{Graph!gerichteter}%
\label{def_gerichteter_graph}
Ein gerichteter Graph ist ein Paar $(V,E)$ bestehend aus einer
Mengen $V$ von Ecken (Vertizes) und einer Menge von Paaren $E\subset V\times V$,
den Kanten.
\end{definition}

Die Kanten des Graphen in Abbildung~\ref{grundlagen:gerichtetergraph}
sind Paare
\begin{align*}
e_0&=(v_0,v_4)
&
e_1&=(v_0,v_3)
&
e_2&=(v_0,v_1)
&
e_3&=(v_4,v_2)
\\
e_4&=(v_1,v_3)
&
e_5&=(v_2,v_3)
&
e_6&=(v_1,v_2)
&
e_7&=(v_5,v_1)
\\
e_8&=(v_1,v_6)
&
e_9&=(v_5,v_2)
&
e_{10}&=(v_5,v_6)
\end{align*}

\begin{figure}
\begin{center}
\includegraphics{images/graph-4.pdf}
\end{center}
\caption{Gerichteter beschrifteter Graph\label{grundlagen:beschrgraph}}
\end{figure}
Ein Verkehrsnetz wie auch die später zu definierenden endlichen
Automaten ist aber noch komplizierter: auf den Verbindungen zwischen
den Netzwerken sind auch verschiedene Bahnlinien im Einsatz, so gibt
es zwischen Rapperswil und Pfäffikon zum Beispiel je eine Kante, die
mit ``S5'' bzw.~mit ``Voralpenexpress'' angeschrieben ist
(Abbildung~\ref{grundlagen:beschrgraph}). Dies
wird von folgender Definition eingefangen:

\begin{definition}
\label{def_gerichteter_beschrifteter_graph}
\index{Graph!gerichteter!beschrifteter}%
Ein gerichteter beschrifteter Graph ist ein Tripel $(V,E,L)$ bestehend
aus einer Menge $V$ von Ecken, einer Menge $L$ von Beschriftungen (Labels) und 
einer Menge $E$ von Tripeln $(a,e,l)\in V\times V\times L$. Das
Tripel $(a,e,l)$ heisst Kante von $a$ nach $e$ mit Beschriftung $l$.
\end{definition}

Der in Abbildung~\ref{grundlagen:beschrgraph} abgebildete Graph
hat die Kantenmenge 
\begin{align*}
E=\{\;
&
(v_0,v_1,\text{\tt Fussweg}),\;
(v_0,v_3,\text{\tt ICE}),\;
(v_0,v_4,\text{\tt NCC 1701-D}),\;
\\&
(v_1,v_2,\text{\tt S5}),\;
(v_1,v_3,\text{\tt S7}),\;
(v_1,v_6,\text{\tt Taxi}),\;
(v_2,v_3,\text{\tt S5}),\;
\\&
(v_3,v_0,\text{\tt ICE}),\;
(v_5,v_1,\text{\tt S5}),\;
(v_5,v_1,\text{\tt S7}),\;
(v_5,v_6,\text{\tt Bus})\;
\}
\end{align*}
Man beachte, dass es zwei verschieden beschriftete Kanten gibt
von $v_5$ nach $v_1$, und dass die Richtung der beiden gleich
beschrifteten Kanten zwischen $v_0$ und $v_3$ verschieden ist.
Beides Möglichkeiten, die in einem Graphen verboten sind.
