%
% beweis.tex -- Beweistechinken
%
% (c) 2019 Prof Dr Andreas Müller, Hochschule Rapperswil
%
\rhead{Beweistechniken}
\section{Beweistechniken}
Da in dieser Vorlesung Beweise zentral sind, hier drei grundlegende 
Techniken zur Erinnerung.

\subsection{Konstruktion}
\index{Beweis!konstruktiver}%
Ein konstruktiver Beweis gibt einfach eine Konstruktion an, die
das behauptete Objekt liefert oder für die die behauptete Eigenschaft
offensichtlich ist.
Ein solches Vorgehen wird auch oft eine Herleitung genannt.

\begin{satz}
\index{Gleichung!quadratische}%
Falls $a\ne0$ und $b^2-4ac>0$ hat die quadratische Gleichung
\[
ax^2+bx+c=0
\]
zwei verschiedene Lösungen.
\end{satz}

\begin{proof}[Beweis]
Man kann die quadratische Gleichung mit vollständigem Ergänzen wie
folgt umformen:
{
\allowdisplaybreaks
\begin{align*}
ax^2+bx+c&=0\\
x^2+2\cdot \frac{b}{2a} x +\frac{c}a&=0\\
x^2+2\cdot \frac{b}{2a} x 
+\left(\frac{b}{2a}\right)^2
+\frac{c}a&=0
+\left(\frac{b}{2a}\right)^2\\
\left(x+\frac{b}{2a}\right)^2 &= \left(\frac{b}{2a}\right)^2 -\frac{c}a \\
\left(x+\frac{b}{2a}\right)^2 &=
\frac{b^2-4ac}{4a^2}\\
x+\frac{b}{2a}&=\frac{\pm\sqrt{b^2-4ac}}{2a}\tag{*}\\
x_{\pm}&=\frac{-b\pm\sqrt{b^2-4ac}}{2a}
\end{align*}
}
Die Voraussetzung $a>0$ wird im ersten Schritt dieser Ableitung verwendet.
Da $b^2-4ac>0$ kann im Schritt (*) tatsächlich die Wurzel gezogen werden,
und die beiden Wurzeln sind auch verschieden. Man hat also zwei verschiedene
Lösungen konstruiert, womit die Behauptung bewiesen ist.
\end{proof}

\subsection{Widerspruch\label{widerspruchsbeweis}}
\index{Beweis!mit Widerspruch}%
\index{Widerspruch}%
In der Mathematik darf es keine Widersprüche geben, denn aus einem
Widerspruch liesse sich jede beliebige Aussage ableiten. Wäre zum
Beispiel $P$ eine Aussage, die sowohl wahr wie auch falsch ist,
dann hätte die Menge
\[
\{x\in\{P\}\,|\, \text{$x$ ist wahr}\}
\]
je nachdem ob $P$ nun wahr oder falsch ist $1$ oder $0$ Elemente.
Da das aber natürlich immer die gleiche Menge ist, müsste man
folgern $0=1$. Daraus lässt sich dann ableiten, dass alle natürlichen
Zahlen gleich sind (zum Beispiel mit vollständiger Induktion, siehe
unten), dass alle Mengen gleich viele Elemente haben, nämlich gar
keine, dass es die ganze Mathematik also gar nicht gibt.

Wenn man also eine Aussage $P$ beweisen will, kann man annehmen,
dass $P$ falsch ist, oder $\neg P$ wahr ist. Wenn daraus jetzt
eine Aussage $Q$ folgt, von der wir bereits wissen, dass sie falsch
ist, dann bedeutet das, dass die Annahme $\neg P$ nicht haltbar
war.

Und noch etwas formaler: wenn $\neg P\Rightarrow \neg Q$, dann können
wir das umformen zu $\neg Q\Rightarrow P$ (nach Kontrapositionsformel
(\ref{kontraposition})), wir wissen bereits, dass
$Q$ falsch ist, die linke Seite also wahr ist, also muss auch $P$
wahr sein.

Das klassische Beispiel für einen Beweis mit Widerspruch ist das 
folgende:
\begin{satz}$\sqrt{2}$ ist irrational: $\sqrt{2}\not\in\mathbb Q$.
\index{irrational}%
\end{satz}
\begin{proof}[Beweis]
Wir nehmen $\sqrt{2}\in\mathbb Q$ und führen dies zu einem
Widerspruch. Wenn $\sqrt{2}\in\mathbb Q$ ist, kann man $\sqrt{2}$
als gekürzten Bruch $\frac{p}{q}$ schreiben. Dann gilt aber
auch
\[
\sqrt{2}^2=2=\frac{p^2}{q^2}\quad\Rightarrow\quad p^2=2q^2.
\]
Wenn $p$ den Primfaktor $2$ enthält, dann enthält $p^2$
eine gerade Anzahl von Primfaktoren $2$. Ebenso enthält
$q^2$ eine gerade Anzahl von Primfaktoren $2$, also $2q^2$
eine ungerade Anzahl von Primfaktoren $2$.
Im Ausdruck $p^2=2q^2$ steht also links eine Zahl mit einer
geraden Anzahl von Primfaktoren $2$, und rechts eine Zahl mit
einer ungeraden Anzahl von Primfaktoren $2$.
Dieser Widerspruch zeigt, dass die
Annahme $\frac{p}q\in\mathbb Q$ nicht zu halten ist.
\end{proof}

\subsection{Induktion}
\index{Beweis!mit Induktion}%
\index{Induktion}%
Möchte man eine Aussage mit einem natürlichen Parameter $n$ für jeden
möglichen Wert von $n$ beweisen, kann man dazu vollständige Induktion
verwenden. Dazu geht man wie folgt vor:
\begin{description}
\index{Verankerung}%
\item[Verankerung:] Beweise die Aussage für den kleinsten Wert,
für den sie bewiesen werden soll.
\index{Induktionsannahme}%
\item[Induktionsannahme:] Nehme an, für einen bestimmten Wert
$n$ sei die Aussage bereits bewiesen (nach der Verankerung ist sie 
für den kleinsten möglichen Wert ja tatsächlich bereits bewiesen).
\index{Induktionsschritt}%
\item[Induktionsschritt:] Beweise jetzt die Aussage für den Wert $n+1$.
\end{description}

Als Beispiel soll der folgene Satz dienen:

\begin{satz} Für alle natürlichen Zahlen $n$ gilt
\[
\sum_{k=1}^nk=\frac{n(n+1)}2
\]
\end{satz}
\begin{proof}[Beweis]
\begin{description}
\item[Verankerung:] für $n=0$ ist die Summe leer, ergibt also $0$.
Auf der rechten Seite steht $0(0+1)/2=0$, die Formel trifft also zu.
\item[Induktionsannahme:]Wir nehmen also an, dass
\[
\sum_{k=1}^nk=\frac{n(n+1)}2
\]
\item[Induktionsschritt:] Wir müssen die Behauptung für $n+1$ zeigen. Dazu
berechnen wir
\begin{align}
\sum_{k=1}^{n+1}k&=\biggl(\sum_{k=1}^nk\biggr) + (n+1)\notag\\
&=\frac{n(n+1)}2+(n+1)\label{verwendung_annahme}\\
&=\frac{n(n+1)+2(n+1)}2\notag\\
&=\frac{(n+1)(n+2)}2=\frac{(n+1)((n+1)+1)}2\notag
\end{align}
In Schritt (\ref{verwendung_annahme}) haben wir die Induktionsannahme
verwendet. In der letzten Zeile auf der rechten Seite steht tatsächlich
die Formel des Satzes, in der $n$ durch $n+1$ ersetzt worden ist.
Damit ist die Formel für $n+1$ bewiesen.
\end{description}
\end{proof}
