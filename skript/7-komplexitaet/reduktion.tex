%
% reduktion.tex -- Polynomielle Reduktion
%
% (c) 2011 Prof Dr Andreas Mueller, Hochschule Rapperswil
%
\section{Reduktion}
\rhead{Reduktion}
Die Beispiele des vorangegangenen Abschnittes haben gezeigt,
dass der Nachweis, dass eine Sprache in P ist, mit grossem
Aufwand verbunden sein kann, weil ein geeigneter Algorithmus
gefunden werden muss. Ein ähnliches Problem konnte in der
Entscheidungstheorie durch die Verwendung einer Reduktionsabbildung
$f\colon A\le B$
gelöst werden. Eine solche musste berechenbar sein, und erlaubte
die Entscheidung, ob ein Wort in $A$ ist, in die Sprache
$B$ zu transportieren, und dort zu entscheiden.

Im Zusammenhang der Komplexität möchte man das ebenfalls
tun, doch muss die Reduktionsabbildung jetzt auch noch 
polynomielles Laufzeitverhalten haben, damit sie nützlich
bleibt.

\begin{definition}
\index{Reduktion}%
Eine berechenbare Abbildung $f\colon \Sigma^*\to\Sigma^*$
mit den Eigenschaften
\begin{compactenum}
\item $w\in A\quad\Leftrightarrow\quad f(w)\in B$
\item Es gibt eine Turingmaschine mit polynomieller Laufzeit, die
$f(w)$ berechnet.
\end{compactenum}
heisst eine polynomielle Reduktion $A\le_P B$.
\end{definition}

\begin{satz}
\label{polynomiellreduction}
Ist $A\le_P B$ und $B$ in P, dann ist auch $A$ in P.
\end{satz}

\begin{proof}[Beweis]
Da $B\in\text{P}$ gibt es eine Turingmaschine $M$ mit polynomieller Laufzeit,
die $w\in B$ entscheiden kann. Der folgende Algorithmus entscheidet
jetzt auch $w\in A$:
\begin{compactenum}
\item Berechne $f(w)$
\item Wende $M$ auf $f(w)$ an.
\end{compactenum}
Wir müssen uns nur noch versichern, dass dieser Algorithmus 
polynomielle Laufzeit hat. Die Laufzeit der Turingmaschine ist
$t_M(|f(w)|)$. Aber $|f(w)|$ kann nicht länger als die Laufzeit 
der Berechnung von $f(w)$ sein, man kann ja in jedem Schritt höchstens
ein Zeichen schreiben. Schreiben wir $t_f(n)$ für die Laufzeit
der Berechnung von $f$, ist die Gesamtlaufzeit des Algorithmus
$t_M(t_f(|w|))$. Sowohl $t_M$ als auch $t_f$ sind Polynome, 
also ist auch die Zusammensetzung ein Polynom, der Algorithmus
hat somit polynomielle Laufzeit.
\end{proof}

\begin{satz}
Ist $A\le_P B$ und hat $B$ einen polynomiellen Verifizierer, dann
hat auch $A$ einen polynomiellen Verifizierer. Falls $A\le_P B$ und 
$B\in\text{NP}$, dann ist auch $A\in\text{NP}$.
\end{satz}

\begin{proof}[Beweis]
Aus einem polynomiellen Verifizierer $V$ für $B$ kann man durch
Zusammensetzen mit der Reduktionsabbildung $f\colon A\le_P B$
einen polynomiellen Verifizierer für $A$ konstruieren.
\end{proof}

Die letzten zwei Sätze zeigen, dass sich Sprachen in $P$ und $NP$
mit Hilfe der polynomiellen Reduktion in ``leichte'' und ``schwierige'',
in ``schnell'' bzw.~``langsam'' zu lösende unterteilen lassen
(Abbildung \ref{pnporder}).
\begin{figure}
\begin{center}
\includegraphics{images/lang-6}
\end{center}
\caption{Sprachen werden durch die polynomielle Reduktion $\le_P$
nach ``Schwierigkeit'' geordnet.\label{pnporder}}
\end{figure}%

\begin{beispiel}[\bf Stundenplan und das Färbeproblem]
\index{Stundenplanproblem}%
Das Studenplanproblem $S$ besteht darin
einen Stundenplan so zu erstellen, dass die Studenten alle
Vorlesungen besuchen können, für die sie sich angemeldet haben.
Die Lektionen müssen dabei auf Zeitfenster verteilt werden,
so dass es keine Kollisionen gibt.

Um dieses Problem zu lösen, muss man es auf ein bekanntes
Problem reduzieren. In diesem Fall bietet sich das Eckenfärbeproblem
\textsl{VERTEX-COLORING}
für Graphen an. Es wird verlangt, die Ecken eines Graphen mit 
verschiedenen Farben so einzufärben, dass keine zwei benachbarten
Ecken die gleiche Farbe haben.

Den verfügbaren Zeitfenstern für Lektionen müssen die Fächer zugeordnet werden
so, dass kein Student zwei Fächer im gleich Zeitfenster
besuchen will.
Die Lektionen bilden als Ecken eines Graphen, zwischen zwei
Ecken gibt es eine Kante, wenn ein Student sich für beide 
Lektionen angemeldet hat. Den Lektionen müssen jetzt Zeitfenster
zugeordnet werden, so dass es keine Kollisionen gibt, die
Zeitfenster sind also die Farben, mit denen die Ecken  des
Graphen eingefärbt werden sollen. Die Einfärbung ist genau
dann möglich, wenn es eine Lösung für das Stundenplanproblem gibt.
Damit haben wir eine Abbildung konstruiert
\[
\begin{tabular}{>{$}r<{$}>{$}c<{$}>{$}l<{$}}
S&\to&\text{$k$-\textsl{VERTEX-COLORING}}\\
\text{Fach}&\mapsto&\text{Vertex}\\
\text{Zeitfenster}&\mapsto&\text{Farbe}\\
\text{Anzahl Zeitfenster}&\mapsto&k\\
\text{Student/Anmeldung}&\mapsto&\text{Kante}
\end{tabular}
\]
Diese Abbildung ist offenbar eine polynomielle Reduktion
\[
S\le_P \text{$k$-\textsl{VERTEX-COLORING}}.
\]
Die Reduktion kann aber auch in umgekehrter Richtung erfolgen,
es gibt also auch eine Reduktion
\[
\text{$k$-\textsl{VERTEX-COLORING}}\le_P S.
\]
\end{beispiel}

\begin{beispiel}[\bf\textsl{3SAT} und \textsl{CLIQUE}]
Eine $k$-Clique in einem Graphen ist ein vollständiger
Untergraph mit $k$ Ecken in einem gegebenen Graphen. Im Cliquen-Problem müssen
in einem gegebenen Graphen $k$ Ecken gefunden werden, so dass
jede mögliche Verbindung zwischen den Ecken auch im Graphen $G$ 
besteht. Als Sprache formuliert, ist es
\[
\text{\textsl{CLIQUE}}
=\{
\langle G,k\rangle\;|\;\text{$G$ ist ein Graph mit einer $k$-Clique}
\}
\]
Wir konstruieren jetzt eine Reduktion 
\[
\text{\textsl{3SAT}}\le_P
\text{\textsl{CLIQUE}}
\]
Man muss also aus jeder Formel in konjunktiver Normalform mit
drei Klauseln einen Graphen konstruieren, der genau dann eine
$k$-Clique besitzt, wenn die Formel erfüllbar ist. Ausserdem
muss die Konstruktion in polynomieller Zeit durchführbar sein.

Sei also die Formel von der Form
\[
\varphi
=
\varphi_1\wedge\varphi_2\wedge\dots\wedge\varphi_k,
\]
wobei jede Teilformel $\varphi_i$ nur eine Disjunktion (Oder-Verknüpfung)
von Variablen oder negierten Variablen ist. Damit $\varphi$
erfüllbar ist, muss es eine Zuordnung von Wahrheitswerten zu
den Variablen geben, so dass jede der Teilformeln $\varphi_i$ wahr
wird. In jeder Teilformel gibt es also eine Variable oder 
negierte Variable, die wahr ist. Es würde also genügen,
die Terme in den $\varphi_i$ zu finden, die alle gleichzeitig
wahr sein können.

Wir konstruieren jetzt den Graphen wie folgt. Die Ecken des Graphen
sind die Variablen der Teilformeln $\varphi_i$. Eine Kante wird
eingezeichnet für jedes Paar von Variablen in verschiedenen
Teilformeln, die gleichzeitig wahr sein können. Die Formel $\varphi$
ist genau dann erfüllbar, wenn der Graph eine $k$-Clique enthält.

\begin{figure}
\[
\entrymodifiers={++[o][F]}
\xymatrix{
*+\txt{}
	&{\bar x_1}
		\ar@{-}[dl] \ar@{-}[drrr] \ar@{-}[ddl] \ar@{-}[dddl]
		\ar@{-}[dddd] \ar@{-}[ddddr] 
		%\ar@{-}[drrr]
		\ar@{-}[ddrrr] \ar@{-}[dddrrr]
		&{\bar x_2}
			\ar@{-}[dddll]
			\ar@{-}[ddddl] \ar@{-}[dddd] \ar@{-}[ddddr]
			\ar@{-}[drr] \ar@{-}[dddrr]
			\ar@{-}[dll] \ar@{-}[ddll]
			&{x_2}
				\ar@{-}[dddlll] \ar@{-}[dddlll]
				\ar@{-}[dddd]
				\ar@{-}[dr] \ar@{-}[ddr]
\\
{\bar x_2}
	\ar@{-}[rrrr] \ar@{-}[rrrrdd]
	\ar@{-}[dddr] \ar@{-}[dddrr] \ar@{-}[dddrrr]
	&*+\txt{}
		&*+\txt{}
			&*+\txt{}
				&{\bar x_1}
					\ar@{-}[dllll] \ar@{-}[ddllll]
					\ar@{-}[dddlll] \ar@{-}[dddll]
\\
{\bar x_2}
	\ar@{-}[rrrrd]
	\ar@{-}[ddr] \ar@{-}[ddrr] \ar@{-}[ddrrr]
	&*+\txt{}
		&*+\txt{}
			&*+\txt{}
				&{x_2}
					\ar@{-}[dllll]
					\ar@{-}[ddl]
\\
{\bar x_1}
	\ar@{-}[rrrr]
	\ar@{-}[dr] \ar@{-}[drr]
	&*+\txt{}
		&*+\txt{}
			&*+\txt{}
				&{\bar x_2}
					\ar@{-}[dlll] \ar@{-}[dll] \ar@{-}[dl]
\\
*+\txt{}
	&{\bar x_2}
		&{\bar x_2}
			&{x_1}
}
\]
\caption{Graph zur Formel $\varphi$\label{phiformel}}
\end{figure}%
Als Beispiel für die Konstruktion des Graphen nehmen wir die Formel
\[
\varphi
=
(\bar x_1\vee \bar x_2\vee x_2)
\wedge
(\bar x_1\vee x_2\vee \bar x_2)
\wedge
(x_1\vee \bar x_2\vee \bar x_2)
\wedge
(\bar x_1\vee \bar x_2\vee \bar x_2)
\]
Der nach obigen Regeln konstruierte Graph ist in Abbildung~\ref{phiformel}
dargestellt.
Natürlich enthält dieser Graph genügend Kanten, so dass es relativ
leicht ist, eine $4$-Clique zu finden, zum Beispiel die in Abbildung~\ref{phiclique} dargestellte. Man kann daraus ablesen, dass die Formel $\varphi$
wahr wird, wenn $x_1$ und $x_2$ falsch sind.
\begin{figure}
\[
\entrymodifiers={++[o][F]}
\xymatrix{
*+\txt{}
	&{\bar x_1}
		&{\bar x_2}
			\ar@{-}[dddll]
			\ar@{-}[dddd]
			\ar@{-}[dddrr]
			&{x_2}
\\
{\bar x_2}
	&*+\txt{}
		&*+\txt{}
			&*+\txt{}
				&{\bar x_1}
\\
{\bar x_2}
	&*+\txt{}
		&*+\txt{}
			&*+\txt{}
				&{x_2}
\\
{\bar x_1}
	\ar@{-}[rrrr]
	\ar@{-}[drr]
	&*+\txt{}
		&*+\txt{}
			&*+\txt{}
				&{\bar x_2}
					\ar@{-}[dll]
\\
*+\txt{}
	&{\bar x_2}
		&{\bar x_2}
			&{x_1}
}
\]
\caption{$4$-Clique im Graph zur Formel $\varphi$\label{phiclique}}
\end{figure}%
\end{beispiel}

