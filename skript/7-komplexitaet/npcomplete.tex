%
% npcomplete.tex -- NP-vollständige Probleme
%
% (c) 2011 Prof Dr Andreas Mueller, Hochschule Rapperswil
%
\section{NP-vollständige Probleme}
\rhead{NP-vollständige Probleme}
\begin{figure}
\begin{center}
\includegraphics{images/lang-4}
\end{center}
\caption{Beziehung zwischen P, NP und NP-vollständigen Problemen.
\label{pnpnpcomplete}}
\end{figure}%
Die polynomielle Reduktion ordnet die Sprachen nach ``Schwierigkeitsgrad''
(Abbildung~\ref{pnpnpcomplete}).
Je ``grösser'' eine Sprache $B$ ist, desto mehr Sprachen $A$ gibt es
mit $A\le_P B$.
Findet man einen polynomiellen Algorithmus für $B$,
löst man damit automatisch auch $A$ für alle diese $A$.

Möchte man $\text{P} = \text{NP}$ beweisen, dann muss man nach
möglichst ``schwierigen'' Problemen suchen, also nach solchen,
deren polynomielle Lösung die polynomielle Lösung vieler anderer
Probleme nach sich ziehen würde.

\begin{definition}
\index{NP-vollständig}%
Eine Sprache $B$ heisst NP-vollständig, wenn 
\begin{compactenum}
\item $B\in\text{NP}$
\item $A\le_P B$ für alle $A\in\text{NP}$
\end{compactenum}
\end{definition}

Wenn man die polynomielle Lösung eines NP-vollständigen Problemes
findet, sind auch alle anderen Probleme in NP in polynomieller Zeit
lösbar:

\begin{satz}
Falls $B$ NP-vollständig ist und $B\in\text{P}$, dann ist
$\text{P}=\text{NP}$.
\end{satz}

\begin{proof}[Beweis]
Ist $A\in\text{NP}$, dann ist $A\le_P B$, weil $B$ NP-vollständig ist.
Nach Satz \ref{polynomiellreduction} ist dann aber auch
$A\in\text{P}$, alle Sprachen
in NP sind also auch in P, oder $\text{P}=\text{NP}$.
\end{proof}

