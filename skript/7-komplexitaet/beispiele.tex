%
% beispiele.tex -- Beispiele 
%
% (c) 2011 Prof Dr Andreas Mueller, Hochschule Rapperswil
%
\section{Beispiele NP-vollständiger Probleme}
\rhead{NP-vollständige Probleme}
Viele praktisch wichtige Probleme sind NP-vollständig.
Der Nachweis der NP-Vollständigkeit ist jedoch nicht immer einfach.
Im Allgemeinen
wird er dadurch geführt, dass eine Reduktion von einem bereits als
NP-vollständig bekannten Problem konstruiert wird.
Es lohnt sich
daher, einen möglichst grossen Katalog von NP-vollständigen
Problemen kennen zu lernen, einerseits um Übung im Konstruieren
von Reduktionen zu erhalten, andererseits um eine grössere Auswahl 
von Problemen zu bekommen, von denen ausgehend eine Reduktion 
konstruiert werden könnte.
In diesem Kapitel werden einige solche Probleme vollständig untersucht.
\subsection{3SAT}
\index{3SAT@\textsl{3SAT}}%
\begin{satz}
$\text{\textsl{3SAT}}$ ist NP-vollständig.
\end{satz}

\begin{proof}[Beweis]
Zum Beweis muss eine polynonmielle Reduktion
$\text{\textsl{SAT}}\le_P\text{\textsl{3SAT}}$ konstruiert werden.
Eine Reduktionsabbilung
$\text{\textsl{SAT}}\to\text{\textsl{3SAT}}$ muss eine allgemeine Formel
$\varphi$ in eine gleichwertige Formel in konjunktiver Normalform umwandeln, 
die ausserdem nur noch maximal drei Variablen in jeder Klausel enthält.

Ein erster Versuch verwendet die Distributivgesetze für die 
logischen Operationen, also
\begin{align*}
a\wedge(b\vee c)&=(a\wedge b)\vee(a\wedge c)\\
a\vee(b\wedge c)&=(a\vee b)\wedge(a\vee c),
\end{align*}
um die $\wedge$ nach ``aussen'' und die $\vee$ nach ``innen'' zu
bringen.
Aus der Formel 
\begin{equation}
(x_1\wedge y_1)
\vee
(x_2\wedge y_2)
\vee
\dots
\vee
(x_n\wedge y_n)
\label{exponentialformula}
\end{equation}
wird, wie schon in (\ref{bigcnf}) bemerkt, eine Konjunktion von $2^n$
Termen der Form
\[
(u_1\vee u_2\vee \dots u_n)
\]
wobei $u_i=x_i$ oder $u_i=y_i$ sein kann.
Die entstehende Konjunktion hat also exponentiell viele Terme, insbesondere
ist es nicht möglich, auf diesem Weg eine Reduktion
$f\colon\text{\textsl{3SAT}}\to\text{\textsl{SAT}}$ zu konstruieren.

Die gesuchte ``gleichwertige'' Formel muss nur im Sinne des
\textsl{SAT}-Problems gleichwertig sein, sie muss nicht eine
logisch äquivalente Formel sein.
Es reicht, wenn die transformierte
Formel genau dann erfüllbar ist, wenn auch die ursprüngliche Formel
erfüllbar ist.
Das bedeutet auch, dass wir neue Variablen einführen
dürfen.

Das Problem in Formel (\ref{exponentialformula}) entstand dadurch, dass
beim ``ausmultiplizieren'' sehr viele Kombinationen entstehen.
Hätte man eine Abkürzung $z_i=x_i\wedge y_i$, würde die Formel viel
kürzer, nämlich nur
\[
z_1\vee z_2\vee\dots\vee z_n.
\]
Diese Formel wäre sogar in konjunktiver Normalform, da man sie
als Formel mit nur einer Klausel betrachten kann.
Die Bedingung, dass man mit $z_i$ eigentlich $x_i\wedge y_i$ 
meint, kann man auch als 
\[
(x_i\wedge y_i)\Rightarrow z_i
=
z_i\vee \neg(x_i\wedge y_i)
=
z_i\vee \bar x_i\vee\bar y_i
\]
formulieren.
Die Übersetzung der Formel (\ref{exponentialformula})
in konjunktive Normalform ist dann
\begin{equation}
(z_1\vee z_2\vee\dots\vee z_n)
\wedge
\bigwedge_{i=1}^n (z_i\vee \bar x_i\vee\bar y_i).
\label{equivalentcnfformula}
\end{equation}
In (\ref{equivalentcnfformula}) kommen $4n$ Variablen vor,
in der ursprünglichen Formel
nur $2n$, die Formel wird also um den Faktor $2$ länger, die
Vergrösserung ist jetzt linear, nicht mehr exponentiell.

Es genügt also Regeln anzugeben, mit denen man Konjunktionen
``nach aussen'' bringen kann, so dass die Erfüllbarkeit der
Formel erhalten bleibt, und so dass die Länge der Formel 
höchstens polynomiell zunimmt.
Für die Formel
\[
\varphi=
(
\varphi_{11}
\wedge
\varphi_{12}
\wedge
\dots
\wedge
\varphi_{1n_1}
)
\vee
(
\varphi_{21}
\wedge
\varphi_{22}
\wedge
\dots
\wedge
\varphi_{2n_2}
)
\vee\dots\vee
(
\varphi_{k1}
\wedge
\varphi_{k2}
\wedge
\dots
\wedge
\varphi_{kn_k}
)
\]
verwendet man wieder Variablen $z_i$, welche dafür die einzelnen
Klammerausdrücke stehen.
Um auszudrücken, dass diese wahr sein
sollen, wenn die Klammerausdrücke wahr sind, fügt man
die Bedingungen
\[
z_i\vee \neg
(
\varphi_{i1}
\wedge
\varphi_{i2}
\wedge
\dots
\wedge
\varphi_{in_i}
)
=
z_i\vee
\neg\varphi_{i1}
\vee
\neg\varphi_{i2}
\vee
\dots
\vee
\neg\varphi_{in_i}
\]
hinzu.
So erhält man die Formel
\[
(z_1\vee z_2\vee \dots\vee z_k)
\wedge
\bigwedge_{i=1}^k \biggl(z_i\vee \bigvee_{j=1}^{n_i} \neg\varphi_{ij}\biggr).
\]
Ihre Länge ist $O(k+\sum_{j=1}^k(1+n_j))=O(|\varphi|)$.
Durch wiederholte Anwendung dieser Methode kann die
Formel also in polynomieller Zeit in konjunktive Normalform gebracht
werden.

Die inzwischen erreichte konjunktive Normalform der Formel kann durchaus
noch mehrere Variablen pro Klausel enthalten.
Sei 
\[
(x_1\vee x_2\vee \dots\vee x_n)
\]
eine solche Klausel.
Wir müssen sie ersetzen durch eine
Formel in konjunktiver Normalform mit drei Variablen pro Klausel.
Dazu kann man den folgenden Trick verwenden.
Die Formel $(\varphi\vee\psi)$ wird wahr, wenn eine der Unterformeln
$\varphi$ oder $\psi$ wahr ist.
Sie wird nicht wahr sein, wenn beide Unterformeln nicht wahr sind.
Sei $z$ eine neue Variable,
wir möchten mit ihr ausdrücken, dass $\psi$ wahr ist.
Wir verlangen $\bar z \vee \psi$, wenn $\psi$ nicht wahr ist, darf auch $z$
nicht wahr sein.
Dann ist die Formel
\[
(\varphi \vee z)\wedge(\bar z\vee \psi)
\]
genau dann erfüllbar, wenn $\varphi\vee \psi$ erfüllbar ist.
Ist $\varphi$ nicht erfüllbar, dann muss $z$ wahr sein, und
damit auch $\varphi$, d.\,h.~auch $\varphi\vee\psi$ ist erfüllbar.
Eine zu lange Disjunktion kann also immer in kürzere Disjunktionen 
aufgespaltet werden.
Rekursive Anwendung liefert jetzt eine Formel
mit maximal drei Literalen pro Klausel.
\end{proof}

\begin{satz}
$\text{\textsl{CLIQUE}}$ ist NP-vollständig.
\end{satz}

\begin{proof}[Beweis]
Da wir bereits früher gezeigt haben, dass
$\text{\textsl{3SAT}}\le_P\text{\textsl{CLIQUE}}$ ist, folgt jetzt auch,
dass $\text{\textsl{CLIQUE}}$ NP-vollständig ist.
\end{proof}

\subsection{Hamiltonscher Pfad}
Ein Hamiltonscher Pfad in einem gerichteten Graphen ist ein Pfad, der jeden
Vertex des Graphen genau einmal besucht.
Als Sprachproblem formuliert 
\[
\text{\textsl{HAMPATH}}
=\{\langle G\rangle\,|\,\text{$G$ ist ein Graph mit einem hamiltonschen Pfad}\}
\]
Es ist leicht, zu verifizieren, ob
ein Pfad ein Hamiltonscher Pfad ist:
\begin{satz} $\text{\textsl{HAMPATH}}\in\text{NP}$.
\end{satz}

\begin{proof}[Beweis]
Als Zertifikat $c$ verwendet man den Pfad.
Seine Länge ist geringer
als die Länge der Beschreibung des Graphen selbst.
Die Punkte des Pfades müssen benachbart sein, dies kann man in 
$O(n)$ testen.
Dann muss man überprüfen, ob jeder Vertex des Graphen vorkommt,
dies ist in Zeit $O(n^2)$ möglich.
Auf dem Pfad darf kein Vertex
zweimal vorkommen, auch dies kann man in $O(n^2)$ verifizieren.
Insgesamt kann man also in polynomieller Zeit nachprüfen, ob ein
Pfad tatsächlich ein hamiltonscher Pfad ist.
\end{proof}

\begin{satz} \textsl{HAMPATH} ist NP-vollständig.
\end{satz}

\begin{proof}[Beweis]
Da bereits bekannt ist dass \textsl{HAMPATH} in NP ist, muss nur
noch eine Reduktion von einem bekanntermassen NP-vollständigen
Problem auf \textsl{HAMPATH} konstruiert werden.
Die hier vorgestellte
ingeniöse Konstruktion reduziert \textsl{3SAT} auf \textsl{HAMPATH}.
Dazu muss zu jeder \textsl{3CNF}-Formel ein gerichteter Graph angegeben
werden, der genau dann einen hamiltonschen Pfad besitzt, wenn die
Formel erfüllbar ist.

Sei also $\varphi$ eine \textsl{3CNF}-Formel mit $k$ Klauseln in den Variablen
$x_1,\dots,x_n$.
Der zu konstruierende Graph muss zu jeder Variablen
$x_i$ einen Teilgraphen enthalten, mit dem ausserdem ausgedrückt
werden kann, ob die Variable wahr oder falsch ist.
Der Graph
\[
\entrymodifiers={++[o][F]}
\xymatrix{
*+\txt{}
	&*+\txt{}
		&*+\txt{}
			&*+\txt{}
				&{} \ar@/_10pt/[ddllll]\ar@/^10pt/[ddrrrr]
					&*+\txt{}
						&*+\txt{}
							&*+\txt{}
								&*+\txt{}
\\
*+\txt{}
\\
{}\ar@/^/[r] \ar@/_10pt/[ddrrrr]
	&{}\ar@/^/[r] \ar@/^/[l]
		&{}\ar@/^/[r] \ar@/^/[l]
			&{}\ar@/^/[r] \ar@/^/[l]
				&*+\txt{\dots}\ar@/^/[r] \ar@/^/[l]
					&{}\ar@/^/[r] \ar@/^/[l]
						&{}\ar@/^/[r] \ar@/^/[l]
							&{}\ar@/^/[r] \ar@/^/[l]
								&{} \ar@/^/[l]\ar@/^10pt/[ddllll]
\\	
*+\txt{}
\\
*+\txt{}
	&*+\txt{}
		&*+\txt{}
			&*+\txt{}
				&{}
					&*+\txt{}
						&*+\txt{}
							&*+\txt{}
								&*+\txt{}
}
\]
hat genau zwei hamiltonsche Pfade, die die ``Perlenkette'' in der
Mitte in entgegengesetzter Richtung durchlaufen.
Wir konstruieren einen Graphen, der
genau $n$ solche Teilgraphen untereinander enthält.
Der $i$-te solche Teilgraph steht für die Variable $x_i$.
Der Durchlaufsinn der ``Perlenkette''
von links nach rechts bedeutet, dass diese Variable wahr sein soll,
der Durchlaufsinn von rechts nach links bedeutet, dass sie falsch ist.

Die Formel $\varphi$ besteht aus $k$ Klauseln, hat also die
Form $c_1\wedge c_2\wedge\dots\wedge c_k$.
Wir wollen die Tatsache,
dass Klausel $c_j$ wahr ist, dadurch ausdrücken, dass der Pfad einen
Vertex $c_j$ besucht.
$c_j$ kann von der Variablen $x_i$ wahr gemacht
werden wird dadurch ausgedrückt, dass der Pfad beim Durchlaufen
der ``Perlenkette'' von Variable $x_i$ einen ``Abstecher'' zu $c_j$
machen kann.
Jedem Paar von Vertizes in der ``Perlenkette'' ordnen
wir daher eine Klausel zu.
Falls $x_i$ in der Klausel $c_j$
vorkommte, fügen wir zwischen den beiden der Klausel $c_j$
zugeordneten Vertizes zusätzliche Kanten nach $c_j$ und wieder
zurück hinzu.
\[
\entrymodifiers={++[o][F]}
\xymatrix{
*+\txt{}
	&*+\txt{}
		&*+\txt{}
			&*+\txt{}
				&*+\txt{}
					&*+\txt{}
						&*+\txt{}
							&*+\txt{}
								&{c_j} \ar@/_15pt/[ddllll]
\\
*+\txt{}
	&*+\txt{}\ar[dl]
		&*+\txt{}
			&*+\txt{}
				&*+\txt{}
					&*+\txt{}
						&*+\txt{}\ar[dr]
							&*+\txt{} 
\\
{}\ar@/^/[r] \ar[dr]
	&{}\ar@/^/[r] \ar@/^/[l]
		&*+\txt{\dots}\ar@/^/[r] \ar@/^/[l]
			&{}\ar@/^/[r] \ar@/^/[l] \ar@/^15pt/[uurrrrr]
				&{}\ar@/^/[r] \ar@/^/[l]
					&*+\txt{\dots}\ar@/^/[r] \ar@/^/[l]
						&{}\ar@/^/[r] \ar@/^/[l]
							&{} \ar@/^/[l]\ar[dl]
\\
*+\txt{}
	&*+\txt{}
		&*+\txt{}
			&*+\txt{}
				\&*+\txt{}
					&*+\txt{}
						&*+\txt{}
							&*+\txt{}
}
\]
Enthält die Klausel $c_j$ die Variable $\bar x_i$ wird der
``Abstecher'' zu $c_j$ in die Durchlaufrichtung von rechts nach
links eingebaut.

Die Auswahl von Wahrheitswerten für die Variablen $x_i$ entspricht
der Entscheidung, in welcher Richtung die ``Perlenketten'' durchlaufen
werden.
Die Formel ist offenbar genau dann erfüllbar, wenn ein
Abstecher zu jeder Klausel $c_j$ möglich ist, oder wenn es einen
hamiltoschen Pfad gibt.

Die Konstruktion erzeugt einen Graphen mit $O(nk)$ Vertizes,
ist also sicher in polynomialer Zeit durchführbar.
Also ist $\text{\textsl{3SAT}}\le_P\text{\textsl{HAMPATH}}$.
\end{proof}

\subsection{SUBSET-SUM}
\index{SUBSET-SUM@\textsl{SUBSET-SUM}}%
Das Problem \textsl{SUBSET-SUM} ist auch bekannt als das Rucksack-Problem.
Gegeben ist eine Menge $S$ von ganzen Zahlen, kann man darin eine
Teilmenge finden, die als Summe einen bestimmten Wert $t$ hat?
Als Sprache formuliert:
\[
\text{\textsl{SUBSET-SUM}}
=\left\{
\langle S,t\rangle\,\left|\,
\begin{minipage}{3truein}
$S$ eine Liste von ganzen Zahlen, in der es eine Teilliste
$T\subset S$ gibt mit
\[
\sum_{x\in T}x=t.
\]
\end{minipage}\right.
\right\}
\]
Der Name ``Rucksack''-Problem rührt daher, dass man sich 
die Zahlen aus $S$ als ``Grösse'' von Gegenständen vorstellt, und
wissen möchte, ob man einen Rucksack der Grösse $t$ exakt füllen
kann mit einer Teilmenge von Gegenständen aus $S$.

\begin{satz}\textsl{SUBSET-SUM} ist NP-vollständig
\end{satz}

\begin{proof}[Beweis]
Es ist ziemlich klar, dass $\text{\textsl{SUBSET-SUM}}\in\operatorname{NP}$,
es ist also nur noch eine Reduktion zu konstruieren, wir wählen
$\text{\textsl{3SAT}}\le_P\text{\textsl{SUBSET-SUM}}$.

Zu einer \textsl{3CNF}-Formel $\varphi$ mit Variablen $x_1,\dots,x_l$
muss eine Menge von Zahlen $S$ und eine Zahl $t$ gefunden werden,
die genau dann eine Teilmenge mit Summe $t$ hat, wenn $\varphi$
erfüllbar ist.

\begin{table}
\begin{center}
\begin{tabular}{|c|ccccc|}
\hline
Zahl&1&2&3&\dots&$l$\\
\hline
$y_1$&1&0&0&\dots&0\\
$z_1$&1&0&0&\dots&0\\
$y_2$& &1&0&\dots&0\\
$z_2$& &1&0&\dots&0\\
$y_3$& & &1&\dots&0\\
$z_3$& & &1&\dots&0\\
\vdots&& & &     & \\
$y_l$& & & &\dots&1\\
$z_l$& & & &\dots&1\\
\hline
$t$&1&1&1&\dots&1\\
\hline
\end{tabular}
\end{center}
\caption{Abbildung der Variablen $x_i$ und $\bar x_i$ in Zahlen
eines \textsl{SUBSET-SUM}-Problems\label{subsetsumnumbers}}
\end{table}

Wir konstruieren Zahlen $y_i$ und $z_i$, die jeweils ausdrücken
ob $x_i$ wahr oder falsch ist.
Wir bauen die Zahlen in einer
Stellendarstellung zu einer später zu bestimmenden genügend
grossen Basis auf.
Sofern die Zahlen nur wenige Stellen haben, die 
von $0$ verschieden sind, gibt es bei der Addition keinen Übertrag.
Die Zahlen $y_i$ und $z_i$ müssen so sein, dass nur jeweils eine
in der Summe auftreten kann.
Dies erreichen wir, wenn wir
sie gemäss Tabelle \ref{subsetsumnumbers} wählen.
Die Auswahl einer Teilmenge mit Summe $t$ ist gleichbedeutend
mit der Wahl einer Belegung der Variablen $x_i$ mit Wahrheitswerten.
\begin{table}
\begin{center}
\begin{tabular}{|c|ccccc|cccc|}
\hline
Zahl&1&2&3&\dots&$l$&$c_1$&$c_2$&\dots&$c_k$\\
\hline
$y_1$&1&0&0&\dots&0&1&0&\dots&0\\
$z_1$&1&0&0&\dots&0&0&0&\dots&0\\
$y_2$& &1&0&\dots&0&0&1&\dots&0\\
$z_2$& &1&0&\dots&0&1&0&\dots&0\\
$y_3$& & &1&\dots&0&1&1&\dots&0\\
$z_3$& & &1&\dots&0&0&0&\dots&0\\
\vdots&& & &     & & & &     & \\
$y_l$& & & &     &1&0&1&\dots&1\\
$z_l$& & & &     &1&0&0&\dots&0\\
\hline
$g_1$& & & &     & &1&0&\dots&0\\
$h_1$& & & &     & &1&0&\dots&0\\
$g_2$& & & &     & & &1&\dots&0\\
$h_2$& & & &     & & &1&\dots&0\\
\vdots&& & &     & & & &     & \\
$g_k$& & & &     & & & &     &1\\
$h_k$& & & &     & & & &     &1\\
\hline
$t$  &1&1&1&\dots&1&3&3&\dots&3\\
\hline
\end{tabular}
\end{center}
\caption{Reduktion von \textsl{3SAT} auf
\textsl{SUBSET-SUM}\label{subsetsumtable}}
\end{table}

Jetzt muss noch codiert werden, in welchen Klauseln von $\varphi$ 
die Variablen $x_i$ auftreten.
Dazu fügen wir der Tabelle für jede
Klausel $c_i$ eine weiter Spalte hinzu, und tragen eine $1$ ein
in der Zeile von $y_i$ wenn $x_i$ in der Klausel $c_j$ vorkommt.
Falls $\bar x_i$ in $c_j$ vorkommt tragen wir eine $1$ in der Zeile
von $z_i$ ein.

Die Formel ist erfüllbar, wenn es eine Belegung mit Wahrheitswerten
gibt, die jede Klausel wahr macht.
Dabei können auch mehrere Literale
in einer Klausel war werden, die Summe der Spalte einer wahren Klausel kann
also Werte zwischen 1 und 3 annehmen.
Da die Summe $t$ exakt erreicht
werden muss, führen wir für jede Klausel zusätzliche Zahlen $g_j$ und
$h_j$ hinzu, die in der Spalte der Klausel $c_j$ eine $1$ haben und sonst
nur $0$.
Indem man zur Teilmenge falls nötig $g_j$ und $h_j$ hinzunimmt,
kann man erreichen, dass die Summe in der Spalte von Klausel $c_j$ immer
$3$ ergibt.
Die Tabelle \ref{subsetsumtable} zeigt die konstruierten Zahlen
für die Formel
\[
(x_1\vee \bar x_2\vee x_3)\wedge(x_2\vee x_3\vee x_l)\wedge \dots\wedge
(\dots\vee \bar x_3).
\]
Da die grösste vorkommende Ziffer eine $3$ ist, kann man die
Zahlen aus $S$  und $t$ im $4$-er-System aus der Tabelle ablesen.

Der Zeitaufwand für die Erstellung der Tabelle \ref{subsetsumtable}
ist $O((k+l)^2)$, also sicher polynomiell in der Länge der
Formel $\varphi$.

Damit ist 
$\text{\textsl{3SAT}}\le_P\text{\textsl{SUBSET-SUM}}$ gezeigt.
\end{proof}


