%
% koufzeitmplexitaet.tex -- Laufzeitkomplexität
%
% (c) 2011 Prof Dr Andreas Mueller, Hochschule Rapperswil
%
\section{Laufzeitkomplexität}
\rhead{Laufzeitkomplexität}
\index{Laufzeitkomplexit\ät}%
\subsection{Inputlänge}
\index{Inputl\änge}%
Wenn die absolute Laufzeit eines Algorithmus nicht als Mass für
die Komplexität eines Problems dienen kann, dann vielleicht
wenigstens die Abhängigkeit der Laufzeit von der Grösse des
Problems. Da wir unter Problemen immer ihre Übersetzung in ein
Sprachproblem verstehen, haben wir für die Grösse des Problems
ein Mass: die Länge des Wortes, welches das Problem beschreibt.

Als Beispiel betrachten wir das Problem
\index{PRIME@\textsl{PRIME}}%
\[
\text{\textsl{PRIME}}=\{ n\;|\; \text{$n\in\mathbb N$ und $n$ ist prim}\}.
\]
Es ist entscheidbar, man kann zum Beispiel alle Teiler durchtesten,
das ist mit $n$ Tests möglich. Ein Entscheider ist also eine Turingmaschine,
der man die Zahl $n$ auf das Band schreibt, und die Turingmaschine
stellt dann fest, ob die Zahl eine Primzahl ist. Die Zahl $n$ muss
als String auf das Band geschrieben werden. Dazu ist eine Codierung
zu wählen, üblicherweise wird dies eine Zifferndarstellung in
irgendeiner Basis $b$ sein. Zur Darstellung der Zahl $n$ in Basis $b$
sind $l=\lfloor \log_bn\rfloor+1$ Zeichen notwendig.

Man könnte aber auch die unäre Darstellung verwenden, also so viele
$0$ auf das Band schreiben, wie die Zahl $n$ angibt. In dieser Codierung
ist also der Input wesentlich länger, nämlich $l=n$.
Für die Beurteilung der
Komplexität kann also die Codierung des Inputs durchaus einen
merklichen Einfluss haben. Nehmen wir an, der Algorithmus hat
eine Laufzeit proportional zu $n$, dann wächst die Laufzeit im
zweiten Fall proportional zur Inputlänge $l$, im ersten Fall
aber wächst die Laufzeit mit $b^l$. Man kann also Algorithmen
nur vergleichen, wenn man die gleiche Inputcodierung verwendet.

\subsection{Laufzeit}
\index{Laufzeit}%
Sei jetzt also eine Turingmaschine $M$ gegeben, die ein Entscheider
ist. Für jedes Inputwort $w$ wird die Maschine eine gewisse Zeit
laufen und dann im Zustand $q_{\text{accept}}$ oder $q_{\text{reject}}$
anhalten. Die Anzahl der Rechenschritte, die die Maschine auf dem
Inputwort $w$ benötigt, bezeichnen wir mit $t(w)$.
Die maximale Laufzeit, die für Inputwörter der Länge nicht
grösser als $n$ benötigt wird, bezeichnen wir mit $t(n)$,
\[
t(n)=\max \{ t(w)\;|\; |w|\le n\}.
\]
Der absolute Wert von $t(n)$ ist nicht interessant, da er sich
selbst unter trivialen Veränderungen der Turingmaschine verändern
kann. Das Verhalten von $t(n)$ für grosse $n$ hingegen scheint
wesentlich robuster zu sein, es interessiert also nur noch, wie
sich verschiedene Funktionen zueinander verhalten, wenn man $n\to\infty$
streben lässt. Falls wir die Turingmaschine angeben wollen, deren
Laufzeit bestimmt wird, schreiben wir sie als Index zu $t$: $t_M(n)$.

\begin{definition}
Sind $f$ und $g$ Funktionen $\mathbb N\to\mathbb R^+$, dann ist
$f(n)=O(g(n))$, falls es eine Konstante $c$ und ein $n_0\in\mathbb N$
gibt mit
\[
f(n)\le cg(n)\quad \forall n\in\mathbb N\text{ mit } n > n_0.
\]
Es ist $f(n)=O(g(n))$ wenn 
\begin{equation}
\lim_{n\to\infty}\frac{f(n)}{g(n)}=0.
\label{oquotientlimit}
\end{equation}
\end{definition}
\index{Laufzeit!polynomielle}%
Man beachte, dass für $f(n)=O(g(n))$ die Ungleichung $f(n)\le
cg(n)$ nicht für alle $n$ gelten muss, insbesondere spielen kleine
Werte von $n$ keine Rolle, es interessiert uns nur das Verhalten
bei grossen Werten von $n$.
Und auch für $f(n)=O(g(n))$ sind die kleinen Werte von $n$ nicht
von Bedeutung, die Funktionswerte für kleine $n$ haben keinen
Einfluss auf den Grenzwert des Quotienten (\ref{oquotientlimit}).

Die Funktion $g$ muss offenbar nicht sonderlich genau bekannt
sein. Ist $g$ zum Beispiel ein Polynom, interessiert nur noch
die höchste Potenz. Es gilt ja zum Beispiel
$ n^r\le n^k $ für jeden Exponenten $r<k$, und damit
$n^r=O(n^k)$. Ausserdem ist $an^r=O(n^k)$, und damit
für $f(n)=a_0+a_1n+a_2n^2+\dots+a_{k-1}n^{k-1}+a_nn^k$ auch
$ f(n)=O(n^k) $. Somit genügt es bei Polynomen für $g$ die
höchste Potenz anzugeben.

\begin{beispiel}
Für das früher genannte Beispiel \textsl{PRIME} wurde 2004 etwas
überraschend ein Algorithmus mit
polynomieller Laufzeit gefunden \cite{skript:aks}.
\end{beispiel}

\subsection{Varianten von Turingmaschinen}
Wieviel schneller arbeitet eine Turingmaschine mit mehreren Bändern 
gegenüber einer Standardturingmaschine? Wieviel schneller ist eine
nichtdeterministische Turingmaschine gegenüber einer deterministischen?

\begin{satz}
\index{Turing-Maschine!mit mehreren B\ändern}%
Eine Turingmaschine mit mehreren Bändern und Laufzeit $t(n)$ kann
in Laufzeit $O(t(n)^2)$ auf einer Standardturingmaschine
simuliert werden.
\end{satz}

\begin{proof}[Beweis]
In Satz \ref{mehrbandturingmaschine} wurde beschrieben, wie man eine
Turingmaschine mit mehreren Bändern auf einer Turingmaschine mit
einem einzigen Band simulieren kann. Dabei ist für jeden der
$t(n)$ Berechnungsschritt ein Durchgang durch das Band nötig, bei
dem die Turingmaschine die Inhalte der markierten Felder zusammensammelt,
um dann die notwendigen Änderungen zu bestimmen. Die Änderungen
auf dem Band einzutragen erfordert dann nochmals einen Durchgang
durch das Band.

Wir nehmen an, dass $t(n)\ge n$, denn sonst könnte die
Turingmaschine nicht einmal ihren Input vollständig lesen.
Der tatsächlich benutzte Teil des Bandes kann dann nicht länger sein
als $t(n)$. Zusammen mit den $t(n)$ durchläufen erhalten wir,
dass die simulierte Turingmaschine die Laufzeit $O(t(n)^2)$ hat.
\end{proof}

Einen wesentlichen Unterschied der Laufzeit erwarten wir aber
bei nichtdeterministischen Turingmaschinen.
Die Berechnung in einer nichtdeterministischen Turingmaschine
kann ja auf ganz verschiedenen Wegen stattfinden, wovon
einer gefunden werden muss, der akzeptiert. Wenn eine simulierende
Turingmaschine alle Wege durchprobiert, spielt die Laufzeit auf
demjenigen Weg, der am längsten braucht, eine wesentliche Rolle.

\begin{definition}
\index{Laufzeit!einer nichtdeterministischen Turingmaschine}%
Sei $N$ eine nicht deterministische Turingmaschine, die auch ein
Entscheider ist. Dann ist ihre Laufzeit $t(n)$ die maximale Anzahl
von Schritten, die jede mögliche Berechnung auf einem Input der
Länge $\le n$ braucht.
\end{definition}

\begin{satz}
\label{exponentialtime}
Eine nichtdeterministische Turingmaschine mit Laufzeit $t(n)$
kann in Laufzeit $2^{O(t(n))}$
auf einer deterministischen Turingmaschine simuliert werden.
\end{satz}

\begin{proof}[Beweis]
In Satz \ref{nichtdeterministischeturingmaschine} wurde gezeigt,
wie man eine nichtdeterministische Turingmaschine auf einer
Mehrband-Maschine simulieren kann. Da wir im vorliegenden Fall
sogar einen Entscheider haben, können wir den Algorithmus noch
etwas vereinfachen. Wir verwenden Band~1 wieder für den Input,
Band~2 als Arbeitsband für die simulierte Turingmaschine und
Band~3 als Steuerband für die nichtdeterministischen
Entscheidungen.

Wir müssen ausrechnen, wie sich die Simulation auf die 
Gesamtlaufzeit auswirkt. Jeder Berechnungspfad der simulierten
nichtdeterministischen Turingmaschine hat Laufzeit nicht
länger als $t(n)$. Die Erzeugung der nächsten möglichen
Berechnungsgeschichte auf Band~3 kann in einem Durchgang
durch Band~3 erfolgen, braucht also Zeit $O(t(n))$.
Die Anzahl der Berechnungsgeschichten ist $|Q\times \Gamma|^{t(n)}$,
also von der Form $c^{t(n)}$. Im schlimmsten Fall ist die
Laufzeit dieses Algorithmus also
\[
c^{t(n)}(t(n) + O(t(n)))
=
2^{t(n) \log_2 c+ \log_2(O(t(n)))}
=
2^{O(t(n))}.
\]
\end{proof}
\subsection{Wie langsam ist exponentielle Laufzeit?}
\index{Laufzeit!exponentielle}%
Satz \ref{exponentialtime} zeigt, dass zur Simulation eines
nichtdeterministischen Automaten eine exponentiell längere
Zeit nötig ist, als die nichtdeterministische Maschine
benötigt hätte. Aber wieviel langsamer ist das?
Zunächst ist für jeden Exponenten $k$
\[
\lim_{n\to\infty}\frac{2^{n}}{n^k}=\infty,
\]
$2^n$ wächst also schneller als jedes Polynom. 
Setzen wir $t=n^r$, dann ist auch
\[
\lim_{n\to\infty}\frac{2^{n^r}}{n^k}
=
\lim_{t\to\infty}\frac{2^t}{t^{k/r}}
=
\infty,
\]
wie zu erwarten war, wächst $2^{n^k}$ noch schneller.
Da nur schon für das Lesen des Input $n$ Schritte notwendig sind,
spielt es gar keine Rolle, wie effizient der Algorithmus der
nichtdeterministischen Turingmaschine ist, es ist unmöglich,
je polynomielle Laufzeit zu erreichen.

Aber wie gross ist denn nun exponentielle Laufzeit?
Nehmen wir an,
ein wir hätten einen Algorithmus mit Laufzeit $n^k$. Vergrössert
man die Input-Länge um $1$ Zeichen, ändert sich die Laufzeit um
\[
\frac{(n+1)^k}{n^k}
=
1+\frac{k}{n} +\dots,
\]
die relative Zunahme wird also immer kleiner. Bei einer Verdoppelung
der Inputlänge wird die Laufzeit mit dem Faktor $2^k$ multipliziert.
Ganz anders bei exponentieller Laufzeit.
Erhöht man die Inputlänge um $1$, wird die Laufzeit mit einem
konstanten Faktor multipliziert. Und die Verdoppelung von $n$
quadriert die Laufzeit.
Ausgehend von einer angenommenen
Zeit pro Schritt von einer Nanosekunde, ergeben sich
für die Laufzeit eines Algorithmus
mit polynomieller Laufzeit, genauer mit $n^5$ und exponentieller Laufzeit,
genauer mit $2^n$ die Werte in Tabelle~\ref{laufzeittabelle}.
\begin{table}
\begin{center}
\begin{tabular}{|l|rr|}
\hline
$n$&$n^5$&$2^n$\\
\hline
  1&             1ns&             2ns\\
  2&            32ns&             4ns\\
  4&       1.0$\mu$s&            16ns\\
  8&      32.7$\mu$s&           256ns\\
 16&           1.0ms&      65.5$\mu$s\\
 32&          33.5ms&           4.3ms\\
 64&            1.0s&       584 Jahre\\
128&           34.4s& $10^{21}$ Jahre\\
\hline
\end{tabular}
\end{center}
\caption{Polynomielle und exponentielle Laufzeit\label{laufzeittabelle}}
\end{table}
Die Zeit von $10^{21}$ Jahren entspricht dem 78 Milliardenfachen des
Alters des Universums. Grosse Probleme mit exponentieller Laufzeit
sind also schlicht unlösbar, während selbst relativ grosse
polynomielle Laufzeiten durchaus im Bereich des erreichbaren sind.
Der Unterschied zwischen polynomieller und exponentieller Laufzeit
ist also für grosse Inputlänge der Unterschied zwischen lösbaren und
unlösbaren Problemen.

