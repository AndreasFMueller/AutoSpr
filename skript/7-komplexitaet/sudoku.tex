\subsection{SUDOKU und SAT\label{subsection:sudoku-und-sat}}
\index{Sudoku}%
Als Vorstufe der angestrebten Erkenntnis, dass sich jedes Problem auf \textsl{SAT}
reduzieren lässt, wollen wir zeigen, dass sich die Frage nach der
Lösbarkeit eines Sudoku-Rätsels auf \textsl{SAT} reduzieren lässt.

Ein $n$-Sudoku ist ein $n^2\times n^2$-Feld, welches mit $n^2$ verschiedenen
Zeichen aus einer Menge $\Sigma$ so gefüllt werden muss,
dass in jeder Zeile, jeder Spalte
und in jedem der $n\times n$-Unterquadrate jedes Zeichen genau einmal
vorkommt.
Die aus Rätselspalten von Zeitungen bekannten $3$-Sudokus verwenden
die Ziffern $1$ bis $9$ für $\Sigma$.
Einige der Zeichen sind bereits vorgegeben, die leeren Felder müssen
so gefüllt werden, dass die Regeln nicht verletzt werden.
Abbildung~\ref{sudoku} zeigt ein lösbares $3$-Sudoku-Rätsel.
\begin{figure}
\begin{center}
\includegraphics[width=0.4\hsize]{images/sudoku-1.pdf}
\end{center}
\caption{$3$-Sudoku, die hellen Ziffern zeigen, dass die Vorgabe (dunkle
Ziffern) erfüllbar ist, das $3$-Sudoko ist lösbar.\label{sudoku}}
\end{figure}
Wir schreiben
\begin{align*}
\textsl{SUDOKU}
=
\{\langle S\rangle\;|\; \text{$S$ ist ein lösbares $n$-Sudoku}\}.
\end{align*}

Ob ein $n$-Sudoku lösbar ist, ist sicher entscheidbar: man kann alle
$|\Sigma|^{n^2\cdot n^2}=n^{2n^4}$ möglichen Belegungen des Feldes mit
Zeichen durchprobieren, was aber natürlich exponentielle Zeit
beansprucht.

\textsl{SUDOKU} ist auch in NP, denn man kann als Lösungszertifikat
die Zeichen im vollständig ausgefüllten Feld verwenden.
Zur Verifikation muss man dann nur überprüfen, ob die
vorgegebenen Zeichen richtig in die Lösung übernommen wurden und
ob die Sudoku-Regeln eingehalten wurden.
Dies ist in Laufzeit $O(n^4)$
möglich, es gibt also einen polynomiellen Verifizierer.

Der Satz~\ref{cooklevin} behauptet, \textsl{SAT} sei NP-vollständig,
es müsste sich also auch \textsl{SUDOKU} polynomiell auf \textsl{SAT}
reduzieren lassen:

\begin{satz} $\textsl{SUDOKU}\le_P\textsl{SAT}$.
\label{skript:satz:sudoku-sat}
\end{satz}

\begin{proof}[Beweis]
Wir müssen eine Reduktion konstrukieren, die jedem Sudoku-Rätsel
eine logische Formel zuordnet, welche genau dann erfüllbar ist, wenn
das Sudoku-Rätsel lösbar ist.
Wir gehen in zwei Schritten vor.
Zunächst konstrieren wir eine Formel, die genau dann erfüllbar ist,
wenn das Sudoku lösbar ist, die aber noch keine logische Formel ist.
Im zweiten Schritt wandeln wir die Formel dann in eine äquivalente
logische Formel um.

1.~Schritt: Sei also $S$ ein gegebenes $n$-Sudoku.
Wenn das Rätsel lösbar
ist, dann gibt es für jedes Feld ein Zeichen, für welches alle Sudoku-Regeln
eingehalten werden.
Wir bezeichnen das Zeichen in dem Feld in Zeile $i$
und Spalte $j$ mit $z_{ij}$.
Für einige Felder gibt es Vorgaben, die wir mit
$v_{ij}$ bezeichnen.
Jetzt müssen die Bedingungen für eine korrekte
Lösung des Sudoku formuliert werden:
\begin{compactenum}
\item {\em Vorgaben sind korrekt:} Die Vorgabe im Feld $(i,j)$ ist erfüllt,
wenn gilt $z_{ij}=v_{ij}$.
Alle Vorgaben sind erfüllt, wenn der Ausdruck
\[
\varphi_{\text{Vorgaben}}=\bigwedge_{\text{Feld $(i,j)$ ist Vorgabefeld}}(z_{ij}=v_{ij})
\]
wahr ist.
\item {\em Zeilenbedingung:} Der Wert $z_{ij}$ in Zeile $i$ ist erlaubt,
wenn jedes andere $z_{ij}$ in der Zeile $i$ verschieden davon ist:
\[
\varphi_{\text{Zeilenbedingung $(i,j)$}}
=
\bigwedge_{k\ne j}(z_{ij}\ne z_{ik}).
\]
\item {\em Spaltenbedingung:} Der Wert $z_{ij}$ in Spalte $j$ ist erlaubt,
wenn jedes andere $z_{kj}$ in der Spalte $j$ verschieden ist:
\[
\varphi_{\text{Spaltenbedingung $(i,j)$}}
=
\bigwedge_{k\ne i}(z_{ij}\ne z_{kj}).
\]
\item {\em Unterfeld-Bedingung:} Der Wert $z_{ij}$ im $n\times n$-Unterfeld,
welches $(i,j)$ enthält, ist zulässig, wenn jedes andere Zeichen im gleichen
Unterfeld davon verschieden ist:
\[
\varphi_
{\text{Unterfeldbedingung $(i,j)$}}
=
\bigwedge_{\myatop{\text{$(k,l)$ im Unterfeld $(i,j)$}}{i\ne k\wedge j\ne l}}(z_{ij}\ne z_{kl}).
\]
\end{compactenum}
Für jedes Feld $(i,j)$ gibt es also drei Bedingungen:
\[
\varphi_{ij}=
\varphi_{\text{Zeilenbedingung $(i,j)$}}
\wedge
\varphi_{\text{Spaltenbedingung $(i,j)$}}
\wedge
\varphi_{\text{Unterfeldbedingung $(i,j)$}}.
\]
Eine Lösung für das Sudoku kann genau dann gefunden werden, wenn man
$z_{ij}$ so wählen kann, dass alle diese Feld-Bedingungen und die
Vorgabe-Bedingungen erfüllt sind, wenn also die Formel
\begin{equation}
\varphi=
\varphi_{\text{Vorgaben}}\wedge
\bigwedge_{1\le i,j\le n^2}
\varphi_{ij}
\label{sudoku-formel}
\end{equation}
erfüllbar ist.

2.~Schritt: Die Formel (\ref{sudoku-formel}) ist nahe am Gesuchten,
sie ist genau dann erfüllbar, wenn das Sudoku lösbar ist.
Allerdings
ist sie keine logische Formel, die Variablen $z_{ij}$ sind Variablen mit
Zeichenwerten.
Wir können Sie aber durch $n^2$ neue logische Variablen
$x_{ijc}$ ersetzen, wobei $c\in \Sigma$ ist, mit der Bedeutung, dass
$x_{ijc}$ genau dann wahr ist, wenn $z_{ij}=c$.

Jede Teilformel von (\ref{sudoku-formel}) setzt sich zusammen
aus Ausdrücken der Form oder $z_{ij}=c$
oder $z_{ij}\ne z_{kl}$, wobei $c\in\Sigma$.
Es reicht also, wenn man jede Teilformel nur durch die Variablen $x_{ijc}$ 
ausdrücken kann.
\begin{compactenum}
\item {\em Nur ein Wert:} Für jedes Paar $(i,j)$ darf nur eine der
Variablen $x_{ijc}$ wahr sein, es kann ja nur ein einziges Zeichen $c$
die Bedingung $z_{ij}=c$ erfüllen.
Die Formel
\[
x_{ijc}\wedge \bigwedge_{d\in\Sigma\setminus\{c\}}\bar x_{ijd}
\]
muss also für mindestens ein $c$ wahr sein:
\[
\varphi_{\text{Eintrag $(i,j)$}}
=
\bigvee_{c\in \Sigma}
\biggl(
x_{ijc}\wedge \bigwedge_{d\in\Sigma\setminus\{c\}}\bar x_{ijd}
\biggr).
\]
\item {\em Vorgabe:} Um die Vorgabe $z_{ij}=c$ auszudrücken, muss man
nur verlangen, dass $x_{ijc}$ wahr ist.
\item {\em Ungleichheit:} Um auszudrücken, dass $z_{ij} \ne z_{kl}$ muss
man nur verlangen, dass $x_{ijc}$ und $x_{klc}$ nicht gleichzeitig wahr
sein können:
\[
\varphi_{\text{Ungleichheit $z_{ij}\ne z_{kl}$}}
=
\bigwedge_{c\in\Sigma}
\overline{x_{ijc}\wedge x_{klc}}.
\]
\end{compactenum}
Durch Anwendung aller dieser Umformungen auf alle Terme in (\ref{sudoku-formel})
erhält man eine neue Formel $\varphi$, die nur noch die logischen
Variablen $x_{ijc}$ enthält.
Die neue Formel ist genau dann erfüllbar,
wenn das Sudoku-Rätsel eine Lösung hat.
\end{proof}

\subsection{Vergleichsformeln}
Aus dem letzten Teil des Beweises von Satz~\ref{skript:satz:sudoku-sat}
können wir ausserdem ablesen, dass die Übersetzung von den Variablen $z_{ij}$
in eine logische Formel mit den logischen Variablen $x_{ijc}$ ebenfalls
algorithmisch möglich ist.
Voraussetzung dafür ist, dass es die Regeln als eine logische Formel
in den Ausdrücken $z_{ij}=z_{kl}$ formuliert werden können.
Wir nennen eine solche Formel eine {\em Vergleichsformel}.

\begin{definition}
Eine {\em Vergleichsformel} in den Variablen $z_{ij}$ ist eine
logische Formel in den Teilformeln $f_{ijkl}=(z_{ij}=z_{kl})$.
\end{definition}

Durch Verwendung der Variablen $x_{ijc}$ mit $c\in\Sigma$ kann man
jeden Ausdruck $f_{ijkl}$ in polynomieller Zeit in eine logische
Formel in den Variablen $x_{ijc}$ umwandeln.
Dies beweist den folgenden Satz.

\begin{satz}
\label{skript:satz:vergleichsformel}
Eine Vergleichsformel in den Variablen $z_{ij}$, deren Werte in $\Sigma$
liegen, kann in polynomieller Zeit in eine logische Formel in Variablen
$x_{ijc}$, $c\in\Sigma$ umgewandelt werden, die genau dann erfüllbar ist,
wenn die Vergleichsformel erfüllbar ist.
\end{satz}





