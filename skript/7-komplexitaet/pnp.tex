%
% pnp.tex -- Klassen P und NP
%
% (c) 2011 Prof Dr Andreas Mueller, Hochschule Rapperswil
%
\section{Klassen P und NP}
\index{Klasse P}%
\rhead{Klassen P und NP}
\subsection{Klasse P}
Bei der Simulation einer Turingmaschine mit mehreren Bändern
auf einer Standardmaschine wird die Laufzeit im schlimmsten
Fall quadriert. Die Funktion $t_M(n)$ ist also keine Invariante
in der Menge der entscheidbaren Sprachen. Für das gleiche Problem kann
je nach eingesetztem Maschinentyp ein völlig anderes Laufzeitverhalten
beobachtet werden. 

Erhalten bleibt aber die Eigenschaft der Funktion $t_M(n)$, 
polynomiell zu sein. Simuliert man eine Turingmaschine $M_1$ mit
mehreren Bändern auf einer Standardturingmaschine $M_2$,
dann ist $t_{M_2}=O(t_{M_1}(n)^2)$.
Wenn also $t_{M_1}(n)=O(n^k)$ ist, dann ist $t_{M_2}(n)=O(n^{2k}).$
Beide Maschinen haben eine Laufzeit, die durch ein Polynom beschränkt
sind.

\begin{definition}
\index{Klasse P}%
Die Klasse $P$ besteht aus den Sprachen, die mit einem Entscheider
mit polynomieller Laufzeit entschieden werden können.
\end{definition}

\subsection{Beispiele von Sprachen in P}
\index{Sprachen!kontextfreie}%
\begin{satz}
Kontextfreie Sprachen sind in P.
\end{satz}

Der Satz besagt, dass es zu jeder kontextfreien Sprache eine
Turingmaschine gibt, die die Zugehörigkeit eines Wortes zur
Sprache in einer Zeit entscheiden kann, die polynomiell ist in
der Länge des Wortes.

\begin{proof}[Beweis]
Der CYK-Algorithmus aus Satz \ref{cyk-algorithm}
kann in Zeit $O(n^3)$ durchgeführt werden.
\end{proof}

Sei $G$ ein gerichteter Graph und $s$ und $t$ zwei Vertizes
des Graphen. Das Pfad-Problem fragt, ob es einen von $s$ nach $t$
führenden Pfad in dem Graphen gibt. Übersetzt in ein Spracheproblem
liefert es uns den folgenden Satz.

\begin{satz}
\index{PATH@\textsl{PATH}}%
Die Sprache
\[
\text{\textsl{PATH}}
=\{
\langle G,s,t\rangle
\;|\;\text{$G$ ist ein gerichteter Graph mit einem Pfad von $s$ nach $t$}
\}
\]
ist in P.
\end{satz}

\begin{proof}[Beweis]
\index{Markierungsalgorithmus}%
Ein Markierungsalgorithmus, der beginnend bei $s$ alle erreichbaren
Vertizes markiert, solange sich noch neue Vertizes markieren lassen,
hat polynomielle Laufzeit.
\end{proof}

\begin{satz}
\index{RELPRIM@\textsl{RELPRIME}}%
Die Sprache
\[
\text{\textsl{RELPRIME}}
=\{
\langle a,b\rangle \;|\;
\text{$a, b\in \mathbb N$ und $a$ und $b$ sind teilerfremd}
\}
\]
ist in P.
\end{satz}

\begin{proof}[Beweis]
\index{Algorithmus!euklidischer}%
Teilerfremdheit kann mit dem euklidischen Algorithmus entscheiden werden.
Dieser funktioniert wie folgt:
\begin{compactenum}
\item Wiederhole Schritte 2 und 3 bis $y=0$:
\item Weise $x$ den Wert $x\mod y$ zu
\item Vertausche $x$ und $y$
\item Gebe Resultat $x$ zurück.
\end{compactenum}
Bei diesem Algorithmus wird $x$ in jedem Schritt mindestens halbiert,
die Laufzeit ist also proportional zur grösseren der
beiden Zahlen $\log_2 a$ und $\log_2 b$, also ist die Laufzeit
polynomiell in der Inputlänge $\log_2 a+\log_2 b$.
\end{proof}

\subsection{Verifizierer}
Sprachen, die von einer nichtdeterministischen Turingmaschine
in polynomieller Zeit entschieden werden, brauchen bei der
Simulation auf einer deterministischen Turingmaschine im
schlimmsten Fall exponentiell länger. Es ist also im
Allgemeinen fast unmöglich, ein solches Problem zu lösen.

Gibt man allerdings die Lösung vor, zum Beispiel indem man
die richtigen nichtdeterministischen Entscheidungen vorgibt,
kann man die Problemlösung in polynomieller Zeit nachvollziehen.
Genau diese Idee formalisiert das Konzept des Verifizierers.

\begin{definition}
\index{Verifizierer}%
Ein Verifizierer für die Sprache $A$ ist eine Turingmaschine
$V$ mit
\[
A=\{
w\;|\;\text{$\exists c\in C^*$ so dass $V$ $\langle w,c\rangle$ akzeptiert}
\},
\]
wobei $C$ eine endliche Menge ist.
Ein Verifizierer heisst polynomiell, wenn seine Laufzeit polynomiell
ist in der Länge des Wortes $w$.
\end{definition}

Man beachte, dass die Länge von $c$ auf die Laufzeit keinen
Einfluss haben darf. Der Verifizierer muss nämlich gar nicht
den ganzen String $c$ anschauen, er muss davon nur soviel
nehmen, wie er für die Verifikation braucht. Aber natürlich
muss er den ganzen String $w$ lesen, von dessen Länge muss
die Laufzeit abhängen.

Ein Verifizierer ist erst nützlich, wenn man auch $c\in C^*$
hat. Im Falle des Akzeptanzproblems durch eine nichtdeterministische
Turingmaschine (siehe Einführungsbeispiel) kann die Folge der
nichtdeterministischen Entscheidungen in der Berechnung diese
Funktion übernehmen. Daraus schliessen wir

\begin{satz} Eine Sprache wird genau dann von einer
nichtdeterministischen Turingmaschine in polynomieller Zeit entschieden,
wenn sie einen polynomiellen Verifizierer hat.
\end{satz}

\begin{proof}[Beweis]
Bereits gezeigt haben wir, dass eine nichtdeterministisch in
polynomieller Zeit entscheidbare Sprache einen polynomiellen Verifizierer hat.
Es ist also
nur noch zu klären, dass auch das Umgekehrte gilt, dass also eine
Sprache mit einem polynomiellen Verifizierer nichtdeterministisch
in polynomieller Zeit entschieden werden kann.

Dazu verwenden wir den folgenden Entscheidungsalgorithmus:
\begin{compactenum}
\item Erzeuge nicht deterministisch $c\in C^*$.
\item Teste, ob $V$ $\langle w,c\rangle$ akzeptiert.
\item falls $V$ akzeptiert: $q_{\text{accept}}$, falls nicht:
$q_{\text{reject}}$
\end{compactenum}
Die Laufzeit dieses Algorithmus wird bestimmt von der Laufzeit von $V$,
ist also polynomiell in der Länge von $w$.
\end{proof}

\begin{beispiel}[Faktorisierung]
Die Faktorisierung eines Produktes von zwei grossen Primzahlen
gilt allgemein als schwierig. Die Sicherheit des RSA-Algorithmus,
der in zahlreichen im Internet verbreiteten kryptographischen
Protokollen verwendet wird, basiert wesentlich auf der Tatsache, dass
es sehr einfach ist, ein Produkt von zwei grossen Primzahlen zu
bilden, aber sehr viel schwieriger, aus dem Produkt die beiden
Faktoren wieder zu ermitteln.

Es geht also um die Sprache
\[
L
=
\{
n\;|\,\text{$n=pq$, wobei $p$ und $q$ Primzahlen sind}.
\}
\]
Erfahrungsgemäss ist es praktisch unmöglich, die Faktoren
einer Zahl in $n$ zu bestimmen, und damit zu entscheiden, ob
$n\in L$ ist. Kennt man jedoch einen der beiden Faktoren,
zum Beispiel $p$, dann kann man mit dem Divisionsalgorithmus
nachprüfen, ob die Division ``aufgeht'', was in einer Anzahl
von Schritten möglich ist, die proportional zur Länge von 
$n$, also zu $\log_b n$ ist. Der Test auf Rest 0 bei Teilung
durch $p$ ist also ein polynomieller Verifizierer mit $c=p$.
\end{beispiel}

\subsection{Klasse NP}
Keine Variante der Turingmaschine macht aus einem Problem,
welches sich in einer Laufzeit $2^{O(n^k)}$ entscheiden
lässt, ein Problem in $P$. Es ist daher zu vermuten,
dass die Sprachen, die nichtdeterministisch in polynomieller
Zeit entschieden werden können, eine wesentlich grössere
Klasse bilden als $P$.

\begin{definition}
\index{Klasse NP}%
Die Klasse der von einer nichtdeterministischen Turingmaschine
in polynomieller Zeit entscheidbaren Sprachen heisst NP.
\end{definition}

\begin{figure}
\begin{center}
\includegraphics{images/lang-5}
\end{center}
\caption{Jede Sprache in P ist auch in NP\label{psubsetnp}}
\end{figure}%

Insbesondere enthält NP alle Sprachen mit einem polynomiellen
Verifizierer. Es ist auch klar, dass $\text{P}\subset\text{NP}$,
siehe Abbildung \ref{psubsetnp}.
Ein offenes Problem ist jedoch, ob $\text{P}\ne \text{NP}$.
Die Bedeutung dieser Frage wird später klar werden.

\subsection{Satisfiability: \textsl{SAT}}
\index{SAT}%
\index{Satisfiability}%
Die Sprache 
\[
\text{\textsl{SAT}}=\{\varphi\;|\;\text{$\varphi$ ist eine erfüllbare logische Formel}\}
\]
ist entscheidbar. Für eine logische Formel $\varphi(x_1,\dots,x_n)$
testet man einfach alle möglichen Belegungen der Variablen $x_1,\dots,x_n$
mit Wahrheitswerten.
Dafür sind $2^n$ Verifikationen notwendig, dies ist also
bestimmt kein Algorithmus in P.

\textsl{SAT} ist aber in NP,
denn wir können einen polynomiellen Verifizierer angeben.
Der Verifizierer verlangt als Lösungszertifikat $c$ eine
Belegung der Variablen mit Wahrheitswerten $c=(x_1,\dots,x_n)$,
und überprüft, ob durch einsetzen der Werte $x_1,\dots,x_n$
die Formel $\varphi(x_1,\dots,x_n)$ wahr wird.

Eine Variante von \textsl{SAT} ist \textsl{3SAT}.
Im Gegensatz zu \textsl{SAT} enthält \textsl{3SAT}
nur Formeln in konjunktiver Normalform, und jede Klausel
enthält genau drei Terme (3cnf-Formel).
Eine typische Formel in \textsl{3SAT} ist 
\[
\varphi=(x_1\vee x_2\vee x_3)\wedge (\bar x_1\vee x_3\vee \bar x_4)\wedge
	(x_1\vee x_3\vee x_5).
\]
Die Sprache \textsl{3SAT} ist
\[
\text{\textsl{3SAT}} =\{\varphi\;|\; \text{$\varphi$ ist eine erfüllbare 3cnf-Formel}\}.
\]
Wie \textsl{SAT} ist \textsl{3SAT} in NP.

\subsection{Existenz von Cliquen: \texorpdfstring{$k$}{k}-\textsl{CLIQUE}}
\begin{figure}
\begin{center}
%\includegraphics[width=0.6\hsize]{images/comp-1}
\includegraphics[width=0.35\hsize]{images/comp-1}
\end{center}
\caption{Graph mit sechs Knoten, keiner 4-Clique, aber
vier 3-Cliquen\label{6graph}}
\end{figure}%
\begin{figure}
\begin{center}
\begin{tabular}{cccc}
\includegraphics[width=0.35\hsize]{images/comp-2}&%
\includegraphics[width=0.35\hsize]{images/comp-3}\\
\includegraphics[width=0.35\hsize]{images/comp-4}&%
\includegraphics[width=0.35\hsize]{images/comp-5}
\end{tabular}
\end{center}
\caption{3-Cliquen des Graphen von Abbildung~\ref{6graph}}
\end{figure}%
\index{Clique}%
\index{Clique@$k$-Clique}%
\index{Cliquen-Problem}%
\index{CLIQUE@\textsl{CLIQUE}}%
Eine $k$-Clique in einem Graph $G$ ist eine Menge von $k$
Ecken des Graphen so, dass in $G$ jede Ecke der Teilmenge mit
jeder anderen Ecke verbunden ist. Im umgangssprachlichen Gebrauch
ist eine Clique eine Gruppe von Leuten, in der jeder jeden kennt.

Das Cliquen-Problem
\[
\text{$k$-\textsl{CLIQUE}} = \{ \langle G\rangle\;|\;
\text{$G$ ist ein Graph mit einer $k$-Clique}\}
\]
ist entscheidbar. Man probiert alle möglichen $k$-elementigen
Teilmengen der Ecken des Graphen durch, ob sie eine Clique
bilden. Da die Zahl der Teilmengen $\binom{n}{k}$ ist, wenn $n$
die Zahl der Ecken ist, und $\binom{n}{k}$ von der Grössenordnung
$O(n^k)$ ist, ist dieser Algorithmus nicht in P.

Das Cliquen-Problem ist aber in NP. Dazu muss wieder ein
Verfizierer angegeben werden. Als Lösungszertifikat verlangt
der Verifizierer die Menge $c=\{v_1,\dots,v_k\}$ der
Vertizes, die angeblich eine $k$-Clique bilden. Dann testet
er jedes Paar von Vertices in $c$ daraufhin, ob sie in $G$
verbunden sind. Dies sind weniger als $k^2$ Tests die nicht
weniger Aufwand als die Grösse des Graphen brauchen, die
Komplexität dieses Algorithmus ist also $O(n)$, das
Cliquen-Problem ist in NP.

\subsection{Färbeproblem: \texorpdfstring{$k$}{k}-\textsl{VERTEX-COLORING}}
\index{Färbeproblem}%
\begin{figure}
\begin{center}
\begin{tabular}{ccc}
\includegraphics{images/comp-6}&
\qquad&\qquad
\includegraphics{images/comp-7}
\end{tabular}
\end{center}
\caption{Zum Färbeproblem: der Graph links kann mit drei Farben
eingefärbt werden, der Graph rechts braucht vier verschiedene Farben
\label{vertex-coloring-examples}}
\end{figure}%
\index{VERTEX-COLORING@\textsl{VERTEX-COLORING}}%
Man sagt, die Vertizes eines Graphen $G$ können mit $k$ Farben
eingefärbt werden, wenn sich für jeden Vertex eine der $k$ Farben
wählen lässt, so dass nie zwei durch eine Kante verbundene Vertizes
die gleiche Farbe bekommen.

Das Färbe-Problem 
\[
\text{$k$-\textsl{VERTEX-COLORING}}
=
\{
\langle G\rangle\;|\;
\text{$G$ ist ein mit $k$ Farben einfärbbarer Graph}\}
\}
\]
ist entscheidbar. Man kann alle $k^n$ möglichen Färbungen
($n$ die Anzahl der Vertizes) durchtesten, ob sie die Bedingung
erfüllen, dass verbundene Vertizes nicht die gleiche Farbe haben
dürfen.

$k$-\textsl{VERTEX-COLORING} ist in NP, denn ein polynomieller Verifizierer
wird als Lösungszertifikat die Farbzuordnung $c=(c_1,\dots,c_n)$ der
Ecken $1,\dots,n$ des Graphen verlangen, und kann damit in polynomieller
Zeit prüfen, ob die Bedingung verschiedener Farben an den Enden
jeder Kante erfüllt ist.

