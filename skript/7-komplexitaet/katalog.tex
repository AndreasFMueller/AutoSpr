%
% katalog.tex -- Katalog NP-vollständiger Probleme
%
% (c) 2011 Prof Dr Andreas Mueller, Hochschule Rapperswil
%
\section{Katalog NP-vollständiger Probleme von Karp}
\rhead{Liste von Karp}
\index{Karp, Richard}%
\index{Karp's Liste}%
Schon kurz nach der Veröffentlichung des Beweises des Satzes
\ref{cooklevin} hat Richard Karp einen Baum von Reduktionen
veröffentlicht, in dem ausgehend von SAT 21 Probleme verzeichnet
waren.
Selbstverständlich lässt
sich jedes dieser Probleme auf jedes andere reduzieren. Zum Beispiel
hatten wir \textsl{3SAT} auf \textsl{CLIQUE} reduziert, während
Karp direkt von \textsl{SAT} ausgeht. Karps Reduktionsbaum
beginnt wie folgt.
\[
\xymatrix{
{}
	&\text{\textsl{SAT}} \ar[dl] \ar[d] \ar[dr]
\\
\text{\textsl{CLIQUE}} \ar[d] \ar[dr]
	&\text{\textsl{BIP}}
		&\text{\textsl{3SAT}} \ar[d]
\\
\text{\textsl{VERTEX-COVER}}
	&\text{\textsl{SET-PACKING}}
		&\text{\textsl{VERTEX-COLORING}}
}
\]
Weiter oben haben wir \textsl{CLIQUE} von \textsl{3SAT} aus
reduziert. \textsl{VERTEX-COLORING} haben wir im Zusammenhang
mit dem Stundenplanproblem getroffen.

\index{BIP@\textsl{BIP}}%
\textsl{BIP} ist ``binary integer programming'', zu einer ganzzahligen
Matrix $C$ und einem ganzzahligen Vektor $d$ ist ein binärer
Vektor $x$ zu finden mit $Cx=d$.

\begin{satz}
\textsl{BIP} ist NP-vollständig.
\end{satz}

\begin{proof}[Beweis]
Es ist klar, dass \textsl{BIP} in NP ist. Es genügt daher, eine
Reduktion
\[
\text{\textsl{SUBSET-SUM}}\le_P\text{\textsl{BIP}}
\]
zu konstruieren.

Dazu muss zu einer Menge $S$ von ganzen Zahlen und einer Summe $t$
eine Matrix und ein Vektor konstruiert werden. Als Matrix nehmen wir
eine Matrix mit einer Zeile, in der die Elemente von $S$ stehen. Als
Vektor $d$ nehmen wir den Vektor mit der einen Komponenten $t$.
Einen binären Vektor $x$ finden mit $Cx=d$ ist jetzt gleichbedeutend
damit, eine Auswahl von Zahlen in $S$ zu finden, deren Summe $t$ ist.
Also ist
$\text{\textsl{SUBSET-SUM}}\le_P\text{\textsl{BIP}}$.
\end{proof}

Im folgenden wollen die von Karp als NP-vollständig erkannten Probleme
ohne Beweis im Sinne einer Referenz zusammenstellen. Hat man ein
Problem als NP-vollständig nachzuweisen, kann man von jedem dieser
Probleme aus eine Reduktion versuchen.

Für die Abhängigkeiten unterhalb von \textsl{VERTEX-COVER} gibt 
Karp folgenden Baum
\[
\xymatrix{
	&\text{\textsl{VERTEX-COVER}} \ar[dl] \ar[d]\ar[dr]\ar[drr]
\\
\begin{minipage}{1.0truein}
\textsl{FEEDBACK-NODE-SET}
\end{minipage}
	&\begin{minipage}{1.0truein}\textsl{FEEDBACK-ARC-SET}\end{minipage}
		&\text{\textsl{HAMPATH}}\ar[d]
			&\text{\textsl{SET-COVERING}}
\\
	&
		&\text{\textsl{UHAMPATH}}
}
\]
\textsl{UHAMPATH} ist das Problem in einem ungerichteten Graphen
einen hamiltonschen Pfad zu finden. Die anderen Probleme sind wie
folgt definiert:
\begin{description}
\index{VERTEX-COVER@\textsl{VERTEX-COVER}}%
\item[\textsl{VERTEX-COVER}:] Gegeben ein Graph $G$ und eine Zahl
$k$, gibt es eine Teilmenge von $k$ Vertizes so, dass jede
Kante des Graphen ein Ende in dieser Teilmenge hat?
\index{FEEDBACK-NODE-SET@\textsl{FEEDBACK-NODE-SET}}%
\item[\textsl{FEEDBACK-NODE-SET}:] Gegeben ein gerichteter Graph $G$
und eine Zahl $k$, gibt es eine endliche Teilmenge von $k$ Vertizes
von $G$ so, dass jeder Zyklus in $G$ einen Vertex in der Teilmenge 
enthält?
\index{FEEDBACK-ARC-SET@\textsl{FEEDBACK-ARC-SET}}%
\item[\textsl{FEEDBACK-ARC-SET}:] Gegeben ein gerichteter Graph $G$
und eine Zahl $k$, gibt es eine Teilmenge von $k$ Kanten so, dass
jeder Zyklus in $G$ eine Kante aus der Teilmenge enthält?
\index{SET-COVERING@\textsl{SET-COVERING}}%
\item[\textsl{SET-COVERING}:] Gegeben eine endliche Familie endlicher
Mengen $(S_j)_{1\le j\le n}$ und eine Zahl $k$, gibt es eine Unterfamilie
bestehend aus $k$ Mengen, die die gleiche Vereinigung hat?
\end{description}
Die Abhängigkeiten unter \textsl{VERTEX-COLORING} sind etwas vielfältiger:
\[
\xymatrix{
	&\begin{minipage}{1truein}
	\textsl{VERTEX-COLORING}
	\end{minipage} \ar[d]\ar[dr]
\\
	&\begin{minipage}{0.6truein}
	\textsl{EXACT-COVER}
	\end{minipage} \ar[dl] \ar[d] \ar[dr] \ar[drr]
		&\begin{minipage}{0.8truein}
		\textsl{CLIQUE-COVER}
		\end{minipage} 
\\
\text{\textsl{3D-MATCHING}}
	&\text{\textsl{SUBSET-SUM}} \ar[dr] \ar[d]
		&\begin{minipage}{0.8truein}
		\textsl{HITTING-SET}
		\end{minipage}
			&\begin{minipage}{0.8truein}
			\textsl{STEINER-TREE}
			\end{minipage}
\\
	&\text{\textsl{SEQUENCING}}
		&\text{\textsl{PARTITION}} \ar[d]
\\
	&
		&\text{\textsl{MAX-CUT}}
}
\]
\begin{description}
\item[\textsl{EXACT-COVER}] Gegeben eine Familie $(S_j)_{1\le j\le n}$
von Teilmengen
einer Menge $U$ gibt es eine Unterfamilie von Mengen, die disjunkt sind,
aber die gleiche Vereinigung haben?
Die Unterfamilie $(S_{j_i})_{1\le i\le m}$ muss also
$S_{j_i}\cap S_{j_k}=\emptyset$ und 
\[
\bigcup_{j=1}^n S_j=\bigcup_{i=1}^mS_{j_i}
\]
erfüllen.
\index{CLIQUE-COVER@\textsl{CLIQUE-COVER}}%
\item[\textsl{CLIQUE-COVER}:] Gegeben ein Graph $G$ und eine positive Zahl
$k$, gibt es $k$ Cliquen so, dass jede Ecke in genau einer der Cliquen ist?
\index{3D-MATCHING@\textsl{3D-MATCHING}}%
\item[\textsl{3D-MATCHING}:] Sei $T$ eine endliche Menge und $U$ eine 
Menge von Tripeln aus $T$: $U\subset T\times T\times T$.
Gibt es eine
Teilmenge $W\subset U$ so, dass $|W|=|T|$ und keine zwei Elemente
von $W$ stimmen in irgendeiner Koordinate überein?
\index{HITTING-SET@\textsl{HITTING-SET}}%
\item[\textsl{HITTING-SET}:] Gegeben eine Menge von Teilmengen $S_i\subset S$,
gibt es eine Menge $H$, die jede Menge in genau einem Punkt trifft, also
$|H\cap S_i|=1\forall i$?
\index{STEINER-TREE@\textsl{STEINER-TREE}}%
\item[\textsl{STEINER-TREE}:]
Gegeben ein Graph $G$, eine Teilmenge $R$ von Vertizes, und eine
Gewichtsfunktion $w\colon E\to\mathbb Z$ und eine postive Zahl $k$, gibt es
einen Baum mit Gewicht $\le k$, dessen Knoten in $R$ enthalten sind?
Das Gewicht des Baumes ist die Summe der Gewichte 
$w(\{u,v\})$ über alle Kanten $\{u,v\}$ im Baum.
\index{SEQUENCING@\textsl{SEQUENCING}}%
\item[\textsl{SEQUENCING}:] Gegeben sei ein Vektor
$(t_1,\dots,t_p)\in\mathbb Z^p$
von Laufzeiten von $p$ Jobs, ein Vektor von spätesten Ausführungszeiten 
$(d_1,\dots,d_p)\in\mathbb Z^p$, einem Strafenvektor 
$(s_1,\dots,s_p)\in\mathbb Z^p$ und eine positive ganze Zahl $k$.
Gibt es eine Permutation $\pi$ der Zahlen $1,\dots,p$, so dass
die Gesamtstrafe für verspätete Ausführung bei der Ausführung der Jobs
in der Reihenfolge $\pi(1),\dots,\pi(p)$ nicht grösser ist als $k$? 
Formal lautet die Bedingung
\[
\sum_{j=1}^p
\vartheta(t_{\pi(1)} +\dots +t_{\pi(j)} - d_{\pi(j)}) s_{\pi(j)} \le k,
\]
darin ist $\vartheta$ die Stufenfunktion definiert durch
\[
\vartheta(x)=\begin{cases}
1&x\ge 0\\
0&x<0.
\end{cases}
\]
\index{PARTITION@\textsl{PARTITION}}%
\item[\textsl{PARTITION}:] Gegeben eine Folge von $s$ ganzen Zahlen
$c_1,c_2,\dots,c_s$, kann man die Indizes $1,2,\dots,s$ in zwei
Teilmengen $I$ und $\bar I$ teilen, so dass die Summe der zugehörigen
$c_i$ identisch ist:
\[
\sum_{i\in I}c_i=\sum_{i\not\in I}c_i
\]
\index{MAX-CUT@\textsl{MAX-CUT}}%
\item[\textsl{MAX-CUT}:] Gegeben ein Graph $G$ mit einer Gewichtsfunktion
$w\colon E\to\mathbb Z$ und eine ganze Zahl $W$.
Gibt es eine Teilmenge
$S$ der Vertizes, so dass das Gesamtgewicht der Kanten, die $S$ mit
seinem Komplement verbinden, mindestens so gross ist wie $W$:
\[
\sum_{\{u,v\}\in E\wedge u\in S\wedge v\not\in S} w(\{u,v\})\ge W
\]
\index{SET-PACKING@\textsl{SET-PACKING}}%
\item[\textsl{SET-PACKING}:] Gegeben eine Familie $(S_i)_{i\in I}$ von
Mengen und eine Zahl $k\in \mathbb N$.
Gibt es eine $k$-elementige Teilfamilie $(S_i)_{i\in J}$ mit $J\subset I$,
d.~h.~$|J|=k$,
derart,
dass die Mengen der Teilfamilie paarweise diskjunkt sind, also 
\[
	S_i \cap S_j = \emptyset \quad\forall i,j\in J \;\text{mit}\; i\ne j?
\]
\end{description}


