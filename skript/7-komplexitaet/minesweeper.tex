%
% minesweeper.tex -- Minesweeper
%
% (c) 2011 Prof Dr Andreas Mueller, Hochschule Rapperswil
%
\section{Schaltungen, Minesweeper und \textsl{SAT}}
\rhead{Minesweeper}
\index{Minesweeper}%
\subsection{Schaltungen}
\begin{figure}
\begin{center}
\includegraphics{images/mine-1}
\end{center}
\caption{Grundgatter AND, OR und NOT\label{gatter}}
\end{figure}%
\begin{figure}
\begin{center}
\includegraphics{images/mine-2}
\end{center}
\caption{Umsetzung der Formel
$(\bar x\wedge y)\vee(x\wedge \bar z)$ mit Gattern aus der
Abbildung~\ref{gatter}\label{gatterformel}}
\end{figure}%
\index{Gatter}%
\index{AND}%
\index{OR}%
\index{NOT}%
In der Computertechnik lernt man, dass sich moderne Computer mit
Hilfe von logischen Grundschaltungen, den Gattern, realisieren lassen.
Abbildung~\ref{gatter} zeigt die Schaltsymbole der Grundgatter AND, OR
und NOT. Wir gehen hier von idealen Gattern aus, deren Ausgänge beliebig
viele Eingänge anderer Gatter treiben können. Mit solchen Gattern
lassen sich beliebig komplexe logische Formeln umsetzen. Zum Beispiel
zeigt Abbildung~\ref{gatterformel}, wie die Formel
$(\bar x\wedge y)\vee(x\wedge \bar z)$ umgesetzt werden kann.

\index{CURCUIT}%
Die Tatsache, dass sich jede Formel in eine äquivalente Schaltung mit Gattern
übersetzen lässt kann man so formulieren: Es gibt eine Sprache
\[
\text{\textsl{CIRCUIT}}=\{
C\;|\; \text{$C$ ist eine Schaltung, deren Ausgang wahr werden kann}\}.
\}
\]
Ausserdem gibt es eine polynomielle Reduktion
$\text{\textsl{SAT}}\le_P\text{\textsl{CIRCUIT}}$. Da wir bereits wissen,
dass \textsl{SAT} NP-vollständig ist, schliessen wir, dass es ein schwieriges
Problem ist, einer logischen Schaltung anzusehen, ob ihr Ausgang überhaupt
je wahr werden kann.

\subsection{Minesweeper}
\index{Minesweeper}%
Beim Spiel Mine-Sweeper, wird dem Spieler von einigen der noch nicht
aufgedeckten Felder die Anzahl benachbarter Fehler angezeigt, unter
denen sich eine Bombe befindet. Der Spieler soll dann nur diejenigen
Felder betreten, unter denen sich keine Bombe versteckt, und alle
Felder markieren, unter denen eine Bombe liegt. Betrachten Sie das
Problem {\it MINE-SWEEPER-CONSISTENCY}, in dem zu einer Belegung der
Felder mit Zahlen (der Anzahl Bomben auf Nachbarfeldern) und eventuell
auch einigen bereits platzierten Bomben entscheiden
werden muss, ob diese Zahlen konsistent sind, ob also Bomben so
auf dem Spielfeld verteilt werden können, dass die Zahlen stimmen.
Es ist ziemlich klar, dass
\[
\text{\textsl{MINE-SWEEPER-CONSISTENCY}}\in \text{NP}.
\]
Als Lösungszertifikat brauchen wir die Verteilung der Bomben.
Zur Verifikation müssen wir dann die Anzahl der Bomben auf den
Nachbarfeldern zählen, was in Laufzeit $O(n^2)$ möglich ist,
wenn $n$ die Länge der längeren Seite des Spielfeldes ist.
Also hat
\textsl{MINE-SWEEPER-CONSISTENCY} einen polynomiellen Verifizierer und
ist damit in NP.

\subsection{Äquivalenz von \textsl{SAT} und \textsl{MINESWEEPER-CONSISTENCY}}
Da \textsl{SAT} NP-vollständig ist, lässt sich jedes NP-Problem auf
\textsl{SAT} reduzieren. Der Beweis des Satzes von Cook-Levin gibt
auch eine Konstruktion, die aber nicht sehr intuitiv ist.
Daher ist die Frage berechtigt, ob man auf etwas intuitivere Art einsehen
kann, dass
\textsl{SAT}
und
\textsl{MINESWEEPER-CONSISTENCY}
äquivalent sind.

Man kann nämlich direkt einsehen, dass \textsl{MINESWEEPER-CONSISTENCY}
ein Ausfüllrätsel ist.
Das Spielfeld muss nämlich mit zwei Zeichen ausgefüllt werden, ``Bombe''
und ``Keine Bombe''.
Es ist die Regel einzuhalten, dass die Zahl der Bomben auf den Nachbarfeldern
mit der Zahl auf dem Feld übereinstimmt.
Ausserdem kann man verifizieren, dass diese Regel als polynomielle Formel
ausgedrückt werden kann.
Also ist \textsl{MINESWEEPER-CONSISTENCY} ein polynomielles
Ausfüllrätsel, und damit polynomiell auf \textsl{SAT} reduzierbar.

Umgekehrt können wir auch eine Reduktion von \textsl{SAT} auf
\textsl{MINESWEEPER-CONSISTENCY} konstruieren, wir tun dies in
zwei Schritten.
Zunächst ist
klar, dass wir jede Formel in eine Schaltung umwandeln können, und
dass dies in polynomieller Zeit geht. Wir haben also auf jeden
Fall eine polynomielle Reduktion $\text{\textsl{SAT}}\le_P\text{\textsl{CIRCUIT}}$.
Insbesondere sind wir fertig, wenn wir eine Schaltung in ein Minesweeper-%
Problem übersetzen können.

Gesucht ist jetzt also eine Reduktion 
\[
\text{\textsl{CIRCUIT}} \le_P \text{\textsl{MINESWEEPER-CONSISTENCY}}.
\]
Aus Abbildung~\ref{gatterformel} können wir sehen, was dazu alles
übersetzt werden können muss. Leitungen, Verzweigungen, und die Grundgatter
müssen alle in geeignete Zahlemuster auf einem Minesweeper-Spielfeld
übersetzt werden, so dass sich genau dann konsistente Bomben
finden lassen, wenn der Ausgang der Schaltung wahr werden kann.

Auf das OR-Gatter kann genau genommen verzichtet werden, denn eine
ODER-Verknüpfung kann man nach den de Morganschen Regeln auch durch
Negationen und UND-Verknüpfungen ausdrücken:
\[
x\vee y = \overline{(\bar x \wedge \bar y)}.
\]
Somit bleibt übrig, dass wir jeden Schaltplan aus AND- und NOT-Gattern
in einen Minesweeper-Spielplan übersetzen können müssen. Dazu gehören
auch Verbindungen zwischen verschiedenen Gattern, und Überkreuzungen von
solchen Verbindungen. Wir brauchen also einen Spielplan, der die Rolle
eines Drahtes in einem Schaltschema übernehmen kann. Eine solchen Spielplan
zeigt Abbildung~\ref{minesweeper-wire}.

\begin{figure}
\begin{center}
\begin{tabular}{cc}
\multicolumn{2}{c}{\includegraphics[width=0.45\hsize]{graphics/wire}}\\
\multicolumn{2}{c}{``Draht'' ohne Signal}\\
&\\
\includegraphics[width=0.45\hsize]{graphics/wire-0}&
\includegraphics[width=0.45\hsize]{graphics/wire-1}\\
Bombenbelegung für logisch ``0''&
Bombenbelegung für logisch ``1''
\end{tabular}
\end{center}
\caption{Minesweeper-``Draht'', Spielplan mit zwei möglichen Bombenbelegungen,
die für die zwei möglichen Zustände stehen können, die entlang des
Drahtes transportiert werden können.\label{minesweeper-wire}}
\end{figure}%

Drähte müssen sich auch überkreuzen, in Abbildung~\ref{minesweeper-cross}
ist ein Spielplan mit zwei sich kreuzenden Drähten gezeigt. Dieses
Kreuz kann zwei verschiedene Inputs vertikal und horizontal haben,
wie in Abbildung~\ref{minesweeper-crosses} dargestellt. Man kann
erkennen, dass sich die Zustände bei der Kreuzung gegenseitig nicht
verändern. Was links als Zustand 0 eingeht, wird auch rechts als Zustand
0 weitergeleitet.
\begin{figure}
\begin{center}
\includegraphics[width=0.6\hsize]{graphics/cross}
\end{center}
\caption{Kreuzung zweier ``Drähte''\label{minesweeper-cross}}
\end{figure}%
\begin{figure}
\begin{center}
\begin{tabular}{|l|c|c|}
\hline
\raisebox{12ex}{$0$}&
\raisebox{0pt}[0.4\hsize][0pt]{%
\includegraphics[width=0.41\hsize]{graphics/cross-11}}&
\includegraphics[width=0.41\hsize]{graphics/cross-10}\\
\hline
\raisebox{12ex}{$1$}&
\raisebox{0pt}[0.4\hsize][0pt]{%
\includegraphics[width=0.41\hsize]{graphics/cross-01}}&
\includegraphics[width=0.41\hsize]{graphics/cross-00}\\
\hline
&\raisebox{0pt}[15pt][7pt]{$0$}&%
\raisebox{0pt}[15pt][7pt]{$1$}\\
\hline
\end{tabular}
\end{center}
\caption{Vier verschiedene mögliche Zustandskombinationen auf
einer Drahtkreuzung\label{minesweeper-crosses}}
\end{figure}%

Das NOT-Gatter muss aus einem Signal das invertierte Signal
machen. Dies schafft der Spielplan in Abbildung~\ref{splitter}.
Der von links eingespeiste Zustand wird rechts invertiert weitergeleitet.
Gleichzeitig wird der eingespeiste Zustand auch unverändert nach
oben und unten abgeleitet, so dass man mit dieser Struktur auch
eine Aufteilung eines Signals in zwei Signale durchführen kann.
\begin{figure}
\begin{center}
\begin{tabular}{|c|c|c|c|}
\hline
\multirow{2}{0.4\hsize}{%
\raisebox{-5ex}[0.4\hsize][0pt]{%
\includegraphics[width=\hsize]{graphics/splitter}%
}}&
\raisebox{11.5ex}{$0$}&
\raisebox{0pt}[0.35\hsize][0pt]{%
\includegraphics[width=0.4\hsize]{graphics/splitter-1}}&
\raisebox{11ex}{$1$}\\
\cline{2-4}
&
\raisebox{11.5ex}{$1$}&
\raisebox{0pt}[0.35\hsize][0pt]{%
\includegraphics[width=0.4\hsize]{graphics/splitter-0}}&
\raisebox{11ex}{$0$}\\
\hline
\end{tabular}
\end{center}
\caption{NOT-Schaltung und Aufspaltung von Zuständen\label{splitter}}
\end{figure}%

Damit bleibt nur noch das AND-Gatter. Dieses ist in Abbildung~\ref{andgate}
dargestellt, und in Abbildung~\ref{andstates} kann man die Funktion des
Gatters verfolgen.

\begin{figure}
\begin{center}
\includegraphics[width=0.55\hsize]{graphics/and}
\end{center}
\caption{Minesweeper-Spielplan für das UND-Gatter\label{andgate}}
\end{figure}%
\begin{figure}
\begin{center}
\begin{tabular}{|c|c|c|c|}
\hline
\raisebox{0pt}[13pt][4pt]{oberer Input}&unterer Input&Resultat&Output\\
\hline
\multirow{2}{10pt}{0}&%
\raisebox{11ex}{$0$}&%
\raisebox{0pt}[0.324\hsize][0pt]{%
\includegraphics[width=0.342\hsize]{graphics/and-00}}&%
\raisebox{11ex}{$0$}%
\\
\cline{2-4}
&\raisebox{11ex}{$1$}&%
\raisebox{0pt}[0.324\hsize][0pt]{%
\includegraphics[width=0.342\hsize]{graphics/and-01}}&%
\raisebox{11ex}{$0$}%
\\
\hline
\multirow{2}{10pt}{1}&%
\raisebox{11ex}{$0$}&%
\raisebox{0pt}[0.324\hsize][0pt]{%
\includegraphics[width=0.342\hsize]{graphics/and-10}}&%
\raisebox{11ex}{$0$}%
\\
\cline{2-4}
&\raisebox{11ex}{$1$}&%
\raisebox{0pt}[0.324\hsize][0pt]{%
\includegraphics[width=0.342\hsize]{graphics/and-11}}&%
\raisebox{11ex}{$1$}%
\\
\hline
\end{tabular}
\end{center}
\caption{Nachweis der konsistenten ``Funktion'' des AND-Gatters
\label{andstates}}
\end{figure}%

Damit ist gezeigt, dass sich jede Schaltung in polynomieller Zeit in 
einen Minesweeper-Plan übersetzen lässt. Für eine polynomielle
Reduktion wird aber verlangt, dass eine erfüllbare Formel auf einen
Plan abgebildet wird, der am Ausgang eine logische 1 haben.
Da in unseren ``Drähten'' eine logische 1 einer Bombe im zweiten
Feld (in Fortpflanzungsrichtung des Signals) entspricht, können
wir der Forderung nach einer logischen 1 dadurch Nachdruck verleihen,
dass wir diese letzte Bombe bereits platzieren. Als letzten Schritt
in der Übersetzung pflanzen wir also im ``letzten'' Feld eine Bombe.
Der so konstruierte Spielplan hat dann genau dann eine konsistente
Bombenbelegung, wenn der ursprüngliche Schaltplan Output 1 haben kann.

Damit ist jetzt gezeigt, dass
\[
\text{\textsl{SAT}}\le_P
\text{\textsl{CIRCUIT}}\le_P\text{\textsl{MINESWEEPER-CONSISTENCY}},
\]
und weil \textsl{SAT} NP-vollständig ist, folgt auch, dass 
\textsl{MINESWEEPER-CONSISTENCY} NP-vollständig ist.

\subsection{Direkter Beweis}
Unter Verwendung der eben entwickelten Ideen könnte es möglich sein,
einen direkten Beweis der NP-Vollständigkeit von
\textsl{MINESWEEPER-CONSISTENCY} zu geben, ohne die Verwendung
von \textsl{SAT}. Dazu braucht man eine Theorie die besagt, dass
Schaltungen und Turing-Maschinen im wesentlichen Äquivalent sind.
So eine Theorie existiert, und wird auch benötigt, um Turing-Maschinen
mit Quantencomputern vergleichen zu können. Letztere sind nämlich
über ``Quantenschaltungen'' definiert. Dann sagt die Maschinerie des
vorangegangenen Abschnitts jedoch, dass man jede Turing-Maschine in
ein Minesweeper-Problem übersetzen kann, was NP-Vollständigkeit
beweist.
