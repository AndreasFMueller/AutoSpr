\section{Beispiele}
In diesem Abschnitt werden zu den NP-vollst"andigen Problemen des
Katalogs von Karp einzelne Beispiele oder Illustrationen geben.
\subsection{VERTEX-COVER}
Diktator Revoc Xetrev unterdr"uckt sein Volk gnadenlos.
Schon l"anger l"asst er von der staatlichen Telefongesellschaft
erheben, welche Telefonanschl"usse genau miteinander telefonieren.
Damit lassen sich hervorragend Leute als Regimefeinde entlarven.
Nachdem allerdings eine ganze Reihe von hohen Milit"ars als
Landesverr"ater verurteilt und erschossen worden waren,
stellte sich heraus, dass eine Gruppe von
inzwischen ebenfalls verurteilten Regimekritikern sich genau dies
zu Nutze machten: sie riefen die Milit"ars gegen deren Willen an.
Da es keine Aufzeichnung "uber den Inhalt der Gespr"ache gab, konnten
die Milit"ars ihre Unschuld nicht beweisen, mit fatalen Folgen f"ur
ihre Karriere.

Diese Aff"are hatte das Milit"ar einige der f"ahigsten K"opfe beraubt,
so dass eine zuverl"assigere Methode gefunden werden musste, um
Regimefeinde zu erkennen.
Wenn sich damit gegen"uber dem Ausland auch noch ein Anschein von
Rechtssystem etablieren liess, w"urde das Diktator Xetrev den Zugang
zu westlichen Waffenlieferanten etwas erleichtern.

Aus technischen Gr"unden ist eine l"uckenlose "Uberwachung aller Anschl"usse
nicht m"oglich.
Die Software kann mit maximal $k$ Anschl"ussen umgehen, die abgeh"ort
werden k"onnen.
Diktator Xetrev befiehlt daher, dass innert einer Woche eine Menge
von $k$ Anschl"ussen definiert werden m"usse, so dass jedes
Telefongespr"ach abgeh"ort werden kann.

Diese Ank"undigung verursacht erst einmal eine neue Fluchtwelle von
Mathematikern und Informatikern.
Nach einer Woche wird der erste Experte wegen Befehlsverweigerung
erschossen.
Im Prozess, der nur 15 Minuten gedauert hattee, hatte er feige behauptet,
es sei nicht m"oglich, in der kurzen
Zeit eine entsprechende Menge zu finden. Ja es sei nicht einmal
m"oglich, zu entscheiden, ob es eine solche Menge "uberhaupt g"abe.
Der Staatsanwalt legte ihm das als Sabotage der Pl"ane des grossen
Diktators Xetrev aus und machte kurzen Prozess mit ihm.

Als aber nach einer weiteren Woche erneut ein Saboteur angeklagt wurde,
der genau die gleiche Verteidigungstrategie verwendete, obwohl
doch mittlerweile bekannt war, dass sie nicht zum Erfolg f"uhren konnte,
begann sich Diktator Xetrev zu fragen, ob darin vielleicht
ein K"ornchen Wahrheit stecken k"onnte. K"onnte es sein, dass 
es tats"achlich zu schwierig ist, eine solche Anschlussauswahl zu treffen?

\medskip

Das Problem, welches die Experten l"osen sollten, ist das Problem
{\it VERTEX-COVER}. Die Telefonanschl"usse bilden die Knoten eines Graphen,
die Kanten verbinden diejenigen Anschl"usse, die tats"achlich miteinander
telefonieren. Gesucht ist eine Menge von $k$ Knoten so, dass jedes
m"ogliche Gespr"ach abgeh"ort werden kann, also jede Kante in einem
ausgew"ahlten Knoten endet.


%\subsection{FEEDBACK-NODE-SET}
%\subsection{FEEDBACK-ARC-SET}
\subsection{SET-COVERING}
Der populistische Politiker Tesco Vering versprach w"ahrend des
Wahlkampfes, das Dickicht von Steuererleichterungen, welches
f"ur jeden B"urger das Ausf"ullen der Steuererkl"arung zur Qual
machte, zu vereinfachen.
Nach seinem Amtsantritt als Ministerpr"asident 
stellte er jedoch fest, dass jeder seiner W"ahler von irgend einer
der Verg"unstigungen profitierte. Um seine Wiederwahl nicht
zu gef"ahrden, beauftragte er daher eine Komission eine Liste
von Verg"unstigungen aufzustellen, so dass jeder seiner W"ahler
immer noch von einer Verg"unstigung profitiert. Niemand soll im
Wahlkampf sagen k"onnen: ``Vering hat mir all Verg"unstigungen weggenommen!''
Kurz vor Beginn des n"achsten Wahlkampfs war die Komission jedoch
immer noch nicht fertig, so dass sich Vering als neues Wahlkampfthema
auf den Kampf gegen die allgemeine Unf"ahigkeit der Veraltung verlegte, um davon
abzulenken, dass er das letzte Wahlversprechen nicht eingel"ost hatte.
Trotzdem wurde er nicht wiedergew"ahlt. Eine detaillierte Analyse der
Wahlresultate ergab, dass Vering vor allem die Stimmen der Mathematiker und
Informatiker gefehlt hatten.
Sie warfen ihm Unverst"andnis f"ur die Komplexit"at seiner Aufgabe vor und 
und hielten ihn deshalb f"ur nicht tauglich f"ur ein politisches Amt.

\medskip

Tats"achlich ist das Programm von Tesco Vering das Problem {\it SET-COVERING}.
Die Steuerverg"unstigungen sind numeriert von $1$ bis $n$.
Sei $S_i$ die Menge der W"ahler von Vering, die von Verg"unstigung $i$
profitieren. Der Auftrag der Komission war, eine Teilmenge
$I=\{i_1,i_2,\dots,i_k\}$ zu finden, so dass
\[
\bigcup_{i=1}^nS_i=\bigcup_{i\in I}S_i
\]
ist.
Dieses Problem ist NP-vollst"andig, nach aktuellem Wissen gibt
es also keinen polynomiellen Algorithmus um zu entscheiden, ob es
"uberhaupt eine L"osung des Problems gibt.

%\subsection{EXACT-COVER}
%\subsection{CLIQUE-COVER}
\subsection{3D-MATCHING}
In der Provinz Gnichtam De herscht grosse Wohnungsnot.
Eine Analyse durch das Innenministerium ergab, dass die Ursache
die vielen Singles sind, die jeweils alleine eine gem"ass
Plan der staatlichen Wohnungsbaubeh"orde f"ur Familien vorgesehene
Wohnung belegen.
W"urde man die Singles paarweise den Wohnungen zuteilen, w"urden die
Wohnungen reichen.

Daher beschliesst der Familienminister ein Programm zur Zwangsverheiratung
der Singles und beautragt seine Beh"orde mit der Ausarbeitung einer Zuordnung,
wer mit wem verheiratet werden soll und wo das frisch verheiratete Paar
wohnen soll.
Nat"urlich ist nicht jede Kombination aus einem Mann,
einer Frau und einer Wohnung akzeptabel, immerhin m"ussen 
Arbeitswege vern"unftig bleiben, die B"urger sollen ja als 
Arbeiter weiterhin produktiv bleiben um die hohen Steuern bezahlen zu k"onnen.
Das Ministerium beginnt also ein Liste von m"oglichen Zwangsfamilien
zu erstellen.

Nach einem Jahr wird der Familienminister ungeduldig, denn obwohl die
genannte Liste nach wenigen Wochen fertig war, konnte daraus immer noch
kein geeignete Familienplanung abgeleitet werden, die Fachleute wissen
noch nicht einmal, ob es "uberhaupt m"oglich ist, eine Planung
zu finden, welche alle geforderten Rahmenbedingungen erf"ullt.

\medskip

Das Problem, welches das Familienministerium der Provinz Gnichtam De
l"osen soll, ist {\it 3D-MATCHING}. Es gibt $n$ unverheiratete Frauen
und M"anner, und $n$ m"ogliche Wohnungen. Die urspr"unglich erstellte
Liste ist eine Menge von Tripeln $(x,y,z)$, wobei die Zahlen $x$,
$y$ und $z$ aus der Menge $T=[n]=\{1,2,3,\dots,n\}$ stammen. Aus dieser
Teilmenge $U\subset T\times T\times T$ soll jetzt eine 
Teilmenge $W\subset U$ von genau $n$ Tripeln ausgew"ahlt werden, so dass
jeder Mann, jede Frau und jede Wohnung in genau einem Tripel vorkommt.
Dieses Problem ist NP-vollst"andig, 
nach heutigem Wissen gibt es keinen polynomiellen Algorithmus,
mit dem entschieden werden k"onnte, ob es "uberhaupt so eine Teilmenge
$W$ gibt.

%\subsection{SUBSET-SUM}
%\subsection{HTTING-SET}
%\subsection{STEINER-TREE}
%\subsection{SEQUENCING}
\subsection{PARTITION}
Ein reicher K"onig ist gestorben, und hinterl"asst seinen beiden
S"ohnen, eineiigen Zwillingen, ein grosses Verm"ogen aus Schl"ossern
und Burgen. Die Berater des K"onigs werden zusammengerufen um
dar"uber zu entscheiden, wie die G"uter gerecht an die beiden S"ohne
verteilt werden k"onnen. Warum sind die Berater auch nach einem Jahr
noch nicht zur Einsicht gelangt, ob eine gerechte Teilung "uberhaupt
m"oglich ist?

Bezeichnen wir mit $c_i$ den Wert der Immobilie $i$, dann besteht
die Aufgabe der Berater darin eine Aufteilung der Menge $A$ aller 
zur Teilung stehenden Immobilien in zwei Mengen $I$ und $A\setminus I$
zu vollziehen, die Immobilien jeder Menge jeweils einem der S"ohne zugeteilt
werden sollen.
Nat"urlich sollen die beiden Mengen gleichen Gesamtwert haben.
Der Gesamtwert ist jeweils die Summe der einzelnen Werte, gesucht ist also
$I\subset A$ so dass
\[
\sum_{i\in I}c_i =\sum_{i\in A\setminus I}c_i
\]
Diese ist das Problem {\it PARTITION}, es ist NP-vollst"andig.
Daher gibt es nach aktuellem Wissen keinen polynomiellen Algorithmus,
der es entscheiden k"onnte.


%\subsection{MAX-CUT}
