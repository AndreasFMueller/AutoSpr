\section{Beispiele}
In diesem Abschnitt werden zu den NP-vollst"andigen Problemen des
Katalogs von Karp einzelne Beispiele oder Illustrationen geben.
\subsection{VERTEX-COVER}
Diktator Revoc Xetrev unterdr"uckt sein Volk gnadenlos.
Schon l"anger l"asst er von der staatlichen Telefongesellschaft
erheben, welche Telefonanschl"usse genau miteinander telefonieren.
Damit lassen sich hervorragend Leute als Regimefeinde entlarven.
Nachdem allerdings eine ganze Reihe von hohen Milit"ars als
Landesverr"ater verurteilt und erschossen worden waren,
stellte sich heraus, dass eine Gruppe von
inzwischen ebenfalls verurteilten Regimekritikern sich genau dies
zu Nutze machten: sie riefen die Milit"ars gegen deren Willen an.
Da es keine Aufzeichnung "uber den Inhalt der Gespr"ache gab, konnten
die Milit"ars ihre Unschuld nicht beweisen, mit fatalen Folgen f"ur
ihre Karriere.

Diese Aff"are hatte das Milit"ar einige der f"ahigsten K"opfe beraubt,
so dass eine zuverl"assigere Methode gefunden werden musste, um
Regimefeinde zu erkennen.
Wenn sich damit gegen"uber dem Ausland auch noch ein Anschein von
Rechtssystem etablieren liess, w"urde das Diktator Xetrev den Zugang
zu westlichen Waffenlieferanten etwas erleichtern.

Aus technischen Gr"unden ist eine l"uckenlose "Uberwachung aller Anschl"usse
nicht m"oglich.
Die Software kann mit maximal $k$ Anschl"ussen umgehen, die abgeh"ort
werden k"onnen.
Diktator Xetrev befiehlt daher, dass innert einer Woche eine Menge
von $k$ Anschl"ussen definiert werden m"usse, so dass jedes
Telefongespr"ach abgeh"ort werden kann.

Diese Ank"undigung verursacht erst einmal eine neue Fluchtwelle von
Mathematikern und Informatikern.
Nach einer Woche wird der erste Experte wegen Befehlsverweigerung
erschossen.
Im Prozess, der nur 15 Minuten gedauert hattee, hatte er feige behauptet,
es sei nicht m"oglich, in der kurzen
Zeit eine entsprechende Menge zu finden. Ja es sei nicht einmal
m"oglich, zu entscheiden, ob es eine solche Menge "uberhaupt g"abe.
Der Staatsanwalt legte ihm das als Sabotage der Pl"ane des grossen
Diktators Xetrev aus und machte kurzen Prozess mit ihm.

Als aber nach einer weiteren Woche erneut ein Saboteur angeklagt wurde,
der genau die gleiche Verteidigungstrategie verwendete, obwohl
doch mittlerweile bekannt war, dass sie nicht zum Erfolg f"uhren konnte,
begann sich Diktator Xetrev zu fragen, ob darin vielleicht
ein K"ornchen Wahrheit stecken k"onnte. K"onnte es sein, dass 
es tats"achlich zu schwierig ist, eine solche Anschlussauswahl zu treffen?

\medskip

Das Problem, welches die Experten l"osen sollten, ist das Problem
{\it VERTEX-COVER}. Die Telefonanschl"usse bilden die Knoten eines Graphen,
die Kanten verbinden diejenigen Anschl"usse, die tats"achlich miteinander
telefonieren. Gesucht ist eine Menge von $k$ Knoten so, dass jedes
m"ogliche Gespr"ach abgeh"ort werden kann, also jede Kante in einem
ausgew"ahlten Knoten endet.


\subsection{FEEDBACK-NODE-SET}
In einer grossen Fabrik bringen Transportroboter die Teile zu den
Verarbeitungsstationen.  Die Roboter fahren dabei entlang vorgegebener
Bahnen, die sie mit Hilfe von Bodenmarkierungen einhalten.  Die Roboter
sind immer f"ur die gleichen Prozesse im Einsatz und fahren daher mehrmals
am Tag immer die gleichen Runden ab.

Die Wartungsvorschriften verlangen, dass die Roboter t"aglich einmal einem
Sicherheitscheck unterzogen werden, der nur wenige Minuten dauert. Wegen
des weitl"aufigen Fabrikgel"andes lohnt es sich nicht, alle Roboter jeden
Tag an eine zentrale Kontrollstelle zu fahren, wo sie "uberpr"uft werden
k"onnen. Sie w"urden viel zu lange ausfallen. Daher wurde entschieden, an
einigen Stellen des Netzes der Roboter-Fahrwege Pr"ufstellen einzurichten.

Sachbearbeiter Heiri Muster soll herausfinden, ob das Budget reicht,
um gen"ugend Pr"ufstellen einzurichten, so dass jeder Roboter t"aglich
gepr"uft werden kann.
Drei Monate sp"ater wird er wegen Burnout freigestellt.

\medskip

Das Budget reicht f"ur die Ausr"ustung von $k$ Pr"ufstellen.
Die Roboter bewegen sich auf dem gerichteten Graphen der Bodenmarkierungen.
Sie fahren die Zyklen des Graphen ab. Gesucht wird jetzt eine Menge
von $k$ Netzknoten f"ur die Pr"ufstellen, so dass jeder Zyklus des Graphen
durch eine dieser Pr"ufstellen verl"auft.
Dies ist das NP-vollst"andige Problem {\it FEEDBACK-NODE-SET}.
Nach aktuellem Wissen gibt es keinen Algorithmus mit polynomieller
Laufzeit, der entscheiden k"onnte, ob das Problem "uberhaupt eine
L"osung hat.

\subsection{FEEDBACK-ARC-SET}
Kurz nachdem die Pl"ane f"ur das Pr"ufstellennetz aus dem vorangegangenen
Beispiel von der Gesch"aftsletiung bewilligt worden war, "anderte
der Hersteller der Roboter bei einem Softwareupdate die Wartungsvorschriften.
Die Pr"ufung erfolgt jetzt nicht mehr im Stehen, sondern w"ahrend der
Fahrt.
Durch diese dynamische Pr"ufung werden Problem erkennbar, die bei der
statischen Pr"ufung nicht gefunden werden konnten. Heiri Musters
Nachfolger wird beauftragt, die Pl"ane entsprechend anzupassen.
Drei Monate sp"ater ist auch er wegen Burnout freigestellt.

\medskip

Das Problem wurde durch die Umstellung zu einer Instanz des
Problems {\it FEEDBACK-ARC-SET} modifiziert.
Es m"ussen $k$ Kanten gefunden werden, so dass jeder Zyklus eine
diese Kanten durchl"auft. Auch dieses Problem ist NP-vollst"andig,
und d"urfte daher f"ur ein kompliziertes Streckennetz
nicht nur einen Sachbearbeiter "uberfordern.

\subsection{SET-COVERING}
Der populistische Politiker Tesco Vering versprach w"ahrend des
Wahlkampfes, das Dickicht von Steuererleichterungen, welches
f"ur jeden B"urger das Ausf"ullen der Steuererkl"arung zur Qual
machte, zu vereinfachen.
Nach seinem Amtsantritt als Ministerpr"asident 
stellte er jedoch fest, dass jeder seiner W"ahler von irgend einer
der Verg"unstigungen profitierte. Um seine Wiederwahl nicht
zu gef"ahrden, beauftragte er daher eine Komission eine Liste
von Verg"unstigungen aufzustellen, so dass jeder seiner W"ahler
immer noch von einer Verg"unstigung profitiert. Niemand soll im
Wahlkampf sagen k"onnen: ``Vering hat mir all Verg"unstigungen weggenommen!''
Kurz vor Beginn des n"achsten Wahlkampfs war die Komission jedoch
immer noch nicht fertig, so dass sich Vering als neues Wahlkampfthema
auf den Kampf gegen die allgemeine Unf"ahigkeit der Veraltung verlegte, um davon
abzulenken, dass er das letzte Wahlversprechen nicht eingel"ost hatte.
Trotzdem wurde er nicht wiedergew"ahlt. Eine detaillierte Analyse der
Wahlresultate ergab, dass Vering vor allem die Stimmen der Mathematiker und
Informatiker gefehlt hatten.
Sie warfen ihm Unverst"andnis f"ur die Komplexit"at seiner Aufgabe vor und 
und hielten ihn deshalb f"ur nicht tauglich f"ur ein politisches Amt.

\medskip

Tats"achlich ist das Programm von Tesco Vering das Problem {\it SET-COVERING}.
Die Steuerverg"unstigungen sind numeriert von $1$ bis $n$.
Sei $S_i$ die Menge der W"ahler von Vering, die von Verg"unstigung $i$
profitieren. Der Auftrag der Komission war, eine Teilmenge
$I=\{i_1,i_2,\dots,i_k\}$ zu finden, so dass
\[
\bigcup_{i=1}^nS_i=\bigcup_{i\in I}S_i
\]
ist.
Dieses Problem ist NP-vollst"andig, nach aktuellem Wissen gibt
es also keinen polynomiellen Algorithmus um zu entscheiden, ob es
"uberhaupt eine L"osung des Problems gibt.

\subsection{EXACT-COVER}
In einem Straflager m"ussen Str"aflinge in diversen Bauprojekten
der Umgebung arbeiten.
Viele der Str"aflinge sind jedoch eher unbegabt und daher kaum
f"ahig, diese Arbeiten zur Zufriedenheit der Lagerleitung auszuf"uhren.
Jeder Str"afling ist f"ur mindestens eines der Projekte nicht geeignet.
Der sadistische Lagerkommandant ersinnt sich dazu noch einen teuflischen
Plan, wie er m"oglichst viele Str"aflinge wegen Sabotage bestrafen kann.
Er gibt seinen Schergen folgende Regeln f"ur die Zuteilung der Str"aflinge
zu den Projekten: Jeder Str"afling darf nur zu Projekten aufgeboten werden,
f"ur die er nicht geeignet ist, und jeder Str"afling muss besch"aftigt
werden.
Im Lagerapport am n"achsten Tag r"ucken die Aufseher damit heraus, dass
sie keine Zuteilung gefunden h"atten, die diesen Regeln entspricht, und
dass deshalb an diesem Tag nichts gearbeitet worden sei.
Der Lagerkommandant sch"aumt vor Wut und droht damit die Aufseher eigenh"andig
zu erschiessen, wenn sie sich weiterhin seinen Befehlen widersetzen
w"urden.

\medskip

Der Lagerkommandant hat seinen Aufsehern ein Problem gestellt, welches
"aquivalent zu {\it EXACT-COVER} ist.
Die Menge aller Str"aflinge ist $U$.
Die Projekte seien numeriert
von $1$ bis $n$. Die Menge $S_i\subset U$ besteht aus den Str"aflingen, die
f"ur das Projekt $i$ ungeeignet sind. Weil jeder Str"afling f"ur mindestens
ein Projekt ungeeignet ist, ist 
\[
U=\bigcup_{i=1}^n S_i.
\]
Gesucht werden jetzt Teilmenge von Projekten
$I=\{i_1,\dots,i_m\}\subset\{1,\dots,n\}$, so dass 
jeder Str"afling auf einem Projekt arbeitet:
\[
\bigcup_{j=1}^m S_{i_j}=U
\]
aber kein Str"afling mehr als einem Projekt zugeteilt wird:
\[
S_{i_j}\cap S_{i_k}=\emptyset\qquad \forall j\ne k.
\]
Dieses Problem ist NP-vollst"andig, nach aktuellem Wissen gibt es
also keinen polynomiellen Algorithmus, mit dem entschieden werden
k"onnte, ob es "uberhaupt eine L"osung hat.

\subsection{CLIQUE-COVER}
In einem Online-Spiel treten die Spieler jeweils paarweise gegeneinander an.
Eine zuverl"assige Rangliste der Spieler kann nur innerhalb
einer Gruppe erstellt werden, in der jeder Spieler gegen
jeden anderen gespielt hat. Trotzdem soll jetzt versucht werden,
ein Gesamt-Ranking zu erstellen. Grundlage daf"ur ist die Auswahl
einer Menge von Gruppen, in der jeder gegen jeden gespielt hat.
Die Auswahl soll m"oglichst klein sein, denn man hofft so, die
Schnittmengen, in denen systembedingt am ehesten Ranking-Widerspr"uche
auftreten k"onnten, m"oglichst klein zu halten. Allerdings stellt es
sich als schwierig heraus, eine solche Auswahl zu treffen, warum?

\medskip

Die Aufgabe verlangt, in dem Graphen bestehend aus den Spielern
als Knoten und Kanten, die angeben, ob die zwei Spieler schon gegeneinander
gespielt hat, eine "Uberdeckung mit einer m"oglichst
kleinen Anzahl Cliquen zu finden. Dies ist das NP-vollst"andige Problem
{\it CLIQUE-COVER}. Nach aktuellem Wissen gibt es daf"ur keinen
Algorithmus mit polynomieller Laufzeit.

\subsection{3D-MATCHING}
In der Provinz Gnichtam De herscht grosse Wohnungsnot.
Eine Analyse durch das Innenministerium ergab, dass die Ursache
die vielen Singles sind, die jeweils alleine eine gem"ass
Plan der staatlichen Wohnungsbaubeh"orde f"ur Familien vorgesehene
Wohnung belegen.
W"urde man die Singles paarweise den Wohnungen zuteilen, w"urden die
Wohnungen reichen.

Daher beschliesst der Familienminister ein Programm zur Zwangsverheiratung
der Singles und beautragt seine Beh"orde mit der Ausarbeitung einer Zuordnung,
wer mit wem verheiratet werden soll und wo das frisch verheiratete Paar
wohnen soll.
Nat"urlich ist nicht jede Kombination aus einem Mann,
einer Frau und einer Wohnung akzeptabel, immerhin m"ussen 
Arbeitswege vern"unftig bleiben, die B"urger sollen ja als 
Arbeiter weiterhin produktiv bleiben um die hohen Steuern bezahlen zu k"onnen.
Das Ministerium beginnt also ein Liste von m"oglichen Zwangsfamilien
zu erstellen.

Nach einem Jahr wird der Familienminister ungeduldig, denn obwohl die
genannte Liste nach wenigen Wochen fertig war, konnte daraus immer noch
kein geeignete Familienplanung abgeleitet werden, die Fachleute wissen
noch nicht einmal, ob es "uberhaupt m"oglich ist, eine Planung
zu finden, welche alle geforderten Rahmenbedingungen erf"ullt.

\medskip

Das Problem, welches das Familienministerium der Provinz Gnichtam De
l"osen soll, ist {\it 3D-MATCHING}. Es gibt $n$ unverheiratete Frauen
und M"anner, und $n$ m"ogliche Wohnungen. Die urspr"unglich erstellte
Liste ist eine Menge von Tripeln $(x,y,z)$, wobei die Zahlen $x$,
$y$ und $z$ aus der Menge $T=[n]=\{1,2,3,\dots,n\}$ stammen. Aus dieser
Teilmenge $U\subset T\times T\times T$ soll jetzt eine 
Teilmenge $W\subset U$ von genau $n$ Tripeln ausgew"ahlt werden, so dass
jeder Mann, jede Frau und jede Wohnung in genau einem Tripel vorkommt.
Dieses Problem ist NP-vollst"andig, 
nach heutigem Wissen gibt es keinen polynomiellen Algorithmus,
mit dem entschieden werden k"onnte, ob es "uberhaupt so eine Teilmenge
$W$ gibt.

\subsection{HITTING-SET}
An einer Hochschule soll geplant werden, welche Dozenten welche
F"acher unterrichten sollen.
Die meisten Dozenten sind kompetent, mehr als eines der F"acher
zu unterrichten, und es ist bekannt, dass die vorhandenen Kompetenzen
unter den Dozenten alle F"acher abdecken.
Warum dauert es trotzdem so lange, eine F"acherzuteilung zu finden? 

\medskip

Das Problem der F"acherzuteilung ist "aquivalent zum Problem {\it HITTING-SET}.
Zu jedem Fach $i$ gibt es eine Menge $S_i$ von Dozenten, die das Fach
unterrichten k"onnten.
Gesucht ist eine Menge $H$ von Dozenten, so dass jedes Fach genau
einen Dozenten erh"alt, also $H\cap S_i$ genau ein Element enth"alt.

Dieses Problem ist NP-vollst"andig, nach aktuellem Wissen gibt es
keinen Algorithmus mit polynomieller Laufzeit, der entscheiden
k"onnte, ob es eine Aufgabenverteilung gibt.

\subsection{STEINER-TREE}
Eine Bank hat bis anhin ihre Filialen immer vom Hauptsitz aus mit
Geld versorgt. Die daf"ur n"otigen Hochsicherheitstransporte sind
jedoch teuer, daher soll ein neues Transportmodell gefunden werden.
Neu sollen auch Transporte zwischen Filialen m"oglich sein.
Am Hauptsitz werden dazu die Betr"age f"ur mehrere Filialen
verladen, die Filialen m"ussen die Lieferungen
auseinander nehmen und auf neue Transporte verladen. Das
Management will wissen, ob sich mit dieser Methode die
Kosten unter den Betrag $k$ senken lassen. 

\medskip

F"ur jedes Paar von Filialen sind die Kosten eines Transportes
bekannt. Der billigste Logistikplan wird einen Baum verwenden.
W"are das gew"ahlte Verfahren nicht ein Baum, k"onnte man
durch Umverteilung eines Teils des Geldes n"amlich einen
Transport weglassen, wodurch die Kosten geringer w"urden.
Es liegt also eine Instanz des Problems {\it STEINER-TREE}
vor, welches NP-vollst"andig ist und daher nach allem, was
wir wissen, nicht von einem Algorithmus mit polynomieller
Laufzeit gel"ost werden kann.

\subsection{SEQUENCING}


\subsection{SUBSET-SUM}
Kurz vor Jahresende in einer grossen Software-Firma: Wie jedes Jahr stellt
der Gruppenleiter Software-Entwicklung fest, dass noch nicht sein ganzes
Budget verbraucht ist.  Er entscheidet, dass diesmal der gesamt Restbetrag
bis auf den letzten Rappen
ausgegeben werden soll.  Daher sammelt der Gruppenleiter von seinem
Team Vorschl"age, was mit dem verbleibenden Geld gemacht werden k"onnte,
und gibt die Liste seiner Sekret"arin.  Sie soll daraus so einige Dinge
ausw"ahlen, dass genau das Restbudget verbraucht wird. Einige Tag sp"ater
wundert er sich, dass die Sekret"arin total "uberfordert ist. Ist er
am richtigen Ort?

\medskip

Nein, der Gruppenleiter h"atte erkennen m"ussen, das er der Sekret"arin
ein NP-vollst"andiges Problem gestellt hat, wof"ur es nach aktuellem
Wissen keinen Algorithmus mit polynomieller Laufzeit gibt. Die Vorschl"age
des Teams bilden eine Menge $\{b_1,b_2,\dots,b_n\}$ von Betr"agen $b_i$.
Daraus soll ein Teilmenge $I=\{i_1,\dots,i_m\}$ gebildet werden, so dass
so dass der Restbetrag $r$ dadurch aufgebraucht wurde:
\[
\sum_{k=1}^m b_{i_k} = r.
\]
Dies ist das Problem {\it SUBSET-SUM}.

\subsection{PARTITION}
Ein reicher K"onig ist gestorben, und hinterl"asst seinen beiden
S"ohnen, eineiigen Zwillingen, ein grosses Verm"ogen aus Schl"ossern
und Burgen. Die Berater des K"onigs werden zusammengerufen um
dar"uber zu entscheiden, wie die G"uter gerecht an die beiden S"ohne
verteilt werden k"onnen. Warum sind die Berater auch nach einem Jahr
noch nicht zur Einsicht gelangt, ob eine gerechte Teilung "uberhaupt
m"oglich ist?

Bezeichnen wir mit $c_i$ den Wert der Immobilie $i$, dann besteht
die Aufgabe der Berater darin eine Aufteilung der Menge $A$ aller 
zur Teilung stehenden Immobilien in zwei Mengen $I$ und $A\setminus I$
zu vollziehen, die Immobilien jeder Menge jeweils einem der S"ohne zugeteilt
werden sollen.
Nat"urlich sollen die beiden Mengen gleichen Gesamtwert haben.
Der Gesamtwert ist jeweils die Summe der einzelnen Werte, gesucht ist also
$I\subset A$ so dass
\[
\sum_{i\in I}c_i =\sum_{i\in A\setminus I}c_i
\]
Diese ist das Problem {\it PARTITION}, es ist NP-vollst"andig.
Daher gibt es nach aktuellem Wissen keinen polynomiellen Algorithmus,
der es entscheiden k"onnte.


\subsection{MAX-CUT}
