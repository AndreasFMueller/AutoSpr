\section{Beispiele}
In diesem Abschnitt werden zu den NP-vollst"andigen Problemen des
Katalogs von Karp einzelne Beispiele oder Illustrationen geben.
%\subsection{VERTEX-COVER}
%\subsection{FEEDBACK-NODE-SET}
%\subsection{FEEDBACK-ARC-SET}
%\subsection{SET-COVERING}
%\subsection{EXACT-COVER}
%\subsection{CLIQUE-COVER}
%\subsection{3D-MATCHING}
%\subsection{SUBSET-SUM}
%\subsection{HTTING-SET}
%\subsection{STEINER-TREE}
%\subsection{SEQUENCING}
\subsection{PARTITION}
Aufgabe:
Ein reicher K"onig ist gestorben, und hinterl"asst seinen beiden
S"ohnen, eineiige Zwillinge, ein grosses Verm"ogen aus Schl"ossern
und Burgen. Die Berater des K"onigs werden zusammengerufen um
dar"uber zu entscheiden, wie die G"uter gerecht an die beiden S"ohne
verteilt werden sollen. Warum sind die Berater auch nach einem Jahr
noch nicht zur Einsicht gelangt, ob die gerechte Teilung "uberhaupt
m"oglich ist?

Bezeichnen wir mit $c_i$ den Wert der Immobilie $i$, dann besteht
die Aufgabe der Berater darin eine Aufteilung der Menge $A$ aller 
zur Teilung stehenden Immobilien in zwei Mengen $I$ und $A\setminus I$
zu vollziehen, die Immobilien jeder Menge jeweils einem der S"ohne zugeteilt
werden sollen.
Nat"urlich sollen die beiden Mengen gleichen Gesamtwert haben.
Der Gesamtwert ist jeweils die Summe der einzelnen Werte, gesucht ist also
$I\subset A$ so dass
\[
\sum_{i\in I}c_i =\sum_{i\in A\setminus I}c_i
\]
Diese ist das Problem {\it PARTITION}, es ist NP-vollst"andig.
Daher gibt es nach aktuellem Wissen keinen polynomiellen Algorithmus,
der es entscheiden k"onnte.


%\subsection{MAX-CUT}
