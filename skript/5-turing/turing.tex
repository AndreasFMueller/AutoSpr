%
% turing.tex -- Turing Maschinen und rekursiv aufzaehlbare Sprachen
%
% (c) 2009 Prof Dr Andreas Müller, Hochschule Rapperswil
%

\section{Turing Maschinen}
\rhead{Turingmaschinen}
\subsection{Zwei Stacks: Random Access}
Ein Stack hat die Möglichkeiten eines endlichen Automaten dramatisch
erweitert, ein Stackautomat kann die bedeutend grössere Klasse der
kontextfreien Sprachen analysieren.
Wir haben aber auch schon gesehen, dass es nicht kontextfreie Sprachen gibt.
Mit welcher möglichst kleinen
Erweiterung können wir eine Sprache wie $\{a^nb^nc^n\,|\, n\ge 0\}$
analysieren?

Eine erste Idee könnte sein, einen zweiten Stack hinzuzufügen.
Damit könnte man mit Sicherheit die Sprache $\{a^nb^nc^n\,|\,n\ge 0\}$
behandeln.
Doch man fügt damit wesentlich mehr hinzu.
Ein Buch ist sicher ein Random-Access Speichermedium.
Durch Blättern im Buch kann man jede beliebige Seite erreichen.
Wenn man aber ein Buch
aufschlägt, besteht es eigentlich nur aus zwei Stapeln von Blättern.
Blättern bedeutet nichts anderes, als dass man ein Blatt des Buches vom
einen Stapel wegnimmt und auf den anderen Stapel legt.
Zwei Stacks bilden also bereits ein Random-Access Medium.

Antike Schriftrollen sind auch Bücher, die aber nicht als Stapel
von Seiten organisiert sind, sondern als langes Band.
Es ist aber klar,
dass man durch zerschneiden und binden jede Schriftrolle in ein
Buch verwandeln kann, und ein Buch durch zusammenkleben der einzelnen
Seiten zu einem Band in eine Schriftrolle.
Die heutigen Bücher (Kodices) haben sich gegenüber den Schriftrollen
etwa im ersten Jahrhundert CE durchgesetzt.
Für unsere Anwendung ist die Vorstellung eines Random-Access Mediums
als endloses Band von Speicherzellen praktischer.
Sie passt besser zu
unserem Bild vom Hauptspeicher eines Computers oder von Massenspeichermedien.
\subsection{Definition}
\index{Band}%
\index{Bandalphabet}%
\index{Schreib-/Lese-Kopf}%
Wir wollen einem endlichen Automaten statt eines Stacks einen unendlich
grossen Speicher zur Verfügung stellen.
Wir stellen uns diesen
Speicher als ein in beide Richtungen unendlich langes Band vor,
welches in einzelne Speicherzellen aufgeteilt ist, in die jeweils
genau ein Zeichen geschrieben werden kann (Abbildung~\ref{turingfig}).
Wie beim Stackautomaten gehen
wir davon aus, dass dieser Speicher auch Dinge speichern kann, die nicht
im Input vorkommen, wir verwenden daher ein Bandalphabet $\Gamma$, welches
das Input-Alphabet umfasst: $\Gamma\supset \Sigma$.

\begin{figure}
\begin{center}
%\includegraphics[width=0.8\hsize]{images/turing-1}
\includegraphics{images/turing-1}
\end{center}
\caption{Schematische Darstellung einer Turing-Maschine\label{turingfig}}
\end{figure}

Der Zugriff auf den Speicher erfolgt jeweils zellenweise, wir stellen
uns einen Schreib-/Lese-Kopf vor, der auf dem Band positioniert werden
kann, und genau ein Zeichen lesen kann, oder das auf dem Band vorhandene
Zeichen überschreiben kann.
Um zu Beginn einen wohldefinierten Zustand
zu haben, brauchen wir ein zusätzliches ``Blank''-Zeichen $\blank$,
welches in $\Gamma$ aber nicht in $\Sigma$ enthalten ist\footnote{Wäre
das ``Blank''-Zeichen $\blank$ auch in $\Sigma$, könnte man das
leere Band nicht vom Input unterscheiden.}.

Da wir jetzt diesen unendlich grossen Speicher haben, gibt es keinen
Grund mehr, sich vorzustellen, dass der Input Zeichen für Zeichen
in den endlichen Automaten gefüttert wird.
Stattdessen gehen wir
davon aus, dass der Input auf das Band geschrieben wird, der
Schreib-/Lesekopf auf das erste Feld des Input positioniert wird,
und man dann die Maschine arbeiten lässt.

Die Maschine muss jetzt Zeichen vom Band lesen, und zusammen mit
ihrem aktuellen Zustand entscheiden, in welchem neuen Zustand
sie nach der Verarbeitung dieses Zeichens sein soll, welches Zeichen
an dieser Stelle auf dem Band stehen soll, und ob der Schreib-/Lesekopf
bewegt werden soll.
Wir erlauben nur zwei Bewegungen: L und R, insbesondere
ist es nicht zulässig, dass der Kopf nach der Verarbeitung stehen
bleibt.

Im Gegensatz zu DEA oder Stackautomat ist die Berechnung in diesem
Modell nach dem Ende des Input-Strings noch nicht zu Ende.
Die Maschine
kann mehrmals über die auf dem Band gespeicherten Daten fahren und
sie immer wieder modifizieren.
Sie könnte also im Prinzip auch gar nie aufhören zu arbeiten.
Daher brauchen wir Zustände, die die Maschine anhalten.
Wenn die Maschine angehalten hat, müssen wir
zudem wissen, ob das auf das Band geschriebene Input-Wort für die
Sprache akzeptabel war oder nicht.

All dies führt uns auf folgende formale Definition einer Turing-Maschine
\begin{definition}
\index{Turing-Maschine}%
Eine Turing-Maschine ist ein $7$-tupel
$M=(Q,\Sigma,\Gamma,\delta,q_0,q_{\text{accept}},q_{\text{reject}})$,
wobei $Q$, $\Sigma$ und $\Gamma$ endliche Mengen sind mit folgenden
zusätzlichen Eigenschaften
\begin{compactenum}
\index{Zustand}%
\item $Q$ heisst die Menge der Zustände
\index{Inputalphabet}%
\index{Eingabe-Alphabet}%
\index{blank@Blank}%
\index{$\blank$|see{Blank}}%
\item $\Sigma$ heisst das Inputalphabet, es enthält das spezielle
``Blank''-Zeichen $\blank$ nicht.
\index{Bandalphabet}%
\item $\Gamma$ ist das Bandalphabet, es gilt $\blank\in\Gamma$ und
$\Sigma\subset\Gamma$.
\index{Uebergangsfunktion@Übergangsfunktion}%
\item $\delta\colon Q\times \Gamma\to Q\times\Gamma\times\{\text{L},\text{R}\}$
ist die Übergangsfunktion
\index{Startzustand}%
\item $q_0\in Q$ ist der Startzustand
\index{Akzeptierzustand}%
\item $q_{\text{accept}}\in Q$ ist der Akzeptierzustand
\index{Ablehnungszustand}%
\item $q_{\text{reject}}\in Q$ ist der Ablehnungszustand
\end{compactenum}
\end{definition}
Die Übergangsfunktion liefert also zu jedem Zustand und zum
Inhalt des Feldes unter dem Schreib-/Lesekopf einen neuen
Zustand, einen neuen Bandinhalt und die Information, ob
der Kopf nach links (L) oder rechts (R) bewegt werden muss.

\subsection{Sprache}
Sei eine Turingmaschine
$M=(Q,\Sigma,\Gamma,\delta,q_0,q_{\text{accept}},q_{\text{reject}})$
gegeben.
Zu $M$ kann man auf folgende Weise eine Sprache konstruieren:
Um zu entscheiden, ob ein Wort $w\in\Sigma^*$
zu der Sprache gehört,
schreibt man es auf das Band, platziert den Schreib-/Lesekopf auf
das erste Zeichen und bringt die Maschine in den Zustand $q_0$.
\index{Konfiguration}%
Man beschreibt diese Konfiguration der Maschine mit der Zeichenkette
$q_0w$, die Maschine ist im Zustand $q_0$ und der Schreib-/Lesekopf
steht über dem ersten Zeichen von $w$.

Mit Hilfe der Übergangsfunktion wird dann der nächste Zustand,
der neue Inhalt des Feldes unter dem Schreib-/Lesekopf und die
Kopfbewegung ermittelt.
Auch diesen neuen Zustand kann man wieder als Konfiguration hinschreiben.
Der Schreib-/Lesekopf steht
jetzt womöglich mitten im Wort drin, und die Maschine befindet
sich in irgend einem Zustand $q$: $w_1qw_2$.

Nehmen wir an, die Maschine befindet sich zu Beginn eines Zyklus im
Zustand $q_1$ und hat den Kopf auf das Zeichen $a_k$ positioniert.
Die Funktion $\delta$ berechnet daraus ein Tripel $(q_1,b_k,R)$
oder $(q_2,b_k,L)$.
Die Konfiguration ändert sich dabei wie folgt:
\begin{align*}
&\text{Ausgangskonfiguration:}&a_1\dots a_{k-1}&\;q_1\;a_k\dots a_n\\
&\text{Übergang mit } \delta(q_1,a_k)=(q_2,b_k,L)&a_1\dots a_{k-2}&\;q_2\;a_{k-1}b_k\dots a_n\\
&\text{Übergang mit } \delta(q_1,a_k)=(q_2,b_k,R)&a_1\dots a_{k-1}b_k&\;q_2\;a_{k+1}\dots a_n
\end{align*}

Die Maschine arbeitet so immer weiter, bis einer der Zustände $q_{\text{accept}}$
oder $q_{\text{reject}}$ erreicht wird, dann hält sie an.
Das Inputwort gilt als akzeptiert, wenn der Zustand $q_{\text{accept}}$
erreicht wurde, es gilt als verworfen, wenn $q_{\text{reject}}$ erreicht
wurde.

\index{Berechnungsgeschichte}%
Die Folge von Konfigurationen, die die Turing-Maschine
während der Berechnung durchläuft, nennt man auch
{\em Berechnungsgeschichte}.

\begin{definition}
\index{Sprache!von einer Turing-Maschine erkannte}%
Ist $M$ eine Turingmaschine, dann heisst
\[
L(M)=\{w\in\Sigma^*\;|\;\text{$M$ akzeptiert $w$}\}.
\]
die von $M$ erkannte Sprache.
\end{definition}

\begin{definition}
\index{Turing-erkennbar}%
Eine Sprache $L$ heisst Turing-erkennbar, wenn es eine Turing-Maschine
$M$ gibt mit $L=L(M)$.
\end{definition}

Die Definition der erkannten Sprache erlaubt, dass die zur Erkennung
verwendete Turing-Maschine auf einigen Wörtern $w\in\Sigma^*$ nicht
anhält.
Diese Wörter werden nie akzeptiert, und gehören
damit nicht zur Sprache.
Testet man sie mit $M$, wird die Berechnung
aber nie anhalten, man weiss also eigentlich nie, ob das Wort
nicht zur Sprache gehört, oder ob man einfach noch etwas mehr
Geduld braucht.

Gewissheit hat man erst, wenn man sicher sein kann, dass $M$ auf
jedem Input anhält.

\begin{definition}
\index{Entscheider}%
Ein Entscheider ist eine Turingmaschine, die auf jedem Input $w\in\Sigma^*$
anhält.
Eine Sprache heisst entscheidbar, wenn ein Entscheider sie erkennt.
\index{entscheidbar}%
\end{definition}
Es ist klar, dass eine Turing-entscheidbare Sprache auch Turing-erkennbar
ist.
Die Definition der Eigenschaft ``Turing-entscheidbar'' unterscheidet sich
von der Definition der Eigenschaft ``Turing-erkennbar'' nur dadurch,
dass bei einer entscheidbaren Sprache die Turing-Maschine auf jedem
beliebigen Input anhalten muss, während bei einer nur erkennbaren
Sprache einzelne Input-Wörter auch dazu führen können, dass die
Turing-Maschine endlos weiterrechnet.

\begin{beispiel}
Die kleinsten Mersenne-Primzahlen sind $3=2^2-1$, $7=2^3-1$, $31=2^5-1$,
andererseits ist $2^{11}-1=2047=23\cdot 89$ keine Primzahl.
Das finden von Mersenneschen Primzahlen ist mit GIMPS (Great Internet
Mersenne Prime Search \cite{skript:gimps}) zu einem Volkssport geworden.
Es ist nicht bekannt, ob es unendlich viele Mersennesche Primzahlen gibt.

Wir bilden jetzt die folgende Sprache über dem Alphabet
$\Sigma=\{\texttt{0},\texttt{1}\}$:
\[
L=\{w\in\Sigma^*\,|\,\text{die Anzahl Mersennescher Primzahlen ist grösser als $w$}\}.
\]
Ein Turing-Maschinen-Programm, welches prüfen muss, ob $w\in L$ ist,
muss herausfinden, ob es mindestens $w$ (interpretiert als Zahl)
Mersennsche Primzahlen gibt.
Dazu müssen der Reihe nach alle Zweierpotenzen $2^k$ daraufhin getestet werden,
ob $2^k-1$ eine Primzahl ist.
Wenn mindestens $n$ Mersennsche Primzahlen gefunden wurden, wird $n$
akzeptiert, sonst nicht.
Falls $n\in L$ ist, wird der Algorithmus
die Zahl $n$ akzeptieren, die Sprache $L$ ist also Turing-erkennbar.

Falls es nur endlich viele Mersennesche Primzahlen gibt, zum Beispiel
nur $N$, dann wird dieser Algorithmus für jede Zahl $n>N$ nicht
aufhören zu laufen.
Der Algorithmus kann ja nicht abschätzen, ob noch eine Mersennesche
Primzahl kommen wird, ob es also einfach noch etwas mehr Geduld
braucht.
Diese Sprache $L$ ist also nicht Turing-entscheidbar.
\end{beispiel}

\subsection{Zustandsdiagramm}
\index{Graph!gerichteter!beschrifteter!einer Turing-Maschine}%
Auch eine Turingmaschine kann man als gerichteten beschrifteten Graphen
darstellen.
Übergänge zwischen zwei Zuständen sind immer begleitet von
einer Änderung des Inhaltes des Feldes unter dem Schreib-/Lesekopf
und von einer Kopfbewegung.
Beide müssen mit der Beschriftung der Pfeile codiert werden.
In Anlehnung an die bei den Stackautomaten
verwendete Notation schreiben wir für den Übergang
$\delta(q_1,a)=(q_2,b,\text{R})$
die Notation
\[
\entrymodifiers={++[o][F]}
\xymatrix @+5mm{
{q_1}\ar[r]^{a\to b,\text{R}}
	&{q_2}
}
\]
und für den Übergang
$\delta(q_1,a)=(q_2,b,\text{L})$
\[
\entrymodifiers={++[o][F]}
\xymatrix @+5mm{
{q_1}\ar[r]^{a\to b,\text{L}}
	&{q_2}
}
\]

\subsection{Beispiel}
Wir konstruieren eine Turingmaschine, welche die bereits als weder regulär
noch kontextfrei erkannte Sprache $L=\{\texttt{0}^{2^n}\;|\; n\in\mathbb N\}$
erkennt.

Zweierpotenzen kann man daran erkennen, dass man die Zahl ohne
Rest durch zwei teilen kann, bis nur noch 1 übrig bleibt.
Genau diesen Prozess kann man mit einer Turingmaschine auf dem Band
nachbilden.
Die Maschine muss also folgendes tun:

Auf dem Input $w$:
\begin{compactenum}
\item Fahre von links nach rechts über das Band und streiche jedes zweite $\texttt{0}$.
\item Falls im Schritt 1 das Band genau eine $\texttt{0}$ enthielt, {\it akzeptiere}.
\item Falls im Schritt 1 das Band eine ungerade Zahl und mehr als ein $\texttt{0}$
enthielt, {\it verwerfe}.
\item Fahre mit dem Kopf zurück zum linken Ende des Bandes
\item Weiter bei Schritt 1
\end{compactenum}
In jedem Durchgang wird die Anzahl der Nullen halbiert, es sei denn,
sie war nicht gerade.
In diesem Fall bleibt nach dem Durchgang eine ungerade Anzahl $\texttt{0}$
auf dem Band stehen, was beim darauffolgenden Durchgang
in Schritt~3 erkannt wird.
Geht die Division immer auf, bleibt am Schluss
nur genau eine $\texttt{0}$ stehen, was im Schritt~2 erkannt wird.
Daher funktioniert dieser Algorithmus.

Um den Algorithmus mit einer Turing-Maschine zu realisieren, brauchen wir
ein neues Zeichen {\tt x}, mit dem wir gestrichene Zeichen markieren
können.
Es ist also $\Gamma = \{\texttt{0},{\tt x},\blank\}$.
Ausserdem brauchen wir
Zustände, mit denen man die Parität der Anzahl $\texttt{0}$ bestimmen kann, und
mit denen man die Aktivitäten in Schritt 4 von den anderen unterscheiden
kann.
Schritt $4$ wird zum Beispiel implementiert durch das Zustandsdiagramm
\[
\entrymodifiers={++[o][F]}
\xymatrix @+5mm{
{}\ar@(ul,ur)^{\genfrac{}{}{0pt}{1}{\texttt{0}\to \texttt{0},\text{L}}{{\tt x}\to{\tt x},\text{L}}}
\ar[r]^{\blank\to\blank,\text{R}}
	&
}
\]
Jede zweite Null streichen wird implementiert durch
\[
\entrymodifiers={++[o][F]}
\xymatrix @+5mm{
*+\txt{}\ar[d]
\\
\ar@/^/[r]^{\texttt{0}\to{\tt x},\text{R}}
\ar@(ul,dl)_{\tt x\to\tt x,\text{R}}
	&\ar@/^/[l]^{\texttt{0}\to \texttt{0},\text{R}}
         \ar@(ur,dr)^{\tt x\to\tt x,\text{R}}
	 \ar[d]^{\blank\to\blank,?}
\\
*+\txt{}
	&{}
}
\]
Dass akzeptiert werden soll, wenn nach dem Ersetzen einer $\texttt{0}$ durch
ein {\tt x} keine weiteren $\texttt{0}$ mehr gefunden werden können, kann durch
\[
\entrymodifiers={++[o][F]}
\xymatrix @+10mm{
\ar[r]^{\texttt{0}\to{\tt x},\text{R}}
	&{} \ar@(dl,dr)_{{\tt x}\to{\tt x},\text{R}}
	    \ar[r]^{\blank\to\blank,\text{R}}
		&*++[o][F=]{q_{\text{accept}}}
}
\]
ausgedrückt werden.
Die Schleife beim mittleren Zustand
bildet ab, dass {\tt x}-Zeichen übersprungen werden sollen.
Mit solchen Elementen können wir die Turingmaschine jetzt schrittweise
aufbauen.

Zunächst stellen wir sicher, dass der Input tatsächlich mit \texttt{0}
beginnt:
\[
\entrymodifiers={++[o][F]}
\xymatrix{
*+\txt{}\ar[d]
\\
{q_1}\ar[d]_{\genfrac{}{}{0pt}{1}{\blank\to\blank,\text{R}}{{\tt x}\to{\tt x},\text{R}}}
\\
*++[o][F=]{q_\text{reject}}
}
\]
Jetzt hängen wir das Segment an, welches prüft, ob es noch genau ein $\texttt{0}$
auf dem Band hat:
\[
\entrymodifiers={++[o][F]}
\xymatrix{
*+\txt{}\ar[d]
\\
{q_1}\ar[d]_{\genfrac{}{}{0pt}{1}{\blank\to\blank,\text{R}}{{\tt x}\to{\tt x},\text{R}}}
	\ar[r]^{\texttt{0}\to{\tt x},\text{R}}
	&{q_2}\ar[d]^{\blank\to\blank,\text{R}}
              \ar@(u,ul)_{{\tt x}\to {\tt x},\text{R}}
\\
*++[o][F=]{q_{\text{reject}}}
	&*++[o][F=]{q_\text{accept}}
}
\]
Falls dies nicht zutrifft, müssen weitere Nullen weggestrichen werden,
wobei über die Parität der noch vorhandenen Nullen Buch geführt werden
muss.
Dafür braucht es die beiden Zustände $q_3$ und $q_4$:
\[
\entrymodifiers={++[o][F]}
\xymatrix{
*+\txt{}\ar[d]
\\
{q_1}\ar[d]_{\genfrac{}{}{0pt}{1}{\blank\to\blank,\text{R}}{{\tt x}\to{\tt x},\text{R}}}
	\ar[r]^{\texttt{0}\to\blank,\text{R}}
	&{q_2}\ar[d]_{\blank\to\blank,\text{R}}
	      \ar[r]^{\texttt{0}\to{\tt x},\text{R}}
              \ar@(u,ul)_{{\tt x}\to {\tt x},\text{R}}
		&{q_3}\ar@(u,r)^{{\tt x}\to{\tt x},\text{R}}
		      \ar@/^/[d]^{\texttt{0}\to \texttt{0},\text{R}}
\\
*++[o][F=]{q_\text{reject}}
	&*++[o][F=]{q_\text{accept}}
		&{q_4}\ar@/^/[u]^{\texttt{0}\to {\tt x},\text{R}}
		      \ar@(d,r)_{{\tt x}\to{\tt x},\text{R}}
}
\]
Wenn die Maschine nach dem Abarbeiten aller \texttt{0} und {\tt x} auf einen
neuen Blank \textvisiblespace\ trifft, ist sie entweder im Zustand
$q_3$ oder in $q_4$.
Im letzten Fall hat sie eine ungerade Anzahl
von \texttt{0} gefunden, gemäss Schritt 3 des Algorithmus muss also verworfen
werden:
\[
\entrymodifiers={++[o][F]}
\xymatrix{
*+\txt{}\ar[d]
\\
{q_1}\ar[d]_{\genfrac{}{}{0pt}{1}{\blank\to\blank,\text{R}}{{\tt x}\to{\tt x},\text{R}}}
	\ar[r]^{\texttt{0}\to\blank,\text{R}}
	&{q_2}\ar[d]_{\blank\to\blank,\text{R}}
	      \ar[r]^{\texttt{0}\to{\tt x},\text{R}}
              \ar@(u,ul)_{{\tt x}\to {\tt x},\text{R}}
		&{q_3}\ar@(u,r)^{{\tt x}\to{\tt x},\text{R}}
		      \ar@/^/[d]^{\texttt{0}\to \texttt{0},\text{R}}
\\
*++[o][F=]{q_\text{reject}}
	&*++[o][F=]{q_\text{accept}}
		&{q_4}\ar@/^/[u]^{\texttt{0}\to {\tt x},\text{R}}
		      \ar@(d,r)_{{\tt x}\to{\tt x},\text{R}}
		      \ar@/^20pt/[ll]^{\blank\to\blank,\text{R}}
}
\]
Wenn sie dagegen im Zustand $q_3$ ist, muss sie nach links fahren,
bis zum ersten Blank, und dann im Zustand $q_2$ ankommen.
Dies erreicht
man mit Hilfe eines weitere Zustands $q_5$.
\[
\entrymodifiers={++[o][F]}
\xymatrix{
*+\txt{}\ar[d]
	&*+\txt{}
		&{q_5}\ar@(ul,ur)^{\genfrac{}{}{0pt}{1}{\texttt{0}\to \texttt{0},\text{L}}{{\tt x}\to {\tt x},\text{L}}}
		      \ar[dl]_{\blank\to\blank,\text{R}}
\\
{q_1}\ar[d]_{\genfrac{}{}{0pt}{1}{\blank\to\blank,\text{R}}{{\tt x}\to{\tt x},\text{R}}}
	\ar[r]^{\texttt{0}\to\blank,\text{R}}
	&{q_2}\ar[d]_{\blank\to\blank,\text{R}}
	      \ar[rr]^{\texttt{0}\to{\tt x},\text{R}}
              \ar@(u,ul)_{{\tt x}\to {\tt x},\text{R}}
		&*+\txt{}
			&{q_3}\ar@(u,r)^{{\tt x}\to{\tt x},\text{R}}
			      \ar@/^/[d]^{\texttt{0}\to \texttt{0},\text{R}}
			      \ar[ul]_{\blank\to\blank,\text{L}}
\\
*++[o][F=]{q_\text{reject}}
	&*++[o][F=]{q_\text{accept}}
		&*+\txt{}
			&{q_4}\ar@/^/[u]^{\texttt{0}\to {\tt x},\text{R}}
			      \ar@(d,r)_{{\tt x}\to{\tt x},\text{R}}
			      \ar@/^20pt/[lll]^{\blank\to\blank,\text{R}}
}
\]
Mit diesem Beispiel haben wir gezeigt, dass die Menge der
Sprachen, die von einer Turingmaschine erkennbar sind, Sprachen
enthält, die mit den bisherigen Mitteln nicht erkennbar waren.

\begin{definition}
Die von einer Turingmaschine $M$ erkannten Wörter bilden die
Sprache $L(M)$.
\end{definition}

