%
% berechenbar.tex -- Berechenbarkeit
%
% (c) 2009 Prof Dr Andreas Müller, Hochschule Rapperswil
%
\section{Berechenbarkeit}
\rhead{Berechenbarkeit}
In diesem Abschnitt wollen wir klären, welche Dinge überhaupt
berechnet werden können. Dabei sehen wir die Tatsache, dass für
die Darstellung einer reellen Zahl eine unendliche Folge von Ziffern
nötig sein kann, nicht als Hindernis an. Wir nennen eine Zahl
berechenbar, wenn wir ein Programm schreiben können, welches
möglicherweise unendlich lange läuft und dabei eine Stelle der
Zahl nach der anderen liefern kann. Die Turingmaschine
funktioniert also als Aufzähler, der die Stellen der zu berechnenden
Zahl aufzählen soll.  In diesem Sinne ist die
Zahl $1/3=0.3333\dots$ berechenbar, aber auch die Zahlen $\sqrt{2}$, $\pi$,
$e$ und weitere. Wir werden allerdings sehen, dass die meisten
Zahlen gar nicht berechenbar sind.

\subsection{Abzählbar und überabzählbar}
Unendlich ist nicht gleich unendlich, die Mengen $\mathbb N$ und
$\mathbb R$ haben zwar beide unendlich viele Elemente, dennoch
gibt es keine bijektive Abbildung $\mathbb N\to \mathbb R$.

\begin{definition}
\index{Machtigkeit@Mächtigkeit}%
Zwei Mengen $A$ und $B$ heissen gleich mächtig, wenn es eine bijektive
Abbildung $A\to B$ gibt.
\end{definition}

Es ist bekannt, dass $\mathbb Q$ und $\mathbb N$ gleich mächtig sind,
aber die Menge $\mathbb R$ scheint wesentlich grösser
zu sein als $\mathbb N$.
Diese Unterschiede zwischen verschiedenen
unendlichen Mengen wird illustriert von der nachfolgenden Geschichte,
die David Hilbert zu erzählen pflegte.

\subsubsection{Hotel ``Unendlich''}
\index{Hotel ``Unendlich''}%
\index{Hilbert!-Hotel}%
In einem fernen Land gibt es ein Hotel, welches Hotel ``Unendlich''
genannt wird. Unsere Geschichte beginnt kurz nachdem der Nachtportier
die Nachtschicht übernommen hat. Ein verspäteter Gast meldet sich
am Eingang, während der Nachtportier damit beschäftigt ist,
die herumlungernden Besoffenen
aus den naheliegenden Pubs zu vertreiben, die auch gerade geschlossen haben.
Der Gast möchte im Hotel übernachten, doch der Nachtportier zeigt sich
unnachgiebig: ``Du kommst hier nicht rein!''

Zum Glück hat der Hotelmanager den Wortwechsel mitbekommen und greift
ein. Selbstverständlich haben wir noch ein Zimmer für jeden noch so
späten Gast. Der Nachportier protestiert, es seien doch alle Zimmer
belegt. ``Kein Problem'' sagt der Manager, und greift zum Mikrofon
der Sprechanlage. Er bittet alle Gäste, aus ihrem aktuellen Zimmer
ins nächste Zimmer umzuziehen. Dieses Ansinnen zu später Stunden
verursacht zwar bei einigen Gästen etwas Unmut, aber jeder ist damit
einverstanden, denn hätten sie sich selbst verspätet, wären sie ja
auch auf das Entgegenkommen der anderen Gäste angewiesen. Auf diese
Weise wird das Zimmer mit der Nummer $0$ frei für den neu angekommenen
Gast.

Kurze Zeit später kommt eine verspätete Reisegruppe aus dreissig
Touristen an, die ebenfalls
Einlass verlangt. Der Nachtportier will nicht nochmals einen schlechten
Eindruck hinterlassen und ruft den Manager. Dieser greift
wieder zum Mikrofon, und bittet wieder alle Gäste, diesmal aus
ihrem Zimmer mit der Nummer $n$ ins Zimmer mit der Nummer $n+30$
umzuziehen. So werden die Zimmer mit den Nummern $0$ bis $29$ frei,
Platz genug für die neu angekommene Reisegruppe.

Diese Nacht ist ziemlich viel los im Hotel ``Unendlich'', denn
kaum waren die dreissig Gäste untergebracht, fährt ein Hilbert-Bus
vor: ein Bus mit unendlich vielen Sitzplätzen, numeriert mit den
natürlichen Zahlen. Der Nachtportier ist schockiert: Unendlich
viele Gäste, ohne Reservation. Dass man für endlich viele Gäste
immer noch Platz schaffen kann, hat er inzwischen verstanden, aber
für unendlich viele Gäste sei das unmöglich, stellt er fest, und
will die Gesellschaft wegschicken. Doch der Hotelmanager möchte
sich das Geschäft nicht entgehen lassen. Er greift erneut zum
Mikrofon und bittet die Gäste, vom Zimmer $n$ ins Zimmer $2n$ umzuziehen.
So werden alle Zimmer mit ungeraden Nummern frei, Platz genug für alle
Passagiere des Hilbert-Buses. Nur die Stimmung unter den Stammgästen
hat sich bereits deutlich verschlechtert.

Schliesslich fahren gleich dreissig Hilbertbusse vor. Ein Hilbert-Jumbo
hatte sich verspätet, und so dass die Feriengäste erst sehr spät
mit ihren Hilbertbussen vom Flughafen zum Hotel abfahren konnten.
Der Nachtportier hat schon einen Verdacht, dass der Manager auch
hierfür eine Lösung bereit hat. Tatsächlich: per Mikrofondurchsage
gibt er wieder alle ungerade Zimmer frei. Dann bringt er die Neuankömmlinge
darin unter, und zwar schön nacheinander immer je einen aus jedem Bus,
also in die ersten dreissig freien Zimmer die jeweils ersten Gäste
aus jedem Bus, dann die zweiten aus jedem Bus in die zweiten dreissig
freien Zimmer und so weiter.

Man ahnt es schon, in dieser Nacht gibt es für die Gäste im Hotel
``Unendlich'' keine Ruhe. Als nächstes findet sich ein Hilbert-Konvoi
vor dem Hotel ein. Also einer Folge von Hilbert-Bussen, die mit natürlichen
Zahlen numeriert waren. Der Nachtportier hat gar nicht erst gewagt,
den Neuankömmlingen abschlägigen Bescheid zu geben, sondern gleich
den Manager gerufen. Der muss zwar auch einen Moment nachdenken, findet
dann aber eine Lösung. Zunächst wendet er nochmals den Trick an,
mit dem er jetzt schon mehrmals die ungeraden Zimmer frei bekommen hat.
Dabei kommt es auch zu wüsten Szenen mit Gästen, die jetzt schon vier mal
geweckt worden sind, und sich ihre Nachtruhe nicht von der Geldgier des
Managers ruinieren lassen wollen.

Die Gäste aus dem Hilbert-Konvoi müssen sich in Einerkolonnen auf
dem Vorplatz aufstellen, der Manager rief sie daraufhin in der
eingezeichneten Reihenfolge ab. Auch dies trägt nicht  unbedingt
zur Zufriedenheit der Gäste bei, hinter vorgehaltener Hand wird
über den ``Kasernenton'' und ``Zustände wie in einem Konzentrationslager''
geschimpft.
Aber jeder kommt früher oder später dran,
am Ende hat jeder Gast ein Zimmer, was die Gemüter wieder
etwas beruhigt.

\[
\xymatrix{
\ar[r]
\cdot	     &\cdot\ar[dl] &\cdot\ar[r] &\cdot\ar[dl]  &\cdot\ar[r] &\cdot\ar[dl] &\cdot
\\
\cdot\ar[d]  &\cdot\ar[ur] &\cdot\ar[dl] &\cdot\ar[ur] &\cdot\ar[dl]&\cdot        &\cdot
\\
\cdot\ar[ur] &\cdot\ar[dl] &\cdot\ar[ur] &\cdot\ar[dl] &\cdot       &\cdot        &\cdot
\\
\cdot\ar[d]  &\cdot\ar[ur] &\cdot\ar[dl] &\cdot        &\cdot       &\cdot        &\cdot
\\
\cdot\ar[ur] &\cdot        &\cdot        &\cdot        &\cdot       &\cdot        &\cdot
}
\]

Ein Ereignis steht allerdings noch bevor, und hier sollte sich die
Geldgier des Managers rächen. Gegen morgen nämlich fährt
ein voller Cantor-Bus vor. Dieses hochmoderne Verkehrsmittel hat Sitze,
\index{Cantor!-Bus}%
die mit den reellen Zahlen angeschrieben sind. Natürlich wollen auch
diese Gäste im Hotel ``Unendlich'' untergebracht werden. Aber so sehr
sich der Manager auch anstrengt, in seinem Hotel kann er diese
Gäste nicht unterbringen. Daraufhin wird er zwar vom Verwaltungsrat
entlassen, der noch weniger Mathematik versteht, was ihm dank üppiger
Abgangsentschädigung jedoch nicht weiter Sorgen macht. Unbestätigten
Gerüchten zufolge soll er jetzt bei einer Hotelkette ``Cantor Hotels''
arbeiten, die mit mit dem Spruch ``Present in uncountable locations
throughout the Universe'' für sich wirbt.

\subsubsection{Lehren aus der Geschichte}
Eine unendliche Menge ist offenbar so gross, dass man darin immer noch
Platz genug für eine Kopie der ganzen Menge finden kann. Oder anders
herum: endliche Mengen sind solche, in denen man niemals Platz finden
könnte:

\begin{satz}
Eine Menge ist endlich, wenn jede injektive Abbildung auch surjektiv ist.
\end{satz}

In der Geschichte kam eine ganze Reihe von unendlichen Mengen auf, die
alle in $\mathbb N$ untergebracht werden konnten, die also nicht
grösser waren als $\mathbb N$ selbst, und das obwohl sie aus natürlichen
Zahlen konstruiert worden waren. Mit $\mathbb N$ vergleichbare Mengen
bilden also eine robuste Klasse von Mengen.

\begin{definition}
\index{abzahlbar unendlich@abzählbar unendlich}%
\index{uberabzahlbar unendlich@überabzählbar unendlich}%
Eine unendliche Menge $A$ heisst abzählbar unendlich, wenn sie
gleichmächtig ist wie die natürlichen Zahlen. $A$ heisst
überabzählbar unendlich, wenn es keine Bijektion zwischen
$\mathbb N$ und $A$ gibt.
\end{definition}
Bildlich gesprochen ist eine abzählbare Menge eine solche,
bei der sich die Elemente in eine mit den natürlichen Zahlen numerierte
Liste einordnen lassen.

Auch die Menge der Paare $(k,l)\in \mathbb N^2$ ist abzählbar.
Ein Gast aus dem Hilbert-Konvoi wird durch seine Busnummer $k$ und
seine Platz-Nummer $l$ identifiziert, der Konvoi hat also gleich
viele Gäste wie $\mathbb N^2$ Paare enthält. Und alle diese Paare
passen in die Menge $\mathbb N$, also die Zimmer des Hotels ``Unendlich''
hinein. Somit sind $\mathbb N^2$ und $\mathbb N$ gleich mächtig.

Die Geschichte hat exemplarisch den folgenden Satz gezeigt:

\begin{satz}Die Vereinigung von endlich vielen abzählbaren
Mengen ist abzählbar. Das kartesische Produkt zweier abzählbarer
Mengen ist abzählbar.
\end{satz}

\begin{proof}[Beweis]
Sei $A_1,\dots,A_n$ eine endliche Familie abzählbarer Mengen,
je mit einer Funktion $f_i\colon \mathbb N\to A_i$, dann
können wir eine Abzählfunktion für die Vereinigung konstruieren:
\[
f\colon \mathbb N\to\bigcup_{i=1}^n A_i\colon nk+l \mapsto f_l(k)
\]
wobei wir verwenden, dass wir jede Zahl eindeutig als $nk+l$
schreiben können mit geeigneten Zahlen $k$ und $l$.

Für das kartesische Produkt müssen wir zeigen, dass sich die
Menge der Paare $\mathbb N^2$ aufzählen lässt, was mit dem
Diagonaltrick des Hotelmanagers geschehen kann.
\end{proof}

Daraus folgen jetzt weitere Sätze über die Kardinalität
der bekannten Zahlmengen.

\begin{satz}
Die Mengen $\mathbb Z$ und $\mathbb Q$ sind abzählbar unendlich.
\end{satz}

\begin{proof}[Beweis]
Die Menge $\mathbb Z$ ist abzählbar, da sie die Vereinigung
von zwei Mengen ist, die offensichtlich abzählbar sind:
\[
\mathbb Z = \mathbb N\cup \{-n\;|\;n\in\mathbb N\}.
\]
Die rationalen Zahlen können durch Brüche dargestellt werden,
also durch Paare $(p,q)$ von ganzen Zahlen:
\[
\mathbb Q=\left\{\left.\frac{p}{q}\;\right|\;p\in \mathbb Z,q\in\mathbb N\setminus\{0\}\right\}
\]
Da sich einige Brüche noch kürzen lassen, ist $\mathbb Q$ noch
kleiner als $\mathbb N^2$, aber insbesondere abzählbar.
\end{proof}

\begin{satz}
Die Menge $\mathbb R$ ist überabzählbar unendlich.
\end{satz}

\begin{proof}[Beweis]
Wir führen den Beweis mit Hilfe eines Widerspruchs. Nehmen wir
an, $\mathbb R$ wäre abzählbar unendlich. Dann müssten auch
die reellen Zahlen zwischen $0$ und $1$ abzählbar sein,  es gäbe also eine
Liste all dieser Zahlen, wir schreiben die Zahlen
dieser Liste in Dezimaldarstellung
\begin{align*}
0.&r_{11}r_{12}r_{13}r_{14}\dots\\
0.&r_{21}r_{22}r_{23}r_{24}\dots\\
0.&r_{31}r_{32}r_{33}r_{34}\dots\\
0.&r_{41}r_{42}r_{43}r_{44}\dots\\
&\vdots
\end{align*}
Jetzt konstruieren wir eine Zahl $x$, die nicht in dieser Liste vorkommen
kann. Um die $k$-te Stelle $x_k$ von $x$ zu konstruieren, betrachten
wir die $k$-te Stelle der $k$-ten Zahl $r_{kk}$, und setzen
\[
x_k=\begin{cases}
r_{kk}-1&\qquad r_{kk}>0\\
5&\qquad r_{kk}=0
\end{cases}
\]
$x_k$ ist also verschieden von $r_{kk}$, und zwar für jedes $k$.
Die Ziffer $9$ kommt in $x$ nicht vor, es kann also keine Zahl
mit lauter $9$ am Ende sein, die gleichbedeutend mit einer anderen
Zahl ist, die möglicherweise bereits in der Liste ist.
Also unterscheidet sich $x$ von jeder der Zahlen aus der Liste, $x$
kommt also in der Liste nicht vor. Wir hatten angenommen, dass die
Liste alle reellen Zahlen zwischen $0$ und $1$ umfasst, dieser
Widerspruch zeigt, dass es keine solche Liste geben kann, die
reellen Zahlen sind also überabzählbar.
\end{proof}

\index{Cantor!Georg}%
Georg Cantor (1845-1918) leistete wichtige Beiträge zur Mengelehre,
unter anderem entdeckte er den Unterschied der Mächtigkeit von
natürlichen und reellen Zahlen. Daher haben wir in der Geschichte
den Bus, dessen Sitzplätze mit reellen Zahlen angeschrieben waren,
als Cantor-Bus bezeichnet.

Verwandt damit und mit einem ähnlichen Beweis kann man auch einsehen,
dass die Potenzmenge einer abzählbaren Menge überabzählbar ist.

\begin{satz}\label{powersetuncountable}
Ist $A$ eine abzählbar unendliche Menge, dann ist $P(A)$
überabzählbar.
\end{satz}

\begin{proof}[Beweis]
Da $A$ abzählbar ist gibt es eine bijektive Abbildung
$i\mapsto a_i\in A$. Nehmen wir an, es gäbe eine bijektive
Abbildung von $\mathbb N$ in $P(A)$, also $i\mapsto A_i\subset A$.
Dann kann man wie folgt eine Menge bilden, die unter den Mengen
$A_i$ nicht vorkommt.
\[
a_i\in B\quad\Leftrightarrow\quad a_i\not\in A_i
\]
Die Menge $B$ unterscheidet sich von jeder Menge $A_i$. Falls
$A_i$ das Elemente $a_i$ enthält, enthält $B$ es nicht und
umgekehrt. Somit ist $B\ne A_i\forall i$, im Widerspruch zur Annahme
dass $i\mapsto A_i$ eine Bijektion ist.
\end{proof}

Jetzt können wir auch einige Mengen aus der bisher betrachteten
Theorie der Sprachen auf ihre Mächtigkeit untersuchen.

\begin{satz}
$\Sigma^*$ ist abzählbar unendlich. Die Menge aller Sprachen über dem
Alphabet $\Sigma$ ist überabzählbar unendlich.
\end{satz}

\begin{proof}[Beweis]
Eine Aufzählung von $\Sigma^*$ kann man konstruieren, indem man
die Wörter von $\Sigma^*$ der Länge nach sortiert, und innerhalb
der Wörter gleicher Länge die lexikographische Ordnung verwendet.
Dazu braucht man natürlich eine Anordnung der Zeichen des Alphabets $\Sigma$,
da dieses aber endlich ist, kann man immer eine solche Anordnung finden.

Die Menge aller Sprache ist die Potenzmenge der Menge $\Sigma^*$, welche
abzählbar unendlich ist. Nach Satz \ref{powersetuncountable} ist
$P(\Sigma^*)$ überabzählbar unendlich.
\end{proof}

\begin{satz}\label{countablefinite}
Eine abzählbare Vereinigung $\bigcup_{i\in\mathbb N}A_i$ von endlichen
Mengen $(|A_i|<\infty)$ ist abzählbar.
\end{satz}

\begin{proof}[Beweis]
Endliche Mengen kann man immer abzählen, also kann man eine Abzählung
von $\bigcup_{i\in\mathbb N}A_i$ einfach dadurch konstruieren, dass
man zuerst die Elemente von $A_0$ durchzählt, dann die von $A_1$, und
so weiter. Weil keine der Mengen $A_i$ unendlich ist, kommt jede
Menge irgendwann dran.
\end{proof}

\begin{satz} Sei $\Sigma$ ein festes Alphabet. Dann sind die folgenden
Mengen alle abzählbar unendlich:
\begin{enumerate}
\item Die Menge aller deterministischen endlichen Automaten.
\item Die Menge aller nichtdeterministischen endlichen Automaten.
\item Die Menge der regulären Sprachen.
\item Die Menge aller kontextfreien Grammatiken.
\item Die Menge aller kontextfreien Sprachen.
\item Die Menge aller Stackautomaten.
\item Die Menge aller Turingmaschinen.
\end{enumerate}
\end{satz}

\begin{proof}[Beweis]
\begin{enumerate}
\item Es genügt zu zeigen, dass die Menge der deterministischen
endlichen Automaten mit $k$ Zuständen endlich ist, dann ist nach
Satz \ref{countablefinite} auch die Vereinigung abzählbar. Ein
deterministischer endlicher Automat ist aber durch die Tabellendarstellung
gegeben. In der Tabelle sind $|Q|\cdot|\Sigma|$ Felder mit Zuständen
zu besetzen, dafür gibt es $|Q|^{|Q|\cdot|\Sigma|}$ Möglichkeiten.
Für jede solche Möglichkeit ist ausserdem festzulegen, welche Zustände
Akzeptierzustände sind, das sind $|P(Q)|=2^{|Q|}$ Möglichkeiten. Es gibt
also
\[
|Q|^{|Q|\cdot|\Sigma|}\cdot 2^{|Q|}
\]
DEAs mit $|Q|$ Zuständen.
\item Obwohl in der Definition der nichtdeterministischen endlichen
Automaten auch die Potenzmenge vorkommt, wird dadurch die Menge
noch nicht über\-abzählbar. Der Funktionswert der Übergangsfunktion
$\delta$ ist ja immer eine Teilmenge einer endlichen Menge, die Potenzmenge
einer endlichen Menge ist aber auch endlich. Es ändert sich am Argument
nur, dass in die Tabelle eines Automaten mit $|Q|$ Zuständen und
$|\Sigma|$ Zeichen im Alphabet an $|Q|\cdot|\Sigma|$ Stellen
nicht $|Q|$ verschieden Objekte einfüllen lassen, sondern $2^{|Q|}$.
An der Abzählbarkeit ändert dies nichts.
\item Die regulären Sprachen werden von deterministischen endlichen
Automaten akzeptiert, es gibt also eine surjektive Abbildung von
den deterministischen endlichen Automaten auf die regulären Sprachen,
eine Aufzählung der deterministischen endlichen Automaten liefert also
auch automatisch eine Aufzählung der regulären Sprachen.
\item Es gibt nur endliche viele Grammatiken mit $n$ verschiedenen
Variablen und höchstens $n$ Zeichen langen rechten Seiten der Regeln,
einfach weil es nur endlich viele rechte Seiten gibt. Also ist die
Menge aller kontextfreien Grammatiken abzählbar.
\item Kontextfreie Sprachen werden von einer kontextfreien Grammatik
erzeugt. Die Abbildung $G\mapsto L(G)$ macht aus einer Aufzählung
der kontextfreien Grammatiken eine Aufzählung der kontextfreien Sprachen.
\item Stackautomaten sind nichtdeterministische endliche Automaten mit
zusätzlicher Beschriftung der Übergangspfeile. Trotzdem bleibt die
Anzahl der Stackautomaten mit $k$ Zuständen endlich, also ist die
Menge der Stackautomaten abzählbar unendlich.
\item Turingmaschinen sind im wesentlichen deterministische endliche Automaten
mit zusätzlichen Beschriftungen der Pfeile. Wie in 1.~gibt es nur endlich
viele Turingmaschinen mit $k$ Zuständen, also ist die Menge der Turingmaschinen
abzählbar unendlich.
\end{enumerate}
\end{proof}

\subsection{Nicht berechenbare Zahlen}

\begin{satz}
Die rationalen Zahlen sind berechenbar.
\end{satz}

\begin{proof}[Beweis]
Der Algorithmus der schriftlichen Division erlaubt alle Stellen
zu finden, man kann sich gut vorstellen, dass er sich auf einer
Turingmaschine als Aufzähler implementieren lässt. Zähler und
Nenner werden zu Beginn auf das Band geschrieben, der Algorithmus
berechnet dann alle Stellen des Quotienten. Also sind die rationalen
Zahlen berechenbar.
\end{proof}

\begin{satz}Die algebraischen Zahlen, also die Nullstellen von
Polynomen mit rationalen Koeffizienten sind berechenbar.
\end{satz}

\begin{proof}[Beweis]
Zunächst können wir das Polynom mit dem gemeinsamen Nenner der
Koeffizienten multiplizieren und erhalten ein Polynom mit ganzzahligen
Koeffizienten.
Die Nullstellen eines Polynoms können mit dem Newton-Algorithmus
mit beliebiger Genauigkeit bestimmt werden. Dazu muss nur eine
Schätzung $\hat x_0$ für die Nullstelle bekannt sein, dann können
mit der Iteration
\[
x_{n+1}=x_n-\frac{f(x_n)}{f'(x_n)},\quad x_0 = \hat x_0
\]
immer genauere Approximationen berechnet werden. Alle Operationen
in der Iterationsformel sind Operationen in rationalen Zahlen, sind
also berechenbar.
\end{proof}

\begin{satz}
Die Menge der nicht berechenbaren Zahlen ist überabzählbar.
\end{satz}

\begin{proof}[Beweis]
Die berechenbaren Zahlen sind abzählbar. Wir können nämlich
eine Liste aller Turingmaschinen erstellen, zunächst schreiben
wir die Turingmaschinen mit nur einem Zustand hin, dann all jene
mit genau zwei Zuständen u.\,s.\,w. Da jede Turingmaschine genau eine
Zahl berechnen kann, ist die Menge der berechnenbaren Zahlen abzählbar,
die Menge der reellen Zahlen ist also überabzählbar, daher müssen die
nicht berechenbaren Zahlen auch überabzählbar sein.
\end{proof}

\subsection{Das 10.~Hilbertsche Problem}
\index{Hilbert!David}%
\index{Hilbertsche Probleme}%
Im Jahre 1900 hielt der deutsche Mathematiker David Hilbert am
\index{Internationaler Kongress}%
internationalen Mathematikerkongress in Paris einen berühmten Vortrag,
in dem er eine Reihe von Problemen zusammenstellte, deren Lösung
nach seiner Meinung die Mathematik im zwanzigsten Jahrhundert
entscheidend voranbringen würden. Einige dieser Probleme wurden
inzwischen gelöst, andere, darunter die Riemannsche Vermutung,
sind immer noch offen.

Von besonderem Interesse für unser Thema war das zehnte Problem:
Gibt es ein Verfahren, das für eine beliebige diophantische Gleichung
entscheidet, ob sie lösbar ist?

Schon die Problemstellung wirft einige Fragen auf.
Zunächst zu den Begriffen:
\index{Gleichung!diophantische}%
eine diophantische Gleichung ist eine
Polynomgleichung mit mehreren Variablen aber ausschliesslich
ganzzahligen Koeffizienten. Die Gleichung
\[
x^2+y^2-z^2=0
\]
ist eine diophantische Gleichung, und es ist auch bekannt, dass sie
Lösungen hat, zum Beispiel $x=3$, $y=4$ und $z=5$. Seit wenigen
Jahren ist auch bekannt, dass
\[
x^n+y^n-z^n=0,
\]
ebenfalls eine diophantische Gleichung, nur ganz wenige Lösungen hat.
Dies ist die berühmte Fermatsche Vermutung, die Andrew Wiles 1995
\index{Fermatsche Vermutung}%
\index{Wiles, Andrew}%
vollständig bewiesen hat.

Die schwierigerere Frage aber ist: Was für eine Art von Verfahren
ist gemeint? Hilbert hat keine Definition gegeben. Heute würden
wir wohl fragen, ob ein Algorithmus angegeben werden kann, mit dem
die Frage entschieden werden kann. Doch das führt uns nur wieder
auf die Frage nach einer  mathematisch strengen Definition was
ein Algorithmus ist.

\index{Turing, Alan}%
Im Jahre 1936 gab Alan Turing eine Antwort: ein Algorithmus ist eine
Rechenvorschrift, die sich mit einer Turingmaschine implementieren
lässt. Probleme, für die es einen Lösungsalgorithmus gibt, sind
also solche, deren Inputdaten man als Wort auf das Band einer
Turingmaschine schreiben kann, die Turingmaschine verarbeitet die
Problembeschreibung, schreibt die Lösung auf das Band und hält im
Zustand $q_{\text{accept}}$ an. Falls das Problem keine Lösung hat,
hält die Maschine im Zustand $q_{\text{reject}}$ an. Solche
Probleme definieren also auch eine Sprache: $L(M)$ ist die
Menge der Inputwörter, für die eine Lösung existiert.

Ein Beispiel soll dies illustrieren. Wir möchten die Frage
von einem Algorithmus beantworten lassen, ob es in einem Graphen $G$
einen Weg gibt, der alle Knoten trifft.  Dazu müssen wir eine
Beschreibung des Graphen erzeugen, wir nennen sie $\langle G\rangle$,
diese auf das Band einer geeignet programmierten Turingmaschine $M$ schreiben,
diese laufen lassen, und warten, bis sie die Frage beantwortet. $L(M)$
besteht aus Beschreibungen $\langle G\rangle$ von Graphen $G$, die
einen solchen Weg haben.

Auf dieser Basis gelang es 1970 Yuri Matijasevi\v c übrigens,
\index{Matijasevi\v c, Yuri}%
das zehnte Hilbertsche Problem etwas überraschend zu lösen: es gibt
keinen Algorithmus, mit dem entschieden werden kann, ob eine
diophantische Gleichung eine Lösung hat. Es ist also durchaus
nicht selbstverständlich, wenn ein Problem eine algorithmische
Lösung hat. Wir werden im nächsten Kapitel sehen, dass es viele
Probleme gibt, die keine algorithmische Lösung haben. Im übernächsten
Kapitel werden wir dann Probleme kennenlernen, die zwar von einem Computer
gelöst werden können, deren Lösung aber länger dauert als man
jeden realistischen Computer laufen lassen kann.
Die im folgenden entwickelte Theorie zeigt also, wo die Grenzen
der Berechenbarkeit mit Computern liegen.
