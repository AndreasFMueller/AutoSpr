%
% checklist.tex -- Checkliste zur Pruefungsvorbereitung
%
% (c) 2009 Prof. Dr. Andreas Mueller, HSR
% $Id: skript.tex,v 1.34 2008/11/02 22:46:16 afm Exp $
%
\documentclass[a4paper,12pt,twocolumn]{article}
\usepackage{geometry}
\geometry{papersize={200mm,280mm},total={180mm,250mm}}
\usepackage{german}
\usepackage{times}
\usepackage{alltt}
\usepackage{verbatim}
\usepackage{fancyhdr}
\usepackage{amsmath}
\usepackage{amssymb}
\usepackage{amsfonts}
\usepackage{amsthm}
\usepackage{textcomp}
\usepackage{graphicx}
\usepackage{array}
\usepackage{ifthen}
\usepackage{multirow}
\usepackage{txfonts}
\usepackage{paralist}
\begin{document}
\title{Pr"ufungsvorbereitungscheckliste:\\ Automaten und Sprachen}
\date{}
\maketitle
\section{Begriffe}
\begin{compactenum}
\item Alphabet
\item Wort
\item leeres Wort
\item Sprache
\item regul"ar
\item DEA
\item nicht deterministisch
\item NEA
\item akzeptieren
\item akzeptierte Sprache eines DEA
\item regul"arer Ausdruck
\item Grammatik
\item kontextfrei
\item Variable
\item Regel
\item Ableitung
\item Parse Tree
\item erzeugte Sprache einer Grammatik
\item Stackautomat
\item Turing-Maschine
\item erkannte Sprache einer TM
\item Entscheider
\item entscheidbare Sprache
\item Akzeptanzproblem
\item abz"ahlbar unendlich
\item "uberabz"ahlbar
\item Reduktion
\item polynomielle Laufzeit
\item P und NP
\item NP-vollst"andig
\item polynomielle Reduktion
\item SAT
\item 3SAT
\item Clique
\item Turing-vollst"andig
\item LOOP
\end{compactenum}
\vfill
\section{Fragen}
\begin{compactenum}
% Grundlagen
\item Was ist $\Sigma^*$?
\item Was ist der Unterschied zwischen $\varepsilon$, $\emptyset$ und
$\{\varepsilon\}$?
% reguläre Sprachen
\item Wie unterscheidet sich ein deterministischer endlicher Automat
von einem nicht deterministischen endlichen Automaten?
\item Wie kann man zwei endliche Automaten vergleichen?
\item Nennen Sie drei Methoden, mit denen Sie zeigen k"onnen, dass eine Sprache
regul"ar ist.
\item Nennen Sie zwei Methoden, mit denen Sie zeigen k"onnen, dass eine Sprache
nicht regul"ar ist.
\item Beschreiben Sie den Zusammenhang zwischen DEAs, NEAs und regul"aren
Ausdr"ucken.
\item Beschreiben Sie die f"unf typischen Schritte, die notwendig sind,
um mit dem Pumping Lemma zu beweisen, dass eine Sprache nicht regul"ar ist.
\item Beschreiben Sie DEAs f"ur die Sprachen $\emptyset$, $\{\varepsilon\}$
und $\Sigma^*$.
\item Warum sind endliche Sprachen regul"ar?
\item Geben Sie ein typisches Beispiel f"ur eine nicht regul"are Sprache.
% CFG
\item Was ist eine kontextfreie Grammatik?
\item Was bedeutet $w\in L(G)$?
\item Welche Eigenschaften hat eine kontextfreie Grammatik in
Chomsky-Normalform?
\item Wie kann man eine Grammatik in Chomsky-Normalform bringen?
\item Ist die Chomsky-Normalform eindeutig?
\item Was m"ussen Sie tun um nachzuweisen, dass eine Sprache kontextfrei ist?
\item Wie funktioniert ein Stack-Automat?
\item Welche Eigenschaften muss eine Sprache haben, damit es einen
Stack-Automaten gibt, der sie akzeptieren kann?
\item Beschreiben Sie eine Technik, mit der Sie zeigen k"onnen, dass eine
Sprache nicht kontextfrei ist.
\item Geben Sie Grammatiken an f"ur die Sprachen $\emptyset$,
$\{\varepsilon\}$ und $\Sigma^*$.
\item Beschreiben Sie die f"unf typischen Schritte, die notwendig sind,
um mit dem Pumping Lemma zu beweisen, dass eine Sprache nicht regul"ar ist.
\item Geben Sie ein typisches Beispiel f"ur eine nicht kontextfreie Sprache.
% TM
\item Gibt es eine Turing-Maschine mit nur einem Zustand?
\item Wieviele verschiedene Sprachen k"onnen von 
Turing-Maschinen mit zwei Zust"anden erkannt werden. Warum?
\item Z"ahlen Sie drei Varianten von Turing-Maschinen auf.
\item Sei $M$ eine nicht deterministische Turing-Maschine und $w\in\Sigma^*$.
Was heisst $w\in L(M)$?
\item Warum gibt es Sprachen, die nicht Turing-erkennbar sind?
\item Was ist der Unterschied zwischen einer Turing erkennbaren Sprache
und einer Turing entscheidbaren Sprache?
% Entscheidbarkeit
\item Was m"ussen Sie tun um nachzuweisen, dass eine turing-erkennbare
Sprache entscheidbar ist?
\item Beschreiben Sie das prototypische nicht entscheidbare Problem f"ur
Turing-Maschinen.
\item Was bedeutet Reduktion eines Problems auf ein anderes?
\item Erkl"aren Sie eine Standardtechnik, mit der man nachweisen kann,
dass ein Problem nicht entscheidbar ist.
\item Geben Sie ein Beispiel eines nicht entscheidbaren Problems.
\item Was besagt das Halte-Theorem?
% Komplexität
\item Wie "andert sich die Laufzeit eines Algorithmus, wenn man von einer
Variante einer TM zu einer Standard TM "ubergeht?
\item Was bedeutet polynomielle Reduktion?
\item Wie "andert sich die Laufzeit, wenn man eine nichtdeterministische
Maschine auf einer deterministischen Maschine simuliert?
\item Was bedeutet das Problem \textsl{SAT}?
\item Was unterscheidet ein Problem wie \textsl{SAT} von einem Problem wie
die ganzzahlige Division von Zahlen?
\item Wie k"onnen Sie herausfinden, ob eine Programmiersprache
Turing-vollst"andig ist?
\item Warum ist die Sprache \text{LOOP} nicht Turing-vollst"andig?
\item Gibt es Probleme, die man \text{WHILE} l"osen kann, nicht
aber mit \text{GOTO}?
\item Ist es m"oglich, einen Compiler zu schreiben, der C-Code in 
Brainfuck "ubersetzt?
\end{compactenum}
\end{document}
