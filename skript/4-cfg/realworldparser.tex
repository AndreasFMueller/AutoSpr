%
% realworldparser.tex
%
% (c) 2019 Prof Dr Andreas Müller, Hochschule Rapperswil
%
\rhead{Praktische Parser}
\section{Real World Parser}
\subsection{Deterministische Parser}
Kontextfreie Sprachen können von einem nichtdeterministischen
Stackautomaten analysiert werden. Ein deterministischer Algorithmus mit
kubischer Laufzeit ist verfügbar. Doch kubische Laufzeit ist bei grossem
Input immer noch sehr gross, eine Verdoppelung der Inputgrösse hat eine
Verachtfachung der Laufzeit zur Folge. 
Für praktische Zwecke der Syntaxanalyse
zum Beispiel in einem Compiler wird daher ein Verfahren benötigt, welches
Wörter deterministisch in linearer Zeit erkennen kann.

Leider ist das in voller Allgemeinheit nicht möglich. Daher wurden Parser-Techniken
erfunden, welche deterministisch und in linearer Laufzeit Sprachen erkennen,
deren Grammatiken zusätzliche Eigenschaften haben. Natürlich sollen diese
Techniken möglichst nicht mehr als einen Stackautomaten benötigen, den Input
von links nach rechts lesen und wenn möglich laufend die Informationen ausgeben,
die zum Beispiel für die Erzeugung von Code in einem Compiler benötigt wird.
Eine besonders wichtige Familie von Parsern sind die $LR(k)$-Parser.
Ihr Name leitet sich aus der Tatsache ab, dass der Input von Links gelesen wird,
dabei Rechtsreduktionen durchgeführt werden und höchstens $k$ Inputzeichen
vorausgelesen werden müssen, um zu entscheiden, welche Produktionsregel der
Grammatik für die Reduktion angewendet werden soll.

Der nicht deterministische Stackautomat ist so konstruiert worden, dass er aus
der Startvariablen auf dem Stack durch Anwendung der Produktionsregeln 
ein Wort konstruiert. Das Resultat der bereits angewendeten Regeln
liegt auf dem Stack. Sobald zuoberst auf dem Stack ein Terminalsymbol auftaucht,
kann es mit dem nächsten Zeichen des Input verglichen und bei Übereinstimmung
gelesen werden. Das Input-Wort wird akzeptiert, wenn der Stack und der Input
gleichzeitig leer werden. Nicht deterministisch ist jeweils die Auswahl der
Produktionsregel.

Der Prozess wird also sozusagen vom Orakel gesteuert,
welches die Produktionsregeln vorschlägt. Der Input spielt dabei gar keine
Rolle, man hat es nur den guten Eigenschaften des Orakels zu verdanken,
dass der Input auf den vom Prozess erzeugten String passt.

\subsection{Shift-Reduce}
\index{shift-reduce Parser}%
Ein deterministischer Prozess muss stattdessen vom Input kontrolliert werden.
Das Input-Zeichen oder der aktuelle Stackinhalt müssen bestimmen, was
als nächstes zu geschehen hat. Wie das erreicht werden könnte, kann man am
Beispiel der Sprache $L=\{\texttt{0}^n\texttt{1}^n|n> 0\}$ erahnen, welche die
Grammatik
\begin{align}
w&\rightarrow \texttt{0}\;w\;\texttt{1}\tag{1}\\
&\rightarrow \texttt{01}\tag{2}
\end{align}
hat. Ein Parser könnte bei der Analyse des Wortes $\texttt{000111}$ wie folgt vorgehen.
Er schiebt Zeichen um Zeichen vom Input auf den Stack. Sobald
er erkennt, dass auf dem Stack eine Folge von Zeichen liegt, die auf die rechte
Seite einer Regel passt, wendet er die Regel an, reduziert das Wort zu einer
Variablen und legt das Resultat auf den Stack. Dies ist in Tabelle \ref{shiftreduce}
dargestellt.

\begin{table}[H]
\begin{center}
\begin{tabular}{|r|l|l|}
\hline
Input&Stack&Operation\\
\hline
$\texttt{000111}$&&Schiebe\\
$\texttt{00111}$&$\texttt{0}$&Schiebe\\
$\texttt{0111}$&$\texttt{00}$&Schiebe\\
$\texttt{111}$&$\texttt{000}$&Schiebe\\
$\texttt{11}$&$\texttt{0001}$&Reduziere mit Regel 2\\
$\texttt{11}$&$\texttt{00}w$&Schiebe\\
$\texttt{1}$&$\texttt{00}w\texttt{1}$&Reduziere mit Regel 1\\
$\texttt{1}$&$\texttt{0}w$&Schiebe\\
&$\texttt{0}w\texttt{1}$&Reduziere mit Regel 1\\
&$w$&\\
\hline
\end{tabular}
\end{center}

\caption{Vorgehen eins Shift-Reduce-Parser am Beispiel der Sprache $L=\{\texttt{0}^n\texttt{1}^n|n>0\}$
\label{shiftreduce}}
\end{table}
Offenbar konnte das Wort durch systematisches Anwenden der Regeln aus dem Input
auf die Startvariable zurückgeführt werden.
In der dritten Spalte kann man nachlesen, wie der Parse-Tree aufgebaut ist,
die Regeln treten dabei in der Reihenfolge ``von unten nach oben'' auf.

Ein solcher Parser heisst eine Shift-Reduce-Parser, er baut den Parse-Tree von unten
her (bottom-up) auf, und operiert offenbar vollständig deterministisch.
Er kann jedoch nicht mit einem Stackautomaten implementiert werden,
weil Die Aufforderung ``Schaue nach, ob die obersten $n$ Elemente auf dem Stack
auf die rechte Seite einer Regel passen'', ist für bei einem Stack nicht
durchführbar. Man kann immer nur das oberste Element einsehen.

\subsection{\texorpdfstring{$LR(0)$}{LR(0)}-Elemente}
\index{LR(0)-Element@$LR(0)$-Element}%
Ein Stackautomat kann sich nur dann an den den Stackinhalt unterhalb des obersten
Stackelementes erinnern, wenn er diese Information in Zuständen codieren kann.
Im Beispiel kamen folgende relevanten Stackinhalte vor 
\[
\texttt{0}
\qquad
\texttt{01}
\qquad
\texttt{0}w
\qquad
\texttt{0}w\texttt{1}
\]
Die Inhalte $\texttt{01}$ und $\texttt{0}w\texttt{1}$ haben die Anwendung einer wohl bestimmten
Reduktionsregel
hervorgerufen, die anderen zwei waren nur Vorstufen dazu.
Fasst man diese Wörter
zusammen als Namen der Zustände, können wir den Parse-Prozess wie in
Tabelle \ref{lrelemente} durchführen.
\begin{table}[H]
\begin{center}
\begin{tabular}{|r|l|c|l|l|}
\hline
Input&Stack&Zustand&Reduktion&Operation\\
\hline
$\texttt{000111}$
	&
		&&&Schiebe $\texttt{0}$\\
$\texttt{00111}$
	&$\boxed{\texttt{0}}$
		&$\boxed{\texttt{0}}$&&Schiebe $\texttt{0}$\\
$\texttt{0111}$
	&$\boxed{\texttt{0}}\,\boxed{\texttt{0}}$
		&$\boxed{\texttt{0}}$&&Schiebe $\texttt{0}$\\
$\texttt{111}$
	&$\boxed{\texttt{0}}\,\boxed{\texttt{0}}\,\boxed{\texttt{0}}$
		&$\boxed{\texttt{0}}$&&Schiebe $\texttt{1}$\\
$\texttt{11}$
	&$\boxed{\texttt{0}}\,\boxed{\texttt{0}}\,\boxed{\texttt{0}}\,\boxed{\texttt{01}}$
		&$\boxed{\texttt{01}}$&&Reduziere, Regel 2\\
$\texttt{11}$
	&$\boxed{\texttt{0}}\,\boxed{\texttt{0}}$
		&$\boxed{\texttt{0}}$&$w$&Schiebe $w$\\
$\texttt{11}$
	&$\boxed{\texttt{0}}\,\boxed{\texttt{0}}\,\boxed{\texttt{0}w}$
		&$\boxed{\texttt{0}w}$&&Schiebe $\texttt{1}$\\
$\texttt{1}$
	&$\boxed{\texttt{0}}\,\boxed{\texttt{0}}\,\boxed{\texttt{0}w}\,\boxed{\texttt{0}w\texttt{1}}$
		&$\boxed{\texttt{0}w\texttt{1}}$&&Reduziere, Regel 1\\
$\texttt{1}$
	&$\boxed{\texttt{0}}$
		&$\boxed{\texttt{0}}$&$w$&Schiebe $w$\\
$\texttt{1}$
	&$\boxed{\texttt{0}}\,\boxed{\texttt{0}w}$
		&$\boxed{\texttt{0}w}$&&Schiebe $\texttt{1}$\\
	&$\boxed{\texttt{0}}\,\boxed{\texttt{0}w}\,\boxed{\texttt{0}w\texttt{1}}$
		&$\boxed{\texttt{0}w\texttt{1}}$&&Reduziere, Regel 1\\
	&
		&$\boxed{\phantom{\texttt{0}}}$&$w$&zum Endzustand\\
\hline
\end{tabular}
\end{center}

\caption{Shift-Reduce mit $LR(0)$-Elementen\label{lrelemente}}
\end{table}
Der Automat legt also immer seinen neuen Zustand zuoberst auf den
Stack, behält aber die alten Zustände. Der Name des Zustandes ist
die teilweise gelesene rechte Seite einer Regel, man kann also aus
dem Namen des Zustandes auch ablesen, welche Regeln für den nächsten
Reduktionsschritt in Frage kommen könnten.

Wir markieren innerhalb einer Regel die ``Position des Lesers'' mit
einem Punkt, wir erhalten dann aus der Grammatik
\begin{align}
\cdot\;\texttt{0}\;w\;\texttt{1}&\leftarrow w\tag{1}\\
\texttt{0}\;\cdot\;w\;\texttt{1}&\leftarrow w\tag{1}\\
\texttt{0}\;w\;\cdot\;\texttt{1}&\leftarrow w\tag{1}\\
\texttt{0}\;w\;\texttt{1}\;\cdot&\leftarrow w\tag{1}\\
\cdot\;\texttt{0}\;\texttt{1}&\leftarrow w\tag{2}\\
\texttt{0}\;\cdot\;\texttt{1}&\leftarrow w\tag{2}\\
\texttt{0}\;\texttt{1}\;\cdot&\leftarrow w\tag{2}
\end{align}
Der Teil links vom $\cdot$ zeigt an, was bereits gelesen wurde,
rechts davon findet
man die Zeichen, die als nächste gelesen werden dürfen.
Befindet sich $\cdot$
am Ende des Wortes, kann die Reduktion agewendet werden.
Man nennt die so erweiterten
Produktionsregeln die {\em $LR(0)$-Elemente}.

\subsection{Übergangsfunktion des \texorpdfstring{$LR(0)$}{LR(0)}-Parsers}
Wenn eine Reduktionsregel angewendet werden kann, werden so viele Elemente
aus dem Stack entfernt, wie die rechte Seite der Regel Zeichen hat. Das
neue oberste Zeichen auf dem Stack ist der letzte Zustand des Automaten
bevor die jetzt reduzierten Zeichen gelesen wurden. Dieser Zustand
bestimmt zusammen mit dem Reduktionsresultat den nächsten Zustand
des Automaten.

Es gilt jetzt zu überprüfen, dass diese Operationen tatsächlich mit
Hilfe eines Stackautomaten durchführbar sind. Die einfachste ist natürlich
die Shift-Ope\-ration. Da diese jedoch abhängig ist vom aktuellen Element zuoberst
auf dem Stack, braucht es dafür zwei Schritte:
\[
\boxed{\texttt{0}}\xrightarrow{\texttt{0},\scalebox{0.7}{\boxed{\texttt{0}}$\rightarrow$\boxed{\texttt{0}}}}\cdot
\xrightarrow{\varepsilon,\varepsilon\rightarrow\scalebox{0.7}{\boxed{\texttt{0}}}}\boxed{\texttt{0}}
\]
oder
\[
\boxed{\texttt{0}}\xrightarrow{\texttt{1},\scalebox{0.7}{\boxed{\texttt{0}}$\rightarrow$\boxed{\texttt{0}}}}\cdot
\xrightarrow{\varepsilon,\varepsilon\rightarrow\scalebox{0.7}{\boxed{\texttt{01}}}}\boxed{\texttt{01}},
\]
in beiden Fällen wird ein neuer Zwischenzustand benötigt, symbolisiert mit dem Zeichen
`$\cdot$'.

Die Reduktion mit der Regel 2 muss zwei Zustände vom Stack entfernen.
Wenn anschliessend der Zustand $\boxed{\texttt{0}}$ auf dem Stack liegt, ist der
nächste Zustand $\boxed{\texttt{0}w}$. Dies kann man mit folgenden Übergängen
realisieren:
\[
\boxed{\texttt{01}}
\xrightarrow{\varepsilon,\scalebox{0.7}{\boxed{\texttt{01}}}\to\varepsilon}
\cdot
\xrightarrow{\varepsilon,\scalebox{0.7}{\boxed{\texttt{0}}}\to\varepsilon}
\cdot
\xrightarrow{\varepsilon,\scalebox{0.7}{\boxed{\texttt{0}}}\to\scalebox{0.7}{\boxed{\texttt{0}}}}
\cdot
\xrightarrow{\varepsilon,\varepsilon\to\scalebox{0.7}{\boxed{\texttt{0}w}}}
\boxed{\texttt{0}w}
\]
Analog kann die Reduktion mit der Regel 1 durchgeführt werden:
\[
\boxed{\texttt{0}w\texttt{1}}
\xrightarrow{\varepsilon,\scalebox{0.7}{\boxed{\texttt{0}w\texttt{1}}}\to\varepsilon}
\cdot
\xrightarrow{\varepsilon,\scalebox{0.7}{\boxed{\texttt{0}w}}\to\varepsilon}
\cdot
\xrightarrow{\varepsilon,\scalebox{0.7}{\boxed{\texttt{0}}}\to\varepsilon}
\cdot
\xrightarrow{\varepsilon,\scalebox{0.7}{\boxed{\texttt{0}}}\to\scalebox{0.7}{\boxed{\texttt{0}}}}
\cdot
\xrightarrow{\varepsilon,\varepsilon\to\scalebox{0.7}{\boxed{\texttt{0}w}}}
\boxed{\texttt{0}w}
\]
Wir können diese Übergänge etwas einfacher schreiben, wenn wir in der
Beschriftung der Pfeile erlauben, mehrere Operationen zusammenzufassen.
Die Shift-Ope\-rationen werden dann zu
\begin{gather*}
\boxed{\texttt{0}}\xrightarrow{\texttt{0},\scalebox{0.7}{\boxed{\texttt{0}}$\rightarrow$\boxed{\texttt{0}}\,\boxed{\texttt{0}}}}\boxed{\texttt{0}}
\\
\boxed{\texttt{0}}\xrightarrow{\texttt{1},\scalebox{0.7}{\boxed{\texttt{0}}$\rightarrow$\boxed{\texttt{0}}\,\boxed{\texttt{01}}}}\boxed{\texttt{01}}
\end{gather*}
die Reduktionsoperationen zu
\begin{gather*}
\boxed{\texttt{01}}
\xrightarrow{\varepsilon,\scalebox{0.7}{\boxed{\texttt{0}}\,\boxed{\texttt{01}}}\to\varepsilon}
\cdot
\xrightarrow{\varepsilon,\scalebox{0.7}{\boxed{\texttt{0}}}\to\scalebox{0.7}{\boxed{\texttt{0}}\,\boxed{\texttt{0}w}}}
\boxed{\texttt{0}w}
\\
\boxed{\texttt{0}w\texttt{1}}
\xrightarrow{\varepsilon,\scalebox{0.7}{\boxed{\texttt{0}}\,\boxed{\texttt{0}w}\,\boxed{\texttt{0}w\texttt{1}}}\to\varepsilon}
\cdot
\xrightarrow{\varepsilon,\scalebox{0.7}{\boxed{\texttt{0}}}\to\scalebox{0.7}{\boxed{\texttt{0}}\,\boxed{\texttt{0}w}}}
\boxed{\texttt{0}w}
\end{gather*}
In diesem einfachen Bespiel gibt es nur eine Möglichkeit für den zweiten
Schritt, in einer komplexeren Grammatik könnte nach der Reduktion jedoch
eine Vielzahl anderer Zustände auf dem Stack zurückbleiben, so dass je
nach vorherigem Zustand $q$ weitere Übergänge der Form
\[
\cdot\xrightarrow{\varepsilon,q\to q q'}q'
\]
hinzugenommen werden müssen. Es folgt, dass obiger $LR(0)$-Parser mit
Hilfe eines Stackautomaten implementiert werden kann.

\subsection{Parser-Tabellen}
Da gezeigt wurde, dass der Beispiel-$LR(0)$-Parser in einem Stackautomaten
implementiert werden kann, kann die Übergangsfunktion mit Hilfe
einer Tabelle codiert werden. Die Zeilen dieser Tabelle sind die
Zustände, also die $LR(0)$-Elemente. In den Spalten müssen die
Schiebe- und Reduktions-Schritte codiert sein. In einem Zustand
entscheidet man sich immer entweder für einen Reduktionsschritt
unabhängig vom Input, oder für eine Schiebeoperation, wobei
nicht alle Input-Zeichen akzeptabel sind, und verschiedene Eingabezeichen
zu verschiedenen Ausgabezuständen führen können. Das Beispiel
$L=\{\texttt{0}^n\texttt{1}^n|n>0\}$ führt auf Tabelle \ref{parsetable}.
\begin{table}[H]
\begin{center}
\begin{tabular}{|c|c|cc|c|}
\hline
Zustand&$LR(0)$-Element&$0$&$1$&$w$\\
\hline
0&$\boxed{\phantom{\texttt{0}}}$&s1&&\\
1&$\boxed{\texttt{0}}$&s1&s3&4\\
2&$\boxed{w}$&&&\\
3&$\boxed{\texttt{01}}$&r2&r2&\\
4&$\boxed{\texttt{0}w}$&&s5&\\
5&$\boxed{\texttt{0}w\texttt{1}}$&r1&r1&\\
\hline
\end{tabular}
\end{center}

\caption{Parsertabelle für $L=\{\texttt{0}^n\texttt{1}^n|n>0\}$\label{parsetable}}
\end{table}
Die Tabelle ist wie folgt zu lesen. In Spalten 3 und 4 wird jeweils
angegeben, was für eine Aktion ausgeführt werden soll, wenn das
Zeichen $\texttt{0}$ bzw.~$\texttt{1}$ vom Input gelesen wird. Ein s$n$ bedeutet, dass
eine Schiebeoperation ausgeführt wird, und anschliessend in den
Zustand $n$ gewechselt wird. Dagegen bedeutet r$n$,
dass die Reduktionsregel mit der Nummer $n$ angewendet wird.
Dazu gehören folgende Schritte:
\begin{enumerate}
\item
Entferne so viele Elemente vom Stack, wie die rechte Seite der angewendeten
Regel lang ist.
\item
Bestimme den Zustand, der durch das oberste Element auf dem Stack
angezeigt wird.
\item
Bestimme in der eben bestimmten Zeile unter dem Resultat der Reduktion,
welches der nächste Zustand ist.
\item
Lege diesen Zustand zuoberst auf den Stack.
\end{enumerate}
\index{Bison}%
\index{Yacc}%
Die Konstruktion der Parsertabellen ist aufwendig und fehlerträchtig,
doch stehen dafür Softwarewerkzeuge zur Verfügung, zum Beispiel
der Parser-Generator Bison.

\subsection{Konflikte}
\index{Konflikt}%
Diese Art von Parser kann funktionieren, wenn jederzeit klar ist, ob
geschoben werden kann, oder ob eine Reduktion erfolgen soll.
Sind mehrere Operationen möglich, spricht man von einem Konflikt.
\subsubsection{Reduce-Reduce Konflikt}
\index{Konflikt!reduce-reduce}%
In der folgenden Grammatik ist nicht klar, welche der beiden Reduktionsregeln
angewendet werden soll
\begin{align*}
A&\rightarrow \texttt{01}\\
B&\rightarrow \texttt{01}
\end{align*}

\subsubsection{Shift-Reduce Konflikt}
\index{Konflikt!shift-reduce}%
In der folgenden Grammatik ist nach dem Lesen einer $1$ nicht klar
\begin{align*}
A&\rightarrow \texttt{1}\tag{1}\\
A&\rightarrow \texttt{1}\,A\tag{2}
\end{align*}
ob man die Regel (1) anwenden soll (Reduce), oder ob man
in der Hoffnung, irgendwann ein $A$ zu finden, weiterlesen
soll. Diese Frage kann offenbar nur dadurch entschieden werden, dass
man vorausschaut. Dies ist in einem $LR(0)$-Parser nicht erlaubt.
Formuliert man obige Grammatik jedoch mit den Regeln
\begin{align*}
A&\rightarrow \texttt{1}\tag{1}\\
A&\rightarrow A\,\texttt{1}\tag{2}
\end{align*}
kann die Regel (1) jederzeit angewendet werden, der Konflikt ist
weggefallen. 

\section{Real existierende Parser}
Parsergeneratoren wie Yacc oder Bison erweitern die gegebene Grammatik
selb\-ständig um zusätzliche Symbole, die ihnen helfen, das Ende
des Input oder des Stack zu erkennen. Sie fügen dazu eine
Regel
\[
\text{\$accept}\rightarrow S\; \text{\$end}
\]
hinzu. Wenn also am Ende des Input alles auf die Variable $S$ reduziert
werden konnte, liegt offenbar ein Wort vor, welches akzeptiert werden
muss.

\section{Arithmetische Ausdrücke}
\index{expression-term-factor Grammatik}%
\index{Ausdrücke!arithmetische}%
Hier soll an Hand einer Beispiel-Grammatik gezeigt werden, wie ein 
von Bison erzeugter Parser einen arithmetischen Ausdruck
erkennt. Wir verwenden die Grammatik:
\begin{align}
\textsl{expr}&\rightarrow \textsl{term}\tag{2}\\
\textsl{expr}&\rightarrow \textsl{expr}\;\text{'{\tt +}'}\;\textsl{term}\tag{3}\\
\textsl{expr}&\rightarrow \textsl{expr}\;\text{'{\tt -}'}\;\textsl{term}\tag{4}\\
\textsl{term}&\rightarrow \textsl{factor}\tag{5}\\
\textsl{term}&\rightarrow \textsl{term}\;\text{'{\tt *}'}\;\textsl{factor}\tag{6}\\
\textsl{term}&\rightarrow \textsl{term}\;\text{'{\tt /}'}\;\textsl{factor}\tag{7}\\
\textsl{factor}&\rightarrow\text{'{\tt (}'}\;\textsl{expr}\;\text{'{\tt )}'}\tag{8}\\
\textsl{factor}&\rightarrow\textsl{number}\tag{9}\\
\textsl{number}&\rightarrow\textsl{number}\;\textsl{digit}\tag{10}\\
\textsl{number}&\rightarrow\textsl{digit}\tag{11}\\
\textsl{digit}&\rightarrow\text{'{\tt 0}'}\tag{12}\\
\vdots&\qquad\vdots\notag\\
\textsl{digit}&\rightarrow\text{'{\tt 9}'}\tag{21}
\end{align}
Die Nummerierung der Regeln bezieht sich auf die später gezeigten Regeln
der von Bison erzeugten Parsetabellen. Damit diese Grammatik in einem
Programm einfach demonstriert werden kann ist es sinnvoll, ein Endzeichen
für die arithmetischen Ausdrücke einzuführen. Wir fügen zu diesem
Zweck die Regel 
\begin{equation}
\textsl{exprline}\rightarrow \textsl{expr}\;\text{'{\tt $\setminus$n}'}
\tag{1}
\end{equation}
hinzu. Bison fügt dann seinerseits eine Regel mit der Nummer 0 hinzu, so ergibt
sich folgende Grammatik in Bisons Schreibweise:
\verbatiminput{4-cfg/bison/bisongrammar.txt}
Man kann Bison nicht nur mit einer Kommandozeilenoption dazu bringen, diese
Grammatik im Klartext auszugeben, er zeigt auch die erzeugten Parsetabellen
an. Exemplarisch soll dargestellt werden, wie Zustand 14 verarbeitet
wird:
\verbatiminput{4-cfg/bison/state14.txt}
Das zugehörige $LR(0)$-Element beginnt also mit \text{term}. Die Zeichen
`{\tt *}' und `{\tt /}' lösen jeweils eine Schiebeoperation aus, andernfalls
wird mit Hilfe der Regel 2 reduziert.

