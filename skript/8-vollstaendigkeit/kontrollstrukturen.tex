%
% kontrollstrukturen.tex
%
% (c) 2011 Prof Dr Andreas Mueller, Hochschule Rapperswil
%
\section{Kontrollstrukturen und Turing-Vollständigkeit}
\rhead{Kontrollstrukturen}
Das Turing-Vollstandigkeits-Kriterium von Satz
\ref{turingvollstaendigkeitskriterium} verlangt, dass man einen
Turing-Maschinen-Simulator in der gewählten Sprache schreiben muss.
Dies ist in jedem Fall eine nicht triviale Aufgabe.
Daher wäre es nützlich Kriterien zu erhalten, welche einfacher
anzuwenden sind. Einen wesentlichen Einfluss auf die Möglichkeiten,
was sich mit einer Programmiersprache ausdrücken lassen, haben die
Kontrollstrukturen.

\newcommand{\assignment}{\mathbin{\texttt{:=}}}

\subsection{Grundlegende Syntaxelemente%
\label{subsection:grundlegende-syntaxelement}}
In den folgenden Abschnitten werden verschiedene vereinfachte Sprachen
diskutiert, die aber einen gemeinsamen Kern fundamentaler Anweisungen
beinahlten.
In allen Sprachen gibt es nur einen einzigen Datentyp, nämlich
natürliche Zahlen.
Einerseits können Konstanten beliebig grosse natürliche Werte haben,
andererseits lassen sich in Variablen beliebig grosse natürliche Zahlen
speichern.
Dazu sind die folgenden Sytanxelemente notwendig:
\begin{compactitem}
\item Konstanten: {\tt 0}, {\tt 1}, {\tt 2}, \dots
\item Variablen $x_0$, $x_1$, $x_2$,\dots
\item Zuweisung: $\assignment$
\item Trennung von Anweisungen: {\tt ;}
\item Operatoren: {\tt +} und {\tt -}
\end{compactitem}
Die einzige Möglichkeit, den Wert einer Variablen zu ändern ist die
Zuweisung.  Diese ist entweder die Zuweisung einer Konstante als
Wert einer Variable:
\begin{algorithmic}
\STATE $x_i\assignment c$
\end{algorithmic}
oder eine Berechnung mit den beiden vorhandenen Operatoren
\begin{algorithmic}
\STATE $x_i \assignment x_j$ {\tt +} $c$
\STATE $x_i \assignment x_j$ {\tt -} $c$
\end{algorithmic}
Dabei ist das Resultat der Subtraktion als $0$ definiert, wenn
der Minuend kleiner ist als der Subtrahend.

Die verschiedenen Sprachen unterschieden sich in den Kontrollstrukturen
mit denen diese Zuweisungsbefehle miteinander verbunden werden können.

\subsection{LOOP}
\index{LOOP}%
Die Programmiersprache
LOOP\footnote{Die in diesem Abschnitt beschriebene
LOOP Sprache darf nicht verwechselt werden mit dem gleichnamigen
Projekt einer objektorientierten parallelen Sprache seit
2001 auf Sourceforge.}
hat als einzige Kontrollstruktur die
Iteration eines Anweisungs-Blocks mit einer festen, innerhalb des
Blocks nicht veränderbaren Anzahl von Durchläufen.

\subsubsection{Syntax}
LOOP verwendet die 
Schlüsselwörter: {\tt LOOP}, {\tt DO}, {\tt END}
Programme werden daraus wie folgt aufgebaut:
\begin{itemize}
\item Das leere Programm $\varepsilon$ ist eine LOOP-Programm,
während der Ausführung tut es nichts.
\item Wertzuweisungen sind LOOP-Programme
\item Sind $P_1$ und $P_2$ LOOP-Programme, dann auch
$P_1{\tt ;}P_2$. Um dieses Programm auszuführen wird zuerst $P_1$
ausgeführt und anschliessend $P_2$.
\item Ist $P$ ein LOOP-Programm, dann ist auch
\begin{algorithmic}
\STATE {\tt LOOP} $x_i$ {\tt DO} $P$ {\tt END}
\end{algorithmic}
ein LOOP-Programm. Die Ausführung wiederholt $P$ so oft, wie der
Wert der Variable $x_i$  zu Beginn angibt.
\end{itemize}
Durch Setzen der Variablen $x_i$ kann einem LOOP-Progamm Input übergeben
werden.
Ein LOOP-Programm definiert also eine Abbildung $\mathbb N^k\to\mathbb N$,
wobei $k$ die Anzahl der Variablen ist, in denen Input zu übergeben ist.

\subsubsection{Beispiel: Summer zweier Variablen}
LOOP kann nur jeweils eine Konstante hinzuaddieren, das genügt aber
bereits, um auch die Summe zweier Variablen zu berechnen. Der folgende
Code berechnet die Summe von $x_1$ und $x_2$ und legt das Resultat in
$x_0$ ab:
\begin{algorithmic}
\STATE $x_0 \assignment x_1$
\STATE {\tt LOOP} $x_2$ {\tt DO}
\STATE{\tt \ \ \ \ }$x_0 \assignment x_0$ {\tt +} $1$
\STATE{\tt END}
\end{algorithmic}

\subsubsection{Beispiel: Produkt zweier Variablen}
\index{Multiplikation}%
Die Sprache LOOP ist offenbar relativ primitiv, trotzdem
ist sie leistungsfähig genug, um die Multiplikation von zwei
Zahlen, die in den Variablen $x_1$ und $x_2$ übergeben werden,
zu berechnen:

\begin{algorithmic}
\STATE $x_0 \assignment 0$
\STATE {\tt LOOP} $x_1$ {\tt DO}
\STATE{\tt \ \ \ \ LOOP} $x_2$ {\tt DO}
\STATE{\tt \ \ \ \ \ \ \ \ }$x_0 \assignment x_0$ {\tt +} $1$
\STATE{\tt \ \ \ \ END}
\STATE{\tt END}
\end{algorithmic}

\subsubsection{IF-Anweisung}
In LOOP fehlt eine IF-Anweisung, auf die die wenigsten Programmiersprachen
verzichten. Es ist jedoch leicht, eine solche in LOOP nachzubilden.
Eine Anweisung
\begin{algorithmic}
\STATE {\tt IF} $x = 0$ {\tt THEN } $P$ {\tt END}
\end{algorithmic}
kann unter Verwendung einer zusätzlichen Variable $y$ in LOOP ausgedrückt
werden:
\begin{algorithmic}[1]
\STATE $y \assignment 1${\tt ;}
\STATE {\tt LOOP }$x${\tt\ DO }$y \assignment 0$ {\tt END;}
\STATE {\tt LOOP }$y${\tt\ DO }$P$ {\tt END;}
\end{algorithmic}
Natürlich wird die {\tt LOOP}-Anweisung in Zeile 2 meistens viel öfter
als nötig ausgeführt, aber es stehen hier ja auch nicht Fragen
der Effizienz sonder der prinzipiellen Machbarkeit zur Diskussion.

\subsubsection{LOOP ist nicht Turing-vollständig}
\begin{satz}
LOOP-Programme terminieren immer.
\end{satz}

\begin{proof}[Beweis]
Der Beweis kann mit Induktion über die Schachtelungstiefe
von {\tt LOOP}-Anweisungen geführt werden. Die LOOP-Programme
ohne {\tt LOOP}-Anweisung, also die Programme mit Schachtelungstiefe
$0$ terminieren offensichtlich immer. Ebenso die LOOP-Programme
mit Schachtelungstiefe $1$, da die Anzahl der Schleifendurchläufe
bereits zu Beginn der Schleife festliegt.

Nehmen wir jetzt an wir wüssten bereits, dass jedes LOOP-Programm mit
Schachtelungstiefe $n$ der {\tt LOOP}-Anweisungen immer terminiert.
Ein LOOP-Programm mit Schachtelungstiefe $n+1$ ist dann eine
Abfolge von Teilprogrammen mit Schachtelungstiefe $n$, die nach
Voraussetzung alle terminieren, und {\tt LOOP}-Anweisungen der
Form
\begin{algorithmic}
\STATE{\tt LOOP }$x_i${\tt\ DO }$P${\tt\ END}
\end{algorithmic}
wobei $P$ ein LOOP-Programm mit Schachtelungstiefe $n$ ist, das also
ebenfalls immer terminiert. $P$ wird genau so oft ausgeführt, wie
$x_i$ zu Beginn angibt. Die Laufzeit von $P$ kann dabei jedesmal
anders sein, $P$ wird aber auf jeden Fall terminieren. Damit ist
gezeigt, dass auch alle LOOP-Programme mit Schachtelungstiefe $n+1$
terminieren.
\end{proof}

\begin{satz}
LOOP ist nicht Turing vollständig.
\end{satz}

\begin{proof}[Beweis]
Es gibt Turing-Maschinen, die nicht terminieren. Gäbe es einen
Turing-Maschinen-Simulator in LOOP, dürfte dieser bei der
Simulation einer solchen Turing-Maschine nicht terminieren, im
Widerspruch zur Tatsache, dass LOOP-Programme immer terminieren.
Also kann es keinen Turing-Maschinen-Simulator in LOOP geben.
\end{proof}

\subsection{WHILE}
\index{WHILE}%
WHILE-Programme können zusätzlich zur {\tt LOOP}-Anweisung
eine {\tt WHILE}-Anweisung der Form
\begin{algorithmic}
\STATE{\tt WHILE }$x_i>0${\tt\ DO }$P${\tt\ END}
\end{algorithmic}
Dadurch ist die {\tt LOOP}-Anweisung nicht mehr unbedingt
notwendig, denn
\begin{algorithmic}
\STATE{\tt LOOP }$x${\tt\ DO }$P${\tt\ END}
\end{algorithmic}
kann durch
\begin{algorithmic}
\STATE$y \assignment x$;
\STATE{\tt WHILE }$y>0${\tt\ DO }$P$; $y \assignment y-1${\tt\ END}
\end{algorithmic}
nachgebildet werden.

\subsection{GOTO}
\index{GOTO}%
GOTO-Programme bestehen aus einer markierten Folge von Anweisungen
\begin{center}
\begin{tabular}{rl}
$M_1$:&$A_1$\\
$M_2$:&$A_2$\\
$M_3$:&$A_3$\\
$\dots$:&$\dots$\\
$M_k$:&$A_k$
\end{tabular}
\end{center}
Zu den bereits bekannten,
Abschnitt~\ref{subsection:grundlegende-syntaxelement} beschriebenen
Zuweisungen
$x_i \assignment c$ 
und
$x_i \assignment x_j \pm c$ 
kommt bei GOTO eine bedingte Sprunganweisung
\begin{center}
\begin{tabular}{rl}
$M_l$:&{\tt IF\ }$x_i=c${\tt\ THEN GOTO\ }$M_j$
\end{tabular}
\end{center}
Natürlich lässt sich damit auch eine unbedingte Sprunganweisung
implementieren:
\begin{center}
\begin{tabular}{rl}
$M_l$:&$x_i \assignment c$;\\
$M_{l+1}$:&{\tt IF\ }$x_i=c${\tt\ THEN GOTO\ }$M_j$
\end{tabular}
\end{center}
In ähnlicher Weise lassen sich auch andere bedingte Anweisungen
konstruieren, zum Beispiel ein
Konstrukt {\tt IF }\dots{\tt\ THEN }\dots{\tt\ ELSE }\dots{\tt\ END}, welches
einen ganzen Anweisungsblock enthalten kann.

\subsection{Äquivalenz von WHILE und GOTO}
Die Verwendung eines Sprungbefehles wie GOTO ist in der modernen
Softwareentwicklung verpönt. Sie führe leichter zu Spaghetti-Code,
der kaum mehr wartbar ist. Gewisse Sprachen verbannen daher
GOTO vollständig aus ihrer Syntax, und propagieren dagegen
die Verwendung von `strukturierten' Kontrollstrukturen wie
WHILE. Die Aufregung ist allerdings unnötig: GOTO und WHILE sind äquivalent.


\begin{satz}
Eine Funktion ist genau dann mit einem GOTO-Programm berechenbar,
wenn sie mit einem WHILE-Programm berechenbar ist.
\end{satz}
\begin{proof}[Beweis]
Man braucht nur zu zeigen, dass man ein GOTO-Programm in ein äquivalentes
WHILE-Programm übersetzen kann, und umgekehrt.

Um eine GOTO-Programm zu übersetzen, verwenden wir eine zusätzliche Variable
$z$, die die Funktion des Programm-Zählers übernimmt.
Aus dem GOTO-Programm machen wir dann folgendes WHILE-Programm
\begin{algorithmic}
\STATE $z \assignment 1$
\STATE{\tt WHILE\ }$z>0${\tt\ DO}
\STATE{\tt IF\ }$z=1${\tt\ THEN\ }$A_1'${\tt\ END};
\STATE{\tt IF\ }$z=2${\tt\ THEN\ }$A_2'${\tt\ END};
\STATE{\tt IF\ }$z=3${\tt\ THEN\ }$A_3'${\tt\ END};
\STATE\dots
\STATE{\tt IF\ }$z=k${\tt\ THEN\ }$A_k'${\tt\ END};
\STATE{\tt IF\ }$z=k+1${\tt\ THEN\ }$z \assignment 0${\tt\ END};
\STATE{\tt END}
\end{algorithmic}
Die Anweisung $A_i'$ entsteht aus der Anweisung $A_i$ nach folgenden
Regeln
\begin{itemize}
\item Falls $A_i$ eine Zuweisung ist, wird ihr eine weitere Zuweisung
\begin{algorithmic}
\STATE $z \assignment z+1$
\end{algorithmic}
angehängt.
Dies hat zur Folge, dass nach $A_z'$ als nächste Anweisung
$A_{z+1}$ ausgeführt wird.
\item
Falls $A_i$ eine bedingte Sprunganweisung
\begin{center}
\begin{tabular}{rl}
$M_l$:&{\tt IF\ }$x_i=c${\tt\ THEN GOTO\ }$M_j$
\end{tabular}
\end{center}
ist, wird $A_i'$
\begin{algorithmic}
\STATE{\tt IF\ }$x_i=c${\tt\ THEN\ }$z \assignment j${\tt\ ELSE }$z=z+1$;
\end{algorithmic}
Dies ist zwar keine WHILE-Anweisung, aber es wurde bereits
früher gezeigt, wie man sie in WHILE übersetzen kann.
\end{itemize}
Damit ist gezeigt, dass ein GOTO-Programm in ein äquivalentes WHILE-Programm
mit genau einer WHILE-Schleife übersetzt werden kann.

Umgekehrt zeigen wir, dass jede WHILE-Schleife mit Hilfe von GOTO
implementiert werden kann. Dazu übersetzt man jede WHILE-Schleife
der Form
\begin{algorithmic}
\STATE{\tt WHILE\ }$x_i>0${\tt\ DO }$P${\tt\ END}
\end{algorithmic}
in ein GOTO-Programm-Segment der Form
\begin{algorithmic}[1]
\STATE{\tt IF\ }$x_i=0${\tt\ THEN GOTO }4
\STATE$P$
\STATE{\tt GOTO\ }1
\STATE
\end{algorithmic}
wobei die Zeilennummern durch geeignete Marken ersetzt werden müssen.
Damit haben wir einen Algorithmus spezifiziert, der WHILE-Programme in
GOTO-Programme übersetzen kann.
\end{proof}

\subsection{Turing-Vollständigkeit von WHILE und GOTO}
Da WHILE und GOTO äquivalent sind, braucht die Turing-Vollständigkeit
nur für eine der Sprachen gezeigt zu werden.
Wir skizzieren, wie man eine Turing-Maschine in ein GOTO-Programm
übersetzen kann. Dies genügt, da man nur die universelle
Turing-Maschine zu übersetzen braucht, um damit jede andere
Turing-Maschine ausführen zu können.

\subsubsection{Alphabet, Zustände und Band}
Die Zeichen des Bandalphabetes werden durch natürliche Zahlen
dargestellt.  Wir nehmen an, dass das Bandalphabet $k$ verschiedene Zeichen
umfasst. Das Leerzeichen $\text{\textvisiblespace}$ wird durch die Zahl $0$
dargestellt.

Auch die Zustände der Turing-Maschine werden durch natürliche Zahlen
dargestellt,
die Variable $s$ dient dazu, den aktuellen Zustand zu
speichern.

Der Inhalt des Bandes kann durch eine einzige Variable $b$ dargestellt
werden. Schreibt man die Zahl im System zur Basis $k$, können die
die Zeichen in den einzelnen Felder des Bandes als die Ziffern
der Zahl $b$ interpretiert werden.

Die Kopfposition wird durch eine Zahl $h$ dargestellt. Befindet sich
der Kopf im Feld mit der Nummer $i$, wird $h$ auf den Wert $k^i$
gesetzt.

\subsubsection{Arithmetik}
Alle Komponenten der Turing-Maschine werden mit natürlichen Zahlen
und arithmetischen Operationen dargestellt.
Zwar beherrscht LOOP nur die Addition oder Subtraktion einer Konstanten,
aber durch wiederholte Addition einer $1$ kann damit jede beliebige
Addition oder Subtraktion implementiert werden.

Ebenso können Multiplikation und Division auf wiederholte Addition
zurückgeführt werden.
Im Folgenden nehmen wir daher an, dass die arithmetischen Operationen
zur Verfügung stehen.

\subsubsection{Lesen eines Feldes}
Um den Inhalt eines Feldes zu lesen, muss die Stelle von $b$ an der
aktuellen Kopfposition ermittelt werden.
Dies kann durch die Rechnung
\begin{equation}
z=b / h \mod k
\label{getchar}
\end{equation}
ermittelt werden, wobei $/$ für eine ganzzahlige Division steht.
Beide Operationen können mit einem GOTO-Programm ermittelt werden.

\subsubsection{Löschen eines Feldes}
Das Feld an der Kopfposition kann wie folgt gelöscht werden.
Zunächst ermittelt man mit (\ref{getchar}) den aktuellen Feldinhalt.
Dann berechnet man
\begin{equation}
b' = b - z\cdot h.
\label{clearchar}
\end{equation}
$b'$ enthält an der Stelle der Kopfposition ein $0$.

\subsubsection{Schreiben eines Feldes}
Soll das Feld an der Kopfposition mit dem Zeichen $x$ überschrieben
werden, wird mit (\ref{clearchar}) zuerst das Feld gelöscht.
Anschliessend wird das Feld durch
\[
b'=b+x\cdot h
\]
neu gesetzt.

\subsubsection{Kopfbewegung}
Die Kopfposition wird durch die Zahl $h$ dargestellt.
Da $h$ immer eine Potenz von $k$ ist, und die Nummer des Feldes der
Exponent ist, brauchen wir nur Operationen, die den Exponenten
ändern, also
\[
h'=h/k\qquad\text{bzw.}\qquad h'=h\cdot k
\]

\subsubsection{Übergangsfunktion}
Die Übergangsfunktion
\[
\delta\colon Q\times \Gamma\to Q\times \Gamma\times\{1, 2\}
\]
ermittelt aus aktuellem Zustand $s$ und
aktuellem Zeichen $z$ den neuen Zustand, das neue Zeichen auf
dem Band und die Kopfbewegung ermittelt. Im Gegensatz zur früheren
Definition verwenden wir jetzt die Zahlen $1$ und $2$ für die
Kopfbewegung L bzw.~R.
Wir schreiben $\delta_i$ für die $i$-te Komponente von $\delta$.
Der folgende GOTO-Pseudocode
beschreibt also ein Programm, welches die Turing-Maschine implementiert
\begin{algorithmic}[1]
\STATE Bestimme das Zeichen $z$ unter der aktuellen Kopfposition $h$
\STATE Lösche das aktuelle Zeichen auf dem Band
\STATE {\tt IF\ }$s=0${\tt\ THEN}
\STATE {\tt \ \ \ \ IF\ }$z=0${\tt\ THEN}
\STATE {\tt \ \ \ \ \ \ \ \ }$s\assignment\delta_1(0,0)$
\STATE {\tt \ \ \ \ \ \ \ \ }$z\assignment\delta_2(0,0)$
\STATE {\tt \ \ \ \ \ \ \ \ }$m\assignment\delta_3(0,0)$
\STATE {\tt \ \ \ \ END}
\STATE {\tt END}
\STATE {\tt IF\ }$s=1${\tt\ THEN}
\STATE {\tt \ \ \ \ }\dots
\STATE {\tt END}
\STATE Zeichen $z$ schreiben
\STATE {\tt IF\ }$m=1${\tt\ THEN }$h\assignment h/k$
\STATE {\tt IF\ }$m=2${\tt\ THEN }$h\assignment h\cdot k$
\STATE \dots
\STATE {\tt GOTO\ }1
\end{algorithmic}
Damit ist gezeigt, dass eine gegebene Turingmaschine in ein
GOTO-Programm übersetzt werden kann. Übersetzt man die universelle
Turing-Maschine, erhält man ein GOTO-Programm, welches jede beliebige
Turing-Maschine simulieren kann. Somit ist GOTO und damit auch WHILE
Turing-vollständig.

\subsection{Esoterische Programmiersprachen}
\index{Programmiersprache!esoterische}%
Zur Illustration der Tatsache, dass eine sehr primitive Sprache
ausreichen kann, um Turing-Vollständigkeit zu erreichen, wurden
verschiedene esoterische Programmiersprachen erfunden.
Ihre Nützlichkeit liegt darin, ein bestimmtest Konzept der Theorie
möglichst klar hervorzuheben, die Verwendbarkeit für irgend einen
praktischen Zweck ist nicht notwendig, und manchmal explizit unerwünscht.

\subsubsection{Brainfuck}
\index{Brainfuck}%
Brainfuck
wurde von Urban Müller 1993 entwickelt mit dem Ziel, dass der
Compiler für diese Sprache möglichst klein sein sollte. In der
Tat ist der kleinste Brainfuck-Compiler für Linux nur 171 Bytes
lang.

Brainfuck basiert auf einem einzelnen Pointer {\tt ptr}, welcher
im Programm inkrementiert oder dekrementiert werden kann.
Jeder Pointer-Wert zeigt auf eine Zelle, deren Inhalt inkrementiert
oder dekrementiert werden kann.
Dies erinnert an die Position des Kopfes einer Turing-Maschine.
Zwei Instruktionen für Eingabe und Ausgabe eines Zeichens
an der Pointer-Position ermöglichen Datenein- und -ausgabe.
Als einzige Kontrollstruktur steht WHILE zur Verfügung. Damit
die Sprache von einem minimalisitischen Compiler kompiliert
werden kann, wird jede Anweisung durch ein einziges Zeichen
dargestellt. Die Befehle sind in der folgenden Tabelle
zusammen mit ihrem C-Äquivalent zusammengstellt:
\begin{center}
\begin{tabular}{|c|l|}
\hline
Brainfuck&C-Äquivalent\\
\hline
{\tt >}&\verb/++ptr;/\\
{\tt <}&\verb/--ptr;/\\
{\tt +}&\verb/++*ptr;/\\
{\tt -}&\verb/--*ptr;/\\
{\tt .}&\verb/putchar(*ptr);/\\
{\tt ,}&\verb/*ptr = getchar();/\\
{\tt [}&\verb/while (*ptr) {/\\
{\tt ]}&\verb/}/\\
\hline
\end{tabular}
\end{center}
Mit Hilfe des Pointers lassen sich offenbar beliebige Speicherzellen
adressieren, und diese können durch Wiederholung der Befehle {\tt +}
und {\tt -} auch um konstante Werte vergrössert
oder verkleinert werden. Etwas mehr Arbeit erfordert die Zuweisung
eines Wertes zu einer Variablen. Ist dies jedoch geschafft, kann
man WHILE in Brainfuck übersetzen, und hat damit gezeigt, dass
Brainfuck Turing-vollständig ist.

\subsubsection{Ook}
\index{Ook}%
Die Sprache Ook verwendet als syntaktische Element das Wort {\tt Ook} gefolgt
von `{\tt .}', `{\tt !}' oder `{\tt ?}'. Wer beim Lesen eines Ook-Programmes
den Eindruck hat, zum Affen gemacht zu werden, liegt nicht ganz falsch:
Ook ist eine einfache Umcodierung von Brainfuck:
\begin{center}
\begin{tabular}{|c|c|}
\hline
Ook&Brainfuck\\
\hline
{\tt Ook. Ook?}&{\tt >}\\
{\tt Ook? Ook.}&{\tt <}\\
{\tt Ook. Ook.}&{\tt +}\\
{\tt Ook! Ook!}&{\tt -}\\
{\tt Ook! Ook.}&{\tt .}\\
{\tt Ook. Ook!}&{\tt ,}\\
{\tt Ook! Ook?}&{\tt [}\\
{\tt Ook? Ook!}&{\tt ]}\\
\hline
\end{tabular}
\end{center}
Da Brainfuck Turing-vollständig ist, ist auch Ook Turing-vollständig.
