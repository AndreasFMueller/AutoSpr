%
% vollstaendig.tex -- Turing-Vollständigkeit
%
% (c) 2011 Prof Dr Andreas Mueller, Hochschule Rapperswil
%

\section{Turing-vollständige Programmiersprachen}
\rhead{Turing-vollständige Programmiersprachen}
Die Turing-Maschine liefert einen wohldefinierten Begriff der
Berechenbarkeit, der auch robust gegenüber milden Änderungen
der Definition einer Turing-Maschine ist.
Der Aufbau aus einem endlichen Automaten mit zusätzlichem
Speicher und der einfache Kalkül mit Konfigurationen hat
sie ausserdem Beweisen vieler wichtiger Eigenschaften zugänglich
gemacht. Die vorangegangenen Kapitel über Entscheidbarkeit und
Komplexität legen davon eindrücklich Zeugnis ab. Am direkten
Nutzen dieser Theorie kann jedoch immer noch ein gewisser Zweifel
bestehen, da ein moderner Entwickler seine Programme ja nicht
direkt für eine Turing-Maschine schreibt, sondern nur mittelbar,
da er eine Programmiersprache verwendet, deren Code anschliessend
von einem Compiler oder Interpreter übersetzt und von einer realen
Maschine ausgeführt wird.

Der Aufbau der realen Maschine ist sehr
nahe an einer Turing-Maschine, ein Prozessor liest und schreibt
jeweils einzelne
Speicherzellen eines mindestens für praktische Zwecke unendlich
grossen Speichers und ändert bei Verarbeitung der gelesenen
Inhalte seinen eigenen Zustand. Natürlich ist die Menge der
Zustände eines modernen Prozessors sehr gross, nur schon die $n$
Register der Länge $l$ tragen $2^{nl}$ verschiedene Zustände bei,
und jedes andere Zustandsbit verdoppelt die Zustandsmenge nochmals.
Trotzdem ist die Zustandsmenge endlich, und es braucht nicht viel
Fantasie, sich den Prozessor mit seinem Hauptspeicher als Turingmaschine
vorzustellen. Es gibt also kaum Zweifel, dass die Computer-Hardware
zu all dem fähig ist, was ihr in den letzten zwei Kapiteln an
Fähigkeiten zugesprochen wurde.

Die einzige Einschränkung der Fähigkeiten realer Computer gegenüber
Turing-Maschinen ist
die Tatsache, dass reale Computer nur über einen endlichen Speicher
verfügen, während eine Turing-Maschine ein undendlich langes Band
als Speicher verwenden kann. Da jedoch eine endliche Berechnung auch
nur endlich viel Speicher verwenden kann, sind alle auf einer Turing-Maschine
durchführbaren Berechnungen, die man auch tatsächlich durchführen
will, auch von einem realen Computer durchführbar. Für praktische
Zwecke darf man also annehmen, dass die realen Computer echte Turing-Maschinen
sind.

Trotzdem ist nicht sicher, ob die Programmierung in einer übersetzten
oder interpretierten Sprache alle diese Fähigkeiten auch einem
Anwendungsprogrammierer zugänglich macht.
Letztlich äussert sich dies auch darin, dass Computernutzer
für verschiedene Problemstellung auch verschiedene Werkzeuge
verwenden. Wer tabellarische Daten summieren will, wird gerne
zu einer Spreadsheet-Software greifen, aber nicht erwarten, dass er
damit auch einen Näherungsalgorithmus für das Cliquen-Problem wird
programmieren können. Die Tabellenkalkulation definiert ein eingeschränktes,
an das Problem angepasstes Berechnungsmodell, welches aber
höchstwahrscheinlich weniger leistungsfähig ist als die Hardware, auf
der es läuft. Es ist also durchaus möglich und je nach Anwendung auch
zweckmässig, dass ein Anwender nicht die volle Leistung einer
Turing-Maschine zur Verfügung hat.

Damit stellt sich jetzt die Frage, wie man einem Berechnungsmodell und
das heisst letztlich der Sprache, in der der Berechnungsauftrag
formuliert wird, ansehen kann, ob sie gleich mächtig ist wie eine
Turing-Maschine.

\subsection{Programmiersprachen}
Eine Programmiersprache ist zwar eine Sprache im Sinne dieses Skriptes,
für den Programmierer wesentlich ist jedoch die Semantik, die bisher
nicht Bestandteil der Diskussion war. Für ihn ist die Tatsache wichtig,
dass die Semantik der Sprache Berechnungen beschreibt,
wie sie mit einer Turing-Maschine ausgeführt werden können.

\begin{definition}
\index{Programmiersprache}%
Eine Sprache $A$ heisst eine {\em Programmiersprache}, wenn es eine Abbildung
\[
c\colon A\to \Sigma^*\colon w\mapsto c(w)
\]
gibt, die einem Wort der Sprache die Beschreibung einer Turing-Maschine
zuordnet. Die Abbildung $c$ heisst {\em Compiler} für die Sprache $A$.
\end{definition}
Die Forderung, dass $c(w)$ die Beschreibung einer Turing-Maschine
sein muss, ist nach obiger Diskussion nicht wesentlich.

\subsection{Interaktion}
Man beachte, dass in dieser Definition einer Programmiersprache kein Platz ist
für Input oder Output während des Programmlaufes.
Das Band der Turing-Maschine, bzw.~sein Inhalt bildet den Input, der Output
kann nach Ende der Berechnung vom Band gelesen werden.
Man könnte dies als Mangel dieses Modells ansehen, in der Tat ist aber
keine Erweiterung nötig, um Interaktion abzubilden.
Interaktionen mit einem Benutzer bestehen immer aus einem Strom von
Ereignissen, die dem Benutzer zufliessen (Änderungen des Bildschirminhaltes,
Signaltöne) oder die der Benutzer veranlasst (Bewegungen des Maus-Zeigers,
Maus-Klicks, Tastatureingaben). Alle diese Ereignisse kann man sich codiert
auf ein Band geschrieben denken, welches die Turing-Maschine bei Bedarf
liest.

Der Inhalt des Bandes einer Standard-Turing-Maschine kann während des
Programmlaufes nur von der Turing-Maschine selbst verändert werden.
Da sich die Turing-Maschine aber nicht daran erinnern kann, was beim
letzten Besuch eines Feldes dort stand, ist es für eine Turing-Maschine
auch durchaus akzeptabel, wenn der Inhalt eines Feldes von aussen
geändert wird. Natürlich werden damit das Laufzeitverhalten der
Turing-Maschine verändert. Doch in Anbetracht der
Tatsache, dass von einer Turing-Maschine im Allgemeinen nicht einmal
entschieden werden kann, ob sie anhalten wird, ist wohl nicht mehr
viel zu verlieren.

Die Ausgaben eines Programmes sind deterministisch, und was der Benutzer
erreichen will, sowie die Ereignisse, die er einspeisen wird, sind es ebenfalls.
Man kann also im Prinzip im Voraus wissen, was ausgegeben werden wird
und welche Ereignisse ein Benutzer auslösen wird. Schreibt man diese
vorgängig auf das Band, so wie man es auch beim automatisierten Testen
eines Userinterfaces tut, entsteht aus dem interaktiven Programm eines,
welches ohne Zutun des Benutzers zur Laufzeit funktionieren kann.

\subsection{Die universelle Turing-Maschine}
\index{Turing, Alan}%
\index{Turing-Maschine!universelle}%
In seinem Paper von 1936 hat Alan Turing gezeigt, dass man eine
Turing-Maschine definieren kann,
der man die Beschreibung
$\langle M,w\rangle$
einer Turing-Maschine $M$ und eines Wortes $w$
und die $M$ auf dem Input-Wort $w$ simuliert.
Diese spezielle Turing-Maschine ist also leistungsfähig genug, jede
beliebige andere Turing-Maschine zu simulieren. Sie heisst die {\em universelle
Turing-Maschine}.

Die universelle Turing-Maschine kann die Entscheidung vereinfachen,
ob eine Funktion Turing-berechenbar ist. Statt eine Turing-Maschine
zu beschreiben, die die Funktion berechnet, reicht es, ein Programm
in der Programmiersprache $A$ zu beschreiben, das Programm mit dem
Compiler $c$ zu übersetzen, und die Beschreibung mit der universellen
Turing-Maschine auszuführen.

\index{Church-Turing-Hypothese}%
Die Church-Turing-Hypothese besagt, dass sich alles, was man berechnen
kann, auch mit einer Turing-Maschine berechnen lässt. Die universelle
Turing-Maschine zeigt, dass jede berechenbare Funktion von der
universellen Turing-Maschine berechnet werden kann.
Etwas leistungsfähigeres als eine Turing-Maschine gibt es nicht.

\subsection{Turing-Vollständigkeit}
Jede Funktion, die in der Programmiersprache $A$ implementiert werden
kann, ist Turing-berechenbar.
Der Compiler kann
aber durchaus Fähigkeiten unzugänglich machen, die Programmiersprache
$A$ kann dann gewisse Berechnungen, die mit einer Turing-Maschine
möglich wären, nicht formulieren. Besonderes interessant sind daher
die Sprachen, bei denen ein solcher Verlust nicht eintritt.

\begin{definition}
\index{Turing-vollständig}%
Eine Programmiersprache heisst Turing-vollständig, wenn sich jede
berechenbare Abbildung in dieser Sprache formulieren lässt.
Zu jeder berechenbaren Abbildung $f\colon\Sigma^*\to \Sigma^*$ gibt
es also ein Programm $w$ so, dass $c(w)$ die Funktion $f$ berechnet.
\end{definition}

Zu einer berechenbaren Abbildung gibt es eine Turing-Maschine, die
sie berechnet, es würde also genügen, wenn man diese Turing-Maschine
von einem in der Sprache $A$ geschriebenen Turing-Maschinen-Simulator
ausführen lassen könnte. Dieser Begriff muss noch etwas klarer gefasst
werden:

\begin{definition}
\index{Turing-Maschinen-Simulator}%
Ein Turing-Maschinen-Simulator ist eine Turing-Maschine $S$, die als Input
die Beschreibung $\langle M,w\rangle$ einer Turing-Maschine $M$ und eines
Input-Wortes für $M$ erhält, und die Berechnung durchführt, die $M$ auf $w$
ausführen würde.
Ein Turing-Maschinen-Simulator in der Programmiersprache $A$ ist
ein Wort $s\in A$ so, dass $c(s)$ ein Turing-Maschinen-Simulator ist.
\end{definition}

Damit erhalten wir ein Kriterium für Turing-Vollständigkeit:

\begin{satz}
\label{turingvollstaendigkeitskriterium}
Eine Programmiersprache $A$ ist Turing-vollständig, genau dann
wenn es einen Turing-Maschinen-Simulator in $A$ gibt.
\end{satz}


\subsection{Beispiele}
Die üblichen Programmiersprachen sind alle Turing-vollständig, denn es
ist eine einfache Programmierübung, eines Turing-Maschinen-Simulator
in einer dieser Sprachen zu schreiben. In einigen Programmiersprachen
ist dies jedoch schwieriger als in anderen.

\subsubsection{Javascript}
\index{Javascript}%
Fabrice Bellard hat 2011 einen PC-Emulator in Javascript geschrieben, der
leistungsfähig genug ist, Linux zu booten. Auf seiner Website
\url{http://bellard.org/jslinux/} kann man den Emulator im eigenen Browser
starten. Das gebootete Linux enthält auch einen C-Compiler. Da C
Turing-vollständig ist, gibt es einen Turing-Maschinen-Simulator in
C, den man auch auf dieses Linux bringen und mit dem C-Compiler
kompilieren kann. Somit gibt es einen Turing-Maschinen-Simulator in
Javascript, Javascript ist Turing-vollständig.

\subsubsection{XSLT}
\index{XSLT}%
XSLT ist eine XML-basierte Sprache, die Transformationen von XML-Dokumenten
zu beschreiben erlaubt. XSLT ist jedoch leistungsfähig, eine Turing-Maschine
zu simulieren. Bob Lyons hat auf seiner Website
\url{http://www.unidex.com/turing/utm.htm} ein XSL-Stylesheet publiziert,
welches einen Simulator implementiert. Als Input verlangt es
ein
XML-Dokument, welches die Beschreibung der Turing-Maschine in einem
zu diesem Zweck definierten XML-Format namens Turing Machine Markup
Language (TMML) enthält. TMML definiert XML-Elemente, die das Alphabet
(\verb+<symbols>+),
die Zustandsmenge $Q$ (\verb+<state>...</state>+)
und die Übergangsfunktion $\delta$ in \verb+<mapping>+-Elementen
der Form
\begin{verbatim}
<mapping>
    <from current-state="moveRight1" current-symbol=" " />
    <to next-state="check1" next-symbol=" " movement="left" />
</mapping>
\end{verbatim}
beschreiben. Der initiale Bandinhalt wird als Parameter \verb+tape+
auf der Kommandozeile übergeben.
Das Stylesheet wandelt das TMML Dokument in eine ausführliche
Berechnungsgeschichte um, aus der auch der Bandinhalt am Ende der Berechnung
abzulesen ist. Es beweist somit, dass XSLT einen Turing-Maschinen-Simulator
hat, also Turing-vollständdig ist.

\subsubsection{\LaTeX}
\index{LaTeX@\LaTeX}%
\index{Knuth, Don}%
Don Knuth, der Autor von \TeX, hat sich lange davor gedrückt, seiner
Schriftsatz-Sprache auch eine Turing-vollständige Programmiersprache
zu spendieren. Schliesslich kam er nicht mehr darum herum, und wurde
von Guy Steeles richtigegehend dazu gedrängt, wie er in
\url{http://maps.aanhet.net/maps/pdf/16\_15.pdf}
gesteht.

Dass \TeX Turing-vollständig ist beweist ein Satz von \LaTeX-Macros, den
man auf
\url{https://www.informatik.uni-augsburg.de/en/chairs/swt/ti/staff/mark/projects/turingtex/}
finden kann.
Um ihn zu verwenden, formuliert man die Beschreibung
von Turing-Maschine und initialem Bandinhalt als eine Menge von
\LaTeX-Makros. Ebenso ruft man den Makro \verb+\RunTuringMachine+ auf,
der die Turing-Machine simuliert und die Berechnungsgeschichte im
\TeX-üblichen perfekten Schriftsatz ausgibt.



