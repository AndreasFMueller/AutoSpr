%
% skript.tex -- Skript zur Vorlesung Automaten und Sprachen
%               gehalten an der Hochschule Rapperswil seit dem Wintersemester 09
%
% (c) 2009-2020 Prof. Dr. Andreas Mueller, HSR
% (c) 2020 Prof. Dr. Andreas Mueller, OST
%
\documentclass[a4paper,12pt]{book}
\usepackage[ngerman]{babel}
\usepackage[utf8]{inputenc}
\usepackage[T1]{fontenc}
\usepackage{csquotes}
\usepackage{float} %added see: https://tex.stackexchange.com/questions/8625/force-figure-placement-in-text
\usepackage{times}
\usepackage{geometry}
\geometry{papersize={210mm,297mm},total={160mm,240mm},top=31mm,bindingoffset=15mm}
\usepackage{alltt}
\usepackage{verbatim}
\usepackage{fancyhdr}
\usepackage{amsmath}
\usepackage{amssymb}
\usepackage{amsfonts}
\usepackage{amsthm}
\usepackage{textcomp}
\usepackage{graphicx}
\usepackage{array}
\usepackage{ifthen}
\usepackage{multirow}
\usepackage{txfonts}
%\usepackage[basic]{circ}
\usepackage[all]{xy}
\usepackage{algorithm}
\usepackage{algorithmic}
\usepackage{makeidx}
\usepackage{paralist}
\usepackage[colorlinks=true]{hyperref}
\usepackage[backend=bibtex]{biblatex}
\usepackage{tikz}
\usetikzlibrary{calc}
\addbibresource{references.bib}
\makeindex
\setlength{\headheight}{15pt}
\newcommand\myatop[2]{\genfrac{}{}{0pt}{}{#1}{#2}}
\begin{document}
\pagestyle{fancy}
\lhead{Automaten und Sprachen}
\rhead{}
\frontmatter
\newcommand\HRule{\noindent\rule{\linewidth}{1.5pt}}
\begin{titlepage}
\vspace*{\stretch{1}}
\HRule
\vspace*{10pt}
\begin{flushright}
{\Huge Automaten und Sprachen}
\end{flushright}
\HRule
\begin{flushright}
\vspace{30pt}
\LARGE
Andreas Müller
\end{flushright}
\vspace*{\stretch{2}}
\begin{center}
Hochschule für Technik, Rapperswil, 2011-2020\\
OST Ostschweizer Fachhochschule, 2020-2021
\end{center}
\end{titlepage}
\hypersetup{
    linktoc=all,
    linkcolor=blue
}
\rhead{Inhaltsverzeichnis}
\tableofcontents
\newtheorem{satz}{Satz}[chapter]
\newtheorem{hilfssatz}[satz]{Hilfssatz}
\newtheorem{definition}[satz]{Definition}
\newtheorem{annahme}[satz]{Annahme}
\newenvironment{beispiel}[1][Beispiel]{%
\begin{proof}[#1]%
\renewcommand{\qedsymbol}{$\bigcirc$}
}{\end{proof}}
\def\blank{\text{\textvisiblespace}}
\mainmatter
\begin{refsection}
%
% einleitung.tex
%
% (c) 2009 Prof Dr Andreas Mueller, Hochschule Rapperswil
%
\lhead{Einleitung}
\rhead{}
\chapter*{Worum geht es?\label{chapter-intro}}
Ursprünglich wurden Computer dazu entwickelt, aufwendige
Berechnungen zu automatisieren. Inzwischen wird nur noch ein
ganz kleiner Teil der weltweit installierten Rechenleistung für
diese Art von Problemen genutzt. 
Die heutige Informatik befasst sich vorwiegend mit der maschinellen
Verarbeitung von Zeichenfolgen. 

Während früher noch die Frage nach der effizientesten Lösung von
Berechnungsaufgaben mit dem Computer, also die Numerik, im Vordergrund stand,  
ist heute die Lösung komplizierterer und oft nicht numerischer Probleme 
gefragt. Zum Beispiel ist der kürzeste Weg zu finden, den ein Spediteur
zur Auslieferung aller ihm anvertrauten Sendungen verwenden kann, eine
Aufgabe, in der die Numerik offensichtlich eine untergeordnete Rolle spielt.

Trotzdem bleiben eine Reihe von Fragen, die grundsätzlich mathematischer
Art sind:
\begin{enumerate}
\item Gibt es Probleme, die mit dem Computer grundsätzlich nicht
lösbar sind?
\item Können wir darüber eine Aussage machen, wie schnell ein Computer
ein bestimmtes Problem lösen kann? Gibt es eine maximale Geschwindigkeit,
die nicht zu übertreffen ist?
\item Welchen Einfluss hat die Architektur eines Rechners auf die Klasse
von Problemen, die damit lösbar sind?
\item Gibt es Probleme, die sich besonders schnell lösen lassen?
\end{enumerate}
Die im vorliegenden Skript behandelte mathematische Theorie befasst sich
genau mit den logischen Grundlagen dieses Prozesses.
Sie definiert, welche Mengen von Zeichenfolgen überhaupt untersucht
werden sollen, und nennt sie {\em Sprachen}.
Sie studiert, welche abstrakten Typen von
Rechnern welche Arten von Zeichenfolgen erkennen oder hervorbringen
können. 

Es geht weniger darum, Methoden oder Algorithmen zur Verarbeitung
von Zeichenfolgen zu behandeln.
Vielmehr sollen mathematische Sätze aufgestellt werden, welche
Arten von Sprachen mit welcher Art von Rechnermodell bearbeitet
werden können.
Zum Beispiel werden wir Resultate wie die folgenden finden:
\begin{enumerate}
\item Ein Zustandsautomat kann die korrekte Schachtelung von
Klammern nicht erkennen.
\item Mit regulären Ausdrücken kann man korrekte arithmetische
Ausdrücke nicht erkennen.
\item Um korrekte arithmetische Ausdrücke zu erkennen, braucht man als
Speicher mindestens einen Stack.
\item Es gibt kein Programm, das ein beliebiges anderes Programm
analysieren kann und mit Sicherheit herausfinden kann, ob es je anhalten
wird.
\item Es gibt wahrscheinlich keinen Algorithmus mit polynomialer Laufzeit,
der entscheidet, ob es einen fehlerfreien Stundenplan gibt (d.\,h.~alle 
Studenten können alle Veranstaltungen besuchen, für die sie sich
angemeldet haben, es gibt keine Überschneidungen).
\item Fast alle reellen Zahlen können gar nicht berechnet werden.
\end{enumerate}
Die zu erwartenden Resultate sind also nicht von der Art von Kochrezepten:
``Um $X$ zu erhalten, mache man erst $Y$, dann $Z$ und zum Schluss $V$''.
Die Fragestellungen drehen sich alle um die fundamentale Frage:
\begin{quote}
Welche grundsätzlichen Möglichkeiten und Einschränkungen haben
Computer?
\end{quote}
Die Aussagen sind also absolute Wahrheiten, die unabhängig vom Stand
der Com\-pu\-ter-Technik oder der Programmier-Kunst gelten.
Solche Aussagen sind nur dann glaub\-würdig, wenn man sich von deren
Richtigkeit mit Hilfe eines mathematischen Beweises jederzeit überzeugen
kann. In dieser Vorlesung geht es also vor allem auch um Beweise, 
warum eine Sprache eine Eigenschaft hat, oder ob eine Eigenschaft
unter bestimmten Voraussetzungen überhaupt vorhanden sein kann.
Daher werden im ersten Kapitel kurz einige Beweistechniken rekapituliert.

Problem 5 in obiger Aufzählung kümmert sich nicht darum, ob eine
Aufgabe lösbar ist, sondern darum, wie effizient sie lösbar ist.
Dabei geht es nicht darum, wer den schnellsten Algorithmus findet,
sondern um die grundsätzliche Frage, ob es einen Algorithmus mit
einer gewissen Laufzeitkomplexität überhaupt geben kann.
Wiederum geht es um eine absolute mathematische Aussage, welche
einen Beweis erfordert. 

In dieser Vorlesung wird nichts berechnet.
In anderen
Mathematikvorlesungen werden zum Beispiel Techniken zur Berechnung
von Lösungen von
Gleichungen, zur Flächenberechnung oder zur Modellierung eines 
physikalischen Prozesses behandelt werden.
Dazu wird ein Kalkül mit einer eigenen Formelsprache entwickelt,
mit dem diese Berechnungen durchgeführt werden können.
In dieser Vorlesung werden kaum Formeln vorkommen. 
Und wenn Formeln vorkommen, dann werden es logische Formeln sein,
nicht algebraische.

Nachdem jetzt geklärt ist, was in dieser Vorlesung nicht behandelt
wird, hier einige Hinweise, was in den kommenden Kapiteln untersucht
wird:
\begin{description}
\item[Kapitel \ref{chapter-regular}:] Deterministische endliche Automaten (DEA). Von einem
DEA akzeptierte Sprache. Rekonstruktion des DEA aus der Sprache.
Reguläre Sprachen und DEAs. Reguläre Ausdrücke. Nicht deterministische
endliche Automaten (NEA). Wie beweist man, dass eine Sprache nicht regulär
ist?

In diesem Kapitel geht es nicht darum, ein Virtuose in der
Anwendung regulärer Ausdrücke zu werden. Viel wichtiger ist, Klarheit darüber
zu bekommen, in welchen Fällen reguläre Ausdrücke anwendbar sind,
und wo ihre Grenzen sind.

\item[Kapitel \ref{chapter-cfl}:] Stack-Automaten und von einem Stack-Automaten
erzeugte Sprache. Kontextfreie Grammatiken und davon erzeugte Sprachen. 
Normalform einer Grammatik. Komplexität des Parse-Problems.
Wiederum geht es nicht darum, besonders gewandt im Umgang mit Grammatiken
zu werden.

Kontextfreie Grammatiken werden zur Beschreibung von Programmiersprachen verwendet.
Trotzdem ist das Ziel dieses Abschnittes nicht Gewandtheit in der Spezifikation
von Programmiersprachen, sondern das Verständnis, welche Eigenschaften eine
Sprache haben muss, damit sie mit Hilfe einer Grammatik beschrieben werden kann.
In einem Beispiel wird auch gezeigt, wie man aus einer Grammatik mit geeigneten
Software-Werkzeugen direkt einen Parser für die von der Grammatik beschriebene
Sprache generieren kann. Wieder ist nicht die Beherrschung solcher Tools das
Ziel, dies wird in der Vorlesung Compilerbau geschehen.

\item[Kapitel \ref{chapter-turing}:] Turing Maschinen und davon erkannte Sprachen. 
Rekursiv aufzählbare Sprachen und Aufzähler.

Eine Turing Maschine ist ein stark vereinfachtes Modell eines Computers.
Alle modernen Computer sind im Prinzip nur besonders raffinierte Turing Maschinen.
Turing Maschinen sind aber einfach genug, um sich mit ihrer Hilfe darüber Gedanken
zu machen, ob und mit welcher Geschwindigkeit ein Problem mit einem
Computer gelöst werden kann. In diesem Zusammenhang müssen wir uns
darüber Gedanken machen, was es heissen soll, dass ein Problem mit
einem Computer lösbar sein soll. Dabei werden wir zeigen, dass es Probleme
gibt, die nicht mit dem Computer gelöst werden können. Allerdings wird
es uns schwerfallen, solche Probleme explizit zu beschreiben. 
Gelingen wird uns dies erst im Kapitel 6.

\item[Kapitel \ref{chapter-entscheidbarkeit}:] Entscheidbare Sprachen. Unlösbare Probleme.

Ziel dieses Kapitels ist zu lernen, wie man ein beliebiges Problem als
Sprache formulieren kann, und damit die bisher gelernte Theorie darauf
anwenden kann. Damit wird es möglich, konkrete und für die Praxis eines
Informatikers wichtige Probleme zu beschreiben, die
nicht mit Computern gelöst werden können.
Es ist zum Beispiel unmöglich, Programme maschinell daraufhin
zu prüfen, ob sie je anhalten werden.
Ausserdem lernen wir eine Technik kennen, mit der wir Probleme 
nach ``Schwierigkeitsgrad'' miteinander vergleichen können. 

\item[Kapitel \ref{chapter-komplexitaet}:] Komplexitätstheorie. In diesem Kapitel sollen 
die prinzipiellen Grenzen der Geschwindigkeit ermittelt werden,
mit der ein Problem mit dem Computer gelöst werden kann. Es wird sich
herausstellen, dass es zwar viele Probleme gibt, die mit aktuellen Computern in
vernünftiger Zeit lösbar sind, dass aber auch eine Klasse von Aufgaben
existiert, für die es nach aktuellem Wissen keinen Algorithmus geben kann,
der die Aufgabe bei etwas grösseren Problemen lösen kann. Diese Probleme sind
zwar im Prinzip lösbar, aber alle Rechenleistung der Welt kann die Probleme nicht lösen.

\item[Kapitel \ref{chapter-vollstaendigkeit}:]
Turing-Maschinen haben eine universelle Eigenschaft: eine Turing-Maschine
kann jede andere Turing-Maschine simulieren. Doch in der Praxis
programmiert niemand direkt eine Turing-Maschine, dazu werden
spezialisierte Programmiersprachen verwendet.
Welche Eigenschaften muss eine Sprache haben, damit
man all das machen kann, was man mit einer Turing-Maschine
machen kann?
\end{description}



%
% Turing-Vollstaendigkeit
%
% (c) 2011 Prof Dr Andreas Mueller, Hochschule Rapperswil
% $Id$
%
\chapter{Turing-Vollständigkeit\label{chapter-vollstaendigkeit}}
\lhead{Vollständigkeit}
%
% vollstaendig.tex -- Turing-Vollständigkeit
%
% (c) 2011 Prof Dr Andreas Mueller, Hochschule Rapperswil
%

\section{Turing-vollständige Programmiersprachen}
\rhead{Turing-vollständige Programmiersprachen}
Die Turing-Maschine liefert einen wohldefinierten Begriff der
Berechenbarkeit, der auch robust gegenüber milden Änderungen
der Definition einer Turing-Maschine ist.
Der Aufbau aus einem endlichen Automaten mit zusätzlichem
Speicher und der einfache Kalkül mit Konfigurationen hat
sie ausserdem Beweisen vieler wichtiger Eigenschaften zugänglich
gemacht. Die vorangegangenen Kapitel über Entscheidbarkeit und
Komplexität legen davon eindrücklich Zeugnis ab. Am direkten
Nutzen dieser Theorie kann jedoch immer noch ein gewisser Zweifel
bestehen, da ein moderner Entwickler seine Programme ja nicht
direkt für eine Turing-Maschine schreibt, sondern nur mittelbar,
da er eine Programmiersprache verwendet, deren Code anschliessend
von einem Compiler oder Interpreter übersetzt und von einer realen
Maschine ausgeführt wird.

Der Aufbau der realen Maschine ist sehr
nahe an einer Turing-Maschine, ein Prozessor liest und schreibt
jeweils einzelne
Speicherzellen eines mindestens für praktische Zwecke unendlich
grossen Speichers und ändert bei Verarbeitung der gelesenen
Inhalte seinen eigenen Zustand. Natürlich ist die Menge der
Zustände eines modernen Prozessors sehr gross, nur schon die $n$
Register der Länge $l$ tragen $2^{nl}$ verschiedene Zustände bei,
und jedes andere Zustandsbit verdoppelt die Zustandsmenge nochmals.
Trotzdem ist die Zustandsmenge endlich, und es braucht nicht viel
Fantasie, sich den Prozessor mit seinem Hauptspeicher als Turingmaschine
vorzustellen. Es gibt also kaum Zweifel, dass die Computer-Hardware
zu all dem fähig ist, was ihr in den letzten zwei Kapiteln an
Fähigkeiten zugesprochen wurde.

Die einzige Einschränkung der Fähigkeiten realer Computer gegenüber
Turing-Maschinen ist
die Tatsache, dass reale Computer nur über einen endlichen Speicher
verfügen, während eine Turing-Maschine ein undendlich langes Band
als Speicher verwenden kann. Da jedoch eine endliche Berechnung auch
nur endlich viel Speicher verwenden kann, sind alle auf einer Turing-Maschine
durchführbaren Berechnungen, die man auch tatsächlich durchführen
will, auch von einem realen Computer durchführbar. Für praktische
Zwecke darf man also annehmen, dass die realen Computer echte Turing-Maschinen
sind.

Trotzdem ist nicht sicher, ob die Programmierung in einer übersetzten
oder interpretierten Sprache alle diese Fähigkeiten auch einem
Anwendungsprogrammierer zugänglich macht.
Letztlich äussert sich dies auch darin, dass Computernutzer
für verschiedene Problemstellung auch verschiedene Werkzeuge
verwenden. Wer tabellarische Daten summieren will, wird gerne
zu einer Spreadsheet-Software greifen, aber nicht erwarten, dass er
damit auch einen Näherungsalgorithmus für das Cliquen-Problem wird
programmieren können. Die Tabellenkalkulation definiert ein eingeschränktes,
an das Problem angepasstes Berechnungsmodell, welches aber
höchstwahrscheinlich weniger leistungsfähig ist als die Hardware, auf
der es läuft. Es ist also durchaus möglich und je nach Anwendung auch
zweckmässig, dass ein Anwender nicht die volle Leistung einer
Turing-Maschine zur Verfügung hat.

Damit stellt sich jetzt die Frage, wie man einem Berechnungsmodell und
das heisst letztlich der Sprache, in der der Berechnungsauftrag
formuliert wird, ansehen kann, ob sie gleich mächtig ist wie eine
Turing-Maschine.

\subsection{Programmiersprachen}
Eine Programmiersprache ist zwar eine Sprache im Sinne dieses Skriptes,
für den Programmierer wesentlich ist jedoch die Semantik, die bisher
nicht Bestandteil der Diskussion war. Für ihn ist die Tatsache wichtig,
dass die Semantik der Sprache Berechnungen beschreibt,
wie sie mit einer Turing-Maschine ausgeführt werden können.

\begin{definition}
\index{Programmiersprache}%
Eine Sprache $A$ heisst eine {\em Programmiersprache}, wenn es eine Abbildung
\[
c\colon A\to \Sigma^*\colon w\mapsto c(w)
\]
gibt, die einem Wort der Sprache die Beschreibung einer Turing-Maschine
zuordnet. Die Abbildung $c$ heisst {\em Compiler} für die Sprache $A$.
\end{definition}
Die Forderung, dass $c(w)$ die Beschreibung einer Turing-Maschine
sein muss, ist nach obiger Diskussion nicht wesentlich.

\subsection{Interaktion}
Man beachte, dass in dieser Definition einer Programmiersprache kein Platz ist
für Input oder Output während des Programmlaufes.
Das Band der Turing-Maschine, bzw.~sein Inhalt bildet den Input, der Output
kann nach Ende der Berechnung vom Band gelesen werden.
Man könnte dies als Mangel dieses Modells ansehen, in der Tat ist aber
keine Erweiterung nötig, um Interaktion abzubilden.
Interaktionen mit einem Benutzer bestehen immer aus einem Strom von
Ereignissen, die dem Benutzer zufliessen (Änderungen des Bildschirminhaltes,
Signaltöne) oder die der Benutzer veranlasst (Bewegungen des Maus-Zeigers,
Maus-Klicks, Tastatureingaben). Alle diese Ereignisse kann man sich codiert
auf ein Band geschrieben denken, welches die Turing-Maschine bei Bedarf
liest.

Der Inhalt des Bandes einer Standard-Turing-Maschine kann während des
Programmlaufes nur von der Turing-Maschine selbst verändert werden.
Da sich die Turing-Maschine aber nicht daran erinnern kann, was beim
letzten Besuch eines Feldes dort stand, ist es für eine Turing-Maschine
auch durchaus akzeptabel, wenn der Inhalt eines Feldes von aussen
geändert wird. Natürlich werden damit das Laufzeitverhalten der
Turing-Maschine verändert. Doch in Anbetracht der
Tatsache, dass von einer Turing-Maschine im Allgemeinen nicht einmal
entschieden werden kann, ob sie anhalten wird, ist wohl nicht mehr
viel zu verlieren.

Die Ausgaben eines Programmes sind deterministisch, und was der Benutzer
erreichen will, sowie die Ereignisse, die er einspeisen wird, sind es ebenfalls.
Man kann also im Prinzip im Voraus wissen, was ausgegeben werden wird
und welche Ereignisse ein Benutzer auslösen wird. Schreibt man diese
vorgängig auf das Band, so wie man es auch beim automatisierten Testen
eines Userinterfaces tut, entsteht aus dem interaktiven Programm eines,
welches ohne Zutun des Benutzers zur Laufzeit funktionieren kann.

\subsection{Die universelle Turing-Maschine}
\index{Turing, Alan}%
\index{Turing-Maschine!universelle}%
In seinem Paper von 1936 hat Alan Turing gezeigt, dass man eine
Turing-Maschine definieren kann,
der man die Beschreibung
$\langle M,w\rangle$
einer Turing-Maschine $M$ und eines Wortes $w$
und die $M$ auf dem Input-Wort $w$ simuliert.
Diese spezielle Turing-Maschine ist also leistungsfähig genug, jede
beliebige andere Turing-Maschine zu simulieren. Sie heisst die {\em universelle
Turing-Maschine}.

Die universelle Turing-Maschine kann die Entscheidung vereinfachen,
ob eine Funktion Turing-berechenbar ist. Statt eine Turing-Maschine
zu beschreiben, die die Funktion berechnet, reicht es, ein Programm
in der Programmiersprache $A$ zu beschreiben, das Programm mit dem
Compiler $c$ zu übersetzen, und die Beschreibung mit der universellen
Turing-Maschine auszuführen.

\index{Church-Turing-Hypothese}%
Die Church-Turing-Hypothese besagt, dass sich alles, was man berechnen
kann, auch mit einer Turing-Maschine berechnen lässt. Die universelle
Turing-Maschine zeigt, dass jede berechenbare Funktion von der
universellen Turing-Maschine berechnet werden kann.
Etwas leistungsfähigeres als eine Turing-Maschine gibt es nicht.

\subsection{Turing-Vollständigkeit}
Jede Funktion, die in der Programmiersprache $A$ implementiert werden
kann, ist Turing-berechenbar.
Der Compiler kann
aber durchaus Fähigkeiten unzugänglich machen, die Programmiersprache
$A$ kann dann gewisse Berechnungen, die mit einer Turing-Maschine
möglich wären, nicht formulieren. Besonderes interessant sind daher
die Sprachen, bei denen ein solcher Verlust nicht eintritt.

\begin{definition}
\index{Turing-vollständig}%
Eine Programmiersprache heisst Turing-vollständig, wenn sich jede
berechenbare Abbildung in dieser Sprache formulieren lässt.
Zu jeder berechenbaren Abbildung $f\colon\Sigma^*\to \Sigma^*$ gibt
es also ein Programm $w$ so, dass $c(w)$ die Funktion $f$ berechnet.
\end{definition}

Zu einer berechenbaren Abbildung gibt es eine Turing-Maschine, die
sie berechnet, es würde also genügen, wenn man diese Turing-Maschine
von einem in der Sprache $A$ geschriebenen Turing-Maschinen-Simulator
ausführen lassen könnte. Dieser Begriff muss noch etwas klarer gefasst
werden:

\begin{definition}
\index{Turing-Maschinen-Simulator}%
Ein Turing-Maschinen-Simulator ist eine Turing-Maschine $S$, die als Input
die Beschreibung $\langle M,w\rangle$ einer Turing-Maschine $M$ und eines
Input-Wortes für $M$ erhält, und die Berechnung durchführt, die $M$ auf $w$
ausführen würde.
Ein Turing-Maschinen-Simulator in der Programmiersprache $A$ ist
ein Wort $s\in A$ so, dass $c(s)$ ein Turing-Maschinen-Simulator ist.
\end{definition}

Damit erhalten wir ein Kriterium für Turing-Vollständigkeit:

\begin{satz}
\label{turingvollstaendigkeitskriterium}
Eine Programmiersprache $A$ ist Turing-vollständig, genau dann
wenn es einen Turing-Maschinen-Simulator in $A$ gibt.
\end{satz}


\subsection{Beispiele}
Die üblichen Programmiersprachen sind alle Turing-vollständig, denn es
ist eine einfache Programmierübung, eines Turing-Maschinen-Simulator
in einer dieser Sprachen zu schreiben. In einigen Programmiersprachen
ist dies jedoch schwieriger als in anderen.

\subsubsection{Javascript}
\index{Javascript}%
Fabrice Bellard hat 2011 einen PC-Emulator in Javascript geschrieben, der
leistungsfähig genug ist, Linux zu booten. Auf seiner Website
\url{http://bellard.org/jslinux/} kann man den Emulator im eigenen Browser
starten. Das gebootete Linux enthält auch einen C-Compiler. Da C
Turing-vollständig ist, gibt es einen Turing-Maschinen-Simulator in
C, den man auch auf dieses Linux bringen und mit dem C-Compiler
kompilieren kann. Somit gibt es einen Turing-Maschinen-Simulator in
Javascript, Javascript ist Turing-vollständig.

\subsubsection{XSLT}
\index{XSLT}%
XSLT ist eine XML-basierte Sprache, die Transformationen von XML-Dokumenten
zu beschreiben erlaubt. XSLT ist jedoch leistungsfähig, eine Turing-Maschine
zu simulieren. Bob Lyons hat auf seiner Website
\url{http://www.unidex.com/turing/utm.htm} ein XSL-Stylesheet publiziert,
welches einen Simulator implementiert. Als Input verlangt es
ein
XML-Dokument, welches die Beschreibung der Turing-Maschine in einem
zu diesem Zweck definierten XML-Format namens Turing Machine Markup
Language (TMML) enthält. TMML definiert XML-Elemente, die das Alphabet
(\verb+<symbols>+),
die Zustandsmenge $Q$ (\verb+<state>...</state>+)
und die Übergangsfunktion $\delta$ in \verb+<mapping>+-Elementen
der Form
\begin{verbatim}
<mapping>
    <from current-state="moveRight1" current-symbol=" " />
    <to next-state="check1" next-symbol=" " movement="left" />
</mapping>
\end{verbatim}
beschreiben. Der initiale Bandinhalt wird als Parameter \verb+tape+
auf der Kommandozeile übergeben.
Das Stylesheet wandelt das TMML Dokument in eine ausführliche
Berechnungsgeschichte um, aus der auch der Bandinhalt am Ende der Berechnung
abzulesen ist. Es beweist somit, dass XSLT einen Turing-Maschinen-Simulator
hat, also Turing-vollständdig ist.

\subsubsection{\LaTeX}
\index{LaTeX@\LaTeX}%
\index{Knuth, Don}%
Don Knuth, der Autor von \TeX, hat sich lange davor gedrückt, seiner
Schriftsatz-Sprache auch eine Turing-vollständige Programmiersprache
zu spendieren. Schliesslich kam er nicht mehr darum herum, und wurde
von Guy Steeles richtigegehend dazu gedrängt, wie er in
\url{http://maps.aanhet.net/maps/pdf/16\_15.pdf}
gesteht.

Dass \TeX Turing-vollständig ist beweist ein Satz von \LaTeX-Macros, den
man auf
\url{https://www.informatik.uni-augsburg.de/en/chairs/swt/ti/staff/mark/projects/turingtex/}
finden kann.
Um ihn zu verwenden, formuliert man die Beschreibung
von Turing-Maschine und initialem Bandinhalt als eine Menge von
\LaTeX-Makros. Ebenso ruft man den Makro \verb+\RunTuringMachine+ auf,
der die Turing-Machine simuliert und die Berechnungsgeschichte im
\TeX-üblichen perfekten Schriftsatz ausgibt.




%
% kontrollstrukturen.tex
%
% (c) 2011 Prof Dr Andreas Mueller, Hochschule Rapperswil
%
\section{Kontrollstrukturen und Turing-Vollständigkeit}
\rhead{Kontrollstrukturen}
Das Turing-Vollstandigkeits-Kriterium von Satz
\ref{turingvollstaendigkeitskriterium} verlangt, dass man einen
Turing-Maschinen-Simulator in der gewählten Sprache schreiben muss.
Dies ist in jedem Fall eine nicht triviale Aufgabe.
Daher wäre es nützlich Kriterien zu erhalten, welche einfacher
anzuwenden sind. Einen wesentlichen Einfluss auf die Möglichkeiten,
was sich mit einer Programmiersprache ausdrücken lassen, haben die
Kontrollstrukturen.

\newcommand{\assignment}{\mathbin{\texttt{:=}}}

\subsection{Grundlegende Syntaxelemente%
\label{subsection:grundlegende-syntaxelement}}
In den folgenden Abschnitten werden verschiedene vereinfachte Sprachen
diskutiert, die aber einen gemeinsamen Kern fundamentaler Anweisungen
beinahlten.
In allen Sprachen gibt es nur einen einzigen Datentyp, nämlich
natürliche Zahlen.
Einerseits können Konstanten beliebig grosse natürliche Werte haben,
andererseits lassen sich in Variablen beliebig grosse natürliche Zahlen
speichern.
Dazu sind die folgenden Sytanxelemente notwendig:
\begin{compactitem}
\item Konstanten: {\tt 0}, {\tt 1}, {\tt 2}, \dots
\item Variablen $x_0$, $x_1$, $x_2$,\dots
\item Zuweisung: $\assignment$
\item Trennung von Anweisungen: {\tt ;}
\item Operatoren: {\tt +} und {\tt -}
\end{compactitem}
Die einzige Möglichkeit, den Wert einer Variablen zu ändern ist die
Zuweisung.  Diese ist entweder die Zuweisung einer Konstante als
Wert einer Variable:
\begin{algorithmic}
\STATE $x_i\assignment c$
\end{algorithmic}
oder eine Berechnung mit den beiden vorhandenen Operatoren
\begin{algorithmic}
\STATE $x_i \assignment x_j$ {\tt +} $c$
\STATE $x_i \assignment x_j$ {\tt -} $c$
\end{algorithmic}
Dabei ist das Resultat der Subtraktion als $0$ definiert, wenn
der Minuend kleiner ist als der Subtrahend.

Die verschiedenen Sprachen unterschieden sich in den Kontrollstrukturen
mit denen diese Zuweisungsbefehle miteinander verbunden werden können.

\subsection{LOOP}
\index{LOOP}%
Die Programmiersprache
LOOP\footnote{Die in diesem Abschnitt beschriebene
LOOP Sprache darf nicht verwechselt werden mit dem gleichnamigen
Projekt einer objektorientierten parallelen Sprache seit
2001 auf Sourceforge.}
hat als einzige Kontrollstruktur die
Iteration eines Anweisungs-Blocks mit einer festen, innerhalb des
Blocks nicht veränderbaren Anzahl von Durchläufen.

\subsubsection{Syntax}
LOOP verwendet die 
Schlüsselwörter: {\tt LOOP}, {\tt DO}, {\tt END}
Programme werden daraus wie folgt aufgebaut:
\begin{itemize}
\item Das leere Programm $\varepsilon$ ist eine LOOP-Programm,
während der Ausführung tut es nichts.
\item Wertzuweisungen sind LOOP-Programme
\item Sind $P_1$ und $P_2$ LOOP-Programme, dann auch
$P_1{\tt ;}P_2$. Um dieses Programm auszuführen wird zuerst $P_1$
ausgeführt und anschliessend $P_2$.
\item Ist $P$ ein LOOP-Programm, dann ist auch
\begin{algorithmic}
\STATE {\tt LOOP} $x_i$ {\tt DO} $P$ {\tt END}
\end{algorithmic}
ein LOOP-Programm. Die Ausführung wiederholt $P$ so oft, wie der
Wert der Variable $x_i$  zu Beginn angibt.
\end{itemize}
Durch Setzen der Variablen $x_i$ kann einem LOOP-Progamm Input übergeben
werden.
Ein LOOP-Programm definiert also eine Abbildung $\mathbb N^k\to\mathbb N$,
wobei $k$ die Anzahl der Variablen ist, in denen Input zu übergeben ist.

\subsubsection{Beispiel: Summer zweier Variablen}
LOOP kann nur jeweils eine Konstante hinzuaddieren, das genügt aber
bereits, um auch die Summe zweier Variablen zu berechnen. Der folgende
Code berechnet die Summe von $x_1$ und $x_2$ und legt das Resultat in
$x_0$ ab:
\begin{algorithmic}
\STATE $x_0 \assignment x_1$
\STATE {\tt LOOP} $x_2$ {\tt DO}
\STATE{\tt \ \ \ \ }$x_0 \assignment x_0$ {\tt +} $1$
\STATE{\tt END}
\end{algorithmic}

\subsubsection{Beispiel: Produkt zweier Variablen}
\index{Multiplikation}%
Die Sprache LOOP ist offenbar relativ primitiv, trotzdem
ist sie leistungsfähig genug, um die Multiplikation von zwei
Zahlen, die in den Variablen $x_1$ und $x_2$ übergeben werden,
zu berechnen:

\begin{algorithmic}
\STATE $x_0 \assignment 0$
\STATE {\tt LOOP} $x_1$ {\tt DO}
\STATE{\tt \ \ \ \ LOOP} $x_2$ {\tt DO}
\STATE{\tt \ \ \ \ \ \ \ \ }$x_0 \assignment x_0$ {\tt +} $1$
\STATE{\tt \ \ \ \ END}
\STATE{\tt END}
\end{algorithmic}

\subsubsection{IF-Anweisung}
In LOOP fehlt eine IF-Anweisung, auf die die wenigsten Programmiersprachen
verzichten. Es ist jedoch leicht, eine solche in LOOP nachzubilden.
Eine Anweisung
\begin{algorithmic}
\STATE {\tt IF} $x = 0$ {\tt THEN } $P$ {\tt END}
\end{algorithmic}
kann unter Verwendung einer zusätzlichen Variable $y$ in LOOP ausgedrückt
werden:
\begin{algorithmic}[1]
\STATE $y \assignment 1${\tt ;}
\STATE {\tt LOOP }$x${\tt\ DO }$y \assignment 0$ {\tt END;}
\STATE {\tt LOOP }$y${\tt\ DO }$P$ {\tt END;}
\end{algorithmic}
Natürlich wird die {\tt LOOP}-Anweisung in Zeile 2 meistens viel öfter
als nötig ausgeführt, aber es stehen hier ja auch nicht Fragen
der Effizienz sonder der prinzipiellen Machbarkeit zur Diskussion.

\subsubsection{LOOP ist nicht Turing-vollständig}
\begin{satz}
LOOP-Programme terminieren immer.
\end{satz}

\begin{proof}[Beweis]
Der Beweis kann mit Induktion über die Schachtelungstiefe
von {\tt LOOP}-Anweisungen geführt werden. Die LOOP-Programme
ohne {\tt LOOP}-Anweisung, also die Programme mit Schachtelungstiefe
$0$ terminieren offensichtlich immer. Ebenso die LOOP-Programme
mit Schachtelungstiefe $1$, da die Anzahl der Schleifendurchläufe
bereits zu Beginn der Schleife festliegt.

Nehmen wir jetzt an wir wüssten bereits, dass jedes LOOP-Programm mit
Schachtelungstiefe $n$ der {\tt LOOP}-Anweisungen immer terminiert.
Ein LOOP-Programm mit Schachtelungstiefe $n+1$ ist dann eine
Abfolge von Teilprogrammen mit Schachtelungstiefe $n$, die nach
Voraussetzung alle terminieren, und {\tt LOOP}-Anweisungen der
Form
\begin{algorithmic}
\STATE{\tt LOOP }$x_i${\tt\ DO }$P${\tt\ END}
\end{algorithmic}
wobei $P$ ein LOOP-Programm mit Schachtelungstiefe $n$ ist, das also
ebenfalls immer terminiert. $P$ wird genau so oft ausgeführt, wie
$x_i$ zu Beginn angibt. Die Laufzeit von $P$ kann dabei jedesmal
anders sein, $P$ wird aber auf jeden Fall terminieren. Damit ist
gezeigt, dass auch alle LOOP-Programme mit Schachtelungstiefe $n+1$
terminieren.
\end{proof}

\begin{satz}
LOOP ist nicht Turing vollständig.
\end{satz}

\begin{proof}[Beweis]
Es gibt Turing-Maschinen, die nicht terminieren. Gäbe es einen
Turing-Maschinen-Simulator in LOOP, dürfte dieser bei der
Simulation einer solchen Turing-Maschine nicht terminieren, im
Widerspruch zur Tatsache, dass LOOP-Programme immer terminieren.
Also kann es keinen Turing-Maschinen-Simulator in LOOP geben.
\end{proof}

\subsection{WHILE}
\index{WHILE}%
WHILE-Programme können zusätzlich zur {\tt LOOP}-Anweisung
eine {\tt WHILE}-Anweisung der Form
\begin{algorithmic}
\STATE{\tt WHILE }$x_i>0${\tt\ DO }$P${\tt\ END}
\end{algorithmic}
Dadurch ist die {\tt LOOP}-Anweisung nicht mehr unbedingt
notwendig, denn
\begin{algorithmic}
\STATE{\tt LOOP }$x${\tt\ DO }$P${\tt\ END}
\end{algorithmic}
kann durch
\begin{algorithmic}
\STATE$y \assignment x$;
\STATE{\tt WHILE }$y>0${\tt\ DO }$P$; $y \assignment y-1${\tt\ END}
\end{algorithmic}
nachgebildet werden.

\subsection{GOTO}
\index{GOTO}%
GOTO-Programme bestehen aus einer markierten Folge von Anweisungen
\begin{center}
\begin{tabular}{rl}
$M_1$:&$A_1$\\
$M_2$:&$A_2$\\
$M_3$:&$A_3$\\
$\dots$:&$\dots$\\
$M_k$:&$A_k$
\end{tabular}
\end{center}
Zu den bereits bekannten,
Abschnitt~\ref{subsection:grundlegende-syntaxelement} beschriebenen
Zuweisungen
$x_i \assignment c$ 
und
$x_i \assignment x_j \pm c$ 
kommt bei GOTO eine bedingte Sprunganweisung
\begin{center}
\begin{tabular}{rl}
$M_l$:&{\tt IF\ }$x_i=c${\tt\ THEN GOTO\ }$M_j$
\end{tabular}
\end{center}
Natürlich lässt sich damit auch eine unbedingte Sprunganweisung
implementieren:
\begin{center}
\begin{tabular}{rl}
$M_l$:&$x_i \assignment c$;\\
$M_{l+1}$:&{\tt IF\ }$x_i=c${\tt\ THEN GOTO\ }$M_j$
\end{tabular}
\end{center}
In ähnlicher Weise lassen sich auch andere bedingte Anweisungen
konstruieren, zum Beispiel ein
Konstrukt {\tt IF }\dots{\tt\ THEN }\dots{\tt\ ELSE }\dots{\tt\ END}, welches
einen ganzen Anweisungsblock enthalten kann.

\subsection{Äquivalenz von WHILE und GOTO}
Die Verwendung eines Sprungbefehles wie GOTO ist in der modernen
Softwareentwicklung verpönt. Sie führe leichter zu Spaghetti-Code,
der kaum mehr wartbar ist. Gewisse Sprachen verbannen daher
GOTO vollständig aus ihrer Syntax, und propagieren dagegen
die Verwendung von `strukturierten' Kontrollstrukturen wie
WHILE. Die Aufregung ist allerdings unnötig: GOTO und WHILE sind äquivalent.


\begin{satz}
Eine Funktion ist genau dann mit einem GOTO-Programm berechenbar,
wenn sie mit einem WHILE-Programm berechenbar ist.
\end{satz}
\begin{proof}[Beweis]
Man braucht nur zu zeigen, dass man ein GOTO-Programm in ein äquivalentes
WHILE-Programm übersetzen kann, und umgekehrt.

Um eine GOTO-Programm zu übersetzen, verwenden wir eine zusätzliche Variable
$z$, die die Funktion des Programm-Zählers übernimmt.
Aus dem GOTO-Programm machen wir dann folgendes WHILE-Programm
\begin{algorithmic}
\STATE $z \assignment 1$
\STATE{\tt WHILE\ }$z>0${\tt\ DO}
\STATE{\tt IF\ }$z=1${\tt\ THEN\ }$A_1'${\tt\ END};
\STATE{\tt IF\ }$z=2${\tt\ THEN\ }$A_2'${\tt\ END};
\STATE{\tt IF\ }$z=3${\tt\ THEN\ }$A_3'${\tt\ END};
\STATE\dots
\STATE{\tt IF\ }$z=k${\tt\ THEN\ }$A_k'${\tt\ END};
\STATE{\tt IF\ }$z=k+1${\tt\ THEN\ }$z \assignment 0${\tt\ END};
\STATE{\tt END}
\end{algorithmic}
Die Anweisung $A_i'$ entsteht aus der Anweisung $A_i$ nach folgenden
Regeln
\begin{itemize}
\item Falls $A_i$ eine Zuweisung ist, wird ihr eine weitere Zuweisung
\begin{algorithmic}
\STATE $z \assignment z+1$
\end{algorithmic}
angehängt.
Dies hat zur Folge, dass nach $A_z'$ als nächste Anweisung
$A_{z+1}$ ausgeführt wird.
\item
Falls $A_i$ eine bedingte Sprunganweisung
\begin{center}
\begin{tabular}{rl}
$M_l$:&{\tt IF\ }$x_i=c${\tt\ THEN GOTO\ }$M_j$
\end{tabular}
\end{center}
ist, wird $A_i'$
\begin{algorithmic}
\STATE{\tt IF\ }$x_i=c${\tt\ THEN\ }$z \assignment j${\tt\ ELSE }$z=z+1$;
\end{algorithmic}
Dies ist zwar keine WHILE-Anweisung, aber es wurde bereits
früher gezeigt, wie man sie in WHILE übersetzen kann.
\end{itemize}
Damit ist gezeigt, dass ein GOTO-Programm in ein äquivalentes WHILE-Programm
mit genau einer WHILE-Schleife übersetzt werden kann.

Umgekehrt zeigen wir, dass jede WHILE-Schleife mit Hilfe von GOTO
implementiert werden kann. Dazu übersetzt man jede WHILE-Schleife
der Form
\begin{algorithmic}
\STATE{\tt WHILE\ }$x_i>0${\tt\ DO }$P${\tt\ END}
\end{algorithmic}
in ein GOTO-Programm-Segment der Form
\begin{algorithmic}[1]
\STATE{\tt IF\ }$x_i=0${\tt\ THEN GOTO }4
\STATE$P$
\STATE{\tt GOTO\ }1
\STATE
\end{algorithmic}
wobei die Zeilennummern durch geeignete Marken ersetzt werden müssen.
Damit haben wir einen Algorithmus spezifiziert, der WHILE-Programme in
GOTO-Programme übersetzen kann.
\end{proof}

\subsection{Turing-Vollständigkeit von WHILE und GOTO}
Da WHILE und GOTO äquivalent sind, braucht die Turing-Vollständigkeit
nur für eine der Sprachen gezeigt zu werden.
Wir skizzieren, wie man eine Turing-Maschine in ein GOTO-Programm
übersetzen kann. Dies genügt, da man nur die universelle
Turing-Maschine zu übersetzen braucht, um damit jede andere
Turing-Maschine ausführen zu können.

\subsubsection{Alphabet, Zustände und Band}
Die Zeichen des Bandalphabetes werden durch natürliche Zahlen
dargestellt.  Wir nehmen an, dass das Bandalphabet $k$ verschiedene Zeichen
umfasst. Das Leerzeichen $\text{\textvisiblespace}$ wird durch die Zahl $0$
dargestellt.

Auch die Zustände der Turing-Maschine werden durch natürliche Zahlen
dargestellt,
die Variable $s$ dient dazu, den aktuellen Zustand zu
speichern.

Der Inhalt des Bandes kann durch eine einzige Variable $b$ dargestellt
werden. Schreibt man die Zahl im System zur Basis $k$, können die
die Zeichen in den einzelnen Felder des Bandes als die Ziffern
der Zahl $b$ interpretiert werden.

Die Kopfposition wird durch eine Zahl $h$ dargestellt. Befindet sich
der Kopf im Feld mit der Nummer $i$, wird $h$ auf den Wert $k^i$
gesetzt.

\subsubsection{Arithmetik}
Alle Komponenten der Turing-Maschine werden mit natürlichen Zahlen
und arithmetischen Operationen dargestellt.
Zwar beherrscht LOOP nur die Addition oder Subtraktion einer Konstanten,
aber durch wiederholte Addition einer $1$ kann damit jede beliebige
Addition oder Subtraktion implementiert werden.

Ebenso können Multiplikation und Division auf wiederholte Addition
zurückgeführt werden.
Im Folgenden nehmen wir daher an, dass die arithmetischen Operationen
zur Verfügung stehen.

\subsubsection{Lesen eines Feldes}
Um den Inhalt eines Feldes zu lesen, muss die Stelle von $b$ an der
aktuellen Kopfposition ermittelt werden.
Dies kann durch die Rechnung
\begin{equation}
z=b / h \mod k
\label{getchar}
\end{equation}
ermittelt werden, wobei $/$ für eine ganzzahlige Division steht.
Beide Operationen können mit einem GOTO-Programm ermittelt werden.

\subsubsection{Löschen eines Feldes}
Das Feld an der Kopfposition kann wie folgt gelöscht werden.
Zunächst ermittelt man mit (\ref{getchar}) den aktuellen Feldinhalt.
Dann berechnet man
\begin{equation}
b' = b - z\cdot h.
\label{clearchar}
\end{equation}
$b'$ enthält an der Stelle der Kopfposition ein $0$.

\subsubsection{Schreiben eines Feldes}
Soll das Feld an der Kopfposition mit dem Zeichen $x$ überschrieben
werden, wird mit (\ref{clearchar}) zuerst das Feld gelöscht.
Anschliessend wird das Feld durch
\[
b'=b+x\cdot h
\]
neu gesetzt.

\subsubsection{Kopfbewegung}
Die Kopfposition wird durch die Zahl $h$ dargestellt.
Da $h$ immer eine Potenz von $k$ ist, und die Nummer des Feldes der
Exponent ist, brauchen wir nur Operationen, die den Exponenten
ändern, also
\[
h'=h/k\qquad\text{bzw.}\qquad h'=h\cdot k
\]

\subsubsection{Übergangsfunktion}
Die Übergangsfunktion
\[
\delta\colon Q\times \Gamma\to Q\times \Gamma\times\{1, 2\}
\]
ermittelt aus aktuellem Zustand $s$ und
aktuellem Zeichen $z$ den neuen Zustand, das neue Zeichen auf
dem Band und die Kopfbewegung ermittelt. Im Gegensatz zur früheren
Definition verwenden wir jetzt die Zahlen $1$ und $2$ für die
Kopfbewegung L bzw.~R.
Wir schreiben $\delta_i$ für die $i$-te Komponente von $\delta$.
Der folgende GOTO-Pseudocode
beschreibt also ein Programm, welches die Turing-Maschine implementiert
\begin{algorithmic}[1]
\STATE Bestimme das Zeichen $z$ unter der aktuellen Kopfposition $h$
\STATE Lösche das aktuelle Zeichen auf dem Band
\STATE {\tt IF\ }$s=0${\tt\ THEN}
\STATE {\tt \ \ \ \ IF\ }$z=0${\tt\ THEN}
\STATE {\tt \ \ \ \ \ \ \ \ }$s\assignment\delta_1(0,0)$
\STATE {\tt \ \ \ \ \ \ \ \ }$z\assignment\delta_2(0,0)$
\STATE {\tt \ \ \ \ \ \ \ \ }$m\assignment\delta_3(0,0)$
\STATE {\tt \ \ \ \ END}
\STATE {\tt END}
\STATE {\tt IF\ }$s=1${\tt\ THEN}
\STATE {\tt \ \ \ \ }\dots
\STATE {\tt END}
\STATE Zeichen $z$ schreiben
\STATE {\tt IF\ }$m=1${\tt\ THEN }$h\assignment h/k$
\STATE {\tt IF\ }$m=2${\tt\ THEN }$h\assignment h\cdot k$
\STATE \dots
\STATE {\tt GOTO\ }1
\end{algorithmic}
Damit ist gezeigt, dass eine gegebene Turingmaschine in ein
GOTO-Programm übersetzt werden kann. Übersetzt man die universelle
Turing-Maschine, erhält man ein GOTO-Programm, welches jede beliebige
Turing-Maschine simulieren kann. Somit ist GOTO und damit auch WHILE
Turing-vollständig.

\subsection{Esoterische Programmiersprachen}
\index{Programmiersprache!esoterische}%
Zur Illustration der Tatsache, dass eine sehr primitive Sprache
ausreichen kann, um Turing-Vollständigkeit zu erreichen, wurden
verschiedene esoterische Programmiersprachen erfunden.
Ihre Nützlichkeit liegt darin, ein bestimmtest Konzept der Theorie
möglichst klar hervorzuheben, die Verwendbarkeit für irgend einen
praktischen Zweck ist nicht notwendig, und manchmal explizit unerwünscht.

\subsubsection{Brainfuck}
\index{Brainfuck}%
Brainfuck
wurde von Urban Müller 1993 entwickelt mit dem Ziel, dass der
Compiler für diese Sprache möglichst klein sein sollte. In der
Tat ist der kleinste Brainfuck-Compiler für Linux nur 171 Bytes
lang.

Brainfuck basiert auf einem einzelnen Pointer {\tt ptr}, welcher
im Programm inkrementiert oder dekrementiert werden kann.
Jeder Pointer-Wert zeigt auf eine Zelle, deren Inhalt inkrementiert
oder dekrementiert werden kann.
Dies erinnert an die Position des Kopfes einer Turing-Maschine.
Zwei Instruktionen für Eingabe und Ausgabe eines Zeichens
an der Pointer-Position ermöglichen Datenein- und -ausgabe.
Als einzige Kontrollstruktur steht WHILE zur Verfügung. Damit
die Sprache von einem minimalisitischen Compiler kompiliert
werden kann, wird jede Anweisung durch ein einziges Zeichen
dargestellt. Die Befehle sind in der folgenden Tabelle
zusammen mit ihrem C-Äquivalent zusammengstellt:
\begin{center}
\begin{tabular}{|c|l|}
\hline
Brainfuck&C-Äquivalent\\
\hline
{\tt >}&\verb/++ptr;/\\
{\tt <}&\verb/--ptr;/\\
{\tt +}&\verb/++*ptr;/\\
{\tt -}&\verb/--*ptr;/\\
{\tt .}&\verb/putchar(*ptr);/\\
{\tt ,}&\verb/*ptr = getchar();/\\
{\tt [}&\verb/while (*ptr) {/\\
{\tt ]}&\verb/}/\\
\hline
\end{tabular}
\end{center}
Mit Hilfe des Pointers lassen sich offenbar beliebige Speicherzellen
adressieren, und diese können durch Wiederholung der Befehle {\tt +}
und {\tt -} auch um konstante Werte vergrössert
oder verkleinert werden. Etwas mehr Arbeit erfordert die Zuweisung
eines Wertes zu einer Variablen. Ist dies jedoch geschafft, kann
man WHILE in Brainfuck übersetzen, und hat damit gezeigt, dass
Brainfuck Turing-vollständig ist.

\subsubsection{Ook}
\index{Ook}%
Die Sprache Ook verwendet als syntaktische Element das Wort {\tt Ook} gefolgt
von `{\tt .}', `{\tt !}' oder `{\tt ?}'. Wer beim Lesen eines Ook-Programmes
den Eindruck hat, zum Affen gemacht zu werden, liegt nicht ganz falsch:
Ook ist eine einfache Umcodierung von Brainfuck:
\begin{center}
\begin{tabular}{|c|c|}
\hline
Ook&Brainfuck\\
\hline
{\tt Ook. Ook?}&{\tt >}\\
{\tt Ook? Ook.}&{\tt <}\\
{\tt Ook. Ook.}&{\tt +}\\
{\tt Ook! Ook!}&{\tt -}\\
{\tt Ook! Ook.}&{\tt .}\\
{\tt Ook. Ook!}&{\tt ,}\\
{\tt Ook! Ook?}&{\tt [}\\
{\tt Ook? Ook!}&{\tt ]}\\
\hline
\end{tabular}
\end{center}
Da Brainfuck Turing-vollständig ist, ist auch Ook Turing-vollständig.



%
% sprachen.tex
%
% (c) 2011 Prof Dr Andreas Mueller, Hochschule Rapperswil
%
\chapter{Sprachen\label{chapter-sprachen}}
\lhead{Sprachen}
Diese Vorlesung betrachtet Computer in erster Linie als
Maschinen, die Zeichenketten verarbeiten.
Natürlich wird man
nicht jede Zeichenkette als sinnvollen Input oder Output betrachten,
nur eine Teilmenge aller möglichen Zeichenketten würden wir
eine ``Sprache'' nennen.
Diese intuitive Vorstellung wollen wir
jetzt formalisieren.

\index{Alphabet}%
Die Basis einer Sprachdefinition muss die Auswahl eines geeigneten
Alphabetes sein.
Während in der Informatik das Alphabet meist 
durch die Maschine vorgegeben ist (sie verarbeitet zum Beispiel
einzelne Bytes, also Zahlen zwischen 0 und 255), möchten wir für
unsere theoretischen Überlegungen mehr Freiheit, und akzeptieren
jede beliebige nichtleere endliche Menge als Alphabet.
Alphabete
bezeichnen wir häufig mit grossen griechischen Buchstaben, zum
Beispiel
\begin{align*}
\Sigma&=\{{\tt 0},{\tt 1}\}\\
\Sigma&=\{{\tt 1}\}\\
\Gamma&=\{{\tt 0},{\tt 1},\blank\}\\
\Delta&=\{{\tt a},\dots,{\tt z}\}
\end{align*}
Zeichenketten sind Tupel aus Elementen eines Alphabets.
\begin{definition}\label{def_wort}
\index{Wort}%
\index{leeres Wort}%
Ein Wort $w$ der Länge $n$ über dem Alphabet $\Sigma$ ist ein $n$-Tupel,
$w\in\Sigma^n$.
Es gibt genau ein Wort der Länge $0$, es heisst das
leere Wort und wird mit $\varepsilon$ bezeichnet, $\Sigma^0=\{\varepsilon\}$.
Die Menge aller Wörter
ist die Vereinigung aller $\Sigma^n$ und wird mit $\Sigma^*$ bezeichnet:
\[
\Sigma^*=\{\varepsilon\}\cup \Sigma^1\cup\Sigma^2\cup\Sigma^3\cup\dots
=\bigcup_{k=0}^\infty\Sigma^k.
\]
\end{definition}

Natürlich kann man von jedem Wort $w\in\Sigma^*$ auch wieder bestimmen,
aus welchem $\Sigma^k$ es stammt, indem man die Länge der Zeichenkette
auszählt.
Wir bezeichnen die Länge von $w$ mit $|w|$.
Es ist $|{\tt 1111}|=4$ und $|\varepsilon|=0$.
Manchmal ist es wichtig zu wissen, wie oft ein bestimmtes
Zeichen in einem Wort vorkommt.
Wir bezeichnen mit
$|w|_{a}$ die Anzahl Vorkommnisse des Zeichens $a$, also zum Beispiel
$|{\tt 0010111}|_{\tt 0}=3$ und $|{\tt 0110011}|_{\tt 1}=4$.

Eine Sprache muss nicht alle möglichen Wörter umfassen, oft wird sogar
nur eine endliche Menge von Wörtern aus der unendlichen Menge $\Sigma^*$
ausgewählt.

\begin{definition}
\index{Sprache}%
Eine Sprache $L$ über dem Alphabet $\Sigma$ ist eine Teilmenge
von $\Sigma^*$, also $L\subset \Sigma^*$.
Über jedem Alphabet
gibt es die leere Sprache $\emptyset$ und die Sprache, die nur
aus dem leeren Wort besteht $\{\varepsilon\}$.
\end{definition}

{\parindent0pt Beispiele:}
\begin{enumerate}
\item Sei $\Sigma=\{{\tt 1}\}$, dann ist
\[
\Sigma^*=\{\varepsilon, {\tt 1}, {\tt 11}, {\tt 111}, {\tt 1111},\dots\}
\]
Die Wörter über $\Sigma$ sind also durch ihre Länge charakterisiert.
Es gibt eine Abbildung
\[
\mathbb N\to\Sigma^*\colon n\mapsto \underbrace{1\dots 1}_{\text{$n$ Zeichen}}=1^n
\]
Eine Sprache über dem Alphabet $\{{\tt 1}\}$ entspricht unter dieser
Abbildung genau einer Teilmenge der natürlichen Zahlen.
Die Darstellung
einer Zahl $n$ als Folge von $n$ Zeichen {\tt 1} heisst
{\em unäre Darstellung}
\index{unaere Darstellung@unäre Darstellung}%
von $n$.
\item Sei $\Sigma=\{{\tt (}, {\tt )}\}$.
$\Sigma^*$ besteht aus allen Ketten von Klammern.
Die korrekt geschachtelten Klammerausdrücke bilden
darin eine Teilmenge, also eine Sprache.
\item Sei $\Sigma=\{{\tt 0},{\tt 1},\dots,{\tt 9}\}$.
$\Sigma^*$ besteht aus allen Ziffernfolgen.
Jede solche Ziffernfolge hat natürlich auch
einen numerischen Wert, indem man die Ziffernfolge als Zehnersystemdarstellung
einer Zahl interpretiert.
Wegen möglicher führender Nullen können verschiedene
Ziffernfolgen den gleichen Wert haben.
Es gibt also eine Abbildung
\[
v\colon\Sigma^*\to \mathbb N,
\]
die den Wert einer Ziffernfolge ermittelt.
Damit können wir die Sprache
der Ziffernfolgen mit Zweierpotenzwerten definieren:
\[
\{w\in\Sigma^*\,|\, \exists k (v(w)=2^k)\}=\{
{\tt 1},
{\tt 2},
{\tt 4},
{\tt 8},\dots
{\tt 01},
{\tt 02},
{\tt 04},
{\tt 08},\dots
{\tt 001},
{\tt 002},
{\tt 004},
{\tt 008},\dots\}
\]
\item Sei $\Sigma=\{{\tt 0}, {\tt 1}\}$.
$\Sigma^*$ besteht dann aus den binären Zeichenketten.
Darin können wir eine Reihe von Sprachen auszeichnen:
\begin{align*}
L_1&=\{ {\tt 0}^n{\tt 1}^n\,|\,n\ge 0\}\\
L_2&=\{ w\in\Sigma^*\,|\, |w|_{\tt 0}=|w|_{\tt 1}\}\\
L_3&=\{ w\in\Sigma^*\,|\, \text{Zahlenwert von $w$ ist durch 3 teilbar}\}
\end{align*}
\item Sei $\Sigma$ die Menge der ASCII-Zeichen.
Dann ist $\Sigma^*$ die Menge aller ASCII-Texte, wie unsinnig sie auch
immer sein mögen.
Interessant ist die Sprache
\[
C=\{w\in\Sigma^*\,|\,\text{$w$ wird von GCC akzeptiert}\}.
\]
Die Sprache $C$ heisst GNU-Dialekt von C.
\item Sei $\Sigma$ die Menge der Unicode-Zeichen.
$\Sigma^*$ besteht dann aus allen Unicode-Zeichenfolgen.
Darin können wir zum Beispiel die
Sprache Java auswählen
\[
J
=\{w\in\Sigma^*\,|\, {\text{$w$ wird von einem Java-Compiler akzeptiert}}\}.
\]
\end{enumerate}

Oft kommen Zeichenketten aus vielen identischen Zeichen oder Zeichenketten vor.
Wir verwenden für solche Zeichenketten eine Potenz-Notation:
\begin{align*}
{\tt 0}^3{\tt 1}^5&={\tt 00011111}\\
({\tt 01})^4&={\tt 01010101}
\end{align*}
Die Sprache
\[
L=\{ {\tt 0}^n{\tt 1}^n\,|\,n\in\mathbb R\}
\]
besteht also aus allen Folgen von {\tt 0} und {\tt 1}, in denen eine Anzahl
von Nullen gefolgt wird von einer identischen Anzahl Einsen.

%
% Turing-Vollstaendigkeit
%
% (c) 2011 Prof Dr Andreas Mueller, Hochschule Rapperswil
% $Id$
%
\chapter{Turing-Vollständigkeit\label{chapter-vollstaendigkeit}}
\lhead{Vollständigkeit}
%
% vollstaendig.tex -- Turing-Vollständigkeit
%
% (c) 2011 Prof Dr Andreas Mueller, Hochschule Rapperswil
%

\section{Turing-vollständige Programmiersprachen}
\rhead{Turing-vollständige Programmiersprachen}
Die Turing-Maschine liefert einen wohldefinierten Begriff der
Berechenbarkeit, der auch robust gegenüber milden Änderungen
der Definition einer Turing-Maschine ist.
Der Aufbau aus einem endlichen Automaten mit zusätzlichem
Speicher und der einfache Kalkül mit Konfigurationen hat
sie ausserdem Beweisen vieler wichtiger Eigenschaften zugänglich
gemacht. Die vorangegangenen Kapitel über Entscheidbarkeit und
Komplexität legen davon eindrücklich Zeugnis ab. Am direkten
Nutzen dieser Theorie kann jedoch immer noch ein gewisser Zweifel
bestehen, da ein moderner Entwickler seine Programme ja nicht
direkt für eine Turing-Maschine schreibt, sondern nur mittelbar,
da er eine Programmiersprache verwendet, deren Code anschliessend
von einem Compiler oder Interpreter übersetzt und von einer realen
Maschine ausgeführt wird.

Der Aufbau der realen Maschine ist sehr
nahe an einer Turing-Maschine, ein Prozessor liest und schreibt
jeweils einzelne
Speicherzellen eines mindestens für praktische Zwecke unendlich
grossen Speichers und ändert bei Verarbeitung der gelesenen
Inhalte seinen eigenen Zustand. Natürlich ist die Menge der
Zustände eines modernen Prozessors sehr gross, nur schon die $n$
Register der Länge $l$ tragen $2^{nl}$ verschiedene Zustände bei,
und jedes andere Zustandsbit verdoppelt die Zustandsmenge nochmals.
Trotzdem ist die Zustandsmenge endlich, und es braucht nicht viel
Fantasie, sich den Prozessor mit seinem Hauptspeicher als Turingmaschine
vorzustellen. Es gibt also kaum Zweifel, dass die Computer-Hardware
zu all dem fähig ist, was ihr in den letzten zwei Kapiteln an
Fähigkeiten zugesprochen wurde.

Die einzige Einschränkung der Fähigkeiten realer Computer gegenüber
Turing-Maschinen ist
die Tatsache, dass reale Computer nur über einen endlichen Speicher
verfügen, während eine Turing-Maschine ein undendlich langes Band
als Speicher verwenden kann. Da jedoch eine endliche Berechnung auch
nur endlich viel Speicher verwenden kann, sind alle auf einer Turing-Maschine
durchführbaren Berechnungen, die man auch tatsächlich durchführen
will, auch von einem realen Computer durchführbar. Für praktische
Zwecke darf man also annehmen, dass die realen Computer echte Turing-Maschinen
sind.

Trotzdem ist nicht sicher, ob die Programmierung in einer übersetzten
oder interpretierten Sprache alle diese Fähigkeiten auch einem
Anwendungsprogrammierer zugänglich macht.
Letztlich äussert sich dies auch darin, dass Computernutzer
für verschiedene Problemstellung auch verschiedene Werkzeuge
verwenden. Wer tabellarische Daten summieren will, wird gerne
zu einer Spreadsheet-Software greifen, aber nicht erwarten, dass er
damit auch einen Näherungsalgorithmus für das Cliquen-Problem wird
programmieren können. Die Tabellenkalkulation definiert ein eingeschränktes,
an das Problem angepasstes Berechnungsmodell, welches aber
höchstwahrscheinlich weniger leistungsfähig ist als die Hardware, auf
der es läuft. Es ist also durchaus möglich und je nach Anwendung auch
zweckmässig, dass ein Anwender nicht die volle Leistung einer
Turing-Maschine zur Verfügung hat.

Damit stellt sich jetzt die Frage, wie man einem Berechnungsmodell und
das heisst letztlich der Sprache, in der der Berechnungsauftrag
formuliert wird, ansehen kann, ob sie gleich mächtig ist wie eine
Turing-Maschine.

\subsection{Programmiersprachen}
Eine Programmiersprache ist zwar eine Sprache im Sinne dieses Skriptes,
für den Programmierer wesentlich ist jedoch die Semantik, die bisher
nicht Bestandteil der Diskussion war. Für ihn ist die Tatsache wichtig,
dass die Semantik der Sprache Berechnungen beschreibt,
wie sie mit einer Turing-Maschine ausgeführt werden können.

\begin{definition}
\index{Programmiersprache}%
Eine Sprache $A$ heisst eine {\em Programmiersprache}, wenn es eine Abbildung
\[
c\colon A\to \Sigma^*\colon w\mapsto c(w)
\]
gibt, die einem Wort der Sprache die Beschreibung einer Turing-Maschine
zuordnet. Die Abbildung $c$ heisst {\em Compiler} für die Sprache $A$.
\end{definition}
Die Forderung, dass $c(w)$ die Beschreibung einer Turing-Maschine
sein muss, ist nach obiger Diskussion nicht wesentlich.

\subsection{Interaktion}
Man beachte, dass in dieser Definition einer Programmiersprache kein Platz ist
für Input oder Output während des Programmlaufes.
Das Band der Turing-Maschine, bzw.~sein Inhalt bildet den Input, der Output
kann nach Ende der Berechnung vom Band gelesen werden.
Man könnte dies als Mangel dieses Modells ansehen, in der Tat ist aber
keine Erweiterung nötig, um Interaktion abzubilden.
Interaktionen mit einem Benutzer bestehen immer aus einem Strom von
Ereignissen, die dem Benutzer zufliessen (Änderungen des Bildschirminhaltes,
Signaltöne) oder die der Benutzer veranlasst (Bewegungen des Maus-Zeigers,
Maus-Klicks, Tastatureingaben). Alle diese Ereignisse kann man sich codiert
auf ein Band geschrieben denken, welches die Turing-Maschine bei Bedarf
liest.

Der Inhalt des Bandes einer Standard-Turing-Maschine kann während des
Programmlaufes nur von der Turing-Maschine selbst verändert werden.
Da sich die Turing-Maschine aber nicht daran erinnern kann, was beim
letzten Besuch eines Feldes dort stand, ist es für eine Turing-Maschine
auch durchaus akzeptabel, wenn der Inhalt eines Feldes von aussen
geändert wird. Natürlich werden damit das Laufzeitverhalten der
Turing-Maschine verändert. Doch in Anbetracht der
Tatsache, dass von einer Turing-Maschine im Allgemeinen nicht einmal
entschieden werden kann, ob sie anhalten wird, ist wohl nicht mehr
viel zu verlieren.

Die Ausgaben eines Programmes sind deterministisch, und was der Benutzer
erreichen will, sowie die Ereignisse, die er einspeisen wird, sind es ebenfalls.
Man kann also im Prinzip im Voraus wissen, was ausgegeben werden wird
und welche Ereignisse ein Benutzer auslösen wird. Schreibt man diese
vorgängig auf das Band, so wie man es auch beim automatisierten Testen
eines Userinterfaces tut, entsteht aus dem interaktiven Programm eines,
welches ohne Zutun des Benutzers zur Laufzeit funktionieren kann.

\subsection{Die universelle Turing-Maschine}
\index{Turing, Alan}%
\index{Turing-Maschine!universelle}%
In seinem Paper von 1936 hat Alan Turing gezeigt, dass man eine
Turing-Maschine definieren kann,
der man die Beschreibung
$\langle M,w\rangle$
einer Turing-Maschine $M$ und eines Wortes $w$
und die $M$ auf dem Input-Wort $w$ simuliert.
Diese spezielle Turing-Maschine ist also leistungsfähig genug, jede
beliebige andere Turing-Maschine zu simulieren. Sie heisst die {\em universelle
Turing-Maschine}.

Die universelle Turing-Maschine kann die Entscheidung vereinfachen,
ob eine Funktion Turing-berechenbar ist. Statt eine Turing-Maschine
zu beschreiben, die die Funktion berechnet, reicht es, ein Programm
in der Programmiersprache $A$ zu beschreiben, das Programm mit dem
Compiler $c$ zu übersetzen, und die Beschreibung mit der universellen
Turing-Maschine auszuführen.

\index{Church-Turing-Hypothese}%
Die Church-Turing-Hypothese besagt, dass sich alles, was man berechnen
kann, auch mit einer Turing-Maschine berechnen lässt. Die universelle
Turing-Maschine zeigt, dass jede berechenbare Funktion von der
universellen Turing-Maschine berechnet werden kann.
Etwas leistungsfähigeres als eine Turing-Maschine gibt es nicht.

\subsection{Turing-Vollständigkeit}
Jede Funktion, die in der Programmiersprache $A$ implementiert werden
kann, ist Turing-berechenbar.
Der Compiler kann
aber durchaus Fähigkeiten unzugänglich machen, die Programmiersprache
$A$ kann dann gewisse Berechnungen, die mit einer Turing-Maschine
möglich wären, nicht formulieren. Besonderes interessant sind daher
die Sprachen, bei denen ein solcher Verlust nicht eintritt.

\begin{definition}
\index{Turing-vollständig}%
Eine Programmiersprache heisst Turing-vollständig, wenn sich jede
berechenbare Abbildung in dieser Sprache formulieren lässt.
Zu jeder berechenbaren Abbildung $f\colon\Sigma^*\to \Sigma^*$ gibt
es also ein Programm $w$ so, dass $c(w)$ die Funktion $f$ berechnet.
\end{definition}

Zu einer berechenbaren Abbildung gibt es eine Turing-Maschine, die
sie berechnet, es würde also genügen, wenn man diese Turing-Maschine
von einem in der Sprache $A$ geschriebenen Turing-Maschinen-Simulator
ausführen lassen könnte. Dieser Begriff muss noch etwas klarer gefasst
werden:

\begin{definition}
\index{Turing-Maschinen-Simulator}%
Ein Turing-Maschinen-Simulator ist eine Turing-Maschine $S$, die als Input
die Beschreibung $\langle M,w\rangle$ einer Turing-Maschine $M$ und eines
Input-Wortes für $M$ erhält, und die Berechnung durchführt, die $M$ auf $w$
ausführen würde.
Ein Turing-Maschinen-Simulator in der Programmiersprache $A$ ist
ein Wort $s\in A$ so, dass $c(s)$ ein Turing-Maschinen-Simulator ist.
\end{definition}

Damit erhalten wir ein Kriterium für Turing-Vollständigkeit:

\begin{satz}
\label{turingvollstaendigkeitskriterium}
Eine Programmiersprache $A$ ist Turing-vollständig, genau dann
wenn es einen Turing-Maschinen-Simulator in $A$ gibt.
\end{satz}


\subsection{Beispiele}
Die üblichen Programmiersprachen sind alle Turing-vollständig, denn es
ist eine einfache Programmierübung, eines Turing-Maschinen-Simulator
in einer dieser Sprachen zu schreiben. In einigen Programmiersprachen
ist dies jedoch schwieriger als in anderen.

\subsubsection{Javascript}
\index{Javascript}%
Fabrice Bellard hat 2011 einen PC-Emulator in Javascript geschrieben, der
leistungsfähig genug ist, Linux zu booten. Auf seiner Website
\url{http://bellard.org/jslinux/} kann man den Emulator im eigenen Browser
starten. Das gebootete Linux enthält auch einen C-Compiler. Da C
Turing-vollständig ist, gibt es einen Turing-Maschinen-Simulator in
C, den man auch auf dieses Linux bringen und mit dem C-Compiler
kompilieren kann. Somit gibt es einen Turing-Maschinen-Simulator in
Javascript, Javascript ist Turing-vollständig.

\subsubsection{XSLT}
\index{XSLT}%
XSLT ist eine XML-basierte Sprache, die Transformationen von XML-Dokumenten
zu beschreiben erlaubt. XSLT ist jedoch leistungsfähig, eine Turing-Maschine
zu simulieren. Bob Lyons hat auf seiner Website
\url{http://www.unidex.com/turing/utm.htm} ein XSL-Stylesheet publiziert,
welches einen Simulator implementiert. Als Input verlangt es
ein
XML-Dokument, welches die Beschreibung der Turing-Maschine in einem
zu diesem Zweck definierten XML-Format namens Turing Machine Markup
Language (TMML) enthält. TMML definiert XML-Elemente, die das Alphabet
(\verb+<symbols>+),
die Zustandsmenge $Q$ (\verb+<state>...</state>+)
und die Übergangsfunktion $\delta$ in \verb+<mapping>+-Elementen
der Form
\begin{verbatim}
<mapping>
    <from current-state="moveRight1" current-symbol=" " />
    <to next-state="check1" next-symbol=" " movement="left" />
</mapping>
\end{verbatim}
beschreiben. Der initiale Bandinhalt wird als Parameter \verb+tape+
auf der Kommandozeile übergeben.
Das Stylesheet wandelt das TMML Dokument in eine ausführliche
Berechnungsgeschichte um, aus der auch der Bandinhalt am Ende der Berechnung
abzulesen ist. Es beweist somit, dass XSLT einen Turing-Maschinen-Simulator
hat, also Turing-vollständdig ist.

\subsubsection{\LaTeX}
\index{LaTeX@\LaTeX}%
\index{Knuth, Don}%
Don Knuth, der Autor von \TeX, hat sich lange davor gedrückt, seiner
Schriftsatz-Sprache auch eine Turing-vollständige Programmiersprache
zu spendieren. Schliesslich kam er nicht mehr darum herum, und wurde
von Guy Steeles richtigegehend dazu gedrängt, wie er in
\url{http://maps.aanhet.net/maps/pdf/16\_15.pdf}
gesteht.

Dass \TeX Turing-vollständig ist beweist ein Satz von \LaTeX-Macros, den
man auf
\url{https://www.informatik.uni-augsburg.de/en/chairs/swt/ti/staff/mark/projects/turingtex/}
finden kann.
Um ihn zu verwenden, formuliert man die Beschreibung
von Turing-Maschine und initialem Bandinhalt als eine Menge von
\LaTeX-Makros. Ebenso ruft man den Makro \verb+\RunTuringMachine+ auf,
der die Turing-Machine simuliert und die Berechnungsgeschichte im
\TeX-üblichen perfekten Schriftsatz ausgibt.




%
% kontrollstrukturen.tex
%
% (c) 2011 Prof Dr Andreas Mueller, Hochschule Rapperswil
%
\section{Kontrollstrukturen und Turing-Vollständigkeit}
\rhead{Kontrollstrukturen}
Das Turing-Vollstandigkeits-Kriterium von Satz
\ref{turingvollstaendigkeitskriterium} verlangt, dass man einen
Turing-Maschinen-Simulator in der gewählten Sprache schreiben muss.
Dies ist in jedem Fall eine nicht triviale Aufgabe.
Daher wäre es nützlich Kriterien zu erhalten, welche einfacher
anzuwenden sind. Einen wesentlichen Einfluss auf die Möglichkeiten,
was sich mit einer Programmiersprache ausdrücken lassen, haben die
Kontrollstrukturen.

\newcommand{\assignment}{\mathbin{\texttt{:=}}}

\subsection{Grundlegende Syntaxelemente%
\label{subsection:grundlegende-syntaxelement}}
In den folgenden Abschnitten werden verschiedene vereinfachte Sprachen
diskutiert, die aber einen gemeinsamen Kern fundamentaler Anweisungen
beinahlten.
In allen Sprachen gibt es nur einen einzigen Datentyp, nämlich
natürliche Zahlen.
Einerseits können Konstanten beliebig grosse natürliche Werte haben,
andererseits lassen sich in Variablen beliebig grosse natürliche Zahlen
speichern.
Dazu sind die folgenden Sytanxelemente notwendig:
\begin{compactitem}
\item Konstanten: {\tt 0}, {\tt 1}, {\tt 2}, \dots
\item Variablen $x_0$, $x_1$, $x_2$,\dots
\item Zuweisung: $\assignment$
\item Trennung von Anweisungen: {\tt ;}
\item Operatoren: {\tt +} und {\tt -}
\end{compactitem}
Die einzige Möglichkeit, den Wert einer Variablen zu ändern ist die
Zuweisung.  Diese ist entweder die Zuweisung einer Konstante als
Wert einer Variable:
\begin{algorithmic}
\STATE $x_i\assignment c$
\end{algorithmic}
oder eine Berechnung mit den beiden vorhandenen Operatoren
\begin{algorithmic}
\STATE $x_i \assignment x_j$ {\tt +} $c$
\STATE $x_i \assignment x_j$ {\tt -} $c$
\end{algorithmic}
Dabei ist das Resultat der Subtraktion als $0$ definiert, wenn
der Minuend kleiner ist als der Subtrahend.

Die verschiedenen Sprachen unterschieden sich in den Kontrollstrukturen
mit denen diese Zuweisungsbefehle miteinander verbunden werden können.

\subsection{LOOP}
\index{LOOP}%
Die Programmiersprache
LOOP\footnote{Die in diesem Abschnitt beschriebene
LOOP Sprache darf nicht verwechselt werden mit dem gleichnamigen
Projekt einer objektorientierten parallelen Sprache seit
2001 auf Sourceforge.}
hat als einzige Kontrollstruktur die
Iteration eines Anweisungs-Blocks mit einer festen, innerhalb des
Blocks nicht veränderbaren Anzahl von Durchläufen.

\subsubsection{Syntax}
LOOP verwendet die 
Schlüsselwörter: {\tt LOOP}, {\tt DO}, {\tt END}
Programme werden daraus wie folgt aufgebaut:
\begin{itemize}
\item Das leere Programm $\varepsilon$ ist eine LOOP-Programm,
während der Ausführung tut es nichts.
\item Wertzuweisungen sind LOOP-Programme
\item Sind $P_1$ und $P_2$ LOOP-Programme, dann auch
$P_1{\tt ;}P_2$. Um dieses Programm auszuführen wird zuerst $P_1$
ausgeführt und anschliessend $P_2$.
\item Ist $P$ ein LOOP-Programm, dann ist auch
\begin{algorithmic}
\STATE {\tt LOOP} $x_i$ {\tt DO} $P$ {\tt END}
\end{algorithmic}
ein LOOP-Programm. Die Ausführung wiederholt $P$ so oft, wie der
Wert der Variable $x_i$  zu Beginn angibt.
\end{itemize}
Durch Setzen der Variablen $x_i$ kann einem LOOP-Progamm Input übergeben
werden.
Ein LOOP-Programm definiert also eine Abbildung $\mathbb N^k\to\mathbb N$,
wobei $k$ die Anzahl der Variablen ist, in denen Input zu übergeben ist.

\subsubsection{Beispiel: Summer zweier Variablen}
LOOP kann nur jeweils eine Konstante hinzuaddieren, das genügt aber
bereits, um auch die Summe zweier Variablen zu berechnen. Der folgende
Code berechnet die Summe von $x_1$ und $x_2$ und legt das Resultat in
$x_0$ ab:
\begin{algorithmic}
\STATE $x_0 \assignment x_1$
\STATE {\tt LOOP} $x_2$ {\tt DO}
\STATE{\tt \ \ \ \ }$x_0 \assignment x_0$ {\tt +} $1$
\STATE{\tt END}
\end{algorithmic}

\subsubsection{Beispiel: Produkt zweier Variablen}
\index{Multiplikation}%
Die Sprache LOOP ist offenbar relativ primitiv, trotzdem
ist sie leistungsfähig genug, um die Multiplikation von zwei
Zahlen, die in den Variablen $x_1$ und $x_2$ übergeben werden,
zu berechnen:

\begin{algorithmic}
\STATE $x_0 \assignment 0$
\STATE {\tt LOOP} $x_1$ {\tt DO}
\STATE{\tt \ \ \ \ LOOP} $x_2$ {\tt DO}
\STATE{\tt \ \ \ \ \ \ \ \ }$x_0 \assignment x_0$ {\tt +} $1$
\STATE{\tt \ \ \ \ END}
\STATE{\tt END}
\end{algorithmic}

\subsubsection{IF-Anweisung}
In LOOP fehlt eine IF-Anweisung, auf die die wenigsten Programmiersprachen
verzichten. Es ist jedoch leicht, eine solche in LOOP nachzubilden.
Eine Anweisung
\begin{algorithmic}
\STATE {\tt IF} $x = 0$ {\tt THEN } $P$ {\tt END}
\end{algorithmic}
kann unter Verwendung einer zusätzlichen Variable $y$ in LOOP ausgedrückt
werden:
\begin{algorithmic}[1]
\STATE $y \assignment 1${\tt ;}
\STATE {\tt LOOP }$x${\tt\ DO }$y \assignment 0$ {\tt END;}
\STATE {\tt LOOP }$y${\tt\ DO }$P$ {\tt END;}
\end{algorithmic}
Natürlich wird die {\tt LOOP}-Anweisung in Zeile 2 meistens viel öfter
als nötig ausgeführt, aber es stehen hier ja auch nicht Fragen
der Effizienz sonder der prinzipiellen Machbarkeit zur Diskussion.

\subsubsection{LOOP ist nicht Turing-vollständig}
\begin{satz}
LOOP-Programme terminieren immer.
\end{satz}

\begin{proof}[Beweis]
Der Beweis kann mit Induktion über die Schachtelungstiefe
von {\tt LOOP}-Anweisungen geführt werden. Die LOOP-Programme
ohne {\tt LOOP}-Anweisung, also die Programme mit Schachtelungstiefe
$0$ terminieren offensichtlich immer. Ebenso die LOOP-Programme
mit Schachtelungstiefe $1$, da die Anzahl der Schleifendurchläufe
bereits zu Beginn der Schleife festliegt.

Nehmen wir jetzt an wir wüssten bereits, dass jedes LOOP-Programm mit
Schachtelungstiefe $n$ der {\tt LOOP}-Anweisungen immer terminiert.
Ein LOOP-Programm mit Schachtelungstiefe $n+1$ ist dann eine
Abfolge von Teilprogrammen mit Schachtelungstiefe $n$, die nach
Voraussetzung alle terminieren, und {\tt LOOP}-Anweisungen der
Form
\begin{algorithmic}
\STATE{\tt LOOP }$x_i${\tt\ DO }$P${\tt\ END}
\end{algorithmic}
wobei $P$ ein LOOP-Programm mit Schachtelungstiefe $n$ ist, das also
ebenfalls immer terminiert. $P$ wird genau so oft ausgeführt, wie
$x_i$ zu Beginn angibt. Die Laufzeit von $P$ kann dabei jedesmal
anders sein, $P$ wird aber auf jeden Fall terminieren. Damit ist
gezeigt, dass auch alle LOOP-Programme mit Schachtelungstiefe $n+1$
terminieren.
\end{proof}

\begin{satz}
LOOP ist nicht Turing vollständig.
\end{satz}

\begin{proof}[Beweis]
Es gibt Turing-Maschinen, die nicht terminieren. Gäbe es einen
Turing-Maschinen-Simulator in LOOP, dürfte dieser bei der
Simulation einer solchen Turing-Maschine nicht terminieren, im
Widerspruch zur Tatsache, dass LOOP-Programme immer terminieren.
Also kann es keinen Turing-Maschinen-Simulator in LOOP geben.
\end{proof}

\subsection{WHILE}
\index{WHILE}%
WHILE-Programme können zusätzlich zur {\tt LOOP}-Anweisung
eine {\tt WHILE}-Anweisung der Form
\begin{algorithmic}
\STATE{\tt WHILE }$x_i>0${\tt\ DO }$P${\tt\ END}
\end{algorithmic}
Dadurch ist die {\tt LOOP}-Anweisung nicht mehr unbedingt
notwendig, denn
\begin{algorithmic}
\STATE{\tt LOOP }$x${\tt\ DO }$P${\tt\ END}
\end{algorithmic}
kann durch
\begin{algorithmic}
\STATE$y \assignment x$;
\STATE{\tt WHILE }$y>0${\tt\ DO }$P$; $y \assignment y-1${\tt\ END}
\end{algorithmic}
nachgebildet werden.

\subsection{GOTO}
\index{GOTO}%
GOTO-Programme bestehen aus einer markierten Folge von Anweisungen
\begin{center}
\begin{tabular}{rl}
$M_1$:&$A_1$\\
$M_2$:&$A_2$\\
$M_3$:&$A_3$\\
$\dots$:&$\dots$\\
$M_k$:&$A_k$
\end{tabular}
\end{center}
Zu den bereits bekannten,
Abschnitt~\ref{subsection:grundlegende-syntaxelement} beschriebenen
Zuweisungen
$x_i \assignment c$ 
und
$x_i \assignment x_j \pm c$ 
kommt bei GOTO eine bedingte Sprunganweisung
\begin{center}
\begin{tabular}{rl}
$M_l$:&{\tt IF\ }$x_i=c${\tt\ THEN GOTO\ }$M_j$
\end{tabular}
\end{center}
Natürlich lässt sich damit auch eine unbedingte Sprunganweisung
implementieren:
\begin{center}
\begin{tabular}{rl}
$M_l$:&$x_i \assignment c$;\\
$M_{l+1}$:&{\tt IF\ }$x_i=c${\tt\ THEN GOTO\ }$M_j$
\end{tabular}
\end{center}
In ähnlicher Weise lassen sich auch andere bedingte Anweisungen
konstruieren, zum Beispiel ein
Konstrukt {\tt IF }\dots{\tt\ THEN }\dots{\tt\ ELSE }\dots{\tt\ END}, welches
einen ganzen Anweisungsblock enthalten kann.

\subsection{Äquivalenz von WHILE und GOTO}
Die Verwendung eines Sprungbefehles wie GOTO ist in der modernen
Softwareentwicklung verpönt. Sie führe leichter zu Spaghetti-Code,
der kaum mehr wartbar ist. Gewisse Sprachen verbannen daher
GOTO vollständig aus ihrer Syntax, und propagieren dagegen
die Verwendung von `strukturierten' Kontrollstrukturen wie
WHILE. Die Aufregung ist allerdings unnötig: GOTO und WHILE sind äquivalent.


\begin{satz}
Eine Funktion ist genau dann mit einem GOTO-Programm berechenbar,
wenn sie mit einem WHILE-Programm berechenbar ist.
\end{satz}
\begin{proof}[Beweis]
Man braucht nur zu zeigen, dass man ein GOTO-Programm in ein äquivalentes
WHILE-Programm übersetzen kann, und umgekehrt.

Um eine GOTO-Programm zu übersetzen, verwenden wir eine zusätzliche Variable
$z$, die die Funktion des Programm-Zählers übernimmt.
Aus dem GOTO-Programm machen wir dann folgendes WHILE-Programm
\begin{algorithmic}
\STATE $z \assignment 1$
\STATE{\tt WHILE\ }$z>0${\tt\ DO}
\STATE{\tt IF\ }$z=1${\tt\ THEN\ }$A_1'${\tt\ END};
\STATE{\tt IF\ }$z=2${\tt\ THEN\ }$A_2'${\tt\ END};
\STATE{\tt IF\ }$z=3${\tt\ THEN\ }$A_3'${\tt\ END};
\STATE\dots
\STATE{\tt IF\ }$z=k${\tt\ THEN\ }$A_k'${\tt\ END};
\STATE{\tt IF\ }$z=k+1${\tt\ THEN\ }$z \assignment 0${\tt\ END};
\STATE{\tt END}
\end{algorithmic}
Die Anweisung $A_i'$ entsteht aus der Anweisung $A_i$ nach folgenden
Regeln
\begin{itemize}
\item Falls $A_i$ eine Zuweisung ist, wird ihr eine weitere Zuweisung
\begin{algorithmic}
\STATE $z \assignment z+1$
\end{algorithmic}
angehängt.
Dies hat zur Folge, dass nach $A_z'$ als nächste Anweisung
$A_{z+1}$ ausgeführt wird.
\item
Falls $A_i$ eine bedingte Sprunganweisung
\begin{center}
\begin{tabular}{rl}
$M_l$:&{\tt IF\ }$x_i=c${\tt\ THEN GOTO\ }$M_j$
\end{tabular}
\end{center}
ist, wird $A_i'$
\begin{algorithmic}
\STATE{\tt IF\ }$x_i=c${\tt\ THEN\ }$z \assignment j${\tt\ ELSE }$z=z+1$;
\end{algorithmic}
Dies ist zwar keine WHILE-Anweisung, aber es wurde bereits
früher gezeigt, wie man sie in WHILE übersetzen kann.
\end{itemize}
Damit ist gezeigt, dass ein GOTO-Programm in ein äquivalentes WHILE-Programm
mit genau einer WHILE-Schleife übersetzt werden kann.

Umgekehrt zeigen wir, dass jede WHILE-Schleife mit Hilfe von GOTO
implementiert werden kann. Dazu übersetzt man jede WHILE-Schleife
der Form
\begin{algorithmic}
\STATE{\tt WHILE\ }$x_i>0${\tt\ DO }$P${\tt\ END}
\end{algorithmic}
in ein GOTO-Programm-Segment der Form
\begin{algorithmic}[1]
\STATE{\tt IF\ }$x_i=0${\tt\ THEN GOTO }4
\STATE$P$
\STATE{\tt GOTO\ }1
\STATE
\end{algorithmic}
wobei die Zeilennummern durch geeignete Marken ersetzt werden müssen.
Damit haben wir einen Algorithmus spezifiziert, der WHILE-Programme in
GOTO-Programme übersetzen kann.
\end{proof}

\subsection{Turing-Vollständigkeit von WHILE und GOTO}
Da WHILE und GOTO äquivalent sind, braucht die Turing-Vollständigkeit
nur für eine der Sprachen gezeigt zu werden.
Wir skizzieren, wie man eine Turing-Maschine in ein GOTO-Programm
übersetzen kann. Dies genügt, da man nur die universelle
Turing-Maschine zu übersetzen braucht, um damit jede andere
Turing-Maschine ausführen zu können.

\subsubsection{Alphabet, Zustände und Band}
Die Zeichen des Bandalphabetes werden durch natürliche Zahlen
dargestellt.  Wir nehmen an, dass das Bandalphabet $k$ verschiedene Zeichen
umfasst. Das Leerzeichen $\text{\textvisiblespace}$ wird durch die Zahl $0$
dargestellt.

Auch die Zustände der Turing-Maschine werden durch natürliche Zahlen
dargestellt,
die Variable $s$ dient dazu, den aktuellen Zustand zu
speichern.

Der Inhalt des Bandes kann durch eine einzige Variable $b$ dargestellt
werden. Schreibt man die Zahl im System zur Basis $k$, können die
die Zeichen in den einzelnen Felder des Bandes als die Ziffern
der Zahl $b$ interpretiert werden.

Die Kopfposition wird durch eine Zahl $h$ dargestellt. Befindet sich
der Kopf im Feld mit der Nummer $i$, wird $h$ auf den Wert $k^i$
gesetzt.

\subsubsection{Arithmetik}
Alle Komponenten der Turing-Maschine werden mit natürlichen Zahlen
und arithmetischen Operationen dargestellt.
Zwar beherrscht LOOP nur die Addition oder Subtraktion einer Konstanten,
aber durch wiederholte Addition einer $1$ kann damit jede beliebige
Addition oder Subtraktion implementiert werden.

Ebenso können Multiplikation und Division auf wiederholte Addition
zurückgeführt werden.
Im Folgenden nehmen wir daher an, dass die arithmetischen Operationen
zur Verfügung stehen.

\subsubsection{Lesen eines Feldes}
Um den Inhalt eines Feldes zu lesen, muss die Stelle von $b$ an der
aktuellen Kopfposition ermittelt werden.
Dies kann durch die Rechnung
\begin{equation}
z=b / h \mod k
\label{getchar}
\end{equation}
ermittelt werden, wobei $/$ für eine ganzzahlige Division steht.
Beide Operationen können mit einem GOTO-Programm ermittelt werden.

\subsubsection{Löschen eines Feldes}
Das Feld an der Kopfposition kann wie folgt gelöscht werden.
Zunächst ermittelt man mit (\ref{getchar}) den aktuellen Feldinhalt.
Dann berechnet man
\begin{equation}
b' = b - z\cdot h.
\label{clearchar}
\end{equation}
$b'$ enthält an der Stelle der Kopfposition ein $0$.

\subsubsection{Schreiben eines Feldes}
Soll das Feld an der Kopfposition mit dem Zeichen $x$ überschrieben
werden, wird mit (\ref{clearchar}) zuerst das Feld gelöscht.
Anschliessend wird das Feld durch
\[
b'=b+x\cdot h
\]
neu gesetzt.

\subsubsection{Kopfbewegung}
Die Kopfposition wird durch die Zahl $h$ dargestellt.
Da $h$ immer eine Potenz von $k$ ist, und die Nummer des Feldes der
Exponent ist, brauchen wir nur Operationen, die den Exponenten
ändern, also
\[
h'=h/k\qquad\text{bzw.}\qquad h'=h\cdot k
\]

\subsubsection{Übergangsfunktion}
Die Übergangsfunktion
\[
\delta\colon Q\times \Gamma\to Q\times \Gamma\times\{1, 2\}
\]
ermittelt aus aktuellem Zustand $s$ und
aktuellem Zeichen $z$ den neuen Zustand, das neue Zeichen auf
dem Band und die Kopfbewegung ermittelt. Im Gegensatz zur früheren
Definition verwenden wir jetzt die Zahlen $1$ und $2$ für die
Kopfbewegung L bzw.~R.
Wir schreiben $\delta_i$ für die $i$-te Komponente von $\delta$.
Der folgende GOTO-Pseudocode
beschreibt also ein Programm, welches die Turing-Maschine implementiert
\begin{algorithmic}[1]
\STATE Bestimme das Zeichen $z$ unter der aktuellen Kopfposition $h$
\STATE Lösche das aktuelle Zeichen auf dem Band
\STATE {\tt IF\ }$s=0${\tt\ THEN}
\STATE {\tt \ \ \ \ IF\ }$z=0${\tt\ THEN}
\STATE {\tt \ \ \ \ \ \ \ \ }$s\assignment\delta_1(0,0)$
\STATE {\tt \ \ \ \ \ \ \ \ }$z\assignment\delta_2(0,0)$
\STATE {\tt \ \ \ \ \ \ \ \ }$m\assignment\delta_3(0,0)$
\STATE {\tt \ \ \ \ END}
\STATE {\tt END}
\STATE {\tt IF\ }$s=1${\tt\ THEN}
\STATE {\tt \ \ \ \ }\dots
\STATE {\tt END}
\STATE Zeichen $z$ schreiben
\STATE {\tt IF\ }$m=1${\tt\ THEN }$h\assignment h/k$
\STATE {\tt IF\ }$m=2${\tt\ THEN }$h\assignment h\cdot k$
\STATE \dots
\STATE {\tt GOTO\ }1
\end{algorithmic}
Damit ist gezeigt, dass eine gegebene Turingmaschine in ein
GOTO-Programm übersetzt werden kann. Übersetzt man die universelle
Turing-Maschine, erhält man ein GOTO-Programm, welches jede beliebige
Turing-Maschine simulieren kann. Somit ist GOTO und damit auch WHILE
Turing-vollständig.

\subsection{Esoterische Programmiersprachen}
\index{Programmiersprache!esoterische}%
Zur Illustration der Tatsache, dass eine sehr primitive Sprache
ausreichen kann, um Turing-Vollständigkeit zu erreichen, wurden
verschiedene esoterische Programmiersprachen erfunden.
Ihre Nützlichkeit liegt darin, ein bestimmtest Konzept der Theorie
möglichst klar hervorzuheben, die Verwendbarkeit für irgend einen
praktischen Zweck ist nicht notwendig, und manchmal explizit unerwünscht.

\subsubsection{Brainfuck}
\index{Brainfuck}%
Brainfuck
wurde von Urban Müller 1993 entwickelt mit dem Ziel, dass der
Compiler für diese Sprache möglichst klein sein sollte. In der
Tat ist der kleinste Brainfuck-Compiler für Linux nur 171 Bytes
lang.

Brainfuck basiert auf einem einzelnen Pointer {\tt ptr}, welcher
im Programm inkrementiert oder dekrementiert werden kann.
Jeder Pointer-Wert zeigt auf eine Zelle, deren Inhalt inkrementiert
oder dekrementiert werden kann.
Dies erinnert an die Position des Kopfes einer Turing-Maschine.
Zwei Instruktionen für Eingabe und Ausgabe eines Zeichens
an der Pointer-Position ermöglichen Datenein- und -ausgabe.
Als einzige Kontrollstruktur steht WHILE zur Verfügung. Damit
die Sprache von einem minimalisitischen Compiler kompiliert
werden kann, wird jede Anweisung durch ein einziges Zeichen
dargestellt. Die Befehle sind in der folgenden Tabelle
zusammen mit ihrem C-Äquivalent zusammengstellt:
\begin{center}
\begin{tabular}{|c|l|}
\hline
Brainfuck&C-Äquivalent\\
\hline
{\tt >}&\verb/++ptr;/\\
{\tt <}&\verb/--ptr;/\\
{\tt +}&\verb/++*ptr;/\\
{\tt -}&\verb/--*ptr;/\\
{\tt .}&\verb/putchar(*ptr);/\\
{\tt ,}&\verb/*ptr = getchar();/\\
{\tt [}&\verb/while (*ptr) {/\\
{\tt ]}&\verb/}/\\
\hline
\end{tabular}
\end{center}
Mit Hilfe des Pointers lassen sich offenbar beliebige Speicherzellen
adressieren, und diese können durch Wiederholung der Befehle {\tt +}
und {\tt -} auch um konstante Werte vergrössert
oder verkleinert werden. Etwas mehr Arbeit erfordert die Zuweisung
eines Wertes zu einer Variablen. Ist dies jedoch geschafft, kann
man WHILE in Brainfuck übersetzen, und hat damit gezeigt, dass
Brainfuck Turing-vollständig ist.

\subsubsection{Ook}
\index{Ook}%
Die Sprache Ook verwendet als syntaktische Element das Wort {\tt Ook} gefolgt
von `{\tt .}', `{\tt !}' oder `{\tt ?}'. Wer beim Lesen eines Ook-Programmes
den Eindruck hat, zum Affen gemacht zu werden, liegt nicht ganz falsch:
Ook ist eine einfache Umcodierung von Brainfuck:
\begin{center}
\begin{tabular}{|c|c|}
\hline
Ook&Brainfuck\\
\hline
{\tt Ook. Ook?}&{\tt >}\\
{\tt Ook? Ook.}&{\tt <}\\
{\tt Ook. Ook.}&{\tt +}\\
{\tt Ook! Ook!}&{\tt -}\\
{\tt Ook! Ook.}&{\tt .}\\
{\tt Ook. Ook!}&{\tt ,}\\
{\tt Ook! Ook?}&{\tt [}\\
{\tt Ook? Ook!}&{\tt ]}\\
\hline
\end{tabular}
\end{center}
Da Brainfuck Turing-vollständig ist, ist auch Ook Turing-vollständig.



%
% Turing-Vollstaendigkeit
%
% (c) 2011 Prof Dr Andreas Mueller, Hochschule Rapperswil
% $Id$
%
\chapter{Turing-Vollständigkeit\label{chapter-vollstaendigkeit}}
\lhead{Vollständigkeit}
%
% vollstaendig.tex -- Turing-Vollständigkeit
%
% (c) 2011 Prof Dr Andreas Mueller, Hochschule Rapperswil
%

\section{Turing-vollständige Programmiersprachen}
\rhead{Turing-vollständige Programmiersprachen}
Die Turing-Maschine liefert einen wohldefinierten Begriff der
Berechenbarkeit, der auch robust gegenüber milden Änderungen
der Definition einer Turing-Maschine ist.
Der Aufbau aus einem endlichen Automaten mit zusätzlichem
Speicher und der einfache Kalkül mit Konfigurationen hat
sie ausserdem Beweisen vieler wichtiger Eigenschaften zugänglich
gemacht. Die vorangegangenen Kapitel über Entscheidbarkeit und
Komplexität legen davon eindrücklich Zeugnis ab. Am direkten
Nutzen dieser Theorie kann jedoch immer noch ein gewisser Zweifel
bestehen, da ein moderner Entwickler seine Programme ja nicht
direkt für eine Turing-Maschine schreibt, sondern nur mittelbar,
da er eine Programmiersprache verwendet, deren Code anschliessend
von einem Compiler oder Interpreter übersetzt und von einer realen
Maschine ausgeführt wird.

Der Aufbau der realen Maschine ist sehr
nahe an einer Turing-Maschine, ein Prozessor liest und schreibt
jeweils einzelne
Speicherzellen eines mindestens für praktische Zwecke unendlich
grossen Speichers und ändert bei Verarbeitung der gelesenen
Inhalte seinen eigenen Zustand. Natürlich ist die Menge der
Zustände eines modernen Prozessors sehr gross, nur schon die $n$
Register der Länge $l$ tragen $2^{nl}$ verschiedene Zustände bei,
und jedes andere Zustandsbit verdoppelt die Zustandsmenge nochmals.
Trotzdem ist die Zustandsmenge endlich, und es braucht nicht viel
Fantasie, sich den Prozessor mit seinem Hauptspeicher als Turingmaschine
vorzustellen. Es gibt also kaum Zweifel, dass die Computer-Hardware
zu all dem fähig ist, was ihr in den letzten zwei Kapiteln an
Fähigkeiten zugesprochen wurde.

Die einzige Einschränkung der Fähigkeiten realer Computer gegenüber
Turing-Maschinen ist
die Tatsache, dass reale Computer nur über einen endlichen Speicher
verfügen, während eine Turing-Maschine ein undendlich langes Band
als Speicher verwenden kann. Da jedoch eine endliche Berechnung auch
nur endlich viel Speicher verwenden kann, sind alle auf einer Turing-Maschine
durchführbaren Berechnungen, die man auch tatsächlich durchführen
will, auch von einem realen Computer durchführbar. Für praktische
Zwecke darf man also annehmen, dass die realen Computer echte Turing-Maschinen
sind.

Trotzdem ist nicht sicher, ob die Programmierung in einer übersetzten
oder interpretierten Sprache alle diese Fähigkeiten auch einem
Anwendungsprogrammierer zugänglich macht.
Letztlich äussert sich dies auch darin, dass Computernutzer
für verschiedene Problemstellung auch verschiedene Werkzeuge
verwenden. Wer tabellarische Daten summieren will, wird gerne
zu einer Spreadsheet-Software greifen, aber nicht erwarten, dass er
damit auch einen Näherungsalgorithmus für das Cliquen-Problem wird
programmieren können. Die Tabellenkalkulation definiert ein eingeschränktes,
an das Problem angepasstes Berechnungsmodell, welches aber
höchstwahrscheinlich weniger leistungsfähig ist als die Hardware, auf
der es läuft. Es ist also durchaus möglich und je nach Anwendung auch
zweckmässig, dass ein Anwender nicht die volle Leistung einer
Turing-Maschine zur Verfügung hat.

Damit stellt sich jetzt die Frage, wie man einem Berechnungsmodell und
das heisst letztlich der Sprache, in der der Berechnungsauftrag
formuliert wird, ansehen kann, ob sie gleich mächtig ist wie eine
Turing-Maschine.

\subsection{Programmiersprachen}
Eine Programmiersprache ist zwar eine Sprache im Sinne dieses Skriptes,
für den Programmierer wesentlich ist jedoch die Semantik, die bisher
nicht Bestandteil der Diskussion war. Für ihn ist die Tatsache wichtig,
dass die Semantik der Sprache Berechnungen beschreibt,
wie sie mit einer Turing-Maschine ausgeführt werden können.

\begin{definition}
\index{Programmiersprache}%
Eine Sprache $A$ heisst eine {\em Programmiersprache}, wenn es eine Abbildung
\[
c\colon A\to \Sigma^*\colon w\mapsto c(w)
\]
gibt, die einem Wort der Sprache die Beschreibung einer Turing-Maschine
zuordnet. Die Abbildung $c$ heisst {\em Compiler} für die Sprache $A$.
\end{definition}
Die Forderung, dass $c(w)$ die Beschreibung einer Turing-Maschine
sein muss, ist nach obiger Diskussion nicht wesentlich.

\subsection{Interaktion}
Man beachte, dass in dieser Definition einer Programmiersprache kein Platz ist
für Input oder Output während des Programmlaufes.
Das Band der Turing-Maschine, bzw.~sein Inhalt bildet den Input, der Output
kann nach Ende der Berechnung vom Band gelesen werden.
Man könnte dies als Mangel dieses Modells ansehen, in der Tat ist aber
keine Erweiterung nötig, um Interaktion abzubilden.
Interaktionen mit einem Benutzer bestehen immer aus einem Strom von
Ereignissen, die dem Benutzer zufliessen (Änderungen des Bildschirminhaltes,
Signaltöne) oder die der Benutzer veranlasst (Bewegungen des Maus-Zeigers,
Maus-Klicks, Tastatureingaben). Alle diese Ereignisse kann man sich codiert
auf ein Band geschrieben denken, welches die Turing-Maschine bei Bedarf
liest.

Der Inhalt des Bandes einer Standard-Turing-Maschine kann während des
Programmlaufes nur von der Turing-Maschine selbst verändert werden.
Da sich die Turing-Maschine aber nicht daran erinnern kann, was beim
letzten Besuch eines Feldes dort stand, ist es für eine Turing-Maschine
auch durchaus akzeptabel, wenn der Inhalt eines Feldes von aussen
geändert wird. Natürlich werden damit das Laufzeitverhalten der
Turing-Maschine verändert. Doch in Anbetracht der
Tatsache, dass von einer Turing-Maschine im Allgemeinen nicht einmal
entschieden werden kann, ob sie anhalten wird, ist wohl nicht mehr
viel zu verlieren.

Die Ausgaben eines Programmes sind deterministisch, und was der Benutzer
erreichen will, sowie die Ereignisse, die er einspeisen wird, sind es ebenfalls.
Man kann also im Prinzip im Voraus wissen, was ausgegeben werden wird
und welche Ereignisse ein Benutzer auslösen wird. Schreibt man diese
vorgängig auf das Band, so wie man es auch beim automatisierten Testen
eines Userinterfaces tut, entsteht aus dem interaktiven Programm eines,
welches ohne Zutun des Benutzers zur Laufzeit funktionieren kann.

\subsection{Die universelle Turing-Maschine}
\index{Turing, Alan}%
\index{Turing-Maschine!universelle}%
In seinem Paper von 1936 hat Alan Turing gezeigt, dass man eine
Turing-Maschine definieren kann,
der man die Beschreibung
$\langle M,w\rangle$
einer Turing-Maschine $M$ und eines Wortes $w$
und die $M$ auf dem Input-Wort $w$ simuliert.
Diese spezielle Turing-Maschine ist also leistungsfähig genug, jede
beliebige andere Turing-Maschine zu simulieren. Sie heisst die {\em universelle
Turing-Maschine}.

Die universelle Turing-Maschine kann die Entscheidung vereinfachen,
ob eine Funktion Turing-berechenbar ist. Statt eine Turing-Maschine
zu beschreiben, die die Funktion berechnet, reicht es, ein Programm
in der Programmiersprache $A$ zu beschreiben, das Programm mit dem
Compiler $c$ zu übersetzen, und die Beschreibung mit der universellen
Turing-Maschine auszuführen.

\index{Church-Turing-Hypothese}%
Die Church-Turing-Hypothese besagt, dass sich alles, was man berechnen
kann, auch mit einer Turing-Maschine berechnen lässt. Die universelle
Turing-Maschine zeigt, dass jede berechenbare Funktion von der
universellen Turing-Maschine berechnet werden kann.
Etwas leistungsfähigeres als eine Turing-Maschine gibt es nicht.

\subsection{Turing-Vollständigkeit}
Jede Funktion, die in der Programmiersprache $A$ implementiert werden
kann, ist Turing-berechenbar.
Der Compiler kann
aber durchaus Fähigkeiten unzugänglich machen, die Programmiersprache
$A$ kann dann gewisse Berechnungen, die mit einer Turing-Maschine
möglich wären, nicht formulieren. Besonderes interessant sind daher
die Sprachen, bei denen ein solcher Verlust nicht eintritt.

\begin{definition}
\index{Turing-vollständig}%
Eine Programmiersprache heisst Turing-vollständig, wenn sich jede
berechenbare Abbildung in dieser Sprache formulieren lässt.
Zu jeder berechenbaren Abbildung $f\colon\Sigma^*\to \Sigma^*$ gibt
es also ein Programm $w$ so, dass $c(w)$ die Funktion $f$ berechnet.
\end{definition}

Zu einer berechenbaren Abbildung gibt es eine Turing-Maschine, die
sie berechnet, es würde also genügen, wenn man diese Turing-Maschine
von einem in der Sprache $A$ geschriebenen Turing-Maschinen-Simulator
ausführen lassen könnte. Dieser Begriff muss noch etwas klarer gefasst
werden:

\begin{definition}
\index{Turing-Maschinen-Simulator}%
Ein Turing-Maschinen-Simulator ist eine Turing-Maschine $S$, die als Input
die Beschreibung $\langle M,w\rangle$ einer Turing-Maschine $M$ und eines
Input-Wortes für $M$ erhält, und die Berechnung durchführt, die $M$ auf $w$
ausführen würde.
Ein Turing-Maschinen-Simulator in der Programmiersprache $A$ ist
ein Wort $s\in A$ so, dass $c(s)$ ein Turing-Maschinen-Simulator ist.
\end{definition}

Damit erhalten wir ein Kriterium für Turing-Vollständigkeit:

\begin{satz}
\label{turingvollstaendigkeitskriterium}
Eine Programmiersprache $A$ ist Turing-vollständig, genau dann
wenn es einen Turing-Maschinen-Simulator in $A$ gibt.
\end{satz}


\subsection{Beispiele}
Die üblichen Programmiersprachen sind alle Turing-vollständig, denn es
ist eine einfache Programmierübung, eines Turing-Maschinen-Simulator
in einer dieser Sprachen zu schreiben. In einigen Programmiersprachen
ist dies jedoch schwieriger als in anderen.

\subsubsection{Javascript}
\index{Javascript}%
Fabrice Bellard hat 2011 einen PC-Emulator in Javascript geschrieben, der
leistungsfähig genug ist, Linux zu booten. Auf seiner Website
\url{http://bellard.org/jslinux/} kann man den Emulator im eigenen Browser
starten. Das gebootete Linux enthält auch einen C-Compiler. Da C
Turing-vollständig ist, gibt es einen Turing-Maschinen-Simulator in
C, den man auch auf dieses Linux bringen und mit dem C-Compiler
kompilieren kann. Somit gibt es einen Turing-Maschinen-Simulator in
Javascript, Javascript ist Turing-vollständig.

\subsubsection{XSLT}
\index{XSLT}%
XSLT ist eine XML-basierte Sprache, die Transformationen von XML-Dokumenten
zu beschreiben erlaubt. XSLT ist jedoch leistungsfähig, eine Turing-Maschine
zu simulieren. Bob Lyons hat auf seiner Website
\url{http://www.unidex.com/turing/utm.htm} ein XSL-Stylesheet publiziert,
welches einen Simulator implementiert. Als Input verlangt es
ein
XML-Dokument, welches die Beschreibung der Turing-Maschine in einem
zu diesem Zweck definierten XML-Format namens Turing Machine Markup
Language (TMML) enthält. TMML definiert XML-Elemente, die das Alphabet
(\verb+<symbols>+),
die Zustandsmenge $Q$ (\verb+<state>...</state>+)
und die Übergangsfunktion $\delta$ in \verb+<mapping>+-Elementen
der Form
\begin{verbatim}
<mapping>
    <from current-state="moveRight1" current-symbol=" " />
    <to next-state="check1" next-symbol=" " movement="left" />
</mapping>
\end{verbatim}
beschreiben. Der initiale Bandinhalt wird als Parameter \verb+tape+
auf der Kommandozeile übergeben.
Das Stylesheet wandelt das TMML Dokument in eine ausführliche
Berechnungsgeschichte um, aus der auch der Bandinhalt am Ende der Berechnung
abzulesen ist. Es beweist somit, dass XSLT einen Turing-Maschinen-Simulator
hat, also Turing-vollständdig ist.

\subsubsection{\LaTeX}
\index{LaTeX@\LaTeX}%
\index{Knuth, Don}%
Don Knuth, der Autor von \TeX, hat sich lange davor gedrückt, seiner
Schriftsatz-Sprache auch eine Turing-vollständige Programmiersprache
zu spendieren. Schliesslich kam er nicht mehr darum herum, und wurde
von Guy Steeles richtigegehend dazu gedrängt, wie er in
\url{http://maps.aanhet.net/maps/pdf/16\_15.pdf}
gesteht.

Dass \TeX Turing-vollständig ist beweist ein Satz von \LaTeX-Macros, den
man auf
\url{https://www.informatik.uni-augsburg.de/en/chairs/swt/ti/staff/mark/projects/turingtex/}
finden kann.
Um ihn zu verwenden, formuliert man die Beschreibung
von Turing-Maschine und initialem Bandinhalt als eine Menge von
\LaTeX-Makros. Ebenso ruft man den Makro \verb+\RunTuringMachine+ auf,
der die Turing-Machine simuliert und die Berechnungsgeschichte im
\TeX-üblichen perfekten Schriftsatz ausgibt.




%
% kontrollstrukturen.tex
%
% (c) 2011 Prof Dr Andreas Mueller, Hochschule Rapperswil
%
\section{Kontrollstrukturen und Turing-Vollständigkeit}
\rhead{Kontrollstrukturen}
Das Turing-Vollstandigkeits-Kriterium von Satz
\ref{turingvollstaendigkeitskriterium} verlangt, dass man einen
Turing-Maschinen-Simulator in der gewählten Sprache schreiben muss.
Dies ist in jedem Fall eine nicht triviale Aufgabe.
Daher wäre es nützlich Kriterien zu erhalten, welche einfacher
anzuwenden sind. Einen wesentlichen Einfluss auf die Möglichkeiten,
was sich mit einer Programmiersprache ausdrücken lassen, haben die
Kontrollstrukturen.

\newcommand{\assignment}{\mathbin{\texttt{:=}}}

\subsection{Grundlegende Syntaxelemente%
\label{subsection:grundlegende-syntaxelement}}
In den folgenden Abschnitten werden verschiedene vereinfachte Sprachen
diskutiert, die aber einen gemeinsamen Kern fundamentaler Anweisungen
beinahlten.
In allen Sprachen gibt es nur einen einzigen Datentyp, nämlich
natürliche Zahlen.
Einerseits können Konstanten beliebig grosse natürliche Werte haben,
andererseits lassen sich in Variablen beliebig grosse natürliche Zahlen
speichern.
Dazu sind die folgenden Sytanxelemente notwendig:
\begin{compactitem}
\item Konstanten: {\tt 0}, {\tt 1}, {\tt 2}, \dots
\item Variablen $x_0$, $x_1$, $x_2$,\dots
\item Zuweisung: $\assignment$
\item Trennung von Anweisungen: {\tt ;}
\item Operatoren: {\tt +} und {\tt -}
\end{compactitem}
Die einzige Möglichkeit, den Wert einer Variablen zu ändern ist die
Zuweisung.  Diese ist entweder die Zuweisung einer Konstante als
Wert einer Variable:
\begin{algorithmic}
\STATE $x_i\assignment c$
\end{algorithmic}
oder eine Berechnung mit den beiden vorhandenen Operatoren
\begin{algorithmic}
\STATE $x_i \assignment x_j$ {\tt +} $c$
\STATE $x_i \assignment x_j$ {\tt -} $c$
\end{algorithmic}
Dabei ist das Resultat der Subtraktion als $0$ definiert, wenn
der Minuend kleiner ist als der Subtrahend.

Die verschiedenen Sprachen unterschieden sich in den Kontrollstrukturen
mit denen diese Zuweisungsbefehle miteinander verbunden werden können.

\subsection{LOOP}
\index{LOOP}%
Die Programmiersprache
LOOP\footnote{Die in diesem Abschnitt beschriebene
LOOP Sprache darf nicht verwechselt werden mit dem gleichnamigen
Projekt einer objektorientierten parallelen Sprache seit
2001 auf Sourceforge.}
hat als einzige Kontrollstruktur die
Iteration eines Anweisungs-Blocks mit einer festen, innerhalb des
Blocks nicht veränderbaren Anzahl von Durchläufen.

\subsubsection{Syntax}
LOOP verwendet die 
Schlüsselwörter: {\tt LOOP}, {\tt DO}, {\tt END}
Programme werden daraus wie folgt aufgebaut:
\begin{itemize}
\item Das leere Programm $\varepsilon$ ist eine LOOP-Programm,
während der Ausführung tut es nichts.
\item Wertzuweisungen sind LOOP-Programme
\item Sind $P_1$ und $P_2$ LOOP-Programme, dann auch
$P_1{\tt ;}P_2$. Um dieses Programm auszuführen wird zuerst $P_1$
ausgeführt und anschliessend $P_2$.
\item Ist $P$ ein LOOP-Programm, dann ist auch
\begin{algorithmic}
\STATE {\tt LOOP} $x_i$ {\tt DO} $P$ {\tt END}
\end{algorithmic}
ein LOOP-Programm. Die Ausführung wiederholt $P$ so oft, wie der
Wert der Variable $x_i$  zu Beginn angibt.
\end{itemize}
Durch Setzen der Variablen $x_i$ kann einem LOOP-Progamm Input übergeben
werden.
Ein LOOP-Programm definiert also eine Abbildung $\mathbb N^k\to\mathbb N$,
wobei $k$ die Anzahl der Variablen ist, in denen Input zu übergeben ist.

\subsubsection{Beispiel: Summer zweier Variablen}
LOOP kann nur jeweils eine Konstante hinzuaddieren, das genügt aber
bereits, um auch die Summe zweier Variablen zu berechnen. Der folgende
Code berechnet die Summe von $x_1$ und $x_2$ und legt das Resultat in
$x_0$ ab:
\begin{algorithmic}
\STATE $x_0 \assignment x_1$
\STATE {\tt LOOP} $x_2$ {\tt DO}
\STATE{\tt \ \ \ \ }$x_0 \assignment x_0$ {\tt +} $1$
\STATE{\tt END}
\end{algorithmic}

\subsubsection{Beispiel: Produkt zweier Variablen}
\index{Multiplikation}%
Die Sprache LOOP ist offenbar relativ primitiv, trotzdem
ist sie leistungsfähig genug, um die Multiplikation von zwei
Zahlen, die in den Variablen $x_1$ und $x_2$ übergeben werden,
zu berechnen:

\begin{algorithmic}
\STATE $x_0 \assignment 0$
\STATE {\tt LOOP} $x_1$ {\tt DO}
\STATE{\tt \ \ \ \ LOOP} $x_2$ {\tt DO}
\STATE{\tt \ \ \ \ \ \ \ \ }$x_0 \assignment x_0$ {\tt +} $1$
\STATE{\tt \ \ \ \ END}
\STATE{\tt END}
\end{algorithmic}

\subsubsection{IF-Anweisung}
In LOOP fehlt eine IF-Anweisung, auf die die wenigsten Programmiersprachen
verzichten. Es ist jedoch leicht, eine solche in LOOP nachzubilden.
Eine Anweisung
\begin{algorithmic}
\STATE {\tt IF} $x = 0$ {\tt THEN } $P$ {\tt END}
\end{algorithmic}
kann unter Verwendung einer zusätzlichen Variable $y$ in LOOP ausgedrückt
werden:
\begin{algorithmic}[1]
\STATE $y \assignment 1${\tt ;}
\STATE {\tt LOOP }$x${\tt\ DO }$y \assignment 0$ {\tt END;}
\STATE {\tt LOOP }$y${\tt\ DO }$P$ {\tt END;}
\end{algorithmic}
Natürlich wird die {\tt LOOP}-Anweisung in Zeile 2 meistens viel öfter
als nötig ausgeführt, aber es stehen hier ja auch nicht Fragen
der Effizienz sonder der prinzipiellen Machbarkeit zur Diskussion.

\subsubsection{LOOP ist nicht Turing-vollständig}
\begin{satz}
LOOP-Programme terminieren immer.
\end{satz}

\begin{proof}[Beweis]
Der Beweis kann mit Induktion über die Schachtelungstiefe
von {\tt LOOP}-Anweisungen geführt werden. Die LOOP-Programme
ohne {\tt LOOP}-Anweisung, also die Programme mit Schachtelungstiefe
$0$ terminieren offensichtlich immer. Ebenso die LOOP-Programme
mit Schachtelungstiefe $1$, da die Anzahl der Schleifendurchläufe
bereits zu Beginn der Schleife festliegt.

Nehmen wir jetzt an wir wüssten bereits, dass jedes LOOP-Programm mit
Schachtelungstiefe $n$ der {\tt LOOP}-Anweisungen immer terminiert.
Ein LOOP-Programm mit Schachtelungstiefe $n+1$ ist dann eine
Abfolge von Teilprogrammen mit Schachtelungstiefe $n$, die nach
Voraussetzung alle terminieren, und {\tt LOOP}-Anweisungen der
Form
\begin{algorithmic}
\STATE{\tt LOOP }$x_i${\tt\ DO }$P${\tt\ END}
\end{algorithmic}
wobei $P$ ein LOOP-Programm mit Schachtelungstiefe $n$ ist, das also
ebenfalls immer terminiert. $P$ wird genau so oft ausgeführt, wie
$x_i$ zu Beginn angibt. Die Laufzeit von $P$ kann dabei jedesmal
anders sein, $P$ wird aber auf jeden Fall terminieren. Damit ist
gezeigt, dass auch alle LOOP-Programme mit Schachtelungstiefe $n+1$
terminieren.
\end{proof}

\begin{satz}
LOOP ist nicht Turing vollständig.
\end{satz}

\begin{proof}[Beweis]
Es gibt Turing-Maschinen, die nicht terminieren. Gäbe es einen
Turing-Maschinen-Simulator in LOOP, dürfte dieser bei der
Simulation einer solchen Turing-Maschine nicht terminieren, im
Widerspruch zur Tatsache, dass LOOP-Programme immer terminieren.
Also kann es keinen Turing-Maschinen-Simulator in LOOP geben.
\end{proof}

\subsection{WHILE}
\index{WHILE}%
WHILE-Programme können zusätzlich zur {\tt LOOP}-Anweisung
eine {\tt WHILE}-Anweisung der Form
\begin{algorithmic}
\STATE{\tt WHILE }$x_i>0${\tt\ DO }$P${\tt\ END}
\end{algorithmic}
Dadurch ist die {\tt LOOP}-Anweisung nicht mehr unbedingt
notwendig, denn
\begin{algorithmic}
\STATE{\tt LOOP }$x${\tt\ DO }$P${\tt\ END}
\end{algorithmic}
kann durch
\begin{algorithmic}
\STATE$y \assignment x$;
\STATE{\tt WHILE }$y>0${\tt\ DO }$P$; $y \assignment y-1${\tt\ END}
\end{algorithmic}
nachgebildet werden.

\subsection{GOTO}
\index{GOTO}%
GOTO-Programme bestehen aus einer markierten Folge von Anweisungen
\begin{center}
\begin{tabular}{rl}
$M_1$:&$A_1$\\
$M_2$:&$A_2$\\
$M_3$:&$A_3$\\
$\dots$:&$\dots$\\
$M_k$:&$A_k$
\end{tabular}
\end{center}
Zu den bereits bekannten,
Abschnitt~\ref{subsection:grundlegende-syntaxelement} beschriebenen
Zuweisungen
$x_i \assignment c$ 
und
$x_i \assignment x_j \pm c$ 
kommt bei GOTO eine bedingte Sprunganweisung
\begin{center}
\begin{tabular}{rl}
$M_l$:&{\tt IF\ }$x_i=c${\tt\ THEN GOTO\ }$M_j$
\end{tabular}
\end{center}
Natürlich lässt sich damit auch eine unbedingte Sprunganweisung
implementieren:
\begin{center}
\begin{tabular}{rl}
$M_l$:&$x_i \assignment c$;\\
$M_{l+1}$:&{\tt IF\ }$x_i=c${\tt\ THEN GOTO\ }$M_j$
\end{tabular}
\end{center}
In ähnlicher Weise lassen sich auch andere bedingte Anweisungen
konstruieren, zum Beispiel ein
Konstrukt {\tt IF }\dots{\tt\ THEN }\dots{\tt\ ELSE }\dots{\tt\ END}, welches
einen ganzen Anweisungsblock enthalten kann.

\subsection{Äquivalenz von WHILE und GOTO}
Die Verwendung eines Sprungbefehles wie GOTO ist in der modernen
Softwareentwicklung verpönt. Sie führe leichter zu Spaghetti-Code,
der kaum mehr wartbar ist. Gewisse Sprachen verbannen daher
GOTO vollständig aus ihrer Syntax, und propagieren dagegen
die Verwendung von `strukturierten' Kontrollstrukturen wie
WHILE. Die Aufregung ist allerdings unnötig: GOTO und WHILE sind äquivalent.


\begin{satz}
Eine Funktion ist genau dann mit einem GOTO-Programm berechenbar,
wenn sie mit einem WHILE-Programm berechenbar ist.
\end{satz}
\begin{proof}[Beweis]
Man braucht nur zu zeigen, dass man ein GOTO-Programm in ein äquivalentes
WHILE-Programm übersetzen kann, und umgekehrt.

Um eine GOTO-Programm zu übersetzen, verwenden wir eine zusätzliche Variable
$z$, die die Funktion des Programm-Zählers übernimmt.
Aus dem GOTO-Programm machen wir dann folgendes WHILE-Programm
\begin{algorithmic}
\STATE $z \assignment 1$
\STATE{\tt WHILE\ }$z>0${\tt\ DO}
\STATE{\tt IF\ }$z=1${\tt\ THEN\ }$A_1'${\tt\ END};
\STATE{\tt IF\ }$z=2${\tt\ THEN\ }$A_2'${\tt\ END};
\STATE{\tt IF\ }$z=3${\tt\ THEN\ }$A_3'${\tt\ END};
\STATE\dots
\STATE{\tt IF\ }$z=k${\tt\ THEN\ }$A_k'${\tt\ END};
\STATE{\tt IF\ }$z=k+1${\tt\ THEN\ }$z \assignment 0${\tt\ END};
\STATE{\tt END}
\end{algorithmic}
Die Anweisung $A_i'$ entsteht aus der Anweisung $A_i$ nach folgenden
Regeln
\begin{itemize}
\item Falls $A_i$ eine Zuweisung ist, wird ihr eine weitere Zuweisung
\begin{algorithmic}
\STATE $z \assignment z+1$
\end{algorithmic}
angehängt.
Dies hat zur Folge, dass nach $A_z'$ als nächste Anweisung
$A_{z+1}$ ausgeführt wird.
\item
Falls $A_i$ eine bedingte Sprunganweisung
\begin{center}
\begin{tabular}{rl}
$M_l$:&{\tt IF\ }$x_i=c${\tt\ THEN GOTO\ }$M_j$
\end{tabular}
\end{center}
ist, wird $A_i'$
\begin{algorithmic}
\STATE{\tt IF\ }$x_i=c${\tt\ THEN\ }$z \assignment j${\tt\ ELSE }$z=z+1$;
\end{algorithmic}
Dies ist zwar keine WHILE-Anweisung, aber es wurde bereits
früher gezeigt, wie man sie in WHILE übersetzen kann.
\end{itemize}
Damit ist gezeigt, dass ein GOTO-Programm in ein äquivalentes WHILE-Programm
mit genau einer WHILE-Schleife übersetzt werden kann.

Umgekehrt zeigen wir, dass jede WHILE-Schleife mit Hilfe von GOTO
implementiert werden kann. Dazu übersetzt man jede WHILE-Schleife
der Form
\begin{algorithmic}
\STATE{\tt WHILE\ }$x_i>0${\tt\ DO }$P${\tt\ END}
\end{algorithmic}
in ein GOTO-Programm-Segment der Form
\begin{algorithmic}[1]
\STATE{\tt IF\ }$x_i=0${\tt\ THEN GOTO }4
\STATE$P$
\STATE{\tt GOTO\ }1
\STATE
\end{algorithmic}
wobei die Zeilennummern durch geeignete Marken ersetzt werden müssen.
Damit haben wir einen Algorithmus spezifiziert, der WHILE-Programme in
GOTO-Programme übersetzen kann.
\end{proof}

\subsection{Turing-Vollständigkeit von WHILE und GOTO}
Da WHILE und GOTO äquivalent sind, braucht die Turing-Vollständigkeit
nur für eine der Sprachen gezeigt zu werden.
Wir skizzieren, wie man eine Turing-Maschine in ein GOTO-Programm
übersetzen kann. Dies genügt, da man nur die universelle
Turing-Maschine zu übersetzen braucht, um damit jede andere
Turing-Maschine ausführen zu können.

\subsubsection{Alphabet, Zustände und Band}
Die Zeichen des Bandalphabetes werden durch natürliche Zahlen
dargestellt.  Wir nehmen an, dass das Bandalphabet $k$ verschiedene Zeichen
umfasst. Das Leerzeichen $\text{\textvisiblespace}$ wird durch die Zahl $0$
dargestellt.

Auch die Zustände der Turing-Maschine werden durch natürliche Zahlen
dargestellt,
die Variable $s$ dient dazu, den aktuellen Zustand zu
speichern.

Der Inhalt des Bandes kann durch eine einzige Variable $b$ dargestellt
werden. Schreibt man die Zahl im System zur Basis $k$, können die
die Zeichen in den einzelnen Felder des Bandes als die Ziffern
der Zahl $b$ interpretiert werden.

Die Kopfposition wird durch eine Zahl $h$ dargestellt. Befindet sich
der Kopf im Feld mit der Nummer $i$, wird $h$ auf den Wert $k^i$
gesetzt.

\subsubsection{Arithmetik}
Alle Komponenten der Turing-Maschine werden mit natürlichen Zahlen
und arithmetischen Operationen dargestellt.
Zwar beherrscht LOOP nur die Addition oder Subtraktion einer Konstanten,
aber durch wiederholte Addition einer $1$ kann damit jede beliebige
Addition oder Subtraktion implementiert werden.

Ebenso können Multiplikation und Division auf wiederholte Addition
zurückgeführt werden.
Im Folgenden nehmen wir daher an, dass die arithmetischen Operationen
zur Verfügung stehen.

\subsubsection{Lesen eines Feldes}
Um den Inhalt eines Feldes zu lesen, muss die Stelle von $b$ an der
aktuellen Kopfposition ermittelt werden.
Dies kann durch die Rechnung
\begin{equation}
z=b / h \mod k
\label{getchar}
\end{equation}
ermittelt werden, wobei $/$ für eine ganzzahlige Division steht.
Beide Operationen können mit einem GOTO-Programm ermittelt werden.

\subsubsection{Löschen eines Feldes}
Das Feld an der Kopfposition kann wie folgt gelöscht werden.
Zunächst ermittelt man mit (\ref{getchar}) den aktuellen Feldinhalt.
Dann berechnet man
\begin{equation}
b' = b - z\cdot h.
\label{clearchar}
\end{equation}
$b'$ enthält an der Stelle der Kopfposition ein $0$.

\subsubsection{Schreiben eines Feldes}
Soll das Feld an der Kopfposition mit dem Zeichen $x$ überschrieben
werden, wird mit (\ref{clearchar}) zuerst das Feld gelöscht.
Anschliessend wird das Feld durch
\[
b'=b+x\cdot h
\]
neu gesetzt.

\subsubsection{Kopfbewegung}
Die Kopfposition wird durch die Zahl $h$ dargestellt.
Da $h$ immer eine Potenz von $k$ ist, und die Nummer des Feldes der
Exponent ist, brauchen wir nur Operationen, die den Exponenten
ändern, also
\[
h'=h/k\qquad\text{bzw.}\qquad h'=h\cdot k
\]

\subsubsection{Übergangsfunktion}
Die Übergangsfunktion
\[
\delta\colon Q\times \Gamma\to Q\times \Gamma\times\{1, 2\}
\]
ermittelt aus aktuellem Zustand $s$ und
aktuellem Zeichen $z$ den neuen Zustand, das neue Zeichen auf
dem Band und die Kopfbewegung ermittelt. Im Gegensatz zur früheren
Definition verwenden wir jetzt die Zahlen $1$ und $2$ für die
Kopfbewegung L bzw.~R.
Wir schreiben $\delta_i$ für die $i$-te Komponente von $\delta$.
Der folgende GOTO-Pseudocode
beschreibt also ein Programm, welches die Turing-Maschine implementiert
\begin{algorithmic}[1]
\STATE Bestimme das Zeichen $z$ unter der aktuellen Kopfposition $h$
\STATE Lösche das aktuelle Zeichen auf dem Band
\STATE {\tt IF\ }$s=0${\tt\ THEN}
\STATE {\tt \ \ \ \ IF\ }$z=0${\tt\ THEN}
\STATE {\tt \ \ \ \ \ \ \ \ }$s\assignment\delta_1(0,0)$
\STATE {\tt \ \ \ \ \ \ \ \ }$z\assignment\delta_2(0,0)$
\STATE {\tt \ \ \ \ \ \ \ \ }$m\assignment\delta_3(0,0)$
\STATE {\tt \ \ \ \ END}
\STATE {\tt END}
\STATE {\tt IF\ }$s=1${\tt\ THEN}
\STATE {\tt \ \ \ \ }\dots
\STATE {\tt END}
\STATE Zeichen $z$ schreiben
\STATE {\tt IF\ }$m=1${\tt\ THEN }$h\assignment h/k$
\STATE {\tt IF\ }$m=2${\tt\ THEN }$h\assignment h\cdot k$
\STATE \dots
\STATE {\tt GOTO\ }1
\end{algorithmic}
Damit ist gezeigt, dass eine gegebene Turingmaschine in ein
GOTO-Programm übersetzt werden kann. Übersetzt man die universelle
Turing-Maschine, erhält man ein GOTO-Programm, welches jede beliebige
Turing-Maschine simulieren kann. Somit ist GOTO und damit auch WHILE
Turing-vollständig.

\subsection{Esoterische Programmiersprachen}
\index{Programmiersprache!esoterische}%
Zur Illustration der Tatsache, dass eine sehr primitive Sprache
ausreichen kann, um Turing-Vollständigkeit zu erreichen, wurden
verschiedene esoterische Programmiersprachen erfunden.
Ihre Nützlichkeit liegt darin, ein bestimmtest Konzept der Theorie
möglichst klar hervorzuheben, die Verwendbarkeit für irgend einen
praktischen Zweck ist nicht notwendig, und manchmal explizit unerwünscht.

\subsubsection{Brainfuck}
\index{Brainfuck}%
Brainfuck
wurde von Urban Müller 1993 entwickelt mit dem Ziel, dass der
Compiler für diese Sprache möglichst klein sein sollte. In der
Tat ist der kleinste Brainfuck-Compiler für Linux nur 171 Bytes
lang.

Brainfuck basiert auf einem einzelnen Pointer {\tt ptr}, welcher
im Programm inkrementiert oder dekrementiert werden kann.
Jeder Pointer-Wert zeigt auf eine Zelle, deren Inhalt inkrementiert
oder dekrementiert werden kann.
Dies erinnert an die Position des Kopfes einer Turing-Maschine.
Zwei Instruktionen für Eingabe und Ausgabe eines Zeichens
an der Pointer-Position ermöglichen Datenein- und -ausgabe.
Als einzige Kontrollstruktur steht WHILE zur Verfügung. Damit
die Sprache von einem minimalisitischen Compiler kompiliert
werden kann, wird jede Anweisung durch ein einziges Zeichen
dargestellt. Die Befehle sind in der folgenden Tabelle
zusammen mit ihrem C-Äquivalent zusammengstellt:
\begin{center}
\begin{tabular}{|c|l|}
\hline
Brainfuck&C-Äquivalent\\
\hline
{\tt >}&\verb/++ptr;/\\
{\tt <}&\verb/--ptr;/\\
{\tt +}&\verb/++*ptr;/\\
{\tt -}&\verb/--*ptr;/\\
{\tt .}&\verb/putchar(*ptr);/\\
{\tt ,}&\verb/*ptr = getchar();/\\
{\tt [}&\verb/while (*ptr) {/\\
{\tt ]}&\verb/}/\\
\hline
\end{tabular}
\end{center}
Mit Hilfe des Pointers lassen sich offenbar beliebige Speicherzellen
adressieren, und diese können durch Wiederholung der Befehle {\tt +}
und {\tt -} auch um konstante Werte vergrössert
oder verkleinert werden. Etwas mehr Arbeit erfordert die Zuweisung
eines Wertes zu einer Variablen. Ist dies jedoch geschafft, kann
man WHILE in Brainfuck übersetzen, und hat damit gezeigt, dass
Brainfuck Turing-vollständig ist.

\subsubsection{Ook}
\index{Ook}%
Die Sprache Ook verwendet als syntaktische Element das Wort {\tt Ook} gefolgt
von `{\tt .}', `{\tt !}' oder `{\tt ?}'. Wer beim Lesen eines Ook-Programmes
den Eindruck hat, zum Affen gemacht zu werden, liegt nicht ganz falsch:
Ook ist eine einfache Umcodierung von Brainfuck:
\begin{center}
\begin{tabular}{|c|c|}
\hline
Ook&Brainfuck\\
\hline
{\tt Ook. Ook?}&{\tt >}\\
{\tt Ook? Ook.}&{\tt <}\\
{\tt Ook. Ook.}&{\tt +}\\
{\tt Ook! Ook!}&{\tt -}\\
{\tt Ook! Ook.}&{\tt .}\\
{\tt Ook. Ook!}&{\tt ,}\\
{\tt Ook! Ook?}&{\tt [}\\
{\tt Ook? Ook!}&{\tt ]}\\
\hline
\end{tabular}
\end{center}
Da Brainfuck Turing-vollständig ist, ist auch Ook Turing-vollständig.



%
% Turing-Vollstaendigkeit
%
% (c) 2011 Prof Dr Andreas Mueller, Hochschule Rapperswil
% $Id$
%
\chapter{Turing-Vollständigkeit\label{chapter-vollstaendigkeit}}
\lhead{Vollständigkeit}
%
% vollstaendig.tex -- Turing-Vollständigkeit
%
% (c) 2011 Prof Dr Andreas Mueller, Hochschule Rapperswil
%

\section{Turing-vollständige Programmiersprachen}
\rhead{Turing-vollständige Programmiersprachen}
Die Turing-Maschine liefert einen wohldefinierten Begriff der
Berechenbarkeit, der auch robust gegenüber milden Änderungen
der Definition einer Turing-Maschine ist.
Der Aufbau aus einem endlichen Automaten mit zusätzlichem
Speicher und der einfache Kalkül mit Konfigurationen hat
sie ausserdem Beweisen vieler wichtiger Eigenschaften zugänglich
gemacht. Die vorangegangenen Kapitel über Entscheidbarkeit und
Komplexität legen davon eindrücklich Zeugnis ab. Am direkten
Nutzen dieser Theorie kann jedoch immer noch ein gewisser Zweifel
bestehen, da ein moderner Entwickler seine Programme ja nicht
direkt für eine Turing-Maschine schreibt, sondern nur mittelbar,
da er eine Programmiersprache verwendet, deren Code anschliessend
von einem Compiler oder Interpreter übersetzt und von einer realen
Maschine ausgeführt wird.

Der Aufbau der realen Maschine ist sehr
nahe an einer Turing-Maschine, ein Prozessor liest und schreibt
jeweils einzelne
Speicherzellen eines mindestens für praktische Zwecke unendlich
grossen Speichers und ändert bei Verarbeitung der gelesenen
Inhalte seinen eigenen Zustand. Natürlich ist die Menge der
Zustände eines modernen Prozessors sehr gross, nur schon die $n$
Register der Länge $l$ tragen $2^{nl}$ verschiedene Zustände bei,
und jedes andere Zustandsbit verdoppelt die Zustandsmenge nochmals.
Trotzdem ist die Zustandsmenge endlich, und es braucht nicht viel
Fantasie, sich den Prozessor mit seinem Hauptspeicher als Turingmaschine
vorzustellen. Es gibt also kaum Zweifel, dass die Computer-Hardware
zu all dem fähig ist, was ihr in den letzten zwei Kapiteln an
Fähigkeiten zugesprochen wurde.

Die einzige Einschränkung der Fähigkeiten realer Computer gegenüber
Turing-Maschinen ist
die Tatsache, dass reale Computer nur über einen endlichen Speicher
verfügen, während eine Turing-Maschine ein undendlich langes Band
als Speicher verwenden kann. Da jedoch eine endliche Berechnung auch
nur endlich viel Speicher verwenden kann, sind alle auf einer Turing-Maschine
durchführbaren Berechnungen, die man auch tatsächlich durchführen
will, auch von einem realen Computer durchführbar. Für praktische
Zwecke darf man also annehmen, dass die realen Computer echte Turing-Maschinen
sind.

Trotzdem ist nicht sicher, ob die Programmierung in einer übersetzten
oder interpretierten Sprache alle diese Fähigkeiten auch einem
Anwendungsprogrammierer zugänglich macht.
Letztlich äussert sich dies auch darin, dass Computernutzer
für verschiedene Problemstellung auch verschiedene Werkzeuge
verwenden. Wer tabellarische Daten summieren will, wird gerne
zu einer Spreadsheet-Software greifen, aber nicht erwarten, dass er
damit auch einen Näherungsalgorithmus für das Cliquen-Problem wird
programmieren können. Die Tabellenkalkulation definiert ein eingeschränktes,
an das Problem angepasstes Berechnungsmodell, welches aber
höchstwahrscheinlich weniger leistungsfähig ist als die Hardware, auf
der es läuft. Es ist also durchaus möglich und je nach Anwendung auch
zweckmässig, dass ein Anwender nicht die volle Leistung einer
Turing-Maschine zur Verfügung hat.

Damit stellt sich jetzt die Frage, wie man einem Berechnungsmodell und
das heisst letztlich der Sprache, in der der Berechnungsauftrag
formuliert wird, ansehen kann, ob sie gleich mächtig ist wie eine
Turing-Maschine.

\subsection{Programmiersprachen}
Eine Programmiersprache ist zwar eine Sprache im Sinne dieses Skriptes,
für den Programmierer wesentlich ist jedoch die Semantik, die bisher
nicht Bestandteil der Diskussion war. Für ihn ist die Tatsache wichtig,
dass die Semantik der Sprache Berechnungen beschreibt,
wie sie mit einer Turing-Maschine ausgeführt werden können.

\begin{definition}
\index{Programmiersprache}%
Eine Sprache $A$ heisst eine {\em Programmiersprache}, wenn es eine Abbildung
\[
c\colon A\to \Sigma^*\colon w\mapsto c(w)
\]
gibt, die einem Wort der Sprache die Beschreibung einer Turing-Maschine
zuordnet. Die Abbildung $c$ heisst {\em Compiler} für die Sprache $A$.
\end{definition}
Die Forderung, dass $c(w)$ die Beschreibung einer Turing-Maschine
sein muss, ist nach obiger Diskussion nicht wesentlich.

\subsection{Interaktion}
Man beachte, dass in dieser Definition einer Programmiersprache kein Platz ist
für Input oder Output während des Programmlaufes.
Das Band der Turing-Maschine, bzw.~sein Inhalt bildet den Input, der Output
kann nach Ende der Berechnung vom Band gelesen werden.
Man könnte dies als Mangel dieses Modells ansehen, in der Tat ist aber
keine Erweiterung nötig, um Interaktion abzubilden.
Interaktionen mit einem Benutzer bestehen immer aus einem Strom von
Ereignissen, die dem Benutzer zufliessen (Änderungen des Bildschirminhaltes,
Signaltöne) oder die der Benutzer veranlasst (Bewegungen des Maus-Zeigers,
Maus-Klicks, Tastatureingaben). Alle diese Ereignisse kann man sich codiert
auf ein Band geschrieben denken, welches die Turing-Maschine bei Bedarf
liest.

Der Inhalt des Bandes einer Standard-Turing-Maschine kann während des
Programmlaufes nur von der Turing-Maschine selbst verändert werden.
Da sich die Turing-Maschine aber nicht daran erinnern kann, was beim
letzten Besuch eines Feldes dort stand, ist es für eine Turing-Maschine
auch durchaus akzeptabel, wenn der Inhalt eines Feldes von aussen
geändert wird. Natürlich werden damit das Laufzeitverhalten der
Turing-Maschine verändert. Doch in Anbetracht der
Tatsache, dass von einer Turing-Maschine im Allgemeinen nicht einmal
entschieden werden kann, ob sie anhalten wird, ist wohl nicht mehr
viel zu verlieren.

Die Ausgaben eines Programmes sind deterministisch, und was der Benutzer
erreichen will, sowie die Ereignisse, die er einspeisen wird, sind es ebenfalls.
Man kann also im Prinzip im Voraus wissen, was ausgegeben werden wird
und welche Ereignisse ein Benutzer auslösen wird. Schreibt man diese
vorgängig auf das Band, so wie man es auch beim automatisierten Testen
eines Userinterfaces tut, entsteht aus dem interaktiven Programm eines,
welches ohne Zutun des Benutzers zur Laufzeit funktionieren kann.

\subsection{Die universelle Turing-Maschine}
\index{Turing, Alan}%
\index{Turing-Maschine!universelle}%
In seinem Paper von 1936 hat Alan Turing gezeigt, dass man eine
Turing-Maschine definieren kann,
der man die Beschreibung
$\langle M,w\rangle$
einer Turing-Maschine $M$ und eines Wortes $w$
und die $M$ auf dem Input-Wort $w$ simuliert.
Diese spezielle Turing-Maschine ist also leistungsfähig genug, jede
beliebige andere Turing-Maschine zu simulieren. Sie heisst die {\em universelle
Turing-Maschine}.

Die universelle Turing-Maschine kann die Entscheidung vereinfachen,
ob eine Funktion Turing-berechenbar ist. Statt eine Turing-Maschine
zu beschreiben, die die Funktion berechnet, reicht es, ein Programm
in der Programmiersprache $A$ zu beschreiben, das Programm mit dem
Compiler $c$ zu übersetzen, und die Beschreibung mit der universellen
Turing-Maschine auszuführen.

\index{Church-Turing-Hypothese}%
Die Church-Turing-Hypothese besagt, dass sich alles, was man berechnen
kann, auch mit einer Turing-Maschine berechnen lässt. Die universelle
Turing-Maschine zeigt, dass jede berechenbare Funktion von der
universellen Turing-Maschine berechnet werden kann.
Etwas leistungsfähigeres als eine Turing-Maschine gibt es nicht.

\subsection{Turing-Vollständigkeit}
Jede Funktion, die in der Programmiersprache $A$ implementiert werden
kann, ist Turing-berechenbar.
Der Compiler kann
aber durchaus Fähigkeiten unzugänglich machen, die Programmiersprache
$A$ kann dann gewisse Berechnungen, die mit einer Turing-Maschine
möglich wären, nicht formulieren. Besonderes interessant sind daher
die Sprachen, bei denen ein solcher Verlust nicht eintritt.

\begin{definition}
\index{Turing-vollständig}%
Eine Programmiersprache heisst Turing-vollständig, wenn sich jede
berechenbare Abbildung in dieser Sprache formulieren lässt.
Zu jeder berechenbaren Abbildung $f\colon\Sigma^*\to \Sigma^*$ gibt
es also ein Programm $w$ so, dass $c(w)$ die Funktion $f$ berechnet.
\end{definition}

Zu einer berechenbaren Abbildung gibt es eine Turing-Maschine, die
sie berechnet, es würde also genügen, wenn man diese Turing-Maschine
von einem in der Sprache $A$ geschriebenen Turing-Maschinen-Simulator
ausführen lassen könnte. Dieser Begriff muss noch etwas klarer gefasst
werden:

\begin{definition}
\index{Turing-Maschinen-Simulator}%
Ein Turing-Maschinen-Simulator ist eine Turing-Maschine $S$, die als Input
die Beschreibung $\langle M,w\rangle$ einer Turing-Maschine $M$ und eines
Input-Wortes für $M$ erhält, und die Berechnung durchführt, die $M$ auf $w$
ausführen würde.
Ein Turing-Maschinen-Simulator in der Programmiersprache $A$ ist
ein Wort $s\in A$ so, dass $c(s)$ ein Turing-Maschinen-Simulator ist.
\end{definition}

Damit erhalten wir ein Kriterium für Turing-Vollständigkeit:

\begin{satz}
\label{turingvollstaendigkeitskriterium}
Eine Programmiersprache $A$ ist Turing-vollständig, genau dann
wenn es einen Turing-Maschinen-Simulator in $A$ gibt.
\end{satz}


\subsection{Beispiele}
Die üblichen Programmiersprachen sind alle Turing-vollständig, denn es
ist eine einfache Programmierübung, eines Turing-Maschinen-Simulator
in einer dieser Sprachen zu schreiben. In einigen Programmiersprachen
ist dies jedoch schwieriger als in anderen.

\subsubsection{Javascript}
\index{Javascript}%
Fabrice Bellard hat 2011 einen PC-Emulator in Javascript geschrieben, der
leistungsfähig genug ist, Linux zu booten. Auf seiner Website
\url{http://bellard.org/jslinux/} kann man den Emulator im eigenen Browser
starten. Das gebootete Linux enthält auch einen C-Compiler. Da C
Turing-vollständig ist, gibt es einen Turing-Maschinen-Simulator in
C, den man auch auf dieses Linux bringen und mit dem C-Compiler
kompilieren kann. Somit gibt es einen Turing-Maschinen-Simulator in
Javascript, Javascript ist Turing-vollständig.

\subsubsection{XSLT}
\index{XSLT}%
XSLT ist eine XML-basierte Sprache, die Transformationen von XML-Dokumenten
zu beschreiben erlaubt. XSLT ist jedoch leistungsfähig, eine Turing-Maschine
zu simulieren. Bob Lyons hat auf seiner Website
\url{http://www.unidex.com/turing/utm.htm} ein XSL-Stylesheet publiziert,
welches einen Simulator implementiert. Als Input verlangt es
ein
XML-Dokument, welches die Beschreibung der Turing-Maschine in einem
zu diesem Zweck definierten XML-Format namens Turing Machine Markup
Language (TMML) enthält. TMML definiert XML-Elemente, die das Alphabet
(\verb+<symbols>+),
die Zustandsmenge $Q$ (\verb+<state>...</state>+)
und die Übergangsfunktion $\delta$ in \verb+<mapping>+-Elementen
der Form
\begin{verbatim}
<mapping>
    <from current-state="moveRight1" current-symbol=" " />
    <to next-state="check1" next-symbol=" " movement="left" />
</mapping>
\end{verbatim}
beschreiben. Der initiale Bandinhalt wird als Parameter \verb+tape+
auf der Kommandozeile übergeben.
Das Stylesheet wandelt das TMML Dokument in eine ausführliche
Berechnungsgeschichte um, aus der auch der Bandinhalt am Ende der Berechnung
abzulesen ist. Es beweist somit, dass XSLT einen Turing-Maschinen-Simulator
hat, also Turing-vollständdig ist.

\subsubsection{\LaTeX}
\index{LaTeX@\LaTeX}%
\index{Knuth, Don}%
Don Knuth, der Autor von \TeX, hat sich lange davor gedrückt, seiner
Schriftsatz-Sprache auch eine Turing-vollständige Programmiersprache
zu spendieren. Schliesslich kam er nicht mehr darum herum, und wurde
von Guy Steeles richtigegehend dazu gedrängt, wie er in
\url{http://maps.aanhet.net/maps/pdf/16\_15.pdf}
gesteht.

Dass \TeX Turing-vollständig ist beweist ein Satz von \LaTeX-Macros, den
man auf
\url{https://www.informatik.uni-augsburg.de/en/chairs/swt/ti/staff/mark/projects/turingtex/}
finden kann.
Um ihn zu verwenden, formuliert man die Beschreibung
von Turing-Maschine und initialem Bandinhalt als eine Menge von
\LaTeX-Makros. Ebenso ruft man den Makro \verb+\RunTuringMachine+ auf,
der die Turing-Machine simuliert und die Berechnungsgeschichte im
\TeX-üblichen perfekten Schriftsatz ausgibt.




%
% kontrollstrukturen.tex
%
% (c) 2011 Prof Dr Andreas Mueller, Hochschule Rapperswil
%
\section{Kontrollstrukturen und Turing-Vollständigkeit}
\rhead{Kontrollstrukturen}
Das Turing-Vollstandigkeits-Kriterium von Satz
\ref{turingvollstaendigkeitskriterium} verlangt, dass man einen
Turing-Maschinen-Simulator in der gewählten Sprache schreiben muss.
Dies ist in jedem Fall eine nicht triviale Aufgabe.
Daher wäre es nützlich Kriterien zu erhalten, welche einfacher
anzuwenden sind. Einen wesentlichen Einfluss auf die Möglichkeiten,
was sich mit einer Programmiersprache ausdrücken lassen, haben die
Kontrollstrukturen.

\newcommand{\assignment}{\mathbin{\texttt{:=}}}

\subsection{Grundlegende Syntaxelemente%
\label{subsection:grundlegende-syntaxelement}}
In den folgenden Abschnitten werden verschiedene vereinfachte Sprachen
diskutiert, die aber einen gemeinsamen Kern fundamentaler Anweisungen
beinahlten.
In allen Sprachen gibt es nur einen einzigen Datentyp, nämlich
natürliche Zahlen.
Einerseits können Konstanten beliebig grosse natürliche Werte haben,
andererseits lassen sich in Variablen beliebig grosse natürliche Zahlen
speichern.
Dazu sind die folgenden Sytanxelemente notwendig:
\begin{compactitem}
\item Konstanten: {\tt 0}, {\tt 1}, {\tt 2}, \dots
\item Variablen $x_0$, $x_1$, $x_2$,\dots
\item Zuweisung: $\assignment$
\item Trennung von Anweisungen: {\tt ;}
\item Operatoren: {\tt +} und {\tt -}
\end{compactitem}
Die einzige Möglichkeit, den Wert einer Variablen zu ändern ist die
Zuweisung.  Diese ist entweder die Zuweisung einer Konstante als
Wert einer Variable:
\begin{algorithmic}
\STATE $x_i\assignment c$
\end{algorithmic}
oder eine Berechnung mit den beiden vorhandenen Operatoren
\begin{algorithmic}
\STATE $x_i \assignment x_j$ {\tt +} $c$
\STATE $x_i \assignment x_j$ {\tt -} $c$
\end{algorithmic}
Dabei ist das Resultat der Subtraktion als $0$ definiert, wenn
der Minuend kleiner ist als der Subtrahend.

Die verschiedenen Sprachen unterschieden sich in den Kontrollstrukturen
mit denen diese Zuweisungsbefehle miteinander verbunden werden können.

\subsection{LOOP}
\index{LOOP}%
Die Programmiersprache
LOOP\footnote{Die in diesem Abschnitt beschriebene
LOOP Sprache darf nicht verwechselt werden mit dem gleichnamigen
Projekt einer objektorientierten parallelen Sprache seit
2001 auf Sourceforge.}
hat als einzige Kontrollstruktur die
Iteration eines Anweisungs-Blocks mit einer festen, innerhalb des
Blocks nicht veränderbaren Anzahl von Durchläufen.

\subsubsection{Syntax}
LOOP verwendet die 
Schlüsselwörter: {\tt LOOP}, {\tt DO}, {\tt END}
Programme werden daraus wie folgt aufgebaut:
\begin{itemize}
\item Das leere Programm $\varepsilon$ ist eine LOOP-Programm,
während der Ausführung tut es nichts.
\item Wertzuweisungen sind LOOP-Programme
\item Sind $P_1$ und $P_2$ LOOP-Programme, dann auch
$P_1{\tt ;}P_2$. Um dieses Programm auszuführen wird zuerst $P_1$
ausgeführt und anschliessend $P_2$.
\item Ist $P$ ein LOOP-Programm, dann ist auch
\begin{algorithmic}
\STATE {\tt LOOP} $x_i$ {\tt DO} $P$ {\tt END}
\end{algorithmic}
ein LOOP-Programm. Die Ausführung wiederholt $P$ so oft, wie der
Wert der Variable $x_i$  zu Beginn angibt.
\end{itemize}
Durch Setzen der Variablen $x_i$ kann einem LOOP-Progamm Input übergeben
werden.
Ein LOOP-Programm definiert also eine Abbildung $\mathbb N^k\to\mathbb N$,
wobei $k$ die Anzahl der Variablen ist, in denen Input zu übergeben ist.

\subsubsection{Beispiel: Summer zweier Variablen}
LOOP kann nur jeweils eine Konstante hinzuaddieren, das genügt aber
bereits, um auch die Summe zweier Variablen zu berechnen. Der folgende
Code berechnet die Summe von $x_1$ und $x_2$ und legt das Resultat in
$x_0$ ab:
\begin{algorithmic}
\STATE $x_0 \assignment x_1$
\STATE {\tt LOOP} $x_2$ {\tt DO}
\STATE{\tt \ \ \ \ }$x_0 \assignment x_0$ {\tt +} $1$
\STATE{\tt END}
\end{algorithmic}

\subsubsection{Beispiel: Produkt zweier Variablen}
\index{Multiplikation}%
Die Sprache LOOP ist offenbar relativ primitiv, trotzdem
ist sie leistungsfähig genug, um die Multiplikation von zwei
Zahlen, die in den Variablen $x_1$ und $x_2$ übergeben werden,
zu berechnen:

\begin{algorithmic}
\STATE $x_0 \assignment 0$
\STATE {\tt LOOP} $x_1$ {\tt DO}
\STATE{\tt \ \ \ \ LOOP} $x_2$ {\tt DO}
\STATE{\tt \ \ \ \ \ \ \ \ }$x_0 \assignment x_0$ {\tt +} $1$
\STATE{\tt \ \ \ \ END}
\STATE{\tt END}
\end{algorithmic}

\subsubsection{IF-Anweisung}
In LOOP fehlt eine IF-Anweisung, auf die die wenigsten Programmiersprachen
verzichten. Es ist jedoch leicht, eine solche in LOOP nachzubilden.
Eine Anweisung
\begin{algorithmic}
\STATE {\tt IF} $x = 0$ {\tt THEN } $P$ {\tt END}
\end{algorithmic}
kann unter Verwendung einer zusätzlichen Variable $y$ in LOOP ausgedrückt
werden:
\begin{algorithmic}[1]
\STATE $y \assignment 1${\tt ;}
\STATE {\tt LOOP }$x${\tt\ DO }$y \assignment 0$ {\tt END;}
\STATE {\tt LOOP }$y${\tt\ DO }$P$ {\tt END;}
\end{algorithmic}
Natürlich wird die {\tt LOOP}-Anweisung in Zeile 2 meistens viel öfter
als nötig ausgeführt, aber es stehen hier ja auch nicht Fragen
der Effizienz sonder der prinzipiellen Machbarkeit zur Diskussion.

\subsubsection{LOOP ist nicht Turing-vollständig}
\begin{satz}
LOOP-Programme terminieren immer.
\end{satz}

\begin{proof}[Beweis]
Der Beweis kann mit Induktion über die Schachtelungstiefe
von {\tt LOOP}-Anweisungen geführt werden. Die LOOP-Programme
ohne {\tt LOOP}-Anweisung, also die Programme mit Schachtelungstiefe
$0$ terminieren offensichtlich immer. Ebenso die LOOP-Programme
mit Schachtelungstiefe $1$, da die Anzahl der Schleifendurchläufe
bereits zu Beginn der Schleife festliegt.

Nehmen wir jetzt an wir wüssten bereits, dass jedes LOOP-Programm mit
Schachtelungstiefe $n$ der {\tt LOOP}-Anweisungen immer terminiert.
Ein LOOP-Programm mit Schachtelungstiefe $n+1$ ist dann eine
Abfolge von Teilprogrammen mit Schachtelungstiefe $n$, die nach
Voraussetzung alle terminieren, und {\tt LOOP}-Anweisungen der
Form
\begin{algorithmic}
\STATE{\tt LOOP }$x_i${\tt\ DO }$P${\tt\ END}
\end{algorithmic}
wobei $P$ ein LOOP-Programm mit Schachtelungstiefe $n$ ist, das also
ebenfalls immer terminiert. $P$ wird genau so oft ausgeführt, wie
$x_i$ zu Beginn angibt. Die Laufzeit von $P$ kann dabei jedesmal
anders sein, $P$ wird aber auf jeden Fall terminieren. Damit ist
gezeigt, dass auch alle LOOP-Programme mit Schachtelungstiefe $n+1$
terminieren.
\end{proof}

\begin{satz}
LOOP ist nicht Turing vollständig.
\end{satz}

\begin{proof}[Beweis]
Es gibt Turing-Maschinen, die nicht terminieren. Gäbe es einen
Turing-Maschinen-Simulator in LOOP, dürfte dieser bei der
Simulation einer solchen Turing-Maschine nicht terminieren, im
Widerspruch zur Tatsache, dass LOOP-Programme immer terminieren.
Also kann es keinen Turing-Maschinen-Simulator in LOOP geben.
\end{proof}

\subsection{WHILE}
\index{WHILE}%
WHILE-Programme können zusätzlich zur {\tt LOOP}-Anweisung
eine {\tt WHILE}-Anweisung der Form
\begin{algorithmic}
\STATE{\tt WHILE }$x_i>0${\tt\ DO }$P${\tt\ END}
\end{algorithmic}
Dadurch ist die {\tt LOOP}-Anweisung nicht mehr unbedingt
notwendig, denn
\begin{algorithmic}
\STATE{\tt LOOP }$x${\tt\ DO }$P${\tt\ END}
\end{algorithmic}
kann durch
\begin{algorithmic}
\STATE$y \assignment x$;
\STATE{\tt WHILE }$y>0${\tt\ DO }$P$; $y \assignment y-1${\tt\ END}
\end{algorithmic}
nachgebildet werden.

\subsection{GOTO}
\index{GOTO}%
GOTO-Programme bestehen aus einer markierten Folge von Anweisungen
\begin{center}
\begin{tabular}{rl}
$M_1$:&$A_1$\\
$M_2$:&$A_2$\\
$M_3$:&$A_3$\\
$\dots$:&$\dots$\\
$M_k$:&$A_k$
\end{tabular}
\end{center}
Zu den bereits bekannten,
Abschnitt~\ref{subsection:grundlegende-syntaxelement} beschriebenen
Zuweisungen
$x_i \assignment c$ 
und
$x_i \assignment x_j \pm c$ 
kommt bei GOTO eine bedingte Sprunganweisung
\begin{center}
\begin{tabular}{rl}
$M_l$:&{\tt IF\ }$x_i=c${\tt\ THEN GOTO\ }$M_j$
\end{tabular}
\end{center}
Natürlich lässt sich damit auch eine unbedingte Sprunganweisung
implementieren:
\begin{center}
\begin{tabular}{rl}
$M_l$:&$x_i \assignment c$;\\
$M_{l+1}$:&{\tt IF\ }$x_i=c${\tt\ THEN GOTO\ }$M_j$
\end{tabular}
\end{center}
In ähnlicher Weise lassen sich auch andere bedingte Anweisungen
konstruieren, zum Beispiel ein
Konstrukt {\tt IF }\dots{\tt\ THEN }\dots{\tt\ ELSE }\dots{\tt\ END}, welches
einen ganzen Anweisungsblock enthalten kann.

\subsection{Äquivalenz von WHILE und GOTO}
Die Verwendung eines Sprungbefehles wie GOTO ist in der modernen
Softwareentwicklung verpönt. Sie führe leichter zu Spaghetti-Code,
der kaum mehr wartbar ist. Gewisse Sprachen verbannen daher
GOTO vollständig aus ihrer Syntax, und propagieren dagegen
die Verwendung von `strukturierten' Kontrollstrukturen wie
WHILE. Die Aufregung ist allerdings unnötig: GOTO und WHILE sind äquivalent.


\begin{satz}
Eine Funktion ist genau dann mit einem GOTO-Programm berechenbar,
wenn sie mit einem WHILE-Programm berechenbar ist.
\end{satz}
\begin{proof}[Beweis]
Man braucht nur zu zeigen, dass man ein GOTO-Programm in ein äquivalentes
WHILE-Programm übersetzen kann, und umgekehrt.

Um eine GOTO-Programm zu übersetzen, verwenden wir eine zusätzliche Variable
$z$, die die Funktion des Programm-Zählers übernimmt.
Aus dem GOTO-Programm machen wir dann folgendes WHILE-Programm
\begin{algorithmic}
\STATE $z \assignment 1$
\STATE{\tt WHILE\ }$z>0${\tt\ DO}
\STATE{\tt IF\ }$z=1${\tt\ THEN\ }$A_1'${\tt\ END};
\STATE{\tt IF\ }$z=2${\tt\ THEN\ }$A_2'${\tt\ END};
\STATE{\tt IF\ }$z=3${\tt\ THEN\ }$A_3'${\tt\ END};
\STATE\dots
\STATE{\tt IF\ }$z=k${\tt\ THEN\ }$A_k'${\tt\ END};
\STATE{\tt IF\ }$z=k+1${\tt\ THEN\ }$z \assignment 0${\tt\ END};
\STATE{\tt END}
\end{algorithmic}
Die Anweisung $A_i'$ entsteht aus der Anweisung $A_i$ nach folgenden
Regeln
\begin{itemize}
\item Falls $A_i$ eine Zuweisung ist, wird ihr eine weitere Zuweisung
\begin{algorithmic}
\STATE $z \assignment z+1$
\end{algorithmic}
angehängt.
Dies hat zur Folge, dass nach $A_z'$ als nächste Anweisung
$A_{z+1}$ ausgeführt wird.
\item
Falls $A_i$ eine bedingte Sprunganweisung
\begin{center}
\begin{tabular}{rl}
$M_l$:&{\tt IF\ }$x_i=c${\tt\ THEN GOTO\ }$M_j$
\end{tabular}
\end{center}
ist, wird $A_i'$
\begin{algorithmic}
\STATE{\tt IF\ }$x_i=c${\tt\ THEN\ }$z \assignment j${\tt\ ELSE }$z=z+1$;
\end{algorithmic}
Dies ist zwar keine WHILE-Anweisung, aber es wurde bereits
früher gezeigt, wie man sie in WHILE übersetzen kann.
\end{itemize}
Damit ist gezeigt, dass ein GOTO-Programm in ein äquivalentes WHILE-Programm
mit genau einer WHILE-Schleife übersetzt werden kann.

Umgekehrt zeigen wir, dass jede WHILE-Schleife mit Hilfe von GOTO
implementiert werden kann. Dazu übersetzt man jede WHILE-Schleife
der Form
\begin{algorithmic}
\STATE{\tt WHILE\ }$x_i>0${\tt\ DO }$P${\tt\ END}
\end{algorithmic}
in ein GOTO-Programm-Segment der Form
\begin{algorithmic}[1]
\STATE{\tt IF\ }$x_i=0${\tt\ THEN GOTO }4
\STATE$P$
\STATE{\tt GOTO\ }1
\STATE
\end{algorithmic}
wobei die Zeilennummern durch geeignete Marken ersetzt werden müssen.
Damit haben wir einen Algorithmus spezifiziert, der WHILE-Programme in
GOTO-Programme übersetzen kann.
\end{proof}

\subsection{Turing-Vollständigkeit von WHILE und GOTO}
Da WHILE und GOTO äquivalent sind, braucht die Turing-Vollständigkeit
nur für eine der Sprachen gezeigt zu werden.
Wir skizzieren, wie man eine Turing-Maschine in ein GOTO-Programm
übersetzen kann. Dies genügt, da man nur die universelle
Turing-Maschine zu übersetzen braucht, um damit jede andere
Turing-Maschine ausführen zu können.

\subsubsection{Alphabet, Zustände und Band}
Die Zeichen des Bandalphabetes werden durch natürliche Zahlen
dargestellt.  Wir nehmen an, dass das Bandalphabet $k$ verschiedene Zeichen
umfasst. Das Leerzeichen $\text{\textvisiblespace}$ wird durch die Zahl $0$
dargestellt.

Auch die Zustände der Turing-Maschine werden durch natürliche Zahlen
dargestellt,
die Variable $s$ dient dazu, den aktuellen Zustand zu
speichern.

Der Inhalt des Bandes kann durch eine einzige Variable $b$ dargestellt
werden. Schreibt man die Zahl im System zur Basis $k$, können die
die Zeichen in den einzelnen Felder des Bandes als die Ziffern
der Zahl $b$ interpretiert werden.

Die Kopfposition wird durch eine Zahl $h$ dargestellt. Befindet sich
der Kopf im Feld mit der Nummer $i$, wird $h$ auf den Wert $k^i$
gesetzt.

\subsubsection{Arithmetik}
Alle Komponenten der Turing-Maschine werden mit natürlichen Zahlen
und arithmetischen Operationen dargestellt.
Zwar beherrscht LOOP nur die Addition oder Subtraktion einer Konstanten,
aber durch wiederholte Addition einer $1$ kann damit jede beliebige
Addition oder Subtraktion implementiert werden.

Ebenso können Multiplikation und Division auf wiederholte Addition
zurückgeführt werden.
Im Folgenden nehmen wir daher an, dass die arithmetischen Operationen
zur Verfügung stehen.

\subsubsection{Lesen eines Feldes}
Um den Inhalt eines Feldes zu lesen, muss die Stelle von $b$ an der
aktuellen Kopfposition ermittelt werden.
Dies kann durch die Rechnung
\begin{equation}
z=b / h \mod k
\label{getchar}
\end{equation}
ermittelt werden, wobei $/$ für eine ganzzahlige Division steht.
Beide Operationen können mit einem GOTO-Programm ermittelt werden.

\subsubsection{Löschen eines Feldes}
Das Feld an der Kopfposition kann wie folgt gelöscht werden.
Zunächst ermittelt man mit (\ref{getchar}) den aktuellen Feldinhalt.
Dann berechnet man
\begin{equation}
b' = b - z\cdot h.
\label{clearchar}
\end{equation}
$b'$ enthält an der Stelle der Kopfposition ein $0$.

\subsubsection{Schreiben eines Feldes}
Soll das Feld an der Kopfposition mit dem Zeichen $x$ überschrieben
werden, wird mit (\ref{clearchar}) zuerst das Feld gelöscht.
Anschliessend wird das Feld durch
\[
b'=b+x\cdot h
\]
neu gesetzt.

\subsubsection{Kopfbewegung}
Die Kopfposition wird durch die Zahl $h$ dargestellt.
Da $h$ immer eine Potenz von $k$ ist, und die Nummer des Feldes der
Exponent ist, brauchen wir nur Operationen, die den Exponenten
ändern, also
\[
h'=h/k\qquad\text{bzw.}\qquad h'=h\cdot k
\]

\subsubsection{Übergangsfunktion}
Die Übergangsfunktion
\[
\delta\colon Q\times \Gamma\to Q\times \Gamma\times\{1, 2\}
\]
ermittelt aus aktuellem Zustand $s$ und
aktuellem Zeichen $z$ den neuen Zustand, das neue Zeichen auf
dem Band und die Kopfbewegung ermittelt. Im Gegensatz zur früheren
Definition verwenden wir jetzt die Zahlen $1$ und $2$ für die
Kopfbewegung L bzw.~R.
Wir schreiben $\delta_i$ für die $i$-te Komponente von $\delta$.
Der folgende GOTO-Pseudocode
beschreibt also ein Programm, welches die Turing-Maschine implementiert
\begin{algorithmic}[1]
\STATE Bestimme das Zeichen $z$ unter der aktuellen Kopfposition $h$
\STATE Lösche das aktuelle Zeichen auf dem Band
\STATE {\tt IF\ }$s=0${\tt\ THEN}
\STATE {\tt \ \ \ \ IF\ }$z=0${\tt\ THEN}
\STATE {\tt \ \ \ \ \ \ \ \ }$s\assignment\delta_1(0,0)$
\STATE {\tt \ \ \ \ \ \ \ \ }$z\assignment\delta_2(0,0)$
\STATE {\tt \ \ \ \ \ \ \ \ }$m\assignment\delta_3(0,0)$
\STATE {\tt \ \ \ \ END}
\STATE {\tt END}
\STATE {\tt IF\ }$s=1${\tt\ THEN}
\STATE {\tt \ \ \ \ }\dots
\STATE {\tt END}
\STATE Zeichen $z$ schreiben
\STATE {\tt IF\ }$m=1${\tt\ THEN }$h\assignment h/k$
\STATE {\tt IF\ }$m=2${\tt\ THEN }$h\assignment h\cdot k$
\STATE \dots
\STATE {\tt GOTO\ }1
\end{algorithmic}
Damit ist gezeigt, dass eine gegebene Turingmaschine in ein
GOTO-Programm übersetzt werden kann. Übersetzt man die universelle
Turing-Maschine, erhält man ein GOTO-Programm, welches jede beliebige
Turing-Maschine simulieren kann. Somit ist GOTO und damit auch WHILE
Turing-vollständig.

\subsection{Esoterische Programmiersprachen}
\index{Programmiersprache!esoterische}%
Zur Illustration der Tatsache, dass eine sehr primitive Sprache
ausreichen kann, um Turing-Vollständigkeit zu erreichen, wurden
verschiedene esoterische Programmiersprachen erfunden.
Ihre Nützlichkeit liegt darin, ein bestimmtest Konzept der Theorie
möglichst klar hervorzuheben, die Verwendbarkeit für irgend einen
praktischen Zweck ist nicht notwendig, und manchmal explizit unerwünscht.

\subsubsection{Brainfuck}
\index{Brainfuck}%
Brainfuck
wurde von Urban Müller 1993 entwickelt mit dem Ziel, dass der
Compiler für diese Sprache möglichst klein sein sollte. In der
Tat ist der kleinste Brainfuck-Compiler für Linux nur 171 Bytes
lang.

Brainfuck basiert auf einem einzelnen Pointer {\tt ptr}, welcher
im Programm inkrementiert oder dekrementiert werden kann.
Jeder Pointer-Wert zeigt auf eine Zelle, deren Inhalt inkrementiert
oder dekrementiert werden kann.
Dies erinnert an die Position des Kopfes einer Turing-Maschine.
Zwei Instruktionen für Eingabe und Ausgabe eines Zeichens
an der Pointer-Position ermöglichen Datenein- und -ausgabe.
Als einzige Kontrollstruktur steht WHILE zur Verfügung. Damit
die Sprache von einem minimalisitischen Compiler kompiliert
werden kann, wird jede Anweisung durch ein einziges Zeichen
dargestellt. Die Befehle sind in der folgenden Tabelle
zusammen mit ihrem C-Äquivalent zusammengstellt:
\begin{center}
\begin{tabular}{|c|l|}
\hline
Brainfuck&C-Äquivalent\\
\hline
{\tt >}&\verb/++ptr;/\\
{\tt <}&\verb/--ptr;/\\
{\tt +}&\verb/++*ptr;/\\
{\tt -}&\verb/--*ptr;/\\
{\tt .}&\verb/putchar(*ptr);/\\
{\tt ,}&\verb/*ptr = getchar();/\\
{\tt [}&\verb/while (*ptr) {/\\
{\tt ]}&\verb/}/\\
\hline
\end{tabular}
\end{center}
Mit Hilfe des Pointers lassen sich offenbar beliebige Speicherzellen
adressieren, und diese können durch Wiederholung der Befehle {\tt +}
und {\tt -} auch um konstante Werte vergrössert
oder verkleinert werden. Etwas mehr Arbeit erfordert die Zuweisung
eines Wertes zu einer Variablen. Ist dies jedoch geschafft, kann
man WHILE in Brainfuck übersetzen, und hat damit gezeigt, dass
Brainfuck Turing-vollständig ist.

\subsubsection{Ook}
\index{Ook}%
Die Sprache Ook verwendet als syntaktische Element das Wort {\tt Ook} gefolgt
von `{\tt .}', `{\tt !}' oder `{\tt ?}'. Wer beim Lesen eines Ook-Programmes
den Eindruck hat, zum Affen gemacht zu werden, liegt nicht ganz falsch:
Ook ist eine einfache Umcodierung von Brainfuck:
\begin{center}
\begin{tabular}{|c|c|}
\hline
Ook&Brainfuck\\
\hline
{\tt Ook. Ook?}&{\tt >}\\
{\tt Ook? Ook.}&{\tt <}\\
{\tt Ook. Ook.}&{\tt +}\\
{\tt Ook! Ook!}&{\tt -}\\
{\tt Ook! Ook.}&{\tt .}\\
{\tt Ook. Ook!}&{\tt ,}\\
{\tt Ook! Ook?}&{\tt [}\\
{\tt Ook? Ook!}&{\tt ]}\\
\hline
\end{tabular}
\end{center}
Da Brainfuck Turing-vollständig ist, ist auch Ook Turing-vollständig.



%
% Turing-Vollstaendigkeit
%
% (c) 2011 Prof Dr Andreas Mueller, Hochschule Rapperswil
% $Id$
%
\chapter{Turing-Vollständigkeit\label{chapter-vollstaendigkeit}}
\lhead{Vollständigkeit}
%
% vollstaendig.tex -- Turing-Vollständigkeit
%
% (c) 2011 Prof Dr Andreas Mueller, Hochschule Rapperswil
%

\section{Turing-vollständige Programmiersprachen}
\rhead{Turing-vollständige Programmiersprachen}
Die Turing-Maschine liefert einen wohldefinierten Begriff der
Berechenbarkeit, der auch robust gegenüber milden Änderungen
der Definition einer Turing-Maschine ist.
Der Aufbau aus einem endlichen Automaten mit zusätzlichem
Speicher und der einfache Kalkül mit Konfigurationen hat
sie ausserdem Beweisen vieler wichtiger Eigenschaften zugänglich
gemacht. Die vorangegangenen Kapitel über Entscheidbarkeit und
Komplexität legen davon eindrücklich Zeugnis ab. Am direkten
Nutzen dieser Theorie kann jedoch immer noch ein gewisser Zweifel
bestehen, da ein moderner Entwickler seine Programme ja nicht
direkt für eine Turing-Maschine schreibt, sondern nur mittelbar,
da er eine Programmiersprache verwendet, deren Code anschliessend
von einem Compiler oder Interpreter übersetzt und von einer realen
Maschine ausgeführt wird.

Der Aufbau der realen Maschine ist sehr
nahe an einer Turing-Maschine, ein Prozessor liest und schreibt
jeweils einzelne
Speicherzellen eines mindestens für praktische Zwecke unendlich
grossen Speichers und ändert bei Verarbeitung der gelesenen
Inhalte seinen eigenen Zustand. Natürlich ist die Menge der
Zustände eines modernen Prozessors sehr gross, nur schon die $n$
Register der Länge $l$ tragen $2^{nl}$ verschiedene Zustände bei,
und jedes andere Zustandsbit verdoppelt die Zustandsmenge nochmals.
Trotzdem ist die Zustandsmenge endlich, und es braucht nicht viel
Fantasie, sich den Prozessor mit seinem Hauptspeicher als Turingmaschine
vorzustellen. Es gibt also kaum Zweifel, dass die Computer-Hardware
zu all dem fähig ist, was ihr in den letzten zwei Kapiteln an
Fähigkeiten zugesprochen wurde.

Die einzige Einschränkung der Fähigkeiten realer Computer gegenüber
Turing-Maschinen ist
die Tatsache, dass reale Computer nur über einen endlichen Speicher
verfügen, während eine Turing-Maschine ein undendlich langes Band
als Speicher verwenden kann. Da jedoch eine endliche Berechnung auch
nur endlich viel Speicher verwenden kann, sind alle auf einer Turing-Maschine
durchführbaren Berechnungen, die man auch tatsächlich durchführen
will, auch von einem realen Computer durchführbar. Für praktische
Zwecke darf man also annehmen, dass die realen Computer echte Turing-Maschinen
sind.

Trotzdem ist nicht sicher, ob die Programmierung in einer übersetzten
oder interpretierten Sprache alle diese Fähigkeiten auch einem
Anwendungsprogrammierer zugänglich macht.
Letztlich äussert sich dies auch darin, dass Computernutzer
für verschiedene Problemstellung auch verschiedene Werkzeuge
verwenden. Wer tabellarische Daten summieren will, wird gerne
zu einer Spreadsheet-Software greifen, aber nicht erwarten, dass er
damit auch einen Näherungsalgorithmus für das Cliquen-Problem wird
programmieren können. Die Tabellenkalkulation definiert ein eingeschränktes,
an das Problem angepasstes Berechnungsmodell, welches aber
höchstwahrscheinlich weniger leistungsfähig ist als die Hardware, auf
der es läuft. Es ist also durchaus möglich und je nach Anwendung auch
zweckmässig, dass ein Anwender nicht die volle Leistung einer
Turing-Maschine zur Verfügung hat.

Damit stellt sich jetzt die Frage, wie man einem Berechnungsmodell und
das heisst letztlich der Sprache, in der der Berechnungsauftrag
formuliert wird, ansehen kann, ob sie gleich mächtig ist wie eine
Turing-Maschine.

\subsection{Programmiersprachen}
Eine Programmiersprache ist zwar eine Sprache im Sinne dieses Skriptes,
für den Programmierer wesentlich ist jedoch die Semantik, die bisher
nicht Bestandteil der Diskussion war. Für ihn ist die Tatsache wichtig,
dass die Semantik der Sprache Berechnungen beschreibt,
wie sie mit einer Turing-Maschine ausgeführt werden können.

\begin{definition}
\index{Programmiersprache}%
Eine Sprache $A$ heisst eine {\em Programmiersprache}, wenn es eine Abbildung
\[
c\colon A\to \Sigma^*\colon w\mapsto c(w)
\]
gibt, die einem Wort der Sprache die Beschreibung einer Turing-Maschine
zuordnet. Die Abbildung $c$ heisst {\em Compiler} für die Sprache $A$.
\end{definition}
Die Forderung, dass $c(w)$ die Beschreibung einer Turing-Maschine
sein muss, ist nach obiger Diskussion nicht wesentlich.

\subsection{Interaktion}
Man beachte, dass in dieser Definition einer Programmiersprache kein Platz ist
für Input oder Output während des Programmlaufes.
Das Band der Turing-Maschine, bzw.~sein Inhalt bildet den Input, der Output
kann nach Ende der Berechnung vom Band gelesen werden.
Man könnte dies als Mangel dieses Modells ansehen, in der Tat ist aber
keine Erweiterung nötig, um Interaktion abzubilden.
Interaktionen mit einem Benutzer bestehen immer aus einem Strom von
Ereignissen, die dem Benutzer zufliessen (Änderungen des Bildschirminhaltes,
Signaltöne) oder die der Benutzer veranlasst (Bewegungen des Maus-Zeigers,
Maus-Klicks, Tastatureingaben). Alle diese Ereignisse kann man sich codiert
auf ein Band geschrieben denken, welches die Turing-Maschine bei Bedarf
liest.

Der Inhalt des Bandes einer Standard-Turing-Maschine kann während des
Programmlaufes nur von der Turing-Maschine selbst verändert werden.
Da sich die Turing-Maschine aber nicht daran erinnern kann, was beim
letzten Besuch eines Feldes dort stand, ist es für eine Turing-Maschine
auch durchaus akzeptabel, wenn der Inhalt eines Feldes von aussen
geändert wird. Natürlich werden damit das Laufzeitverhalten der
Turing-Maschine verändert. Doch in Anbetracht der
Tatsache, dass von einer Turing-Maschine im Allgemeinen nicht einmal
entschieden werden kann, ob sie anhalten wird, ist wohl nicht mehr
viel zu verlieren.

Die Ausgaben eines Programmes sind deterministisch, und was der Benutzer
erreichen will, sowie die Ereignisse, die er einspeisen wird, sind es ebenfalls.
Man kann also im Prinzip im Voraus wissen, was ausgegeben werden wird
und welche Ereignisse ein Benutzer auslösen wird. Schreibt man diese
vorgängig auf das Band, so wie man es auch beim automatisierten Testen
eines Userinterfaces tut, entsteht aus dem interaktiven Programm eines,
welches ohne Zutun des Benutzers zur Laufzeit funktionieren kann.

\subsection{Die universelle Turing-Maschine}
\index{Turing, Alan}%
\index{Turing-Maschine!universelle}%
In seinem Paper von 1936 hat Alan Turing gezeigt, dass man eine
Turing-Maschine definieren kann,
der man die Beschreibung
$\langle M,w\rangle$
einer Turing-Maschine $M$ und eines Wortes $w$
und die $M$ auf dem Input-Wort $w$ simuliert.
Diese spezielle Turing-Maschine ist also leistungsfähig genug, jede
beliebige andere Turing-Maschine zu simulieren. Sie heisst die {\em universelle
Turing-Maschine}.

Die universelle Turing-Maschine kann die Entscheidung vereinfachen,
ob eine Funktion Turing-berechenbar ist. Statt eine Turing-Maschine
zu beschreiben, die die Funktion berechnet, reicht es, ein Programm
in der Programmiersprache $A$ zu beschreiben, das Programm mit dem
Compiler $c$ zu übersetzen, und die Beschreibung mit der universellen
Turing-Maschine auszuführen.

\index{Church-Turing-Hypothese}%
Die Church-Turing-Hypothese besagt, dass sich alles, was man berechnen
kann, auch mit einer Turing-Maschine berechnen lässt. Die universelle
Turing-Maschine zeigt, dass jede berechenbare Funktion von der
universellen Turing-Maschine berechnet werden kann.
Etwas leistungsfähigeres als eine Turing-Maschine gibt es nicht.

\subsection{Turing-Vollständigkeit}
Jede Funktion, die in der Programmiersprache $A$ implementiert werden
kann, ist Turing-berechenbar.
Der Compiler kann
aber durchaus Fähigkeiten unzugänglich machen, die Programmiersprache
$A$ kann dann gewisse Berechnungen, die mit einer Turing-Maschine
möglich wären, nicht formulieren. Besonderes interessant sind daher
die Sprachen, bei denen ein solcher Verlust nicht eintritt.

\begin{definition}
\index{Turing-vollständig}%
Eine Programmiersprache heisst Turing-vollständig, wenn sich jede
berechenbare Abbildung in dieser Sprache formulieren lässt.
Zu jeder berechenbaren Abbildung $f\colon\Sigma^*\to \Sigma^*$ gibt
es also ein Programm $w$ so, dass $c(w)$ die Funktion $f$ berechnet.
\end{definition}

Zu einer berechenbaren Abbildung gibt es eine Turing-Maschine, die
sie berechnet, es würde also genügen, wenn man diese Turing-Maschine
von einem in der Sprache $A$ geschriebenen Turing-Maschinen-Simulator
ausführen lassen könnte. Dieser Begriff muss noch etwas klarer gefasst
werden:

\begin{definition}
\index{Turing-Maschinen-Simulator}%
Ein Turing-Maschinen-Simulator ist eine Turing-Maschine $S$, die als Input
die Beschreibung $\langle M,w\rangle$ einer Turing-Maschine $M$ und eines
Input-Wortes für $M$ erhält, und die Berechnung durchführt, die $M$ auf $w$
ausführen würde.
Ein Turing-Maschinen-Simulator in der Programmiersprache $A$ ist
ein Wort $s\in A$ so, dass $c(s)$ ein Turing-Maschinen-Simulator ist.
\end{definition}

Damit erhalten wir ein Kriterium für Turing-Vollständigkeit:

\begin{satz}
\label{turingvollstaendigkeitskriterium}
Eine Programmiersprache $A$ ist Turing-vollständig, genau dann
wenn es einen Turing-Maschinen-Simulator in $A$ gibt.
\end{satz}


\subsection{Beispiele}
Die üblichen Programmiersprachen sind alle Turing-vollständig, denn es
ist eine einfache Programmierübung, eines Turing-Maschinen-Simulator
in einer dieser Sprachen zu schreiben. In einigen Programmiersprachen
ist dies jedoch schwieriger als in anderen.

\subsubsection{Javascript}
\index{Javascript}%
Fabrice Bellard hat 2011 einen PC-Emulator in Javascript geschrieben, der
leistungsfähig genug ist, Linux zu booten. Auf seiner Website
\url{http://bellard.org/jslinux/} kann man den Emulator im eigenen Browser
starten. Das gebootete Linux enthält auch einen C-Compiler. Da C
Turing-vollständig ist, gibt es einen Turing-Maschinen-Simulator in
C, den man auch auf dieses Linux bringen und mit dem C-Compiler
kompilieren kann. Somit gibt es einen Turing-Maschinen-Simulator in
Javascript, Javascript ist Turing-vollständig.

\subsubsection{XSLT}
\index{XSLT}%
XSLT ist eine XML-basierte Sprache, die Transformationen von XML-Dokumenten
zu beschreiben erlaubt. XSLT ist jedoch leistungsfähig, eine Turing-Maschine
zu simulieren. Bob Lyons hat auf seiner Website
\url{http://www.unidex.com/turing/utm.htm} ein XSL-Stylesheet publiziert,
welches einen Simulator implementiert. Als Input verlangt es
ein
XML-Dokument, welches die Beschreibung der Turing-Maschine in einem
zu diesem Zweck definierten XML-Format namens Turing Machine Markup
Language (TMML) enthält. TMML definiert XML-Elemente, die das Alphabet
(\verb+<symbols>+),
die Zustandsmenge $Q$ (\verb+<state>...</state>+)
und die Übergangsfunktion $\delta$ in \verb+<mapping>+-Elementen
der Form
\begin{verbatim}
<mapping>
    <from current-state="moveRight1" current-symbol=" " />
    <to next-state="check1" next-symbol=" " movement="left" />
</mapping>
\end{verbatim}
beschreiben. Der initiale Bandinhalt wird als Parameter \verb+tape+
auf der Kommandozeile übergeben.
Das Stylesheet wandelt das TMML Dokument in eine ausführliche
Berechnungsgeschichte um, aus der auch der Bandinhalt am Ende der Berechnung
abzulesen ist. Es beweist somit, dass XSLT einen Turing-Maschinen-Simulator
hat, also Turing-vollständdig ist.

\subsubsection{\LaTeX}
\index{LaTeX@\LaTeX}%
\index{Knuth, Don}%
Don Knuth, der Autor von \TeX, hat sich lange davor gedrückt, seiner
Schriftsatz-Sprache auch eine Turing-vollständige Programmiersprache
zu spendieren. Schliesslich kam er nicht mehr darum herum, und wurde
von Guy Steeles richtigegehend dazu gedrängt, wie er in
\url{http://maps.aanhet.net/maps/pdf/16\_15.pdf}
gesteht.

Dass \TeX Turing-vollständig ist beweist ein Satz von \LaTeX-Macros, den
man auf
\url{https://www.informatik.uni-augsburg.de/en/chairs/swt/ti/staff/mark/projects/turingtex/}
finden kann.
Um ihn zu verwenden, formuliert man die Beschreibung
von Turing-Maschine und initialem Bandinhalt als eine Menge von
\LaTeX-Makros. Ebenso ruft man den Makro \verb+\RunTuringMachine+ auf,
der die Turing-Machine simuliert und die Berechnungsgeschichte im
\TeX-üblichen perfekten Schriftsatz ausgibt.




%
% kontrollstrukturen.tex
%
% (c) 2011 Prof Dr Andreas Mueller, Hochschule Rapperswil
%
\section{Kontrollstrukturen und Turing-Vollständigkeit}
\rhead{Kontrollstrukturen}
Das Turing-Vollstandigkeits-Kriterium von Satz
\ref{turingvollstaendigkeitskriterium} verlangt, dass man einen
Turing-Maschinen-Simulator in der gewählten Sprache schreiben muss.
Dies ist in jedem Fall eine nicht triviale Aufgabe.
Daher wäre es nützlich Kriterien zu erhalten, welche einfacher
anzuwenden sind. Einen wesentlichen Einfluss auf die Möglichkeiten,
was sich mit einer Programmiersprache ausdrücken lassen, haben die
Kontrollstrukturen.

\newcommand{\assignment}{\mathbin{\texttt{:=}}}

\subsection{Grundlegende Syntaxelemente%
\label{subsection:grundlegende-syntaxelement}}
In den folgenden Abschnitten werden verschiedene vereinfachte Sprachen
diskutiert, die aber einen gemeinsamen Kern fundamentaler Anweisungen
beinahlten.
In allen Sprachen gibt es nur einen einzigen Datentyp, nämlich
natürliche Zahlen.
Einerseits können Konstanten beliebig grosse natürliche Werte haben,
andererseits lassen sich in Variablen beliebig grosse natürliche Zahlen
speichern.
Dazu sind die folgenden Sytanxelemente notwendig:
\begin{compactitem}
\item Konstanten: {\tt 0}, {\tt 1}, {\tt 2}, \dots
\item Variablen $x_0$, $x_1$, $x_2$,\dots
\item Zuweisung: $\assignment$
\item Trennung von Anweisungen: {\tt ;}
\item Operatoren: {\tt +} und {\tt -}
\end{compactitem}
Die einzige Möglichkeit, den Wert einer Variablen zu ändern ist die
Zuweisung.  Diese ist entweder die Zuweisung einer Konstante als
Wert einer Variable:
\begin{algorithmic}
\STATE $x_i\assignment c$
\end{algorithmic}
oder eine Berechnung mit den beiden vorhandenen Operatoren
\begin{algorithmic}
\STATE $x_i \assignment x_j$ {\tt +} $c$
\STATE $x_i \assignment x_j$ {\tt -} $c$
\end{algorithmic}
Dabei ist das Resultat der Subtraktion als $0$ definiert, wenn
der Minuend kleiner ist als der Subtrahend.

Die verschiedenen Sprachen unterschieden sich in den Kontrollstrukturen
mit denen diese Zuweisungsbefehle miteinander verbunden werden können.

\subsection{LOOP}
\index{LOOP}%
Die Programmiersprache
LOOP\footnote{Die in diesem Abschnitt beschriebene
LOOP Sprache darf nicht verwechselt werden mit dem gleichnamigen
Projekt einer objektorientierten parallelen Sprache seit
2001 auf Sourceforge.}
hat als einzige Kontrollstruktur die
Iteration eines Anweisungs-Blocks mit einer festen, innerhalb des
Blocks nicht veränderbaren Anzahl von Durchläufen.

\subsubsection{Syntax}
LOOP verwendet die 
Schlüsselwörter: {\tt LOOP}, {\tt DO}, {\tt END}
Programme werden daraus wie folgt aufgebaut:
\begin{itemize}
\item Das leere Programm $\varepsilon$ ist eine LOOP-Programm,
während der Ausführung tut es nichts.
\item Wertzuweisungen sind LOOP-Programme
\item Sind $P_1$ und $P_2$ LOOP-Programme, dann auch
$P_1{\tt ;}P_2$. Um dieses Programm auszuführen wird zuerst $P_1$
ausgeführt und anschliessend $P_2$.
\item Ist $P$ ein LOOP-Programm, dann ist auch
\begin{algorithmic}
\STATE {\tt LOOP} $x_i$ {\tt DO} $P$ {\tt END}
\end{algorithmic}
ein LOOP-Programm. Die Ausführung wiederholt $P$ so oft, wie der
Wert der Variable $x_i$  zu Beginn angibt.
\end{itemize}
Durch Setzen der Variablen $x_i$ kann einem LOOP-Progamm Input übergeben
werden.
Ein LOOP-Programm definiert also eine Abbildung $\mathbb N^k\to\mathbb N$,
wobei $k$ die Anzahl der Variablen ist, in denen Input zu übergeben ist.

\subsubsection{Beispiel: Summer zweier Variablen}
LOOP kann nur jeweils eine Konstante hinzuaddieren, das genügt aber
bereits, um auch die Summe zweier Variablen zu berechnen. Der folgende
Code berechnet die Summe von $x_1$ und $x_2$ und legt das Resultat in
$x_0$ ab:
\begin{algorithmic}
\STATE $x_0 \assignment x_1$
\STATE {\tt LOOP} $x_2$ {\tt DO}
\STATE{\tt \ \ \ \ }$x_0 \assignment x_0$ {\tt +} $1$
\STATE{\tt END}
\end{algorithmic}

\subsubsection{Beispiel: Produkt zweier Variablen}
\index{Multiplikation}%
Die Sprache LOOP ist offenbar relativ primitiv, trotzdem
ist sie leistungsfähig genug, um die Multiplikation von zwei
Zahlen, die in den Variablen $x_1$ und $x_2$ übergeben werden,
zu berechnen:

\begin{algorithmic}
\STATE $x_0 \assignment 0$
\STATE {\tt LOOP} $x_1$ {\tt DO}
\STATE{\tt \ \ \ \ LOOP} $x_2$ {\tt DO}
\STATE{\tt \ \ \ \ \ \ \ \ }$x_0 \assignment x_0$ {\tt +} $1$
\STATE{\tt \ \ \ \ END}
\STATE{\tt END}
\end{algorithmic}

\subsubsection{IF-Anweisung}
In LOOP fehlt eine IF-Anweisung, auf die die wenigsten Programmiersprachen
verzichten. Es ist jedoch leicht, eine solche in LOOP nachzubilden.
Eine Anweisung
\begin{algorithmic}
\STATE {\tt IF} $x = 0$ {\tt THEN } $P$ {\tt END}
\end{algorithmic}
kann unter Verwendung einer zusätzlichen Variable $y$ in LOOP ausgedrückt
werden:
\begin{algorithmic}[1]
\STATE $y \assignment 1${\tt ;}
\STATE {\tt LOOP }$x${\tt\ DO }$y \assignment 0$ {\tt END;}
\STATE {\tt LOOP }$y${\tt\ DO }$P$ {\tt END;}
\end{algorithmic}
Natürlich wird die {\tt LOOP}-Anweisung in Zeile 2 meistens viel öfter
als nötig ausgeführt, aber es stehen hier ja auch nicht Fragen
der Effizienz sonder der prinzipiellen Machbarkeit zur Diskussion.

\subsubsection{LOOP ist nicht Turing-vollständig}
\begin{satz}
LOOP-Programme terminieren immer.
\end{satz}

\begin{proof}[Beweis]
Der Beweis kann mit Induktion über die Schachtelungstiefe
von {\tt LOOP}-Anweisungen geführt werden. Die LOOP-Programme
ohne {\tt LOOP}-Anweisung, also die Programme mit Schachtelungstiefe
$0$ terminieren offensichtlich immer. Ebenso die LOOP-Programme
mit Schachtelungstiefe $1$, da die Anzahl der Schleifendurchläufe
bereits zu Beginn der Schleife festliegt.

Nehmen wir jetzt an wir wüssten bereits, dass jedes LOOP-Programm mit
Schachtelungstiefe $n$ der {\tt LOOP}-Anweisungen immer terminiert.
Ein LOOP-Programm mit Schachtelungstiefe $n+1$ ist dann eine
Abfolge von Teilprogrammen mit Schachtelungstiefe $n$, die nach
Voraussetzung alle terminieren, und {\tt LOOP}-Anweisungen der
Form
\begin{algorithmic}
\STATE{\tt LOOP }$x_i${\tt\ DO }$P${\tt\ END}
\end{algorithmic}
wobei $P$ ein LOOP-Programm mit Schachtelungstiefe $n$ ist, das also
ebenfalls immer terminiert. $P$ wird genau so oft ausgeführt, wie
$x_i$ zu Beginn angibt. Die Laufzeit von $P$ kann dabei jedesmal
anders sein, $P$ wird aber auf jeden Fall terminieren. Damit ist
gezeigt, dass auch alle LOOP-Programme mit Schachtelungstiefe $n+1$
terminieren.
\end{proof}

\begin{satz}
LOOP ist nicht Turing vollständig.
\end{satz}

\begin{proof}[Beweis]
Es gibt Turing-Maschinen, die nicht terminieren. Gäbe es einen
Turing-Maschinen-Simulator in LOOP, dürfte dieser bei der
Simulation einer solchen Turing-Maschine nicht terminieren, im
Widerspruch zur Tatsache, dass LOOP-Programme immer terminieren.
Also kann es keinen Turing-Maschinen-Simulator in LOOP geben.
\end{proof}

\subsection{WHILE}
\index{WHILE}%
WHILE-Programme können zusätzlich zur {\tt LOOP}-Anweisung
eine {\tt WHILE}-Anweisung der Form
\begin{algorithmic}
\STATE{\tt WHILE }$x_i>0${\tt\ DO }$P${\tt\ END}
\end{algorithmic}
Dadurch ist die {\tt LOOP}-Anweisung nicht mehr unbedingt
notwendig, denn
\begin{algorithmic}
\STATE{\tt LOOP }$x${\tt\ DO }$P${\tt\ END}
\end{algorithmic}
kann durch
\begin{algorithmic}
\STATE$y \assignment x$;
\STATE{\tt WHILE }$y>0${\tt\ DO }$P$; $y \assignment y-1${\tt\ END}
\end{algorithmic}
nachgebildet werden.

\subsection{GOTO}
\index{GOTO}%
GOTO-Programme bestehen aus einer markierten Folge von Anweisungen
\begin{center}
\begin{tabular}{rl}
$M_1$:&$A_1$\\
$M_2$:&$A_2$\\
$M_3$:&$A_3$\\
$\dots$:&$\dots$\\
$M_k$:&$A_k$
\end{tabular}
\end{center}
Zu den bereits bekannten,
Abschnitt~\ref{subsection:grundlegende-syntaxelement} beschriebenen
Zuweisungen
$x_i \assignment c$ 
und
$x_i \assignment x_j \pm c$ 
kommt bei GOTO eine bedingte Sprunganweisung
\begin{center}
\begin{tabular}{rl}
$M_l$:&{\tt IF\ }$x_i=c${\tt\ THEN GOTO\ }$M_j$
\end{tabular}
\end{center}
Natürlich lässt sich damit auch eine unbedingte Sprunganweisung
implementieren:
\begin{center}
\begin{tabular}{rl}
$M_l$:&$x_i \assignment c$;\\
$M_{l+1}$:&{\tt IF\ }$x_i=c${\tt\ THEN GOTO\ }$M_j$
\end{tabular}
\end{center}
In ähnlicher Weise lassen sich auch andere bedingte Anweisungen
konstruieren, zum Beispiel ein
Konstrukt {\tt IF }\dots{\tt\ THEN }\dots{\tt\ ELSE }\dots{\tt\ END}, welches
einen ganzen Anweisungsblock enthalten kann.

\subsection{Äquivalenz von WHILE und GOTO}
Die Verwendung eines Sprungbefehles wie GOTO ist in der modernen
Softwareentwicklung verpönt. Sie führe leichter zu Spaghetti-Code,
der kaum mehr wartbar ist. Gewisse Sprachen verbannen daher
GOTO vollständig aus ihrer Syntax, und propagieren dagegen
die Verwendung von `strukturierten' Kontrollstrukturen wie
WHILE. Die Aufregung ist allerdings unnötig: GOTO und WHILE sind äquivalent.


\begin{satz}
Eine Funktion ist genau dann mit einem GOTO-Programm berechenbar,
wenn sie mit einem WHILE-Programm berechenbar ist.
\end{satz}
\begin{proof}[Beweis]
Man braucht nur zu zeigen, dass man ein GOTO-Programm in ein äquivalentes
WHILE-Programm übersetzen kann, und umgekehrt.

Um eine GOTO-Programm zu übersetzen, verwenden wir eine zusätzliche Variable
$z$, die die Funktion des Programm-Zählers übernimmt.
Aus dem GOTO-Programm machen wir dann folgendes WHILE-Programm
\begin{algorithmic}
\STATE $z \assignment 1$
\STATE{\tt WHILE\ }$z>0${\tt\ DO}
\STATE{\tt IF\ }$z=1${\tt\ THEN\ }$A_1'${\tt\ END};
\STATE{\tt IF\ }$z=2${\tt\ THEN\ }$A_2'${\tt\ END};
\STATE{\tt IF\ }$z=3${\tt\ THEN\ }$A_3'${\tt\ END};
\STATE\dots
\STATE{\tt IF\ }$z=k${\tt\ THEN\ }$A_k'${\tt\ END};
\STATE{\tt IF\ }$z=k+1${\tt\ THEN\ }$z \assignment 0${\tt\ END};
\STATE{\tt END}
\end{algorithmic}
Die Anweisung $A_i'$ entsteht aus der Anweisung $A_i$ nach folgenden
Regeln
\begin{itemize}
\item Falls $A_i$ eine Zuweisung ist, wird ihr eine weitere Zuweisung
\begin{algorithmic}
\STATE $z \assignment z+1$
\end{algorithmic}
angehängt.
Dies hat zur Folge, dass nach $A_z'$ als nächste Anweisung
$A_{z+1}$ ausgeführt wird.
\item
Falls $A_i$ eine bedingte Sprunganweisung
\begin{center}
\begin{tabular}{rl}
$M_l$:&{\tt IF\ }$x_i=c${\tt\ THEN GOTO\ }$M_j$
\end{tabular}
\end{center}
ist, wird $A_i'$
\begin{algorithmic}
\STATE{\tt IF\ }$x_i=c${\tt\ THEN\ }$z \assignment j${\tt\ ELSE }$z=z+1$;
\end{algorithmic}
Dies ist zwar keine WHILE-Anweisung, aber es wurde bereits
früher gezeigt, wie man sie in WHILE übersetzen kann.
\end{itemize}
Damit ist gezeigt, dass ein GOTO-Programm in ein äquivalentes WHILE-Programm
mit genau einer WHILE-Schleife übersetzt werden kann.

Umgekehrt zeigen wir, dass jede WHILE-Schleife mit Hilfe von GOTO
implementiert werden kann. Dazu übersetzt man jede WHILE-Schleife
der Form
\begin{algorithmic}
\STATE{\tt WHILE\ }$x_i>0${\tt\ DO }$P${\tt\ END}
\end{algorithmic}
in ein GOTO-Programm-Segment der Form
\begin{algorithmic}[1]
\STATE{\tt IF\ }$x_i=0${\tt\ THEN GOTO }4
\STATE$P$
\STATE{\tt GOTO\ }1
\STATE
\end{algorithmic}
wobei die Zeilennummern durch geeignete Marken ersetzt werden müssen.
Damit haben wir einen Algorithmus spezifiziert, der WHILE-Programme in
GOTO-Programme übersetzen kann.
\end{proof}

\subsection{Turing-Vollständigkeit von WHILE und GOTO}
Da WHILE und GOTO äquivalent sind, braucht die Turing-Vollständigkeit
nur für eine der Sprachen gezeigt zu werden.
Wir skizzieren, wie man eine Turing-Maschine in ein GOTO-Programm
übersetzen kann. Dies genügt, da man nur die universelle
Turing-Maschine zu übersetzen braucht, um damit jede andere
Turing-Maschine ausführen zu können.

\subsubsection{Alphabet, Zustände und Band}
Die Zeichen des Bandalphabetes werden durch natürliche Zahlen
dargestellt.  Wir nehmen an, dass das Bandalphabet $k$ verschiedene Zeichen
umfasst. Das Leerzeichen $\text{\textvisiblespace}$ wird durch die Zahl $0$
dargestellt.

Auch die Zustände der Turing-Maschine werden durch natürliche Zahlen
dargestellt,
die Variable $s$ dient dazu, den aktuellen Zustand zu
speichern.

Der Inhalt des Bandes kann durch eine einzige Variable $b$ dargestellt
werden. Schreibt man die Zahl im System zur Basis $k$, können die
die Zeichen in den einzelnen Felder des Bandes als die Ziffern
der Zahl $b$ interpretiert werden.

Die Kopfposition wird durch eine Zahl $h$ dargestellt. Befindet sich
der Kopf im Feld mit der Nummer $i$, wird $h$ auf den Wert $k^i$
gesetzt.

\subsubsection{Arithmetik}
Alle Komponenten der Turing-Maschine werden mit natürlichen Zahlen
und arithmetischen Operationen dargestellt.
Zwar beherrscht LOOP nur die Addition oder Subtraktion einer Konstanten,
aber durch wiederholte Addition einer $1$ kann damit jede beliebige
Addition oder Subtraktion implementiert werden.

Ebenso können Multiplikation und Division auf wiederholte Addition
zurückgeführt werden.
Im Folgenden nehmen wir daher an, dass die arithmetischen Operationen
zur Verfügung stehen.

\subsubsection{Lesen eines Feldes}
Um den Inhalt eines Feldes zu lesen, muss die Stelle von $b$ an der
aktuellen Kopfposition ermittelt werden.
Dies kann durch die Rechnung
\begin{equation}
z=b / h \mod k
\label{getchar}
\end{equation}
ermittelt werden, wobei $/$ für eine ganzzahlige Division steht.
Beide Operationen können mit einem GOTO-Programm ermittelt werden.

\subsubsection{Löschen eines Feldes}
Das Feld an der Kopfposition kann wie folgt gelöscht werden.
Zunächst ermittelt man mit (\ref{getchar}) den aktuellen Feldinhalt.
Dann berechnet man
\begin{equation}
b' = b - z\cdot h.
\label{clearchar}
\end{equation}
$b'$ enthält an der Stelle der Kopfposition ein $0$.

\subsubsection{Schreiben eines Feldes}
Soll das Feld an der Kopfposition mit dem Zeichen $x$ überschrieben
werden, wird mit (\ref{clearchar}) zuerst das Feld gelöscht.
Anschliessend wird das Feld durch
\[
b'=b+x\cdot h
\]
neu gesetzt.

\subsubsection{Kopfbewegung}
Die Kopfposition wird durch die Zahl $h$ dargestellt.
Da $h$ immer eine Potenz von $k$ ist, und die Nummer des Feldes der
Exponent ist, brauchen wir nur Operationen, die den Exponenten
ändern, also
\[
h'=h/k\qquad\text{bzw.}\qquad h'=h\cdot k
\]

\subsubsection{Übergangsfunktion}
Die Übergangsfunktion
\[
\delta\colon Q\times \Gamma\to Q\times \Gamma\times\{1, 2\}
\]
ermittelt aus aktuellem Zustand $s$ und
aktuellem Zeichen $z$ den neuen Zustand, das neue Zeichen auf
dem Band und die Kopfbewegung ermittelt. Im Gegensatz zur früheren
Definition verwenden wir jetzt die Zahlen $1$ und $2$ für die
Kopfbewegung L bzw.~R.
Wir schreiben $\delta_i$ für die $i$-te Komponente von $\delta$.
Der folgende GOTO-Pseudocode
beschreibt also ein Programm, welches die Turing-Maschine implementiert
\begin{algorithmic}[1]
\STATE Bestimme das Zeichen $z$ unter der aktuellen Kopfposition $h$
\STATE Lösche das aktuelle Zeichen auf dem Band
\STATE {\tt IF\ }$s=0${\tt\ THEN}
\STATE {\tt \ \ \ \ IF\ }$z=0${\tt\ THEN}
\STATE {\tt \ \ \ \ \ \ \ \ }$s\assignment\delta_1(0,0)$
\STATE {\tt \ \ \ \ \ \ \ \ }$z\assignment\delta_2(0,0)$
\STATE {\tt \ \ \ \ \ \ \ \ }$m\assignment\delta_3(0,0)$
\STATE {\tt \ \ \ \ END}
\STATE {\tt END}
\STATE {\tt IF\ }$s=1${\tt\ THEN}
\STATE {\tt \ \ \ \ }\dots
\STATE {\tt END}
\STATE Zeichen $z$ schreiben
\STATE {\tt IF\ }$m=1${\tt\ THEN }$h\assignment h/k$
\STATE {\tt IF\ }$m=2${\tt\ THEN }$h\assignment h\cdot k$
\STATE \dots
\STATE {\tt GOTO\ }1
\end{algorithmic}
Damit ist gezeigt, dass eine gegebene Turingmaschine in ein
GOTO-Programm übersetzt werden kann. Übersetzt man die universelle
Turing-Maschine, erhält man ein GOTO-Programm, welches jede beliebige
Turing-Maschine simulieren kann. Somit ist GOTO und damit auch WHILE
Turing-vollständig.

\subsection{Esoterische Programmiersprachen}
\index{Programmiersprache!esoterische}%
Zur Illustration der Tatsache, dass eine sehr primitive Sprache
ausreichen kann, um Turing-Vollständigkeit zu erreichen, wurden
verschiedene esoterische Programmiersprachen erfunden.
Ihre Nützlichkeit liegt darin, ein bestimmtest Konzept der Theorie
möglichst klar hervorzuheben, die Verwendbarkeit für irgend einen
praktischen Zweck ist nicht notwendig, und manchmal explizit unerwünscht.

\subsubsection{Brainfuck}
\index{Brainfuck}%
Brainfuck
wurde von Urban Müller 1993 entwickelt mit dem Ziel, dass der
Compiler für diese Sprache möglichst klein sein sollte. In der
Tat ist der kleinste Brainfuck-Compiler für Linux nur 171 Bytes
lang.

Brainfuck basiert auf einem einzelnen Pointer {\tt ptr}, welcher
im Programm inkrementiert oder dekrementiert werden kann.
Jeder Pointer-Wert zeigt auf eine Zelle, deren Inhalt inkrementiert
oder dekrementiert werden kann.
Dies erinnert an die Position des Kopfes einer Turing-Maschine.
Zwei Instruktionen für Eingabe und Ausgabe eines Zeichens
an der Pointer-Position ermöglichen Datenein- und -ausgabe.
Als einzige Kontrollstruktur steht WHILE zur Verfügung. Damit
die Sprache von einem minimalisitischen Compiler kompiliert
werden kann, wird jede Anweisung durch ein einziges Zeichen
dargestellt. Die Befehle sind in der folgenden Tabelle
zusammen mit ihrem C-Äquivalent zusammengstellt:
\begin{center}
\begin{tabular}{|c|l|}
\hline
Brainfuck&C-Äquivalent\\
\hline
{\tt >}&\verb/++ptr;/\\
{\tt <}&\verb/--ptr;/\\
{\tt +}&\verb/++*ptr;/\\
{\tt -}&\verb/--*ptr;/\\
{\tt .}&\verb/putchar(*ptr);/\\
{\tt ,}&\verb/*ptr = getchar();/\\
{\tt [}&\verb/while (*ptr) {/\\
{\tt ]}&\verb/}/\\
\hline
\end{tabular}
\end{center}
Mit Hilfe des Pointers lassen sich offenbar beliebige Speicherzellen
adressieren, und diese können durch Wiederholung der Befehle {\tt +}
und {\tt -} auch um konstante Werte vergrössert
oder verkleinert werden. Etwas mehr Arbeit erfordert die Zuweisung
eines Wertes zu einer Variablen. Ist dies jedoch geschafft, kann
man WHILE in Brainfuck übersetzen, und hat damit gezeigt, dass
Brainfuck Turing-vollständig ist.

\subsubsection{Ook}
\index{Ook}%
Die Sprache Ook verwendet als syntaktische Element das Wort {\tt Ook} gefolgt
von `{\tt .}', `{\tt !}' oder `{\tt ?}'. Wer beim Lesen eines Ook-Programmes
den Eindruck hat, zum Affen gemacht zu werden, liegt nicht ganz falsch:
Ook ist eine einfache Umcodierung von Brainfuck:
\begin{center}
\begin{tabular}{|c|c|}
\hline
Ook&Brainfuck\\
\hline
{\tt Ook. Ook?}&{\tt >}\\
{\tt Ook? Ook.}&{\tt <}\\
{\tt Ook. Ook.}&{\tt +}\\
{\tt Ook! Ook!}&{\tt -}\\
{\tt Ook! Ook.}&{\tt .}\\
{\tt Ook. Ook!}&{\tt ,}\\
{\tt Ook! Ook?}&{\tt [}\\
{\tt Ook? Ook!}&{\tt ]}\\
\hline
\end{tabular}
\end{center}
Da Brainfuck Turing-vollständig ist, ist auch Ook Turing-vollständig.



%
% Turing-Vollstaendigkeit
%
% (c) 2011 Prof Dr Andreas Mueller, Hochschule Rapperswil
% $Id$
%
\chapter{Turing-Vollständigkeit\label{chapter-vollstaendigkeit}}
\lhead{Vollständigkeit}
%
% vollstaendig.tex -- Turing-Vollständigkeit
%
% (c) 2011 Prof Dr Andreas Mueller, Hochschule Rapperswil
%

\section{Turing-vollständige Programmiersprachen}
\rhead{Turing-vollständige Programmiersprachen}
Die Turing-Maschine liefert einen wohldefinierten Begriff der
Berechenbarkeit, der auch robust gegenüber milden Änderungen
der Definition einer Turing-Maschine ist.
Der Aufbau aus einem endlichen Automaten mit zusätzlichem
Speicher und der einfache Kalkül mit Konfigurationen hat
sie ausserdem Beweisen vieler wichtiger Eigenschaften zugänglich
gemacht. Die vorangegangenen Kapitel über Entscheidbarkeit und
Komplexität legen davon eindrücklich Zeugnis ab. Am direkten
Nutzen dieser Theorie kann jedoch immer noch ein gewisser Zweifel
bestehen, da ein moderner Entwickler seine Programme ja nicht
direkt für eine Turing-Maschine schreibt, sondern nur mittelbar,
da er eine Programmiersprache verwendet, deren Code anschliessend
von einem Compiler oder Interpreter übersetzt und von einer realen
Maschine ausgeführt wird.

Der Aufbau der realen Maschine ist sehr
nahe an einer Turing-Maschine, ein Prozessor liest und schreibt
jeweils einzelne
Speicherzellen eines mindestens für praktische Zwecke unendlich
grossen Speichers und ändert bei Verarbeitung der gelesenen
Inhalte seinen eigenen Zustand. Natürlich ist die Menge der
Zustände eines modernen Prozessors sehr gross, nur schon die $n$
Register der Länge $l$ tragen $2^{nl}$ verschiedene Zustände bei,
und jedes andere Zustandsbit verdoppelt die Zustandsmenge nochmals.
Trotzdem ist die Zustandsmenge endlich, und es braucht nicht viel
Fantasie, sich den Prozessor mit seinem Hauptspeicher als Turingmaschine
vorzustellen. Es gibt also kaum Zweifel, dass die Computer-Hardware
zu all dem fähig ist, was ihr in den letzten zwei Kapiteln an
Fähigkeiten zugesprochen wurde.

Die einzige Einschränkung der Fähigkeiten realer Computer gegenüber
Turing-Maschinen ist
die Tatsache, dass reale Computer nur über einen endlichen Speicher
verfügen, während eine Turing-Maschine ein undendlich langes Band
als Speicher verwenden kann. Da jedoch eine endliche Berechnung auch
nur endlich viel Speicher verwenden kann, sind alle auf einer Turing-Maschine
durchführbaren Berechnungen, die man auch tatsächlich durchführen
will, auch von einem realen Computer durchführbar. Für praktische
Zwecke darf man also annehmen, dass die realen Computer echte Turing-Maschinen
sind.

Trotzdem ist nicht sicher, ob die Programmierung in einer übersetzten
oder interpretierten Sprache alle diese Fähigkeiten auch einem
Anwendungsprogrammierer zugänglich macht.
Letztlich äussert sich dies auch darin, dass Computernutzer
für verschiedene Problemstellung auch verschiedene Werkzeuge
verwenden. Wer tabellarische Daten summieren will, wird gerne
zu einer Spreadsheet-Software greifen, aber nicht erwarten, dass er
damit auch einen Näherungsalgorithmus für das Cliquen-Problem wird
programmieren können. Die Tabellenkalkulation definiert ein eingeschränktes,
an das Problem angepasstes Berechnungsmodell, welches aber
höchstwahrscheinlich weniger leistungsfähig ist als die Hardware, auf
der es läuft. Es ist also durchaus möglich und je nach Anwendung auch
zweckmässig, dass ein Anwender nicht die volle Leistung einer
Turing-Maschine zur Verfügung hat.

Damit stellt sich jetzt die Frage, wie man einem Berechnungsmodell und
das heisst letztlich der Sprache, in der der Berechnungsauftrag
formuliert wird, ansehen kann, ob sie gleich mächtig ist wie eine
Turing-Maschine.

\subsection{Programmiersprachen}
Eine Programmiersprache ist zwar eine Sprache im Sinne dieses Skriptes,
für den Programmierer wesentlich ist jedoch die Semantik, die bisher
nicht Bestandteil der Diskussion war. Für ihn ist die Tatsache wichtig,
dass die Semantik der Sprache Berechnungen beschreibt,
wie sie mit einer Turing-Maschine ausgeführt werden können.

\begin{definition}
\index{Programmiersprache}%
Eine Sprache $A$ heisst eine {\em Programmiersprache}, wenn es eine Abbildung
\[
c\colon A\to \Sigma^*\colon w\mapsto c(w)
\]
gibt, die einem Wort der Sprache die Beschreibung einer Turing-Maschine
zuordnet. Die Abbildung $c$ heisst {\em Compiler} für die Sprache $A$.
\end{definition}
Die Forderung, dass $c(w)$ die Beschreibung einer Turing-Maschine
sein muss, ist nach obiger Diskussion nicht wesentlich.

\subsection{Interaktion}
Man beachte, dass in dieser Definition einer Programmiersprache kein Platz ist
für Input oder Output während des Programmlaufes.
Das Band der Turing-Maschine, bzw.~sein Inhalt bildet den Input, der Output
kann nach Ende der Berechnung vom Band gelesen werden.
Man könnte dies als Mangel dieses Modells ansehen, in der Tat ist aber
keine Erweiterung nötig, um Interaktion abzubilden.
Interaktionen mit einem Benutzer bestehen immer aus einem Strom von
Ereignissen, die dem Benutzer zufliessen (Änderungen des Bildschirminhaltes,
Signaltöne) oder die der Benutzer veranlasst (Bewegungen des Maus-Zeigers,
Maus-Klicks, Tastatureingaben). Alle diese Ereignisse kann man sich codiert
auf ein Band geschrieben denken, welches die Turing-Maschine bei Bedarf
liest.

Der Inhalt des Bandes einer Standard-Turing-Maschine kann während des
Programmlaufes nur von der Turing-Maschine selbst verändert werden.
Da sich die Turing-Maschine aber nicht daran erinnern kann, was beim
letzten Besuch eines Feldes dort stand, ist es für eine Turing-Maschine
auch durchaus akzeptabel, wenn der Inhalt eines Feldes von aussen
geändert wird. Natürlich werden damit das Laufzeitverhalten der
Turing-Maschine verändert. Doch in Anbetracht der
Tatsache, dass von einer Turing-Maschine im Allgemeinen nicht einmal
entschieden werden kann, ob sie anhalten wird, ist wohl nicht mehr
viel zu verlieren.

Die Ausgaben eines Programmes sind deterministisch, und was der Benutzer
erreichen will, sowie die Ereignisse, die er einspeisen wird, sind es ebenfalls.
Man kann also im Prinzip im Voraus wissen, was ausgegeben werden wird
und welche Ereignisse ein Benutzer auslösen wird. Schreibt man diese
vorgängig auf das Band, so wie man es auch beim automatisierten Testen
eines Userinterfaces tut, entsteht aus dem interaktiven Programm eines,
welches ohne Zutun des Benutzers zur Laufzeit funktionieren kann.

\subsection{Die universelle Turing-Maschine}
\index{Turing, Alan}%
\index{Turing-Maschine!universelle}%
In seinem Paper von 1936 hat Alan Turing gezeigt, dass man eine
Turing-Maschine definieren kann,
der man die Beschreibung
$\langle M,w\rangle$
einer Turing-Maschine $M$ und eines Wortes $w$
und die $M$ auf dem Input-Wort $w$ simuliert.
Diese spezielle Turing-Maschine ist also leistungsfähig genug, jede
beliebige andere Turing-Maschine zu simulieren. Sie heisst die {\em universelle
Turing-Maschine}.

Die universelle Turing-Maschine kann die Entscheidung vereinfachen,
ob eine Funktion Turing-berechenbar ist. Statt eine Turing-Maschine
zu beschreiben, die die Funktion berechnet, reicht es, ein Programm
in der Programmiersprache $A$ zu beschreiben, das Programm mit dem
Compiler $c$ zu übersetzen, und die Beschreibung mit der universellen
Turing-Maschine auszuführen.

\index{Church-Turing-Hypothese}%
Die Church-Turing-Hypothese besagt, dass sich alles, was man berechnen
kann, auch mit einer Turing-Maschine berechnen lässt. Die universelle
Turing-Maschine zeigt, dass jede berechenbare Funktion von der
universellen Turing-Maschine berechnet werden kann.
Etwas leistungsfähigeres als eine Turing-Maschine gibt es nicht.

\subsection{Turing-Vollständigkeit}
Jede Funktion, die in der Programmiersprache $A$ implementiert werden
kann, ist Turing-berechenbar.
Der Compiler kann
aber durchaus Fähigkeiten unzugänglich machen, die Programmiersprache
$A$ kann dann gewisse Berechnungen, die mit einer Turing-Maschine
möglich wären, nicht formulieren. Besonderes interessant sind daher
die Sprachen, bei denen ein solcher Verlust nicht eintritt.

\begin{definition}
\index{Turing-vollständig}%
Eine Programmiersprache heisst Turing-vollständig, wenn sich jede
berechenbare Abbildung in dieser Sprache formulieren lässt.
Zu jeder berechenbaren Abbildung $f\colon\Sigma^*\to \Sigma^*$ gibt
es also ein Programm $w$ so, dass $c(w)$ die Funktion $f$ berechnet.
\end{definition}

Zu einer berechenbaren Abbildung gibt es eine Turing-Maschine, die
sie berechnet, es würde also genügen, wenn man diese Turing-Maschine
von einem in der Sprache $A$ geschriebenen Turing-Maschinen-Simulator
ausführen lassen könnte. Dieser Begriff muss noch etwas klarer gefasst
werden:

\begin{definition}
\index{Turing-Maschinen-Simulator}%
Ein Turing-Maschinen-Simulator ist eine Turing-Maschine $S$, die als Input
die Beschreibung $\langle M,w\rangle$ einer Turing-Maschine $M$ und eines
Input-Wortes für $M$ erhält, und die Berechnung durchführt, die $M$ auf $w$
ausführen würde.
Ein Turing-Maschinen-Simulator in der Programmiersprache $A$ ist
ein Wort $s\in A$ so, dass $c(s)$ ein Turing-Maschinen-Simulator ist.
\end{definition}

Damit erhalten wir ein Kriterium für Turing-Vollständigkeit:

\begin{satz}
\label{turingvollstaendigkeitskriterium}
Eine Programmiersprache $A$ ist Turing-vollständig, genau dann
wenn es einen Turing-Maschinen-Simulator in $A$ gibt.
\end{satz}


\subsection{Beispiele}
Die üblichen Programmiersprachen sind alle Turing-vollständig, denn es
ist eine einfache Programmierübung, eines Turing-Maschinen-Simulator
in einer dieser Sprachen zu schreiben. In einigen Programmiersprachen
ist dies jedoch schwieriger als in anderen.

\subsubsection{Javascript}
\index{Javascript}%
Fabrice Bellard hat 2011 einen PC-Emulator in Javascript geschrieben, der
leistungsfähig genug ist, Linux zu booten. Auf seiner Website
\url{http://bellard.org/jslinux/} kann man den Emulator im eigenen Browser
starten. Das gebootete Linux enthält auch einen C-Compiler. Da C
Turing-vollständig ist, gibt es einen Turing-Maschinen-Simulator in
C, den man auch auf dieses Linux bringen und mit dem C-Compiler
kompilieren kann. Somit gibt es einen Turing-Maschinen-Simulator in
Javascript, Javascript ist Turing-vollständig.

\subsubsection{XSLT}
\index{XSLT}%
XSLT ist eine XML-basierte Sprache, die Transformationen von XML-Dokumenten
zu beschreiben erlaubt. XSLT ist jedoch leistungsfähig, eine Turing-Maschine
zu simulieren. Bob Lyons hat auf seiner Website
\url{http://www.unidex.com/turing/utm.htm} ein XSL-Stylesheet publiziert,
welches einen Simulator implementiert. Als Input verlangt es
ein
XML-Dokument, welches die Beschreibung der Turing-Maschine in einem
zu diesem Zweck definierten XML-Format namens Turing Machine Markup
Language (TMML) enthält. TMML definiert XML-Elemente, die das Alphabet
(\verb+<symbols>+),
die Zustandsmenge $Q$ (\verb+<state>...</state>+)
und die Übergangsfunktion $\delta$ in \verb+<mapping>+-Elementen
der Form
\begin{verbatim}
<mapping>
    <from current-state="moveRight1" current-symbol=" " />
    <to next-state="check1" next-symbol=" " movement="left" />
</mapping>
\end{verbatim}
beschreiben. Der initiale Bandinhalt wird als Parameter \verb+tape+
auf der Kommandozeile übergeben.
Das Stylesheet wandelt das TMML Dokument in eine ausführliche
Berechnungsgeschichte um, aus der auch der Bandinhalt am Ende der Berechnung
abzulesen ist. Es beweist somit, dass XSLT einen Turing-Maschinen-Simulator
hat, also Turing-vollständdig ist.

\subsubsection{\LaTeX}
\index{LaTeX@\LaTeX}%
\index{Knuth, Don}%
Don Knuth, der Autor von \TeX, hat sich lange davor gedrückt, seiner
Schriftsatz-Sprache auch eine Turing-vollständige Programmiersprache
zu spendieren. Schliesslich kam er nicht mehr darum herum, und wurde
von Guy Steeles richtigegehend dazu gedrängt, wie er in
\url{http://maps.aanhet.net/maps/pdf/16\_15.pdf}
gesteht.

Dass \TeX Turing-vollständig ist beweist ein Satz von \LaTeX-Macros, den
man auf
\url{https://www.informatik.uni-augsburg.de/en/chairs/swt/ti/staff/mark/projects/turingtex/}
finden kann.
Um ihn zu verwenden, formuliert man die Beschreibung
von Turing-Maschine und initialem Bandinhalt als eine Menge von
\LaTeX-Makros. Ebenso ruft man den Makro \verb+\RunTuringMachine+ auf,
der die Turing-Machine simuliert und die Berechnungsgeschichte im
\TeX-üblichen perfekten Schriftsatz ausgibt.




%
% kontrollstrukturen.tex
%
% (c) 2011 Prof Dr Andreas Mueller, Hochschule Rapperswil
%
\section{Kontrollstrukturen und Turing-Vollständigkeit}
\rhead{Kontrollstrukturen}
Das Turing-Vollstandigkeits-Kriterium von Satz
\ref{turingvollstaendigkeitskriterium} verlangt, dass man einen
Turing-Maschinen-Simulator in der gewählten Sprache schreiben muss.
Dies ist in jedem Fall eine nicht triviale Aufgabe.
Daher wäre es nützlich Kriterien zu erhalten, welche einfacher
anzuwenden sind. Einen wesentlichen Einfluss auf die Möglichkeiten,
was sich mit einer Programmiersprache ausdrücken lassen, haben die
Kontrollstrukturen.

\newcommand{\assignment}{\mathbin{\texttt{:=}}}

\subsection{Grundlegende Syntaxelemente%
\label{subsection:grundlegende-syntaxelement}}
In den folgenden Abschnitten werden verschiedene vereinfachte Sprachen
diskutiert, die aber einen gemeinsamen Kern fundamentaler Anweisungen
beinahlten.
In allen Sprachen gibt es nur einen einzigen Datentyp, nämlich
natürliche Zahlen.
Einerseits können Konstanten beliebig grosse natürliche Werte haben,
andererseits lassen sich in Variablen beliebig grosse natürliche Zahlen
speichern.
Dazu sind die folgenden Sytanxelemente notwendig:
\begin{compactitem}
\item Konstanten: {\tt 0}, {\tt 1}, {\tt 2}, \dots
\item Variablen $x_0$, $x_1$, $x_2$,\dots
\item Zuweisung: $\assignment$
\item Trennung von Anweisungen: {\tt ;}
\item Operatoren: {\tt +} und {\tt -}
\end{compactitem}
Die einzige Möglichkeit, den Wert einer Variablen zu ändern ist die
Zuweisung.  Diese ist entweder die Zuweisung einer Konstante als
Wert einer Variable:
\begin{algorithmic}
\STATE $x_i\assignment c$
\end{algorithmic}
oder eine Berechnung mit den beiden vorhandenen Operatoren
\begin{algorithmic}
\STATE $x_i \assignment x_j$ {\tt +} $c$
\STATE $x_i \assignment x_j$ {\tt -} $c$
\end{algorithmic}
Dabei ist das Resultat der Subtraktion als $0$ definiert, wenn
der Minuend kleiner ist als der Subtrahend.

Die verschiedenen Sprachen unterschieden sich in den Kontrollstrukturen
mit denen diese Zuweisungsbefehle miteinander verbunden werden können.

\subsection{LOOP}
\index{LOOP}%
Die Programmiersprache
LOOP\footnote{Die in diesem Abschnitt beschriebene
LOOP Sprache darf nicht verwechselt werden mit dem gleichnamigen
Projekt einer objektorientierten parallelen Sprache seit
2001 auf Sourceforge.}
hat als einzige Kontrollstruktur die
Iteration eines Anweisungs-Blocks mit einer festen, innerhalb des
Blocks nicht veränderbaren Anzahl von Durchläufen.

\subsubsection{Syntax}
LOOP verwendet die 
Schlüsselwörter: {\tt LOOP}, {\tt DO}, {\tt END}
Programme werden daraus wie folgt aufgebaut:
\begin{itemize}
\item Das leere Programm $\varepsilon$ ist eine LOOP-Programm,
während der Ausführung tut es nichts.
\item Wertzuweisungen sind LOOP-Programme
\item Sind $P_1$ und $P_2$ LOOP-Programme, dann auch
$P_1{\tt ;}P_2$. Um dieses Programm auszuführen wird zuerst $P_1$
ausgeführt und anschliessend $P_2$.
\item Ist $P$ ein LOOP-Programm, dann ist auch
\begin{algorithmic}
\STATE {\tt LOOP} $x_i$ {\tt DO} $P$ {\tt END}
\end{algorithmic}
ein LOOP-Programm. Die Ausführung wiederholt $P$ so oft, wie der
Wert der Variable $x_i$  zu Beginn angibt.
\end{itemize}
Durch Setzen der Variablen $x_i$ kann einem LOOP-Progamm Input übergeben
werden.
Ein LOOP-Programm definiert also eine Abbildung $\mathbb N^k\to\mathbb N$,
wobei $k$ die Anzahl der Variablen ist, in denen Input zu übergeben ist.

\subsubsection{Beispiel: Summer zweier Variablen}
LOOP kann nur jeweils eine Konstante hinzuaddieren, das genügt aber
bereits, um auch die Summe zweier Variablen zu berechnen. Der folgende
Code berechnet die Summe von $x_1$ und $x_2$ und legt das Resultat in
$x_0$ ab:
\begin{algorithmic}
\STATE $x_0 \assignment x_1$
\STATE {\tt LOOP} $x_2$ {\tt DO}
\STATE{\tt \ \ \ \ }$x_0 \assignment x_0$ {\tt +} $1$
\STATE{\tt END}
\end{algorithmic}

\subsubsection{Beispiel: Produkt zweier Variablen}
\index{Multiplikation}%
Die Sprache LOOP ist offenbar relativ primitiv, trotzdem
ist sie leistungsfähig genug, um die Multiplikation von zwei
Zahlen, die in den Variablen $x_1$ und $x_2$ übergeben werden,
zu berechnen:

\begin{algorithmic}
\STATE $x_0 \assignment 0$
\STATE {\tt LOOP} $x_1$ {\tt DO}
\STATE{\tt \ \ \ \ LOOP} $x_2$ {\tt DO}
\STATE{\tt \ \ \ \ \ \ \ \ }$x_0 \assignment x_0$ {\tt +} $1$
\STATE{\tt \ \ \ \ END}
\STATE{\tt END}
\end{algorithmic}

\subsubsection{IF-Anweisung}
In LOOP fehlt eine IF-Anweisung, auf die die wenigsten Programmiersprachen
verzichten. Es ist jedoch leicht, eine solche in LOOP nachzubilden.
Eine Anweisung
\begin{algorithmic}
\STATE {\tt IF} $x = 0$ {\tt THEN } $P$ {\tt END}
\end{algorithmic}
kann unter Verwendung einer zusätzlichen Variable $y$ in LOOP ausgedrückt
werden:
\begin{algorithmic}[1]
\STATE $y \assignment 1${\tt ;}
\STATE {\tt LOOP }$x${\tt\ DO }$y \assignment 0$ {\tt END;}
\STATE {\tt LOOP }$y${\tt\ DO }$P$ {\tt END;}
\end{algorithmic}
Natürlich wird die {\tt LOOP}-Anweisung in Zeile 2 meistens viel öfter
als nötig ausgeführt, aber es stehen hier ja auch nicht Fragen
der Effizienz sonder der prinzipiellen Machbarkeit zur Diskussion.

\subsubsection{LOOP ist nicht Turing-vollständig}
\begin{satz}
LOOP-Programme terminieren immer.
\end{satz}

\begin{proof}[Beweis]
Der Beweis kann mit Induktion über die Schachtelungstiefe
von {\tt LOOP}-Anweisungen geführt werden. Die LOOP-Programme
ohne {\tt LOOP}-Anweisung, also die Programme mit Schachtelungstiefe
$0$ terminieren offensichtlich immer. Ebenso die LOOP-Programme
mit Schachtelungstiefe $1$, da die Anzahl der Schleifendurchläufe
bereits zu Beginn der Schleife festliegt.

Nehmen wir jetzt an wir wüssten bereits, dass jedes LOOP-Programm mit
Schachtelungstiefe $n$ der {\tt LOOP}-Anweisungen immer terminiert.
Ein LOOP-Programm mit Schachtelungstiefe $n+1$ ist dann eine
Abfolge von Teilprogrammen mit Schachtelungstiefe $n$, die nach
Voraussetzung alle terminieren, und {\tt LOOP}-Anweisungen der
Form
\begin{algorithmic}
\STATE{\tt LOOP }$x_i${\tt\ DO }$P${\tt\ END}
\end{algorithmic}
wobei $P$ ein LOOP-Programm mit Schachtelungstiefe $n$ ist, das also
ebenfalls immer terminiert. $P$ wird genau so oft ausgeführt, wie
$x_i$ zu Beginn angibt. Die Laufzeit von $P$ kann dabei jedesmal
anders sein, $P$ wird aber auf jeden Fall terminieren. Damit ist
gezeigt, dass auch alle LOOP-Programme mit Schachtelungstiefe $n+1$
terminieren.
\end{proof}

\begin{satz}
LOOP ist nicht Turing vollständig.
\end{satz}

\begin{proof}[Beweis]
Es gibt Turing-Maschinen, die nicht terminieren. Gäbe es einen
Turing-Maschinen-Simulator in LOOP, dürfte dieser bei der
Simulation einer solchen Turing-Maschine nicht terminieren, im
Widerspruch zur Tatsache, dass LOOP-Programme immer terminieren.
Also kann es keinen Turing-Maschinen-Simulator in LOOP geben.
\end{proof}

\subsection{WHILE}
\index{WHILE}%
WHILE-Programme können zusätzlich zur {\tt LOOP}-Anweisung
eine {\tt WHILE}-Anweisung der Form
\begin{algorithmic}
\STATE{\tt WHILE }$x_i>0${\tt\ DO }$P${\tt\ END}
\end{algorithmic}
Dadurch ist die {\tt LOOP}-Anweisung nicht mehr unbedingt
notwendig, denn
\begin{algorithmic}
\STATE{\tt LOOP }$x${\tt\ DO }$P${\tt\ END}
\end{algorithmic}
kann durch
\begin{algorithmic}
\STATE$y \assignment x$;
\STATE{\tt WHILE }$y>0${\tt\ DO }$P$; $y \assignment y-1${\tt\ END}
\end{algorithmic}
nachgebildet werden.

\subsection{GOTO}
\index{GOTO}%
GOTO-Programme bestehen aus einer markierten Folge von Anweisungen
\begin{center}
\begin{tabular}{rl}
$M_1$:&$A_1$\\
$M_2$:&$A_2$\\
$M_3$:&$A_3$\\
$\dots$:&$\dots$\\
$M_k$:&$A_k$
\end{tabular}
\end{center}
Zu den bereits bekannten,
Abschnitt~\ref{subsection:grundlegende-syntaxelement} beschriebenen
Zuweisungen
$x_i \assignment c$ 
und
$x_i \assignment x_j \pm c$ 
kommt bei GOTO eine bedingte Sprunganweisung
\begin{center}
\begin{tabular}{rl}
$M_l$:&{\tt IF\ }$x_i=c${\tt\ THEN GOTO\ }$M_j$
\end{tabular}
\end{center}
Natürlich lässt sich damit auch eine unbedingte Sprunganweisung
implementieren:
\begin{center}
\begin{tabular}{rl}
$M_l$:&$x_i \assignment c$;\\
$M_{l+1}$:&{\tt IF\ }$x_i=c${\tt\ THEN GOTO\ }$M_j$
\end{tabular}
\end{center}
In ähnlicher Weise lassen sich auch andere bedingte Anweisungen
konstruieren, zum Beispiel ein
Konstrukt {\tt IF }\dots{\tt\ THEN }\dots{\tt\ ELSE }\dots{\tt\ END}, welches
einen ganzen Anweisungsblock enthalten kann.

\subsection{Äquivalenz von WHILE und GOTO}
Die Verwendung eines Sprungbefehles wie GOTO ist in der modernen
Softwareentwicklung verpönt. Sie führe leichter zu Spaghetti-Code,
der kaum mehr wartbar ist. Gewisse Sprachen verbannen daher
GOTO vollständig aus ihrer Syntax, und propagieren dagegen
die Verwendung von `strukturierten' Kontrollstrukturen wie
WHILE. Die Aufregung ist allerdings unnötig: GOTO und WHILE sind äquivalent.


\begin{satz}
Eine Funktion ist genau dann mit einem GOTO-Programm berechenbar,
wenn sie mit einem WHILE-Programm berechenbar ist.
\end{satz}
\begin{proof}[Beweis]
Man braucht nur zu zeigen, dass man ein GOTO-Programm in ein äquivalentes
WHILE-Programm übersetzen kann, und umgekehrt.

Um eine GOTO-Programm zu übersetzen, verwenden wir eine zusätzliche Variable
$z$, die die Funktion des Programm-Zählers übernimmt.
Aus dem GOTO-Programm machen wir dann folgendes WHILE-Programm
\begin{algorithmic}
\STATE $z \assignment 1$
\STATE{\tt WHILE\ }$z>0${\tt\ DO}
\STATE{\tt IF\ }$z=1${\tt\ THEN\ }$A_1'${\tt\ END};
\STATE{\tt IF\ }$z=2${\tt\ THEN\ }$A_2'${\tt\ END};
\STATE{\tt IF\ }$z=3${\tt\ THEN\ }$A_3'${\tt\ END};
\STATE\dots
\STATE{\tt IF\ }$z=k${\tt\ THEN\ }$A_k'${\tt\ END};
\STATE{\tt IF\ }$z=k+1${\tt\ THEN\ }$z \assignment 0${\tt\ END};
\STATE{\tt END}
\end{algorithmic}
Die Anweisung $A_i'$ entsteht aus der Anweisung $A_i$ nach folgenden
Regeln
\begin{itemize}
\item Falls $A_i$ eine Zuweisung ist, wird ihr eine weitere Zuweisung
\begin{algorithmic}
\STATE $z \assignment z+1$
\end{algorithmic}
angehängt.
Dies hat zur Folge, dass nach $A_z'$ als nächste Anweisung
$A_{z+1}$ ausgeführt wird.
\item
Falls $A_i$ eine bedingte Sprunganweisung
\begin{center}
\begin{tabular}{rl}
$M_l$:&{\tt IF\ }$x_i=c${\tt\ THEN GOTO\ }$M_j$
\end{tabular}
\end{center}
ist, wird $A_i'$
\begin{algorithmic}
\STATE{\tt IF\ }$x_i=c${\tt\ THEN\ }$z \assignment j${\tt\ ELSE }$z=z+1$;
\end{algorithmic}
Dies ist zwar keine WHILE-Anweisung, aber es wurde bereits
früher gezeigt, wie man sie in WHILE übersetzen kann.
\end{itemize}
Damit ist gezeigt, dass ein GOTO-Programm in ein äquivalentes WHILE-Programm
mit genau einer WHILE-Schleife übersetzt werden kann.

Umgekehrt zeigen wir, dass jede WHILE-Schleife mit Hilfe von GOTO
implementiert werden kann. Dazu übersetzt man jede WHILE-Schleife
der Form
\begin{algorithmic}
\STATE{\tt WHILE\ }$x_i>0${\tt\ DO }$P${\tt\ END}
\end{algorithmic}
in ein GOTO-Programm-Segment der Form
\begin{algorithmic}[1]
\STATE{\tt IF\ }$x_i=0${\tt\ THEN GOTO }4
\STATE$P$
\STATE{\tt GOTO\ }1
\STATE
\end{algorithmic}
wobei die Zeilennummern durch geeignete Marken ersetzt werden müssen.
Damit haben wir einen Algorithmus spezifiziert, der WHILE-Programme in
GOTO-Programme übersetzen kann.
\end{proof}

\subsection{Turing-Vollständigkeit von WHILE und GOTO}
Da WHILE und GOTO äquivalent sind, braucht die Turing-Vollständigkeit
nur für eine der Sprachen gezeigt zu werden.
Wir skizzieren, wie man eine Turing-Maschine in ein GOTO-Programm
übersetzen kann. Dies genügt, da man nur die universelle
Turing-Maschine zu übersetzen braucht, um damit jede andere
Turing-Maschine ausführen zu können.

\subsubsection{Alphabet, Zustände und Band}
Die Zeichen des Bandalphabetes werden durch natürliche Zahlen
dargestellt.  Wir nehmen an, dass das Bandalphabet $k$ verschiedene Zeichen
umfasst. Das Leerzeichen $\text{\textvisiblespace}$ wird durch die Zahl $0$
dargestellt.

Auch die Zustände der Turing-Maschine werden durch natürliche Zahlen
dargestellt,
die Variable $s$ dient dazu, den aktuellen Zustand zu
speichern.

Der Inhalt des Bandes kann durch eine einzige Variable $b$ dargestellt
werden. Schreibt man die Zahl im System zur Basis $k$, können die
die Zeichen in den einzelnen Felder des Bandes als die Ziffern
der Zahl $b$ interpretiert werden.

Die Kopfposition wird durch eine Zahl $h$ dargestellt. Befindet sich
der Kopf im Feld mit der Nummer $i$, wird $h$ auf den Wert $k^i$
gesetzt.

\subsubsection{Arithmetik}
Alle Komponenten der Turing-Maschine werden mit natürlichen Zahlen
und arithmetischen Operationen dargestellt.
Zwar beherrscht LOOP nur die Addition oder Subtraktion einer Konstanten,
aber durch wiederholte Addition einer $1$ kann damit jede beliebige
Addition oder Subtraktion implementiert werden.

Ebenso können Multiplikation und Division auf wiederholte Addition
zurückgeführt werden.
Im Folgenden nehmen wir daher an, dass die arithmetischen Operationen
zur Verfügung stehen.

\subsubsection{Lesen eines Feldes}
Um den Inhalt eines Feldes zu lesen, muss die Stelle von $b$ an der
aktuellen Kopfposition ermittelt werden.
Dies kann durch die Rechnung
\begin{equation}
z=b / h \mod k
\label{getchar}
\end{equation}
ermittelt werden, wobei $/$ für eine ganzzahlige Division steht.
Beide Operationen können mit einem GOTO-Programm ermittelt werden.

\subsubsection{Löschen eines Feldes}
Das Feld an der Kopfposition kann wie folgt gelöscht werden.
Zunächst ermittelt man mit (\ref{getchar}) den aktuellen Feldinhalt.
Dann berechnet man
\begin{equation}
b' = b - z\cdot h.
\label{clearchar}
\end{equation}
$b'$ enthält an der Stelle der Kopfposition ein $0$.

\subsubsection{Schreiben eines Feldes}
Soll das Feld an der Kopfposition mit dem Zeichen $x$ überschrieben
werden, wird mit (\ref{clearchar}) zuerst das Feld gelöscht.
Anschliessend wird das Feld durch
\[
b'=b+x\cdot h
\]
neu gesetzt.

\subsubsection{Kopfbewegung}
Die Kopfposition wird durch die Zahl $h$ dargestellt.
Da $h$ immer eine Potenz von $k$ ist, und die Nummer des Feldes der
Exponent ist, brauchen wir nur Operationen, die den Exponenten
ändern, also
\[
h'=h/k\qquad\text{bzw.}\qquad h'=h\cdot k
\]

\subsubsection{Übergangsfunktion}
Die Übergangsfunktion
\[
\delta\colon Q\times \Gamma\to Q\times \Gamma\times\{1, 2\}
\]
ermittelt aus aktuellem Zustand $s$ und
aktuellem Zeichen $z$ den neuen Zustand, das neue Zeichen auf
dem Band und die Kopfbewegung ermittelt. Im Gegensatz zur früheren
Definition verwenden wir jetzt die Zahlen $1$ und $2$ für die
Kopfbewegung L bzw.~R.
Wir schreiben $\delta_i$ für die $i$-te Komponente von $\delta$.
Der folgende GOTO-Pseudocode
beschreibt also ein Programm, welches die Turing-Maschine implementiert
\begin{algorithmic}[1]
\STATE Bestimme das Zeichen $z$ unter der aktuellen Kopfposition $h$
\STATE Lösche das aktuelle Zeichen auf dem Band
\STATE {\tt IF\ }$s=0${\tt\ THEN}
\STATE {\tt \ \ \ \ IF\ }$z=0${\tt\ THEN}
\STATE {\tt \ \ \ \ \ \ \ \ }$s\assignment\delta_1(0,0)$
\STATE {\tt \ \ \ \ \ \ \ \ }$z\assignment\delta_2(0,0)$
\STATE {\tt \ \ \ \ \ \ \ \ }$m\assignment\delta_3(0,0)$
\STATE {\tt \ \ \ \ END}
\STATE {\tt END}
\STATE {\tt IF\ }$s=1${\tt\ THEN}
\STATE {\tt \ \ \ \ }\dots
\STATE {\tt END}
\STATE Zeichen $z$ schreiben
\STATE {\tt IF\ }$m=1${\tt\ THEN }$h\assignment h/k$
\STATE {\tt IF\ }$m=2${\tt\ THEN }$h\assignment h\cdot k$
\STATE \dots
\STATE {\tt GOTO\ }1
\end{algorithmic}
Damit ist gezeigt, dass eine gegebene Turingmaschine in ein
GOTO-Programm übersetzt werden kann. Übersetzt man die universelle
Turing-Maschine, erhält man ein GOTO-Programm, welches jede beliebige
Turing-Maschine simulieren kann. Somit ist GOTO und damit auch WHILE
Turing-vollständig.

\subsection{Esoterische Programmiersprachen}
\index{Programmiersprache!esoterische}%
Zur Illustration der Tatsache, dass eine sehr primitive Sprache
ausreichen kann, um Turing-Vollständigkeit zu erreichen, wurden
verschiedene esoterische Programmiersprachen erfunden.
Ihre Nützlichkeit liegt darin, ein bestimmtest Konzept der Theorie
möglichst klar hervorzuheben, die Verwendbarkeit für irgend einen
praktischen Zweck ist nicht notwendig, und manchmal explizit unerwünscht.

\subsubsection{Brainfuck}
\index{Brainfuck}%
Brainfuck
wurde von Urban Müller 1993 entwickelt mit dem Ziel, dass der
Compiler für diese Sprache möglichst klein sein sollte. In der
Tat ist der kleinste Brainfuck-Compiler für Linux nur 171 Bytes
lang.

Brainfuck basiert auf einem einzelnen Pointer {\tt ptr}, welcher
im Programm inkrementiert oder dekrementiert werden kann.
Jeder Pointer-Wert zeigt auf eine Zelle, deren Inhalt inkrementiert
oder dekrementiert werden kann.
Dies erinnert an die Position des Kopfes einer Turing-Maschine.
Zwei Instruktionen für Eingabe und Ausgabe eines Zeichens
an der Pointer-Position ermöglichen Datenein- und -ausgabe.
Als einzige Kontrollstruktur steht WHILE zur Verfügung. Damit
die Sprache von einem minimalisitischen Compiler kompiliert
werden kann, wird jede Anweisung durch ein einziges Zeichen
dargestellt. Die Befehle sind in der folgenden Tabelle
zusammen mit ihrem C-Äquivalent zusammengstellt:
\begin{center}
\begin{tabular}{|c|l|}
\hline
Brainfuck&C-Äquivalent\\
\hline
{\tt >}&\verb/++ptr;/\\
{\tt <}&\verb/--ptr;/\\
{\tt +}&\verb/++*ptr;/\\
{\tt -}&\verb/--*ptr;/\\
{\tt .}&\verb/putchar(*ptr);/\\
{\tt ,}&\verb/*ptr = getchar();/\\
{\tt [}&\verb/while (*ptr) {/\\
{\tt ]}&\verb/}/\\
\hline
\end{tabular}
\end{center}
Mit Hilfe des Pointers lassen sich offenbar beliebige Speicherzellen
adressieren, und diese können durch Wiederholung der Befehle {\tt +}
und {\tt -} auch um konstante Werte vergrössert
oder verkleinert werden. Etwas mehr Arbeit erfordert die Zuweisung
eines Wertes zu einer Variablen. Ist dies jedoch geschafft, kann
man WHILE in Brainfuck übersetzen, und hat damit gezeigt, dass
Brainfuck Turing-vollständig ist.

\subsubsection{Ook}
\index{Ook}%
Die Sprache Ook verwendet als syntaktische Element das Wort {\tt Ook} gefolgt
von `{\tt .}', `{\tt !}' oder `{\tt ?}'. Wer beim Lesen eines Ook-Programmes
den Eindruck hat, zum Affen gemacht zu werden, liegt nicht ganz falsch:
Ook ist eine einfache Umcodierung von Brainfuck:
\begin{center}
\begin{tabular}{|c|c|}
\hline
Ook&Brainfuck\\
\hline
{\tt Ook. Ook?}&{\tt >}\\
{\tt Ook? Ook.}&{\tt <}\\
{\tt Ook. Ook.}&{\tt +}\\
{\tt Ook! Ook!}&{\tt -}\\
{\tt Ook! Ook.}&{\tt .}\\
{\tt Ook. Ook!}&{\tt ,}\\
{\tt Ook! Ook?}&{\tt [}\\
{\tt Ook? Ook!}&{\tt ]}\\
\hline
\end{tabular}
\end{center}
Da Brainfuck Turing-vollständig ist, ist auch Ook Turing-vollständig.



%
% Turing-Vollstaendigkeit
%
% (c) 2011 Prof Dr Andreas Mueller, Hochschule Rapperswil
% $Id$
%
\chapter{Turing-Vollständigkeit\label{chapter-vollstaendigkeit}}
\lhead{Vollständigkeit}
%
% vollstaendig.tex -- Turing-Vollständigkeit
%
% (c) 2011 Prof Dr Andreas Mueller, Hochschule Rapperswil
%

\section{Turing-vollständige Programmiersprachen}
\rhead{Turing-vollständige Programmiersprachen}
Die Turing-Maschine liefert einen wohldefinierten Begriff der
Berechenbarkeit, der auch robust gegenüber milden Änderungen
der Definition einer Turing-Maschine ist.
Der Aufbau aus einem endlichen Automaten mit zusätzlichem
Speicher und der einfache Kalkül mit Konfigurationen hat
sie ausserdem Beweisen vieler wichtiger Eigenschaften zugänglich
gemacht. Die vorangegangenen Kapitel über Entscheidbarkeit und
Komplexität legen davon eindrücklich Zeugnis ab. Am direkten
Nutzen dieser Theorie kann jedoch immer noch ein gewisser Zweifel
bestehen, da ein moderner Entwickler seine Programme ja nicht
direkt für eine Turing-Maschine schreibt, sondern nur mittelbar,
da er eine Programmiersprache verwendet, deren Code anschliessend
von einem Compiler oder Interpreter übersetzt und von einer realen
Maschine ausgeführt wird.

Der Aufbau der realen Maschine ist sehr
nahe an einer Turing-Maschine, ein Prozessor liest und schreibt
jeweils einzelne
Speicherzellen eines mindestens für praktische Zwecke unendlich
grossen Speichers und ändert bei Verarbeitung der gelesenen
Inhalte seinen eigenen Zustand. Natürlich ist die Menge der
Zustände eines modernen Prozessors sehr gross, nur schon die $n$
Register der Länge $l$ tragen $2^{nl}$ verschiedene Zustände bei,
und jedes andere Zustandsbit verdoppelt die Zustandsmenge nochmals.
Trotzdem ist die Zustandsmenge endlich, und es braucht nicht viel
Fantasie, sich den Prozessor mit seinem Hauptspeicher als Turingmaschine
vorzustellen. Es gibt also kaum Zweifel, dass die Computer-Hardware
zu all dem fähig ist, was ihr in den letzten zwei Kapiteln an
Fähigkeiten zugesprochen wurde.

Die einzige Einschränkung der Fähigkeiten realer Computer gegenüber
Turing-Maschinen ist
die Tatsache, dass reale Computer nur über einen endlichen Speicher
verfügen, während eine Turing-Maschine ein undendlich langes Band
als Speicher verwenden kann. Da jedoch eine endliche Berechnung auch
nur endlich viel Speicher verwenden kann, sind alle auf einer Turing-Maschine
durchführbaren Berechnungen, die man auch tatsächlich durchführen
will, auch von einem realen Computer durchführbar. Für praktische
Zwecke darf man also annehmen, dass die realen Computer echte Turing-Maschinen
sind.

Trotzdem ist nicht sicher, ob die Programmierung in einer übersetzten
oder interpretierten Sprache alle diese Fähigkeiten auch einem
Anwendungsprogrammierer zugänglich macht.
Letztlich äussert sich dies auch darin, dass Computernutzer
für verschiedene Problemstellung auch verschiedene Werkzeuge
verwenden. Wer tabellarische Daten summieren will, wird gerne
zu einer Spreadsheet-Software greifen, aber nicht erwarten, dass er
damit auch einen Näherungsalgorithmus für das Cliquen-Problem wird
programmieren können. Die Tabellenkalkulation definiert ein eingeschränktes,
an das Problem angepasstes Berechnungsmodell, welches aber
höchstwahrscheinlich weniger leistungsfähig ist als die Hardware, auf
der es läuft. Es ist also durchaus möglich und je nach Anwendung auch
zweckmässig, dass ein Anwender nicht die volle Leistung einer
Turing-Maschine zur Verfügung hat.

Damit stellt sich jetzt die Frage, wie man einem Berechnungsmodell und
das heisst letztlich der Sprache, in der der Berechnungsauftrag
formuliert wird, ansehen kann, ob sie gleich mächtig ist wie eine
Turing-Maschine.

\subsection{Programmiersprachen}
Eine Programmiersprache ist zwar eine Sprache im Sinne dieses Skriptes,
für den Programmierer wesentlich ist jedoch die Semantik, die bisher
nicht Bestandteil der Diskussion war. Für ihn ist die Tatsache wichtig,
dass die Semantik der Sprache Berechnungen beschreibt,
wie sie mit einer Turing-Maschine ausgeführt werden können.

\begin{definition}
\index{Programmiersprache}%
Eine Sprache $A$ heisst eine {\em Programmiersprache}, wenn es eine Abbildung
\[
c\colon A\to \Sigma^*\colon w\mapsto c(w)
\]
gibt, die einem Wort der Sprache die Beschreibung einer Turing-Maschine
zuordnet. Die Abbildung $c$ heisst {\em Compiler} für die Sprache $A$.
\end{definition}
Die Forderung, dass $c(w)$ die Beschreibung einer Turing-Maschine
sein muss, ist nach obiger Diskussion nicht wesentlich.

\subsection{Interaktion}
Man beachte, dass in dieser Definition einer Programmiersprache kein Platz ist
für Input oder Output während des Programmlaufes.
Das Band der Turing-Maschine, bzw.~sein Inhalt bildet den Input, der Output
kann nach Ende der Berechnung vom Band gelesen werden.
Man könnte dies als Mangel dieses Modells ansehen, in der Tat ist aber
keine Erweiterung nötig, um Interaktion abzubilden.
Interaktionen mit einem Benutzer bestehen immer aus einem Strom von
Ereignissen, die dem Benutzer zufliessen (Änderungen des Bildschirminhaltes,
Signaltöne) oder die der Benutzer veranlasst (Bewegungen des Maus-Zeigers,
Maus-Klicks, Tastatureingaben). Alle diese Ereignisse kann man sich codiert
auf ein Band geschrieben denken, welches die Turing-Maschine bei Bedarf
liest.

Der Inhalt des Bandes einer Standard-Turing-Maschine kann während des
Programmlaufes nur von der Turing-Maschine selbst verändert werden.
Da sich die Turing-Maschine aber nicht daran erinnern kann, was beim
letzten Besuch eines Feldes dort stand, ist es für eine Turing-Maschine
auch durchaus akzeptabel, wenn der Inhalt eines Feldes von aussen
geändert wird. Natürlich werden damit das Laufzeitverhalten der
Turing-Maschine verändert. Doch in Anbetracht der
Tatsache, dass von einer Turing-Maschine im Allgemeinen nicht einmal
entschieden werden kann, ob sie anhalten wird, ist wohl nicht mehr
viel zu verlieren.

Die Ausgaben eines Programmes sind deterministisch, und was der Benutzer
erreichen will, sowie die Ereignisse, die er einspeisen wird, sind es ebenfalls.
Man kann also im Prinzip im Voraus wissen, was ausgegeben werden wird
und welche Ereignisse ein Benutzer auslösen wird. Schreibt man diese
vorgängig auf das Band, so wie man es auch beim automatisierten Testen
eines Userinterfaces tut, entsteht aus dem interaktiven Programm eines,
welches ohne Zutun des Benutzers zur Laufzeit funktionieren kann.

\subsection{Die universelle Turing-Maschine}
\index{Turing, Alan}%
\index{Turing-Maschine!universelle}%
In seinem Paper von 1936 hat Alan Turing gezeigt, dass man eine
Turing-Maschine definieren kann,
der man die Beschreibung
$\langle M,w\rangle$
einer Turing-Maschine $M$ und eines Wortes $w$
und die $M$ auf dem Input-Wort $w$ simuliert.
Diese spezielle Turing-Maschine ist also leistungsfähig genug, jede
beliebige andere Turing-Maschine zu simulieren. Sie heisst die {\em universelle
Turing-Maschine}.

Die universelle Turing-Maschine kann die Entscheidung vereinfachen,
ob eine Funktion Turing-berechenbar ist. Statt eine Turing-Maschine
zu beschreiben, die die Funktion berechnet, reicht es, ein Programm
in der Programmiersprache $A$ zu beschreiben, das Programm mit dem
Compiler $c$ zu übersetzen, und die Beschreibung mit der universellen
Turing-Maschine auszuführen.

\index{Church-Turing-Hypothese}%
Die Church-Turing-Hypothese besagt, dass sich alles, was man berechnen
kann, auch mit einer Turing-Maschine berechnen lässt. Die universelle
Turing-Maschine zeigt, dass jede berechenbare Funktion von der
universellen Turing-Maschine berechnet werden kann.
Etwas leistungsfähigeres als eine Turing-Maschine gibt es nicht.

\subsection{Turing-Vollständigkeit}
Jede Funktion, die in der Programmiersprache $A$ implementiert werden
kann, ist Turing-berechenbar.
Der Compiler kann
aber durchaus Fähigkeiten unzugänglich machen, die Programmiersprache
$A$ kann dann gewisse Berechnungen, die mit einer Turing-Maschine
möglich wären, nicht formulieren. Besonderes interessant sind daher
die Sprachen, bei denen ein solcher Verlust nicht eintritt.

\begin{definition}
\index{Turing-vollständig}%
Eine Programmiersprache heisst Turing-vollständig, wenn sich jede
berechenbare Abbildung in dieser Sprache formulieren lässt.
Zu jeder berechenbaren Abbildung $f\colon\Sigma^*\to \Sigma^*$ gibt
es also ein Programm $w$ so, dass $c(w)$ die Funktion $f$ berechnet.
\end{definition}

Zu einer berechenbaren Abbildung gibt es eine Turing-Maschine, die
sie berechnet, es würde also genügen, wenn man diese Turing-Maschine
von einem in der Sprache $A$ geschriebenen Turing-Maschinen-Simulator
ausführen lassen könnte. Dieser Begriff muss noch etwas klarer gefasst
werden:

\begin{definition}
\index{Turing-Maschinen-Simulator}%
Ein Turing-Maschinen-Simulator ist eine Turing-Maschine $S$, die als Input
die Beschreibung $\langle M,w\rangle$ einer Turing-Maschine $M$ und eines
Input-Wortes für $M$ erhält, und die Berechnung durchführt, die $M$ auf $w$
ausführen würde.
Ein Turing-Maschinen-Simulator in der Programmiersprache $A$ ist
ein Wort $s\in A$ so, dass $c(s)$ ein Turing-Maschinen-Simulator ist.
\end{definition}

Damit erhalten wir ein Kriterium für Turing-Vollständigkeit:

\begin{satz}
\label{turingvollstaendigkeitskriterium}
Eine Programmiersprache $A$ ist Turing-vollständig, genau dann
wenn es einen Turing-Maschinen-Simulator in $A$ gibt.
\end{satz}


\subsection{Beispiele}
Die üblichen Programmiersprachen sind alle Turing-vollständig, denn es
ist eine einfache Programmierübung, eines Turing-Maschinen-Simulator
in einer dieser Sprachen zu schreiben. In einigen Programmiersprachen
ist dies jedoch schwieriger als in anderen.

\subsubsection{Javascript}
\index{Javascript}%
Fabrice Bellard hat 2011 einen PC-Emulator in Javascript geschrieben, der
leistungsfähig genug ist, Linux zu booten. Auf seiner Website
\url{http://bellard.org/jslinux/} kann man den Emulator im eigenen Browser
starten. Das gebootete Linux enthält auch einen C-Compiler. Da C
Turing-vollständig ist, gibt es einen Turing-Maschinen-Simulator in
C, den man auch auf dieses Linux bringen und mit dem C-Compiler
kompilieren kann. Somit gibt es einen Turing-Maschinen-Simulator in
Javascript, Javascript ist Turing-vollständig.

\subsubsection{XSLT}
\index{XSLT}%
XSLT ist eine XML-basierte Sprache, die Transformationen von XML-Dokumenten
zu beschreiben erlaubt. XSLT ist jedoch leistungsfähig, eine Turing-Maschine
zu simulieren. Bob Lyons hat auf seiner Website
\url{http://www.unidex.com/turing/utm.htm} ein XSL-Stylesheet publiziert,
welches einen Simulator implementiert. Als Input verlangt es
ein
XML-Dokument, welches die Beschreibung der Turing-Maschine in einem
zu diesem Zweck definierten XML-Format namens Turing Machine Markup
Language (TMML) enthält. TMML definiert XML-Elemente, die das Alphabet
(\verb+<symbols>+),
die Zustandsmenge $Q$ (\verb+<state>...</state>+)
und die Übergangsfunktion $\delta$ in \verb+<mapping>+-Elementen
der Form
\begin{verbatim}
<mapping>
    <from current-state="moveRight1" current-symbol=" " />
    <to next-state="check1" next-symbol=" " movement="left" />
</mapping>
\end{verbatim}
beschreiben. Der initiale Bandinhalt wird als Parameter \verb+tape+
auf der Kommandozeile übergeben.
Das Stylesheet wandelt das TMML Dokument in eine ausführliche
Berechnungsgeschichte um, aus der auch der Bandinhalt am Ende der Berechnung
abzulesen ist. Es beweist somit, dass XSLT einen Turing-Maschinen-Simulator
hat, also Turing-vollständdig ist.

\subsubsection{\LaTeX}
\index{LaTeX@\LaTeX}%
\index{Knuth, Don}%
Don Knuth, der Autor von \TeX, hat sich lange davor gedrückt, seiner
Schriftsatz-Sprache auch eine Turing-vollständige Programmiersprache
zu spendieren. Schliesslich kam er nicht mehr darum herum, und wurde
von Guy Steeles richtigegehend dazu gedrängt, wie er in
\url{http://maps.aanhet.net/maps/pdf/16\_15.pdf}
gesteht.

Dass \TeX Turing-vollständig ist beweist ein Satz von \LaTeX-Macros, den
man auf
\url{https://www.informatik.uni-augsburg.de/en/chairs/swt/ti/staff/mark/projects/turingtex/}
finden kann.
Um ihn zu verwenden, formuliert man die Beschreibung
von Turing-Maschine und initialem Bandinhalt als eine Menge von
\LaTeX-Makros. Ebenso ruft man den Makro \verb+\RunTuringMachine+ auf,
der die Turing-Machine simuliert und die Berechnungsgeschichte im
\TeX-üblichen perfekten Schriftsatz ausgibt.




%
% kontrollstrukturen.tex
%
% (c) 2011 Prof Dr Andreas Mueller, Hochschule Rapperswil
%
\section{Kontrollstrukturen und Turing-Vollständigkeit}
\rhead{Kontrollstrukturen}
Das Turing-Vollstandigkeits-Kriterium von Satz
\ref{turingvollstaendigkeitskriterium} verlangt, dass man einen
Turing-Maschinen-Simulator in der gewählten Sprache schreiben muss.
Dies ist in jedem Fall eine nicht triviale Aufgabe.
Daher wäre es nützlich Kriterien zu erhalten, welche einfacher
anzuwenden sind. Einen wesentlichen Einfluss auf die Möglichkeiten,
was sich mit einer Programmiersprache ausdrücken lassen, haben die
Kontrollstrukturen.

\newcommand{\assignment}{\mathbin{\texttt{:=}}}

\subsection{Grundlegende Syntaxelemente%
\label{subsection:grundlegende-syntaxelement}}
In den folgenden Abschnitten werden verschiedene vereinfachte Sprachen
diskutiert, die aber einen gemeinsamen Kern fundamentaler Anweisungen
beinahlten.
In allen Sprachen gibt es nur einen einzigen Datentyp, nämlich
natürliche Zahlen.
Einerseits können Konstanten beliebig grosse natürliche Werte haben,
andererseits lassen sich in Variablen beliebig grosse natürliche Zahlen
speichern.
Dazu sind die folgenden Sytanxelemente notwendig:
\begin{compactitem}
\item Konstanten: {\tt 0}, {\tt 1}, {\tt 2}, \dots
\item Variablen $x_0$, $x_1$, $x_2$,\dots
\item Zuweisung: $\assignment$
\item Trennung von Anweisungen: {\tt ;}
\item Operatoren: {\tt +} und {\tt -}
\end{compactitem}
Die einzige Möglichkeit, den Wert einer Variablen zu ändern ist die
Zuweisung.  Diese ist entweder die Zuweisung einer Konstante als
Wert einer Variable:
\begin{algorithmic}
\STATE $x_i\assignment c$
\end{algorithmic}
oder eine Berechnung mit den beiden vorhandenen Operatoren
\begin{algorithmic}
\STATE $x_i \assignment x_j$ {\tt +} $c$
\STATE $x_i \assignment x_j$ {\tt -} $c$
\end{algorithmic}
Dabei ist das Resultat der Subtraktion als $0$ definiert, wenn
der Minuend kleiner ist als der Subtrahend.

Die verschiedenen Sprachen unterschieden sich in den Kontrollstrukturen
mit denen diese Zuweisungsbefehle miteinander verbunden werden können.

\subsection{LOOP}
\index{LOOP}%
Die Programmiersprache
LOOP\footnote{Die in diesem Abschnitt beschriebene
LOOP Sprache darf nicht verwechselt werden mit dem gleichnamigen
Projekt einer objektorientierten parallelen Sprache seit
2001 auf Sourceforge.}
hat als einzige Kontrollstruktur die
Iteration eines Anweisungs-Blocks mit einer festen, innerhalb des
Blocks nicht veränderbaren Anzahl von Durchläufen.

\subsubsection{Syntax}
LOOP verwendet die 
Schlüsselwörter: {\tt LOOP}, {\tt DO}, {\tt END}
Programme werden daraus wie folgt aufgebaut:
\begin{itemize}
\item Das leere Programm $\varepsilon$ ist eine LOOP-Programm,
während der Ausführung tut es nichts.
\item Wertzuweisungen sind LOOP-Programme
\item Sind $P_1$ und $P_2$ LOOP-Programme, dann auch
$P_1{\tt ;}P_2$. Um dieses Programm auszuführen wird zuerst $P_1$
ausgeführt und anschliessend $P_2$.
\item Ist $P$ ein LOOP-Programm, dann ist auch
\begin{algorithmic}
\STATE {\tt LOOP} $x_i$ {\tt DO} $P$ {\tt END}
\end{algorithmic}
ein LOOP-Programm. Die Ausführung wiederholt $P$ so oft, wie der
Wert der Variable $x_i$  zu Beginn angibt.
\end{itemize}
Durch Setzen der Variablen $x_i$ kann einem LOOP-Progamm Input übergeben
werden.
Ein LOOP-Programm definiert also eine Abbildung $\mathbb N^k\to\mathbb N$,
wobei $k$ die Anzahl der Variablen ist, in denen Input zu übergeben ist.

\subsubsection{Beispiel: Summer zweier Variablen}
LOOP kann nur jeweils eine Konstante hinzuaddieren, das genügt aber
bereits, um auch die Summe zweier Variablen zu berechnen. Der folgende
Code berechnet die Summe von $x_1$ und $x_2$ und legt das Resultat in
$x_0$ ab:
\begin{algorithmic}
\STATE $x_0 \assignment x_1$
\STATE {\tt LOOP} $x_2$ {\tt DO}
\STATE{\tt \ \ \ \ }$x_0 \assignment x_0$ {\tt +} $1$
\STATE{\tt END}
\end{algorithmic}

\subsubsection{Beispiel: Produkt zweier Variablen}
\index{Multiplikation}%
Die Sprache LOOP ist offenbar relativ primitiv, trotzdem
ist sie leistungsfähig genug, um die Multiplikation von zwei
Zahlen, die in den Variablen $x_1$ und $x_2$ übergeben werden,
zu berechnen:

\begin{algorithmic}
\STATE $x_0 \assignment 0$
\STATE {\tt LOOP} $x_1$ {\tt DO}
\STATE{\tt \ \ \ \ LOOP} $x_2$ {\tt DO}
\STATE{\tt \ \ \ \ \ \ \ \ }$x_0 \assignment x_0$ {\tt +} $1$
\STATE{\tt \ \ \ \ END}
\STATE{\tt END}
\end{algorithmic}

\subsubsection{IF-Anweisung}
In LOOP fehlt eine IF-Anweisung, auf die die wenigsten Programmiersprachen
verzichten. Es ist jedoch leicht, eine solche in LOOP nachzubilden.
Eine Anweisung
\begin{algorithmic}
\STATE {\tt IF} $x = 0$ {\tt THEN } $P$ {\tt END}
\end{algorithmic}
kann unter Verwendung einer zusätzlichen Variable $y$ in LOOP ausgedrückt
werden:
\begin{algorithmic}[1]
\STATE $y \assignment 1${\tt ;}
\STATE {\tt LOOP }$x${\tt\ DO }$y \assignment 0$ {\tt END;}
\STATE {\tt LOOP }$y${\tt\ DO }$P$ {\tt END;}
\end{algorithmic}
Natürlich wird die {\tt LOOP}-Anweisung in Zeile 2 meistens viel öfter
als nötig ausgeführt, aber es stehen hier ja auch nicht Fragen
der Effizienz sonder der prinzipiellen Machbarkeit zur Diskussion.

\subsubsection{LOOP ist nicht Turing-vollständig}
\begin{satz}
LOOP-Programme terminieren immer.
\end{satz}

\begin{proof}[Beweis]
Der Beweis kann mit Induktion über die Schachtelungstiefe
von {\tt LOOP}-Anweisungen geführt werden. Die LOOP-Programme
ohne {\tt LOOP}-Anweisung, also die Programme mit Schachtelungstiefe
$0$ terminieren offensichtlich immer. Ebenso die LOOP-Programme
mit Schachtelungstiefe $1$, da die Anzahl der Schleifendurchläufe
bereits zu Beginn der Schleife festliegt.

Nehmen wir jetzt an wir wüssten bereits, dass jedes LOOP-Programm mit
Schachtelungstiefe $n$ der {\tt LOOP}-Anweisungen immer terminiert.
Ein LOOP-Programm mit Schachtelungstiefe $n+1$ ist dann eine
Abfolge von Teilprogrammen mit Schachtelungstiefe $n$, die nach
Voraussetzung alle terminieren, und {\tt LOOP}-Anweisungen der
Form
\begin{algorithmic}
\STATE{\tt LOOP }$x_i${\tt\ DO }$P${\tt\ END}
\end{algorithmic}
wobei $P$ ein LOOP-Programm mit Schachtelungstiefe $n$ ist, das also
ebenfalls immer terminiert. $P$ wird genau so oft ausgeführt, wie
$x_i$ zu Beginn angibt. Die Laufzeit von $P$ kann dabei jedesmal
anders sein, $P$ wird aber auf jeden Fall terminieren. Damit ist
gezeigt, dass auch alle LOOP-Programme mit Schachtelungstiefe $n+1$
terminieren.
\end{proof}

\begin{satz}
LOOP ist nicht Turing vollständig.
\end{satz}

\begin{proof}[Beweis]
Es gibt Turing-Maschinen, die nicht terminieren. Gäbe es einen
Turing-Maschinen-Simulator in LOOP, dürfte dieser bei der
Simulation einer solchen Turing-Maschine nicht terminieren, im
Widerspruch zur Tatsache, dass LOOP-Programme immer terminieren.
Also kann es keinen Turing-Maschinen-Simulator in LOOP geben.
\end{proof}

\subsection{WHILE}
\index{WHILE}%
WHILE-Programme können zusätzlich zur {\tt LOOP}-Anweisung
eine {\tt WHILE}-Anweisung der Form
\begin{algorithmic}
\STATE{\tt WHILE }$x_i>0${\tt\ DO }$P${\tt\ END}
\end{algorithmic}
Dadurch ist die {\tt LOOP}-Anweisung nicht mehr unbedingt
notwendig, denn
\begin{algorithmic}
\STATE{\tt LOOP }$x${\tt\ DO }$P${\tt\ END}
\end{algorithmic}
kann durch
\begin{algorithmic}
\STATE$y \assignment x$;
\STATE{\tt WHILE }$y>0${\tt\ DO }$P$; $y \assignment y-1${\tt\ END}
\end{algorithmic}
nachgebildet werden.

\subsection{GOTO}
\index{GOTO}%
GOTO-Programme bestehen aus einer markierten Folge von Anweisungen
\begin{center}
\begin{tabular}{rl}
$M_1$:&$A_1$\\
$M_2$:&$A_2$\\
$M_3$:&$A_3$\\
$\dots$:&$\dots$\\
$M_k$:&$A_k$
\end{tabular}
\end{center}
Zu den bereits bekannten,
Abschnitt~\ref{subsection:grundlegende-syntaxelement} beschriebenen
Zuweisungen
$x_i \assignment c$ 
und
$x_i \assignment x_j \pm c$ 
kommt bei GOTO eine bedingte Sprunganweisung
\begin{center}
\begin{tabular}{rl}
$M_l$:&{\tt IF\ }$x_i=c${\tt\ THEN GOTO\ }$M_j$
\end{tabular}
\end{center}
Natürlich lässt sich damit auch eine unbedingte Sprunganweisung
implementieren:
\begin{center}
\begin{tabular}{rl}
$M_l$:&$x_i \assignment c$;\\
$M_{l+1}$:&{\tt IF\ }$x_i=c${\tt\ THEN GOTO\ }$M_j$
\end{tabular}
\end{center}
In ähnlicher Weise lassen sich auch andere bedingte Anweisungen
konstruieren, zum Beispiel ein
Konstrukt {\tt IF }\dots{\tt\ THEN }\dots{\tt\ ELSE }\dots{\tt\ END}, welches
einen ganzen Anweisungsblock enthalten kann.

\subsection{Äquivalenz von WHILE und GOTO}
Die Verwendung eines Sprungbefehles wie GOTO ist in der modernen
Softwareentwicklung verpönt. Sie führe leichter zu Spaghetti-Code,
der kaum mehr wartbar ist. Gewisse Sprachen verbannen daher
GOTO vollständig aus ihrer Syntax, und propagieren dagegen
die Verwendung von `strukturierten' Kontrollstrukturen wie
WHILE. Die Aufregung ist allerdings unnötig: GOTO und WHILE sind äquivalent.


\begin{satz}
Eine Funktion ist genau dann mit einem GOTO-Programm berechenbar,
wenn sie mit einem WHILE-Programm berechenbar ist.
\end{satz}
\begin{proof}[Beweis]
Man braucht nur zu zeigen, dass man ein GOTO-Programm in ein äquivalentes
WHILE-Programm übersetzen kann, und umgekehrt.

Um eine GOTO-Programm zu übersetzen, verwenden wir eine zusätzliche Variable
$z$, die die Funktion des Programm-Zählers übernimmt.
Aus dem GOTO-Programm machen wir dann folgendes WHILE-Programm
\begin{algorithmic}
\STATE $z \assignment 1$
\STATE{\tt WHILE\ }$z>0${\tt\ DO}
\STATE{\tt IF\ }$z=1${\tt\ THEN\ }$A_1'${\tt\ END};
\STATE{\tt IF\ }$z=2${\tt\ THEN\ }$A_2'${\tt\ END};
\STATE{\tt IF\ }$z=3${\tt\ THEN\ }$A_3'${\tt\ END};
\STATE\dots
\STATE{\tt IF\ }$z=k${\tt\ THEN\ }$A_k'${\tt\ END};
\STATE{\tt IF\ }$z=k+1${\tt\ THEN\ }$z \assignment 0${\tt\ END};
\STATE{\tt END}
\end{algorithmic}
Die Anweisung $A_i'$ entsteht aus der Anweisung $A_i$ nach folgenden
Regeln
\begin{itemize}
\item Falls $A_i$ eine Zuweisung ist, wird ihr eine weitere Zuweisung
\begin{algorithmic}
\STATE $z \assignment z+1$
\end{algorithmic}
angehängt.
Dies hat zur Folge, dass nach $A_z'$ als nächste Anweisung
$A_{z+1}$ ausgeführt wird.
\item
Falls $A_i$ eine bedingte Sprunganweisung
\begin{center}
\begin{tabular}{rl}
$M_l$:&{\tt IF\ }$x_i=c${\tt\ THEN GOTO\ }$M_j$
\end{tabular}
\end{center}
ist, wird $A_i'$
\begin{algorithmic}
\STATE{\tt IF\ }$x_i=c${\tt\ THEN\ }$z \assignment j${\tt\ ELSE }$z=z+1$;
\end{algorithmic}
Dies ist zwar keine WHILE-Anweisung, aber es wurde bereits
früher gezeigt, wie man sie in WHILE übersetzen kann.
\end{itemize}
Damit ist gezeigt, dass ein GOTO-Programm in ein äquivalentes WHILE-Programm
mit genau einer WHILE-Schleife übersetzt werden kann.

Umgekehrt zeigen wir, dass jede WHILE-Schleife mit Hilfe von GOTO
implementiert werden kann. Dazu übersetzt man jede WHILE-Schleife
der Form
\begin{algorithmic}
\STATE{\tt WHILE\ }$x_i>0${\tt\ DO }$P${\tt\ END}
\end{algorithmic}
in ein GOTO-Programm-Segment der Form
\begin{algorithmic}[1]
\STATE{\tt IF\ }$x_i=0${\tt\ THEN GOTO }4
\STATE$P$
\STATE{\tt GOTO\ }1
\STATE
\end{algorithmic}
wobei die Zeilennummern durch geeignete Marken ersetzt werden müssen.
Damit haben wir einen Algorithmus spezifiziert, der WHILE-Programme in
GOTO-Programme übersetzen kann.
\end{proof}

\subsection{Turing-Vollständigkeit von WHILE und GOTO}
Da WHILE und GOTO äquivalent sind, braucht die Turing-Vollständigkeit
nur für eine der Sprachen gezeigt zu werden.
Wir skizzieren, wie man eine Turing-Maschine in ein GOTO-Programm
übersetzen kann. Dies genügt, da man nur die universelle
Turing-Maschine zu übersetzen braucht, um damit jede andere
Turing-Maschine ausführen zu können.

\subsubsection{Alphabet, Zustände und Band}
Die Zeichen des Bandalphabetes werden durch natürliche Zahlen
dargestellt.  Wir nehmen an, dass das Bandalphabet $k$ verschiedene Zeichen
umfasst. Das Leerzeichen $\text{\textvisiblespace}$ wird durch die Zahl $0$
dargestellt.

Auch die Zustände der Turing-Maschine werden durch natürliche Zahlen
dargestellt,
die Variable $s$ dient dazu, den aktuellen Zustand zu
speichern.

Der Inhalt des Bandes kann durch eine einzige Variable $b$ dargestellt
werden. Schreibt man die Zahl im System zur Basis $k$, können die
die Zeichen in den einzelnen Felder des Bandes als die Ziffern
der Zahl $b$ interpretiert werden.

Die Kopfposition wird durch eine Zahl $h$ dargestellt. Befindet sich
der Kopf im Feld mit der Nummer $i$, wird $h$ auf den Wert $k^i$
gesetzt.

\subsubsection{Arithmetik}
Alle Komponenten der Turing-Maschine werden mit natürlichen Zahlen
und arithmetischen Operationen dargestellt.
Zwar beherrscht LOOP nur die Addition oder Subtraktion einer Konstanten,
aber durch wiederholte Addition einer $1$ kann damit jede beliebige
Addition oder Subtraktion implementiert werden.

Ebenso können Multiplikation und Division auf wiederholte Addition
zurückgeführt werden.
Im Folgenden nehmen wir daher an, dass die arithmetischen Operationen
zur Verfügung stehen.

\subsubsection{Lesen eines Feldes}
Um den Inhalt eines Feldes zu lesen, muss die Stelle von $b$ an der
aktuellen Kopfposition ermittelt werden.
Dies kann durch die Rechnung
\begin{equation}
z=b / h \mod k
\label{getchar}
\end{equation}
ermittelt werden, wobei $/$ für eine ganzzahlige Division steht.
Beide Operationen können mit einem GOTO-Programm ermittelt werden.

\subsubsection{Löschen eines Feldes}
Das Feld an der Kopfposition kann wie folgt gelöscht werden.
Zunächst ermittelt man mit (\ref{getchar}) den aktuellen Feldinhalt.
Dann berechnet man
\begin{equation}
b' = b - z\cdot h.
\label{clearchar}
\end{equation}
$b'$ enthält an der Stelle der Kopfposition ein $0$.

\subsubsection{Schreiben eines Feldes}
Soll das Feld an der Kopfposition mit dem Zeichen $x$ überschrieben
werden, wird mit (\ref{clearchar}) zuerst das Feld gelöscht.
Anschliessend wird das Feld durch
\[
b'=b+x\cdot h
\]
neu gesetzt.

\subsubsection{Kopfbewegung}
Die Kopfposition wird durch die Zahl $h$ dargestellt.
Da $h$ immer eine Potenz von $k$ ist, und die Nummer des Feldes der
Exponent ist, brauchen wir nur Operationen, die den Exponenten
ändern, also
\[
h'=h/k\qquad\text{bzw.}\qquad h'=h\cdot k
\]

\subsubsection{Übergangsfunktion}
Die Übergangsfunktion
\[
\delta\colon Q\times \Gamma\to Q\times \Gamma\times\{1, 2\}
\]
ermittelt aus aktuellem Zustand $s$ und
aktuellem Zeichen $z$ den neuen Zustand, das neue Zeichen auf
dem Band und die Kopfbewegung ermittelt. Im Gegensatz zur früheren
Definition verwenden wir jetzt die Zahlen $1$ und $2$ für die
Kopfbewegung L bzw.~R.
Wir schreiben $\delta_i$ für die $i$-te Komponente von $\delta$.
Der folgende GOTO-Pseudocode
beschreibt also ein Programm, welches die Turing-Maschine implementiert
\begin{algorithmic}[1]
\STATE Bestimme das Zeichen $z$ unter der aktuellen Kopfposition $h$
\STATE Lösche das aktuelle Zeichen auf dem Band
\STATE {\tt IF\ }$s=0${\tt\ THEN}
\STATE {\tt \ \ \ \ IF\ }$z=0${\tt\ THEN}
\STATE {\tt \ \ \ \ \ \ \ \ }$s\assignment\delta_1(0,0)$
\STATE {\tt \ \ \ \ \ \ \ \ }$z\assignment\delta_2(0,0)$
\STATE {\tt \ \ \ \ \ \ \ \ }$m\assignment\delta_3(0,0)$
\STATE {\tt \ \ \ \ END}
\STATE {\tt END}
\STATE {\tt IF\ }$s=1${\tt\ THEN}
\STATE {\tt \ \ \ \ }\dots
\STATE {\tt END}
\STATE Zeichen $z$ schreiben
\STATE {\tt IF\ }$m=1${\tt\ THEN }$h\assignment h/k$
\STATE {\tt IF\ }$m=2${\tt\ THEN }$h\assignment h\cdot k$
\STATE \dots
\STATE {\tt GOTO\ }1
\end{algorithmic}
Damit ist gezeigt, dass eine gegebene Turingmaschine in ein
GOTO-Programm übersetzt werden kann. Übersetzt man die universelle
Turing-Maschine, erhält man ein GOTO-Programm, welches jede beliebige
Turing-Maschine simulieren kann. Somit ist GOTO und damit auch WHILE
Turing-vollständig.

\subsection{Esoterische Programmiersprachen}
\index{Programmiersprache!esoterische}%
Zur Illustration der Tatsache, dass eine sehr primitive Sprache
ausreichen kann, um Turing-Vollständigkeit zu erreichen, wurden
verschiedene esoterische Programmiersprachen erfunden.
Ihre Nützlichkeit liegt darin, ein bestimmtest Konzept der Theorie
möglichst klar hervorzuheben, die Verwendbarkeit für irgend einen
praktischen Zweck ist nicht notwendig, und manchmal explizit unerwünscht.

\subsubsection{Brainfuck}
\index{Brainfuck}%
Brainfuck
wurde von Urban Müller 1993 entwickelt mit dem Ziel, dass der
Compiler für diese Sprache möglichst klein sein sollte. In der
Tat ist der kleinste Brainfuck-Compiler für Linux nur 171 Bytes
lang.

Brainfuck basiert auf einem einzelnen Pointer {\tt ptr}, welcher
im Programm inkrementiert oder dekrementiert werden kann.
Jeder Pointer-Wert zeigt auf eine Zelle, deren Inhalt inkrementiert
oder dekrementiert werden kann.
Dies erinnert an die Position des Kopfes einer Turing-Maschine.
Zwei Instruktionen für Eingabe und Ausgabe eines Zeichens
an der Pointer-Position ermöglichen Datenein- und -ausgabe.
Als einzige Kontrollstruktur steht WHILE zur Verfügung. Damit
die Sprache von einem minimalisitischen Compiler kompiliert
werden kann, wird jede Anweisung durch ein einziges Zeichen
dargestellt. Die Befehle sind in der folgenden Tabelle
zusammen mit ihrem C-Äquivalent zusammengstellt:
\begin{center}
\begin{tabular}{|c|l|}
\hline
Brainfuck&C-Äquivalent\\
\hline
{\tt >}&\verb/++ptr;/\\
{\tt <}&\verb/--ptr;/\\
{\tt +}&\verb/++*ptr;/\\
{\tt -}&\verb/--*ptr;/\\
{\tt .}&\verb/putchar(*ptr);/\\
{\tt ,}&\verb/*ptr = getchar();/\\
{\tt [}&\verb/while (*ptr) {/\\
{\tt ]}&\verb/}/\\
\hline
\end{tabular}
\end{center}
Mit Hilfe des Pointers lassen sich offenbar beliebige Speicherzellen
adressieren, und diese können durch Wiederholung der Befehle {\tt +}
und {\tt -} auch um konstante Werte vergrössert
oder verkleinert werden. Etwas mehr Arbeit erfordert die Zuweisung
eines Wertes zu einer Variablen. Ist dies jedoch geschafft, kann
man WHILE in Brainfuck übersetzen, und hat damit gezeigt, dass
Brainfuck Turing-vollständig ist.

\subsubsection{Ook}
\index{Ook}%
Die Sprache Ook verwendet als syntaktische Element das Wort {\tt Ook} gefolgt
von `{\tt .}', `{\tt !}' oder `{\tt ?}'. Wer beim Lesen eines Ook-Programmes
den Eindruck hat, zum Affen gemacht zu werden, liegt nicht ganz falsch:
Ook ist eine einfache Umcodierung von Brainfuck:
\begin{center}
\begin{tabular}{|c|c|}
\hline
Ook&Brainfuck\\
\hline
{\tt Ook. Ook?}&{\tt >}\\
{\tt Ook? Ook.}&{\tt <}\\
{\tt Ook. Ook.}&{\tt +}\\
{\tt Ook! Ook!}&{\tt -}\\
{\tt Ook! Ook.}&{\tt .}\\
{\tt Ook. Ook!}&{\tt ,}\\
{\tt Ook! Ook?}&{\tt [}\\
{\tt Ook? Ook!}&{\tt ]}\\
\hline
\end{tabular}
\end{center}
Da Brainfuck Turing-vollständig ist, ist auch Ook Turing-vollständig.



\appendix
\chapter{Algorithmen-Übersicht\label{skript:algorithmen}}
\lhead{Algorithmen-Übersicht}
\rhead{}
In den vergangenen Kapiteln wurde eine grosse Zahl von Algorithmen
für die verschiedensten Probleme formuliert. Hier werden die wichtigsten
im Sinne einer Übersicht zusammengestellt, mit Verweisen auf die 
detaillierte Beschreibung weiter vorne im Text.
\section{Endliche Automaten und reguläre Sprachen}
\subsection{Minimalautomat}
\newtheorem*{Minimalautomat}{Minimalautomat}
\begin{Minimalautomat}
Zu jedem deterministischen endlichen Automaten $A$ finde den minimalen
Automaten $A'$.
\end{Minimalautomat}
Der Satz~\ref{satz_minimalautomat} beschreibt die Eigenschaften des
Minimalautomaten, anschliessend im Text wird der ``Kreuzchen''-Algorithmus
beschrieben, mit dem der Minimalautomat gefunden werden kann.

\newtheorem*{Automatenvergleich}{Automatenvergleich}
\begin{Automatenvergleich}
Gegeben zwei endliche Automaten $A$ und $B$ finde heraus, ob die
beiden Automaten die gleiche Sprache akzeptieren, also $L(A)=L(B)$.
\end{Automatenvergleich}
Das Problem, ob zwei Automaten die gleiche Sprache akzeptieren,
ist mit Hilfe des Minimalautomaten (Seite \pageref{algorithmus:minimalautomat})
entscheidbar. 
Der Algorithmus wurde auch im Satz~\ref{satz:eqdea} verwendet, wo die
Entscheidbarkeit von $\textsl{EQ}_\textsl{DEA}$ gezeigt wurde.


\subsection{NEA}
\newtheorem*{NEA}{Umwandlung NEA $\to$ DEA}
\begin{NEA}
Ein nicht deterministischer endlicher Automat $A$ kann in einen 
deterministischen endlichen Automaten $B$ umgewandelt werden, der die
gleiche Sprache akzeptiert, $L(A)=L(B)$.
\end{NEA}

Der Umwandlungsalgorithmus NEA $\to$ DEA wird in
Abschnitt~\ref{regulaer:nea-dea}
auf
Seite~\pageref{regulaer:nea-dea}
beschrieben.

\subsection{Mengenoperationen}
Die Menge der reguläre Sprachen ist abgeschlossen bezüglich der
Mengenoperationen, es muss also Algorithmen geben, die die Mengenoperationen
auf Automatenebene implementieren.

\newtheorem*{RegVereinigung}{Vereinigung regulärer Sprachen}
\begin{RegVereinigung}
Zu zwei deterministischen endlichen Automaten $A$ und $B$ berechne einen
Automaten $C$ derart, dass $L(C)=L(A)\cup L(B)$.
\end{RegVereinigung}

Die Vereinigung wird in Satz~\ref{satz_union} beschrieben, sie tritt auch als
Alternative bei den regulären Operationen auf.

\newtheorem*{RegSchnitt}{Schnittemenge regulärer Sprachen}
\begin{RegSchnitt}
Zu zwei deterministischen endlichen Automaten $A$ und $B$ berechne einen
Automaten $C$ derart, dass $L(C)=L(A)\cap L(B)$.
\end{RegSchnitt}

Satz~\ref{satz_intersection} beschreibt eine Konstruktion, mit der
ein deterministischer endlicher Automat für die Schnittmenge gefunden
werden kann. Sie verwendet die Produktautomaten-Konstruktion von Seite
\pageref{reg_produktautomat}.

\newtheorem*{RegNegation}{Komplement einer regulären Sprache}
\begin{RegNegation}
Zu einem deterministischen endlichen Automaten $A$ berechne den Automaten
$B$ mit der Eigenschaft $L(B)=\overline{L(A)}$.
\end{RegNegation}

Der Algorithmus tauscht Akzeptier- und Nichtakzeptierzustände aus,
und funktioniert in dieser Form nur für deterministische endliche
Automaten (Satz~\ref{satz_regcomplement}).

\newtheorem*{RegDifferenz}{Differenz zweier regulärer Sprachen}
\begin{RegDifferenz}
Zu zwei gegebenen deterministischen endlichen Automaten $A$ und $B$ 
berechne einen deterministischen endlichen Automaten $C$ mit
$L(C)=L(A)\setminus L(B)$.
\end{RegDifferenz}

Die Mengendifferenz ist die Schnittmenge mit dem Komplement:
$L(A)\setminus L(B)=L(A)\cap\overline{L(B)}$, der Algorithmus wird im Beweis
von Satz~\ref{satz_regcomplement} dargestellt.

\subsection{Reguläre Operationen}
Reguläre Operationen können in Operationen mit nicht deterministischen
endlichen Automaten übersetzt werden.
Die Verkettung wird in Satz~\ref{satz_concat} beschrieben, die *-Operation
in Satz~\ref{satz_star}.

\newtheorem*{Alternative}{Alternative}
\begin{Alternative}Zu zwei endlichen Automaten $A$ und $B$ berechne einen
Automaten $C$, der die Alternative der beiden Sprachen akzeptiert, also
$L(C)=L(A)|L(B)=L(A)\cup L(B)$.
\end{Alternative}

\newtheorem*{Verkettung}{Verkettung}
\begin{Verkettung}
Zu zwei endlichen Automaten $A$ und $B$ berechnen einen Automaten $C$,
der die Verkettung der Sprachen von $A$ und $B$ akzeptiert: $L(C)=L(A)L(B)$.
\end{Verkettung}

\newtheorem*{Sternoperation}{Stern-Operation}
\begin{Sternoperation}
Zu einem endlichen Automaten $A$ berechne einen Automaten $B$, der die
Stern-Operation der Sprache von $A$ akzeptiert: $L(B)=L(A)^*$.
\end{Sternoperation}



\subsection{Reguläre Ausdrücke}
\newtheorem*{RegexDea}{Regulären Ausdruck in DEA umwandeln}
\begin{RegexDea} Berechne aus einem regulären Ausdruck $r$ einen 
deterministischen endlichen Automaten $A$, der die gleiche Sprache
akzeptiert, also $L(A)=L(r)$.
\end{RegexDea}

Die Umwandlung eines regulären Ausdrucks in einen endlichen Automaten
wird in Abschnitt~\ref{regulaer:regulaere-ausdruecke} beschrieben.

\newtheorem*{DeaRegex}{DEA in regulären Ausdruck umwandeln}
\begin{DeaRegex}
Zu einem deterministischen endlichen Automaten $A$ berechne einen
regulären Ausdruck $r$, der die gleiche Sprache akzeptiert,
also $L(A)=L(r)$.
\end{DeaRegex}

Die Umwandlung eines endlichen Automaten in einen äquivalenten
regulären Ausdrucks 
wird in Abschnitt~\ref{regulaer:dea-re} beschrieben.

\section{Stackautomaten und kontextfreie Grammatiken}

\subsection{Stackautomaten}
Zum Beweis der Äquivalenz von Stackautomaten und kontextfreien Grammatiken
wurden im Abschnitt~\ref{sect:aequivalenz-cfg} zwei Algorithmen beschrieben,
wie man eine kontextfreie Grammatik in einen äquivalenten Stackautomanten
umwandenl kann und umgekehrt.

\newtheorem*{CfgPDA}{Stackautomat einer Grammatik}
\begin{CfgPDA}
Zu einer kontextfreien Grammatik $G$ finde einen Stackautomaten $P$, 
der genau die von $G$ produzierte Sprache akzeptiert: $L(G)=L(P)$.
\end{CfgPDA}

\newtheorem*{PdaCfg}{Grammatik eines Stackautomaten}
\begin{PdaCfg}
Zu einem Stackautomaten $P$ findet eine kontextfreie Grammatik $G$, die
die gleiche Sprache produziert: $L(G)=L(P)$.
\end{PdaCfg}

\subsection{Chomsky-Normalform}
Die Chomsky-Normalform wird in Definition~\ref{definition:cnf} definiert.
Im Beweis von Satz~\ref{satz:cnf} wird die Umwandlung in eine äquivalente
Grammatik in Chomsky-Normalform dargestellt.

\newtheorem*{CNF}{Chomsky-Normalform einer Grammatik}
\begin{CNF}
Zu einer kontextfreien Grammatik $G$ finde eine kontextfreie Grammatik $G'$
in Chomsky-Normalform, die die gleiche Sprache akzeptiert.
\end{CNF}

\subsection{Mengenoperationen}
Nur für die Vereinigung zweier kontextfreier Sprachen haben wir einen
Algorithmus.

\newtheorem*{CfgUnion}{Grammatik einer Vereinigung}
\begin{CfgUnion}
Aus zwei kontextfreien Grammatiken $G_1$ und $G_2$ berechne
eine kontextfreie Grammatik $G$, die die
Vereinigung der von $G_1$ und $G_2$ produzierten Sprachen prodziert:
$L(G)=L(G_1)\cup L(G_2)$.
\end{CfgUnion}

\subsection{Reguläre Operationen}
Die regulären Operationen auf Grammatiken wurden in
Abschnitt~\ref{sect:cfg-regulaer} beschrieben.
Die Alternative ist in Satz~\ref{satz:cfg-union} erklärt,
die Verkettung in Satz~\ref{satz:cfg-verkettung} und die *-Operation
in Satz~\ref{satz:cfg-star}.

\newtheorem*{CfgAlternative}{Grammatik für eine Alterative}
\begin{CfgAlternative}
Zu zwei kontextfreien Grammatiken $G_1$ und $G_2$ finde eine
kontextfreie Grammatik $G$, die die
Alternative der von $G_1$ und $G_2$ produzierten Sprachen produziert:
$L(G)=L(G_1)|L(G_2)=L(G_1)\cup L(G_2)$.
\end{CfgAlternative}

\newtheorem*{CfgConcatenation}{Grammatik einer Verkettung}
\begin{CfgConcatenation}
Zu zwei kontextfreien Grammatiken $G_1$ und $G_2$ berechne eine 
kontextfreie Grammatik $G$, die die Verkettung der von $G_1$ und $G_2$
produzierten Sprachen produziert: $L(G)=L(G_1)L(G_2)$.
\end{CfgConcatenation}

\newtheorem*{CfgStar}{Grammatik für die Sternoperation}
\begin{CfgStar}
Zu einer kontextfreien Grammatik $G$ berechne eine kontextfreie Grammatik
$G'$, die die *-Operation der von $G$ produziert Sprache produziert:
$L(G')=L(G)^*$.
\end{CfgStar}

\subsection{Parser}
Stackautomaten sind nicht deterministisch und eignen sich daher nicht
dazu, ein Wort $w$ daraufhinzu zu prüfen, ob es von einer kontextfreien
Grammatik $G$ produziert werden kann: $w\in L(G)$. 
Es gibt aber einen deterministischen Algorithmus, der in
Satz~\ref{cyk-algorithm} beschrieben wird.
Dieser Algorithmus wird auch zur Entscheidung des Problems $A_\textsl{CFG}$
in Satz~\ref{satz:acfg-entscheidbar}
benötigt.

\newtheorem*{CYK}{Cocke-Younger-Kasami Algorithmus}
\begin{CYK}
Zu einer kontextfreien Grammatik $G$ in Chomsky-Nor\-mal\-form und einem
Wort $w$ berechne den Ableitungsbaum (Parse-Tree) von $w$.
\end{CYK}

%\section{Graphen}

%\section{Turingmaschinen}


\vfill
\pagebreak
\ifodd\value{page}\else\null\clearpage\fi
\lhead{}
\rhead{}
\printbibliography[heading=subbibliography]
\end{refsection}

\input skript.ind
\end{document}
