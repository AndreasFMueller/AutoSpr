%
% skript.tex -- Skript zur Vorlesung Automaten und Sprachen
%               gehalten an der Hochschule Rapperswil im Wintersemester 09
%
% (c) 2009 Prof. Dr. Andreas Mueller, HSR
% $Id: skript.tex,v 1.34 2008/11/02 22:46:16 afm Exp $
%
\documentclass[a4paper,12pt]{book}
\usepackage[ngerman]{babel}
\usepackage[T1]{fontenc}
\usepackage{csquotes}
\usepackage{float} %added see: https://tex.stackexchange.com/questions/8625/force-figure-placement-in-text
\usepackage{times}
\usepackage{geometry}
\geometry{papersize={210mm,297mm},total={160mm,240mm},top=31mm,bindingoffset=15mm}
\usepackage{alltt}
\usepackage{verbatim}
\usepackage{fancyhdr}
\usepackage{amsmath}
\usepackage{amssymb}
\usepackage{amsfonts}
\usepackage{amsthm}
\usepackage{textcomp}
\usepackage{graphicx}
\usepackage{array}
\usepackage{ifthen}
\usepackage{multirow}
\usepackage{txfonts}
%\usepackage[basic]{circ}
\usepackage[all]{xy}
\usepackage{algorithm}
\usepackage{algorithmic}
\usepackage{makeidx}
\usepackage{paralist}
\usepackage[utf8]{inputenc}
\usepackage[colorlinks=true]{hyperref}
\usepackage[backend=bibtex]{biblatex}
\addbibresource{references.bib}
\makeindex
\setlength{\headheight}{15pt}
\newcommand\myatop[2]{\genfrac{}{}{0pt}{}{#1}{#2}}
\begin{document}
\pagestyle{fancy}
\lhead{Automaten und Sprachen}
\rhead{}
\frontmatter
\newcommand\HRule{\noindent\rule{\linewidth}{1.5pt}}
\begin{titlepage}
\vspace*{\stretch{1}}
\HRule
\vspace*{10pt}
\begin{flushright}
{\Huge Automaten und Sprachen}
\end{flushright}
\HRule
\begin{flushright}
\vspace{30pt}
\LARGE
Andreas Müller
\end{flushright}
\vspace*{\stretch{2}}
\begin{center}
Hochschule für Technik, Rapperswil, 2011-2019
\end{center}
\end{titlepage}
\hypersetup{
    linktoc=all,
    linkcolor=blue
}
\rhead{Inhaltsverzeichnis}
\tableofcontents
\newtheorem{satz}{Satz}[chapter]
\newtheorem{hilfssatz}[satz]{Hilfssatz}
\newtheorem{definition}[satz]{Definition}
\newtheorem{annahme}[satz]{Annahme}
\newenvironment{beispiel}[1][Beispiel]{%
\begin{proof}[#1]%
\renewcommand{\qedsymbol}{$\bigcirc$}
}{\end{proof}}
\def\blank{\text{\textvisiblespace}}
\mainmatter
\begin{refsection}
%
% einleitung.tex
%
% (c) 2009 Prof Dr Andreas Mueller, Hochschule Rapperswil
%
\lhead{Einleitung}
\rhead{}
\chapter*{Worum geht es?\label{chapter-intro}}
Ursprünglich wurden Computer dazu entwickelt, aufwendige
Berechnungen zu automatisieren. Inzwischen wird nur noch ein
ganz kleiner Teil der weltweit installierten Rechenleistung für
diese Art von Problemen genutzt. 
Die heutige Informatik befasst sich vorwiegend mit der maschinellen
Verarbeitung von Zeichenfolgen. 

Während früher noch die Frage nach der effizientesten Lösung von
Berechnungsaufgaben mit dem Computer, also die Numerik, im Vordergrund stand,  
ist heute die Lösung komplizierterer und oft nicht numerischer Probleme 
gefragt. Zum Beispiel ist der kürzeste Weg zu finden, den ein Spediteur
zur Auslieferung aller ihm anvertrauten Sendungen verwenden kann, eine
Aufgabe, in der die Numerik offensichtlich eine untergeordnete Rolle spielt.

Trotzdem bleiben eine Reihe von Fragen, die grundsätzlich mathematischer
Art sind:
\begin{enumerate}
\item Gibt es Probleme, die mit dem Computer grundsätzlich nicht
lösbar sind?
\item Können wir darüber eine Aussage machen, wie schnell ein Computer
ein bestimmtes Problem lösen kann? Gibt es eine maximale Geschwindigkeit,
die nicht zu übertreffen ist?
\item Welchen Einfluss hat die Architektur eines Rechners auf die Klasse
von Problemen, die damit lösbar sind?
\item Gibt es Probleme, die sich besonders schnell lösen lassen?
\end{enumerate}
Die im vorliegenden Skript behandelte mathematische Theorie befasst sich
genau mit den logischen Grundlagen dieses Prozesses.
Sie definiert, welche Mengen von Zeichenfolgen überhaupt untersucht
werden sollen, und nennt sie {\em Sprachen}.
Sie studiert, welche abstrakten Typen von
Rechnern welche Arten von Zeichenfolgen erkennen oder hervorbringen
können. 

Es geht weniger darum, Methoden oder Algorithmen zur Verarbeitung
von Zeichenfolgen zu behandeln.
Vielmehr sollen mathematische Sätze aufgestellt werden, welche
Arten von Sprachen mit welcher Art von Rechnermodell bearbeitet
werden können.
Zum Beispiel werden wir Resultate wie die folgenden finden:
\begin{enumerate}
\item Ein Zustandsautomat kann die korrekte Schachtelung von
Klammern nicht erkennen.
\item Mit regulären Ausdrücken kann man korrekte arithmetische
Ausdrücke nicht erkennen.
\item Um korrekte arithmetische Ausdrücke zu erkennen, braucht man als
Speicher mindestens einen Stack.
\item Es gibt kein Programm, das ein beliebiges anderes Programm
analysieren kann und mit Sicherheit herausfinden kann, ob es je anhalten
wird.
\item Es gibt wahrscheinlich keinen Algorithmus mit polynomialer Laufzeit,
der entscheidet, ob es einen fehlerfreien Stundenplan gibt (d.\,h.~alle 
Studenten können alle Veranstaltungen besuchen, für die sie sich
angemeldet haben, es gibt keine Überschneidungen).
\item Fast alle reellen Zahlen können gar nicht berechnet werden.
\end{enumerate}
Die zu erwartenden Resultate sind also nicht von der Art von Kochrezepten:
``Um $X$ zu erhalten, mache man erst $Y$, dann $Z$ und zum Schluss $V$''.
Die Fragestellungen drehen sich alle um die fundamentale Frage:
\begin{quote}
Welche grundsätzlichen Möglichkeiten und Einschränkungen haben
Computer?
\end{quote}
Die Aussagen sind also absolute Wahrheiten, die unabhängig vom Stand
der Com\-pu\-ter-Technik oder der Programmier-Kunst gelten.
Solche Aussagen sind nur dann glaub\-würdig, wenn man sich von deren
Richtigkeit mit Hilfe eines mathematischen Beweises jederzeit überzeugen
kann. In dieser Vorlesung geht es also vor allem auch um Beweise, 
warum eine Sprache eine Eigenschaft hat, oder ob eine Eigenschaft
unter bestimmten Voraussetzungen überhaupt vorhanden sein kann.
Daher werden im ersten Kapitel kurz einige Beweistechniken rekapituliert.

Problem 5 in obiger Aufzählung kümmert sich nicht darum, ob eine
Aufgabe lösbar ist, sondern darum, wie effizient sie lösbar ist.
Dabei geht es nicht darum, wer den schnellsten Algorithmus findet,
sondern um die grundsätzliche Frage, ob es einen Algorithmus mit
einer gewissen Laufzeitkomplexität überhaupt geben kann.
Wiederum geht es um eine absolute mathematische Aussage, welche
einen Beweis erfordert. 

In dieser Vorlesung wird nichts berechnet.
In anderen
Mathematikvorlesungen werden zum Beispiel Techniken zur Berechnung
von Lösungen von
Gleichungen, zur Flächenberechnung oder zur Modellierung eines 
physikalischen Prozesses behandelt werden.
Dazu wird ein Kalkül mit einer eigenen Formelsprache entwickelt,
mit dem diese Berechnungen durchgeführt werden können.
In dieser Vorlesung werden kaum Formeln vorkommen. 
Und wenn Formeln vorkommen, dann werden es logische Formeln sein,
nicht algebraische.

Nachdem jetzt geklärt ist, was in dieser Vorlesung nicht behandelt
wird, hier einige Hinweise, was in den kommenden Kapiteln untersucht
wird:
\begin{description}
\item[Kapitel \ref{chapter-regular}:] Deterministische endliche Automaten (DEA). Von einem
DEA akzeptierte Sprache. Rekonstruktion des DEA aus der Sprache.
Reguläre Sprachen und DEAs. Reguläre Ausdrücke. Nicht deterministische
endliche Automaten (NEA). Wie beweist man, dass eine Sprache nicht regulär
ist?

In diesem Kapitel geht es nicht darum, ein Virtuose in der
Anwendung regulärer Ausdrücke zu werden. Viel wichtiger ist, Klarheit darüber
zu bekommen, in welchen Fällen reguläre Ausdrücke anwendbar sind,
und wo ihre Grenzen sind.

\item[Kapitel \ref{chapter-cfl}:] Stack-Automaten und von einem Stack-Automaten
erzeugte Sprache. Kontextfreie Grammatiken und davon erzeugte Sprachen. 
Normalform einer Grammatik. Komplexität des Parse-Problems.
Wiederum geht es nicht darum, besonders gewandt im Umgang mit Grammatiken
zu werden.

Kontextfreie Grammatiken werden zur Beschreibung von Programmiersprachen verwendet.
Trotzdem ist das Ziel dieses Abschnittes nicht Gewandtheit in der Spezifikation
von Programmiersprachen, sondern das Verständnis, welche Eigenschaften eine
Sprache haben muss, damit sie mit Hilfe einer Grammatik beschrieben werden kann.
In einem Beispiel wird auch gezeigt, wie man aus einer Grammatik mit geeigneten
Software-Werkzeugen direkt einen Parser für die von der Grammatik beschriebene
Sprache generieren kann. Wieder ist nicht die Beherrschung solcher Tools das
Ziel, dies wird in der Vorlesung Compilerbau geschehen.

\item[Kapitel \ref{chapter-turing}:] Turing Maschinen und davon erkannte Sprachen. 
Rekursiv aufzählbare Sprachen und Aufzähler.

Eine Turing Maschine ist ein stark vereinfachtes Modell eines Computers.
Alle modernen Computer sind im Prinzip nur besonders raffinierte Turing Maschinen.
Turing Maschinen sind aber einfach genug, um sich mit ihrer Hilfe darüber Gedanken
zu machen, ob und mit welcher Geschwindigkeit ein Problem mit einem
Computer gelöst werden kann. In diesem Zusammenhang müssen wir uns
darüber Gedanken machen, was es heissen soll, dass ein Problem mit
einem Computer lösbar sein soll. Dabei werden wir zeigen, dass es Probleme
gibt, die nicht mit dem Computer gelöst werden können. Allerdings wird
es uns schwerfallen, solche Probleme explizit zu beschreiben. 
Gelingen wird uns dies erst im Kapitel 6.

\item[Kapitel \ref{chapter-entscheidbarkeit}:] Entscheidbare Sprachen. Unlösbare Probleme.

Ziel dieses Kapitels ist zu lernen, wie man ein beliebiges Problem als
Sprache formulieren kann, und damit die bisher gelernte Theorie darauf
anwenden kann. Damit wird es möglich, konkrete und für die Praxis eines
Informatikers wichtige Probleme zu beschreiben, die
nicht mit Computern gelöst werden können.
Es ist zum Beispiel unmöglich, Programme maschinell daraufhin
zu prüfen, ob sie je anhalten werden.
Ausserdem lernen wir eine Technik kennen, mit der wir Probleme 
nach ``Schwierigkeitsgrad'' miteinander vergleichen können. 

\item[Kapitel \ref{chapter-komplexitaet}:] Komplexitätstheorie. In diesem Kapitel sollen 
die prinzipiellen Grenzen der Geschwindigkeit ermittelt werden,
mit der ein Problem mit dem Computer gelöst werden kann. Es wird sich
herausstellen, dass es zwar viele Probleme gibt, die mit aktuellen Computern in
vernünftiger Zeit lösbar sind, dass aber auch eine Klasse von Aufgaben
existiert, für die es nach aktuellem Wissen keinen Algorithmus geben kann,
der die Aufgabe bei etwas grösseren Problemen lösen kann. Diese Probleme sind
zwar im Prinzip lösbar, aber alle Rechenleistung der Welt kann die Probleme nicht lösen.

\item[Kapitel \ref{chapter-vollstaendigkeit}:]
Turing-Maschinen haben eine universelle Eigenschaft: eine Turing-Maschine
kann jede andere Turing-Maschine simulieren. Doch in der Praxis
programmiert niemand direkt eine Turing-Maschine, dazu werden
spezialisierte Programmiersprachen verwendet.
Welche Eigenschaften muss eine Sprache haben, damit
man all das machen kann, was man mit einer Turing-Maschine
machen kann?
\end{description}



%
% grundlagen.tex
%
% (c) 2009 Prof Dr Andreas Mueller
%
\chapter{Grundlagen\label{chapter-grundlagen}}
\lhead{Grundlagen}
\rhead{Notation}
\section{Notation}
In diesem Abschnitt stellen wir eine Reihe bereits bekannter Notationen
zusammen.
\subsection{Logik}
\index{Logik}
\index{Pr\ädikatenlogik}
\subsubsection{Prädikate}
Wir verwenden die Prädikatenlogik.
\index{Pr\ädikat}
{\em Prädikate} sind formale Aussagen über mathematische Objekte, die auch
variable Teile enthalten können. Prädikate sind also ``Funktionen'' mit
Wahrheitswerten als Rückgabewerten.
\begin{align*}
P(x,y)&=x < y&&\text{wahr falls $x$ kleiner als $y$}\\
P(n)&=n \equiv 0\mod 2&&\text{wahr falls $n$ gerade}
\end{align*}
Prädikate können durch logische Verknüpfungen zu neuen Aussagen kombiniert
werden:
\begin{center}
\begin{tabular}{|l|c|c|l|}
\hline
Name&Verknüpfung&Zeichen&Bedeutung\\
\hline
\index{Konjunktion}
Konjunktion&UND&$P\wedge Q$&wahr falls sowohl $P$ als auch $Q$ wahr sind\\
\index{Disjunktion}
Disjunktion&ODER&$P\vee Q$&wahr falls $P$ oder $Q$ (oder beide) wahr sind\\
\index{Negation}
Negation&NICHT&$\neg P$&wahr falls $P$ nicht wahr ist\\
\hline
\end{tabular}
\end{center}
Die Kombination $\neg P\vee Q$ ist also wahr, wenn $P$ nicht
wahr ist, oder falls $P$ wahr ist, auf jeden Fall auch $Q$
wahr ist. Die Aussage, dass $\neg P\vee Q$ wahr sei, heisst also
nichts anderes, als dass $Q$ folgt, wenn $P$ wahr ist. Daher schreibt
man
\[
\neg P\vee Q\qquad =\qquad P\Rightarrow Q.
\]

Hat man mit einer grossen Zahl von Prädikaten $P_1,P_2,\dots$ zu tun,
dann kann man deren Konjunktion oder Disjunktion ähnlich wie beim Summenzeichen
mit einem ``grossen'' Verknüpfungszeichen schreiben:
\begin{align*}
\bigwedge_{i=1}^n P_i&=P_1\wedge P_2\wedge P_3\wedge\dots\wedge P_n\\
\bigvee_{i=1}^n P_i&=P_1\vee P_2\vee P_3\vee\dots\vee P_n
\end{align*}
Die Indizes müssen nicht eine fortlaufende Folge bilden, sie können
auch aus einer beliebigen Indexmenge $I$ stammen, wofür man dann
schreibt
\[
\bigwedge_{i\in I}P_i
\qquad
\text{bzw.}
\qquad
\bigvee_{i\in I}P_i
\]

Die beiden Operationen $\wedge$ und $\vee$ sind miteinander verträglich,
es gelten die {\em Distributiv-Gesetze}
\begin{align*}
P\wedge(Q\vee R)&=(P\wedge Q)\vee (P\wedge R)\\
P\vee(Q\wedge R)&=(P\vee Q)\wedge (P\vee R).
\end{align*}
\index{Distributivgesetz}

\subsubsection{Normalformen}
\index{Normalform}
\index{Normalform!konjunktive}
Durch wiederholte Anwendung der Distributivgesetze kann man 
eine Formel in eine von zwei Normalformen bringen, je nachdem
ob man die ``$\wedge$'' oder die ``$\vee$'' nach aussen bringt.
Sind die ``äusseren'' Verknüpfungen die ``$\wedge$'', spricht
man von {\em konjunktiver Normalform}, eine solche Formel sieht so aus:
\[
(x_1\vee x_3)\wedge(x_2\vee \bar x_4\vee x_5)\wedge\dots
\]
Die Klammerausdrücke heissen {\em Klauseln}.
\index{Klausel}

Bringt man stattdessen die ``$\vee$'' nach aussen, spricht man von
{\em disjunktiver Normalform}, also eine Formel der Form
\index{Normalform!diskjunktive}
\[
(x_1\wedge x_3)\vee(x_2\wedge \bar x_4\wedge x_5)\vee\dots
\]

Die Umwandlung zwischen den Normalformen kann sehr verschwenderisch sein.
Die Formel
\[
(x_1\wedge y_1)\vee(x_2\wedge y_2)\vee\dots\vee (x_n\wedge y_n)
\]
in disjunktiver Normalform wird bei dieser Art von Umwandlung zu
\begin{equation}
(x_1\vee x_2\vee\dots\vee x_{n-1}\vee x_n)
\wedge
(y_1\vee y_2\vee\dots\vee y_{n-1}\vee y_n)
\wedge
\dots 
=\bigwedge (z_1\vee z_2\vee\dots \vee z_{n-1}\vee z_n),
\label{bigcnf}
\end{equation}
wobei $(z_1,\dots,z_n)$ eine beliebige Kombination $z_i=x_i$ oder $z_i=y_i$
ist. Die Formel (\ref{bigcnf}) hat also $2^n$ Klauseln.

\subsubsection{de Morgansche Regeln}
\index{de Morgansche Regeln}
Die de Morganschen Regeln erlauben, ODER in UND zu verwandeln, indem
man die einzelnen Terme negiert:
\begin{align}
P\wedge Q&= \neg(\neg P\vee\neg Q)\notag\\
P\vee Q&= \neg(\neg P\wedge\neg Q)\notag\\
P\Rightarrow Q&=\neg P\vee Q=\neg\neg(\neg P\vee Q)=\neg(P\wedge\neg Q)=\neg\neg Q\vee \neg P\notag\\
&=\neg Q\Rightarrow \neg P\label{kontraposition}
\end{align}
\index{Kontraposition}
Die letzte Formel heisst auch die Kontraposition, sie ist der Kern
eines Beweises mit Widerspruch (siehe Abschnitt \ref{widerspruchsbeweis})


\subsubsection{Quantoren}
\index{Quantor}
Prädikate mit mindestens einer freien Variablen können mit Hilfe
von Quantoren zu neuen Präkdikaten zusammengebaut werden. 
Die Aussage $n^2\ge 0$ ist zum Beispiel für alle natürlichen
Zahlen richtig, man schreibt dafür
\[
\forall n\in\mathbb N(n^2\ge 0).
\]
\index{All-Quantor}
Das Zeichen $\forall$ wird ``für alle'' gelesen, es heisst
{\em All-Quantor}.
\index{Existenz-Quantor}
Jede positive reelle Zahl $x$ hat eine Wurzel, es gibt also eine
Zahl $w$ mit der Eigenschaft $x=w^2$. Formal drückt man dies mit
dem {\em Existenz-Quantor} $\exists$ aus:
\[
\exists w\in\mathbb R(x =w^2)
\]
Dies liest man ``es gibt eine reelle Zahl $w$, deren Quadrat $x$ ist.
Will man obigen Satz, dass jede positive relle Zahl eine Wurzel hat,
formal ausdrücken, schreibt man
\[
\forall x>0\exists w\in\mathbb R(x=w^2)
\]

Zwei Aussagen $P$ und $Q$ sind {\em äquivalent}, wenn sowohl
$P\Rightarrow Q$ als auch $Q\Rightarrow P$, kurz $P\Leftrightarrow Q$.

\subsection{Mengenlehre}
\index{Mengenlehre}
Wir gehen von einem intuitiven Mengenverständnis aus, und bezeichnen
Mengen meistens mit grossen Buchstaben $A$, $B$, $C$ usw.
Aus einer Menge $A$ kann eine neue Menge gebildet werden, indem
nur die Elemente ausgewählt werden, die eine bestimmte Eigenschaft
$E$ haben:
\[
\{a\in A\;|\;E(a)\}
\]
Die Menge der $Q$ der Quadratzahlen besteht zum Beispiel aus den
Zahlen $n$, für
die es eine andere Zahl $w$ gibt (die Wurzel), mit der Eigenschaft
$n=w^2$.  In Formeln:
\[
Q=\{n\; |\; \exists m\in\mathbb N(n=m^2)\}
\]

Sind $A$ und $B$ zwei Mengen, dann können wir daraus neue Mengen
bilden:
\begin{center}
\begin{tabular}{|l|c|l|}
\hline
Operation&Zeichen&Bedeutung\\
\hline
\index{Durchschnitt}
Durchschnitt&$A\cap B$&Enthält die Elemente, die in $A$ {\bf und} $B$ sind\\
\index{Vereinigung}
Vereinigung&$A\cup B$&Enthält die Elemente, die in $A$ {\bf oder} $B$ sind\\
\index{Komplement}
Komplement&$\bar A$&Enthält die Elemente, die {\bf nicht} in $A$ sind\\
\index{Differenz}
Differenz&$A\setminus B$&Enthält die Elemente aus A, die nicht in $B$ sind\\
\hline
\end{tabular}
\end{center}
\index{symmetrische Differenz}
Für die Anwendungen wichtig ist ausserdem die {\em symmetrische Differenz}
zweier Mengen: 
\[
A{\;\Delta\;}B = (A\setminus B)\cup (B\setminus A).
\]
Die symmetrische Differenz verschwindet genau dann, wenn die beiden
Mengen gleich sind:
\[
A{\;\Delta\;}B = \emptyset
\quad\Leftrightarrow\quad
A=B.
\]
\begin{figure}
\begin{center}
\includegraphics[width=.4\hsize]{images/turing-4}
\end{center}
\caption{Symmetrische Differenz der Mengen $A$ und $B$ (schraffiert)
\label{symdiff}}
\end{figure}

Die folgenden Zahlmengen sind wohlbekannt und wir werden sie ebenfalls
oft verwenden:
\begin{center}
\begin{tabular}{|c|l|}
\hline
Symbol&Beschreibung\\
\hline
\index{nat\ürliche Zahlen}
$\mathbb N$&Menge der natürlichen Zahlen $\{0,1,2,\dots\}$\\
$\mathbb N^*$&Menge der positiven natürlichen Zahlen $\{1,2,3,\dots\}$\\
\index{ganze Zahlen}
$\mathbb Z$&Menge der ganzen Zahlen $\{\dots,-2,-1,0,1,2,\dots\}$\\
\index{rationale Zahlen}
$\mathbb Q$&Menge der rationalen Zahlen $\{\frac{p}{q}|p\in\mathbb Z,q\in\mathbb N^*\}$\\
\index{reelle Zahlen}
$\mathbb R$&Menge der reellen Zahlen\\
\index{leere Menge}
$\emptyset=\{\}$&leere Menge\\
\hline
\end{tabular}
\end{center}
Daraus lassen sich weitere Mengen konstruieren:
\begin{align*}
\mathbb Z^*&=\mathbb Z\setminus \{0\}\\
\mathbb Q^*&=\mathbb Q\setminus \{0\}\\
\mathbb R^*&=\mathbb R\setminus \{0\}\\
\mathbb R_{> 0}&=\mathbb R^+=\{x\in\mathbb R\,|\,x>0\}\\
[n]&=\{x\in \mathbb N^*\,|\,x\le n\}
\end{align*}
\index{Machtigkeit@M\ächtigkeit}
Enthält eine Menge $A$ nur endlich viele Elemente, schreiben wir die Anzahl
ihrer Elemente $|A|$, sie heisst auch die {\em Mächtigkeit} von $A$.
\[
|\{n\in\mathbb N\,|\,\text{$n$ prim}\wedge n<10\}|=|\{2,3,5,7\}|=4.
\]
Die Mengen $[n]$ enthalten genau $n$ Elemente, $|[n]|=n$.

Enthält eine Menge $B$ alle Elemente von $A$, sagt man, $A$ sei in $B$
enthalten, schreibt dafür $A\subset B$ und sagt, $A$ sei eine
{\em Teilmenge} von $B$. Ist $A\subset B$ und auch
$B\subset A$, dann enthalten $A$ und $B$ die gleichen Elemente, also
$A=B$.

\index{Potenzmenge}
Die Menge aller Teilmengen von $A$ heisst die {\em Potenzmenge} $P(A)$ von $A$:
\[
P(A)=\{ T\,|\,T\subset A\}
\]
Die Potenzmenge einer endlichen Anzahl Elemente $A$ kann man wie folgt bilden.
Um eine Teilmenge von $A$ zu bilden, muss man für jedes Element
von $A$ entscheiden, ob es in die Teilmenge kommen soll oder nicht.
Man hat also für jedes der $|A|$ Elemente eine Entscheidung mit zwei
möglichen Ausgängen zu fällen, was auf
\[
\underbrace{2\cdot\dots\cdot 2}_{\text{$|A|$ Faktoren}}=2^{|A|}
\]
möglich ist, also $|P(A)|=2^{|A|}$.

\subsection{Paare und Tupel}
\index{Paar}
\index{Tupel@$n$-Tupel}
\index{kartesisches Produkt}
Die Beschreibung eines Punktes in der Ebene erfordert die Angabe zweier
Koordinaten, die man üblicherweise als Paar $(x,y)$ schreibt. Allgemein
kann man aus den Elementen $a\in A$ und $b\in B$ zweier Mengen $A$ und $B$
die Menge der Paare (2-Tupel) bilden, diese Menge heisst das
{\em kartesische Produkt} der beiden Mengen:
\[
A\times B = \{(a,b)\,|\,a\in A\wedge b\in B\}.
\]
\index{Tripel}
Analog kann man aus drei Mengen die Menge aller Tripel
\[
A\times B\times C=\{(a,b,c)\,|\,a\in A\wedge b\in B\wedge c\in C\}
\]
bilden, oder für $n$ Mengen die Menge der $n$-Tupel
\[
A_1\times A_2\times\dots\times A_n
=\{(a_1,a_2,\dots,a_n)\,|a_1\in A_1\wedge a_2\in A_2\wedge\dots\wedge a_n\in A_n\}.
\]
Falls die Faktormengen alle identisch sind, also $A_1=A_2=\dots=A_n=A$
schreiben wir auch
\[
A^n= \underbrace{A\times \dots \times A}_{\text{$n$ Faktoren}}
\]
für das kartesische Produkt von $n$ Faktoren $A$.

\subsection{Relationen}
\index{Relation}
Eine zweistellige {\em Relation} $R$ ist ein Spezialfall eines Prädikates.
Realisieren könnte man eine Relation dadurch, dass man die Menge
aller Paare bildet, für die die Relation erfüllt ist. Um zu entscheiden,
ob $xRy$ gilt, muss man also nur noch in der Menge nachschauen,
ob $(x,y)$ dort drin ist. Man kann also die Relation mit der
Menge
\[
\{(x,y)\in A\times B\,| xRy\}
\]
identifizieren.
\index{Graph!einer Relation}
Diese Menge heisst auch der {\em Graph} der Relation.

Besonders wichtig sind {\em Äquivalenzrelationen}, die sich durch folgende
Eigenschaften auszeichnen.
\begin{compactenum}
\item $R$ ist {\em reflexiv}, falls für jedes $x$ gilt $xRx$: $\forall x(xRx)$
\item $R$ heisst {\em symmetrisch}, falls mit $xRy$ auch $yRx$ gilt: $\forall x\forall y(xRy\Rightarrow yRx)$
\item $R$ heisst {\em transitiv}, falls mit $xRy$ und $yRz$ auch $xRz$ gilt.
\end{compactenum}
Beispiele von Äquivalenzrelationen sind:
\begin{itemize}
\item Gleichheit
\item $xRy$ für $x,y\in\mathbb N$ falls $x$ und $y$ den gleichen
Rest bei Teilung durch eine feste Primzahl $p$ haben.
\item $xRy$ für Dreiecke $x$ und $y$, falls $x$ und $y$ kongruent sind.
\end{itemize}

\subsection{Funktionen, Abbildungen}
\index{Funktion}
\index{Abbildung}
Eine {\em Funktion} oder {\em Abbildung} $f\colon A\to B$ ist
eine Zuordnung, die
jedem Element von $A$ genau ein Element von $B$ zuordnet.
\index{Bild}
\index{Urbild}
Das Element
$b=f(a)$ heisst das {\em Bild} von $a$, $a$ heisst ein {\em Urbild} von $b$. Ein
Element in $b$ kann mehrere Urbilder haben.

Eine Funktion ist eine spezielle Relation: 
$a$  und $b$ stehen in der Relation zueinander, wenn $f(a)=b$. 
\index{Graph!einer Abbildung}
Die zugehörige Menge von Paaren ist
\[
G(f)=\{(a,b)\,|\,b=f(a)\},
\]
und heisst der {\em Graph}. Ist $A=\mathbb R$, $B=\mathbb R$ und
$f\colon\mathbb R\to\mathbb R$, dann ist $G(f)$ die Menge
der Punkte $(x,y)$ in der Ebene, die $y=f(x)$ erfüllen, also
genau der Graph im üblichen Sinne.

\index{Menge aller Abbildungen}
Die Menge aller Abbildungen von $A$ nach $B$ schreibt man auch
$B^A$. Die Notation wird verständlich, wenn man für endliche
Mengen $A$ und $B$ zählt, wieviel
Abbildungen zwischen $A$ und $B$ es gibt. Um eine Abbildung von
$A$ nach $B$ zu konstruieren, muss man für jedes der $|A|$ Elemente von $A$
eines der $|B|$ Elemente von $B$ auswählen. Man muss also $|A|$ mal
eine Auswahl mit $|B|$ Möglichkeiten treffen, man hat also
\[
\underbrace{|B|\cdot\dots\cdot|B|}_{\text{$|A|$ Faktoren}}=|B|^{|A|}
\]
Abbildungen.

Die Elemente der Potenzmenge von $A$ entstehen dadurch, dass man für jedes
Element von $A$ einen Wert $0$ oder $1$ auswählen muss, $0$ gibt an,
dass das Element nicht in die Teilmenge kommt, $1$ gibt an, dass es dazugehört.
Eine Teilmenge von $A$ entspricht also genau einer Abbildung $A\to\{0,1\}$,
man kann die Potenzmenge mit der Menge der Abbildungen $A\to\{0,1\}$
identifizieren:
\[
P(A) = \{0,1\}^{A}.
\]

Ein $n$-Tupel von Elementen aus $A$ ordnet jedem der Plätze
$1,\dots,n$ genau ein Element aus $A$ zu, ein Tupel ist also eigentlich
eine Abbildung $[n]=\{1,\dots,n\}\to A$, die Menge der Tupel ist somit
\[
A^n=A^{\{1,\dots,n\}}=A^{[n]}.
\]

\subsection{Graphen}
\index{Kante}
\index{Ecke}
\index{Vertex}
Ein Graph besteht aus Ecken (Vertizes) und Kanten,
die die Ecken verbinden.
Dabei
soll es aber anders als zum Beispiel bei einem Verkehrsnetz zwischen zwei
Ecken immer nur eine Verbindung geben können. Die Kante ist also durch
die beiden Endpunkte vollständig bestimmt, ausserdem spielt deren
Reihenfolge keine Rolle. Um eine Kante zu beschreiben, braucht man
also nur die Menge $\{a,b\}$ der Endpunkte zu kennen.
Wir fassen das in folgende Definition zusammen.

\begin{definition}
\index{Graph}
\label{def_graph}
Ein Graph ist ein Paar $(V,E)$ bestehend aus einer Menge $V$ von Ecken
(Vertices),
und einer Menge $E$ von zweielementigen Teilmengen von $V$, die Kanten
genannt werden.
\end{definition}

\begin{figure}
\begin{center}
\includegraphics{images/graph-1.pdf}
\end{center}
\caption{Graph\label{grundlagen:graph}}
\end{figure}
Die Abbildung\ref{grundlagen:graph} zeigt einen Graphen mit Vertex-Menge
\[
V=\{v_0,
v_1,
v_2,
v_3,
v_4,
v_5,
v_6\},
\]
und Kanten
\begin{align*}
e_0&=\{v_0,v_4\},&
e_1&=\{v_0,v_3\},&
e_2&=\{v_0,v_1\},&
e_3&=\{v_2,v_4\},
\\
e_4&=\{v_1,v_3\},&
e_5&=\{v_2,v_3\},&
e_6&=\{v_1,v_2\},&
e_7&=\{v_1,v_5\},
\\
e_8&=\{v_1,v_6\},&
e_9&=\{v_2,v_5\},&
e_{10}&=\{v_5,v_6\}.
\end{align*}
Die Menge $E$ besteht also aus den Elementen $e_1,\dots,e_{10}$.

\begin{figure}
\begin{center}
\includegraphics{images/graph-2.pdf}
\end{center}
\caption{Kein Graph, jede Kante eines Graphen hat zwei verschiedene
Ecken.\label{grundlagen:keingraph}}
\end{figure}
Man beachte, dass ein Graph keine Kante haben kann, die zurück zum 
selben Vertex führen kann.
Das Gebilde in Abbildung~\ref{grundlagen:keingraph}
ist also kein Graph, weil jede Kante eines Graphen zwei verschiedene
Ecken hat.

\index{Grad}
Der {\em Grad} einer Ecke in einem Graphen ist die Anzahl der Kanten,
die von dieser Ecke ausgehen.

\index{Pfad}
Ein {\em Pfad} in einem Graphen ist eine Folge von Ecken, die durch Kanten
verbunden sind.
\index{Pfad!einfacher}
Ein {\em einfacher Pfad} ist ein Pfad, der keine Ecke mehr
als einmal trifft.
\index{Zyklus}
Ein {\em Zyklus} ist ein Pfad, der mit der gleichen Ecke
endet, mit der er begonnen hat.
\index{Zyklus!einfacher}
Ein einfacher Zyklus ist ein Zyklus,
der keine Ecke zweimal besucht.

\index{Baum}
Ein {\em Baum} ist ein Graph ohne Zyklen. Ein Baum kann einen ausgezeichneten
Punkt, die {\em Wurzel} des Baumes, enthalten. Alle anderen Ecken vom Grad 1
\index{Blatt}
heissen Blätter des Baumes.

\begin{figure}
\begin{center}
\includegraphics{images/graph-3.pdf}
\end{center}
\caption{Gerichteter Graph\label{grundlagen:gerichtetergraph}}
\end{figure}
Möchte man die Richtung der Kanten berücksichtigen,
hat man mit einem gerichteten
Graphen zu tun (Abbildung~\ref{grundlagen:gerichtetergraph}).
Da es jetzt auf die Reihenfolge der Endpunkte an kommt,
sich insbesondere die Kante von $a$ nach $b$ von der Kante von $b$ nach
$a$ unterscheidet, können wir die Kanten nicht mehr durch die ungeordneten
Mengen beschreiben, sondern müssen geordnete Tupel verwenden.

\begin{definition}
\index{Graph!gerichteter}
\label{def_gerichteter_graph}
Ein gerichteter Graph ist ein Paar $(V,E)$ bestehend aus einer
Mengen $V$ von Ecken (Vertizes) und einer Menge von Paaren $E\subset V\times V$,
den Kanten.
\end{definition}

Die Kanten des Graphen in Abbildung~\ref{grundlagen:gerichtetergraph}
sind Paare
\begin{align*}
e_0&=(v_0,v_4)
&
e_1&=(v_0,v_3)
&
e_2&=(v_0,v_1)
&
e_3&=(v_4,v_2)
\\
e_4&=(v_1,v_3)
&
e_5&=(v_2,v_3)
&
e_6&=(v_1,v_2)
&
e_7&=(v_5,v_1)
\\
e_8&=(v_1,v_6)
&
e_9&=(v_5,v_2)
&
e_{10}&=(v_5,v_6)
\end{align*}

\begin{figure}
\begin{center}
\includegraphics{images/graph-4.pdf}
\end{center}
\caption{Gerichteter beschrifteter Graph\label{grundlagen:beschrgraph}}
\end{figure}
Ein Verkehrsnetz wie auch die später zu definierenden endlichen
Automaten ist aber noch komplizierter: auf den Verbindungen zwischen
den Netzwerken sind auch verschiedene Bahnlinien im Einsatz, so gibt
es zwischen Rapperswil und Pfäffikon zum Beispiel je eine Kante, die
mit ``S5'' bzw.~mit ``Voralpenexpress'' angeschrieben ist
(Abbildung~\ref{grundlagen:beschrgraph}). Dies
wird von folgender Definition eingefangen:

\begin{definition}
\label{def_gerichteter_beschrifteter_graph}
\index{Graph!gerichteter!beschrifteter}
Ein gerichteter beschrifteter Graph ist ein Tripel $(V,E,L)$ bestehend
aus einer Menge $V$ von Ecken, einer Menge $L$ von Beschriftungen (Labels) und 
einer Menge $E$ von Tripeln $(a,e,l)\in V\times V\times L$. Das
Tripel $(a,e,l)$ heisst Kante von $a$ nach $e$ mit Beschriftung $l$.
\end{definition}

Der in Abbildung~\ref{grundlagen:beschrgraph} abgebildete Graph
hat die Kantenmenge 
\begin{align*}
E=\{\;
&
(v_0,v_1,\text{\tt Fussweg}),\;
(v_0,v_3,\text{\tt ICE}),\;
(v_0,v_4,\text{\tt NCC 1701-D}),\;
\\&
(v_1,v_2,\text{\tt S5}),\;
(v_1,v_3,\text{\tt S7}),\;
(v_1,v_6,\text{\tt Taxi}),\;
(v_2,v_3,\text{\tt S5}),\;
\\&
(v_3,v_0,\text{\tt ICE}),\;
(v_5,v_1,\text{\tt S5}),\;
(v_5,v_1,\text{\tt S7}),\;
(v_5,v_6,\text{\tt Bus})\;
\}
\end{align*}
Man beachte, dass es zwei verschieden beschriftete Kanten gibt
von $v_5$ nach $v_1$, und dass die Richtung der beiden gleich
beschrifteten Kanten zwischen $v_0$ und $v_3$ verschieden ist.
Beides Möglichkeiten, die in einem Graphen verboten sind.

\rhead{Beweistechniken}
\section{Beweistechniken}
Da in dieser Vorlesung Beweise zentral sind, hier drei grundlegende 
Techniken zur Erinnerung.

\subsection{Konstruktion}
\index{Beweis!konstruktiver}
Ein konstruktiver Beweis gibt einfach eine Konstruktion an, die
das behauptete Objekt liefert oder für die die behauptete Eigenschaft
offensichtlich ist.
Ein solches Vorgehen wird auch oft eine Herleitung genannt.

\begin{satz}
\index{Gleichung!quadratische}
Falls $a\ne0$ und $b^2-4ac>0$ hat die quadratische Gleichung
\[
ax^2+bx+c=0
\]
zwei verschiedene Lösungen.
\end{satz}

\begin{proof}[Beweis]
Man kann die quadratische Gleichung mit vollständigem Ergänzen wie
folgt umformen:
{
\allowdisplaybreaks
\begin{align*}
ax^2+bx+c&=0\\
x^2+2\cdot \frac{b}{2a} x +\frac{c}a&=0\\
x^2+2\cdot \frac{b}{2a} x 
+\left(\frac{b}{2a}\right)^2
+\frac{c}a&=0
+\left(\frac{b}{2a}\right)^2\\
\left(x+\frac{b}{2a}\right)^2 &= \left(\frac{b}{2a}\right)^2 -\frac{c}a \\
\left(x+\frac{b}{2a}\right)^2 &=
\frac{b^2-4ac}{4a^2}\\
x+\frac{b}{2a}&=\frac{\pm\sqrt{b^2-4ac}}{2a}\tag{*}\\
x_{\pm}&=\frac{-b\pm\sqrt{b^2-4ac}}{2a}
\end{align*}
}
Die Voraussetzung $a>0$ wird im ersten Schritt dieser Ableitung verwendet.
Da $b^2-4ac>0$ kann im Schritt (*) tatsächlich die Wurzel gezogen werden,
und die beiden Wurzeln sind auch verschieden. Man hat also zwei verschiedene
Lösungen konstruiert, womit die Behauptung bewiesen ist.
\end{proof}

\subsection{Widerspruch\label{widerspruchsbeweis}}
\index{Beweis!mit Widerspruch}
\index{Widerspruch}
In der Mathematik darf es keine Widersprüche geben, denn aus einem
Widerspruch liesse sich jede beliebige Aussage ableiten. Wäre zum
Beispiel $P$ eine Aussage, die sowohl wahr wie auch falsch ist,
dann hätte die Menge
\[
\{x\in\{P\}\,|\, \text{$x$ ist wahr}\}
\]
je nachdem ob $P$ nun wahr oder falsch ist $1$ oder $0$ Elemente.
Da das aber natürlich immer die gleiche Menge ist, müsste man
folgern $0=1$. Daraus lässt sich dann ableiten, dass alle natürlichen
Zahlen gleich sind (zum Beispiel mit vollständiger Induktion, siehe
unten), dass alle Mengen gleich viele Elemente haben, nämlich gar
keine, dass es die ganze Mathematik also gar nicht gibt.

Wenn man also eine Aussage $P$ beweisen will, kann man annehmen,
dass $P$ falsch ist, oder $\neg P$ wahr ist. Wenn daraus jetzt
eine Aussage $Q$ folgt, von der wir bereits wissen, dass sie falsch
ist, dann bedeutet das, dass die Annahme $\neg P$ nicht haltbar
war.

Und noch etwas formaler: wenn $\neg P\Rightarrow \neg Q$, dann können
wir das umformen zu $\neg Q\Rightarrow P$ (nach Kontrapositionsformel
(\ref{kontraposition})), wir wissen bereits, dass
$Q$ falsch ist, die linke Seite also wahr ist, also muss auch $P$
wahr sein.

Das klassische Beispiel für einen Beweis mit Widerspruch ist das 
folgende:
\begin{satz}$\sqrt{2}$ ist irrational: $\sqrt{2}\not\in\mathbb Q$.
\index{irrational}
\end{satz}
\begin{proof}[Beweis]
Wir nehmen $\sqrt{2}\in\mathbb Q$ und führen dies zu einem
Widerspruch. Wenn $\sqrt{2}\in\mathbb Q$ ist, kann man $\sqrt{2}$
als gekürzten Bruch $\frac{p}{q}$ schreiben. Dann gilt aber
auch
\[
\sqrt{2}^2=2=\frac{p^2}{q^2}\quad\Rightarrow\quad p^2=2q^2.
\]
Wenn $p$ den Primfaktor $2$ enthält, dann enthält $p^2$
eine gerade Anzahl von Primfaktoren $2$. Ebenso enthält
$q^2$ eine gerade Anzahl von Primfaktoren $2$, also $2q^2$
eine ungerade Anzahl von Primfaktoren $2$.
Im Ausdruck $p^2=2q^2$ steht also links eine Zahl mit einer
geraden Anzahl von Primfaktoren $2$, und rechts eine Zahl mit
einer ungeraden Anzahl von Primfaktoren $2$.
Dieser Widerspruch zeigt, dass die
Annahme $\frac{p}q\in\mathbb Q$ nicht zu halten ist.
\end{proof}

\subsection{Induktion}
\index{Beweis!mit Induktion}
\index{Induktion}
Möchte man eine Aussage mit einem natürlichen Parameter $n$ für jeden
möglichen Wert von $n$ beweisen, kann man dazu vollständige Induktion
verwenden. Dazu geht man wie folgt vor:
\begin{description}
\index{Verankerung}
\item[Verankerung:] Beweise die Aussage für den kleinsten Wert,
für den sie bewiesen werden soll.
\index{Induktionsannahme}
\item[Induktionsannahme:] Nehme an, für einen bestimmten Wert
$n$ sei die Aussage bereits bewiesen (nach der Verankerung ist sie 
für den kleinsten möglichen Wert ja tatsächlich bereits bewiesen).
\index{Induktionsschritt}
\item[Induktionsschritt:] Beweise jetzt die Aussage für den Wert $n+1$.
\end{description}

Als Beispiel soll der folgene Satz dienen:

\begin{satz} Für alle natürlichen Zahlen $n$ gilt
\[
\sum_{k=1}^nk=\frac{n(n+1)}2
\]
\end{satz}
\begin{proof}[Beweis]
\begin{description}
\item[Verankerung:] für $n=0$ ist die Summe leer, ergibt also $0$.
Auf der rechten Seite steht $0(0+1)/2=0$, die Formel trifft also zu.
\item[Induktionsannahme:]Wir nehmen also an, dass
\[
\sum_{k=1}^nk=\frac{n(n+1)}2
\]
\item[Induktionsschritt:] Wir müssen die Behauptung für $n+1$ zeigen. Dazu
berechnen wir
\begin{align}
\sum_{k=1}^{n+1}k&=\biggl(\sum_{k=1}^nk\biggr) + (n+1)\notag\\
&=\frac{n(n+1)}2+(n+1)\label{verwendung_annahme}\\
&=\frac{n(n+1)+2(n+1)}2\notag\\
&=\frac{(n+1)(n+2)}2=\frac{(n+1)((n+1)+1)}2\notag
\end{align}
In Schritt (\ref{verwendung_annahme}) haben wir die Induktionsannahme
verwendet. In der letzten Zeile auf der rechten Seite steht tatsächlich
die Formel des Satzes, in der $n$ durch $n+1$ ersetzt worden ist.
Damit ist die Formel für $n+1$ bewiesen.
\end{description}
\end{proof}

%
% sprachen.tex
%
% (c) 2011 Prof Dr Andreas Mueller, Hochschule Rapperswil
%
\chapter{Sprachen\label{chapter-sprachen}}
\lhead{Sprachen}
Diese Vorlesung betrachtet Computer in erster Linie als
Maschinen, die Zeichenketten verarbeiten.
Natürlich wird man
nicht jede Zeichenkette als sinnvollen Input oder Output betrachten,
nur eine Teilmenge aller möglichen Zeichenketten würden wir
eine ``Sprache'' nennen.
Diese intuitive Vorstellung wollen wir
jetzt formalisieren.

\index{Alphabet}%
Die Basis einer Sprachdefinition muss die Auswahl eines geeigneten
Alphabetes sein.
Während in der Informatik das Alphabet meist 
durch die Maschine vorgegeben ist (sie verarbeitet zum Beispiel
einzelne Bytes, also Zahlen zwischen 0 und 255), möchten wir für
unsere theoretischen Überlegungen mehr Freiheit, und akzeptieren
jede beliebige nichtleere endliche Menge als Alphabet.
Alphabete
bezeichnen wir häufig mit grossen griechischen Buchstaben, zum
Beispiel
\begin{align*}
\Sigma&=\{{\tt 0},{\tt 1}\}\\
\Sigma&=\{{\tt 1}\}\\
\Gamma&=\{{\tt 0},{\tt 1},\blank\}\\
\Delta&=\{{\tt a},\dots,{\tt z}\}
\end{align*}
Zeichenketten sind Tupel aus Elementen eines Alphabets.
\begin{definition}\label{def_wort}
\index{Wort}%
\index{leeres Wort}%
Ein Wort $w$ der Länge $n$ über dem Alphabet $\Sigma$ ist ein $n$-Tupel,
$w\in\Sigma^n$.
Es gibt genau ein Wort der Länge $0$, es heisst das
leere Wort und wird mit $\varepsilon$ bezeichnet, $\Sigma^0=\{\varepsilon\}$.
Die Menge aller Wörter
ist die Vereinigung aller $\Sigma^n$ und wird mit $\Sigma^*$ bezeichnet:
\[
\Sigma^*=\{\varepsilon\}\cup \Sigma^1\cup\Sigma^2\cup\Sigma^3\cup\dots
=\bigcup_{k=0}^\infty\Sigma^k.
\]
\end{definition}

Natürlich kann man von jedem Wort $w\in\Sigma^*$ auch wieder bestimmen,
aus welchem $\Sigma^k$ es stammt, indem man die Länge der Zeichenkette
auszählt.
Wir bezeichnen die Länge von $w$ mit $|w|$.
Es ist $|{\tt 1111}|=4$ und $|\varepsilon|=0$.
Manchmal ist es wichtig zu wissen, wie oft ein bestimmtes
Zeichen in einem Wort vorkommt.
Wir bezeichnen mit
$|w|_{a}$ die Anzahl Vorkommnisse des Zeichens $a$, also zum Beispiel
$|{\tt 0010111}|_{\tt 0}=3$ und $|{\tt 0110011}|_{\tt 1}=4$.

Eine Sprache muss nicht alle möglichen Wörter umfassen, oft wird sogar
nur eine endliche Menge von Wörtern aus der unendlichen Menge $\Sigma^*$
ausgewählt.

\begin{definition}
\index{Sprache}%
Eine Sprache $L$ über dem Alphabet $\Sigma$ ist eine Teilmenge
von $\Sigma^*$, also $L\subset \Sigma^*$.
Über jedem Alphabet
gibt es die leere Sprache $\emptyset$ und die Sprache, die nur
aus dem leeren Wort besteht $\{\varepsilon\}$.
\end{definition}

{\parindent0pt Beispiele:}
\begin{enumerate}
\item Sei $\Sigma=\{{\tt 1}\}$, dann ist
\[
\Sigma^*=\{\varepsilon, {\tt 1}, {\tt 11}, {\tt 111}, {\tt 1111},\dots\}
\]
Die Wörter über $\Sigma$ sind also durch ihre Länge charakterisiert.
Es gibt eine Abbildung
\[
\mathbb N\to\Sigma^*\colon n\mapsto \underbrace{1\dots 1}_{\text{$n$ Zeichen}}=1^n
\]
Eine Sprache über dem Alphabet $\{{\tt 1}\}$ entspricht unter dieser
Abbildung genau einer Teilmenge der natürlichen Zahlen.
Die Darstellung
einer Zahl $n$ als Folge von $n$ Zeichen {\tt 1} heisst
{\em unäre Darstellung}
\index{unaere Darstellung@unäre Darstellung}%
von $n$.
\item Sei $\Sigma=\{{\tt (}, {\tt )}\}$.
$\Sigma^*$ besteht aus allen Ketten von Klammern.
Die korrekt geschachtelten Klammerausdrücke bilden
darin eine Teilmenge, also eine Sprache.
\item Sei $\Sigma=\{{\tt 0},{\tt 1},\dots,{\tt 9}\}$.
$\Sigma^*$ besteht aus allen Ziffernfolgen.
Jede solche Ziffernfolge hat natürlich auch
einen numerischen Wert, indem man die Ziffernfolge als Zehnersystemdarstellung
einer Zahl interpretiert.
Wegen möglicher führender Nullen können verschiedene
Ziffernfolgen den gleichen Wert haben.
Es gibt also eine Abbildung
\[
v\colon\Sigma^*\to \mathbb N,
\]
die den Wert einer Ziffernfolge ermittelt.
Damit können wir die Sprache
der Ziffernfolgen mit Zweierpotenzwerten definieren:
\[
\{w\in\Sigma^*\,|\, \exists k (v(w)=2^k)\}=\{
{\tt 1},
{\tt 2},
{\tt 4},
{\tt 8},\dots
{\tt 01},
{\tt 02},
{\tt 04},
{\tt 08},\dots
{\tt 001},
{\tt 002},
{\tt 004},
{\tt 008},\dots\}
\]
\item Sei $\Sigma=\{{\tt 0}, {\tt 1}\}$.
$\Sigma^*$ besteht dann aus den binären Zeichenketten.
Darin können wir eine Reihe von Sprachen auszeichnen:
\begin{align*}
L_1&=\{ {\tt 0}^n{\tt 1}^n\,|\,n\ge 0\}\\
L_2&=\{ w\in\Sigma^*\,|\, |w|_{\tt 0}=|w|_{\tt 1}\}\\
L_3&=\{ w\in\Sigma^*\,|\, \text{Zahlenwert von $w$ ist durch 3 teilbar}\}
\end{align*}
\item Sei $\Sigma$ die Menge der ASCII-Zeichen.
Dann ist $\Sigma^*$ die Menge aller ASCII-Texte, wie unsinnig sie auch
immer sein mögen.
Interessant ist die Sprache
\[
C=\{w\in\Sigma^*\,|\,\text{$w$ wird von GCC akzeptiert}\}.
\]
Die Sprache $C$ heisst GNU-Dialekt von C.
\item Sei $\Sigma$ die Menge der Unicode-Zeichen.
$\Sigma^*$ besteht dann aus allen Unicode-Zeichenfolgen.
Darin können wir zum Beispiel die
Sprache Java auswählen
\[
J
=\{w\in\Sigma^*\,|\, {\text{$w$ wird von einem Java-Compiler akzeptiert}}\}.
\]
\end{enumerate}

Oft kommen Zeichenketten aus vielen identischen Zeichen oder Zeichenketten vor.
Wir verwenden für solche Zeichenketten eine Potenz-Notation:
\begin{align*}
{\tt 0}^3{\tt 1}^5&={\tt 00011111}\\
({\tt 01})^4&={\tt 01010101}
\end{align*}
Die Sprache
\[
L=\{ {\tt 0}^n{\tt 1}^n\,|\,n\in\mathbb R\}
\]
besteht also aus allen Folgen von {\tt 0} und {\tt 1}, in denen eine Anzahl
von Nullen gefolgt wird von einer identischen Anzahl Einsen.

%
% Regulaere Sprachen und endliche Automaten
%
\chapter{Endliche Automaten und reguläre Sprachen\label{chapter-regular}}
\lhead{Reguläre Sprachen}
Endliche Automaten stellen das einfachste Berechnungsmodell
dar, welches in diesem Skript untersucht werden soll.
Ausser über seine Zustände verfügt ein Automat über keinen weiteren
Speicher.
Trotzdem defineren endliche Automaten eine Menge von Sprachen, die 
mathematisch und für die Praxis sehr zweckmässige Eigenschaften
haben.
Ausserdem gibt es ein besonders einfachen Syntax, um
solche Sprachen zu beschreiben: die regulären Ausdrücke.
Es geht in diesem Kapitel allerdings nicht darum, das Arbeiten
mit regulären Ausdrücken zu lernen.
Vielmehr geht es vor allem um das Verständnis, welche Eigenschaften
eine Sprache hat, die mit einem endlichen Automaten oder einem
regulären Ausdruck beschrieben werden kann.

\rhead{Deterministische endliche Automaten}
\section{Deterministische endliche Automaten\label{regulaer:dea}}
Ein endlicher Automat modelliert ein System, welches
in einer endlichen Zahl verschiedener Zustände sein kann.
Der Übergang zwischen den Zuständen geschieht deterministisch beim
Eintreffen neuer Daten.

Als Beispiel betrachten wir ein Drehkreuz an einem Skilift.
Es hat zwei mögliche Zustände: {\it  verriegelt} ($V$)
und  {\it entriegelt} ($E$).
Ausserdem gibt es zwei Inputs: {\tt Fahrkarte} ($F$)
und {\tt Drehen} ($D$).
Zu Beginn befindet sich das Drehkreuz in verriegeltem Zustand.
Wenn ein Skifahrer durch will, es zu drehen
versucht, ändert sich daran nichts.
Erst wenn er eine Fahrkarte einschiebt,
wird das Drehkreuz entriegelt.
Jetzt nützt es nichts, noch weitere
Fahrkarten einzuschieben, das Drehkreuz bleibt entriegelt.
Versucht der Skifahrer jetzt, das Drehkreuz zu drehen, wird es
ihn durchlassen, dabei aber wieder in den verriegelten Zustand übergehen.
Schematisch können wird das wie folgt darstellen:
\[
\entrymodifiers={++[o][F]}
\xymatrix{
*+\txt{} \ar[r]
	&{V}\ar@/^/[r]^{F} \ar@(ur,ul)_{D}
		&{E}\ar@/^/[l]^{D} \ar@(ur,ul)_{F}
}
\]
Wir sehen hier bereits eine Möglichkeit, Inputfolgen zu beobachten, 
aber es kann noch nicht entschieden werden, welche Inputs akzeptabel
sind.
Dazu müssen die einzelnen Zustände ausgezeichnet werden.
Zum Beispiel könnte man verlangen, dass nur Abläufe akzeptabel
sind, bei denen das Drehkreuz am Ende im verriegelten Zustand steht,
also nicht offen stehen bleibt.
Symbolisch bezeichnen wir dies durch einen Doppelkreis:
\[
\entrymodifiers={++[o][F]}
\xymatrix{
*+\txt{} \ar[r]
	&*++[o][F=]{V}\ar@/^/[r]^{F} \ar@(ur,ul)_{D}
		&{E}\ar@/^/[l]^{D} \ar@(ur,ul)_{F}
}
\]
Damit haben wir alle Elemente zusammen, die einen deterministischen
endlichen Automaten ausmachen.
\subsection{Definition\label{regulaer:definition-dea}}
\index{Automat!endlicher}%
\index{Automat!deterministischer endlicher}%
\index{DEA|see{deterministischer endlicher Automat}}%
Eine abstrakte Definition muss die Menge der erlaubten Inputs
festlegen, wir brauchen also ein Alphabet $\Sigma$.
Die Zustände bilden eine Menge.
Ausserdem müssen wir festlegen, in welchem Zustand
sich der Automat zu Beginn befindet und wie er auf ein Input-Zeichen
reagiert.
Dies wird erreicht durch die folgende Definition
\begin{definition}
Ein deterministischer endlicher Automat (DEA) ist ein Quintupel
$(Q,\Sigma,\delta, q_0,F)$ mit
\begin{compactenum}
\item $Q$ ist eine beliebige endliche Menge von Zuständen
\item $\Sigma$ ist eine endliche Menge, genannt das Alphabet.
\index{Übergangsfunktion!eines endlichen Automaten}%
\item $\delta\colon Q\times\Sigma\to Q$ heisst Übergangsfunktion
\index{Startzustand}%
\item $q_0\in Q$ heisst Startzustand
\index{Akzeptierzustand}%
\item $F\subset Q$ heisst die Menge der Akzeptierzustände.
\end{compactenum}
\end{definition}
Die Abbildung $\delta\colon Q\times \Sigma\to Q$ berechnet
aus einem Ausgangszustand $q\in Q$ und einem Input-Zeichen $a\in\Sigma$
einen neuen Zustand $\delta(q,a)\in Q$, in den der Automat durch
die Verarbeitung von $a$ übergeht.

\subsubsection{Gerichteter beschrifteter Graph eines DEA}
\index{Graph!gerichteter!beschrifteter!eines DEA}%
Eine Visualisierung ist häufig übersichtlicher.
Die Zustände von $Q$ werden durch Kreise dargestellt.
Die Übergangsfunktion wird durch mit Buchstaben des Alphabets $\Sigma$
angeschriebene Kreise symbolisiert.
Akzeptierzustände werden durch doppelte Kreise markiert.
Der Startzustand wird durch einen
Pfeil angezeigt, der nicht einen Zustand als Ausgangspunkt hat.

Damit haben wir einen gerichteten beschrifteten Graphen definiert.
Die Vertizes des Graphen sind die Zustände, also $V=Q$.
Vom Zustand $q$ aus verläuft eine mit $a\in\Sigma$ beschriftete
Kante zum Zustand $\delta(q,a)$, 
die Kantenmenge
ist also
\[
E=\{(q,\delta(q,a),a)\;|\; q\in Q\wedge a\in\Sigma\}.
\]
Darin nicht enthalten ist der Startzustand, den wir zwar auch als
Pfeil zeichnen, der aber nicht eine Kante des Graphen ist, weil
er nicht zwei Zustände verbindet.
Zusätzlich sind in diesem Graphen jedoch einzelne Zustände ausgezeichnet,
was wir in einem gewöhnlichen gerichteten Graphen nicht vorgesehen
haben.
Ein DEA ist also ein gerichteter beschrifteter Graph, aber nicht
nur.

Wichtig: von jedem Zustand aus gibt es genau einen Pfeil, der mit jedem möglichen
Zeichen des Alphabets angeschrieben sind.
Insbesondere kann man von jedem
Zustand aus mit jedem beliebigen Zeichen auf genau eine Art fortfahren.
\index{deterministisch}%
In diesem Sinne ist der Automat {\em deterministisch}.

\subsubsection{Beispiele}
\begin{beispiel}[\bf Drehkreuz]
Der Drehkreuz-DEA besteht aus den Elementen $Q={V,E}$, $q_0=V$,
$F=\{V\}$, $\Sigma=\{F,D\}$.
Die Übergangsfunktion definiert zu jedem Zustand und jedem möglichen
Eingabewert, welcher zugehörige Zielwert erreicht wird:
\begin{center}
\begin{tabular}{|cc|c|}
\hline
$q$&$a$&$\delta(q,a)$\\
\hline
$V$&$D$&$V$\\
$V$&$F$&$E$\\
$E$&$D$&$V$\\
$E$&$F$&$E$\\
\hline
\end{tabular}
\end{center}
\end{beispiel}
\begin{beispiel}[\bf Gerade Binärzahlen]
Ein DEA, welcher erkennen kann, ob eine Bitfolge einer geraden Zahl entspricht.
Dieser DEA braucht zwei Zustände, denn eine Bitfolge, interpretiert
als Zahl, kann entweder gerade (Zustand $q_0$, Rest $0$ bei Teilung durch 2)
oder ungerade (Zustand $q_1$, Rest $1$) sein.
Die leere Bitfolge entspricht der Zahl $0$, also einer geraden Zahl.
Hinzufügen eines weiteren
Bits ändert den Zustand möglicherweise: hängt man eine $0$ an, entsteht
eine gerade Zahl, hänge man eine $1$ an, entsteht eine ungerade Zahl.
Wir haben also $\Sigma=\{\texttt{0},\texttt{1}\}$, $q_0$ als Startzustand,
$F=\{q_0\}$.
Die Übergangsfunktion ist
\begin{center}
\begin{tabular}{|cc|c|}
\hline
$q$&$a$&$\delta(q,a)$\\
\hline
$q_0$&$\texttt{0}$&$q_0$\\
$q_0$&$\texttt{1}$&$q_1$\\
$q_1$&$\texttt{0}$&$q_0$\\
$q_1$&$\texttt{1}$&$q_1$\\
\hline
\end{tabular}
\end{center}
Die graphische Form ist wiederum
\[
\entrymodifiers={++[o][F]}
\xymatrix{
*+\txt{}\ar[r]
	&*++[o][F=]{q_0} \ar@/^/[r]^{\texttt{1}} \ar@(ur,ul)_{\texttt{0}}
		&{q_1} \ar@/^/[l]^{\texttt{0}} \ar@(ur,ul)_{\texttt{1}}
}
\]
\end{beispiel}
\begin{beispiel}[\bf Dezimale Ganzzahlen]
Ein Automat, der Ganzzahlen ohne führende Nullen erkennt.
Das Alphabet besteht aus den Ziffern und dem optionalen Minus, also
\[
\Sigma=\{
{\tt 0},
{\tt 1},
{\tt 2},
{\tt 3},
{\tt 4},
{\tt 5},
{\tt 6},
{\tt 7},
{\tt 8},
{\tt 9},
{\tt -}
\}
\]
Eine ganze Zahl kann mit einem {\tt -} beginnen, darf
aber nicht mit einer {\tt 0} anfangen.
Danach dürfen beliebige
Ziffern folgen, aber kein {\tt -}.
Sobald ein {\tt -} eingegeben
wird, muss der Automat den Input als Illegal erkennen.
Offenbar gibt es also folgende Fälle: Zahl mit Minus, Zahl ohne Minus,
führende Null oder andere Illegalitäten.
Zusammen mit dem Startzustand sollten wir also mit vier Zuständen auskommen.
Sei $Q={q_0, m,p,e}$, dann wird das Zustandsdiagramm:
\[
\entrymodifiers={++[o][F]}
\xymatrix{
*+\txt{}\ar[r]
	&{q_0}  \ar[dl]_{{\tt 1},\dots,{\tt 9}} \ar[d]^{\tt 0} \ar[dr]^{\tt -}
\\
*++[o][F=]{p}\ar[r]^{\tt -} \ar@(ul,dl)_{{\tt 0},\dots,{\tt 9}}
	&{e} \ar@(dr,dl)^{\Sigma}
		&{m}\ar[l]_{{\tt 0},{\tt -}} \ar@/^40pt/[ll]^{{\tt 1},\dots,{\tt 9}}
}
\]
\end{beispiel}
\begin{beispiel}[\bf Ganzzahlen, diesmal richtig]
Der eben gezeigte Automat ist insofern eine wörtliche Realisierung
der Anforderungen, dass er keine führende $0$ akzeptiert.
Allerdings
möchten wir in der Praxis auch eine einzelne $0$ akzeptieren können.
Dazu ist es nötig, den Automaten um einen Zustand zu erweitern:
\[
\entrymodifiers={++[o][F]}
\xymatrix{
*+\txt{}\ar[r]
	&{q_0}  \ar[ddl]_{{\tt 1},\dots,{\tt 9}} \ar[d]^{\tt 0} \ar[ddr]^{\tt -}
\\
*+\txt{}
	&*++[o][F=]{0}\ar[d]^{\Sigma}
\\
*++[o][F=]{p}\ar[r]^{\tt -} \ar@(ul,dl)_{{\tt 0},\dots,{\tt 9}}
	&{e} \ar@(dr,dl)^{\Sigma}
		&{m}\ar[l]_{{\tt 0},{\tt -}} \ar@/^40pt/[ll]^{{\tt 1},\dots,{\tt 9}}
}
\]
\end{beispiel}

\subsubsection{Tabellenform}
\index{Tabellenform eines DEA}%
Man kann alle Information eines DEA auch in einer einzigen Tabelle 
darstellen.
Die Zeilen werden mit den Zuständen angeschrieben,
die Spalten mit den Zeichen des Alphabetes.
Akzeptierzustände werden
mit {\tt /E} markiert, der Anfangszustand steht immer auf der ersten
Zeile.
Die oben betrachteten Beispiele haben folgende Tabellenform:

\begin{beispiel}[\bf Drehkreuz] $V$ ist Startzustand und gleichzeitig
Akzeptierzustand.

\begin{center}
\begin{tabular}{|c|cc|}
\hline
&$F$&$D$\\
\hline
$V{\tt /E}$&$E$&$V$\\
$E$&$E$&$V$\\
\hline
\end{tabular}
\end{center}

\end{beispiel}

\begin{beispiel}[\bf Gerade Zahlen]
Die Zustände $q_0$ und $q_1$ stehen für gerade und ungerade Zahlen:

\begin{center}
\begin{tabular}{|c|cc|}
\hline
&0&1\\
\hline
$q_0{\tt /E}$&$q_0$&$q_1$\\
$q_1$&$q_0$&$q_1$\\
\hline
\end{tabular}
\end{center}

\end{beispiel}

\begin{beispiel}[\bf Ganzzahl-Automat] Dieser Automat erkennt die $0$
nicht korrekt.

\begin{center}
\begin{tabular}{|c|ccccccccccc|}
\hline
&\tt 0&\tt 1&\tt 2&\tt 3&\tt 4&\tt 5&\tt 6&\tt 7&\tt 8&\tt 9&\tt -\\
\hline
$q_0$&$e$&$p$&$p$&$p$&$p$&$p$&$p$&$p$&$p$&$p$&$m$\\
$p{\tt /E}$&$p$&$p$&$p$&$p$&$p$&$p$&$p$&$p$&$p$&$p$&$e$\\
$m$&$e$&$p$&$p$&$p$&$p$&$p$&$p$&$p$&$p$&$p$&$e$\\
$e$&$e$&$e$&$e$&$e$&$e$&$e$&$e$&$e$&$e$&$e$&$e$\\
\hline
\end{tabular}
\end{center}

\end{beispiel}

\begin{beispiel}[\bf Ganzzahl-Automat mit $0$] Um die $0$ richtig zu erkennen,
wird ein zusätzlicher Zustand benötigt.

\begin{center}
\begin{tabular}{|c|ccccccccccc|}
\hline
&\tt 0&\tt 1&\tt 2&\tt 3&\tt 4&\tt 5&\tt 6&\tt 7&\tt 8&\tt 9&\tt -\\
\hline
$q_0$&$0$&$p$&$p$&$p$&$p$&$p$&$p$&$p$&$p$&$p$&$m$\\
$0{\tt /E}$&$e$&$e$&$e$&$e$&$e$&$e$&$e$&$e$&$e$&$e$&$e$\\
$p{\tt /E}$&$p$&$p$&$p$&$p$&$p$&$p$&$p$&$p$&$p$&$p$&$e$\\
$m$&$e$&$p$&$p$&$p$&$p$&$p$&$p$&$p$&$p$&$p$&$e$\\
$e$&$e$&$e$&$e$&$e$&$e$&$e$&$e$&$e$&$e$&$e$&$e$\\
\hline
\end{tabular}
\end{center}

\end{beispiel}

\subsection{Von einem DEA akzeptierte Sprache\label{regulaer:akzeptiertesprache}}
\index{akzeptierte Sprache eines DEA}%
Ein DEA definiert auf natürliche Weise eine Sprache: Sie besteht aus
allen Wörtern, die als Input für den DEA angewendet, diesen vom
Startzustand in einen Akzeptierzustand überführen.

\begin{definition}
Sei $A$ ein endlicher Automat, dann ist $L(A)$ die Sprache
\[
L(A)=\{w\in\Sigma^*\;|\; \text{$w$ überführt $A$ in einen Akzeptierzustand}\}
\]
\end{definition}

\begin{definition}
\label{regulaer:definition:regulaere-sprache}
\index{Sprache!reguläre}%
Eine Sprache heisst regulär, wenn es einen DEA gibt, der sie
akzeptiert.
\end{definition}

\subsubsection{Beispiele}
\begin{beispiel}[\bf Durch drei teilbare Zahlen] 
Man finde einen Automaten, der die durch drei teilbaren Zahlen in
Binärdarstellung akzeptiert.
Sei also $\Sigma=\{{\tt 0},{\tt 1}\}$,
$L=\{w\in\Sigma^*\;|\; \text{$w$ stellt eine durch drei teilbare Zahl dar}\}.$
Dabei soll das leere Wort für als $0$
interpretiert werden und damit auch als durch drei teilbar gelten.
Die Sprache $L$ ist regulär, weil sie
von folgendem Automaten akzeptiert wird:
\[
\entrymodifiers={++[o][F]}
\xymatrix{
*+\txt{} \ar[r]
	&*++[o][F=]{0} \ar@/^/[rr]^{\tt 1} \ar@(ur,ul)_{\tt 0}
		&*+\txt{}
			&{1} \ar@/^/[dl]^{\tt 0}  \ar@/^/[ll]^{\tt 1}
\\
*+\txt{}
	&*+\txt{}
		&2\ar@/^/[ur]^{\tt 0} \ar@(dl,ul)^{\tt 1}
}
\]
oder in Tabellenform
\begin{center}
\begin{tabular}{|c|cc|}
\hline
&\tt 0&\tt 1\\
\hline
0{\tt /E}&0&1\\
1        &2&0\\
2        &1&2\\
\hline
\end{tabular}
\end{center}
\end{beispiel}

\begin{beispiel}[\bf Wörter mit einer geraden Anzahl $a$]
Sei $\Sigma=\{{\tt a},{\tt b}\}$,
$L=\{w\in\Sigma^*\;|\; |w|_{\tt a}\equiv 0\mod 2\}$.
$L$ besteht aus den Wörtern, die  eine gerade Anzahl von {\tt a}s enthalten.
Diese Sprache ist regulär, weil sie von folgendem Automaten akzeptiert wird:
\[
\entrymodifiers={++[o][F]}
\xymatrix{
*+\txt{} \ar[r]
	&*++[o][F=]{0}\ar@/^/[r]^{\tt a} \ar@(dl,dr)_{\tt b}
		&1 \ar@/^/[l]^{\tt a} \ar@(dl,dr)_{\tt b}
}
\qedhere
\]
\end{beispiel}

\begin{beispiel}[\bf Bedingungen an einzelne Zeichen]
Finde einen Automaten, der die Sprache $L$ über dem Alphabet
$\Sigma=\{{\tt a},{\tt b}\}$ akzeptiert, deren Wörter mit 
einem {\tt a} beginnen und genau ein {\tt b} enthalten.
\[
\entrymodifiers={++[o][F]}
\xymatrix{
*+\txt{} \ar[r]
	&{q_0} \ar[r]^{\tt a} \ar[dr]^{\tt b}
		&{q_1} \ar@(ul,ur)^{\tt a} \ar[r]^{\tt b}
			&*++[o][F=]{q_2}\ar@(ul,ur)^{\tt a}  \ar[dl]^{\tt b}
\\
*+\txt{}
	&*+\txt{}
		&{e}\ar@(dl,dr)_{{\tt a},{\tt b}}
}
\]
\end{beispiel}

\subsection{Rekonstruktion des Automaten\label{regulaer:rekonstruktion}}
Wenn eine Sprache regulär ist, dann muss es einen DEA geben, der
diese Sprache erkennt.
Doch wie findet man aus der Sprache, also aus der Menge $L$, diesen DEA?

Nehmen wir an, der DEA sei schon bekannt, und versuchen wir,
die Zustände zu charakterisieren.
Befindet sich ein Automat im Zustand $q$, dann kann durch zusätzlichen
Input $w_e$ ein akzeptables Wort entstehen.
Sei also $L(q)$ die Menge der Wörter, mit deren Hilfe man vom Zustand $q$
aus einen Akzeptierzustand erreichen kann.
Wenn man mit einem Wort $w_a$ den Zustand $q$ erreicht, dann ist $L(q)$
offenbar auch
\[
L(q)=\{ w_e\;|\; w_aw_e \in L\}= L(w_a)
\]
Die Mengen $L(w)$ können also ganz allein aus der Kenntnis der
Sprache bestimmt werden, die $L(q)$ sind nicht mehr nötig.

Wir zeigen jetzt, dass wir mit den Mengen $L(w)$ als Zuständen einen
DEA bauen können, der genau die Sprache $L$ akzeptiert.
Wir setzen als für die Menge der Zustände 
\[
Q=\{L(w)\;|\;w\in\Sigma^*\}.
\]

Ein Wort muss offenbar genau dann akzeptiert werden, wenn
man es ans leere Wort anhängen kann und damit zu einem Akzeptierzustand
kommt, also $L=L(\varepsilon)$.
Dies ist der Anfangszustand des Automaten, also
\[
q_0=L.
\]

Die Übergangsfunktion $\delta$ führt die Menge $L(w)$ bei Input
$a$ in die Menge $L(wa)$ über, also
\[
\delta \colon Q\times\Sigma\to Q:(L(w),a)\mapsto L(wa).
\]

Akzeptierzustande sind offenbar diejenigen $L(w)$, von denen
aus man nicht mehr weiter gehen muss, um einen Akzeptierzustand
zu erreichen, also jene $L(w)$ die das leere Wort enthalten:
\[
F=\{L(w)\;|\;\varepsilon\in L(w)\}.
\]

Wir nennen den eben konstruierten Automaten $A$.
Jetzt müssen wir nur noch zeigen, dass dieser Automat genau die
Wörter in $L$ akzeptiert.
Sei das Wort $w\in L$ zusammengesetzt aus den
Zeichen $a_1,a_2,\dots,a_n$, also $w=a_1a_2\dots a_n$.
Dann führt der Input $w$ den Anfangszustand $L$ schrittweise über in
\[
L(a_1)\to L(a_1a_2)\to L(a_1a_2a_3)\to \dots \to L(a_1a_2\dots a_n),
\]
da letzterer das leere Wort enthält, ist er ein Akzeptierzustand.
Ein Wort in $L$ wird also automatisch vom Automaten $A$ akzeptiert.

Jetzt müssen wir noch nachweisen, dass ein von $A$ akzeptiertes Wort $w$
auch in $L$ drin liegt.
Dass $A$ das Wort $w$ akzeptiert besagt, dass $\varepsilon \in L(w)$.
Das ist aber gleichbedeutend damit, dass $w\in L$.
Damit ist alles bewiesen und wir haben folgenden Satz.

\begin{satz}[Myhill-Nerode]\label{satz_dea_aus_sprache}
\index{Myhill-Nerode!Satz von}%
Ist $L$ eine reguläre Sprache, dann wird $L$ von dem 
endlichen Automaten $A=(Q,\Sigma,\delta,q_0,F)$ akzeptiert mit
\begin{align*}
Q&=\{L(w)\;|\;w\in\Sigma^*\}\\
q_0&=L\\
F&=\{q\in Q\;|\; \varepsilon\in q\}\\
\delta&\colon Q\times \Sigma\to Q:(L(w),a)\mapsto L(wa)
\end{align*}
\end{satz}

\subsubsection{Beispiel}
Wir rekonstruieren den Automaten für die folgende Sprache über
$\Sigma=\{{\tt a},{\tt b}\}$:
\[
L=\{w\in \Sigma^*\;|\; |w|_{\tt a}\equiv 0\mod 2\}.
\]
Dazu müssen wir die Menge der möglichen $L(w)$ ermitteln:
\begin{center}
\begin{tabular}{|c|l|}
\hline
$w$&$L(w)$\\
\hline
$\varepsilon$&$L$\\
{\tt a}&$L({\tt a})=\{w\in L^*\;|\; \text{$|w|_a$ ungerade}\}$\\
{\tt b}&$L({\tt b})=\{w\in L^*\;|\; \text{$|w|_a$ gerade}\}=L$\\
$|w|_{\tt a}$ ungerade&$L({\tt a})$\\
$|w|_{\tt a}$ gerade&$L$\\
\hline
\end{tabular}
\end{center}
Offenbar gibt es also genau die zwei Zustände $L$ und $L'=L({\tt a})$.
Ein Zeichen {\tt a} führt $L$ in $L({\tt a})$ über und umgekehrt,
das Zeichen {\tt b} ändert den Zustand nicht:
\[
\entrymodifiers={++[o][F]}
\xymatrix{
*+\txt{} \ar[r]
	&*++[o][F=]{L}\ar@/^/[r]^{\tt a} \ar@(ul,ur)^{\tt b}
		&{L'} \ar@/^/[l]^{\tt a} \ar@(ul,ur)^{\tt b}
}
\]
Dieser Automat ist identisch mit dem früher gefundenen.

Mit dem Satz von Myhill-Nerode erhalten wir damit auch einen Satz,
der zu entscheiden erlaubt, ob eine Sprache regulär ist.

\begin{satz}
Sei $L$ eine Sprache über $\Sigma$ und $L(w)=\{ v\in\Sigma^*\;|\; wv\in L\}$.
Falls die Menge 
\[
\{L(w)\;|\; w\in\Sigma^*\}
\]
endlich ist, ist $L$ regulär.
Falls sie unendlich ist, kann $L$ nicht regulär sein.
\end{satz}

Als Anwendung dieses Satzes zeigen wir, dass die Sprache
$\{ \texttt{0}^n\texttt{1}^n \;|\; n\in\mathbb N\}$ über
$\Sigma=\{\texttt{0},\texttt{1}\}$
nicht regulär ist.
Wir müssen also die Mengen $L(w)$ für alle denkbaren Wörter $w\in\Sigma^*$
ermitteln.
Dazu erstellen wir die folgende Tabelle:
\begin{center}
\begin{tabular}{|c|c|}
\hline
$w$&$L(w)$\\
\hline
$\varepsilon$&$L$\\
$\texttt{1}$&$\emptyset$\\
$\texttt{1}^n$&$\emptyset$\\
$\texttt{0}$&$\texttt{0}^n\texttt{1}^{n+1}\;|\; n\in\mathbb N\}$\\
$\texttt{0}^2$&$\texttt{0}^n\texttt{1}^{n+2}\;|\; n\in\mathbb N\}$\\
$\texttt{0}^k$&$\{\texttt{0}^n\texttt{1}^{n+k}\;|\; n\in\mathbb N\}$\\
$\texttt{0}^k\texttt{1}^l$&$\{\texttt{1}^{k-l}\}$ falls $k\ge l$\\
$\texttt{0}^k\texttt{1}^l$&$\emptyset $ falls $k<l$\\
\hline
\end{tabular}
\end{center}
An den Zeilen vier bis sieben kann man ablesen, dass es unendlich
viele verschiedene Mengen $L(w)$ gibt, somit kann $L$ nicht von
einem endlichen Automaten erkannt werden.

\subsection{Minimaler Automat\label{regulaer:minimalautomat}}
\index{Automat!minimaler}%
Zwei endliche Automaten können völlig verschieden aussehen,
und trotzdem die gleiche Sprache akzeptieren.
Kann man auf einfache
Art herausfinden, ob zwei endliche Automaten die gleiche Sprache
akzeptieren?

Gäbe es für einen Automaten die Möglichkeit, ihn in eine Standardform
zu bringen, so könnte man diese für einen solchen Vergleich verwenden.
Zunächst müsste man beide Automaten in die Standardform bringen.
Wenn die Standardformen übereinstimmen, akzeptieren sie die gleiche Sprache.

Eine Standardform eines Automaten können wir über die Konstruktion
aus Satz \ref{satz_dea_aus_sprache} ermitteln: Zu einem Automaten
$A$ ermitteln wir zunächst die Sprache $L=L(A)$, für die wir dann
den Automaten aus Satz \ref{satz_dea_aus_sprache} bilden.

\begin{satz}[Minimalautomat]\label{satz_minimalautomat}
Sei $A$ ein DEA und $A'$ der DEA, der nach
Satz \ref{satz_dea_aus_sprache} gebildet wurde.
Dann gibt es eine surjektive Abbildung der Zustände $f\colon Q\to Q'$, so dass
\begin{compactenum}
\item $f(q_0)=q_0'=L$
\item $\delta'(f(q),a)=f(\delta(q,a))$
\item $q\in F\Rightarrow f(q)\in F'$
\end{compactenum}
Insbesondere ist $A'$ der kleinste aller Automaten, der die Sprache
$L$ akzeptieren kann.
\end{satz}

\begin{proof}[Beweis]
Jedes Wort $w\in\Sigma^*$ führt den Automaten $A$ aus dem
Anfangszustand in einen Zustand $q\in Q$.
Führen zwei Wörter $w_1$ und $w_2$ den DEA $A$ in den gleichen Zustand $q$,
werden von dort aus die gleichen Wörter akzeptiert, d.\,h.~$L(w_1)=L(w_2)$.
Die Abbildung
\[
f\colon Q\to Q': q\mapsto L(w)
\]
ist also unabhängig von dem Wort $w$, mit welchem man den Zustand $q$
erreicht hat.

Jetzt muss gezeigt werden, dass $f$ die genannten Eigenschaften hat:
\begin{compactenum}
\item
$\varepsilon$ führt $A$ in den Anfangszustand über, also
$f(q_0)=L(\varepsilon)=L=q_0'$.
\item Falls $w$ den DEA in den Zustand $q$ führt, führt das Wort
$wa$ den DEA in den Zustand, den man von $q$ mit Input $a$ erreicht.
Also ist
\[
\delta'(f(q),a)=\delta'(L(w), a)=L(wa)=f(\delta(q,a))
\]
\item Sei $q\in F$ und sei $w$ ein Wort, welches den DEA in den Zustand 
$q$ führt.
Dann ist $f(q)=L(w)$, da aber $q$ ein Akzeptierzustand ist,
ist $\varepsilon\in L(w)$, $f(q)$ ist also ein Akzeptierzustand von
$A'$.
\end{compactenum}

\end{proof}

Die Bestimmung des Automaten über den Satz \ref{satz_dea_aus_sprache}
ist eher umständlich.
Der Satz \ref{satz_minimalautomat} zeigt
aber auch, dass wir einfach nach einem DEA suchen müssen, in dem
möglichst viele Zustände zusammengelegt worden sind, bis sich die
Anzahl der Zustände nicht mehr weiter verringern lässt.

Wir müssen also herausfinden, welche Zustände man zusammenlegen
kann.
Das sind natürlich genau diejenigen, die das gleiche $L(q)$
haben.
\index{aquivalent Zustande@äquivalent!Zustände}%
Statt herauszufinden, welche Zustände in diesem Sinne äquivalent
sind, könnte man auch herauszufinden versuchen, welche nicht äquivalent
sind.
Als Beispiel nehmen wir den folgenden Automaten
\[
\entrymodifiers={++[o][F]}
\xymatrix{
*+\txt{}
	&*+\txt{}
		&{z_1}\ar@/_/[dd]_{0} \ar[dr]^{1}
\\
*+\txt{} \ar[r]
	&{z_0}\ar[ur]^{0} \ar[dr]_1
		&*+\txt{}
			&*++[o][F=]{z_3}\ar@(ur,dr)^{0,1}
\\
*+\txt{}
	&*+\txt{}
		&{z_2} \ar@/_/[uu]_{0} \ar[ur]_{1}
}
\]
Von den zwei Zuständen $z_1$ und $z_2$ aus kann man genau die gleichen
akzeptierten Wörter bilden, also $L(z_1)=L(z_2)$, man müsste diese
also zusammenlegen können.

\index{Algorithmus!für den Minimalautomaten}%
\subsubsection{Algorithmus für den Minimalautomaten
\label{algorithmus:minimalautomat}}
Um solche Zustände zu finden, erstellen wir jetzt eine Tabelle, in der
wir aufzeichnen, welche Zustände äquivalent sind oder auch nicht.
Äquivalente Zustände markieren wir mit dem Zeichen $\equiv$, nicht
äquivalente mit einem Kreuz $\times$.
Zunächst wissen wir nur, dass jeder Zustand zu sich selbst äquivalent
ist, also
\begin{center}
\begin{tabular}{ccccc}
     &$z_0$   &$z_1$   &$z_2$   &$z_3$   \\
$z_0$&$\equiv$&        &        &        \\
$z_1$&        &$\equiv$&        &        \\
$z_2$&        &        &$\equiv$&        \\
$z_3$&        &        &        &$\equiv$
\end{tabular}
\end{center}
Zwei Zustände können nicht äquivalent sein, wenn der eine
ein Akzeptierzustand ist, der andere hingegen nicht, wir markieren
also alle solchen Paare mit einem $\times$, füllen aber nur den Teil unterhalb der
Diagonalen aus.
\begin{center}
\begin{tabular}{ccccc}
     &$z_0$   &$z_1$   &$z_2$   &$z_3$   \\
$z_0$&$\equiv$&        &        &        \\
$z_1$&        &$\equiv$&        &        \\
$z_2$&        &        &$\equiv$&        \\
$z_3$&$\times$&$\times$&$\times$&$\equiv$
\end{tabular}
\end{center}
Falls man von den Zuständen $p$ und $q$ mit einem Übergang
ein bereits mit $\times$ markiertes Feld erreichen kann, können die
Zustände $p$ und $q$ nicht äquivalent sein, man muss also auch das
Feld $(p,q)$ mit $\times$ markieren.
Vom Paar $(z_0,z_1)$ aus kann man zum Beispiel mit einer $1$ das
Paar $(z_2,z_3)$ erreichen, welche als nicht äquivalent bekannt
sind.
Ebenso kann man von $(z_0,z_2)$ mit einer $1$ das Paar
$(z_2,z_3)$ erreichen.
So findet man die Tabelle
\begin{center}
\begin{tabular}{ccccc}
     &$z_0$   &$z_1$   &$z_2$   &$z_3$   \\
$z_0$&$\equiv$&        &        &        \\
$z_1$&$\times$&$\equiv$&        &        \\
$z_2$&$\times$&        &$\equiv$&        \\
$z_3$&$\times$&$\times$&$\times$&$\equiv$
\end{tabular}
\end{center}
Das verbleibende Paar $(z_1,z_2)$ wird aber immer in $(z_1,z_2)$ 
oder $(z_3,z_3)$ übergeführt, von beiden Paare können wir nicht
schliessen, dass sie nicht äquivalent sind, also müssen sie äquivalent
sein:
\begin{center}
\begin{tabular}{ccccc}
     &$z_0$   &$z_1$   &$z_2$   &$z_3$   \\
$z_0$&$\equiv$&        &        &        \\
$z_1$&$\times$&$\equiv$&        &        \\
$z_2$&$\times$&$\equiv$&$\equiv$&        \\
$z_3$&$\times$&$\times$&$\times$&$\equiv$
\end{tabular}
\end{center}
Man kann also die beiden Zustände $z_1$ und $z_2$ zusammenlegen,
und erhält den minimalen Automaten:
\[
\entrymodifiers={++[o][F]}
\xymatrix{
*+\txt{} \ar[r]
	&{z_0}\ar[r]^{0,1} 
		&{z_{1,2}}\ar@(ur,ul)_{0}\ar[r]^1
			&*++[o][F=]{z_3}\ar@(ur,dr)^{0,1}
}
\]

Der Vergleich von zwei deterministischen endlichen Automaten ist zwar
mit Hilfe des Minimalautomaten möglich, dies ist jedoch nicht der
bestmögliche Algorithmus.
Ein Algorithmus mit linearer Laufzeit wurde von Hopcroft und Karp
angegeben.
\begin{figure}
\begin{center}
\includegraphics{images/reg-6.pdf}
\end{center}
\caption{Pumping Lemma: Zerlegung eines Wortes $w$ in drei Teile
$w=xyz$.\label{regular:pumpinglemma-graph}}
\end{figure}

\subsection{Pumping Lemma für reguläre Sprachen\label{regulaer:pumpinglemma}}
\index{Pumping Lemma!für reguläre Sprachen}%
Der Satz~\ref{satz_dea_aus_sprache} zeigt, wie man zu einer regulären Sprache
einen endlichen Automaten finden kann.
Als Nebeneffekt kann man daraus oft auch ableiten, ob eine Sprache regulär
ist.
Wenn allerdings die Zahl der Zustände sehr gross ist, kann man beim
Auflisten der Zustände nicht unterscheiden, ob man einfach noch etwas
mehr Geduld haben muss, bis man alle Zustände gefunden hat, oder ob
es tatsächlich unendlich viele verschiedene Mengen unter den $L(w)$
gibt.
Wir brauchen daher eine Methode, mit der man die Frage nach der
Regularität entscheiden können, ohne auf eine Auflistung aller
nötigen Zustände angewiesen zu sein.

Eine reguläre Sprache wird von einem DEA akzeptiert.
Dadurch wird die Struktur der Wörter eingeschränkt.
Ein Wort $w$ beschreibt
einen Pfad durch den gerichteten beschriftenten Graphen des DEA
(Abbildung~\ref{regular:pumpinglemma-graph}).
Wenn das Wort länger ist als die Anzahl der Zustände, dann muss
mindestens ein Zustand mehrmals vorkommen.
Wir können sogar
sagen, dass die erste solche Wiederholung innerhalb der ersten
$N = |Q|$ Zeichen des Wortes $w$ auftreten muss.
Sei $x$ das Wort,
das den Zustand in den ersten wiederholten Zustand $q$ führt.
Sei $y$ das kürzeste Wort, welches von $q$ wieder zu $q$ führt und
$z$ der Rest, also $w=xyz$.
Dann können wir die Schleife $y$
auch mehrmals wiederholen und immer ein Wort erhalten welches
vom Anfangszustand über den Zustand $q$ (evlt.~mehrmals) zu einem
Endzustand führt.
Die Wörter $xy^kz$ werden also alle auch von dem Automaten akzeptiert.
Damit haben wir folgendes bewiesen:
\begin{satz}[Pumping Lemma für reguläre Sprachen]
\index{pumping length!für reguläre Sprachen}%
Ist $L$ eine reguläre Sprache, dann gibt es eine Zahl $N$, die Pumping Length, so dass
jedes Wort $w\in L$ mit $|w|\ge N$ in drei Teile
$w=xyz$ zerlegt werden kann, so dass
\begin{compactenum}
\item $|y| > 0$
\item $|xy|\le N$
\item $xy^kz\in L\quad\forall k\ge 0$
\end{compactenum}
\end{satz}

Das Pumping Lemma dient in erster Linie dazu, von einer Sprache
nachzuweisen, dass sie nicht regulär ist.
Dazu nimmt man an, dass die Sprache regulär ist, für sie also die Aussage
des Pumping Lemma gilt.
Dann leitet man daraus einen Widerspruch ab.


\begin{beispiel}[\bf Eine nicht reguläre Sprache]
Wir führen dies an dem Beispiel der Sprache $L=\{0^n1^n|n\ge 0\}$
durch, welche wir bereits als nicht regulär erkannt haben.
Wir nehmen jetzt also an, dass $L$ regulär sei.
Dann gibt es nach dem Pumping Lemma eine Zahl $N$, so dass Wörter mit
mindestens dieser Länge interessante Eigenschaften haben.
Wir wählen das Wort
\[
0^N1^N\in L
\]
Nach dem Pumping Lemma muss $w$ in drei Teile aufgeteilt werden können
(Abbildung~\ref{plimage}),
$w=xyz$,
wovon die ersten beiden Teile $x$ und $y$ zusammen eine Länge $\le N$
haben, also aus lauter Nullen bestehen müssen.
Der Teil $y$ muss mindestens eine Null enthalten.
Wenn man jetzt $xy^kz$ bildet, vermehrt man die Zahl der Nullen,
nicht aber die Zahl der Einsen, man erhält also ein Wort, welches
mehr Nullen als Einsen enthält.
Das Pumping Lemma sagt, dass $xy^kz\in L$, aber die Definition von $L$ sagt,
dass ein Wort in $L$ gleich viele Nullen wie Einsen haben muss, also
$xy^kz\not\in L$ für $k>1$.
Dieser Widerspruch zeigt, dass die ursprüngliche
Annahme, $L$ sei regulär gewesen, nicht haltbar ist.
Also ist die Sprache nicht regulär.
\begin{figure}
\begin{center}
\begin{tabular}{l}
\includegraphics{images/pl-2.pdf}\\
\includegraphics{images/pl-1.pdf}\\
\includegraphics{images/pl-3.pdf}\\
\includegraphics{images/pl-4.pdf}
\end{tabular}
\end{center}
\caption{Anwendung des Pumping-Lemmas.
Ein genügend langes Wort
der Sprache $w=0^N1^N$ kann nach dem Pumping Lemma in $w=xyz$ 
zerlegt werden (2.~Zeile).
Das Pumping Lemma verspricht, dass
abgepumpte Wörter (3.~Zeile) und aufgepumpte Wörter (4.~Zeile)
auch zur Sprache gehören sollen, tun sie aber nicht.
\label{plimage}}
\end{figure}
\end{beispiel}

\subsubsection{Beispiel}
\index{Palindrom}%
Ein Palindrom ist ein Wort, welches ``vorwärts'' und ``rückwärts''
gleich lautet.
Zum Beispiel {\tt otto}, {\tt anna}, {\tt reittier} oder der
Klassiker
\begin{center}
{\tt ein neger mit gazelle zagt im regen nie}
\end{center}
Formal können wir das so beschreiben: ist $w=a_1a_2\dots a_n\in\Sigma^*$
ein Wort, dann sei $w^t=a_na_{n-1}\dots a_2a_1$ das ``rückwärts
geschriebene Wort:
\[
{\tt esel}^t={\tt lese},\quad {\tt reliefpfeiler}^t={\tt reliefpfeiler}.
\]
Palindrome sind offenbar Wörter, die sich bei Anwendung der
$\mathstrut^t$-Operation nicht ändern: $w^t=w$.
Die Menge der Palindrome
ist natürlich eine Sprache $L=\{w\in\Sigma^*\;|\;w^t=w\}$.

Wir zeigen jetzt, dass die Menge der Palindrome keine reguläre Sprache ist.
Dazu verwenden wir das Pumping-Lemma, wir nehmen also an, die Sprache
sei regulär und führen dies mit dem Pumping Lemma zu einem Widerspruch:
\begin{compactenum}
\item Das Pumping Lemma verspricht, dass es eine Zahl $N$ gibt, die
Pumping Length.
\item Wir bilden jetzt das Wort $0^N10^N$.
Dieses Wort ist ein Palindrom,
also in $L$.
\item Das Pumping Lemma garantiert jetzt eine Zerlegung des Wortes
in drei Teile $xyz$, mit den Eigenschaften $|xy|\le N$, $|y|\ge 1$ und
$xy^kz\in L\forall k\in\mathbb N$.
\item Wir prüfen dies nach für $k=2$ (Abbildung~\ref{regular:pl-palindrome}).
\begin{figure}
\centering
\includegraphics{images/palindrom-1.pdf}
\caption{Anwendung des Pumping-Lemmas auf die Sprache der Palindrome
\label{regular:pl-palindrome}}
\end{figure}
Da $|xy|\le N$, bestehen sowohl $x$ als auch $y$ aus lauter Nullen.
Die $1$ befindet sich im Teil $z$.
Das Wort $xyyz$ hat daher $N+|y|$ Nullen vor der $1$, aber nur $N$ Nullen
nach der $1$, ist also kein Palindrom mehr.
Es ist also $xy^2z\not\in L$,
im Widerspruch zur Ausssage des Pumping Lemma (vorangegangener Schritt).
\end{compactenum}
Der Widerspruch zeigt jetzt, dass die Annahme, $L$ sei regulär, nicht
haltbar ist, $L$ ist also nicht regulär.

\rhead{Nicht deterministische Automaten}
\section{Nicht deterministische Automaten\label{regulaer:nea}}
Ein DEA sieht jeweils nur ein Zeichen weit, kann sich nicht an ältere
Zeichen erinnern und kann seine früheren Entscheidungen beim Eintreffen
neuer Zeichen nicht mehr revidieren.
Es ist daher nicht unmittelbar klar,
wie eine Bedingung wie ``wenn ein
Wort mit einer {\tt 0} aufhört, dann muss es auch mit einer {\tt 0}
beginnen'' implementiert werden müsste.

\index{Automat!nicht deterministischer}%
\index{NEA|see{nicht deterministischer endlicher Automat}}%
Nichtdeterminismus erweitert 
die Definition eines endlichen Automaten, um solche Fälle leichter 
zugänglich zu machen.
Es stellt sich daher die Frage, ob die Menge 
der Sprachen, die von solchen nichtdeterministischen endlichen Automaten (NEA)
erkannt werden können,
grösser ist.
Es wird sich zeigen, dass jeder NEA
in einen äquivalenten DEA umgewandelt werden kann, NEAs erkennen also
die gleichen Sprachen wie DEAs.

\subsection{Definition\label{regulaer:definition-nea}}
In einem DEA gibt es zu jedem Zustand $q\in Q$
und jedem Zeichen $a\in\Sigma$ genau
einen Übergang in den Zustand $\delta(q,a)$.
Ein nicht deterministischer Automat lässt dagegen mehrere Möglichkeiten zu.
Das ``Resultat''
von $\delta$ ist also nicht mehr ein (eindeutiger) neuer Zustand, sondern
eine ganze Menge von Zuständen, also ein Element der Potenzmenge $P(Q)$.

Zudem soll ein
nicht deterministischer Automat auch einen Übergang durchführen
können, wenn gar kein Input anliegt.
Ausser den Zeichen aus $\Sigma$ soll für das zweite Argument
von $\delta$ auch das leere Wort zulässig sein.

\begin{definition}\label{definition_nea}
Ein nichtdeterministischer endlicher Automat (NEA) ist ein Quintupel
$(Q,\Sigma,\delta, q_0,F)$ mit
\begin{compactenum}
\item $Q$ ist eine beliebige endliche Menge von Zuständen
\item $\Sigma$ ist eine endliche Menge, genannt das Alphabet.
\item $\delta\colon Q\times(\Sigma\cup\{\varepsilon\})\to P(Q)$ heisst Übergangsfunktion
\item $q_0\in Q$ heisst Startzustand
\item $F\subset Q$ heisst die Menge der Akzeptierzustände.
\end{compactenum}
\end{definition}
\index{$\varepsilon$-Übergänge}%
Übergänge mit $\varepsilon$ als ``Input'' heissen $\varepsilon$-Übergänge.
Ein NEA kann einen solchen Übergang durchführen ``wann er Lust hat''.
Wir bezeichnen NEAs mit $\varepsilon$-Übergängen auch als
NEA$\mathstrut_\varepsilon$.

Ein DEA ist ein Spezialfall eines NEA.
Ein NEA ist ein DEA, wenn
\begin{align}
\delta(q,\varepsilon)&=\emptyset&&\forall q\in Q\label{deanea1}\\
|\delta(q,a)|&=1&&\forall q\in Q, a\in\Sigma\label{deanea2}
\end{align}
Die erste Bedingung (\ref{deanea1}) bedeutet, dass keine
$\varepsilon$-Übergänge vorkommen.
Die zweite Bedingung (\ref{deanea2}) bedeutet, dass von jedem Zustand zu
jedem Zeichen genau ein neuer Zustand möglich ist.

\subsubsection{Gerichteter beschrifteter Graph eines NEA}
\index{Graph!gerichteter!beschrifteter!eines NEA}%
Auch ein NEA kann durch einen gerichteten beschrifteten Graphen
visualisiert werden.
Die Vertizes sind die Zustände $V=Q$.
Die Kanten sind mit Element aus $\Sigma\cup\{\varepsilon\}$ beschriftet.
Von der Ecke $q\in Q$ aus führt zu jedem Element von $\delta(q,a)\in P(Q)$
eine mit $a$ beschriftete Kante, wobei $a\in \Sigma\cup\{\varepsilon\}$, $a=\varepsilon$ ist also auch möglich.

Es ist also durchaus denkbar, dass von einem Zustand $q$ aus für ein
bestimmtes Eingabezeichen $a$ kein Übergang möglich ist, nämlich
wenn $\delta(q,a)=\emptyset$.
Der Automat ``verklemmt'' sich sozusagen in dieser Situation und ist
sicher nicht in der Lage, ein Wort, welches
ihn in diese Situation geführt hat, zu akzeptieren.


\subsection{Beispiele}
\begin{beispiel}[\bf Ganzzahl-Automat]
Im Ganzzahl-Automat führen viele Übergänge zum Zustand $e$, der
alle fehlerhaften Inputs aufsammelt.
Wir könnten diesen Zustand einfach
weglassen und erhalten folgenden, wesentlich übersichtlicheren NEA:
\[
\entrymodifiers={++[o][F]}
\xymatrix{
*+\txt{}\ar[r]
	&{q_0}  \ar[dl]_{{\tt 1},\dots,{\tt 9}} \ar[dr]^{\tt -}
\\
*++[o][F=]{p}\ar@(ul,dl)_{{\tt 0},\dots,{\tt 9}}
	&*+\txt{}
		&{m} \ar[ll]^{{\tt 1},\dots,{\tt 9}}
}
\]
Aus Zustand $q_0$ ist kein Übergang mit dem Zeichen {\tt 0} möglich,
dadurch werden führende Nullen verboten.
Ein {\tt -} kann nur am Anfang
stehen, befindet sich der Automat bereits im Zustand $m$ oder $p$, kann 
er keine weiteren {\tt -} akzeptieren.
Nach einem {\tt -} darf nur ein $\texttt{1}\dots\texttt{9}$ folgen,
denn vom Zustand $m$
aus gibt es keinen mit \texttt{0} angeschrieben Übergang.
\end{beispiel}

\begin{beispiel}[\bf Bedingung an ein einzelnes Zeichen]
Sei $\Sigma=\{{\tt a},{\tt b}\}$, man finde einen NEA, der die 
Sprache akzeptiert, deren Wörter an der drittletzten Stelle
eine {\tt a} haben.
\[
\entrymodifiers={++[o][F]}
\xymatrix{
*+\txt{} \ar[r]
	&{q_0} \ar[r]^{\tt a} \ar@(dr,dl)^{{\tt a},{\tt b}}
		&{q_1}\ar[r]^{{\tt a},{\tt b}}
			&{q_2}\ar[r]^{{\tt a},{\tt b}}
				&*++[o][F=]{q_3}
}
\]
Der Automat ist nicht deterministisch, weil im Zustand $q_0$ zwei verschiedene
Übergänge für das Zeichen {\tt a} möglich sind.
Der Automat bleibt
sozusagen im Zustand $q_0$ bis er ``weiss'', dass jetzt das drittletzte
Zeichen ansteht.
\end{beispiel}

\begin{beispiel}[\bf Teilbarkeit]
Für $\Sigma=\{{\tt 0}\}$ finde einen NEA für die Sprache 
\[
L=\{w\in \Sigma^*\;|\; \text{$|w|$ ist durch 2 oder 3 teilbar}\}.
\]
Das Problem bei der Konstruktion eines DEA ist, dass wird zu
Beginn noch nicht wissen können, ob wir einen ankommenden
String auf Teilbarkeit durch $2$ oder durch $3$ testen sollen.
Also verwenden wir $\varepsilon$-Übergänge aus dem Startzustand
in zwei verschiedene Automaten, die die Teilbarkeit testen:
\[
\entrymodifiers={++[o][F]}
\xymatrix{
*+\txt{}
	&*+\txt{}
		&*++[o][F=]{q_1} \ar@/^/[r]^{\tt 0}
			&{q_2} \ar@/^/[l]^{\tt 0}
\\
*+\txt{} \ar[r]
	&{q_0}\ar[ur]^{\varepsilon} \ar[dr]^{\varepsilon}
\\
*+\txt{}
	&*+\txt{}
		&*++[o][F=]{q_3}\ar[r]^{\tt 0}
			&{q_4}\ar[dl]^{\tt 0}
\\
*+\txt{}
	&*+\txt{}
		&{q_5}\ar[u]^{\tt 0}
}
\]
\end{beispiel}

\subsection{Erreichbare Zustände\label{regulaer:erreichbarezustaende}}
\index{erreichbare Zustände}%
Im Folgenden werden wir berechnen müssen, welche Menge von Zuständen
durch einen Übergang oder mehrere verkettete Übergänge erreicht werden
können.
Es besteht ein Unterschied, ob der Übergang infolge eines
Input-Zeichens $a\in\Sigma$ erfolgt, oder ob er spontan als
$\varepsilon$-Übergang ausgeführt werden kann.

\subsubsection{Übergang zu Zeichen aus $\Sigma$}
Die Menge $\delta(q,a)\subset Q$ gibt die in einem Schritt mit Inputzeichen $a$
von $q$ aus erreichbaren Zustände von $Q$ an.
Für den Input $a_1a_2$ kann
man von jedem Zustand in $\delta(q,a_1)$ aus weiter Zustände mit einem
Überang mit Inputzeichen $a_2$ erreichen.
Die Menge der erreichbaren Zustände ist
\begin{equation}
\delta(q,a_1a_2)=\bigcup_{q_1\in\delta(q,a_1)}\delta(q_1, a_2)
=\delta(\delta(q,a_1),a_2)
\label{erreichbar}
\end{equation}
Dies kann man wiederholen:
\begin{align*}
\delta(q,a_1a_2a_3)&=
\bigcup_{q_2\in\delta(q, a_1a_2)}\delta(q_2,a_3)
=
\delta(\delta(\delta(q,a_1),a_2),a_3)
\\
\delta(q,a_1\dots a_n)&=\bigcup_{q_{n-1}\in\delta(q,a_1\dots a_{n-1})}\delta(q_{n-1},a_n)
=\delta(\dots \delta(q,a_1)\dots,a_n)
\end{align*}
Die so definierte Menge $\delta(q,a_1\dots a_n)$ umfasst alle von
$q$ aus erreichbaren Zustände.

Falls $q'\in\delta(q,a_1\dots a_n)$ gibt
es insbesondere eine Folge von Zwischenzuständen $q_1,\dots,q_{n-1}$
mit $q_k\in\delta(q_{k-1},a_k)$ für alle $k$.
Ausserdem  ist natürlich $q_k\in\delta(q,a_1\dots a_k)$ für alle $k$.

Eine Menge $M$ von Zuständen stellt man sich am besten vor, indem man die
darin enthaltenen Zustände farbig markiert.
In der Menge $\delta(M,a)$ befinden sich alle Zustände, die von Zuständen
von $M$ aus mit Input $a$ erreichbar sind.

\subsubsection{$\varepsilon$-Übergänge}
Einzelne Zustände kann man auch durch $\varepsilon$-Übergänge
erreichen.
Die Menge der Zustände, die von $q\in Q$ aus durch
$\varepsilon$-Übergänge erreichbar sind, bezeichnen wir mit
$E(q)$.
Sie setzt sich zusammen aus Zuständen, die in einem einzigen 
Schritt erreichbar sind, und solchen, die durch wiederholte
Schritte erreichbar werden.
\[
E(q)=\{q\} \cup \delta(q,\varepsilon) \cup \delta(\delta(q,\varepsilon),\varepsilon)\cup\dots
\]

\subsection{''Könnte''-Automat\label{Thompson-NEA}}
Ein NEA hat bei jedem Übergang eventuell mehrere Zielstände zur Auswahl.
Eine Implementation muss im Prinzip alle diese Möglichkeiten 
durchprobieren.
Alternativ könnte er sich aber auch merken,
welche möglichen Zustände im Laufe der Verarbeitung eines
Wortes erreicht werden konnten.
Die Menge der möglichen erreichten Zustände kann mit jedem
neuen Zeichen des Inputwortes neu berechnet werden.
Eine separate Speicherung ist nicht erforderlich, wenn man die
erreichten Zustände mit einer Markierung versieht.

\begin{figure}
\begin{center}
\includegraphics{images/reg-2}
\end{center}
\caption{Beispiel eines NEA\label{koenntenea}}
\end{figure}

\begin{figure}
\begin{center}
\begin{tabular}{cccc}
&%
\includegraphics{images/reg-2}&
\raisebox{60pt}{$\overset{\displaystyle\text{\tt b}}\longrightarrow$}&
\includegraphics{images/reg-3}\\
\raisebox{60pt}{$\overset{\displaystyle\text{\tt a}}\longrightarrow$}&
\includegraphics{images/reg-4}&
\raisebox{60pt}{$\overset{\displaystyle\text{\tt a}}\longrightarrow$}&
\includegraphics{images/reg-5}
\end{tabular}
\end{center}
\caption{Verarbeitung des Wortes {\tt baa} durch den NEA von
Abbildung~\ref{koenntenea}.
Die möglichen Zustände sind jeweils
rot markiert.\label{koenntebeispiel}
}
\end{figure}
Abbildung~\ref{koenntenea} zeigt einen NEA mit drei Zuständen.
Wir verfolgen die Verarbeitung des Wortes {\tt baa} in
Abbildung~\ref{koenntebeispiel}.

Bevor ein Zeichen verarbeitet wird,
ist der NEA im Startzustand $q_0$, also ist genau dieser Zustand
markiert.
Die Verarbeitung des Zeichens {\tt b} ist deterministisch,
sie bringt den NEA in den Zustand $q_1$.
Das folgende Zeichen {\tt a}
ist jedoch nicht mehr deterministisch, es sind sowohl $q_1$ als auch
$q_2$ als Folgezustände möglich.
Die markierte mögliche Menge von Zuständen nach Verarbeitung von {\tt ba}
ist daher $\{q_1,q_2\}$.
Bei der Verarbeitung des zweiten {\tt a} kommt noch der Zustand $q_0$ hinzu.
Man berechnet also schrittweise die
Menge der erreichbaren Zustände $\delta(q_0,{\tt baa})$.

Nach Verarbeitung des ganzen Wortes kann der NEA in den Zuständen
$\{q_0,q_1,q_2\}$ sein.
Da der Akzeptierzustand $q_0$ in dieser Menge
enthalten ist, gibt es eine Kombination von nichtdeterministischen
Entscheidungen, die zu $q_0$ führen, das Wort {\tt baa} kann also
akzeptiert werden.

\index{Thompson, Ken}%
\index{Thompson-NEA}%
Diese Implementation eines NEA geht auf Ken Thompson zurück, sie wurde in
der Regex-Library der achten Edition von Unix von Rob Pike
\index{Pike, Rob}%
implementiert, die allerdings nicht sehr weit verbreitet war.
Später hat Henry Spencer die Regex-Library neu
implementiert, allerdings ohne die Thompson-NEA Konstruktion, sondern
mit Backtracking.
Da er sie in den public domain freigab, hat sie sich
rasch verbreitet und bildete die Basis Regex-Bibliotheken in Perl, PCRE,
Python und vielen anderen.

\subsection{Transformation NEA \texorpdfstring{$\rightarrow$}{->} DEA\label{regulaer:nea-dea}}
NEAs führen trotz der beträchtlichen Erweiterung durch den
Nichtdeterminismus nicht zu einer grösseren Klasse von akzeptierbaren
Sprachen.
Dazu genügt es zu zeigen, dass sich jeder NEA in einen DEA
transformieren lässt, der die gleiche Sprache akzeptiert.
Dass dies möglich sein sollte, deutet bereits die Konstruktion
des Thompson-NEA in \ref{Thompson-NEA} an.
Der Thompson-NEA wird zu einem DEA, wenn man die Verteilung der
``roten Markierungen'' als Zustand des DEA ansieht.
Ziel dieses Abschnittes ist, dies
mathematisch streng zu formalisieren.

Ein NEA kann aus zwei Gründen kein DEA sein:
er kann $\varepsilon$-Übergänge haben und er kann nicht eindeutige
oder nicht existierende Übergänge haben, also $|\delta(q,a)|\ne 1$
für gewisse $q\in Q$ und $a\in\Sigma$.
Wir transformieren einen 
beliebigen NEA in zwei Schritten in einen DEA:
\begin{enumerate}
\item Wir bauen uns einen Algorithmus, mit dem man einen NEA ohne
$\varepsilon$-Übergänge in einen DEA umwandeln kann.
\item Wir modifizieren den Algorithmus, so dass er auch mit NEAs mit
$\varepsilon$-Übergängen umgehen kann.
\end{enumerate}

\subsubsection{Transformation für NEA ohne $\varepsilon$-Übergänge}
Der konstruierte DEA muss darüber Buch führen, in welchen Zuständen
der NEA sein könnte.
Seine Zustände sind als Mengen von möglichen Zuständen des NEA.
Als Zustandsmenge des DEA wird man also $P(Q)$ 
erwarten.
Wie die übrigen Elemente des DEA konstruiert werden müssen,
besagt der folgende Satz.

\begin{satz}
\label{satz_neadea_eps}
Sei $A=(Q,\Sigma,\delta,q_0,F)$ ein NEA ohne $\varepsilon$-Übergänge.
Dann gibt es einen DEA $A'$, der die gleiche Sprache akzeptiert: $L(A)=L(A')$.
Der DEA $A'$ setzt sich zusammen aus
\[
A'=
(P(Q), \Sigma, \delta', \{q_0\}, F'),
\]
die Übergangsfunktion ist
\[
\delta'\colon P(Q)\times \Sigma: (M, a)\mapsto \bigcup_{q\in M} \delta(q,a)
=\delta(M,a),
\]
und die Akzeptierzustände sind die Teilmengen, die einen Akzeptierzustand
enthalten:
\[
F'=\{M\in P(Q)\;|\;M\cap F\ne \emptyset\}.
\]
\end{satz}
\begin{proof}[Beweis]
Es ist klar, dass mit dieser Übergangsfunktion $\delta'$ und den
Akzeptierzuständen $F'$ ein DEA definiert ist, es muss nur noch
bewiesen werden, dass er die gleiche Sprache akzeptiert, also
$L(A)=L(A')$.
Das wird zutreffen, wenn $L(A)\subset L(A')$ und
$L(A')\subset L(A)$, also jedes Wort von $L(A)$ ist auch in $L(A')$,
und umgekehrt.

Wenn $w=a_1\dots a_n\in L(A)$, dann gibt es einen Pfad durch den gerichteten
Graphen von $A$, der zu einem Akzeptierzustand führt.
Also enthält
die Menge der von $q_0$ aus erreichbaren Zustände einen Akzeptierzustand.
Diese Menge ist aber die Menge der Zustände, die man durch Anwendung
von $\delta'$ aus $\{q_0\}$ und den Inputzeichen $a_1,\dots,a_n$ erhalten
wird.
Da diese Menge einen Akzeptierzustand enthält, ist sie ein
Akzeptierzustand von $A'$ und damit $w\in L(A')$.

Sei jetzt umgekehrt $w\in L(A')$.
Wir schreiben wieder $w=a_1\dots a_n$.
Auf Grund der Konstruktion des Automaten $A'$, liefert die wiederholte
Anwendung von $\delta'(\cdot, a_k)$ aus dem Anfangszustand $q_0'=\{q_0\}$
die Menge $\delta(\dots\delta(q_0, a_1)\dots ,a_n)$.
Da diese Menge ein
Akzeptierzustand von $A'$ ist, gibt es darin einen Akzeptierzustand
von $A$.
Da also ein Akzeptierzustand erreichbar ist, ist $w\in L(A)$,
also auch $L(A')\subset L(A)$.
\end{proof}

\subsubsection{Beispiel}
Der NEA von Abbildung~\ref{koenntenea}
%\[
%\entrymodifiers={++[o][F]}
%\xymatrix{
%*+\txt{}\ar[r]
%	&*++[o][F=]{q_0} \ar[d]_{\tt b} 
%		&{q_2}\ar[l]_{\tt a}
%\\
%*+\txt{}
%	&{q_1}\ar[ur]_{{\tt a},{\tt b}} \ar@(ul,dl)_{\tt a}
%}
%\]
soll in einen DEA umgewandelt werden.
Er ist nicht deterministisch,
weil vom Zustand $q_1$ aus zwei Übergänge mit {\tt a} möglich sind.

Aus dem Satz ist bekannt, dass die Zustände die Potenzmengen von $Q$ sind.
Diese entsprechen den dreistelligen Binärzahlen, wobei jede
Stelle angibt, ob das zugehörige $q_i$ in der Menge dabei ist.
Wir
schreiben also
\begin{align*}
       &          &q_{001}&=\{\phantom{q_2q_1}q_0\}&q_{110}&=\{q_2,q_1\phantom{,q_0}\}&&\\
q_{000}&=\emptyset&q_{010}&=\{\phantom{q_2}q_1\phantom{q_0}\}&q_{101}&=\{q_2,\phantom{q_1,}q_0\}&q_{111}&=\{q_2,q_1,q_0\}\\
       &          &q_{100}&=\{q_2\phantom{q_1q_0}\}&q_{011}&=\{\phantom{q_2,}q_1,q_0\}&&\\
\end{align*}
Akzeptierzustände sind jene Mengen, die $q_0$ enthalten, also
\[
F'=\{ q_{001}, q_{011}, q_{101}, q_{111}\},
\]
Startzustand ist $q_{001}$.
Damit können wir jetzt das Zustandsdiagramm zeichnen
\[
\entrymodifiers={++[o][F]}
\xymatrix{
*+\txt{}\ar[r]
	&*++[o][F=]{q_{001}} \ar[d]^{\tt b}\ar[dl]_{\tt a}
		&{q_{110}}\ar[dr]^{\tt a} \ar[ddl]^{\tt b}
\\
{q_{000}}\ar@(ul,dl)_{{\tt a},{\tt b}}
	&{q_{010}} \ar[ur]^{\tt a} \ar[d]^{\tt b}
		&*++[o][F=]{q_{101}} \ar[ul]^{\tt a} \ar[l]^{\tt b}
			&*++[o][F=]{q_{111}}\ar@(ur,dr)^{{\tt a}} \ar@/_20pt/[ul]_{\tt b}
\\
*+\txt{}
	&{q_{100}}\ar@/^20pt/[uu]^{\tt a}  \ar[ul]^{\tt b}
		&*++[o][F=]{q_{011}} \ar@/_20pt/[uu]_{{\tt a},{\tt b}}
}
\]
Man erkennt sofort, dass die Zustände $q_{101}$ und $q_{011}$
nicht erreicht werden können, also weggelassen werden könnten:
\begin{equation}
\entrymodifiers={++[o][F]}
\xymatrix{
*+\txt{}\ar[r]
	&*++[o][F=]{q_{001}} \ar[d]^{\tt b}\ar[dl]_{\tt a}
		&{q_{110}}\ar[dr]^{\tt a} \ar[ddl]^{\tt b}
\\
{q_{000}}\ar@(ul,dl)_{{\tt a},{\tt b}}
	&{q_{010}} \ar[ur]^{\tt a} \ar[d]^{\tt b}
		&*+\txt{}
			&*++[o][F=]{q_{111}}\ar@(ur,dr)^{{\tt a}} \ar@/_20pt/[ul]_{\tt b}
\\
*+\txt{}
	&{q_{100}}\ar@/^20pt/[uu]^{\tt a}  \ar[ul]^{\tt b}
		&*+\txt{}
}
\label{nea-eps}
\end{equation}
Man kann ebenfalls verifzieren, dass dies tatsächlich ein DEA ist, 
von jedem Zustand gehen genau zwei Pfeile für die Übergänge mit
den Zeichen ${\tt a}$ und ${\tt b}$ aus.
Etwas übersichtlicher geschrieben lautet er
\[
\entrymodifiers={++[o][F]}
\xymatrix{
*+\txt{}\ar[r]
	&*++[o][F=]{1}\ar[r]^{\tt b} \ar[d]_{\tt a}
		&2 \ar[r]^{\tt a} \ar[d]^{\tt b}
			&6\ar@/_/[r]_{\tt a} \ar[dl]^{\tt b}
				&*++[o][F=]{7}\ar@/_/[l]_{\tt b} \ar@(ur,dr)^{\tt a}
\\
*+\txt{}
	&0 \ar@(l,d)_{{\tt a},{\tt b}}
		&4 \ar[ul]_{\tt a} \ar[l]^{\tt b}
}
\]

\subsubsection{NEA mit $\varepsilon$-Übergängen}
Wir erweitern die Konstruktion des DEA aus einem NEA jetzt, um eventuell
vorhandenen $\varepsilon$-Übergängen Rechnung zu tragen.

\begin{satz}
\label{satz_neadea}
Sei $A=(Q,\Sigma,\delta,q_0,F)$ ein NEA, dann gibt es einen
DEA $A'=(Q',\Sigma,\delta',q_0',F')$, der die gleiche Sprache
akzeptiert, $L(A)=L(A')$.
Es ist
\begin{align*}
Q'&=P(Q)\\
\delta'(M,a)&=E(\delta(M, a))\quad M\in P(Q)\\
q_0'&=E(q_0)\\
F'&=\{M\in P(Q)\;|\, M\cap F\ne \emptyset\}
\end{align*}
\end{satz}

\begin{proof}[Beweis]
Gegenüber Satz \ref{satz_neadea} hat sich geändert, dass wir den
Anfangszustand von $\{q_0\}$ auf $E(q_0)$ vergrössert haben.
Dadurch erfassen wir alle Pfade durch den Automaten, die mit einem oder
mehreren $\varepsilon$-Übergängen beginnen.
Falls $E(q_0)\cap F\ne \emptyset$
wird der Anfangszustand auch zu einem Endzustand.

Zudem erweitern wir in der Definition des Bildes $\delta(M,a)$ um alle Zustände,
die durch einen angehängten $\varepsilon$-Übergang erreicht werden
können.
\end{proof}

\subsubsection{Beispiel}
Der NEA 
\[
\entrymodifiers={++[o][F]}
\xymatrix{
*+\txt{}\ar[r]
	&*++[o][F=]{q_0} \ar[d]_{\tt b} \ar@/_/[r]_{\varepsilon}
		&{q_2}\ar@/_/[l]_{\tt a}
\\
*+\txt{}
	&{q_1}\ar[ur]_{{\tt a},{\tt b}} \ar@(ul,dl)_{\tt a}
}
\]
soll in einen DEA umgewandelt werden.

Den NEA ohne den $\varepsilon$-Übergang haben wir bereits in
\ref{nea-eps} umgewandelt.
Jetzt müssen wir nur die beiden
im Satz \ref{satz_neadea} erwähnten Modifikationen durchführen.

Ursprünglich war der Startzustand $q_{001}$.
Da aus dem Startzustand
des NEA mittels $\varepsilon$-Übergang auch der Zustand $q_2$
erreichbar ist, muss der Anfangszustand des DEA um diesen Zustand
erweitert werden und ist jetzt $q_{101}$:
\[
\entrymodifiers={++[o][F]}
\xymatrix{
*+\txt{}
	&*++[o][F=]{q_{001}} \ar[d]^{\tt b}\ar[dl]_{\tt a}
		&{q_{110}}\ar[dr]^{\tt a} \ar[ddl]^{\tt b}
\\
{q_{000}}\ar@(ul,dl)_{{\tt a},{\tt b}}
	&{q_{010}} \ar[ur]^{\tt a} \ar[d]^{\tt b}
		&*++[o][F=]{q_{101}} \ar[ul]^{\tt a} \ar[l]^{\tt b}
			&*++[o][F=]{q_{111}}\ar@(ur,dr)^{{\tt a}} \ar@/_20pt/[ul]_{\tt b}
\\
*+\txt{}
	&{q_{100}}\ar@/^20pt/[uu]^{\tt a} \ar[ul]^{\tt b}
		&*++[o][F=]{q_{011}} \ar@/_20pt/[uu]_{{\tt a},{\tt b}}
			&*+\txt{}\ar[ul]
}
\]


Zusätzlich müssen wir bei jedem Übergang die Möglichkeit in
Betracht ziehen, dass noch ein $\varepsilon$-Übergang angehängt
wird.
Dies bedeutet im vorliegenden Fall, dass bei jedem Übergang,
der zu einem $q_0$ enthaltenden Zustand führt, auch noch der
Zustand $q_2$ hinzugefügt werden muss.
\[
\entrymodifiers={++[o][F]}
\xymatrix{
*+\txt{}
	&*++[o][F=]{q_{001}} \ar[d]^{\tt b}\ar[dl]_{\tt a}
		&{q_{110}}\ar[dr]^{\tt a} \ar[ddl]^{\tt b}
\\
{q_{000}}\ar@(ul,dl)_{{\tt a},{\tt b}}
	&{q_{010}} \ar[ur]^{\tt a} \ar[d]^{\tt b}
		&*++[o][F=]{q_{101}} \ar@(ul,ur)^{\tt a} \ar[l]^{\tt b}
			&*++[o][F=]{q_{111}}\ar@(ur,dr)^{{\tt a}} \ar@/_20pt/[ul]_{\tt b}
\\
*+\txt{}
	&{q_{100}}\ar[ur]^{\tt a} \ar[ul]^{\tt b}
		&*++[o][F=]{q_{011}} \ar@/_20pt/[uu]_{{\tt a},{\tt b}}
			&*+\txt{}\ar[ul]
}
\]
In diesem Automaten sind die Zustände $q_{001}$ und $q_{011}$ nicht
mehr erreichbar, man kann sie weglassen.
\[
\entrymodifiers={++[o][F]}
\xymatrix{
*+\txt{}
	&*+\txt{}
		&{q_{110}}\ar[dr]^{\tt a} \ar[ddl]^{\tt b}
\\
{q_{000}}\ar@(ur,dr)^{{\tt a},{\tt b}}
	&{q_{010}} \ar[ur]^{\tt a} \ar[d]^{\tt b}
		&*++[o][F=]{q_{101}} \ar@(ul,ur)^{\tt a} \ar[l]^{\tt b}
			&*++[o][F=]{q_{111}}\ar@(ur,dr)^{{\tt a}} \ar@/_20pt/[ul]_{\tt b}
\\
*+\txt{}
	&{q_{100}}\ar[ur]^{\tt a} \ar[ul]^{\tt b}
		&*+\txt{}
			&*+\txt{}\ar[ul]
}
\]
Wie man mit dem Algorithmus über den Minimalautomaten kontrollieren
kann, lässt sich dieser Automat nicht mehr weiter reduzieren.
Etwas übersichtlicher geschrieben lautet er
\[
\entrymodifiers={++[o][F]}
\xymatrix{
*+\txt{}\ar[r]
	&*++[o][F=]{5}\ar@(ul,ur)^{\tt a} \ar[r]^{\tt b}
		&2\ar[r]^{\tt a} \ar[d]^{\tt b}
			&6\ar@/_/[r]_{\tt a} \ar[dl]^{\tt b}
				&*++[o][F=]{7}\ar@/_/[l]_{\tt b} \ar@(ur,dr)^{\tt a}
\\
*+\txt{}
	&0\ar@(ul,dl)_{{\tt a},{\tt b}}
		&4\ar[ul]^{\tt a} \ar[l]^{\tt b}
}
\]


Satz \ref{satz_neadea} ist nicht nur ein theoretisch interessantes
Resultat.
Wenn man die endliche vielen Zustände mit den Zahlen
$0,\dots,|Q|-1$ identifiziert, kann man die Teilmengen von $Q$ mit
den $|Q|$-stelligen Binärzahlen identifizieren.
Der Teilmenge $M\subset Q$ entsprechen die Binärzahlen, die Einsen an
den Stellen haben, deren Nummern in $M$ enthalten sind.
Die grosse Vereinigungen ist ebenfalls sehr einfach
zu berechnen, sie entspricht der Oder-Verknüpfung
der einzelnen Mengen in (\ref{erreichbar}).
Somit haben wir nicht nur ein Existenz-Resultat, sondern einen
Algorithmus, mit dem ein beliebiger NEA in einen DEA umgewandelt
werden kann.
Daraus können wir jetzt einen Vergleichsalgorithmus für reguläre
Sprachen ableiten: 

\begin{satz}
Es gibt einen Algorithmus, mit dem entschieden werden kann, ob
zwei Automaten $A$ und $B$ die gleiche Sprache akzeptieren.
\end{satz}

\begin{proof}[Beweis]
Um zu entscheiden, ob $L(A)=L(B)$, wendet man folgenden Algorithmus
an:
\begin{enumerate}
\item Wandle $A$ und $B$
mit dem Algorithmus des Satzes 
\ref{satz_neadea} in die DEAs $A'$ und $B'$ um.
\item Reduziert anschliessend $A'$ und $B'$  mit dem Algorithmus
von Satz \ref{satz_minimalautomat} in Minimalautomaten
$A''$ und $B''$.
\item Die von $A$ und $B$ akzeptierten Sprachen sind genau dann
gleich, wenn $A''$ und $B''$ identisch sind.
\end{enumerate}
\end{proof}

Der hier vorgeschlagene Beweis ist allerdings oft nicht praktikabel.
Der aus dem NEA erzeugte DEA hat im schlechtesten Fall $2^n$
Zustände, wenn der NEA $n$ Zustände hatte, die Laufzeit des
Algorithmus ist also exponentiell in der Grösse des NEA.
Ein wesentlich schnellerer Algorithmus wurde 2015 von Bonchi und Pous
gefunden \cite{skript:bonchi-pous}.

\subsection{Mengenoperationen\label{regulaer:mengenoperationen}}
\index{Mengenoperationen}%
\index{Durchschnitt}%
\index{Vereinigung}%
\index{Differenz}%
Sprachen sind Mengen von Wörtern, also sind auch deren Vereinigung,
Durchschnitt, Differenz usw.~Sprachen.
Sind die Sprachen regulär,
sind dann auch Vereinigung, Durchschnitt, Differenz etc.~regulär? 
NEAs erlauben uns sehr elegant zu zeigen, dass Sie diese Operationen
alle wieder reguläre Sprachen liefern.

\begin{satz}
\index{Vereinigung}%
\label{satz_union}
Sind $L_1$ und $L_2$ reguläre Sprachen, dann
ist auch $L_1\cup L_2$ regulär.
\end{satz}

\begin{figure}
\begin{center}
\includegraphics{images/nea-5}
\qquad \qquad
\includegraphics{images/nea-6}
\end{center}
\caption{Konstruktion eines NEA 
für die Sprache $L(A_1)\cup L(A_2)$.\label{regulaer:vereinigung}}
\end{figure}

\begin{proof}[Beweis]
Da die Sprachen regulär sind, gibt es endliche Automaten 
\begin{align*}
A_1&=(Q_1,\Sigma_1,\delta_1, q_{01}, F_1)\\
A_2&=(Q_2,\Sigma_2,\delta_2, q_{02}, F_2)
\end{align*}
mit $L_1=L(A_1)$ und $L_2=L(A_2)$.
Wir müssen
jetzt einen Automaten
\[
A = (Q, \Sigma, \delta, q_0, F)
\]
konstruieren, der $L_1\cup L_2$ akzeptieren
(Abbildung~\ref{regulaer:vereinigung}).

Für die Vereinigung muss das Alphabet alle Zeichen enthalten,
also $\Sigma = \Sigma_1\cup\Sigma_2$.
Man kann jeden der Automaten 
$A_1$ und $A_2$ auch als Automaten über dem Alphabet $\Sigma$
betrachten, indem man setzt
\[
\delta_i(q, x)=\begin{cases}
\delta_i(q,x)&\qquad q\in Q_i\text{ und } x\in \Sigma_i\\
\emptyset&\qquad q\in Q_i\text{ und } x\in \Sigma_{3-i}\setminus \Sigma_i\\
\end{cases}
\]
Es ist also keine Einschränkung, wenn wir annehmen, dass
$\Sigma=\Sigma_1=\Sigma_2$.

Um den Automaten für $L_1\cup L_2$ zu konstruieren, nehmen wir jetzt
weiter an, dass $Q_1$ und $Q_2$ disjunkt sind.
Für $A$ verwenden wir einen neuen Startzustand $q_0$.
Ein Wort in $L_1\cup L_2$ muss von einem der beiden Automaten
akzeptiert werden.
Der Automat muss sich also zu Beginn nichtdeterministisch
dafür entscheiden, den einen oder anderen Automaten zum
akzeptieren eines Wortes zu verwenden.
Also verwenden wir 
\begin{align*}
Q&=Q_1\cup Q_2\cup \{q_0\}\\
F&=F_1\cup F_2\\
\delta(x,a)&=\begin{cases}
\delta_1(x,a)&\qquad x\in Q_1\\
\delta_2(x,a)&\qquad x\in Q_2\\
\{q_{01}, q_{02}\}&\qquad x=q_0, a=\varepsilon\\
\emptyset&\qquad x=q_0, a\ne\varepsilon
\end{cases}
\end{align*}
\end{proof}

\begin{satz}
\index{Durchschnitt}%
\label{satz_intersection}
Sind $L_1$ und $L_2$ reguläre Sprachen, dann
ist auch $L_1\cap L_2$ regulär.
\end{satz}

\begin{proof}[Beweis]
Da die Sprachen regulär sind, gibt es endliche Automaten 
\begin{align*}
A_1&=(Q_1,\Sigma_1,\delta_1, q_{01}, F_1)\\
A_2&=(Q_2,\Sigma_2,\delta_2, q_{02}, F_2)
\end{align*}
mit $L_1=L(A_1)$ und $L_2=L(A_2)$.
Wir müssen jetzt einen Automaten
\[
A = (Q, \Sigma, \delta, q_0, F)
\]
konstruieren, der $L_1\cap L_2$ akzeptiert.

Für die Schnittmenge brauchen wir einen Automaten, der die Abläufe 
in beiden Teilautomaten gleichzeitig codiert.
Dies ist möglich, indem man Zustände und Übergänge nebeneinander
simuliert.
Als Zustandsmenge verwenden wir daher $Q=Q_1\times Q_2$.
Als Übergangsfunktion verwenden wir
\[
\delta((q',q''),a)=\delta_1(q',a)\times \delta_2(q'',a).
\]
Akzeptiert werden kann ein Wort, wenn beide Komponenten des Paares 
Akzeptierzustände ihrer Automaten sind.
Akzeptierzustände von $A$ sind also $F=F_1\times F_2$.
\end{proof}

\index{Produktautomat}%
Wir nennen diesen mit Hilfe des kartesischen Produktes konstruierten
Automaten auch den
{\em kartesischen Produktautomaten}\label{reg_produktautomat}.

\subsubsection{Beispiel}
Zur Illustration wollen 
wir einen Automaten für die Schnittmenge von
\begin{align*}
L_1&=\{w\in\Sigma^*\;|\; \text{$|w|_0$ gerade}\}\qquad\text{und}\\
L_2&=\{w\in\Sigma^*\;|\,\text{$w$ ist eine durch 3 teilbare Binärzahl}\}.
\end{align*}
konstruieren.
Die einzelnen Teilautomaten sind $A_1$ für $L_1$:
\[
\entrymodifiers={++[o][F]}
\xymatrix{
*+\txt{}\ar[r]
	&*++[o][F=]{} \ar@/^/[r]^{\tt 0} \ar@(ul,ur)^{\tt 1}
		&{} \ar@/^/[l]^{\tt 0} \ar@(ul,ur)^{\tt 1}
}
\]
und $A_2$ für $L_2$
\[
\entrymodifiers={++[o][F]}
\xymatrix{
*+\txt{}\ar[r]
	&*++[o][F=]{0} \ar@/^/[r]^{\tt 1} \ar@(ul,ur)^{\tt 0}
		&{1} \ar@/^/[l]^{\tt 1} \ar@/^/[r]^{\tt 0}
			&{2}\ar@(ur,dr)^{\tt 1} \ar@/^/[l]^{\tt 0}
}
\]
Der Produktautomat ist jetzt
\[
\entrymodifiers={++[o][F]}
\xymatrix{
*+\txt{} \ar[r] \ar[d] \ar[dr]
	&*++[o][F=]{0} \ar@/^/[r]_{\tt 1} \ar@(ul,ur)^{\tt 0}
		&{1} \ar@/^/[l] \ar@/^/[r]_{\tt 0}
			&{2}\ar@(ur,dr)^{\tt 1} \ar@/^/[l]
\\
*++[o][F=]{} \ar@/^/[d] \ar@(ul,dl)_{\tt 1}
	&*++[o][F=]{} \ar@/^/[d]{\tt 0} \ar@/^/[r]_{\tt 1}
		&{} \ar@/^/[l] \ar@/^/[dr]_{\tt 0}
			&{}\ar@(ur,dr)^{\tt 1} \ar@/^/[dl]
\\
{} \ar@/^/[u]_{\tt 0} \ar@(ul,dl)_{\tt 1}
	&{} \ar@/^/[u]_{\tt 0} \ar@/^/[r]_{\tt 1}
		&{} \ar@/^/[l] \ar@/^/[ur]
			&{}\ar@(ur,dr)^{\tt 1} \ar@/^/[ul]
}
\]

Man kann den Produktautomaten auch
verwenden, um einen Automaten zu bauen, der $L_1\cup L_2$
akzeptiert.
Der kartesische Produktautomat simuliert ja sozusagen
beide Teilautomaten in den beiden Komponenten der Zustände.
Akzeptabel ist ein Wort $w$, wenn es von $A_1$
oder von $A_2$ akzeptiert ist, oder wenn der Produktautomat
sich am Ende des Wortes in einem Zustand befindet, der ein
Akzeptierzustand von $A_1$ ist oder von $A_2$.
Verwendet
man also $F_1\times Q_2\cup Q_1\times F_2$ als Akzeptierzustände,
erhält man einen Automaten, der $L_1\cup L_2$ akzeptiert.
Das zugehörige Zustandsdiagramm ist:
\[
\entrymodifiers={++[o][F]}
\xymatrix{
*+\txt{} \ar[r] \ar[d] \ar[dr]
	&*++[o][F=]{0} \ar@/^/[r]_{\tt 1} \ar@(ul,ur)^{\tt 0}
		&{1} \ar@/^/[l] \ar@/^/[r]_{\tt 0}
			&{2}\ar@(ur,dr)^{\tt 1} \ar@/^/[l]
\\
*++[o][F=]{} \ar@/^/[d] \ar@(ul,dl)_{\tt 1}
	&*++[o][F=]{} \ar@/^/[d]{\tt 0} \ar@/^/[r]_{\tt 1}
		&*++[o][F=]{} \ar@/^/[l] \ar@/^/[dr]_{\tt 0}
			&*++[o][F=]{}\ar@(ur,dr)^{\tt 1} \ar@/^/[dl]
\\
{} \ar@/^/[u]_{\tt 0} \ar@(ul,dl)_{\tt 1}
	&*++[o][F=]{} \ar@/^/[u]_{\tt 0} \ar@/^/[r]_{\tt 1}
		&{} \ar@/^/[l] \ar@/^/[ur]
			&{}\ar@(ur,dr)^{\tt 1} \ar@/^/[ul]
}
\]

\begin{satz}
\index{Differenz}%
\label{satz_regcomplement}
Ist $L$ eine reguläre Sprache, dann ist auch $\bar L$ regulär.
Sind $L_1$ und $L_2$ regulär, dann ist auch $L_1\setminus L_2$
regulär.
\end{satz}

\begin{proof}[Beweis]
Da $L$ regulär ist, gibt es einen DEA, welcher $L=L(A)$ erfüllt.
Wir ersetzen in diesem DEA die Menge $F$ der Akzeptierzustände
durch $Q\setminus F$ und erhalten einen neuen DEA $A'$.
Dieser neue DEA akzeptiert genau diejenigen Wörter, die $A$ nicht
akzeptiert hat, also $L(A')=\overline{L(A)}$.
Somit ist $\bar L=L(A')$ regulär.

Die zweite Aussage folgt aus $L_1\setminus L_2=L_1\cap\bar L_2$ und
Satz \ref{satz_intersection}
\end{proof}
Man beachte, dass in diesem Beweis unbedingt ein DEA verwendet werden
muss.
Die beiden NEAs
\[
\entrymodifiers={++[o][F]}
\xymatrix{
*+\txt{}\ar[r]
	&{}\ar@(ul,ur)^{\Sigma} \ar[r]^{\varepsilon}
		&*++[o][F=]{}
&
*+\txt{}\ar[r]
	&*++[o][F=]{}\ar@(ul,ur)^{\Sigma} \ar[r]^{\varepsilon}
		&{}
}
\]
gehen auseinander durch Ersetzung der Akzeptierzustände
$F\leftrightarrow Q\setminus F$ hervor, aber beide akzeptieren
$\Sigma^*$, also nicht das Komplement.

\rhead{Reguläre Ausdrücke}
\section{Reguläre Ausdrücke\label{regulaer:re}}
\index{reguläre Ausdrücke}%
Endliche Automaten beschreiben reguläre Sprachen, sind aber für die
Anwendung eher unhandlich.
In der Praxis haben sich reguläre Ausdrücke
durchgesetzt, die mit einer einfachen Syntax ebenfalls Mengen von
Wörtern zu spezifizieren erlauben.
Ziel dieses Abschnittes ist
zu zeigen, dass reguläre Ausdrücke genau gleich ausdrucksstark sind
wie DEAs.

\subsection{Reguläre Operationen\label{regulaer:regulaere-operationen}}
\index{reguläre Operationen}%
Wir wissen bereits, dass wir die Mengenoperationen auf reguläre
Sprachen anwenden dürfen, ohne die Klasse der regulären Sprachen
zu verlassen.
Es sind jedoch noch zwei weitere Operationen möglich, die ebenfalls
nicht aus der Klasse herausführen:

\begin{definition}
\index{Verkettung}%
Seien $L_1$ und $L_2$ Sprachen, dann ist die
Verkettung von $L_1$ und $L_2$ die Sprache
\[
L_1L_2=\{w_1w_2\;|\;w_1\in L_1,w_2\in L_2\}.
\]
Die mehrfache Verkettung von $L$ wird mit $L^n$ bezeichnet:
\begin{align*}
L^0&=\{\varepsilon\}\\
L^n&=L^{n-1}L
\end{align*}
\end{definition}

\begin{definition}
\index{*-Operation@$*$-Operation}%
Sei $L$ eine Sprache, dann ist die Stern-Operation
von $L$ die Sprache
\[
L^*=\bigcup_{k\in\mathbb N} L^k.
\]
\end{definition}

Wir möchten jetzt zeigen, dass diese Operationen nicht aus den
regulären Sprachen herausführen.
Dazu müssen wir zu gegebenen Automaten $A_1$ und $A_2$ für zwei Sprachen
$L_1$ und $L_2$ neue Automaten konstruieren, welche
die Sprachen $L_1L_2$ oder $L_1^*$ akzeptieren.
Die Automaten $A_1$ und $A_2$ müssen wir dabei als ``Black Box'' betrachten,
wir dürfen daran nichts ändern.

\subsubsection{Automat zu einer Verkettung $L_1L_2$}
\index{Verkettung}%
Für die Verkettung $L_1L_2$ ist es zum Beispiel
nicht zulässig, die Akzeptierzustände von
$A_1$ mit dem Startzustand von $A_2$ zusammenzulegen.
Dadurch
würde es nämlich möglich, auch Wege wieder zurück in
den ersten Automaten zu verwenden.
Im folgenden Beispiel akzeptiert der erste Automat Strings mit
einer ungeraden Anzahl Nullen, der zweite Strings mit einer
ungeraden Anzahl {\tt a}.
\[
\entrymodifiers={++[o][F]}
\xymatrix{
*+\txt{}\ar[r]
	&{} \ar@/^/[r]^{\tt 0}
		&*++[o][F=]{} \ar@/^/[l]^{\tt 0}
			&*\txt{}\ar[r]
				&{} \ar@/^/[r]^{\tt a}
					&*++[o][F=]{} \ar@/^/[l]^{\tt a }
}
\]
Verkettet man die Automaten, indem man den Akzeptierzustand des
ersten mit dem Startzustand des zweiten Automaten zusammenlegt,
erhält man
\[
\entrymodifiers={++[o][F]}
\xymatrix{
*+\txt{}\ar[r]
	&{} \ar@/^/[r]^{\tt 0}
		&{} \ar@/^/[l]^{\tt 0}
				 \ar@/^/[r]^{\tt a}
					&*++[o][F=]{} \ar@/^/[l]^{\tt a }
}
\]
Dieser Automat akzeptiert aber auch das Wort {\tt 0aa00a}, welches
gar nicht in $L_1L_2$ ist! Die Verkettung muss also mit Hilfe
einer ``Einbahnstrasse'' erfolgen, ein $\varepsilon$-Übergang
ist ideal dafür geeignet.
Die folgende Verkettung funktioniert:
\[
\entrymodifiers={++[o][F]}
\xymatrix{
*+\txt{}\ar[r]
	&{} \ar@/^/[r]^{\tt 0}
		&{} \ar@/^/[l]^{\tt 0} \ar[r]^{\varepsilon}
			&{} \ar@/^/[r]^{\tt a}
				&*++[o][F=]{} \ar@/^/[l]^{\tt a }
}
\]

Für die allgemeine Konstruktion gehen wir aus
von zwei Automaten $A_1$ und $A_2$:
\begin{center}
\includegraphics{images/nea-1}
\end{center}
Ein Automat für die Verkettung entsteht, indem man
die Akzeptierzustände im ersten Automaten über
$\varepsilon$-Übergänge mit dem
Startzustand des zweiten verbindet:
\begin{center}
\includegraphics{images/nea-2}
\end{center}
Die $\varepsilon$-Übergänge
fungieren als Einbahnstrassen und stellen sicher, auch innerhalb der
einzelnen Automaten keine neuen akzeptierten Wörter entstehen können.

Etwas formeller kann man das Resultat im folgenden Satz zusammenfassen.

\begin{satz}
\index{Verkettung}%
\label{satz_concat}
Sind $L_1$ und $L_2$ reguläre Sprachen, dann ist auch $L_1L_2$
regulär.
\end{satz}

\begin{proof}[Beweis]
Ein Wort $w\in L_1L_2$ besteht aus zwei Teilen, die von $L_1$
bzw.~$L_2$ akzeptiert werden.
Ein Automat $A$ für $L=L_1L_2$ kann aus Automaten $A_1$ und $A_2$
konstruiert werden, indem man
die Akzeptierzustände von $A_1$ über $\varepsilon$-Übergänge
mit dem Startzustand von $A_2$ verbindet.
Also verwenden wir $Q=Q_1\cup Q_2$.
Startzustand ist der Startzustand $q_{01}$ von $L_1$, Akzeptierzustände
sind die Akzeptierzustände $F_2$ von $A_2$.
Also
\begin{align*}
Q&=Q_1\cup Q_2\\
F&=F_2\\
q_0&=q_{01}\\
\delta(q,a)&=\begin{cases}
\delta_1(\varepsilon,a)\cup\{q_{02}\}&\qquad q\in F_1 \wedge a=\varepsilon\\
\delta_1(q,a)                        &\qquad q\in F_1 \wedge a\ne\varepsilon
\quad\text{oder}\quad q\in Q_1\setminus F_1\\
%\\
%\delta_1(q,a)                        &\qquad q\in Q_1\setminus F_1\\
\delta_2(q,a)                        &\qquad q\in Q_2
\end{cases}
\end{align*}
\end{proof}

\subsubsection{Automat zur *-Operation $L^*$}
\index{*-Operation@$*$-Operation}%
Die Stern-Operation verlangt, dass man einen Automaten mehrfach
durchlaufen können muss.
Dies könnte man mit einem $\varepsilon$-Übergang
von den Akzeptierzuständen direkt zum Startzustand realisieren.
Dadurch ändert man aber den Automaten für die Sprache $L$,
der Automat ist nicht mehr eine wiederverwendbare Einheit.

Wir streben an, die regulären Operationen ohne interne Änderungen
des Automaten zu konstruieren.
Insbesondere dürfen dem Automaten keine neuen Übergänge hinzugefügt
werden.
Es ist hingegen zulässig, einen Akzeptierzustand zu einem
Nichtakzeptierzustand zu degradieren.
Ob nämlich ein Zustand als Akzeptierzustand interpretiert wird,
hängt davon ab, was ein zusammengesetzter Automat mit der Information
macht, dass ein Teilautomat sich in einem Akzeptierzustand befindet.

Zur Verdeutlichung stelle man sich eine Automaten-Objekt $A$ vor mit
folgendem Interface
\begin{verbatim}
interface Automat {
        // Zeichen verarbeiten
        void    process(char zeichen);
        // testen, ob sich der Automat in einem Akzeptierzustand befindet
        boolean accept();
        // in den Startzustand zurueckversetzen
        void    reset();
}
\end{verbatim}
Offensichtlich enthält dieses Interface keine Möglichkeit, den Automaten
zu modifizieren, oder auch nur den inneren Zustand auf eine andere Art
zu ändern als mit einem normalen Übergang.
Das Interface genügt aber, um die *-Operation auszuführen.
Um ein Wort in $L^*$ zu akzeptieren, muss man nur jedes Zeichen
des Wortes in den Automaten $A$ einspeisen und fragen, ob er sich
in einem Akzeptierzustand befindet.
Falls ja, darf man das Teilwort
akzeptieren und den Automaten in den Startzustand zurückversetzen.

Für die *-Operation darf also der Automat
\begin{center}
\includegraphics{images/nea-3}
\end{center}
nicht modifiziert werden, insbesondere darf der naheligende
$\varepsilon$-Übergang von einem Akzeptierzustand zum Startzustand
nicht hinzugefügt werden.
Um das leere Wort muss daher ein zusätzlicher Zustand hinzugefügt werden.
Und für die Wiederholung darf kann man diesen neuen Zustand ebenfalls
verwenden, indem man von den Akzeptierzuständen von $A$ zu diesem
neuen Zustand zurückkehrt.
So erhält man den Automaten
\begin{center}
\includegraphics{images/nea-4}
\end{center}

Formal kann man das Resultat wie folgt zusammenfassen:
\begin{satz}
\index{*-Operation@$*$-Operation}%
\label{satz_star}
Ist $L$ eine reguläre Sprache, dann ist auch $L^*$ regulär.
\end{satz}

Zum Beweis reicht es nicht, den Satz \ref{satz_concat} zu verwenden
um zu zeigen, dass $L^n$ regulär ist, und dann aus wiederholter
Anwendung Satz \ref{satz_union} zu schliessen zu versuchen,
dass die Vereinigung aller $L^n$ auch regulär sei.
Da dies eine unendliche
Vereinigung ist, würde nach der Konstruktion im Beweis von Satz
\ref{satz_union} ein Automat mit unendlich vielen
Zuständen entstehen, also keinen NEA.

\begin{proof}[Beweis]
Aus dem Automaten
\[
A=(Q,\Sigma, \delta,q_0,F)
\]
mit $L(A)=L$ konstruieren wir einen neuen
Automaten
\[
A'=(Q',\Sigma,\delta',q',F')
\]
mit einem neuen Anfangszustand $q'$, der auch
ein Akzeptierzustand ist.
Damit wird sichergestellt, dass der neue Automat das in $L^0$
enthaltene leere Wort akzeptiert.
Zusätzlich wird der neue Anfangszustand mit einem $\varepsilon$-Übergang
mit $q'$ verbunden.
Ebenso werden alle Akzeptierzustände über $\varepsilon$-Übergänge
mit dem neuen Startzustand verbunden:
\begin{align*}
Q'&=Q\cup \{q'\}\\
F'&=\{q'\}\\
\delta'(q',a)&=\emptyset\\
\delta'(q',\varepsilon)&= \{q_0\}\\
\delta'(q,a)&= \delta(q,a)\\
\delta'(q,\varepsilon)&=\begin{cases}
\delta(q,\varepsilon)          &\qquad q\not\in F\\
\delta(q,\varepsilon)\cup\{q'\}&\qquad q\in F
\end{cases}
\end{align*}
\end{proof}
\subsubsection{Variante: mindestens eine Wiederholung}
Die Sprache $LL^*$ besteht aus Verkettungen von Wörtern aus $L$,
im Unterschied zu $L^*$ kommt aber mindestens ein Wort vor.
Einen Automaten dafür kann man auf verschiedene Arten gewinnen,
sie laufen aber alle im Wesentlichen auf die Lösung
\begin{center}
\includegraphics{images/nea-7}
\end{center}
hinaus.
Wesentlich ist, dass man den Automaten durchlaufen muss, bevor
etwas akzeptiert werden kann, und dass man von den Akzeptierzuständen
auch wieder zum Startzustand zurückkehren kann.
Dabei darf allerdings
kein ``Kurzschluss'' entstehen, man muss eine Einbahnstrasse aus
$\varepsilon$-Übergängen verwenden.

\subsubsection{Automat zur Alternative $L_1\cup L_2$}
\index{Alternative}%
\index{Vereinigung}%
Für die Alternative $L_1\cup L_2$ hatten wir bei der Diskussion
des Produkt-Automaten schon einen Automaten gefunden, auch
hierfür lässt sich ein etwas einfacherer Automat im gleichen
Stil wie für Verkettung und *-Operation finden.
Ein neuer Startzustand
wird dazu mit $\varepsilon$-Übergängen mit den Startzuständen
der Automaten $A_1$ und $A_2$ verbunden, wie in Abbildung~\ref{neaalternative}.
\begin{figure}
\begin{center}
\includegraphics{images/nea-5}
\qquad
\qquad
\includegraphics{images/nea-6}
\end{center}
\caption{Automatenkonstruktion für die Alternative, rechts der Automat
für die Sprache $L(A_1)\cup L(A_2)$\label{neaalternative}}
\end{figure}

\subsubsection{Reguläre Operationen}
Damit haben wir alle drei regulären Operationen kennengelernt:

\begin{definition}
Die drei Operationen
\begin{enumerate}
\item Vereinigung $(L_1,L_2)\mapsto L_1\cup L_2$
\item Verkettung $(L_1,L_2)\mapsto L_1L_2$ und
\item Stern-Operation $L\mapsto L^*$
\end{enumerate}
heissen reguläre Operationen.
\end{definition}

Diese Operationen genügen, um alle regulären Sprachen aus
einelementigen Teilmengen von $\Sigma$ aufzubauen.
Dazu werden wir im nächsten Abschnitt eine prägnantere Notation,
die regulären Ausdrücke definieren.
Danach werden wir zeigen,
dass sich jeder DEA in einen regulären Ausdruck umwandeln lässt.

\subsection{Reguläre Ausdrücke\label{regulaer:regulaere-ausdruecke}}
\index{reguläre Ausdrücke}%
\index{Ausdrücke!reguläre}%
\begin{table}
\begin{center}
\begin{tabular}{|l|l|}
\hline
Ausdruck $r$&Bedeutung\\
\hline
{\tt a}&steht für das Zeichen ${\tt a}\in \Sigma$\\
{\tt .}&steht für ein beliebiges Zeichen aus $\Sigma$\\
{\tt [aeiou]}&steht für ein Zeichen aus $\{{\tt a},{\tt e},{\tt i},{\tt o},{\tt u}\}\subset \Sigma$\\
{\tt [1-9]}&steht für die positiven Ziffern\\
$\varepsilon$&steht für das leere Wort\\
$\emptyset$&steht für die leere Sprache\\
\hline
\end{tabular}
\end{center}
\index{reguläre Ausdrücke!primitive}%
\index{Ausdrücke!reguläre!primitive}%
\caption{Primitive reguläre Ausdrücke.\label{regtab1}}
\end{table}
Reguläre Ausdrücke sind Zeichenketten, die (reguläre) Sprachen
beschreiben.
Die einfachsten regulären Sprachen sind die Sprachen mit Wörtern,
die nicht länger als 1 Zeichen sind.
Reguläre Ausdrücke für
diese primitiven Sprachen werden in Tabelle~\ref{regtab1} zusammengestellt.

Reguläre Ausdrücke bauen reguläre Sprachen aus einzelnen
Zeichen und regulären Operationen aus.
Ist $r$ ein regulärer Ausdruck, dann schreiben wir  $L(r)$ für die
Sprache, die vom regulären Ausdruck $r$ beschrieben wird.
Für die regulären Operationen verwenden wir die Notation
gemäss Tabelle~\ref{regtab2}
\begin{table}
\begin{center}
\begin{tabular}{|l|c|l|}
\hline
&Ausdruck&reguläre Operation\\
\hline
\index{Verkettung}%
Verkettung&$r_1r_2$&$L(r_1)L(r_2)$\\
\index{Alternative}%
Alternative&$r_1{\tt |}r_2$&$L(r_1)\cup L(r_2)$\\
\index{*-Operation@$*$-Operation}%
Stern-Operation&$r{\mathstrut^{\tt *}}$&$L(r)^*$\\
\hline
\end{tabular}
\end{center}
\caption{Notation für reguläre Operationen\label{regtab2}}
\end{table}
Falls nötig können Klammern verwendet werden, um anzuzeigen,
welche Gruppen verknüpft werden.
Da mit regulären Operationen
verknüpfte Sprachen wieder regulär sind, können alle mit
regulären Ausdrücken beschriebenen Sprachen von einem DEA
akzeptiert werden.

\subsubsection{Automaten für die primitiven regulären Ausdrücke}
Die primitiven regulären Ausdrücke aus Tabelle~\ref{regtab1} können
von folgenden Automaten akzeptiert werden:
\begin{enumerate}
\item Ein einzelnes Zeichen {\tt a}:
\[
\entrymodifiers={++[o][F]}
\xymatrix{
*+\txt{}\ar[r]
	&{}\ar[r]^{\tt a}
		&*++[o][F=]{}
}
\]
\item Sprache $L=\Sigma$, regulärer Ausdruck $r={\tt .}$:
\[
\entrymodifiers={++[o][F]}
\xymatrix{
*+\txt{}\ar[r]
	&{}\ar[r]^{\Sigma}
		&*++[o][F=]{}
}
\]
\item Das leere Wort:
\[
\entrymodifiers={++[o][F]}
\xymatrix{
*+\txt{}\ar[r]
	&*++[o][F=]{}
}
\]
\item Die leere Sprache:
\[
\entrymodifiers={++[o][F]}
\xymatrix{
*+\txt{}\ar[r]
	&{}
}
\]
\end{enumerate}

\subsubsection{Automat eines regulären Ausdrucks}
Mit Hilfe der Automaten für die primitven regulären Ausdrücke
und den regulären Operationen kann man jetzt zu jedem regulären
Ausdruck einen NEA aufbauen.

\begin{beispiel}[\bf Beispiel 1] NEA des regulären Ausdrucks
${\tt ab}|{\tt cd}$.

Zunächst brauchen wir einen NEA für die Verkettung ${\tt ab}$.
Diesen können wir mit der Verkettungskonstruktion aus primitiven
NEAs für {\tt a} und {\tt b} bilden:
\[
\entrymodifiers={++[o][F]}
\xymatrix{
*+\txt{}\ar[r]
	&{}\ar[r]^{\tt a}
		&{}\ar[r]^{\varepsilon}
			&{}\ar[r]^{\tt b}
				&*++[o][F=]{}
}
\]
Auf die gleiche Art kann auch ein NEA für ${\tt cd}$ gebildet werden.
Diese müssen jetzt mit Hilfe der Konstruktion für die Alternative
zu einem NEA für den Gesamtausdruck zusammengesetzt werden:
\[
\entrymodifiers={++[o][F]}
\xymatrix{
*+\txt{}
	&*+\txt{}
		&{}\ar[r]^{\tt a}
			&{}\ar[r]^{\varepsilon}
				&{}\ar[r]^{\tt b}
					&*++[o][F=]{}
\\
*+\txt{} \ar[r]
	&{}\ar[ur]^{\varepsilon} \ar[dr]^{\varepsilon}
\\
*+\txt{}
	&*+\txt{}
		&{}\ar[r]^{\tt c}
			&{}\ar[r]^{\varepsilon}
				&{}\ar[r]^{\tt d}
					&*++[o][F=]{}
}
\]
\end{beispiel}

\begin{beispiel}[\bf Beispiel 2] NEA für den regulären Ausdruck
$({\tt ab})^*{\tt c}$.

Die einzelnen Teile sind uns schon bekannt, wir müssen nur noch
die *-Konstruktion auf den verketten Automaten anwenden:
\[
\entrymodifiers={++[o][F]}
\xymatrix{
*+\txt{}\ar[r]
	&{}\ar[r]^{\varepsilon}
		&{}\ar[r]^{\varepsilon}
			\ar[d]^{\varepsilon}
			&\ar[r]_{\tt a}
				&{}\ar[r]^{\varepsilon}
					&{}\ar[r]_{\tt b}
						&{}\ar@/_20pt/[llll]_{\varepsilon}
\\
*+\txt{}
	&*+\txt{}
		&{}\ar[r]^{\tt c}
			&*++[o][F=]{}
}
\]
\end{beispiel}

\begin{beispiel}[\bf Beispiel 3:] NEA für den regulären Ausdruck
$({\tt ab})+$.

Das Zeichen ${\tt+}$ bedeutet bei regulären Ausdrücken ``mindestens ein''.
Die einfachste Modifikation der *-Konstruktion, die dies leistet, besteht
darin, den Akzeptierzustand ans Ende des Automaten zu verschieben:
\[
\entrymodifiers={++[o][F]}
\xymatrix{
*+\txt{}\ar[r]
	&{}\ar[r]^{\varepsilon}
		&{}\ar[r]^{\tt a}
			&{}\ar[r]^{\varepsilon}
				&{}\ar[r]^{\tt b}
					&*++[o][F=]{}\ar@/^20pt/[llll]^{\varepsilon}
}
\]
Alternativ könnte man auch einfach den NEA für die Verkettung
nehmen und keinen neuen Startzustand hinzufügen.
Bei der
*-Konstruktion wurde das ja gemacht, um auch das leere Wort
akzeptieren zu können, was bei der {\tt +}-Operation nicht
nötig ist:
\[
\entrymodifiers={++[o][F]}
\xymatrix{
*+\txt{}\ar[r]
	&{}\ar[r]^{\tt a}
		&{}\ar[r]^{\varepsilon}
			&{}\ar[r]^{\tt b}
				&*++[o][F=]{}\ar@/^20pt/[lll]^{\varepsilon}
}
\]
\end{beispiel}
Diese Beispiele illustrieren, dass sich aus jedem regulären Ausdruck
ein NEA bauen lässt.
Die Konstruktion hat die Form eines Algorithmus,
lässt sich also auch mit einem Computer implementieren, sie geht
auf Ken Thompson zurück.
\index{Thompson, Ken}%

\subsection{Regulärer Ausdruck eines DEA\label{regulaer:dea-re}}
Zu jedem DEA gibt es einen regulären Ausdruck, der angibt,
welche Wörter er akzeptiert.
In diesem Abschnitt geben
wir einen Algorithmus an, der einen DEA in einen äquivalenten
regulären Ausdruck umwandelt.

\index{NEA!verallgemeinerter}%
\index{VNEA}%
Wir verwenden dazu den Begriff eines verallgemeinerten NEA (VNEA),
dessen Pfeile nicht mehr nur mit Zeichen des Alphabets oder mit
$\varepsilon$ angeschrieben sei können, sondern mit regulären
Ausdrücken.
Dann wandelt man einen DEA zunächst um in einen
VNEA und reduziert ihn dann auf einen VNEA mit nur zwei Zuständen,
einem Startzustand und einem Akzeptierzustand:
\[
\entrymodifiers={++[o][F]}
\xymatrix{
*+\txt{}\ar[r]
	&{q_0}\ar[r]^{r}
		&*++[o][F=]{q_1}
}
\]
Offenbar werden genau diejenigen Wörter von diesem Automaten
akzeptiert, welche auf den regulären Ausdruck $r$ passen.

Sei jetzt also ein NEA $A$ gegeben.
Da er möglicherweise viele
Endzustände haben kann, wir aber am Schluss nur noch einen
Endzustand haben wollen, fügen wir einen neuen Endzustand
$q_{\text{accept}}$ hinzu.
Da der Automat auch Pfeile haben kann, die im Startzustand enden,
der endgültige VNEA aber keine solche Pfeile hat, fügen wir
auch einen zusätzlichen Startzustand $q_{\text{start}}$ hinzu.

%Natürlich müssen wir jetzt auch noch Pfeile ergänzen. Zusätzlich
%zu den ursprünglichen Pfeilen des NEA ergänzen wir darin
%alle möglichen Pfeile, die es bis jetzt noch nicht gab, und
%beschriften Sie mit dem regulären Ausdruck $\emptyset$, was bedeutet,
%dass diese Pfeile gar nie ``genommen'' werden.

Ausserdem fügen wir vom neuen Startzustand $q_{\text{start}}$
aus einen Pfeil zum ursprünglichen Startzustand des NEAs mit Beschriftung $\varepsilon$ hinzu.

Alle Akzeptierzustände des ursprünglichen NEAs werden jetzt noch mit
$q_{\text{accept}}$ verbunden, beschriftet mit $\varepsilon$.

Damit haben wir jetzt einen VNEA.
Diesen müssen wir jetzt auf einen VNEA reduzieren, der nur noch
die beiden Zustände $q_{\text{start}}$ und $q_{\text{accept}}$
enthält.
Dazu reissen wir nacheinander alle Zustände des ursprünglichen
NEA heraus.
Die entstehenden Löcher müssen wir natürlich reparieren, indem wir die
verbleibenden Pfeile mit erweiterten regulären Ausdrücken
anschreiben, welche die verlorengegangenen Teile ersetzen.

Nehmen wir also an, wir möchten den Zustand $q_ {\text{rip}}$
herausreissen.
Seien weiter $q_i$ und $q_j$ zwei Zustände, die
Übergänge haben, die über $q_{\text{rip}}$ führen:
\[
\entrymodifiers={++[o][F]}
\xymatrix{
{q_i}\ar[rr]^{r_4} \ar[dr]_{r_1}
	&*+\txt{}
		&{q_j}
\\
*+\txt{}
	&{q_{\text{rip}}}\ar[ur]_{r_3} \ar@(dl,dr)_{r_2}
}
\]
Der Weg über $q_{\text{rip}}$ entspricht einem Übergang
von $q_i$ nach $q_j$ mit regulärem Ausdruck $r_1r_2^*r_3$,
der zusätzlich zum Ausdruck $r_4$, möglich ist.
Wenn $q_{\text{rip}}$ entfernt wird, muss also $r_4$ durch 
\[
r_4{\tt |}r_1r_2^{\tt *}r_3
\]
ersetzt werden:
\[
\entrymodifiers={++[o][F]}
\xymatrix{
{q_i}\ar[rr]^{r_4|r_1r_2^*r_3}
	&*+\txt{}
		&{q_j}
}
\]
Die vom VNEA akzeptierte Sprache ändert sich dabei nicht.
Diese Operation wird wiederholt, bis die Zustände des ursprünglichen
NEA vollständig entfernt sind.

Damit  haben wir jetzt folgenden Satz bewiesen.
\begin{satz}
Ist $A$ ein NEA, dann gibt es einen regulären Ausdruck $r$ mit
$L(A)=L(r)$.
\end{satz}

\subsubsection{Beispiel}
Wir wollen den DEA
\[
\entrymodifiers={++[o][F]}
\xymatrix{
*+\txt{}\ar[r]
	&{1}\ar@/^/[r]^{\tt a} \ar@/^/[d]^{\tt b}
		&*++[o][F=]{2} \ar@(ul,ur)^{\tt b} \ar@/^/[l]^{\tt a}
\\
*+\txt{}
	&*++[o][F=]{3} \ar[ur]_{\tt a} \ar@/^/[u]^{\tt b}
}
\]
in einen äquivalenten regulären Ausdruck umwandeln.

Im ersten Schritt fügen wir die neuen Start- und Akzeptierzustände 
hinzu, die Zustände $2$ und $3$ sind damit nicht mehr Akzeptierzustände:
\[
\entrymodifiers={++[o][F]}
\xymatrix{
*+\txt{}\ar[r]
	&S\ar[r]^{\varepsilon}
		&{1}\ar@/^/[r]^{\tt a} \ar@/^/[d]^{\tt b}
			&{2} \ar@(ul,ur)^{\tt b} \ar@/^/[l]^{\tt a}\ar[r]^{\varepsilon}
				&*++[o][F=]{A}
\\
*+\txt{}
	&*+\txt{}
		&{3} \ar[ur]_{\tt a} \ar@/^/[u]^{\tt b} \ar[urr]^{\varepsilon}
}
\]
Jetzt entfernen wir nacheinander alle ``inneren'' Zustände.
Wir beginnen mit dem Zustand $1$, dabei sind die Pfade
$S$-$1$-$2$,
$S$-$1$-$3$,
$2$-$1$-$3$,
$3$-$1$-$2$,
$2$-$1$-$2$
und $3$-$1$-$2$
zu berücksichtigen
\[
\entrymodifiers={++[o][F]}
\xymatrix{
*+\txt{}\ar[r]
	&S\ar[rr]^{\tt a} \ar[dr]_{\tt b}
		&*+\txt{}
			&{2} \ar@(ul,ur)^{\tt aa|b} \ar@/_/[dl]_{\tt ab} \ar[r]^{\varepsilon}
				&*++[o][F=]{A}
\\
*+\txt{}
	&*+\txt{}
		&{3} \ar@/_/[ur]_{\tt a|ba} \ar@/_20pt/[urr]_{\varepsilon} \ar@(dl,dr)_{\tt bb}
}
\]
Jetzt wird der Zustand $2$ entfernt, dabei sind die Pfade 
$S$-$2$-$A$,
$S$-$2$-$3$,
$3$-$2$-$A$,
und
$3$-$2$-$3$
zu berücksichtigen:
\[
\entrymodifiers={++[o][F]}
\xymatrix{
*+\txt{}\ar[r]
	&S\ar[rrr]^{\tt a(aa|b)^*} \ar[dr]_{\tt b|a(aa|b)^*ab}
		&*+\txt{}
			&*+\txt{}
				&*++[o][F=]{A}
\\
*+\txt{}
	&*+\txt{}
		&{3} \ar[urr]_{{\tt (a|ba)(aa|b)^*}|\varepsilon} \ar@(dl,dr)_{{\tt bb|(a|ba)(aa|b)^*ab}}
}
\]
Jetzt muss nur noch der Zustand $3$ entfernt werden:
\[
\entrymodifiers={++[o][F]}
\xymatrix{
*+\txt{}\ar[r]
	&S\ar[rrrrrrrr]^{\tt (a(aa|b)^*)|(b|a(aa|b)^*ab)(bb|(a|ba)(aa|b)^*ab)^*(((a|ba)(aa|b)^*)|\varepsilon)}
		&*+\txt{}
			&*+\txt{}
			&*+\txt{}
			&*+\txt{}
			&*+\txt{}
			&*+\txt{}
			&*+\txt{}
				&*++[o][F=]{A}
}
\]
Damit ist der reguläre Ausdruck gefunden, der die gleiche Sprache
akzeptiert wie der ursprüngliche DEA.

\section{Anhang: Reguläre Ausdrücke in der Praxis\label{regulaer:praxis}}
\rhead{Regex in der Praxis}
\subsection{Flex}
\index{Flex}%
\index{Scanner}%
\index{Scannergenerator}%
Flex ist ein Scannergenerator.
Er verarbeitet eine Spezifikation einer
Sprache als Menge von regulären Ausdrücken in einen deterministischen 
endlichen Automaten.
Als Beispiel soll verfolgt werden, wie die
Flex-Spezifikation
\verbatiminput{lex/example4.l}
in einen endlichen Automaten umgesetzt wird.
Mit der Option {\tt -T}
kann man sich die erzeugten Tabellen des NEA anzeigen lassen:
\verbatiminput{lex/dumpnfa}
Die Tabelle ist wie folgt zu lesen.
In der mittleren Spalte steht das zu verarbeitende Zeichen, also
die Beschriftung eines Pfeils.
Für jedes Zeichen sind zwei Zielzustände
möglich, die in den letzten beiden Spalten eingetragen werden.
Einen
Zustand {\tt 0} gibt es nicht, Übergänge nach ${\tt 0}$ müssen als
nicht verwendete Pfeile interpretiert werden.
$\varepsilon$-Übergänge
werden durch mit dem Symbol {\tt 257} bezeichnet.
Sind von einem Punkt
aus mehr also zwei Alternativen möglich, wie im vorliegenden Fall
bei der Alternative {\tt h|s|r}, dann muss dies mit Hilfe von zusätzlichen
$\varepsilon$-Übergangen aus Zweier-Alternativen zusammengesetzt werden.
Zeichnet man diese NEA, ergibt sich das folgende Zustandsdiagramm:
\[
\entrymodifiers={++[o][F]}
\xymatrix{
*+\txt{}\ar[r]
	&{15} \ar[r]^{\varepsilon} \ar[d]^{\varepsilon}
		&{13}\ar[r]^{{\tt EOF}}
			&*++[o][F=]{14}
\\
*+\txt{}
	&{12}\ar[d]^{\varepsilon} \ar[r]^{\varepsilon}
		&{1}
\\
*+\txt{}
	&{8}\ar[d]^{\varepsilon} \ar[r]^{\varepsilon}
		&{7}\ar[dr]^{\tt r}
\\
*+\txt{}
	&{5}\ar[dr]^{\varepsilon} \ar[r]^{\varepsilon}
		&{3}\ar[r]^{\tt h}
			&{6}\ar[r]^{\varepsilon}
				&{9}\ar[r]^{\tt 1}
					&{10}\ar[r]^{\tt 0}
						&*++[o][F=]{11}
\\
*+\txt{}
	&*+\txt{}
		&{4}\ar[ur]^{\tt s}
\\
*+\txt{}
	&{2}
}
\]
Man kan im unteren Teil sehr schön erkennen, wie der Automat, der
die Dreier-Alternative {\tt h|s|r} akzeptiert, über einen
$\varepsilon$-Übergang mit dem Automaten verkettet wird, der 
den String {\tt 10} akzeptiert.
Zustand {\tt 14} wurde von Flex
hinzugefügt, er erlaubt das Fileende zu erkennen, so dass das
erzeugte Programm am Fileende auf jeden Fall terminieren kann.

Der NEA kann natürlich auch in einen DEA umgewandelt werden.
Auch dies protokolliert Flex: 
\verbatiminput{lex/dumpdfa}
Zu jedem Zustand wird jetzt aufgeführt, welche Zeichen zu welchen
Übergängen führen.
Flex hat die Zeichen in Klassen zusammengefasst,
die Ziffern bedeuten:
\begin{center}
\begin{tabular}{|c|c|}
\hline
Zeichenklasse&Inhalt\\
\hline
{\tt 1}&andere\\
{\tt 2}&{\tt 0}\\
{\tt 3}&{\tt 1}\\
{\tt 4}&{\tt h}\\
{\tt 5}&{\tt s}\\
{\tt 6}&{\tt r}\\
\hline
\end{tabular}
\end{center}
Auch den DEA kann man in ein Zustandsdiagramm zurückübersetzen:
\[
\entrymodifiers={++[o][F]}
\xymatrix{
*+\txt{}
	&*+\txt{}\ar[d]
\\
*+\txt{}
	&{1}\ar[dr]^{{\tt h},{\tt s},{\tt r}} \ar[dl]_{{\tt 0}, {\tt 1}, \text{andere}}
\\
*++[o][F=]{4}
	&*+\txt{}
		&*++[o][F=]{5}\ar[r]^{\tt 1}
			&{6}\ar[r]^{\tt 0}
				&*++[o][F=]{7}
\\
*+\txt{}
	&{2}\ar[ur]_{{\tt h},{\tt s},{\tt r}} \ar[ul]^{{\tt 0}, {\tt 1}, \text{andere}}
\\
*+\txt{}
	&*++[o][F=]{3}
}
\]
Man kann sehen, dass Flex nicht alles wirklich aufräumt, es bleiben
unerreichbare Zustände {\tt 2} und {\tt 3} stehen, einer davon sogar ein
Akzeptierzustand.
Die Zustände 4 und 5 sind als
Akzeptierzustände markiert, aber nur 7 ist als Akzeptierzustand zu
verstehen, der ein auf den regulären Ausdruck passendes Wort
akzeptiert.

\subsection{Performance}
\index{Laufzeit!eines DEA}%
\index{Laufzeit!eines NEA}%
Ein DEA hat immer Laufzeit $O(n)$, d.\,h.~ein Regex-Matcher auf der
Basis eines DEA wird innerhalb einer Zeit proportional zur
Länge des Inputstrings erkennen, ob ein Inputstring passt.
Die einfachste Implementation eines NEA hingegen wird alle Möglichkeiten
von nicht eindeutigen Übergängen durchprobieren müssen, mit
möglicherweise exponentieller Laufzeit.
Normalerweise sind in der
Praxis eingesetzte reguläre Ausdrücke klein und die untersuchten
Strings sind ebenfalls nicht allzu gross.
Man kann allerdings auch
Ausdrücke konstruieren, die den Unterschied zwischen DEA und simplistischer
NEA-Implementation offensichtlich werden lassen.
Ein solcher Ausdruck ist
\begin{center}
\tt a?a?a?a?a?aaaaa
\end{center}
Also $n$ fakultative {\tt a} gefolgt von $n$ obligatorischen {\tt a}.
Um ein Wort bestehend aus $n$ {\tt a} zu akzeptieren, gibt es genau eine
Möglichkeit, die man erhält, wenn man in der Alternativen
${\tt a?} = \varepsilon|{\tt a}$
jeweils das leere Wort wählt.
Falls das immer die zweite ausprobierte 
Möglichkeit ist, ist die Laufzeit dieses Algorithmus proportional zu
$2^n$, wird also bei langen Strings schnell sehr gross.

Es ist einfach, einen geeigneten DEA für den regulären
Ausdruck {\tt a?a?a?aaa} hinzuschreiben:
\[
\entrymodifiers={++[o][F]}
\xymatrix{
*+\txt{}\ar[r]
	&\ar[r]^{\tt a}
		&\ar[r]^{\tt a}
			&\ar[r]^{\tt a}
				&*++[o][F=]{}\ar[r]^{\tt a}
					&*++[o][F=]{}\ar[r]^{\tt a}
						&*++[o][F=]{}\ar[r]^{\tt a}
							&*++[o][F=]{}
}
\]
Und auch einen NEA findet man aus der Konstruktion eines Automaten
aus einem regulären Ausdruck ohne Probleme:
\[
\entrymodifiers={++[o][F]}
\xymatrix{
*+\txt{}\ar[r]
	&\ar@/^/[r]^{\tt a} \ar@/_/[r]_{\varepsilon}
		&\ar@/^/[r]^{\tt a} \ar@/_/[r]_{\varepsilon}
			&\ar@/^/[r]^{\tt a} \ar@/_/[r]_{\varepsilon}
				&\ar[r]^{\tt a}
					&\ar[r]^{\tt a}
						&\ar[r]^{\tt a}
							&*++[o][F=]{}
}
\]
Auch die Vereinfachung der $\varepsilon$-Übergänge ist noch sehr
übersichtlich:
\[
\entrymodifiers={++[o][F]}
\xymatrix{
*+\txt{}\ar[r]
	&\ar[r]^{\tt a}
	\ar@/^20pt/[rr]^{\tt a}
	\ar@/^20pt/[rrr]^{\tt a}
	\ar@/^20pt/[rrrr]^{\tt a}
		&\ar[r]^{\tt a}
		\ar@/_20pt/[rr]_{\tt a}
		\ar@/_20pt/[rrr]_{\tt a}
			&\ar[r]^{\tt a} \ar@/_20pt/[rr]_{\tt a}
				&\ar[r]^{\tt a}
					&\ar[r]^{\tt a}
						&\ar[r]^{\tt a}
							&*++[o][F=]{}
}
\]
Auch in diesem Beispiel kann man sich von Flex helfen lassen,
einen NEA zu erstellen.
Unverkennbar sind auch hier wieder die
Spuren der Konstruktionsschritte, mit denen man aus regulären
Ausdrücken Teilautomaten baut und diese dann zum ganzen Automaten
zusammenbaut.
\[
\entrymodifiers={++[o][F]}
\xymatrix{
*+\txt{}\ar[r]
	&{19}\ar[d]^{\varepsilon}\ar[r]^{\varepsilon}
		&{17}\ar[r]^{\tt EOF}
			&*++[o][F=]{18}
\\
*+\txt{}
	&{16}\ar[d]^{\varepsilon} \ar[r]^{\varepsilon}
		&{1}
\\
*+\txt{}
	&{5} \ar[r]^{\varepsilon}
	      \ar[dr]^{\varepsilon}
		&{3} \ar[d]^{\tt a}
			&{8} \ar[r]^{\varepsilon}
			     \ar[dr]^{\varepsilon}
				&{6} \ar[d]^{\tt a}
					&{11} \ar[r]^{\varepsilon}
					     \ar[dr]^{\varepsilon}
						&{9} \ar[d]^{\tt a}
\\
*+\txt{}
	&*+\txt{}
		&{4} \ar[ur]^{\varepsilon}
			&*+\txt{}
				&{7} \ar[ur]^{\varepsilon}
					&*+\txt{}
						&{10} \ar@/_10pt/[dlll]_{\varepsilon}
\\
*+\txt{}
	&*+\txt{}
		&*+\txt{}
			&{12} \ar[r]^{\tt a}
				&{13} \ar[r]^{\tt a}
					&{14} \ar[r]^{\tt a}
						&*++[o][F=]{15}
}
\]
Daraus macht Flex dann den folgenden DEA:
\[
\entrymodifiers={++[o][F]}
\xymatrix{
*+\txt{}\ar[d]
\\
{1}\ar[r]^{\tt a} \ar[d]^{\text{andere}} \
	&{5}\ar[r]^{\tt a}
		&{6}\ar[r]^{\tt a}
			&*++[o][F=]{7} \ar[r]^{\tt a}
				&*++[o][F=]{8} \ar[r]^{\tt a}
					&*++[o][F=]{9} \ar[r]^{\tt a}
						&*++[o][F=]{10}
\\
{4}
\\
{3}
}
\]
$3$ und $4$ sind wieder Zustände die benötigt werden, um andere Inputs
wie zum Beispiel EOF zu erkennen.
Der resultierend DEA entspricht genau
dem oben vorgeschlagenen DEA.

\begin{sloppypar} % To prevent "java.lang.string" from overflowing
Nicht alle Implementationen verwenden jedoch diese Theorie.
Die Regex-Funktionen in der C-Library verwenden dies und sind entsprechend
schnell.
Alle Unix-Tools, die auf dieser Bibliothek basieren, sind
entsprechend schnell.
Perl hat seine eigene erweiterte Regex-Library,
verwendet keinen DEA und ist langsam.
Ebenfalls langsam ist die
Implementation der {\tt matches} Methode in {\tt java.lang.String}.
Man würde hoffen, dass wenigstens das Package {\tt java.util.regex} einen
besseren Regex-Matcher beinhalten würde.
Leider wurde da nur am Interface herumgebastelt, nicht an der Substanz.
Vielleicht sollten
die Implementatoren von {\tt java.util.regex} mal die Vorlesung AutoSpr
besuchen.
\end{sloppypar}

\section{Zusammenfassung: Das Wichtigste in Kürze}
\begin{enumerate}
\item Zu jedem DEA gibt es einen
minimalen Automaten, der zum Beispiel dazu verwendet werden kann,
DEA zu vergleichen
(\ref{regulaer:minimalautomat}).
\item Eine Sprache ist regulär, wenn sie von einem endlichen Automaten
akzeptiert wird (\ref{regulaer:definition:regulaere-sprache}).
\item Selbst wenn man von einer Sprache nur weiss, dass sie regulär
ist, kann man einen DEA finden, der die Sprache akzeptiert
(Satz \ref{satz_dea_aus_sprache}).
\item Jeder NEA kann in einen DEA umgewandelt werden (\ref{regulaer:nea-dea}).
\item Reguläre Sprachen erfüllen das Pumping Lemma
(\ref{regulaer:pumpinglemma}).
Eine Sprache ist nicht regulär, wenn sie ein Wort enthält, welches
nicht aufgepumpt werden kann: wie auch immer man das Wort
auf ein Pumping Lemma zerlegt, mindestens ein aufgepumptes
oder abgepumptes Wort ist nicht mehr in der Sprache drin.
\item Die regulären Operationen Vereinigung, Verkettung und $*$-Operation
erzeugen aus regulären Sprachen neue reguläre Sprachen
(\ref{regulaer:regulaere-operationen}).
\item Zu jedem endlichen Automaten kann man einen regulären Ausdruck
finden, der die gleiche Sprache akzeptiert wie der Automat.
(\ref{regulaer:dea-re}).
\end{enumerate}

%
% Kontextfreie Sprachen und pushdown-Automaten
%
\lhead{Kontextfreie Sprachen}
\rhead{Kontextfreie Grammatiken}
\chapter{Stackautomaten und kontextfreie Sprachen\label{chapter-cfl}}
Die regulären Sprachen bilden zwar eine interessante Klasse
von Sprachen mit mathematisch sehr attraktiven Eigenschaften,
doch sind viele praktisch wichtige Sprachen nicht regulär.
Bereits korrekt geschachtelte Klammern können nicht mit einem
DEA erkannt werden.

Ursache ist, dass sich ein DEA nicht
daran ``erinnern'' kann, wieviele Klammern schon geöffnet
worden sind. Um das für beliebig viele Klammern tun zu können,
bräuchte er unendlich viele Zustände (Satz von Myhill-Nerode),
das ist in einem DEA nicht möglich. Dazu ist als Erweiterung
eine Art von Speicher nötig. In diesem Kapitel betrachten
wir Automaten, die mit einem Stack ausgestattet worden sind.

\index{Stack}%
\index{Stackautomat}%
Auch reguläre Ausdrücke sind nicht dazu geeigneet, die Schachtelung
von Klammern auszudrücken. Daher brauchen wir auch für die
regulären Ausdrücke eine Alternative, dies werden die kontextfreien
Grammatiken sein. Es wird sich zeigen, dass die von Stackautomaten
akzeptierten Sprachen auch mit einer kontextfreien Grammatik erzeugt
werden können, wir haben also eine ähnliche Situation wie
bei DEAs und regulären Ausdrücken. So entsteht die neue
Kategorie der kontextfreien Sprachen.
\index{Sprache!kontextfreie}%
Natürlich werden auch kontextfreie Sprachen nicht alles
abdecken, so dass wir wieder ein Kriterium brauchen, mit dem
man entscheiden kann, ob eine Sprache kontextfrei ist. Wir
werden auch für kontextfreie Sprachen
ein Pumping Lemma finden.

\section{Kontextfreie Sprachen}
\rhead{Kontextfreie Sprachen}
\subsection{Kontextfreie Grammatiken}
Reguläre Sprachen werden dadurch definiert, dass die Zeichen
eines Wortes von links nach rechts gelesen werden, und nach jedem
Zeichen entschieden werden kann, ob das Wort akzeptabel ist.
Der Fokus liegt also auf der Analyse des Strings, und genau dies
macht das Überprüfen der korrekten Schachtelung von Klammern
schwierig. 

\index{Ausdr\ücke!arithmetische}%
Für Klammern ist man sich jedoch meistens eine andere Vorgehensweise
gewohnt. Wenn man arithmetische Ausdrücke aufbaut, sagt man zum
Beispiel: ``Wenn man die Summe $a+b$ mit $c$ multiplizeren will,
dann {\bf muss man $a+b$ in Klammern setzen}''.
Klammern werden also immer paarweise gesetzt, und man baut den
Ausdruck sozusagen von innen nach aussen auf.

Wir möchten diese Idee ein Stück weit formalisieren. Das Ziel ist,
Wörter aus den Zeichen {\tt (} und {\tt )} aufzubauen, die korrekte
Klammerausdrücke sind. Dazu können wir bereits bekannte korrekte
Klammerausdrücke aneinanderhängen, oder wir können einen bereits
bekannten korrekten Ausdruck ``einklammern''. Schreiben wir $A$
für einen Klammerausdruck, bedeutet das, dass wir $A$ gemäss
folgender Regeln umformen können:
\begin{align*}
A&\to AA\\
A&\to {\tt (}A{\tt )}
\end{align*}
Dabei ist auf der rechten Seite der ersten Regel gemeint, dass wir zwei beliebige
Ausdrücke nehmen können, sie müssen nicht identisch sein. Allerdings
fehlt in diesen Regeln noch ein Anfang, bis jetzt können wir überhaupt
keine Wörter produzieren. Dazu nehmen wir noch eine Regel $A\to\varepsilon$
hinzu, die besagt, dass das leere Wort auch ein akzeptabler
Klammerausdruck ist. Die Regeln
\begin{align}
A&\to\varepsilon       \label{regeln-beispiel-1}\\
A&\to AA               \label{regeln-beispiel-2}\\
A&\to {\tt (}A{\tt )}  \label{regeln-beispiel-3}
\end{align}
erzeugen in diesem Sinne alle korrekt geschachtelten Klammerausdrücke.

Der Buchstabe $A$ steht offenbar nicht immer für das gleiche, 
er ist Platzhalter für einen korrekten Klammerausdruck. Wir nennen
$A$ eine Variable.
Dagegen stehen die Zeichen {\tt (} und {\tt )} nur für sich selbst, sie
können nicht weiter ersetzt werden, sie heissen Terminalsymbole.

Damit sind wir bereit für die formale Definition einer kontextfreien
Grammatik.
\begin{definition}
\index{Grammatik!kontextfreie}%
Eine kontextfreie Grammatik ist ein Quadrupel $(V,\Sigma,R,S)$ mit
\begin{compactenum}
\index{Variable}%
\item $V$ ist eine endliche Menge von Variablen.
\index{Terminalsymbol}%
\item $\Sigma$ ist eine endliche Menge von Zeichen, disjunkt zu $V$,
auch genannt die Terminalsymbole.
\item $R$ ist eine Menge von Regeln, eine Regel besteht aus einer
Variable und einer Kette von Variablen und Terminalsymbolen, geschrieben
in der Form $A\to BC{\tt x}$, wobei rechts eine beliebige Folge von
Variablen (in diesem Fall $A$ und $B$) und Terminalsymbolen (in diesem Fall
\texttt{x}) stehen kann.
\index{Regel!einer kontextfreien Grammatik}%
\item $S\in V$ ist die Startvariable.
\index{Startvariable}%
\end{compactenum}
\end{definition}
Im Beispiel ist $V=\{A\}$, $\Sigma=\{{\tt (},{\tt )}\}$, $S=A$ und
$R$ enthält genau drei Regeln (\ref{regeln-beispiel-1}) bis
(\ref{regeln-beispiel-3}).

Zur Abkürzung erlauben wir, Regeln mit der gleichen Variablen
auf der linken Seite des Pfeils mit einem Vertikalstrich als
Verknüpfungszeichen auf der rechten Seite zu schreiben, zu lesen
als ``oder'':
\[
A \to \varepsilon\;|\; AA\;|\; {\tt (}A{\tt )}.
\]

Der Name ``kontextfrei'' rührt daher, dass es bei der Anwendung
der Regeln nicht auf den Kontext ankommt, in dem eine Variable
auf der linken Seite des Pfeils~$\rightarrow$ vorkommt. Regeln der
Form 
\[
{\tt a}A\to AA, \qquad {\tt bA}\to BB,
\]
sogenannte kontextsensitive Regeln,
\index{kontextsensitiv}%
würden dagegen ausdrücken, dass die Umwandlung der Variablen $A$
unterschiedlich zu erfolgen hat, wenn ihr verschiedene Zeichen
vorangehen. In diesem Fall käme es auf den Kontext an. Solche
Regeln sollen jedoch nicht zugelassen sein.

\subsection{Kontextfreie Sprachen}
Eine Grammatik erzeugt eine Sprache auf die folgende Weise.
Auf die Startvariable werden beliebige Regeln angewendet,
bis keine Variablen mehr vorhanden sind, jetzt steht nur noch
eine Kette von Terminalsymbolen da. Dies wollen wir wie folgt
formalisieren.

\begin{definition}
\index{Regel!erzeugtes Wort}%
\index{Ableitung}%
Falls $A\to w$ eine Regel einer Grammatik $G$ ist, sagt man
dass die Regel aus $uAv$ den String $uwv$ erzeugt,
geschrieben $uAv\Rightarrow uwv$. Man sagt, $v$ lässt sich aus
$u$ ableiten, wenn es eine Folge $u_1,\dots,u_n$ gibt mit
\[
u\Rightarrow u_1\Rightarrow u_2\Rightarrow\dots\Rightarrow u_n\Rightarrow v,
\]
auch geschrieben als $u\overset{*}{\Rightarrow} v$
\end{definition}

\begin{beispiel}
Als Beispiel sei das Wort \texttt{(()())} aus der Grammatik
(\ref{regeln-beispiel-1})
bis
(\ref{regeln-beispiel-2})
für die Klammerausdrücke abgeleitet:
\begin{align*}
A
&\xrightarrow{\text{(\ref{regeln-beispiel-3})}} \texttt{(} A \texttt{)}
 \xrightarrow{\text{(\ref{regeln-beispiel-2})}} \texttt{(} AA \texttt{)}
 \xrightarrow{\text{(\ref{regeln-beispiel-3})}} \texttt{((} A\texttt{)}A \texttt{)}
 \xrightarrow{\text{(\ref{regeln-beispiel-3})}} \texttt{((} A\texttt{)(}A \texttt{))}
 \xrightarrow{\text{(\ref{regeln-beispiel-1})}} \texttt{((} \texttt{)(}A \texttt{))}
 \xrightarrow{\text{(\ref{regeln-beispiel-1})}} \texttt{((} \texttt{)(} \texttt{))}
\end{align*}
Die Reihenfolge der Regelanwendung ist nicht eindeutig.
\end{beispiel}

\begin{definition}
\index{Sprache!von einer CFG erzeugte}%
Die Menge aller Wörter, die von einer kontextfreien Grammatik 
$G$ erzeugt werden können wir mit
\[
L(G)=\{w\in\Sigma^*\;|\; S\overset{*}{\Rightarrow} w\}
\]
bezeichnen.
\end{definition}

\subsection{Beispiele}
\subsubsection{Natürliche Zahlen}
Natürliche Zahlen sind Wörter über dem Alphabet $\Sigma=\{
{\tt 0},
{\tt 1},
{\tt 2},
{\tt 3},
{\tt 4},
{\tt 5},
{\tt 6},
{\tt 7},
{\tt 8},
{\tt 9}\}$, welche von den Grammatikregeln
\begin{align*}
N&\to Z\\
 &\to NZ\\
Z&\to {\tt 0}\;|\;\dots \;|\;{\tt 9}
\end{align*}
erzeugt werden.

Diese Grammatik hat aber den Mangel, dass sie auch Zahlen
mit führenden Nullen erlaubt.
Diese könnten dadurch entfernt
werden, dass wir eine weitere Variable für zulässige
Anfangsziffern einführen.
Die Bezeichnung der Variablen mit einzelnen Buchstaben wird jetzt
bereits etwas unübersichtlich, wir schreiben die Grammatik daher neu
so:
\begin{align*}
\textsl{zahl}           &\to\textsl{nichtnullziffer}\; \textsl{ziffernfolge} \\
                        &\to\textsl{ziffer}\\
\textsl{ziffernfolge}   &\to \textsl{ziffernfolge}\; \textsl{ziffer}\\
                        &\to \textsl{ziffer}\\
\textsl{ziffer}         &\to \texttt{0}\; |\; \textsl{nichtnullziffer}\\
\textsl{nichtnullziffer}&\to \texttt{1}\;|\;\texttt{2}\;|\;\dots\;|\;\texttt{9}
\end{align*}
Damit ist auch gleich der Fall einer einzelnen Null abgehandelt, denn eine
solche ist ja eine \textsl{ziffer}, und eine \textsl{zahl} kann auch eine
einzelne Ziffer sein.

\subsubsection{Einfache arithmetische Ausdrücke}
\index{Ausdr\ücke!arithmetische}%
\index{expression-term-factor Grammatik}%
Auch arithemtische Ausdrücke können mit einer
Grammatik erzeugt werden. Da die Zahl der Variablen schnell grösser
wird, werden wir im Folgenden auch länger Variablennamen zulassen.
Als Alphabet verwenden wir
$\Sigma=\{{\tt 0},\dots,{\tt 9},{\tt +},{\tt *},{\tt (}, {\tt )}\}$.
Die Startvariable drückt aus, was erzeugt werden soll, in diesem Fall
ein Ausdruck, also nennen wir sie \textsl{expression}.  Folgende Grammatik
erzeugt die arithmetischen Ausdrücke:
\begin{align*}
\textsl{expression} &\to \textsl{expression}\;{\tt +}\;\textsl{term}\\
                    &\to \textsl{term}\\
\textsl{term}       &\to \textsl{term}\;{\tt *}\;\textsl{factor}\\
                    &\to \textsl{factor}\\
\textsl{factor}     &\to {\tt (}\textsl{expression}{\tt )}\\
                    &\to N\\
N                   &\to Z\\
                    &\to NZ\\
Z                   &\to {\tt 0}\;|\;\dots \;|\;{\tt 9}
\end{align*}

\subsubsection{Eine nicht reguläre Sprache}
Die Sprache $\{{\tt 0}^n{\tt 1}^n\;|\;n\in\mathbb N\}$ wurde bereits
früher als nicht regulär erkannt. Sie wird aber von der 
Grammatik $G=(\{S\}, \{{\tt 0},{\tt 1}\}, R, S)$ erzeugt
mit den Regeln
\begin{align*}
S&\rightarrow \varepsilon\\
&\rightarrow {\tt 0}S{\tt 1}
\end{align*}

\subsection{Reguläre Operationen\label{sect:cfg-regulaer}}
In diesem Abschnitt beweisen wir, dass reguläre Sprachen auch
kontextfrei sind. Dazu ist zu zeigen, dass
zu jedem regulären Ausdruck eine kontextfreie Grammatik existiert,
die die Sprache erzeugt.

\begin{satz}[Vereinigung]
\label{satz:cfg-union}
\index{Vereinigung}%
Seien $L_1$ und $L_2$ kontextfreie Sprachen über $\Sigma$,
dann ist auch $L_1\cup L_2$ kontextfrei.
\end{satz}

\begin{proof}[Beweis]
Weil $L_i$ kontextfrei sind,  gibt es Grammatiken $G_1$ und $G_2$,
die $L_1$ bzw.~$L_2$ erzeugen. Wir dürfen sogar annehmen, dass
beide Grammatiken keine
gemeinsamen Variablen haben, dass also $V_1$ und $V_2$ 
disjunkt sind. Wir konstruieren jetzt eine neue Grammatik
$G=(V_1\cup V_2\cup\{S\}, \Sigma, R, S)$ mit einem neuen
Startzustand $S$. Die Regelmenge $R$ besteht einerseits aus
den Regeln in $G_1$ und $G_2$ und andererseits aus der neuen
Regel
\[
S\to S_1 | S_2.
\]
Insgesamt ist also
\[
R=R_1\cup R_2\cup \{S\to S_1, S\to S_2\}.
\]
Die Wörter von $L_1$ werden erzeugt, wenn man $S\to S_1$ als
erste Regel anwendet, die Wörter von $L_2$ jedoch, wenn 
man $S\to S_2$ verwendet. Die Regeln können nicht ``gemischt''
angewendet werden, weil $V_1\cap V_2=\emptyset$.
\end{proof}

\begin{satz}[Verkettung]
\label{satz:cfg-verkettung}
\index{Verkettung}%
Sind $L_1$ und $L_2$ kontextfreie Sprachen, dann ist auch $L_1L_2$
kontextfrei.
\end{satz}

\begin{proof}[Beweis]
Seien wieder $G_i$ die Grammatiken, die $L_i$ erzeugen, mit disjunkten
Variablenmengen $V_1\cap V_2=\emptyset$. Dann erzeugt die Grammatik
$G=(V_1\cup V_2\cup\{S\},\Sigma, R,S)$ mit dem neuen Startzustand
und den Regeln
\[
R=R_1\cup R_2\cup \{S\to S_1S_2\}
\]
die Sprache $L_1L_2$.
\end{proof}

\begin{satz}[$*$-Operation]
\index{*-Operation@$*$-Operation}%
\label{satz:cfg-star}
Ist $L$ kontextfrei, dann auch $L^*$.
\end{satz}

\begin{proof}[Beweis]
Sei $G=(V,\Sigma,R,S)$ die Grammatik, die $L$ erzeugt. Dann erzeugt die
Grammatik
\[
G^*=(V\cup \{S_0\}, \Sigma, R_0, S_0)
\]
mit
\[
R_0=R\cup \{ S_0\to S_0S, S_0\to\varepsilon \}
\]
die Sprache $L^*$.
\end{proof}

\begin{satz} Sei $L$ eine reguläre Sprache, dann gibt es eine
kontextfreie Grammatik $G$ mit $L(G)=L$.
\end{satz}

\begin{proof}[Beweis]
Da reguläre Sprachen mit regulären Operationen aus einfacheren
Sprachen aufgebaut werden können, ist nur zu zeigen, dass die
Konstruktionen, die mit regulären Operationen möglich sind,
auch durch Produktionsregeln einer Grammatik ausgedrückt werden
können. 
\begin{enumerate}
\item Leere Sprache: Die leere Sprache ist regulär. Sie wird auch 
von der Grammatik mit $R=\emptyset$ erzeugt, also ist sie auch
kontextfrei.
\item Ein einzelnes Zeichen: $L=\{{\tt a}\}$. Die Sprache $L$ wird von
der Grammatik $G=(\{S\}, \{{\tt a}\}, \{S\to{\tt a}\}, S)$
erzeugt.
\item Alle Sprachen, die sich aus den eben genannten durch reguläre
Operationen konstruieren lassen, sind ebenfalls kontextfrei. Da
dies für alle regulären Sprachen zutrifft folgt, dass alle
regulären Sprachen kontextfrei sind.
\end{enumerate}
\end{proof}

Der Satz zeigt, dass die kontextfreien Sprachen eine echte
Obermenge der regulären Sprachen sind:
\begin{center}
%\includegraphics[width=0.5\hsize]{images/lang-1}
\includegraphics{images/lang-1}
\end{center}
Wir werden später zeigen, dass kontextfreie Sprachen auch durch
die Existenz eines Stackautomaten charakterisiert werden können,
der diese Sprachen akzeptiert. Dann wird sich zeigen, dass 
die regulären Sprachen diejenigen kontextfreien Sprachen sind,
für deren Analyse der Stack gar nicht benötigt wird.

\subsection{Parse Tree}
\index{parse tree}%
\index{Ableitungsbaum}%
Der Ableitungsbaum eines Wortes einer kontextfreien Sprache
ist eine Darstellung der verwendeten Produktionsregeln in Baum-Struktur.
\index{Ausdr\ücke!arithmetische}%
Die Grammatik für arithmetische Ausdrücke produziert zum Beispiel
den Ausdruck {\tt 7 * (3 + 5)}.
Die dabei verwendeten Regeln können in Baumform wie in
Abbildung \ref{etf-parse-tree} dargestellt werden.

\begin{figure}
{
\small
\[
\xymatrix{
	&\textsl{expression} \ar[d]
\\
	&\textsl{term} \ar[dl] \ar[d] \ar[drrr]
\\
\textsl{term}\ar[dddd]
	&{\tt *}\ar[ddddddd]
		&
			&
				&\textsl{factor}\ar@/_80pt/[dddddddll]\ar[d]\ar@/^40pt/[dddddddrr]
					&
						&
\\
	&
		&
			&
				&\textsl{expression}\ar[dl]\ar[dddddd]\ar[ddr]
					&
\\
	&
		&
			&\textsl{expression}\ar[d]
				&
					&
						&
\\
	&
		&
			&\textsl{term}\ar[d]
				&
					&\textsl{term}\ar[d]
						&
\\
\textsl{factor}\ar[d]
	&
		&
			&\textsl{factor}\ar[d]
				&
					&\textsl{factor}\ar[d]
						&
\\
N\ar[d]
	&
		&
			&N\ar[d]
				&
					&N\ar[d]
						&
\\
Z\ar[d]
	&
		&
			&Z\ar[d]
				&
					&Z\ar[d]
						&
\\
{\tt 7}
	&{\tt *}
		&{\tt (}
			&{\tt 3}
				&{\tt +}
					&{\tt 5}
						&{\tt )}
}
\]
}
\caption{Parse-Tree des Ausdrucks \texttt{7 * ( 3 + 5 )} in der
Expression-Term-Factor-Grammatik.
\label{etf-parse-tree}}
\end{figure}

Offenbar spielt es keine Rolle, ob erst die Regelanwendungen
im linken Teil des Baumes geschehen, oder die Regelanwendungen im
rechten Teil. Zwei Ableitungen können daher als gleich angesehen
werden, wenn sich die Ableitungsbäume nicht unterscheiden.

\begin{definition}
\index{Aquivalenz@Äquivalenz!von Ableitungen}%
Zwei Ableitungen eines Wortes $w$ einer kontextfreien Sprache $L(G)$
heissen äquivalent, wenn sie den gleichen Ableitungsbaum haben.
Hat eine Sprache Wörter mit verschiedenen Ableitungen, heisst
sie mehrdeutig (engl.~ambiguous).
\end{definition}

Von besonderem praktischem Interesse sind Grammatiken, in denen
Ableitungen immer eindeutig sind. Die
\index{expression-term-factor Grammatik}%
{\tt expression}-{\tt term}-{\tt factor}-Grammatik für einfache
arithmetische Ausdrücke erfüllt diese Bedingung. 
Ein Beispiel für eine zweideutige Grammatik ist
\[
G=(\{S\}, \{\texttt{0},\texttt{1}\}, \{S\to \texttt{0}S\texttt{1}|\texttt{1}S\texttt{0}|SS|\varepsilon\}, S).
\]
$G$ erzeugt die Sprache
$\{w\in \{\texttt{0},\texttt{1}\}^*\,|\, |w|_{\texttt{0}} = |w|_\texttt{1}\}$, aber das Wort
$\texttt{001011}$ hat mindestens zwei verschiedene Ableitungen:
\[
\xymatrix{
&&&S\ar@/_10pt/[dddddlll]\ar[d]\ar@/^30pt/[dddddrrrr]
\\
&&&S\ar[dl]\ar[drr]
\\
 &&S\ar[dddl]\ar[d]\ar[dddr]&&&S\ar[dddl]\ar[d]\ar[dddr]&&
\\
&&S\ar[d]&&&S\ar[d]&&
\\
 & &\varepsilon& & &\varepsilon& & 
\\
\texttt{0}&\texttt{0}& &\texttt{1}&\texttt{0}& &\texttt{1}&\texttt{1}
}
\]
oder
\[
\xymatrix{
&&&S\ar@/_/[dddddlll]\ar[d]\ar@/^/[dddddrrr]
\\
&&&S\ar@/_/[ddddll]\ar[d]\ar@/^/[ddddrr]&&&
\\
&&&S\ar@/_/[dddl]\ar[d]\ar@/^/[dddr]&&&
\\
&&&S\ar[d]&&&
\\
&&&\varepsilon&&&
\\
\texttt{0}&\texttt{0}&\texttt{1}&&\texttt{0}&\texttt{1}&\texttt{1}
}
\]


\subsection{Chomsky Normalform}
Bei den regulären Sprachen war der Algorithmus zur Reduktion
auf den minimalen Automaten die Voraussetzung, Automaten vergleichen
zu können. Eine ähnliche Normalform wünscht man sich auch für
Grammatiken, und definiert daher:
\begin{definition}
\label{definition:cnf}
\index{Chomsky Normalform}%
Eine kontextfreie Grammatik  ist in Chomsky Normalform, wenn
jede Regel von der Form
\begin{align*}
A&\to BC\qquad\text{oder}\\
A&\to a
\end{align*}
ist, wobei $A\in V$, $a\in\Sigma$ und $B, C\in V\setminus\{S\}$.
Zusätzlich ist die Regel $S\to\varepsilon$ erlaubt. 
\end{definition}

Bei einer Grammatik in Chomsky Normalform kann man sofort abschätzen,
wieviele Regelanwendungen notwendig sind, um ein gegebenes Wort
zu erzeugen. Wendet man eine der Regeln $A\to BC$ an, wird das
Wort verlängert. Dies kann nicht mehr rückgängig gemacht
werden, weil dazu eine Regel notwendig wäre, die eine Variable
in das leere Wort umwandelt. Die Regel $S\to\varepsilon$ ist aber
die einzige erlaubte derartige Regel, aber sie kann nie angewendet
werden weil $B$ und $C$ nicht $S$ sein können. Es werden also
höchstens $|w|-1$ Anwendungen von Regeln $A\to BC$ benötig, und dann
nochmals $|w|$ Anwendungen von Regeln $A\to a$, welche die Variablen
in Terminalsymbole umwandeln. 

Eine Grammatik kann aus folgenden Gründen nicht der Chomsky Normalform
entsprechen:
\begin{enumerate}
\item Die Startvariable $S$ könnte auf der rechten Seite einer
Regel vorkommen.
\item Eine $\varepsilon$-Regel $A\to\varepsilon$ mit einer Variablen
$A$, die nicht die Startvariable ist.
\index{unit rule}%
\index{Einheits-Regeln}%
\item Regeln der Form $A\to B$, sogenannte unit rules oder Einheits-Regeln.
\item Regeln der Form $A\to u_1u_2\dots u_n$, mit $n>2$, wobei
die $u_i$ sowohl andere Variablen wie auch Terminalsymbole sein 
können.
\end{enumerate}
Es ist möglich, alle diese Defekte zu ``reparieren'', und jede
Grammatik in Chomsky Normalform zu überführen, wie der folgende Satz
zeigt.

\begin{satz}
\label{satz:cnf}
Eine kontextfreie Sprache wird erzeugt von einer Grammatik
in Chomsky Normalform.
\end{satz}

\begin{proof}[Beweis]
Eine kontextfreie Sprache $L$ wird von einer kontextfreien Grammatik
$G$ akzeptiert. Es ist zu zeigen, dass diese Grammatik in
äquivalente Chomsky Normalform gebracht werden kann.
Weiter oben haben wir eine Liste
möglicher Defekte zusammengestellt, die eine Grammatik davon abhalten,
Chomsky Normalform zu haben. Um Chomsky Normalform zu erreichen,
müssen also genau diese Defekte behoben werden.
Dazu sind folgende Schritte notwendig
\begin{enumerate}
\item Damit die Startvariable nicht mehr auf der rechten Seite auftreten
kann, wird eine neue Startvariable $S_0$ hinzugefügt, sowie eine
neue Regel
$S_0\to S$, wobei $S$ die ursprüngliche Startvariable war.
\item Entfernung aller $\varepsilon$-Regeln. Solche Regeln erlauben,
eine Variable ``wegzulassen''. Für jede Regel, die auf der rechten
Seite eine solche Variable enthält, fügen wir eine zusätzliche
Regel hinzu, bei der die $\varepsilon$-Regel auf der rechten Seite
angwendet worden ist, d.\,h.~die Variable auf der rechten Seite weggelassen
wurde.  Aus
\[
\left.
\begin{aligned}
A&\to\varepsilon\\
B&\to AC\\
\end{aligned}
\right\}
\qquad
\Rightarrow
\qquad
\left\{
\begin{aligned}
B&\to AC\\
&\to C\\
\end{aligned}
\right.
\]
Die $\varepsilon$-Regeln werden damit unnötig, mit der einzig
möglichen Ausnahme einer Regel $S_0\to\varepsilon$, die nicht
eliminiert werden kann, weil $S_0$ nicht auf der rechten Seite
vorkommt.
\item Entfernung von sogenannten ``unit rules'' oder Einheits-Regeln:
Regeln mit einer einzelnen Variablen auf der rechten Seite, also
der Form $A\to B$ können dadurch eliminert werden, dass man
sie in jeder Regel anwendet, die $B$ auf der linken Seite enthält.
\[
\left.
\begin{aligned}
A&\to B\\
B&\to CD
\end{aligned}
\right\}
\qquad\Rightarrow\qquad
\left\{
\begin{aligned}
A&\to CD\\
B&\to CD
\end{aligned}
\right.
\]
Wieder muss man die ursprüngliche Regel $B\to CD$ natürlich
behalten, denn $B$ könnte ja auch auf andere Art erhalten worden
sein als mit der Regel $A\to B$.
\item Verkettungen: Eine Regel der Form $A\to u_1\dots u_n$ kann mit
Hilfe neuer Variablen $A_1,\dots,A_{n-2}$ in folgende Regeln abgebildet
werden
\begin{align*}
A&\to u_1A_1\\
A_1&\to u_2A_2\\
A_2&\to u_3A_3\\
&\vdots\\
A_{n-2}&\to u_{n-1}u_n
\end{align*}
Falls $u_i$ ein Terminalsymbol ist, ersetzen wir $u_i$ in obigen
Regeln durch eine neue Variable $U_i$ und fügen die Regel
$U_i\to u_i$ hinzu. Damit erhalten alle hinzugefügten Regeln
Chomsky Normalform.
\end{enumerate}
Die so konstruierte Grammatik hat Chomsky Normalform und erzeugt die
gleiche Sprache.
\end{proof}

\subsubsection{Beispiel}
Man bringe die Grammatik über dem Alphabet $\Sigma=\{{\tt a},{\tt b}\}$
mit den Regeln
\begin{align*}
S&\to ASA\;|\; {\tt a}B\\
A&\to B\;|\; S\\
B&\to{\tt b}\;|\; \varepsilon
\end{align*}
in Chomsky Normalform.

\begin{enumerate}
% 1. Schritt: neue Startvariable
\item Neue Startvariable hinzufügen:
\begin{align*}
S_0&\to S\\
S&\to ASA\;|\; {\tt a}B\\
A&\to B\;|\; S\\
B&\to{\tt b}\;|\; \varepsilon
\end{align*}
% 2. Schritt: epsilon-Uebergaenge
\item $\varepsilon$-Übergänge entfernen: Die einzige $\varepsilon$-Regel
ist $B\to\varepsilon$, d.\,h.~zu jeder Regel mit $B$ auf der rechten
Seite gibt es auch eine Regel, in welcher auf der rechten Seite das $B$
weggelassen wurde:
\begin{align*}
S_0&\to S\\
S&\to ASA\;|\; {\tt a}B\;|\;{\tt a}\\
A&\to B \;|\; \varepsilon \;|\; S\\
B&\to{\tt b}
\end{align*}
In diesem Schritt ist eine neue $\varepsilon$-Regel entstanden, die
man auch noch gleichartig behandeln muss:
\begin{align*}
S_0&\to S\\
S&\to ASA \;|\; AS \;|\; SA \;|\; {\tt a}B\;|\;{\tt a}\\
A&\to B \;|\; S\\
B&\to{\tt b}
\end{align*}
% 3. Schrit: Unit rules
\item Einheits-Regeln (unit rules) sind $A\to B$, $A\to S$ und $S_0\to S$,
die alle angewendet werden müssen.
Zuerst $A\to B$:
\begin{align*}
S_0&\to S\\
S&\to ASA \;|\; AS \;|\; SA \;|\; {\tt a}B\;|\;{\tt a}\\
A&\to S\\
B&\to{\tt b}\\
A&\to{\tt b}
\end{align*}
Im Folgenden fassen wir die von $A$ ausgehenden Regeln wieder auf
einer Zeile zusammen.
Als nächstes wenden wir $A\to S$ an, aus $A$ lassen sich die gleichen
Wörter produzieren wie aus $S$, aber zusätzlich noch das $\texttt{b}$:
\begin{align*}
S_0&\to S\\
S&\to ASA \;|\; AS \;|\; SA \;|\; {\tt a}B\;|\;{\tt a}\\
A&\to ASA \;|\; AS \;|\; SA \;|\; {\tt a}B\;|\;{\tt a}\;|\;{\tt b}\\
B&\to{\tt b}
\end{align*}
Und zum Schluss $S_0\to S$
\begin{align*}
S_0&\to ASA \;|\; AS \;|\; SA \;|\; {\tt a}B\;|\;{\tt a}\\
S&\to ASA \;|\; AS \;|\; SA \;|\; {\tt a}B\;|\;{\tt a}\\
A&\to ASA \;|\; AS \;|\; SA \;|\; {\tt a}B\;|\;{\tt a}\;|\;{\tt b}\\
B&\to{\tt b}
\end{align*}
%Wobei wir die von $A$ ausgehenden Regeln zusammenfassen können:
%\begin{align*}
%S_0&\to ASA \;|\; AS \;|\; SA \;|\; {\tt a}B\;|\;{\tt a}\\
%S&\to ASA \;|\; AS \;|\; SA \;|\; {\tt a}B\;|\;{\tt a}\\
%A&\to ASA \;|\; AS \;|\; SA \;|\; {\tt a}B\;|\;{\tt a}\;|\; {\tt b}\\
%B&\to{\tt b}
%\end{align*}
% 4. Schritt
\item Längere Verkettungen gibt es nur bei $S\to ASA$, wir führen
also eine zusätzliche Variable $A_1$ und ersetzen $S\to ASA$ durch
die Regeln $S\to AA_1$ und $A_1\to SA$:
\begin{align*}
S_0&\to AA_1 \;|\; AS \;|\; SA \;|\; {\tt a}B\;|\;{\tt a}\\
S&\to AA_1 \;|\; AS \;|\; SA \;|\; {\tt a}B\;|\;{\tt a}\\
A&\to AA_1 \;|\; AS \;|\; SA \;|\; {\tt a}B\;|\;{\tt a} \;|\; {\tt b}\\
A_1&\to SA\\
B&\to{\tt b}
\end{align*}
Bleiben nur noch die Regeln $S_0\to{\tt a}B$, $S\to{\tt a}B$ und
$A\to{\tt a}B$.
Dazu wird eine weitere Variable $U$ hinzugefügt, die zusammen mit
der Regel $U\to{\tt a}$ das Terminalsymbol {\tt a} in den genannten
Regeln ersetzen kann:
\begin{align*}
S_0&\to AA_1 \;|\; AS \;|\; SA \;|\; UB\;|\;{\tt a}\\
S&\to AA_1 \;|\; AS \;|\; SA \;|\; UB\;|\;{\tt a}\\
A&\to AA_1 \;|\; AS \;|\; SA \;|\; UB\;|\;{\tt a} \;|\; {\tt b}\\
A_1&\to SA\\
U&\to {\tt a}\\
B&\to{\tt b}
\end{align*}
Damit ist die Grammatik in Chomsky Normalform gefunden.
\end{enumerate}

\begin{satz}
Ist $L$ eine kontextfreie Sprache, dann gibt es eine Grammatik,
mit der jedes Wort $w$ mit einem Ableitungsbaum der Tiefe höchstens
$2|w|-1$ abgeleitet werden kann.
\end{satz}

\begin{proof}[Beweis]
Es gibt eine Grammatik in Chomsky Normalform, mit der die Wörter
der Sprache $L$ erzeugt werden können.
Da in Chomsky Normalform nur Regeln der Form $A\to BC$ und $A\to a$
enthalten kann, braucht es genau $|w|-1$ Anwendungen von Regeln der Form
$A\to BC$, um aus der Startvariable einen String aus $|w|$ Zeichen
herzustellen. Weitere $|w|$ Anwendungen von Regeln der Form $A\to a$
wandeln Variablen in Terminalsymbole um. Somit kann ein Wort mit
genau $2|w|-1$ Regelanwendungen produziert werden.
\end{proof}


\subsection{Ein deterministischer Parse-Algorithmus}
Ob ein Wort zu einer regulären Sprache gehört, kann in linearer
Zeit entschieden werden. Dazu erzeugt man einen DEA, der die Sprache
akzeptiert, und testet ein Wort damit.

Für kontextfreie Sprachen gibt es ebenfalls einen deterministischen
Algorithmus, den Cocke-Younger-Kasami Algorithmus. Dazu verwendet
man die Grammatik in Chomsky-Normalform. Der Algorithmus arbeitet
rekursiv um die Frage zu beantworten, ob ein Wort $w$ aus einer
Variablen $A$ der Grammatik $G=(V,\Sigma,R,S)$  abgeleitet
werden kann.

Um das leere Wort zu erzeugen, muss die Regel $S\to\varepsilon$
angewendet werden, falls sie vorhanden ist.

Ein Wort $w$ mit Länge $|w|=1$ kann genau dann aus der Variablen 
$A$ abgeleitet werden, wenn es eine Regel $A\to a$ gibt. 
Es kann höchstens $|\Sigma|$ solche Regeln geben.

Um ein Wort der Länge $|w|>1$ zu erzeugen, muss mindestens eine
Regel der Form
$A\to BC$ verwendet werden. Sie teilt das Wort in in zwei Teile
$w_1$ und $w_2$ mit je geringerer Länge: $w=w_1w_2$ und 
$|w_i|<|w|$. Es gibt $|w|-1$ solche Zerlegungen. Um also zu testen,
ob die Grammatik das Wort $w$ aus der Variablen $A$ ableitet,
muss man also rekursiv alle möglichen
Zerlegungen des Wortes daraufhin testen, ob sie von den Variablen
$B$ bzw.~$C$ erzeugt werden, ob also $B\overset{*}\Rightarrow w_1$
und $C\overset{*}\Rightarrow w_2$ gilt, und dies für jede
Regel $A\to BC\in R$. Es gibt weniger als $|R|$ solche Regeln.

Zur Implementation dieses rekursiven Algorithmus braucht man
also eine Funktion {\tt ableitbar}, die die Frage beantwortet,
ob ein Wort aus einer bestimmten Variable ableitbar ist. Sie
ruft sich für jede Aufteilung des Wortes in zwei Teile
$w_1$ und $w_2$ und für jede Regel $A\to BC$ selbst auf, um
herauszufinden, ob $w_1$ aus $B$ und $w_2$ aus $C$ ableitbar ist.

Der Pseudocode Algorithm~\ref{cyk-algorithm-code} implementiert
diesen Algorithmus.
\begin{algorithm}
\begin{algorithmic}[1]
\STATE {\tt boolean ableitbar(}Variable $V$, Wort $w${\tt ) \{}
\STATE \hskip1em{\tt if ((}$|w| = 0${\tt )} und {\tt (}Regel $V\to\varepsilon$ vorhanden{\tt )) \{}
\STATE \hskip2em {\tt return true; }
\STATE \hskip1em{\tt \}}
\STATE \hskip1em{\tt if ((}$|w| = 1${\tt )} und {\tt (}Regel $V\to w$ vorhanden{\tt ) \{}
\STATE \hskip2em{\tt return true; }
\STATE \hskip1em{\tt \}}
\STATE \hskip1emFür jede Unterteilung $w=w_1w_2$ mit $|w_1| > 0$ und $|w_2| > 0$ {\tt \{}
\STATE \hskip2emFür jede Regel $V\to AB$ {\tt \{}
\STATE \hskip3em{\tt if ((ableitbar(}$A,w_1${\tt ) \&\& (ableitbar(}$B,w_2${\tt )) \{ }
\STATE \hskip4em{\tt return true;}
\STATE \hskip3em{\tt \}}
\STATE \hskip2em{\tt \}}
\STATE \hskip1em{\tt \}}
\STATE \hskip1em{\tt return false;}
\STATE {\tt \}}
\end{algorithmic}
\caption{Algorithmus von Cocke-Younger-Kasami\label{cyk-algorithm-code}}
\end{algorithm}


\begin{satz}[Cocke-Younger-Kasami]
\index{Algorithmus!Cocke-Younger-Kasami}%
\index{Algorithmus!CYK|see{Cocke-Younger-Kasami}}%
\index{Cocke-Younger-Kasami}%
\label{cyk-algorithm}
Es gibt einen deterministischen Algorithmus mit Komplexität
$O(|w|^3)$, welcher entscheidet, ob $w\in L(G)$.
\end{satz}

Das Argument weiter oben erklärt, dass es einen determinisitischen
Algorithmus gibt. Es erklärt aber nicht, warum die Komplexität
nur $O(|w|^3)$ ist. Ähnlich wie man bei einer rekursiven Berechnung
der Fibonacci-Zahlen die Komplexität von $O(2^n)$ auf $O(n)$ reduzieren
kann, indem man bereits berechnet Fibonacci-Zahlen speichert, kann
man auch beim obigen rekursiven Algorithmus die Komplexität wesentlich
reduzieren, indem man bereits geprüfte Ableitungen zwischenspeichert.
Wir verzichten jedoch auf eine detaillierte Durchführung dieses
Beweises.

\section{Stackautomaten}
\index{Stackautomat}%
\rhead{Stackautomaten}
Kontextfreie Grammatiken erzeugen Sprachen, die von einem DEA nicht
akzeptiert werden können. Es ist also eine erweiterte Maschine
nötig, wenn sie solche Sprachen erkennen soll. Die Erweiterung muss
den DEA mit einem Speicher ausstatten, der zum Beispiel erlaubt, 
in der Sprache $\{0^n1^n\,\;|\; n\in\mathbb N\}$ die Anzahl der
$0$ zu speichern, damit später die Anzahl der $1$
damit verglichen werden kann.

Die einfachste Art von Speicher für diesen Zweck ist ein Stack:
man legt die $0$ auf den Stack, und entfernt für jede $1$ eine 
$0$. Wenn der Stack am Schluss leer ist, es ``aufgeht'', ist das
Wort akzeptabel.

\subsection{Formale Definition}
\begin{definition}
Ist $\Sigma$ eine endliche Menge, die das leere Wort $\varepsilon$
nicht enthält, dann setzen wir 
\[
\Sigma_\varepsilon = \Sigma\cup \{\varepsilon\}.
\]
\end{definition}

\begin{definition}
\index{Stackautomat}%
\index{Pushdown-Automat|see{Stackautomat}}%
Ein Stackautomat ist ein $6$-Tupel $(Q,\Sigma,\Gamma,\delta,q_0,F)$
mit endlichen Mengen $Q$, $\Sigma$, $\Gamma$ und $F$ und folgenden
Bezeichungen und Einschränkungen
\begin{enumerate}
\index{Zustand}%
\item Die Elemente von $Q$ heissen Zustände.
\index{Eingabe-Alphabet}%
\item $\Sigma$ ist das Eingabe-Alphabet
\index{Stack-Alphabet}%
\item $\Gamma$ ist das Stack-Alphabet
\item $\delta\colon Q\times \Sigma_\varepsilon\times\Gamma_\varepsilon
\to P(Q\times\Gamma_\varepsilon)$ heisst Übergangsfunktion
\index{Startzustand}%
\item $q_0\in Q$ heisst Startzustand
\index{Akzeptierzustand}%
\item $F\subset Q$ heisst Menge der Akzeptierzustände.
\end{enumerate}
\end{definition}
Die Elemente $Q$, $q_0$ und $F$ scheinen für sich genommen
einen endlichen Automaten zu definieren, jedoch einen, der je nach
dem Wert, der als drittes Argument der Übergangsfunktion
übergeben wird, sich etwas anders verhält. Die Idee ist, dass
dieses dritte Argument von einem Stack gelesen werden soll, der in
der Definition nicht explizit ausgedrückt sein muss, da er
sich für alle Stackautomaten gleich verhält: Man kann Zeichen
aus $\Gamma$ dort hinschreiben, oder von dort lesen\footnote{Man könnte
die Definition eines Stackautomaten auch als C++-Template mit
sechs Template-Argumenten betrachten. Um den Stack zu
implementieren, muss man nur wissen, was man darauf ablegen will,
man muss also nur $\Gamma$ kennen.}.

Im Gegensatz zu den endlichen Automaten, wo wir grossen Wert auf die
Unterscheidung zwischen deterministischen und nicht deterministischen
Automaten gelegt haben, definieren wir Stackautomaten nur
``nichtdeterministisch''.

\subsection{Gerichteter beschrifteter Graph}
% XXX Idee: Zu den Stack-Operationen Bilder hinzufuegen
\index{Graph!gerichteter!beschrifteter!eines Stackautomaten}%
Auch zu einem Stackautomaten gibt es einen gerichteten beschrifteten
Graphen. Die Übergangsfunktion legt jetzt zu jedem Zustand 
fest, was für ein Zeichen verarbeitet wird, was für ein
neuer Zustand erreicht wird, und wie sich der Stackinhalt
verändert.
Dazu wird diese zusätzliche Information der Beschriftung der Pfeile
hinzugefügt.
\[
\xymatrix{
*++[o][F]{p}\ar[r]^{a,b\to c}
	&*++[o][F]{q}
}
\]
bedeutet, dass der Automat bei der Verarbeitung eines Zeichens $a$
vom Zustand $p$ in den Zustand $q$ übergeht, wenn gleichzeitig
ein Zeichen $b$ zuoberst auf dem Stack durch das Zeichen $c$
ersetzt werden kann. Für Übergänge mit dem leeren Wort gilt:
\[
\xymatrix{
*++[o][F]{p}\ar[r]^{a,\varepsilon\to c}
	&*++[o][F]{q}
		&\text{$c$ wird auf den Stack gelegt}
\\
*++[o][F]{p}\ar[r]^{a,b\to\varepsilon}
	&*++[o][F]{q}
		&\text{$b$ wird vom Stack entfernt}
\\
*++[o][F]{p}\ar[r]^{a,\varepsilon\to\varepsilon}
	&*++[o][F]{q}
		&\text{Stack bleibt unverändert}
}
\]
In allen Fällen darf $a$ auch das leere Wort sein, was Operationen
ergibt, die den Stack verändern, ohne dass dazu Input verarbeitet werden
muss.

% XXX weitere Illustrationen der Stack-Operationen hinzufuegen:
% p --- eps,b->eps ---> q
% p --- eps,eps->eps ---> q

\subsection{Beispiele\label{stackbeispiele}}
\subsubsection{Stackautomaten können ``zählen''}
Um die Wörter der nicht reguläre Sprache $L=\{0^n1^n\;|\;n\in \mathbb N\}$
zu erkennen, braucht man einen Zähler, mit dem man die Zahl der $0$
mit der Zahl der $1$ vergleichen kann. In einem endlichen Automaten
gab es dafür keinen Platz, aber der Stack eines Stackautomaten kann
diese Aufgabe übernehmen.

Um zu erkennen, dass der Stack wieder leer ist, braucht man ein Zeichen,
welches den Anfang des Stacks markiert. Das Stack-Alphabet muss also 
etwas grösser sein, wir fügen das Zeichen {\tt \$} hinzu, also
$\Gamma=\{{\tt 0},{\tt 1},{\tt \$}\}$.
Die Idee des konstruierten Stackautomaten ist den Stack als Ablage der
verarbeiteten {\tt 0} zu verwenden, und anschliessend beim verarbeiten
der  {\tt 1} die Nullen wieder vom Stack zu nehmen. Wenn alles ``aufgeht'',
wurde ein Wort mit gleich vielen Nullen wie Einsen verarbeitet, also 
ein Wort in $L$.
\[
\entrymodifiers={++[o][F]}
\xymatrix{
*+\txt{}\ar[r]
	&{}\ar[r]^{\varepsilon,\varepsilon\to{\tt \$}}
		&{} \ar@(ur,dr)^{{\tt 0},\varepsilon\to{\tt 0}}
		    \ar[d]^{\varepsilon,\varepsilon\to\varepsilon}
\\
*+\txt{}
	&*++[o][F=]{}
		&{}\ar[l]^{\varepsilon,{\tt \$}\to\varepsilon}
		   \ar@(ur,dr)^{{\tt 1},{\tt 0}\to\varepsilon}
}
\]
Damit ist gezeigt, dass die nicht reguläre Sprache $L$ von einem Stackautomaten
akzeptiert wird.

\subsubsection{Die Sprache $L=\{a^ib^jc^k\;|\;i,j,k\in\mathbb N, i=j\vee i=k\}$}
Auch in dieser Sprache muss man die $a$ auf den Stack legen, um sie
zu zählen. Dann muss man sich nicht deterministisch entscheiden,
ob man auf gleich viele $b$ oder auf gleich viele $c$ testen will,
beides kann man nicht haben, da nach einem Test der Stack wieder leer
ist. Als Alphabet verwenden wir daher wieder $\Gamma=\{a,b,c,{\tt\$}\}$.
Der Automat muss also zunächst das Zeichen {\tt\$} auf den Stack legen.
Dann beginnt er entweder, die $b$ zu zählen, und die $c$ zu ignorieren,
oder die $b$ zu ignorieren und dann die $c$ zu zählen.
\[
\entrymodifiers={++[o][F]}
\xymatrix{
*+\txt{}
	&*+\txt{}\ar[d]
\\
*+\txt{}
	&{}\ar[d]^{\varepsilon,\varepsilon\to{\tt\$}}
\\
{}\ar@(u,l)_{b,a\to\varepsilon}
  \ar[d]_{\varepsilon,\varepsilon\to\varepsilon}
	&{}\ar@(dl,dr)_{a,\varepsilon\to a}
	   \ar[l]_{\varepsilon,\varepsilon\to\varepsilon}
	   \ar[r]^{\varepsilon,\varepsilon\to\varepsilon}
		&{}\ar@(u,r)^{b,\varepsilon\to\varepsilon}
		   \ar[d]^{\varepsilon,\varepsilon\to\varepsilon}
\\
{}\ar@(l,d)_{c,\varepsilon\to\varepsilon}
  \ar[r]_{\varepsilon,{\tt\$}\to\varepsilon}
	&*++[o][F=]{}
		&\ar@(r,d)^{c,a\to\varepsilon}
		  \ar[l]^{\varepsilon,{\tt\$}\to\varepsilon}
}
\]

\subsection{Äquivalenz von Stackautomaten und CFG\label{sect:aequivalenz-cfg}}
So wie reguläre Ausdrücke und endliche Automaten zur Beschreibung
von regulären Sprachen äquivalent sind, so sind auch kontexfreie Grammatiken
mit Stackautomaten äquivalent:

\begin{satz}
Eine Sprache ist genau dann kontextfrei, wenn sie von einem
Stackautomaten akzeptiert wird.
\end{satz}

Es ist einerseits zu zeigen, dass sich für jede Grammatik ein Stackautomat
finden lässt, der die von der Grammatik erzeugte Sprache akzeptiert.
Andererseits muss auch zu einem beliebigen Stackautomaten eine 
Grammatik gefunden werden, welche die gleiche Sprache erzeugt, die
auch der Stackautomat akzeptiert.

\begin{hilfssatz}\label{hilfssatz_cfg_to_pushdown}
Ist $L$ eine kontextfreie Sprache mit Grammatik $G=(V,\Sigma,R,S)$,
dann gibt es einen Stackautomaten, der $L$ akzeptiert.
\end{hilfssatz}

\begin{proof}[Beweis]
Die Grammatik erzeugt die Wörter dadurch, dass sie auf die Startvariable
immer wieder Regeln aus $R$ anwendet. Diesen Prozess kann man auf dem
Stack nachbilden. Zu Beginn legt man die Startvariable auf den Stack.
Die Anwendung von Regeln besteht darin, eine Variable vom Stack zu nehmen,
und stattdessen die Zeichen auf der rechten Seite der Regel
auf den Stack zu legen. 

Als Stackalphabet verwenden wir $\Gamma = V\cup \Sigma \cup \{{\tt \$}\}$.
Zu Beginn wird das Symbol {\tt\$} auf den Stack gelegt, am Schluss wird 
es wieder entfernt, es sind also insgesamt vier Zustände erforderlich:
\[
\entrymodifiers={++[o][F]}
\xymatrix{
*+\txt{}\ar[r]
	&{q_0} \ar[r]^{\varepsilon,\varepsilon\to{\tt\$}}
		&{S}\ar[r]^{\varepsilon,\varepsilon\to S}
			&{R}\ar[r]^{\varepsilon,{\tt\$}\to\varepsilon}
				&*++[o][F=]{A}
}
\]
Jetzt müssen im Zustand $R$ noch Übergänge hinzugefügt werden, mit denen
die Regeln abgebildet werden.
Dazu dürfen wir annehmen, dass die Grammatik bereits in Chomsky Normalform
ist. Wir müssen also nur noch für Regeln der Form $A\to BC$ und $A\to a$
in Übergänge abbilden. Die Regel $A\to a$ ersetzt die Variable $A$
durch das Terminalsymbol $a$, das erreicht man mit dem Übergang
\[
\entrymodifiers={++[o][F]}
\xymatrix{
{R}\ar@(ur,dr)^{\varepsilon,A\to a}
}
\]
Für die Regel $A\to BC$ brauchen wir einen zusätzlichen Zustand mit
den beiden Übergängen
\[
\entrymodifiers={++[o][F]}
\xymatrix{
R\ar@/^/[r]^{\varepsilon,A\to C}
	&{}\ar@/^/[l]^{\varepsilon,\varepsilon\to B}
}
\]
Sobald auf dem Stack ein Terminalsymbol liegt, kann dieses mit
den Regeln der Grammatik nicht mehr verarbeitet werden, muss 
aber auf das nächste Zeichen des Eingabewortes passen. Die Regel
\[
\entrymodifiers={++[o][F]}
\xymatrix{
{R}\ar@(ur,dr)^{a,a\to\varepsilon}
}
\]
entfernt ein solches Terminalssymbol, wenn das gleiche Symbol
als Input anliegt.

Der so konstruierte Automat akzeptiert genau die
Wörter, die von der Grammatik erzeugt werden.
\end{proof}

Wir könnten diesen Beweis mit folgender Notation noch etwas
prägnanter formulieren. Die Regeln $A\to BC$ werden dadurch abgebildet,
dass ein zusätzlicher Zustand hinzugefügt wird, um in zwei Schritten
die Variable $A$ vom Stack zu entfernen und stattdessen die zwei Variablen
$BC$ dort abzulegen. Man möchte also eigentlich die Operatione
$\varepsilon,A\to BC$ implementieren. Daher soll das Symbol
\[
\entrymodifiers={++[o][F]}
\xymatrix{
{p}\ar[d]^{a,s\to xyz}
\\
{q}
}
\]
implizit zwei neue Zustände $q_1$ und $q_2$ und die zughörigen
Übergänge wie im folgenden Zustandsdiagramm definieren:
\[
\entrymodifiers={++[o][F]}
\xymatrix{
{p}\ar[r]^{a,s\to z}
	&{q_1}\ar[d]^{\varepsilon,\varepsilon\to y}
\\
{q}
	&{q_2}\ar[l]^{\varepsilon,\varepsilon\to x}
}
\]
Analog für beliebige Wörter auf der rechten Seite der Regel,
für ein Wort $w$ der Länge $|w|$ müssen $|w|-1$ neue
Zustände hinzugefügt werden.

\begin{beispiel}[\bf Beispiel]
Für die Grammatik mit den Regeln $S\to {\tt 0}S{\tt 1}\;|\;\varepsilon$
finde man einen Stackautomaten.

Wir haben zwar nur die Übersetzung für Regeln einer Grammatik in
Chomsky Normalform diskutiert, das Prinzip ist jedoch übertragbar.
Wir stellen die Übergänge zusammen, die im Zustand $R$ anzufügen
sind. Für die Verarbeitung der Terminalsymbole brauchen wir
für jedes Terminalsymbol einen Übergang, der gleichzeitig
ein Symbol vom Input wie vom Stack entfernt:
\begin{equation}
\entrymodifiers={++[o][F]}
\xymatrix{
{R}\ar@(ur,dr)^{{\tt 0},{\tt 0}\to\varepsilon}
   \ar@(ul,dl)_{{\tt 1},{\tt 1}\to\varepsilon}
}
\label{beispiel-tregel}
\end{equation}
Die Regel $S\to\varepsilon$ wird übergeführt in einen Übergang
\begin{equation}
\entrymodifiers={++[o][F]}
\xymatrix{
{R}\ar@(ur,dr)^{\varepsilon, S\to\varepsilon}
}
\label{beispiel-epsilon}
\end{equation}
Die Regel $S\to {\tt 0}S{\tt 1}$ muss übergeführt werden
in die drei Schritte
\begin{equation}
\entrymodifiers={++[o][F]}
\xymatrix{
{R}\ar[rr]^{\varepsilon,S\to{\tt 1}}
	&*+\txt{}
		&{q_1}\ar[dl]^{\varepsilon,\varepsilon\to S}
\\
*+\txt{}
	&{q_2}\ar[ul]^{\varepsilon,\varepsilon\to{\tt 0}}
}
\label{beispiel-0s1regel}
\end{equation}
Jetzt kann man den Ablauf des Akzeptierens des Wortes {\tt 0011}
verfolgen. Offenbar ist die Ableitung 
\[
S
\overset{\text{(\ref{beispiel-0s1regel})}}{\longrightarrow}
{\tt 0}S{\tt 1}
\overset{\text{(\ref{beispiel-0s1regel})}}{\longrightarrow}
{\tt 00}S{\tt 11}
\overset{\text{(\ref{beispiel-epsilon})}}{\longrightarrow}
{\tt 0011}
\]
Die Tabellen~\ref{beispiel-tabelle} zeigt den Inhalt des 
Input und des Stacks nach jedem Übergang des Stackautomaten.
\begin{table}
\begin{center}
\begin{tabular}{|c|l|c|l|}
\hline
Übergang&Input&Zustand&Stack\\
\hline
                         &{\tt 0011}&$q_0$&\\
                         &{\tt 0011}&$S$  &${\tt \$}$\\
                         &{\tt 0011}&$R$  &$S\quad {\tt \$}$\\
(\ref{beispiel-0s1regel})&{\tt 0011}&$q_1$&${\tt 1}\quad{\tt \$}$\\
(\ref{beispiel-0s1regel})&          &$q_2$&$S\quad {\tt 1}\quad{\tt \$}$\\
(\ref{beispiel-0s1regel})&          &$R$  &${\tt 0}\quad S\quad {\tt 1}\quad{\tt \$}$\\
(\ref{beispiel-tregel})  &{\tt 011} &$R$  &$S\quad {\tt 1}\quad{\tt \$}$\\
(\ref{beispiel-0s1regel})&{\tt 011}&$q_1$&${\tt 1}\quad {\tt 1}\quad{\tt \$}$\\
(\ref{beispiel-0s1regel})&          &$q_2$&$S\quad {\tt 1}\quad {\tt 1}\quad{\tt \$}$\\
(\ref{beispiel-0s1regel})&          &$R$  &${\tt 0}\quad S\quad {\tt 1}\quad {\tt 1}\quad{\tt \$}$\\
(\ref{beispiel-tregel})  &{\tt 11}  &$R$  &$S\quad {\tt 1}\quad {\tt 1}\quad{\tt \$}$\\
(\ref{beispiel-epsilon}) &{\tt 11}  &$R$  &${\tt 1}\quad {\tt 1}\quad{\tt \$}$\\
(\ref{beispiel-tregel})  &{\tt 1}   &$R$  &${\tt 1}\quad{\tt \$}$\\
(\ref{beispiel-tregel})  &{\tt }    &$R$  &${\tt \$}$\\
                         &{\tt }    &$A$  &\\
\hline
\end{tabular}
\end{center}
\caption{Ablauf des Akzeptierens des Wortes {\tt 0011} mit
Hilfe des Stackautomaten zur Grammatik $S\to {\tt 0}S{\tt 1}\;|\;\varepsilon$
\label{beispiel-tabelle}}
\end{table}
\end{beispiel}

\begin{hilfssatz}\label{pda_has_grammar}
Ist $P$ ein Stackautomat, dann gibt es eine Grammatik $G$, die die
gleiche Sprache produziert: $L(P)=L(G)$.
\end{hilfssatz}

\begin{proof}[Beweis]
Wir müssen eine Grammatik konstruieren, die die gleiche Sprache
erzeugt. Der Stackautomat sollte möglichst nahe an dem in
Hilfssatz \ref{hilfssatz_cfg_to_pushdown} konstruierten liegen, damit es
möglichst einfach wird, sich von der Äquivalenz zu überzeugen.
Wir führen daher an $P$ zunächst drei einfache Modifikationen
durch: 
\begin{enumerate}
\item $P$ hat nur einen einzigen Akzeptierzustand
$q_a$.
Dies erreicht man dadurch, dass man alle Akzeptierzustände über
einen $\varepsilon$-Übergang mit
$q_a$ verbindet, und bisherigen Akzeptierzustände
werden dann zu gewöhnlichen Zuständen degradiert:
\[
\xymatrix{
*++[o][F=]{q}
	&\Rightarrow
		&*++[o][F]{q}\ar[r]^{\varepsilon,\varepsilon\to\varepsilon}
			&*++[o][F=]{q_a}
}		
\]
Der modifizierte Automat akzeptiert die gleichen Wörter wie $P$.
\item Vor dem Akzeptieren wird der Stack geleert. Dazu wird ein zusätzliches
Zeichen benötigt, welches das Ende des Stacks signalisiert. Es wird zu
Beginn auf den Stack geschrieben, und als letzter Übergang wieder
vom Stack genommen. Der Startzustand $q_0$ wird also ersetzt durch einen neuen
Startzustand $q_0'$, mit dem einzigen Übergang 
\[
\entrymodifiers={++[o][F]}
\xymatrix{
*+\txt{}\ar[r]
	&{q_0'}\ar[r]^{\varepsilon,\varepsilon\to {\tt\$}}
		&q_0
}
\]
und der Akzeptierzustand  durch einen neuen Akzeptierzustand $q_a'$ und
den Übergang
\[
\entrymodifiers={++[o][F]}
\xymatrix{
{q_a}\ar[r]^{\varepsilon,{\tt\$}\to\varepsilon}
	&*++[o][F=]{q_a'}
}
\]
\item Jeder Übergang legt ein Zeichen auf den Stack, oder entfernt
eines, aber er macht nicht beides gleichzeitig. Dazu muss jeder Übergang,
der gleichzeitig ein Zeichen vom Stack nimmt und ein neues dort ablegt,
mit Hilfe eines neuen Zustandes in zwei Übergänge aufgeteilt werden:
\[
\xymatrix{
*++[o][F]{p}\ar[dd]^{a,b\to c}
	&*+\txt{}
		&*++[o][F]{p}\ar[d]^{a,b\to\varepsilon}
\\
	&\Rightarrow
		&*++[o][F]{s}\ar[d]^{\varepsilon,\varepsilon\to c}
\\
*++[o][F]{q}
	&
		&*++[o][F]{q}
}
\]
Ausserdem müssen Übergänge, die gar nicht auf dem Stack operieren,
durch Übergänge ersetzt werden, die ein beliebiges Stacksymbol
schreiben und gleich wieder entfernen:
\[
\xymatrix{
*++[o][F]{p}\ar[dd]^{a,\varepsilon\to \varepsilon}
	&*+\txt{}
		&*++[o][F]{p}\ar[d]^{a,\varepsilon\to x}
\\
	&\Rightarrow
		&*++[o][F]{s}\ar[d]^{\varepsilon,x\to \varepsilon}
\\
*++[o][F]{q}
	&
		&*++[o][F]{q}
}
\]
\end{enumerate}
Damit können wir jetzt zur Konstruktion der Grammatik schreiten.
Die Idee dabei ist, Variablen $A_{pq}$ zu verwenden, die alle 
Wörter erzeugen, die Zustandsübergängen entsprechen,
die den Automaten vom Zustand $p$ in den Zustand $q$ bringen,
und den Stack im gleichen Zustand zurücklassen, den sie vorgefunden
haben.
Wir setzen $A_{pq}$ mit der Menge der Wörter gleich, die
auf diese Weise erzeugt werden können.

Dazu müssen wir verstehen, wie die Wörter entstehen, die von $P$
akzeptiert werden. Während der Berechnung, die zum Akzeptieren des
Wortes $w$ führt, wird zunächst
irgendein Zeichen $x$ auf den Stack gelegt, und am Ende muss ein
Zeichen entfernt werden, welches $x$ oder auch etwas Anderes sein kann.

Falls das am Ende entfernte Zeichen $x$ ist, können wir dies durch
die Regel $A_{pq}\to aA_{rs}b$ symbolisieren, wobei $a$ das Zeichen
ist, welches zusammen mit dem Ablegen von $x$ vom Input verarbeitet wird,
und $b$ das Zeichen, das mit dem Entfernen von $x$ verarbeitet wird.
Der Stack ist zwischen den Zuständen $r$ und $s$ unverändert, wir
können ihn schematisch wie in Abbildung~\ref{stacknichtleer} darstellen.
\begin{figure}
\begin{center}
%\includegraphics[width=0.85\hsize]{images/stack-2.pdf}
\includegraphics{images/stack-2.pdf}
\end{center}
\caption{Übergänge, die den Stack in keinem Zwischenzustand leeren\label{stacknichtleer}}
\end{figure}

Falls das am Schluss vom Stack entfernte Zeichen nicht das Gleiche ist,
dann muss das $x$ irgendwann im Laufe der Berechnung entfernt worden
sein, und das neue Zeichen $y$ muss auf dem Stack abgelegt worden
sein. Es gibt also einen Zwischenzustand $r$, in dem der Stack
wieder im selben Zustand wie zu
Beginn der Berechnung ist, wir können dies durch
$A_{pq}\to A_{pr}A_{rq}$ symbolisieren.
Die Abbildung~\ref{stackleer} zeigt diese Situation schematisch.
\begin{figure}
\begin{center}
%\includegraphics[width=0.85\hsize]{images/stack-1.pdf}
\includegraphics{images/stack-1.pdf}
\end{center}
\caption{Übergänge, die zwischenzeitlich den Stack leeren\label{stackleer}}
\end{figure}

Formal konstruieren wir also eine Grammatik mit Variablen
$V=\{A_{pq}\;|\; p,q\in Q\}$, Startvariable ist $A_{q_0,q_a}$.
Die Menge der Regeln bauen wir wie folgt auf:
\begin{itemize}
\item Für $p,q,r,s\in Q$, $t\in\Gamma$ und $a,b\in\Sigma_{\varepsilon}$:
Falls $\delta(p,a,\varepsilon)$ das Paar $(r,t)$ enthält
und $\delta(s,b,t)$ das Paar $(q,\varepsilon)$, füge die Regel
$A_{pq}\to aA_{rs}b$ hinzu.

Diese Regel besagt, dass Wörter zwischen $p$ und $q$ dadurch gebildet
werden können, dass zunächst ein Zeichen $a$ verarbeitet, und
ein Zeichen $t$ auf den Stack geschrieben wird, dann wird ein Wort
in $A_{rs}$ erzeugt, und zum Schluss das Zeichen $t$ unter
gleichzeitiger Verarbeitung des Zeichens $b$ wieder vom Stack
genommen.

\[
\begin{gathered}
\entrymodifiers={++[o][F]}
\xymatrix{
{p}\ar[d]\ar[r]^{a,\varepsilon\to t}
	&{r}\ar[d]
\\
{q}
	&{s}\ar[l]^{b,t\to\varepsilon}
}
\end{gathered}
\qquad\rightsquigarrow\qquad A_{pq}\to aA_{rs}b
\]


\item Für drei Zustände $p,q,r\in Q$ füge die Regel 
$A_{pq}\to A_{pr}A_{rq}$ hinzu.
\[
\begin{gathered}
\entrymodifiers={++[o][F]}
\xymatrix{
{p}\ar[dr]\ar[dd]
\\
*+\txt{}
	&{r}\ar[dl]
\\
{q}
}
\end{gathered}
\qquad\rightsquigarrow\qquad A_{pq}\to A_{pr}A_{rq}
\]
\item Für jeden Zustand $p\in Q$ füge die Regel $A_{pp}\to \varepsilon$
hinzu:
\[
\begin{gathered}
\entrymodifiers={++[o][F]}
\xymatrix{
*+\txt{}
	&{p}
}
\end{gathered}
\qquad\rightsquigarrow\qquad
A_{pp}\to\varepsilon.
\]
\end{itemize}
Jetzt muss man nur noch zeigen, dass ein Wort $w$ genau dann aus $A_{pq}$
abgeleitet werden kann, wenn es $P$ vom Zustand $p$ mit leerem Stack
in den Zustand $q$ mit leerem Stack bringen kann.


\begin{hilfssatz}\label{apq_generates_x_implies}
Falls $A_{pq}$ das Wort $x$ erzeugt, dann kann $x$ $P$ aus dem Zustand
$p$ mit leerem Stack in den Zustand $q$ mit leerem Stack überführen.
\end{hilfssatz}

\begin{proof}[Beweis von Hilfssatz \ref{apq_generates_x_implies}]
Man kann vollständige Induktion für die Länge der be\-nötigten 
Ableitung $A_{pq}\overset{*}{\Rightarrow} x$ verwenden.

Falls die Ableitung mit nur einem Schritt möglich ist, dann muss
dazu eine Regel der Form $A_{pp}\to\varepsilon$ verwendet werden,
denn dies sind die einzigen Regeln, die auf der rechten Seite
keine Variablen enthalten. 

Nehmen wir also an, für Ableitungen der Länge $k$ sei bereits
bekannt, dass sie den Automaten von einem Zustand mit leerem Stack
in einen anderen Zustand mit leerem Stack überführen.

Sei jetzt $A_{pq}\overset{*}{\Rightarrow}x$ eine Ableitung mit $k+1$
Schritten. Der erste Schritt dieser Ableitung ist entweder eine
Regel der Form $A_{pq}\to aA_{rs}b$ oder $A_{pq}\to A_{pr}A_{rq}$.
Im ersten Fall sagt die Regel, dass $x=ayb$, wobei $y$ ein
Wort ist, welches aus $A_{rs}$ in höchstens $k$ Schritten abgeleitet
werden kann. Also kann nach Induktionsannahme $y$ den Automaten vom
Zustand $r$ mit leerem Stack in den Zustand $s$ mit leerem Stack
überführen. Der Übergang von $q$ nach $r$ mit Inputzeichen $a$
legt möglicherweise ein Zeichen $t$ auf den Stack, nach Konstruktion
der Produktionsregeln wird dieses vom Übergang mit Inputzeichen $b$
auch wieder entfernt, so dass der Stack wieder leer ist.

Im zweiten Falls ist nach Induktionsannahme $x=yz$, wobei $y$ den
Automaten vom Zustand $q$ in den Zustand $r$ je mit leerem Stack
überführt, und $z$ ihn von $r$ nach $q$ je mit leerem Stack
führt.

In beiden Fällen folgt, dass $x$ den Automaten von $p$
nach $q$ je mit leerem Stack führen kann.
Damit ist der Induktionsschritt vollzogen, und es folgt, dass
sich jede Ableitung durch Übergänge im Stackautomaten zwischen
Zuständen mit jeweils leerem Stack bilden lassen.
\end{proof}

\begin{hilfssatz}\label{implies_apq_generates_x}
Falls $x$ $P$ aus dem Zustand $p$ mit leerem Stack in den Zustand
$q$ mit leerem Stack überführen kann, dann ist
$A_{pq}\overset{*}{\Rightarrow} x$.
\end{hilfssatz}

\begin{proof}[Beweis von Hilfssatz \ref{implies_apq_generates_x}]
Auch diesen Teil kann man mit vollständiger Induktion beweisen,
diesmal über die Länge der Berechnung. 

Die kürzeste mögliche Berechnung hat $0$ Schritte, d.\,h.~sie endet
im gleichen Zustand, in dem sie begonnen hat.
Und tatsächlich enthält die Grammatik die Regel $A_{pp}\to\varepsilon$, 
so dass das leere Wort tatsächlich aus $A_{pp}$ mit leerem
Stack abgeleitet werden kann.

Nehmen wir jetzt an, dass bereits bekannt ist, dass Berechnungen mit
$k$ Schritten, welche $P$ mit dem Input-Wort $w$ vom Zustand $p$
in den Zustand $q$ je mit leerem Stack dazu führen, dass
$A_{pq}\overset{*}{\Rightarrow} x$.

Sei jetzt also eine Berechnung mit Inputwort $x$
mit $k+1$ Schritten gegeben, die $P$ vom Zustand $p$ in den Zustand $q$ 
je mit leerem Stack führt. Entweder ist der Stack nur ganz zu
Beginn oder ganz am Schluss leer, oder er wird dazwischen einmal
leer.

Im ersten Fall wird ein Symbol $t$ beim ersten Schritt auf den Stack
gelegt, und beim letzten Schritt entfernt. Es gibt also zwei Zustände
$r$ und $s$ und Übergänge
\[
\entrymodifiers={++[o][F]}
\xymatrix{
p\ar[r]^{a,\varepsilon\to t}
	&r\ar[r]
		&s\ar[r]^{b,t\to\varepsilon}
			&q
}
\]
Die Berechnung, die $P$ von $r$ in $s$ überführt, ist kürzer, und kann
mit leerem Stack durchgeführt werden.
Also ist sie nach Induktionsannahme der Teil $y$ in $x=ayb$ aus ableitbar 
$A_{rs}$ ableitbar.

Im zweiten Fall zerfällt die Berechnung, in der $x$ $P$ von $p$ in
$q$  mit leerem Stack überführt in zwei Teile, die mit Inputwörtern
$y$ und $z$ $p$ in $r$ bzw.~$r$ in $q$ je mit leerem Stack überführen:
\[
\entrymodifiers={++[o][F]}
\xymatrix{
p\ar[r]
	&r\ar[r]
		&q
}
\]
Nach Induktionsannahme ist daher
$A_{pr}\overset{*}{\Rightarrow}y$
$A_{rq}\overset{*}{\Rightarrow}z$, und zusammen mit der Regel
$A_{pq}\to A_{pr}A_{rq}$ der Grammatik auch
$A_{pq}\overset{*}{\Rightarrow} x$
\end{proof}

Damit ist auch der Beweis von Hilfssatz \ref{pda_has_grammar} vollständig.
\end{proof}

\begin{beispiel}
Wir möchten die Theorie dazu verwenden, für den Stackautomaten aus
Abschnitt~\ref{stackbeispiele} für die Sprache
$L=\{\texttt{0}^n\texttt{1}^n\;|\; n\ge 0\}$ eine Grammatik zu finden.
Dazu wandeln wir zunächst den Stackautomaten in die Standardform um, die
wir für die Konstruktion der Grammatik brauchen. Dazu müssen wir
im vertikalen Übergang einen Zwischenzustand einfügen:
\[
\entrymodifiers={++[o][F]}
\xymatrix{
*+\txt{}\ar[r]
	&{q_0}\ar[r]^{\varepsilon,\varepsilon\to{\tt \$}}
		&{q_1} \ar@(ur,dr)^{{\tt 0},\varepsilon\to{\tt 0}}
		    \ar[d]^{\varepsilon,\varepsilon\to x}
\\
*+\txt{}
	&*+\txt{}
		&{q_2}\ar[d]^{\varepsilon,x\to\varepsilon}
\\
*+\txt{}
	&*++[o][F=]{q_a}
		&{q_3}\ar[l]^{\varepsilon,{\tt \$}\to\varepsilon}
		   \ar@(ur,dr)^{{\tt 1},{\tt 0}\to\varepsilon}
}
\]
Die Startvariable der Grammatik ist $A_{q_0q_a}$. Die Übergänge, die
das Zeichen $\texttt{\$}$ behandeln, führen zu einer Regel
\begin{equation}
A_{q_0q_a}\to \varepsilon A_{q_1q_3}\varepsilon = A_{q_1q_3}.
\label{q0qa}
\end{equation}
Wenn man im Zustand $q_1$ eine $\texttt{0}$ auf den Stack legt, dann
muss man sie auch im Zustand $q_3$ wieder entfernen. Dieser Prozess
gibt Anlass zu einer Regel
\begin{equation}
A_{q_1q_3}\to \texttt{0}\; A_{q_1q_3}\;\texttt{1}
\label{q1q3}
\end{equation}
Man kann aber auch ohne eine $\texttt{0}$ auf dem Stack von
$q_1$ zu $q_3$ gelangen, das führt zu der Regel 
\begin{equation}
A_{q_1q_3}\to \varepsilon\; A_{q_2q_2}\;\varepsilon
\label{q2q2}
\end{equation}
Zusätzlich hat man noch die Regel $A_{q_2q_2}\to\varepsilon$, da 
$A_{q_2q_2}$ nur in (\ref{q2q2}) gebraucht wird, können wir sie
bereits anwenden, und erhalten so die Grammatik.
\begin{align}
A_{q_0q_a}&\to A_{q_1q_3} \tag{\ref{q0qa}}\\
A_{q_1q_3}&\to \texttt{0}\; A_{q_1q_3}\;\texttt{1} \tag{\ref{q1q3}}\\
A_{q_1q_3}&\to \varepsilon\tag{\ref{q2q2}}
\end{align}
Es werden nur noch zwei Variablen verwendet, aber die Startvariable
wird immer in $A_{q_1q_3}$ umgewandelt, man kann sie also auch noch
einsparen. Schreiben wir $S=A_{q_1q_3}$ bleibt somit als
Grammatik
\begin{align}
S&\to \texttt{0}\; S\; \texttt{1}\tag{\ref{q1q3}}\\
S&\to\varepsilon,\tag{\ref{q2q2}}
\end{align}
genau die Grammatik, die wir früher schon für $L$ gefunden haben.
\end{beispiel}

\section{Nicht kontextfreie Sprachen}
\rhead{Nicht kontextfreie Sprachen}
Mit einem Stack kann man nur über eine Art von eingegangenen
Zeichen Buch führen. Möchte man mehrere Zeichen verfolgen,
bräuchte man freien Zugriff auf verschiedene Zähler für jedes
Zeichen. Eine Maschine mit so vielen Stacks wie Zeichen zur Verfügung
stehen, könnte also auch die Sprache
\[
L=\{ {\tt a}^n {\tt b}^n {\tt c}^n\;|\;n\in\mathbb N\}
\]
erkennen. Ein ``standard'' Stackautomat kann das jedoch nicht,
in diesem Abschnitt soll gezeigt werden, dass $L$ nicht kontextfrei
ist. 

\subsection{Pumping Lemma für kontextfreie Sprachen}
\index{Pumping Lemma!f\ür kontextfreie Sprachen}%
Bei regulären Sprachen ermöglicht das Pumping-Lemma den einfachen
Beweis, dass eine Sprache nicht regulär ist. Die Grundidee dabei ist,
dass sich in einem endlichen Automaten früher oder später ein
Zustand wiederholen muss, und dass die Schleife zwischen dem ersten
und dem zweiten Auftreten dieses Zustandes beliebig oft durchlaufen
werden kann, um immer wieder neue akzeptable Wörter zu liefern.

Die Äquivalenz von kontextfreien Sprachen mit Sprachen, die von einem
Stackautomaten akzeptiert werden können, suggeriert, dass so etwas
auch für kontextfreie Sprachen möglich sein sollte. Stackautomaten
wurden aber grundsätzlich als nicht deterministische Automaten
konstruiert, wo die Argumente, die das Pumping Lemma für reguläre
Sprachen geliefert haben, nicht direkt anwendbar sind. Wir verwenden
daher nicht einen Stackautomaten, sondern direkt die Grammatik, um
das Pumping Lemma herzuleiten.

\begin{figure}
\begin{center}
\includegraphics[width=0.35\hsize]{images/cfg-1}
\end{center}
\caption{Parse Tree für die Erzeugung des Wortes $uvxyz$ aus der
Startvariablen $S$.\label{cfg-tree-1}}
\end{figure}
\begin{figure}
\begin{center}
\begin{tabular}{cc}
\includegraphics[width=0.35\hsize]{images/cfg-2}&%
\includegraphics[width=0.35\hsize]{images/cfg-3}\\
\end{tabular}
\end{center}
\caption{Parse Tree für das aufgepumpte Wort $uv^2xy^2z$ (links) und das
abgepumpte Wort $uxz$ (rechts).\label{cfg-tree-2}}
\end{figure}

Ist ein Wort $w\in L(G)$ genügend lang, gibt es auch lange Pfade im
Ableitungsbaum. Bei Verwendung der Chomsky-Normalform ist die 
Tiefe des Baumes im besten Fall $\log_2 |w|$, im schlimmsten Fall $|w|-1$.
Ist $|w|>2^{|V|}$, muss mindestens eine
Variable auf einem Ast des Baumes zweimal verwendet werden\footnote{
Innerhalb des Baumes werden genau $|w|-1$ Regeln der
Form $A\to BC$ angewendet.
Trotzdem reicht es nicht, $|w|-1>|V|$ zu verlangen, weil die
zweimal verwendete Variable nicht auf dem gleichen Ast des
Baumes zu sein braucht. Für die Durchführung des Argumentes
brauchen wir aber, dass wir die beiden Vorkommnisse der Variablen
über Ableitungsregeln verbinden können.}.
Sei $A$ die unterste Variable im Ableitungsbaum, die wiederholt
wird (Abbildung~\ref{cfg-tree-1}).
Es erzeugt ein Wort $x$. Das nächsthöhere Vorkommen von $A$
erzeugt dagegen ein Wort, welches aus drei Teilen besteht:
$vxy$, die Länge dieses Wortes ist $\le N$. Für das ganze Wort fehlen
noch die Teile, die von Regeln ``weiter oben'' im Ableitungsbaum
erzeugt werden, also $w=uvxyz$.
Da die beiden Vorkommnisse von $A$ veschieden sind, muss mindestens
eines der Teilwörter $v$ und $y$ nicht leer sein, also $|vy|>0$
Indem man den Teil des Ableitungsbaumes
zwischen den Vorkommnissen von $A$ repliziert, kann man jetzt die
Wörter $uv^kxy^kz$ bilden, die alle auch mit der Grammatik $G$ 
abgeleitet werden können, also zu $L(G)$ gehören (Abbildung~\ref{cfg-tree-2}).
Damit haben wir folgenden Satz bewiesen:

\begin{satz}[Pumping Lemma für kontextfreie Sprachen]
\index{Pumping Lemma!f\ür kontextfreie Sprachen}%
\index{pumping length!f\ür kontextfreie Sprachen}%
Sei $L$ eine kontextfreie Sprache, dann gibt es ein Zahl $N$, die pumping
length, so dass jedes Wort $w\in L$ mit $|w|\ge N$ zerlegt werden
kann in fünf Teile $w=uvxyz$
\begin{compactenum}
\item
$|vy|>0$,
\item
$|vxy|\le N$ und
\item
$uv^kxy^kz\in L$ für alle $k\in\mathbb N$.
\end{compactenum}
\end{satz}

\subsection{Beispiele nicht kontextfreier Sprachen}
\subsubsection{Die Sprache $L=\{a^nb^nc^n\,|\,n\in\mathbb N\}$.}
\begin{figure}
\begin{center}
%\includegraphics[width=\hsize]{images/pl-5}\\%
\includegraphics{images/pl-5}\\%
\smallskip
%\includegraphics[width=\hsize]{images/pl-6}\\%
\includegraphics{images/pl-6}\\%
\smallskip
%\includegraphics[width=\hsize]{images/pl-7}\\%
\includegraphics{images/pl-7}\\%
\smallskip
%\includegraphics[width=\hsize]{images/pl-8}%
\includegraphics{images/pl-8}%
\end{center}
\caption{Pumping Lemma für kontextfreie Sprachen angewandt auf das 
Wort ${\tt a}^N{\tt b}^N{\tt c }^N\in L$, wobei $N$ die pumping length
von $L$ ist. Da $w$ lang genug ist, gibt es eine Zerlegung 
$w=uvxyz$ (2.~Zeile).
Abpumpen (3.~Zeile) und Aufpumpen (4.~Zeile) des
Wortes führt zu Wörtern, die nicht mehr in $L$ liegen.\label{pumpingcfgimage}}
\end{figure}

Die Sprache $L=\{a^nb^nc^n\;|\;n\in\mathbb N\}$ ist nicht kontextfrei.
Wir verwenden das Pumping Lemma für kontextfreie Sprachen.  Dazu
nehmen wir zunächst an, die Sprache $L$ sei kontextfrei. Dann
besagt das Pumping Lemma, dass es eine Zahl $N$ gibt, so dass
Wörter mit Länge mindestens $N$ besondere Eigenschaften haben.
Als solches langes Wort nehmen wir $w=a^Nb^Nc^N$. Nach dem
Pumping Lemma gibt es eine Zerlegung in fünf Teile
$w=uvxyz$, wobei der mittlere Teil nicht zu lang ist:
$|vxy|\le N$ (Abbildung~\ref{pumpingcfgimage}).
Insbesondere enthält $|vxy|$ höchstens zwei
Arten von Zeichen, denn es ist zu kurz, die $N$ $b$s in der Mitte
von $w$ zu überspannen. Beim Aufpumpen zu $uv^kxy^kz$ nimmt also
die Zahl dieser beiden Zeichen zu, nicht jedoch die Zahl des nicht
in $vxy$ enthaltenen Zeichens.
Damit kann $uv^kxy^kz$ nicht mehr in $L$ sein, obwohl das
Pumping Lemma dies behauptet. Aus diesem Widerspruch folgt,
dass $L$ nicht kontextfrei sein kann.

\rhead{Praktische Parser}
\input realworldparser.tex

%
% Turing Maschinen und rekursiv aufzaehlbare Sprachen
%
% (c) 2009
%
\def\blank{\text{\textvisiblespace}}
\lhead{Algorithmen}
\chapter{Turing Maschinen\label{chapter-turing}}
Die bisher entwickelten Modelle von Rechenmaschinen haben zwar
interessante Sprachklassen hervorgebracht, sind aber offensichlich nicht
geeignet, die M"oglichkeiten eines Computers abzubilden. Ein
Computer verf"ugt "uber einen Speicher, auf den er, im Gegensatz zum
Stack eines Stackautomen, beliebigen Zugriff hat. Das Ziel dieses
Abschnittes ist daher, einen endlichen Zustandsautomaten so zu
erweitern, dass er mit einem solchen Speicher arbeiten kann.
Weiterhin soll das Modell so formuliert sein, dass sich damit
Sprachen akzeptieren lassen, und wir hoffen nat"urlich, dass dadurch
eine nochmals gr"ossere Klasse von Sprachen entsteht, die diesmal
aber alles umfasst, was wir mit modernen Computern abbilden k"onnen.

\section{Turing Maschinen}
\rhead{Turingmaschinen}
\subsection{Zwei Stacks: Random Access}
Ein Stack hat die M"oglichkeiten eines endlichen Automaten dramatisch
erweitert, ein Stackautomat kann die bedeutend gr"ossere Klasse der
kontextfreien Sprachen analysieren. Wir haben aber auch schon gesehen,
dass es nicht kontextfreie Sprachen gibt. Mit welcher m"oglichst kleinen
Erweiterung k"onnen wir eine Sprache wie $\{a^nb^nc^n\,|\, n\ge 0\}$
analysieren?

Eine erste Idee k"onnte sein, einen zweiten Stack hinzuzuf"ugen.
Damit k"onnte man mit Sicherheit die Sprache $\{a^nb^nc^n\,|\,n\ge 0\}$
behandeln. Doch man f"ugt damit wesentlich mehr hinzu. Ein Buch ist
sicher ein Random-Access Speichermedium. Durch bl"attern im Buch
kann man jede beliebige Seite erreichen. Wenn man aber ein Buch
aufschl"agt, besteht es eigentlich nur aus zwei Stapeln von Bl"attern.
Bl"attern bedeutet nichts anderes, dass man ein Blatt des Buches vom
einen Stapel wegnimmt und auf den anderen Stapel legt. Zwei Stacks
bilden also bereits ein Random-Access Medium.

Antike Schriftrollen sind auch B"ucher, die aber nicht als Stapel
von Seiten organisiert sind, sondern als langes Band. Es ist aber klar,
dass man durch zerschneiden und binden jede Schriftrolle in ein
Buch verwandeln kann, und ein Buch durch zusammenkleben der einzelnen
Seiten zu einem Band in eine Schriftrolle.
Die heutigen B"ucher (Kodices) haben sich gegen"uber den Schriftrollen
etwa im ersten Jahrhundert n.~Chr. durchgesetzt.
F"ur unsere Anwendung ist die Vorstellung eines Random-Access Mediums
als endloses Band von Speicherzellen praktischer. Sie passt besser zu
unserem Bild vom Hauptspeicher eines Computers oder von Massenspeichermedien.
\subsection{Definition}
\index{Band}
\index{Bandalphabet}
\index{Schreib-/Lese-Kopf}
Wir wollen einem endlichen Automaten statt eines Stacks einen unendlich
grossen Speicher zur Verf"ugung stellen. Wir stellen uns diesen
Speicher als ein in beide Richtungen unendlich langes Band vor,
welches in einzelne Speicherzellen aufgeteilt ist, in die jeweils
genau ein Zeichen geschrieben werden kann (Abbildung~\ref{turingfig}).
Wie beim Stackautomaten gehen
wir davon aus, dass dieser Speicher auch Dinge speichern kann, die nicht
im Input vorkommen, wir verwenden daher ein Bandalphabet $\Gamma$, welches
das Input-Alphabet umfasst: $\Gamma\supset \Sigma$.

\begin{figure}
\begin{center}
\includegraphics[width=0.8\hsize]{images/turing-1}
\end{center}
\caption{Schematische Darstellung einer Turing-Maschine\label{turingfig}}
\end{figure}

Der Zugriff auf den Speicher erfolgt jeweils zellenweise, wir stellen
uns einen Schreib-/Lese-Kopf vor, der auf dem Band positioniert werden
kann, und genau ein Zeichen lesen kann, oder das auf dem Band vorhandene
Zeichen "uberschreiben kann. Um zu Beginn einen wohldefinierten Zustand
zu haben, brauchen wir ein zus"atzliches ``Blank''-Zeichen $\blank$,
welches in $\Gamma$ aber nicht in $\Sigma$ enthalten ist\footnote{W"are
das ``Blank''-Zeichen $\blank$ auch in $\Sigma$, k"onnte man das
leere Band nicht vom Input unterscheiden.}.

Da wir jetzt diesen unendlich grossen Speicher haben, gibt es keinen
Grund mehr, sich vorzustellen, dass der Input Zeichen f"ur Zeichen
in den endlichen Automaten gef"uttert wird. Stattdessen gehen wir
davon aus, dass der Input auf das Band geschrieben wird, der
Schreib-/Lesekopf auf das erste Feld des Input positioniert wird,
und man dann die Maschine arbeiten l"asst.

Die Maschine muss jetzt Zeichen vom Band lesen, und zusammen mit
ihrem aktuellen Zustand entscheiden, in welchem neuen Zustand
sie nach der Verarbeitung dieses Zeichens sein soll, welches Zeichen
an dieser Stelle auf dem Band stehen soll, und ob der Schreib-/Lesekopf
bewegt werden soll. Wir erlauben nur zwei Bewegungen: L und R, insbesondere
ist es nicht zul"assig, dass der Kopf nach der Verarbeitung stehen
bleibt.

Im Gegensatz zu DEA oder Stackautomat ist die Berechnung in diesem
Modell nach dem Ende des Input-Strings noch nicht zu Ende. Die Maschine
kann mehrmals "uber die auf dem Band gespeicherten Daten fahren und
sie immer wieder modifizieren. Sie k"onnte also im Prinzip auch
gar nie aufh"oren zu arbeiten. Daher brauchen wir Zust"ande, die
die Maschine anhalten. Wenn die Maschine angehalten hat, m"ussen wir
zudem wissen, ob das auf das Band geschriebene Input-Wort f"ur die
Sprache akzeptabel war oder nicht.

All dies f"uhrt uns auf folgende formale Definition einer Turing-Maschine
\begin{definition}
\index{Turing-Maschine}
Eine Turing-Maschine ist ein $7$-tupel
$M=(Q,\Sigma,\Gamma,\delta,q_0,q_{\text{accept}},q_{\text{reject}})$,
wobei $Q$, $\Sigma$ und $\Gamma$ endliche Mengen sind mit folgenden
zus"atzlichen Eigenschaften
\begin{compactenum}
\index{Zustand}
\item $Q$ heisst die Menge der Zust"ande
\index{Inputalphabet}
\index{Eingabe-Alphabet}
\index{blank@Blank}
\index{$\blank$|see{Blank}}
\item $\Sigma$ heisst das Inputalphabet, es enth"alt das spezielle
``Blank''-Zeichen $\blank$ nicht.
\index{Bandalphabet}
\item $\Gamma$ ist das Bandalphabet, es gilt $\blank\in\Gamma$ und
$\Sigma\subset\Gamma$.
\index{Uebergangsfunktion@\"Ubergangsfunktion}
\item $\delta\colon Q\times \Gamma\to Q\times\Gamma\times\{\text{L},\text{R}\}$
ist die "Ubergangsfunktion
\index{Startzustand}
\item $q_0\in Q$ ist der Startzustand
\index{Akzeptierzustand}
\item $q_{\text{accept}}\in Q$ ist der Akzeptierzustand
\index{Ablehnungszustand}
\item $q_{\text{reject}}\in Q$ ist der Ablehnungszustand
\end{compactenum}
\end{definition}
Die "Ubergangsfunktion liefert also zu jedem Zustand und zum
Inhalt des Feldes unter dem Schreib-/Lesekopf einen neuen
Zustand, einen neuen Bandinhalt und die Information, ob
der Kopf nach links (L) oder rechts (R) bewegt werden muss.

\subsection{Sprache}
Sei eine Turingmaschine
$M=(Q,\Sigma,\Gamma,\delta,q_0,q_{\text{accept}},q_{\text{reject}})$
gegeben.
Zu $M$ kann man auf folgende Weise eine Sprache konstruieren:
Um zu entscheiden, ob ein Wort $w\in\Sigma^*$
zu der Sprache geh"ort,
schreibt man es auf das Band, platziert den Schreib-/Lesekopf auf
das erste Zeichen und bringt die Maschine in den Zustand $q_0$.
\index{Konfiguration}
Man beschreibt diese Konfiguration der Maschine mit der Zeichenkette
$q_0w$, die Maschine ist im Zustand $q_0$ und der Schreib-/Lesekopf
steht "uber dem ersten Zeichen von $w$.

Mit Hilfe der "Ubergangsfunktion wird dann der n"achste Zustand,
der neue Inhalt des Feldes unter dem Schreib-/Lesekopf und die
Kopfbewegung ermittelt. Auch diesen neuen Zustand kann man wieder
als Konfiguration hinschreiben. Der Schreib-/Lesekopf steht
jetzt wom"oglich mitten im Wort drin, und die Maschine befindet
sich in irgend einem Zustand $q$: $w_1qw_2$.

Nehmen wir an, die Maschine befindet sich zu Beginn eines Zyklus im
Zustand $q_1$ und hat den Kopf auf das Zeichen $a_k$ positioniert.
Die Funktion $\delta$ berechnet daraus ein Tripel $(q_1,b_k,R)$
oder $(q_2,b_k,L)$. Die Konfiguration "andert sich dabei wie
folgt:
\begin{align*}
&\text{Ausgangskonfiguration:}&a_1\dots a_{k-1}&\;q_1\;a_k\dots a_n\\
&\text{"Ubergang mit } \delta(q_1,a_k)=(q_2,b_k,L)&a_1\dots a_{k-2}&\;q_2\;a_{k-1}b_k\dots a_n\\
&\text{"Ubergang mit } \delta(q_1,a_k)=(q_2,b_k,R)&a_1\dots a_{k-1}b_k&\;q_2\;a_{k+1}\dots a_n
\end{align*}

Die Maschine arbeit so immer weiter, bis einer der Zust"ande $q_{\text{accept}}$
oder $q_{\text{reject}}$ erreicht wird, dann h"alt sie an.
Das Inputwort gilt als akzeptiert, wenn der Zustand $q_{\text{accept}}$
erreicht wurde, es gilt als verworfen, wenn $q_{\text{reject}}$ erreicht
wurde.

\index{Berechnungsgeschichte}
Die Folge von Konfigurationen, die die Turing-Maschine
w"ahrend der Berechnung durchl"auft, nennt man auch
{\em Berechnungsgeschichte}.

\begin{definition}
\index{Sprache!von einer Turing-Maschine erkannte}
Ist $M$ eine Turingmaschine, dann heisst
\[
L(M)=\{w\in\Sigma^*\;|\;\text{$M$ akzeptiert $w$}\}.
\]
die von $M$ erkannte Sprache.
\end{definition}

\begin{definition}
\index{Turing-erkennbar}
Eine Sprache $L$ heisst Turing-erkennbar, wenn es eine Turing-Maschine
$M$ gibt mit $L=L(M)$.
\end{definition}

Die Definition der erkannten Sprache erlaubt, dass die zur Erkennung
verwendete Turing-Maschine auf einigen W"ortern $w\in\Sigma^*$ nicht
anh"alt. Diese W"orter werden nie akzeptiert, und geh"oren
damit nicht zur Sprache. Testet man sie mit $M$, wird die Berechnung
aber nie anhalten, man weiss also eigentlich nie, ob das Wort
nicht zur Sprache geh"ort, oder ob man einfach noch etwas mehr
Geduld braucht.

Gewissheit hat man erst, wenn man sicher sein kann, dass $M$ auf
jedem Input anh"alt.

\begin{definition}
\index{Entscheider}
Ein Entscheider ist eine Turingmaschine, die auf jedem Input $w\in\Sigma^*$
anh"alt. Eine Sprache heisst entscheidbar, wenn ein Entscheider sie
erkennt.
\index{entscheidbar}
\end{definition}

\subsection{Zustandsdiagramm}
\index{Graph!gerichteter!beschrifteter!einer Turing-Maschine}
Auch eine Turingmaschine kann man als gerichteten beschrifteten Graphen
darstellen. "Uberg"ange zwischen zwei Zust"anden sind immer begleitet von
einer "Anderung des Inhaltes des Feldes unter dem Schreib-/Lesekopf
und von einer Kopfbewegung. Beide m"ussen mit der Beschriftung
der Pfeile codiert werden. In Anlehnung an die bei den Stackoutomaten
verwendete Notation schreiben wir f"ur den "Ubergang
$\delta(q_1,a)=(q_2,b,\text{R})$
die Notation
\[
\entrymodifiers={++[o][F]}
\xymatrix{
{q_1}\ar[r]^{a\to b,\text{R}}
	&{q_2}
}
\]
und f"ur den "Ubergang
$\delta(q_1,a)=(q_2,b,\text{L})$
\[
\entrymodifiers={++[o][F]}
\xymatrix{
{q_1}\ar[r]^{a\to b,\text{L}}
	&{q_2}
}
\]

\subsection{Beispiel}
Wir konstruieren eine Turingmaschine, welche die bereits als weder regul"ar
noch kontextfrei erkannte Sprache $L=\{0^{2^n}\;|\; n\in\mathbb N\}$
erkennt.

Zweierpotenzen kann man daran erkennen, dass man die Zahl ohne
Rest durch zwei teilen kann, bis nur noch 1 "ubrig bleibt. Genau
diesen Prozess kann man mit einer Turingmaschine auf dem Band
nachbilden. Die Maschine muss also folgendes tun:

Auf dem Input $w$:
\begin{compactenum}
\item Fahre von links nach rechts "uber das Band und streiche jedes zweite $0$.
\item Falls im Schritt 1 das Band genau eine $0$ enthielt, {\it akzeptiere}.
\item Falls im Schritt 1 das Band eine ungerade Zahl und mehr als ein $0$
enthielt, {\it verwerfe}.
\item Fahre mit dem Kopf zur"uck zum linken Ende des Bandes
\item Weiter bei Schritt 1
\end{compactenum}
In jedem Durchgang wird die Anzahl der Nullen halbiert, es sei denn,
sie war nicht gerade. In diesem Fall bleibt nach dem Durchgang eine
ungerade Anzahl $0$ auf dem Band stehen, was beim darauffolgenden Durchgang
in Schritt~3 erkannt wird. Geht die Division immer auf, bleibt am Schluss
nur genau eine $0$ stehen, was im Schritt~2 erkannt wird. Daher funktioniert
dieser Algorithmus.

Um den Algorithmus mit einer Turing-Maschine zu realisieren, brauchen wir
ein neues Zeichen {\tt x}, mit dem wir gestrichene Zeichen markieren
k"onnen. Es ist also $\Gamma = \{0,{\tt x},\blank\}$.
Ausserdem brauchen wir
Zust"ande, mit denen man die Parit"at der Anzahl $0$ bestimmen kann, und
mit denen man die Aktivit"aten in Schritt 4 von den anderen unterscheiden
kann. Schritt $4$ wird zum Beispiel implementiert durch das
Zustandsdiagramm
\[
\entrymodifiers={++[o][F]}
\xymatrix{
{}\ar@(ul,ur)^{\genfrac{}{}{0pt}{1}{0\to 0,\text{L}}{{\tt x}\to{\tt x},\text{L}}}
\ar[r]^{\blank\to\blank,\text{R}}
	&
}
\]
Jede zweite Null streichen wird implementiert durch
\[
\entrymodifiers={++[o][F]}
\xymatrix{
*+\txt{}\ar[d]
\\
\ar@/^/[r]^{0\to{\tt x},\text{R}}
\ar@(ul,dl)_{\tt x\to\tt x,\text{R}}
	&\ar@/^/[l]^{0\to 0,\text{R}}
         \ar@(ur,dr)^{\tt x\to\tt x,\text{R}}
	 \ar[d]^{\blank\to\blank,?}
\\
*+\txt{}
	&{}
}
\]
Dass akzeptiert werden soll, wenn nach dem Ersetzen einer $0$ durch
ein {\tt x} keine weiteren $0$ mehr gefunden werden k"onnen, kann durch
\[
\entrymodifiers={++[o][F]}
\xymatrix{
\ar[r]^{0\to{\tt x},\text{R}}
	&{} \ar@(dl,dr)_{{\tt x}\to{\tt x},\text{R}}
	    \ar[r]^{\blank\to\blank,\text{R}}
		&*++[o][F=]{q_a}
}
\]
ausgedr"uckt werden. Die Schleife beim mittleren Zustand
bildet ab, dass {\tt x}-Zeichen "ubersprungen werden sollen.
Mit solchen Elementen k"onnen wir die Turingmaschine jetzt schrittweise
aufbauen.

Zun"achst stellen wir sicher, dass der Input tats"achlich aus Nullen
besteht:
\[
\entrymodifiers={++[o][F]}
\xymatrix{
*+\txt{}\ar[d]
\\
{q_1}\ar[d]_{\genfrac{}{}{0pt}{1}{\blank\to\blank,\text{R}}{{\tt x}\to{\tt x},\text{R}}}
\\
*++[o][F=]{q_r}
}
\]
Jetzt h"angen wir das Segment an, welches pr"uft, ob es noch genau ein $0$
auf dem Band hat:
\[
\entrymodifiers={++[o][F]}
\xymatrix{
*+\txt{}\ar[d]
\\
{q_1}\ar[d]_{\genfrac{}{}{0pt}{1}{\blank\to\blank,\text{R}}{{\tt x}\to{\tt x},\text{R}}}
	\ar[r]^{0\to{\tt x},\text{R}}
	&{q_2}\ar[d]^{\blank\to\blank,\text{R}}
              \ar@(u,ul)_{{\tt x}\to {\tt x},\text{R}}
\\
*++[o][F=]{q_r}
	&*++[o][F=]{q_a}
}
\]
Falls dies nicht zutrifft, m"ussen weitere Nullen weggestrichen werden,
wobei "uber die Parit"at der noch vorhandenen Nullen Buch gef"uhrt werden
muss. Daf"ur braucht es die beiden Zust"ande $q_3$ und $q_4$:
\[
\entrymodifiers={++[o][F]}
\xymatrix{
*+\txt{}\ar[d]
\\
{q_1}\ar[d]_{\genfrac{}{}{0pt}{1}{\blank\to\blank,\text{R}}{{\tt x}\to{\tt x},\text{R}}}
	\ar[r]^{0\to\blank,\text{R}}
	&{q_2}\ar[d]_{\blank\to\blank,\text{R}}
	      \ar[r]^{0\to{\tt x},\text{R}}
              \ar@(u,ul)_{{\tt x}\to {\tt x},\text{R}}
		&{q_3}\ar@(u,r)^{{\tt x}\to{\tt x},\text{R}}
		      \ar@/^/[d]^{0\to 0,\text{R}}
\\
*++[o][F=]{q_r}
	&*++[o][F=]{q_a}
		&{q_4}\ar@/^/[u]^{0\to {\tt x},\text{R}}
		      \ar@(d,r)_{{\tt x}\to{\tt x},\text{R}}
}
\]
Wenn die Maschine nach dem Abarbeiten aller 0 und {\tt x} auf einen
neuen Blank \textvisiblespace\ trifft, ist sie entweder im Zustand
$q_3$ oder in $q_4$. Im letzten Fall hat sie ein ungerade Anzahl
von 0 gefunden, gem"ass Schritt 3 des Algorithmus muss also verworfen
werden:
\[
\entrymodifiers={++[o][F]}
\xymatrix{
*+\txt{}\ar[d]
\\
{q_1}\ar[d]_{\genfrac{}{}{0pt}{1}{\blank\to\blank,\text{R}}{{\tt x}\to{\tt x},\text{R}}}
	\ar[r]^{0\to\blank,\text{R}}
	&{q_2}\ar[d]_{\blank\to\blank,\text{R}}
	      \ar[r]^{0\to{\tt x},\text{R}}
              \ar@(u,ul)_{{\tt x}\to {\tt x},\text{R}}
		&{q_3}\ar@(u,r)^{{\tt x}\to{\tt x},\text{R}}
		      \ar@/^/[d]^{0\to 0,\text{R}}
\\
*++[o][F=]{q_r}
	&*++[o][F=]{q_a}
		&{q_4}\ar@/^/[u]^{0\to {\tt x},\text{R}}
		      \ar@(d,r)_{{\tt x}\to{\tt x},\text{R}}
		      \ar@/^20pt/[ll]^{\blank\to\blank,\text{R}}
}
\]
Wenn sie dagegen im Zustand $q_3$ ist, muss sie nach links fahren,
bis zum ersten Blank, und dann im Zustand $q_2$ ankommen. Dies erreicht
man mit Hilfe eines weitere Zustands $q_5$.
\[
\entrymodifiers={++[o][F]}
\xymatrix{
*+\txt{}\ar[d]
	&*+\txt{}
		&{q_5}\ar@(ul,ur)^{\genfrac{}{}{0pt}{1}{0\to 0,\text{L}}{{\tt x}\to {\tt x},\text{L}}}
		      \ar[dl]_{\blank\to\blank,\text{R}}
\\
{q_1}\ar[d]_{\genfrac{}{}{0pt}{1}{\blank\to\blank,\text{R}}{{\tt x}\to{\tt x},\text{R}}}
	\ar[r]^{0\to\blank,\text{R}}
	&{q_2}\ar[d]_{\blank\to\blank,\text{R}}
	      \ar[rr]^{0\to{\tt x},\text{R}}
              \ar@(u,ul)_{{\tt x}\to {\tt x},\text{R}}
		&*+\txt{}
			&{q_3}\ar@(u,r)^{{\tt x}\to{\tt x},\text{R}}
			      \ar@/^/[d]^{0\to 0,\text{R}}
			      \ar[ul]_{\blank\to\blank,\text{L}}
\\
*++[o][F=]{q_r}
	&*++[o][F=]{q_a}
		&*+\txt{}
			&{q_4}\ar@/^/[u]^{0\to {\tt x},\text{R}}
			      \ar@(d,r)_{{\tt x}\to{\tt x},\text{R}}
			      \ar@/^20pt/[lll]^{\blank\to\blank,\text{R}}
}
\]
Mit diesem Beispiel haben wir gezeigt, dass die Menge der
Sprachen, die von einer Turingmaschine erkennbar sind, Sprachen
enth"alt, die mit den bisherigen Mitteln nicht erkennbar waren.

\begin{definition}
Die von einer Turingmaschine $M$ erkannten W"orter bilden die
Sprache $L(M)$.
\end{definition}

\section{Varianten von Turing Maschinen}
\rhead{Varianten von Turingmaschinen}
Turingmaschinen sollen als universelles Modell f"ur Computer gelten,
dies ist jedoch nur m"oglich, wenn sich die Menge der erkannten Sprachen
nicht "andert, wenn an dem Modell kleine Ver"anderungen vorgenommen
werden.

Das Band einer Turingmaschine k"onnen wir als den RAM-Speicher
eines modernen Computers interpretieren. Moderne Computer verwenden
jedoch verschiedene Wortl"ange bei der Arbeit mit ihrem Speicher,
was auch das Bandalphabet $\Gamma$ ver"andert. Ein Computer mit
Speicherwortl"ange $l$ hat als Bandalphabet die Menge $\Gamma_l=[2^l]$.

\index{von Neumann-Architektur}
\index{Harvard-Architektur}
Computer mit von Neumann-Architektur verwenden nur einen einzigen RAM-Speicher,
der Programm und Daten enth"alt. Computer mit Harvard-Architektur
verwenden dagegen zwei verschiedene Speicher: einen Programm-Speicher
und einen Datenspeicher. Die beiden Speicher k"onnen sogar verschiedene
Wortl"ange haben. AVR-Microcontroller verwenden zum Beispiel
den Flashspeicher, den sie als 16 Bit breiten Speicher adressieren,
w"ahrend sie das RAM als 8 Bit breiten Speicher adressieren.
ARM-Microcontroller dagegen betrachten RAM und Flash einfach als
verschiedene Bereich in einem einzigen grossen Adressraum.
AVR-Microcontroller haben also ``zwei B"ander'' mit unterschiedlichen
Bandalphabeten, ARM-Microcontroller verwenden dagegen ein einziges Band
mit immer dem gleichen Alphabet.

\subsection{Mehrspurige Turingmaschine}
\index{Turing-Maschine!mehrspurige}
\begin{satz}\label{mehrspurigeturingmaschine}
Jede Sprache, die von einer mehrspurigen Turingmaschine
erkannt werden kann, kann auch von einer einspurigen Turingmaschine
erkannten werden.
\end{satz}

\begin{proof}[Beweis]
Sei $L$ eine Sprache, die von der Turingmaschine $M$ mit $n$ Spuren
erkannten wird. Das Bandalphabet ist $\Gamma$. Wir konstruieren aus
$M$ eine neue Turingmaschine $M'$, welches nur noch eine Spur hat,
jedoch als Bandalphabet die Menge $\Gamma^n$. Dadurch wird die ``Wortbreite''
des Bandes erh"oht, es wird nur noch eine breite Spur verwendet, um die
gleiche Information unterzubringen.
\end{proof}

\subsection{Turingmaschine mit mehreren B"andern}
\index{Turing-Maschine!mit mehreren B\"andern}
\begin{figure}
\begin{center}
\includegraphics[width=0.9\hsize]{images/turing-2}
\end{center}
\caption{Turingmaschine mit mehreren B"andern\label{multitapetm}}
\end{figure}
\begin{satz}
\label{mehrbandturingmaschine}
Jede Sprache, die von einer Turingmaschine mit mehreren B"andern
(Abbildung~\ref{multitapetm})
erkannt
wird, kann auch von einer Turing-Maschine mit nur einem Band erkannt
werden.
\end{satz}

\begin{proof}[Beweis]
Die Konfiguration einer Turingmaschine $M$ mit $n$ B"andern beinhaltet
ausser dem Inhalt der zus"atzlichen B"ander auch noch die Position
des Schreib-/Lesekopfes jedes einzelnen Bandes. Um diese Information
zu codieren, konstruieren wir eine neue Turingmaschine $M'$ mit
$2n$ Spuren. Spur $2i$ enth"alt dabei die Daten von Band $i$.
Auf Spur $2i-1$ verwenden wir eine spezielle Marke $\uparrow$, um die Position
des Schreib-/Lesekopfes auf Band $i$ zu markieren. Somit l"asst sich
eine Turingmaschine mit $n$ B"andern auf einer Turingmaschine mit
$2n$ Spuren und Bandalphabet $\Gamma\cup\{\uparrow\}$ codieren.
Und nach Satz \ref{mehrspurigeturingmaschine} l"asst sich eine solche wiederum
auf einer Standardturingmaschine simulieren.
\end{proof}
Die geraden Spuren werden also dazu verwendet, die Position des
Schreib-/Lese-Kopfes zu speichern, w"ahrend die ungeraden Spuren
die Daten enthalten:
\begin{center}
\begin{tabular}{c|c|c|c|c|c|c|c|c}
\hline
&a&b&c&d&e&f&c&\\
\hline
& & &$\uparrow$& & & & &\\
\hline
&1&2&3&4&5&6&7&\\
\hline
& & & & & &$\uparrow$& &\\
\hline
\end{tabular}
\end{center}

\subsection{Bandalphabet}
Auch das Bandalphabet hat keinen Einfluss auf die M"oglichkeiten einer
Turingmaschine. Die Festplatte eines modernen Computers ist eigentlich
ein ``Band'' mit Bandalphabet $\{0,1,\blank\}$, erst zus"atzliche
Logik im Controller macht daraus einzelne Bytes, also ein gr"ossers
Bandalphabet.

\begin{satz}
Jede Turingmaschine mit Bandalphabet $\Gamma$ kann auf einer Turingmaschine
mit Bandalphabet $\{0,1,\blank\}$ simuliert werden.
\end{satz}

\begin{proof}[Beweis]
Sei also $M$ eine Turingmaschine mit Bandalphabet $\Gamma$ gegeben. Wir
wollen daraus eine Turingmaschine $M'$ mit dem Alphabet
$\Gamma_0= \{0,1,\blank\}$ konstruieren, die dieselbe
Sprache erkennt. Dazu m"ussen wir die Zeichen aus $\Gamma$ als
Bitfolgen codieren, eine beliebige injektive Abbildung
\[
e\colon\Gamma\to \Gamma_0^l
\]
l"ost dieses Problem, eine solche existiert, wenn $2^l>|\Gamma|$.
Zus"atzlich k"onnen
wir verlangen, dass $e(x)\in \{0,1\}^l$, die Zeichen $\ne\blank$
also ausschliesslich mit $0$ und $1$ codiert werden, und
$e(\blank)=(\blank,\dots,\blank)$.

Mit der Codierung $e$ wird der Bandinhalt jetzt umcodiert. Der Inhalt
$x$
eines Feldes des Bandes von $M$ wird auf $l$ aufeinanderfolgende Felder
des Bandes von $M'$ verteilt, die mit den einzelnen Komponenten
von $e(x)$ gef"ullt werden.

In den folgenden Darstellungen verwenden wir der gr"osseren
"Ubersichtlichkeit halber $\Gamma=\{0,1,2,3\}$, wobei eines der
Zeichen die Rolle von $\blank$ "ubernimmt. Ausserdem ist $l=2$.

Jetzt m"ussen auch die "Uberg"ange in $M$
durch entsprechend erweiterte Konstruktionen in $M'$ ersetzt werden.
In einem Zyklus m"ussen zun"achst die n"achsten $l$ Zeichen vom Band
von $M'$ gelesen werden. Welches Zeichen dieser Bitfolge entspricht,
muss in Zust"anden von $M'$ festgehalten werden. Ausgehend vom Zustand
$q$ von $M$ kann zur Decodierung
des Bandes folgender Automat verwendet werden, der $l$ Bits liest
und den Kopf wieder auf den Anfang des Codewortes zur"uckf"ahrt:
\[
\entrymodifiers={++[o][F]}
\xymatrix{
*+\txt{}
	&*+\txt{}
		&*+\txt{}
			&q\ar[dll]_{0\to 0,\text{R}} \ar[drr]^{1\to 1,\text{R}}
\\
*+\txt{}
	&\ar[dl]_{0\to 0,\text{R}} \ar[dr]^{1\to 1,\text{R}}
		&*+\txt{}
			&*+\txt{}
				&*+\txt{}
					&\ar[dl]_{0\to 0,\text{R}} \ar[dr]^{1\to 1,\text{R}}
						&*+\txt{}
\\
\ar[d]^{.\to .,\text{L}}
	&*+\txt{}
		&\ar[d]^{.\to .,\text{L}}
			&*+\txt{}
				&\ar[d]^{.\to .,\text{L}}
					&*+\txt{}
						&\ar[d]^{.\to .,\text{L}}
\\
\ar[d]^{.\to .,\text{L}}
	&*+\txt{}
		&\ar[d]^{.\to .,\text{L}}
			&*+\txt{}
				&\ar[d]^{.\to .,\text{L}}
					&*+\txt{}
						&\ar[d]^{.\to .,\text{L}}
\\
q_0
	&*+\txt{}
		&q_1
			&*+\txt{}
				&q_2
					&*+\txt{}
						&q_3
}
\]
Die Zust"ande $q_i$ codieren jetzt zus"atzlich, welches
Zeichen $i\in\Gamma$ sich unter dem Schreib-/Lesekopf befindet.

Ein "Ubergang
\[
\entrymodifiers={++[o][F]}
\xymatrix{
q\ar[r]^{a\to b,\text{R}}
	&p
}
\]
muss jetzt "ubersetzt werden in einen "Ubergang von $q_a$ in einen
Zwischenzustand, von dem aus das Codewort von $b$ geschrieben.
wird. Wird $b$ als $b_1\dots b_l$ codiert, wird daraus also
\[
\entrymodifiers={++[o][F]}
\xymatrix{
q_a\ar[r]^{.\to b_1,\text{R}}
	&\ar[r]^{.\to b_2,\text{R}}
		&p
}
\]
Falls dem "Ubergang jedoch eine Kopfbewegung nach links folgt, wie in
\[
\entrymodifiers={++[o][F]}
\xymatrix{
q\ar[r]^{a\to b,\text{L}}
	&p
}
\]
dann muss diese ebenfalls noch angeh"angt werden:
\[
\entrymodifiers={++[o][F]}
\xymatrix{
q_a\ar[r]^{.\to b_1,\text{R}}
	&\ar[r]^{.\to b_2,\text{R}}
		&\ar[r]^{.\to .,\text{L}}
			&\ar[r]^{.\to .,\text{L}}
				&\ar[r]^{.\to .,\text{L}}
					&\ar[r]^{.\to .,\text{L}}
						&p
}
\]
Diese Konstruktion zeigt, dass sich die Turingmaschine $M$ auf einer
Turingmaschine $M'$ mit Bandalphabet $\{0,1,\blank\}$ simulieren
l"asst.
\end{proof}

\subsection{Aufz"ahler}
\begin{figure}
\begin{center}
\includegraphics[width=\hsize]{images/turing-3}
\end{center}
\caption{Schematische Darstellung eines Aufz"ahlers.\label{turing-aufzaehler}}
\end{figure}
Unser bisheriges Konzept einer Turingmaschine hat keine M"oglichkeit,
Output zu produzieren. Solange die Turingmaschine arbeitet, wissen wir
nicht, welcher Teil des Bandinhaltes m"oglicherweise schon ``fertig''
berechnet ist. Erst wenn sie angehalten hat, weil sie $q_{\text{accept}}$
oder $q_{\text{reject}}$ erreicht hat, ist die Berechnung fertig.

F"ur die Praxis w"unschen wir uns jedoch auch eine M"oglichkeit,
die Maschine ohne Ende weiterlaufen zu lassen, wobei sie immer neue
Resultate produziert. Zum Beispiel k"onnte so eine Maschine der Reihe
nach alle W"orter einer Sprache aufz"ahlen wollen. Zu diesem Zweck
f"ugen wir der Turingmaschine einen Drucker hinzu
(Abbildung~\ref{turing-aufzaehler}). Die Turingmaschine
kann jederzeit ein Wort auf den Drucker schreiben und dann
weiterarbeiten.

\begin{definition}
\index{Aufzahler@Aufz\"ahler}
Eine Turingmaschine mit einem Drucker, die auf einem leeren Band beginnt,
heisst ein Aufz"ahler. Die von einem Aufz"ahler auf dem Drucker ausgegebene
W"orter bilden eine Sprache, die vom Aufz"ahler aufgez"ahlte Sprache.
\end{definition}

Wieder haben wir das Berechnungsmodell erweitert, und es stellt sich
die Frage, ob sich dadurch die Menge der Sprachen erweitert.

\begin{satz}
\index{Turing-erkennbar}
Eine Sprache ist genau dann Turing-erkennbar, wenn sie
von einem Aufz"ahler aufgez"ahlt wird.\end{satz}

\begin{proof}[Beweis]
Da die "Aquivalenz der beiden Modelle zu zeigen ist, sind zwei Implikationen
zu beweisen. Einerseits muss gezeigt werden, dass die von einem
Aufz"ahler aufgez"ahlte Sprache auch von einer Turingmaschine
erkannt werden kann, andererseits muss zu einer Turingmaschine, die
die Sprache erkennt, ein Aufz"ahler konstruiert werden, der die Sprache
aufz"ahlt.

Sei als $A$ ein Aufz"ahler. Wir konstruieren eine Turingmaschine $M$, die
zum Inputwort $w$ folgenden Algorithmus implementiert.
\begin{compactenum}
\item Lasse $A$ laufen. Jedesmal, wenn $A$ ein Wort auf den Drucker schreibt,
vergleiche das Wort mit $w$.
\item Falls das Wort mit $w$ "ubereinstimmt, gehe in den Zustand
$q_{\text{accept}}$.
\end{compactenum}
Falls das Wort zur Sprache geh"ort, wird der Aufz"ahler es fr"uher
oder sp"ater aufz"ahlen, und der Algorithmus wird es akzeptieren.

F"ur die umgekehrte Richtung m"ussen wir zur Turingmaschine $M$
einen Aufz"ahler produzieren.
Ein erster Versuch besteht darin, der Reihe nach alle W"orter aus
$\Sigma^*$ zu produzieren, und jedes mit $M$ zu testen.
Ein Algorithmus, der die W"orter produziert, ist einfach herzustellen,
man produziert zuerst alle W"orter der L"ange 1, dann alle der L"ange 2,
immer lexikographisch geordnet.
Sei also $s_1,s_2,s_3,\dots$ eine Liste aller W"orter von $\Sigma^*$.

Beim Testen der W"orter mit $M$
haben wir die Schwierigkeit, dass $M$ auf einem Inputwort
m"oglicherweise nicht anh"alt. Wir m"ussen also mit einem Trick
simulieren, dass unsere Algorithmus nicht in einer ``Endlosschleife''
in $M$ stecken bleibt. Wir lassen $M$ daher nur jeweils f"ur einige
Schritte laufen.
Genauer:
\begin{compactenum}
\item F"ur $n=1,2,3,\dots$ f"uhre die folgenden zwei Schritte aus:
\item Lasse $M$ auf jedem Wort $s_i, i \le n$ w"ahrend $n$ Schritten
laufen.
\item Falls $M$ das Wort $s_i$ akzeptiert, schreibe es auf den Drucker.
\end{compactenum}
Dieser Algorithmus verhindert, dass $M$ in eine ``Endlosschleife''
ger"at, und druckt alle W"orter aus, von der Turingmaschine erkannt
werden, sogar unendlich oft.
\end{proof}

\subsection{Nicht deterministische Turingmaschinen}
\index{Turing-Maschine!nicht deterministische}
Sowohl bei endlichen Automaten wie auch bei Stackautomaten war
Nichtdeterminismus ein Konzept, welches die Formulierung eines
Automaten wesentlich vereinfachen konnte, ohne die M"oglichkeiten
zu ver"andern. Die einzige "Anderung in der Definition ist die
Definition von $\delta$, welche nicht mehr Werte in
$Q\times \Gamma\times\{\text{L},\text{R}\}$ annimmt, sondern
in der Potenzmenge:
\[
\delta\colon Q\times\Gamma\to
P(Q\times \Gamma\times\{\text{L},\text{R}\}).
\]
Die Berechnung muss in jedem Schritt eine der M"oglichkeiten
aus $\delta(q,a)$ ausw"ahlen.
Eine deterministische Turingmaschine ist offenbar auch eine
nichtdeterministische Turingmaschine, die den nichtdeterminismus
gar nicht ausn"utzt, also $|\delta(q,a)|=1$.

\begin{satz}
\label{nichtdeterministischeturingmaschine}
Jede nichtdeterministische Turingmaschine ist "aquivalent zu einer
deterministischen Turingmaschine.
\end{satz}

\begin{proof}[Beweis]
Die Behauptung ist bewiesen, wenn wir zu einer nicht deterministischen
Turingmaschine $M$ eine deterministische Turingmaschine konstruiert haben,
die die gleichen W"orter erkennt.

Die nicht deterministische Maschine kann viele verschiedene Berechnungswege
verwenden, um ein Wort zu akzeptieren.
Es ist aber auch m"oglich, dass ein solcher Weg in eine Endlosschleife f"uhrt.
Daher k"onnen wir nicht einfach die nichtdeterministische Turingmaschine
auf einem Berechnungsweg laufen lassen, wir w"urden nie mit einem
anderen Weg beginnen, der m"oglicherweise das Wort akzeptiert.
Wir d"urfen also die Maschine immer nur einige Schritte laufen lassen,
und m"ussen dann die anderen Wege durchprobieren.

Wir konstruieren jetzt eine Turingmaschine mit drei B"andern, die die
nichtdeterministische Turingmaschine simuliert. Das erste Band
enth"alt das Input-Wort $w$.
Das zweite Band dient als Arbeitsband f"ur die Maschine $M$. Das
dritte Band hat Bandalphabet $S=Q\times \Gamma\times\{\text{L},\text{M}\}$
und speichert die nicht deterministischen Auswahlen aus $\delta(q,a)$,
die  in jedem Schritt n"otig sein k"onnen. Eine Folge von Auswahlen
ist ein String aus $S^*$. Wir wissen bereits, dass die Strings aus $S^*$
aufgez"ahlt werden k"onnen.

Auf dieser Maschine f"uhren
wir jetzt folgenden Algorithmus aus:

\begin{compactenum}
\item f"ur $n=1,2,3,\dots$ f"uhre die Schritte 2 bis 5 aus:
\item Schreibe den String $s_i$  auf Band 3,
und f"uhre damit die Schritte 3 bis 5 aus.
\item Kopiere $w$ von Band 1 auf Band 2
\item Lasse $M$ f"ur $n$ Schritte auf Band 2 laufen, verwende f"ur die
nichtdeterministischen Entscheidungen in jedem Schritt das entsprechende
Feld auf Band 3.
\item Falls $M$ akzeptiert, akzeptiere.
\end{compactenum}
Dieser Algorithmus probiert nacheinander alle m"oglichen Berechnungswege
durch, l"asst $M$ aber immer nur eine beschr"ankte Anzahl Schritte lang
rechnen. Falls $w$ akzeptiert werden kann, wir der dazu n"otige Berechnungsweg
irgendwann auf Band 3 auftauchen, und in entsprechend vielen Schritten wird
das Wort akzeptiert werden.
\end{proof}

\begin{satz}
Eine Sprache ist Turing-erkennbar, wenn sie von einer nicht deterministischen
Turingmaschine erkannt werden kann.
\end{satz}

Auch eine nicht deterministische Turingmaschine kann ein Entscheider sein.
Da ein Entscheider aber niemals in eine ``Endlosschleife'' geraten darf,
m"ussen wir verlangen, dass alle m"oglichen Berechnungsabl"aufe
irgenwann terminieren.

\begin{definition}
\index{Entscheider}
\index{Berechnungsgeschichte}
Eine nicht deterministische Turingmaschine ist ein Entscheider, wenn
jede m"ogliche Berechnungsgeschichte terminiert.
\end{definition}

\begin{satz}
Eine Sprache ist entscheidbar, wenn sie von einer nicht deterministischen
Turingmaschine entschieden wird.
\end{satz}

Nichtdeterminismus "andert also die erkennbaren oder entscheidbaren Sprachen
nicht. Doch der Algorithmus l"asst bereits vermuten, dass die Simulation
einer nichtdeterministischen Turingmaschine auf einer deterministischen
Maschine sehr zeitaufwendig ist. Wenn man also eine Performance-Vorgabe
macht, kann die Menge der von einer deterministischen Turingmaschine
erkennbaren Sprachen deutlich kleiner sein. Dies wird uns im Kapitel
\ref{chapter-komplexitaet} besch"aftigen.

\section{Berechenbarkeit}
\rhead{Berechenbarkeit}
In diesem Abschnitt wollen wir kl"aren, welche Dinge "uberhaupt
berechnet werden k"onnen. Dabei sehen wir die Tatsache, dass f"ur
die Darstellung einer reellen Zahl eine unendliche Folge von Ziffern
n"otig sein kann, nicht als Hindernis an. Wir nennen eine Zahl
berechenbar, wenn wir ein Programm schreiben k"onnen, welches
m"oglicherweise unendlich lange l"auft und dabei eine Stelle der
Zahl nach der anderen liefern kann. Die Turingmaschine
funktioniert also als Aufz"ahler, der die Stellen der zu berechnenden
Zahl aufz"ahlen soll.  In diesem Sinne ist die
Zahl $1/3=0.3333\dots$ berechenbar, aber auch die Zahlen $\sqrt{2}$, $\pi$,
$e$ und weitere. Wir werden allerdings sehen, dass die meisten
Zahlen gar nicht berechenbar sind.

\subsection{Abz"ahlbar und "uberabz"ahlbar}
Unendlich ist nicht gleich unendlich, die Mengen $\mathbb N$ und
$\mathbb R$ haben zwar beide unendlich viele Elemente, dennoch
gibt es keine bijektive Abbildung $\mathbb N\to \mathbb R$.

\begin{definition}
\index{Machtigkeit@M\"achtigkeit}
Zwei Mengen $A$ und $B$ heissen gleich m"achtig, wenn es eine bijektive
Abbildung $A\to B$ gibt.
\end{definition}

Es ist bekannt, dass $\mathbb Q$ und $\mathbb N$ gleich m"achtig sind,
aber die Menge $\mathbb R$ scheint wesentlich gr"osser
zu sein als $\mathbb N$.
Diese Unterschiede zwischen verschiedenen
unendlichen Mengen wird illustriert von der nachfolgenden Geschichte,
die David Hilbert als Illustration zu erz"ahlen pflegte.

\subsubsection{Hotel ``Unendlich''}
\index{Hotel ``Unendlich''}
\index{Hilbert!-Hotel}
In einem fernen Land gibt es ein Hotel, welches Hotel ``Unendlich''
genannt wird. Unsere Geschichte beginnt kurz nachdem der Nachtportier
die Nachtschicht "ubernommen hat. Ein versp"ateter Gast meldet sich
am Eingang, w"ahrend der Nachtportier damit besch"aftigt war,
die herumlungernden Besoffenen
aus den naheliegenden Pubs vertreibt, die auch gerade geschlossen haben.
Der Gast m"ochte im Hotel "ubernachten, doch der Nachtportier zeigt sich
unnachgiebig: ``Du kommst hier nicht rein!''

Zum Gl"uck hat der Hotelmanager den Wortwechsel mitbekommen und greift
ein. Selbstverst"andlich haben wir noch ein Zimmer f"ur jeden noch so
sp"aten Gast. Der Nachportier protestiert, es seien doch alle Zimmer
belegt. ``Kein Problem'' sagt der Manager, und greift zum Mikrofon
der Sprechanlage. Er bittet alle G"aste, aus ihrem aktuellen Zimmer
ins n"achste Zimmer umzuziehen. Dieses Ansinnen zu sp"ater Stunden
verursacht zwar bei einigen G"asten etwas Unmut, aber jeder ist damit
einverstanden, denn h"atten sie sich selbst versp"atet, w"aren sie ja
auch auf das Entgegenkommen der anderen G"aste angewiesen. Auf diese
Weise wird das Zimmer mit der Nummer $0$ frei f"ur den neu angekommenen
Gast.

Kurze Zeit sp"ater kommt eine versp"atete Reisegruppe aus dreissig
Touristen an, die ebenfalls
Einlass verlangt. Der Nachtportier will nicht nochmals einen schlechten
Eindruck hinterlassen und ruft den Manager. Dieser greift
wieder zum Mikrofon, und bittet wieder alle G"aste, diesmal aus
ihrem Zimmer mit der Nummer $n$ ins Zimmer mit der Nummer $n+30$
umzuziehen. So werden die Zimmer mit den Nummern $0$ bis $29$ frei,
Platz genug f"ur die neu angekommene Reisegruppe.

Diese Nacht ist ziemlich viel los im Hotel ``Unendlich'', denn
kaum waren die dreissig G"aste untergebracht, f"ahrt ein Hilbert-Bus
vor: ein Bus mit unendlich vielen Sitzpl"atzen, numeriert mit den
nat"urlichen Zahlen. Der Nachtportier ist schockiert: Unendlich
viele G"aste, ohne Reservation. Dass man f"ur endlich viele G"aste
immer noch Platz schaffen kann, hat er inzwischen verstanden, aber
f"ur unendlich viele G"aste sei das unm"oglich, stellt er fest und
will die Gesellschaft wegschicken. Doch der Hotelmanager m"ochte
sich das Gesch"aft nicht entgehen lassen. Er greift erneut zum
Mikrofon und bitte die G"aste, vom Zimmer $n$ ins Zimmer $2n$ umzuziehen.
So werden alle Zimmer mit ungeraden Nummer frei, Platz genug, f"ur alle
Passagiere des Hilbert-Buses. Nur die Stimmung unter den Stammg"asten
hat sich bereits deutlich verschlechtert.

Schliesslich fahren gleich dreissig Hilbertbusse vor. Ein Hilbert-Jumbo
hatte sich versp"atet, und so dass die Ferieng"aste erst sehr sp"at
mit ihren Hilbertbussen vom Flughafen zum Hotel abfahren konnten.
Der Nachtportier hat schon einen Verdacht, dass der Manager auch
hierf"ur eine L"osung bereit hat. Tats"achlich: per Mikrofondurchsage
gibt er wieder alle ungerade Zimmer frei. Dann bringt er die Neuank"ommlinge
darin unter, und zwar sch"on nacheinander immer je einen aus jedem Bus,
also in die ersten dreissig freien Zimmer die jeweils ersten G"aste
aus jedem Bus, dann die zweiten aus jedem Bus in die zweiten dreissig
freien Zimmer und so weiter.

Man ahnt es schon, in dieser Nacht gibt es f"ur die G"aste im Hotel
``Unendlich'' keine Ruhe. Als n"achstes findet sich ein Hilbert-Konvoi
vor dem Hotel ein. Also ein Folge von Hilbert-Bussen, die mit nat"urlichen
Zahlen numeriert waren. Der Nachtportier hat gar nicht erst gewagt,
den Neuank"ommlingen abschl"agigen Bescheid zu geben, sondern gleich
den Manager gerufen. Der muss zwar auch einen Moment nachdenken, findet
dann aber eine L"osung. Zun"achst wendet er nochmals den Trick an,
mit dem er jetzt schon mehrmals die ungeraden Zimmer frei bekommen hat.
Dabei kommt es auch zu w"usten Szenen mit G"asten, die jetzt schon vier mal
geweckt worden sind, und sich ihre Nachtruhe nicht von der Geldgier des
Managers ruinieren lassen wollen.

Die G"aste aus dem Hilbert-Konvoi m"ussen sich in Einerkolonnen auf
dem Vorplatz aufstellen, der Manager rief sie daraufhin in der
eingezeichneten Reihenfolge ab. Auch dies tr"agt nicht  unbedingt
zur Zufriedenheit der G"aste bei, hinter vorgehaltener Hand wird
"uber den ``Kasernenton'' und ``Zust"ande wie in einem Konzentrationslager''
geschimpft.
Aber jeder kommt fr"uher oder sp"ater dran,
am Ende hat jeder Gast ein Zimmer, was die Gem"uter wieder
etwas bes"anftigt.

\[
\xymatrix{
\ar[r]
\cdot	     &\cdot\ar[dl] &\cdot\ar[r] &\cdot\ar[dl]  &\cdot\ar[r] &\cdot\ar[dl] &\cdot
\\
\cdot\ar[d]  &\cdot\ar[ur] &\cdot\ar[dl] &\cdot\ar[ur] &\cdot\ar[dl]&\cdot        &\cdot
\\
\cdot\ar[ur] &\cdot\ar[dl] &\cdot\ar[ur] &\cdot\ar[dl] &\cdot       &\cdot        &\cdot
\\
\cdot\ar[d]  &\cdot\ar[ur] &\cdot\ar[dl] &\cdot        &\cdot       &\cdot        &\cdot
\\
\cdot\ar[ur] &\cdot        &\cdot        &\cdot        &\cdot       &\cdot        &\cdot
}
\]

Ein Ereignis steht allerdings noch bevor, und hier sollte sich die
Geldgier des Managers r"achen. Gegen morgen n"amlich f"ahrt
ein voller Cantor-Bus vor. Dieses hochmoderne Verkehrsmittel hat Sitze,
\index{Cantor!-Bus}
die mit den reellen Zahlen angeschrieben sind. Nat"urlich wollen auch
diese G"aste im Hotel ``Unendlich'' untergebracht werden. Aber so sehr
sich der Manager auch anstrengt, in seinem Hotel kann er diese
G"aste nicht unterbringen. Daraufhin wird er zwar vom Verwaltungsrat
entlassen, der noch weniger Mathematik versteht, was ihm dank "uppiger
Abgangsentsch"adigung jedoch nicht weiter Sorgen macht. Unbest"atigten
Ger"uchten zufolge soll er jetzt bei einer Hotelkette ``Cantor Hotels''
arbeiten, die mit mit dem Spruch ``Present in uncountable locations
throughout the Universe'' f"ur sich wirbt.

\subsubsection{Lehren aus der Geschichte}
Eine unendliche Menge ist offenbar so gross, dass man darin immer noch
Platz genug f"ur eine Kopie der ganzen Menge finden kann. Oder anders
herum: endliche Mengen sind solche, in denen man niemals Platz finden
k"onnte:

\begin{satz}
Eine Menge ist endlich, wenn jede injektive Abbildung auch surjektiv ist.
\end{satz}

In der Geschichte kam eine ganze Reihe von unendlichen Mengen auf, die
alle in $\mathbb N$ untergebracht werden konnten, die also nicht
gr"osser waren als $\mathbb N$ selbst, und das obwohl sie aus nat"urlichen
Zahlen konstruiert worden waren. Mit $\mathbb N$ vergleichbare Mengen
bilden also eine robuste Klasse von Mengen.

\begin{definition}
\index{abzahlbar unendlich@abz\"ahlbar unendlich}
\index{uberabzahlbar unendlich@\"uberabz\"ahlbar unendlich}
Eine unendliche Menge $A$ heisst abz"ahlbar unendlich, wenn sie
gleichm"achtig ist wie die nat"urlichen Zahlen. $A$ heisst
"uberabz"ahlbar unendlich, wenn es keine Bijektion zwischen
$\mathbb N$ und $A$ gibt.
\end{definition}
Bildlich gesprochen ist eine abz"ahlbare Menge eine solche,
bei der sich die Elemente in eine mit den nat"urlichen Zahlen numerierte
Liste einordnen lassen.

Auch die Menge der Paare $(k,l)\in \mathbb N^2$ ist abz"ahlbar.
Ein Gast aus dem Hilbert-Konvoi wird durch seine Busnummer $k$ und
seine Platz-Nummer $l$ identifiziert, der Konvoi hat also gleich
viele G"aste wie $\mathbb N^2$ Paare enth"alt. Und alle diese Paare
passen in die Menge $\mathbb N$, also die Zimmer des Hotels ``Unendlich''
hinein. Somit sind $\mathbb N^2$ und $\mathbb N$ gleich m"achtig.

Die Geschichte hat exemplarisch den folgenden Satz gezeigt:

\begin{satz}Die Vereinigung von endlich vielen abz"ahlbaren
Mengen ist abz"ahlbar. Das kartesische Produkt zweier abz"ahlbarer
Mengen ist abz"ahlbar.
\end{satz}

\begin{proof}[Beweis]
Sei $A_1,\dots,A_n$ eine endliche Familie abz"ahlbarer Mengen,
je mit einer Funktion $f_i\colon \mathbb N\to A_i$, dann
k"onnen wir eine Abz"ahlfunktion f"ur die Vereinigung konstruieren:
\[
f\colon \mathbb N\to\bigcup_{i=1}^n A_i\colon nk+l \mapsto f_l(k)
\]
wobei wir verwenden, dass wir jede Zahl eindeutig als $nk+l$
schreiben k"onnen mit geeigneten Zahlen $k$ und $l$.

F"ur das kartesische Produkt m"ussen wir zeigen, dass sich die
Menge der Paare $\mathbb N^2$ aufz"ahlen l"asst, was mit dem
Diagonaltrick des Hotelmanagers geschehen kann.
\end{proof}

Daraus folgen jetzt weitere S"atze "uber die Kardinalit"at
der bekannten Zahlmengen.

\begin{satz}
Die Mengen $\mathbb Z$ und $\mathbb Q$ sind abz"ahlbar unendlich.
\end{satz}

\begin{proof}[Beweis]
Die Menge $\mathbb Z$ ist abz"ahlbar, da sie die Vereinigung
von zwei Mengen ist, die offensichtlich abz"ahlbar sind:
\[
\mathbb Z = \mathbb N\cup \{-n\;|\;n\in\mathbb N\}.
\]
Die rationalen Zahlen k"onnen durch Br"uche dargestellt werden,
also durch Paare $(p,q)$ von ganzen Zahlen:
\[
\mathbb Q=\left\{\left.\frac{p}{q}\;\right|\;p\in \mathbb Z,q\in\mathbb N\setminus\{0\}\right\}
\]
Da sich einige Br"uche noch k"urzen lassen, ist $\mathbb Q$ noch
kleiner als $\mathbb N^2$, aber insbesondere abz"ahlbar.
\end{proof}

\begin{satz}
Die Menge $\mathbb R$ ist "uberabz"ahlbar unendlich.
\end{satz}

\begin{proof}[Beweis]
Wir f"uhren den Beweis mit Hilfe eines Widerspruchs. Nehmen wir
an, $\mathbb R$ w"are abz"ahlbar unendlich. Dann m"ussten auch
die reellen Zahlen zwischen $0$ und $1$ abz"ahlbar sein,  es g"abe also eine
Liste all dieser Zahlen, wir schreiben die Zahlen
dieser Liste in Dezimaldarstellung
\begin{align*}
0.&r_{11}r_{12}r_{13}r_{14}\dots\\
0.&r_{21}r_{22}r_{23}r_{24}\dots\\
0.&r_{31}r_{32}r_{33}r_{34}\dots\\
0.&r_{41}r_{42}r_{43}r_{44}\dots\\
&\vdots
\end{align*}
Jetzt konstruieren wir eine Zahl $x$, die nicht in dieser Liste vorkommen
kann. Um die $k$-te Stelle $x_k$ von $x$ zu konstruieren, betrachten
wir die $k$-te Stelle der $k$-ten Zahl $r_{kk}$, und setzen
\[
x_k=\begin{cases}
r_{kk}-1&\qquad r_{kk}>0\\
5&\qquad r_{kk}=0
\end{cases}
\]
$x_k$ ist also verschieden von $r_{kk}$, und zwar f"ur jedes $k$.
Die Ziffer $9$ kommt in $x$ nicht vor, es kann also keine Zahl
mit lauter $9$ am Ende sein, die gleichbedeutend mit einer anderen
Zahl ist, die m"oglicherweise bereits in der Liste ist.
Also unterscheidet sich $x$ von jeder der Zahlen aus der Liste, $x$
kommt also in der Liste nicht vor. Wir hatten angenommen, dass die
Liste alle reellen Zahlen zwischen $0$ und $1$ umfasst, dieser
Widerspruch zeigt, dass es keine solche Liste geben kann, die
reellen Zahlen sind also "uberabz"ahlbar.
\end{proof}

\index{Cantor!Georg}
Georg Cantor (1845-1918) leistete wichtige Beitr"age zur Mengelehre,
unter anderem entdeckte er den Unterschied der M"achtigkeit von
nat"urlichen und reellen Zahlen. Daher haben wir in der Geschichte
den Bus, dessen Sitzpl"atze mit reellen Zahlen angeschrieben waren,
als Cantor-Bus bezeichnet.

Verwandt damit und mit einem "ahnlichen Beweis kann man auch einsehen,
dass die Potenzmenge einer abz"ahlbaren Menge "uberabz"ahlbar ist.

\begin{satz}\label{powersetuncountable}
Ist $A$ eine abz"ahlbar unendliche Menge, dann ist $P(A)$
"uberabz"ahlbar.
\end{satz}

\begin{proof}[Beweis]
Da $A$ abz"ahlbar ist gibt es eine bijektive Abbildung
$i\mapsto a_i\in A$. Nehmen wir an, es g"abe eine bijektive
Abbildung von $\mathbb N$ in $P(M)$, also $i\mapsto A_i\subset A$.
Dann kann man wie folgt eine Menge bilden, die unter den Mengen
$A_i$ nicht vorkommt.
\[
a_i\in B\quad\Leftrightarrow\quad a_i\not\in A_i
\]
Die Menge $B$ unterscheidet sich von jeder Menge $A_i$. Falls
$A_i$ das Elemente $a_i$ enth"alt, enth"alt $B$ es nicht und
umgekehrt. Somit ist $B\ne A_i\forall i$, im Widerspruch zur Annahme
dass $i\mapsto A_i$ eine Bijektion ist.
\end{proof}

Jetzt k"onnen wir auch einige Mengen aus der bisher betrachteten
Theorie der Sprachen auf ihre M"achtigkeit untersuchen.

\begin{satz}
$\Sigma^*$ ist abz"ahlbar unendlich. Die Menge aller Sprachen "uber dem
Alphabet $\Sigma$ ist "uberabz"ahlbar unendlich.
\end{satz}

\begin{proof}[Beweis]
Eine Aufz"ahlung von $\Sigma^*$ kann man konstruieren, indem man
die W"orter von $\Sigma^*$ der L"ange nach sortiert, und innerhalb
der W"orter gleicher L"ange die lexikographische Ordnung verwendet.
Dazu braucht man nat"urlich eine Anordnung der Zeichen des Alphabets $\Sigma$,
da dieses aber endlich ist, kann man immer eine solche Anordnung finden.

Die Menge aller Sprache ist die Potenzmenge der Menge $\Sigma^*$, welche
abz"ahlbar unendlich ist. Nach Satz \ref{powersetuncountable} ist
$P(\Sigma^*)$ "uberabz"ahlbar unendlich.
\end{proof}

\begin{satz}\label{countablefinite}
Eine abz"ahlbare Vereinigung $\bigcup_{i\in\mathbb N}A_i$ von endlichen
Mengen $(|A_i|<\infty)$ ist abz"ahlbar.
\end{satz}

\begin{proof}[Beweis]
Endliche Mengen kann man immer abz"ahlen, also kann man eine Abz"ahlung
von $\bigcup_{i\in\mathbb N}A_i$ einfach dadurch konstruieren, dass
man zuerst die Elemente von $A_0$ durchz"ahlt, dann die von $A_1$, und
so weiter. Weil keine der Mengen $A_i$ unendlich ist, kommt jede
Menge irgend wann dran.
\end{proof}

\begin{satz} Sei $\Sigma$ ein festes Alphabet. Dann sind die folgenden
Mengen alle abz"ahlbar unendlich:
\begin{enumerate}
\item Die Menge aller deterministischen endlichen Automaten.
\item Die Menge aller nichtdeterministischen endlichen Automaten.
\item Die Menge der regul"aren Sprachen.
\item Die Menge aller kontextfreien Grammatiken.
\item Die Menge aller kontextfreien Sprachen.
\item Die Menge aller Stackautomaten.
\item Die Menge aller Turingmaschinen.
\end{enumerate}
\end{satz}

\begin{proof}[Beweis]
\begin{enumerate}
\item Es gen"ugt zu zeigen, dass die Menge der deterministischen
endlichen Automaten mit $k$ Zust"anden endlich ist, dann ist nach
Satz \ref{countablefinite} auch die Vereinigung abz"ahlbar. Ein
deterministischer endlicher Automat ist aber durch die Tabellendarstellung
gegeben. In der Tabelle sind $|Q|\cdot|\Sigma|$ Felder mit Zust"anden
zu besetzen, daf"ur gibt es $|Q|^{|Q|\cdot|\Sigma|}$ M"oglichkeiten.
F"ur jede solche M"oglichkeit ist ausserdem festzulegen, welche Zust"ande
Akzeptierzust"ande sind, das sind $|P(Q)|=2^{|Q|}$ M"oglichkeiten. Es gibt
also
\[
|Q|^{|Q|\cdot|\Sigma|}\cdot 2^{|Q|}
\]
DEAs mit $|Q|$ Zust"anden.
\item Obwohl in der Definition der nichtdeterministischen endlichen
Automaten auch die Potenzmenge vorkommt, wird dadurch die Menge
noch nicht "uber\-abz"ahlbar. Der Funktionswert der "Ubergangsfunktion
$\delta$ ist ja immer eine Teilmenge einer endlichen Menge, die Potenzmenge
einer endlichen Menge ist aber auch endlich. Es "andert sich am Argument
nur, dass in die Tabelle eines Automaten mit $|Q|$ Zust"anden und
$|\Sigma|$ Zeichen im Alphabet an $|Q|\cdot|\Sigma|$ Stellen
nicht $|Q|$ verschieden Objekte einf"ullen lassen, sondern $2^{|Q|}$.
An der Abz"ahlbarkeit "andert dies nichts.
\item Die regul"aren Sprachen werden von deterministischen endlichen
Automaten akzeptiert, es gibt also eine surjektive Abbildung von
den deterministischen endlichen Automaten auf die regul"aren Sprachen,
eine Aufz"ahlung der deterministischen endlichen Automaten liefert also
auch automatisch eine Aufz"ahlung der regul"aren Sprachen.
\item Es gibt nur endliche viele Grammatiken mit $n$ verschiedenen
Variablen und h"ochstens $n$ Zeichen langen rechten Seiten der Regeln,
einfach weil es nur endlich viele rechte Seiten gibt. Also ist die
Menge aller kontextfreien Grammatiken abz"ahlbar.
\item Kontextfreie Sprachen werden von einer kontextfreie Grammatik
erzeugt. Die Abbildung $G\mapsto L(G)$ macht aus einer Aufz"ahlung
der kontextfreien Grammatiken eine Aufz"ahlung der kontextfreien Sprachen.
\item Stackautomaten sind nichtdeterministische endliche Automaten mit
zus"atzlicher Beschriftung der "Ubergangspfeile. Trotzdem bleibt die
Anzahl der Stackautomaten mit $k$ Zust"anden endlich, also ist die
Menge der Stackautomaten abz"ahlbar.
\item Turingmaschinen sind im wesentlichen deterministsiche endliche Automaten
mit zus"atzlichen Beschriftungen der Pfeile. Wie in 1.~gibt es nur endlich
viele Turingmaschinen mit $k$ Zust"anden, also sind die Turingmaschinen
abz"ahlbar unendlich.
\end{enumerate}
\end{proof}

\subsection{Nicht berechenbare Zahlen}

\begin{satz}
Die rationalen Zahlen sind berechenbar.
\end{satz}

\begin{proof}[Beweis]
Der Algorithmus der schriftlichen Division erlaubt alle Stellen
zu finden, man kann sich gut vorstellen, dass er sich auf einer
Turingmaschine als Aufz"ahler implementieren l"asst. Z"ahler und
Nenner werden zu Beginn auf das Band geschrieben, der Algorithmus
berechnet dann alle Stellen des Quotienten. Also sind die rationalen
Zahlen berechenbar.
\end{proof}

\begin{satz}Die algebraischen Zahlen, also die Nullstellen von
Polynomen mit rationalen Koeffizienten sind berechenbar.
\end{satz}

\begin{proof}[Beweis]
Zun"achst k"onnen wir das Polynom mit dem gemeinsamen Nenner der
Koeffizienten multiplizieren und erhalten ein Polynom mit ganzzahligen
Koeffizienten.
Die Nullstellen eines Polynoms k"onnen mit dem Newton-Algorithmus
mit beliebiger Genauigkeit bestimmt werden. Dazu muss nur eine
Sch"atzung $\hat x_0$ f"ur die Nullstelle bekannt sein, dann k"onnen
mit der Iteration
\[
x_{n+1}=x_n-\frac{f(x_n)}{f'(x_n)},\quad x_0 = \hat x_0
\]
immer genauere Approximationen berechnet werden. Alle Operationen
in der Iterationsformel sind Operationen in rationalen Zahlen, sind
also berechenbar.
\end{proof}

\begin{satz}
Die Menge der nicht berechenbaren Zahlen ist "uberabz"ahlbar.
\end{satz}

\begin{proof}[Beweis]
Die berechenbaren Zahlen sind abz"ahlbar. Wir k"onnen n"amlich
eine Liste aller Turingmaschinen erstellen, zun"achst schreiben
wir die Turingmaschinen mit nur einem Zustand hin, dann all jene
mit genau zwei Zust"anden u.\,s.\,w. Da jede Turingmaschine genau eine
Zahl berechnen kann, ist die Menge der berechnenbaren Zahlen abz"ahlbar,
die Menge der reellen Zahlen ist also "uberabz"ahlbar, daher m"ussen die
nicht berechenbaren Zahlen auch "uberabz"ahlbar sein.
\end{proof}

\subsection{Das 10.~Hilbertsche Problem}
\index{Hilbert!David}
\index{Hilbertsche Probleme}
Im Jahre 1900 hielt der deutsche Mathematiker David Hilbert am
\index{Internationaler Kongress}
internationalen Mathematikerkongress in Paris einen ber"uhmten Vortrag,
in dem er eine Reihe von Problemen zusammenstellte, deren L"osung
nach seiner Meinung die Mathematik im zwanzigsten Jahrhundert
entscheidend voranbringen w"urden. Einige dieser Probleme wurden
inzwischen gel"ost, andere, darunter die Riemannsche Vermutung,
sind immer noch offen.

Von besonderem Interesse f"ur unser Thema war das zehnte Problem:
Gibt es ein Verfahren, das f"ur eine beliebige diophantische Gleichung
entscheidet, ob sie l"osbar ist?

Schon die Problemstellung wirft einige Fragen auf.
Zun"achst zu den Begriffen:
\index{Gleichung!diophantische}
eine diophantische Gleichung ist eine
Polynomgleichung mit mehreren Variablen aber ausschliesslich
ganzzahligen Koeffizienten. Die Gleichung
\[
x^2+y^2-z^2=0
\]
ist eine diophantische Gleichung, und es ist auch bekannt, dass sie
L"osungen hat, zum Beispiel $x=3$, $y=4$ und $z=5$. Seit wenigen
Jahren ist auch bekannt, dass
\[
x^n+y^n-z^n=0,
\]
ebenfalls eine diophantische Gleichung, nur ganz wenige L"osungen hat.
Dies ist die ber"uhmte Fermatsche Vermutung, die Andrew Wiles 1995
\index{Fermatsche Vermutung}
\index{Wiles, Andrew}
vollst"andig bewiesen hat.

Die schwierigerere Frage aber ist: Was f"ur eine Art von Verfahren
ist gemeint? Hilbert hat keine Definition gegeben. Heute w"urden
wir wohl fragen, ob ein Algorithmus angegeben werden kann, mit dem
die Frage entschieden werden kann. Doch das f"uhrt uns nur wieder
auf die Frage nach einer  mathematisch strengen Definition was
ein Algorithmus ist.

\index{Turing, Alan}
Im Jahre 1936 gab Alan Turing eine Antwort: ein Algorithmus ist eine
Rechenvorschrift, die sich mit einer Turingmaschine implementieren
l"asst. Probleme, f"ur die es einen L"osungsalgorithmus gibt, sind
also solche, deren Inputdaten man als Wort auf das Band einer
Turingmaschine schreiben kann, die Turingmaschine verarbeitet die
Problembeschreibung, schreibt die L"osung auf das Band und h"alt im
Zustand $q_{\text{accept}}$ an. Falls das Problem keine L"osung hat,
h"alt die Maschine im Zustand $q_{\text{reject}}$ an. Solche
Problem definieren also auch eine Sprache: $L(M)$ ist die
Menge der Inputw"orter, f"ur die eine L"osung existiert.

Ein Beispiel soll dies illustrieren. Wir m"ochten die Frage
von einem Algorithmus beantworten lassen, ob es in einem Graphen $G$
einen Weg gibt, der alle Knoten trifft.  Dazu m"ussen wir eine
Beschreibung des Graphen erzeugen, wir nennen sie $\langle G\rangle$,
diese auf das Band einer geeignet programmierten Turingmaschine $M$ schreiben,
diese laufen lassen, und warten, bis sie die Frage beantwortet. $L(M)$
besteht aus Beschreibungen $\langle G\rangle$ von Graphen $G$, die
einen solchen Weg haben.

Auf dieser Basis gelang es 1970 Yuri Matijasevi\v c "ubrigens,
\index{Matijasevi\v c, Yuri}
das zehnte Hilbertsche Problem etwas "uberraschend zu l"osen: es gibt
keinen Algorithmus, mit dem entschieden werden kann, ob eine
diophantische Gleichung eine L"osung hat. Es ist also durchaus
nicht selbstverst"andlich, wenn ein Problem eine algorithmische
L"osung hat. Wir werden im n"achsten Kapitel sehen, dass es viele
Probleme gibt, die keine algorithmische L"osung haben. Im "ubern"achsten
werden wir dann Probleme kennenlernen, die zwar von einem Computer
gel"ost werden k"onnen, deren L"osung aber l"anger dauert als man
jeden realistischen Computer laufen lassen kann.
Die im folgenden entwickelte Theorie zeigt also, wo die Grenzen
der Berechenbarkeit mit Computern liegen.

%
%  entscheidbarkeit.tex
%
\chapter{Entscheidbarkeit\label{chapter-entscheidbarkeit}}
\lhead{Entscheidbarkeit}
\rhead{}
Die Menge aller Turing-entscheidbaren Sprachen ist abzählbar,
da die Turingmaschinen abzählbar sind.
Die Menge $\Sigma^*$ aller Wörter über dem Alphabet $\Sigma$ ist abzählbar.
Andererseits ist die Menge aller Sprachen über einem Alphabet $\Sigma$
als Potenzmenge einer abzählbaren Menge überabzählbar.
Es muss also auch Sprachen geben, die nicht entscheidbar sind.
Es ist also
ein durchaus interessantes Unterfangen, nicht entscheidbare Sprachen
zu finden.

In diesem Kapitel zeigen wir, wie Probleme als Sprachprobleme
formuliert werden können und untersucht werden kann, ob sie entscheidbar sind,
also ob es ein Computerprogramm geben kann, mit dem sie gelöst werden
können.

\section{Entscheidbare Sprachen}
\rhead{Entscheidbare Sprachen}
\index{Sprache!entscheidbare}%
Wenn ein Problem von einem Computer gelöst werden soll, dann
muss der Computer die Problembeschreibung und eine mögliche Lösung
lesen können, und entscheiden können, ob die beiden zusammen passen.
Er muss also korrekte Problem- und Lösungs-Beschreibungen von
inkorrekten unterscheiden können.
Die Menge der korrekten
Beschreibungen bilden eine Sprache, das Problem ist also darauf
zurückgeführt, eine Sprache zu entscheiden.

\subsection{Entscheidbare Sprachen}
Eine Sprache heisst entscheidbar, wenn es einen Entscheider gibt, der
dazu verwendet werden kann, herauszufinden, ob ein Wort $w$ in der Sprache
liegt.

\begin{beispiel}
Sei $A$ ein DEA.
Dann kann man folgenden Algorithmus $M_A$ mit Input $w$ bauen:
\begin{compactenum}
\item Simuliere den DEA $A$ auf dem Input $w$
\item Falls $A$ in einem Akzeptierzustand steht: $q_{\text{accept}}$
\item Andernfalls: $q_{\text{reject}}$
\end{compactenum}
Dieser Algorithmus ist sicher ein Entscheider, er terminiert immer.
Ob er ein Wort $w$ akzeptiert, hängt davon ab, ob $A$ das Wort akzeptiert:
\begin{align*}
  A&\text{ akzeptiert $w$}&
   &\Leftrightarrow&
M_A&\text{ erkennt $w$}
\\
  A&\text{ akzeptiert $w$ nicht}&
   &\Leftrightarrow&
M_A&\text{ erkennt $w$ nicht}
\end{align*}
Das bedeutet aber, dass $L(A)=L(M_A)$, oder $M_A$ ist ein Entscheider für
die Sprache $L(A)$.
\end{beispiel}

Zu jeder regulären Sprache mit DEA $A$ gibt es also einen
Algorithmus $M_A$, der die gleiche Sprache erkennt: $L(A)=L(M_A)$.
Kürzer:

\begin{satz}
\label{regulaer-entscheidbar}
Reguläre Sprachen sind entscheidbar.
\end{satz}

\begin{beispiel}
Sei $G$ eine kontextfreie Grammatik.
Dann kann man den folgenden Algorithmus $M_G$ mit Input $w$ bauen:
\begin{compactenum}
\item Wandle $G$ in Chomsky-Normalform $G'$ um
\item Wende den CYK-Algorithmus (Satz \ref{cyk-algorithm}) auf $G'$ und das
Wort $w$ an
\item Wenn der CYK-Algorithmus das Wort $w$ akzeptiert: $q_{\text{accept}}$
\item Andernfalls: $q_{\text{reject}}$.
\end{compactenum}
Weil der CYK-Algorithmus (Satz \ref{cyk-algorithm}) deterministisch 
feststellen kann, ob ein Wort von einer Grammatik erzeugt wird, ist
$M_G$ ein Entscheider.
$M_G$ erkennt ein Wort genau dann, wenn es von
der Grammatik produziert wird, also $L(M_G)=L(G)$.
\end{beispiel}

Zu jeder kontextfreien Sprache mit Grammatik $G$ gibt es also einen 
Algorithmus $M_G$, der die gleiche Sprache erkennt: $L(G)=L(M_G)$.
Kürzer:

\begin{satz}
Kontextfreie Sprachen sind entscheidbar.
\end{satz}
Dieser Satz zeigt, dass die entscheidbaren Sprachen eine 
Obermenge der kontextfreien Sprachen sind, wie in folgender
Grafik dargestellt.
\begin{center}
%\includegraphics[width=0.7\hsize]{images/lang-2}
\includegraphics{images/lang-2}
\end{center}

\subsection{Entscheidbare Probleme für reguläre Sprachen}
Wir wollen jetzt eine Ebene höher gehen.
Unser Fokus sind jetzt nicht
mehr einzelne Sprachen, sondern eine ganze Menge von Sprachen, und eventuell
weitere Daten.
Wir suchen also nicht mehr Algorithmen, die eine Frage
für eine einzelne Sprache beantworten können, sondern Algorithmen, die
die Frage gleich für mehrere Sprachen beantworten.

In Satz~\ref{regulaer-entscheidbar} ging es darum, einen Algorithmus
zu finden, der entscheidet, ob ein Wort $w$ von {\em einem} ganz bestimmten,
vorgegebenen endlichen Automaten $A$ akzeptiert wird.
Jetzt fragen wir, ob es einen Algorithmus gibt, der für {\em jede beliebige}
Kombination von DEA und Wort $w$ diese Entscheidung treffen kann.
Dies ist offenbar ein viel allgemeineres Problem, ob $A$ das Wort
akzeptiert ist ein Spezialfall.
Dieses Problem heisst das
Akzeptanzproblem für endliche Automaten.

\subsubsection{Akzeptanzproblem}
\index{Akzeptanzproblem!für reguläre Sprachen}%
Wir betrachten das folgende Problem: Gegeben sei ein endlicher Automat $B$
und ein Wort $w$,
kann man nun entscheiden, ob der Automat das Wort akzeptiert?
Um dies als Sprachproblem zu formulieren, braucht  man eine Codierung
$\langle B,w\rangle$ von Automat und Wort.
Die Sprache, die zu
entscheiden ist, lautet dann:
\[
A_{\text{DEA}} =\{
\langle B,w\rangle\;|\;\text{$B$ ist ein DEA und $B$ akzeptiert $w$}.
\}
\]
\index{ADEA@$A_{\text{DEA}}$}%
\begin{satz}
\label{adea_decidable}
$A_{\text{DEA}}$ ist entscheidbar.
\end{satz}
\index{Akzeptanzproblem!für DEAs}%

\begin{proof}[Beweis]
Der folgende Algorithmus entscheidet
$A_{\text{DEA}}$:
\medskip
\begin{compactenum}
\item Simuliere $B$ auf einer TM mit Input $w$.
\item Falls $B$ akzeptiert, gehe in den Zustand $q_{\text{accept}}$.
\item Falls $B$ nicht akzeptiert, gehe in den Zustand $q_{\text{reject}}$.
\end{compactenum}
\medskip
\end{proof}

Das gleiche Problem kann man auch für nicht deterministische Automaten
formulieren:

\begin{satz}
\index{Akzeptanzproblem!für NEAs}%
Das Akzeptanzproblem für nicht deterministische endlich Automaten
\[
A_{\text{NEA}} =\{
\langle B,w\rangle\;|\;\text{$B$ ist ein NEA und $B$ akzeptiert $w$}.
\}
\]
\index{ANEA@$A_{\text{NEA}}$}%
ist entscheidbar.
\end{satz}

\begin{proof}[Beweis]
Der folgende Algorithmus entscheidet 
$A_{\text{NEA}}$:
\medskip
\begin{compactenum}
\item Wandle $B$ in einen DEA $B'$ um.
\item Verwende die TM von Satz \ref{adea_decidable}, um zu
entscheiden, ob $B'$ das Wort $w$ akzeptiert.
\end{compactenum}
\medskip
\end{proof}

\begin{satz} Das Akzeptanzproblem für reguläre Ausdrücke
\index{Akzeptanzproblem!für regul\äre Ausdr\ücke}%
\[
A_{\text{REX}}=\{
\langle R,w\rangle\;|\;\text{$R$ ist ein regulärer Ausdruck und $R$ akzeptiert $w$}
\}
\]
\index{AREX@$A_{\text{REX}}$}%
ist entscheidbar.
\end{satz}

\subsubsection{Leerheitsproblem}
\index{Leerheitsproblem!für DEAs}%
Wir wollen einem endlichen Automaten ansehen können, ob er überhaupt
irgend ein Wort akzeptieren wird, und untersuchen daher
\[
E_{\text{DEA}}
=\{
\langle A\rangle \;|\;\text{$A$  ist ein DEA und $L(A)=\emptyset$}
\}.
\]
\begin{satz}
$E_{\text{DEA}}$
\index{EDEA@$E_{\text{DEA}}$}%
ist entscheidbar.
\end{satz}

\begin{proof}[Beweis]
Man muss herausfinden, ob überhaupt je ein Akzeptierzustand erreicht
werden kann.
Der folgende Algorithmus schafft dies:
\medskip
\begin{compactenum}
\item Markiere den Startzustand
\item Solange sich noch neue Zustände markieren liessen:
 Markiere alle Zustände, die sich von den bereits markierten aus
erreichen lassen.
\item Falls ein Akzeptierzustand markiert wurde: $q_{\text{reject}}$,
andernfalls
$q_{\text{accept}}$
\end{compactenum}
\medskip
\end{proof}

\subsubsection{Gleichheitsproblem}
\index{Gleichheitsproblem!für DEAs}%
Akzeptieren zwei DEAs die gleiche Sprache? Im Kapitel \ref{chapter-regular}
haben wir einen Algorithmus skizziert, mit dem dies entschieden werden kann.

\begin{satz}
\label{satz:eqdea}
Die Sprache
\[
\text{\it EQ}_{\text{DEA}}=\{
\langle A,B\rangle\;|\;\text{$A$ und $B$ sind DEAs und $L(A)=L(B)$}
\}
\]
\index{EQDEA@$\textit{EQ}_{\text{DEA}}$}%
ist entscheidbar.
\end{satz}

\begin{proof}[Beweis 1]
In Anlehnung an Kapitel \ref{chapter-regular} können wir den folgenden
Algorithmus verwenden:
\medskip
\begin{compactenum}
\item Erzeuge den minimalen Automaten $A'$ zu $A$.
\item Erzeuge den minimalen Automaten $B'$ zu $B$.
\item Falls $A'=B'$: $q_{\text{accept}}$, andernfalls $q_{\text{reject}}$.
\end{compactenum}
\medskip
Dieser Algorithmus entscheidet offenbar 
$\text{\it EQ}_{\text{DEA}}$.
\end{proof}

\begin{proof}[Beweis 2]
Noch etwas eleganter ist der folgende Beweis.
Ob $L(A)=L(B)$ ist
offenbar gleichbedeutend damit, dass die symmetrische Differenz
(Abbildung~\ref{symdiff})
\[
L(A){\;\Delta\;} L(B)=
(L(A)\setminus L(B)) \cup (L(B)\setminus L(A))
\]
leer ist.
Da sich die Mengenoperationen aber durch Operationen mit
endlichen Automaten abbilden lassen, gibt es einen Algorithmus,
der  zu jedem Paar $(A,B)$
von endlichen Automaten  einen endlichen Automaten $C$ konstruiert mit
$L(C)=L(A){\;\Delta\;} L(B)$.
Dann muss nur noch mit Hilfe eines Entscheiders
für $E_{\text{DEA}}$ entschieden werden, ob $\langle C\rangle\in
E_{\text{DEA}}$.
\end{proof}

Das Prinzip dieses zweiten Beweises wird uns später weiter beschäftigen.
Entscheidend war, dass es einen Algorithmus gibt, mit dem das
Entscheidungsproblem 
$\text{\it EQ}_{\text{DEA}}$
auf das Leerheitsproblem
$E_{\text{DEA}}$ zurückgeführt werden konnte.
Die Entscheidung erfolgte dann mit dem Entscheider für
$E_{\text{DEA}}$.

\subsection{Entscheibarkeitsproblem für kontextfreie Sprachen}
\subsubsection{Akzeptanzproblem}
\index{Akzeptanzproblem!für kontextfreie Grammatiken}%
\begin{satz}
\label{satz:acfg-entscheidbar}
Die Sprache
\[
A_{\text{CFG}}=\{
\langle G,w\rangle\;|\; \text{die kontextfreie Grammatik $G$ erzeugt $w$}
\}
\]
ist entscheidbar.
\index{ACFG@$A_{\text{CFG}}$}%
\end{satz}

\begin{proof}[Beweis 1]
Der CYK-Algorithmus von Satz \ref{cyk-algorithm} entscheidet
$A_{\text{CFG}}$.
\end{proof}

\begin{proof}[Beweis 2]
Der folgende Algorithmus entscheidet
$A_{\text{CFG}}$:
\medskip
\begin{compactenum}
\item Wandle $G$ in Chomsky Normalform um.
\item Erzeuge alle Wörter, die sich durch Anwendung
von $|w|-1$ Regeln der Form $A\to BC$ und $|w|$ Regeln
der Form $A\to a$ ableiten lassen.
\item Falls $w$ darunter vorkommt: $q_{\text{accept}}$, 
andernfalls $q_{\text{reject}}$.
\end{compactenum}
\medskip
\end{proof}

\subsubsection{Leerheitsproblem}
\index{Leerheitsproblem!für kontextfreie Grammatiken}%
\begin{satz}
Die Sprache
\[
E_{\text{CFG}}=\{
\langle G\rangle\;|\; \text{$G$ ist eine kontextfreie Grammatik und $L(G)=\emptyset$}
\}
\]
ist entscheidbar.
\index{ECFG@$E_{\text{CFG}}$}%
\end{satz}

\begin{proof}[Beweis]
Eine Grammatik erzeugt Wörter, wenn alle Variablen, die im Laufe
der Ableitung auftreten, am Ende mit Hilfe von Regeln der Form $A\to a$
in Terminalsymbole umgewandelt werden können.
Man könnte also einen
Markierungsalgorithmus entwickeln, der alle Terminalsymbole markiert,
und alle Symbole, die mit geeigneten Regeln ebenfalls markierte Symbole
erzeugen.
Falls die Startvariable nicht markiert wird, ist die Sprache leer.
Der Algorithmus 
\medskip
\begin{compactenum}
\item Wandle $G$ in Chomsky Normalform um.
\item Markiere alle Terminalsymbole.
\item Solange sich neue Variablen markieren lassen: markiere alle
Variablen $A$, zu denen es eine Regel gibt, so dass alle Symbole auf
der rechten Seite der Regel bereits markiert sind.
\item Falls $S$ markiert wurde: $q_{\text{reject}}$, andernfalls
$q_{\text{accept}}$.
\end{compactenum}
\medskip
entscheidet also
$E_{\text{CFG}}$.
\end{proof}

\subsubsection{Gleichheit}
\index{Gleichheitsproblem!für kontextfreie Grammatiken}%
Es ist verlockend zu vermuten, dass das Gleichheitsproblem
\[
\text{\it EQ}_{\text{CFG}}=\{
\langle G,H\rangle\;|\; \text{$G$  und $H$ sind kontextfreie Grammatiken und $L(G)=L(H)$}
\}
\]
\index{EQCFG@$\textit{EQ}_{\text{CFG}}$}%
mit der gleichen Methode mit Hilfe der symmetrischen Differenz
wie bei regulären Sprachen entschieden werden könnte.
Dem ist
allerdings nicht so, denn das Komplement einer kontextfreien Sprache
ist im Allgemeinen nicht kontextfrei, also auch die symmetrische
Differenz nicht.
In der Tat ist 
$\text{\it EQ}_{\text{CFG}}$ nicht entscheidbar, die Methoden zum Beweis
werden allerdings erst später bereitgestellt.

\section{Das Akzeptanzproblem für Turingmaschinen}
\index{Akzeptanzproblem!für Turingmaschinen}%
\rhead{Akzeptanzproblem für Turingmaschinen}
Das Akzeptanzproblem für Turing-erkennbare Sprachen fragt, ob 
eine Turingmaschine $M$ ein Inputwort $w$ erkennen wird.
Das Problem ist entscheidbar, wenn man aus der Beschreibung
der Maschine $M$ und dem Inputwort mit einem Algorithmus, der
für alle Turingmaschinen und alle Wörter immer anhält, ableiten
kann, ob $M$ das Wort $w$ akzeptieren wird.
Man kann also immer
entscheiden, ob $M$ im Zustand $q_{\text{accept}}$ anhalten wird.

Dabei können wir nicht einfach $M$ laufen lassen, weil $M$
ja auch in eine Endlosschleife fallen könnte, womit sich die
Frage nicht mehr entscheiden liesse.

Als Sprachproblem formuliert, suchen wir jetzt also einen
Entscheider für die Sprache
\[
A_{\text{TM}}=\{
\langle M,w\rangle\;|\; \text{$M$ ist eine TM und $M$ erkennt $w$}
\}
\]
\index{ATM@$A_{\text{TM}}$}%

\begin{satz}
\label{ATM}
$A_{\text{TM}}$ ist nicht entscheidbar.
\end{satz}

Auf den ersten Blick ist dieses Resultat sehr überraschend, warum
soll für Turing-Maschinen alles grundsätzlich anders sein, als
für endliche Automaten und Stackautomaten? Der wesentliche Unterschied
verbirgt sich in dem Wort ``entscheidbar'' selbst.
Ein Entscheider ist eine
TM, welche in diesem Fall Aussagen über andere TMs machen muss, also
zum Beispiel auch über sich selbst.
Bei DEAs und CFGs war die Sache
insofern einfacher, als eine TM Aussagen über einen DEA oder eine
CFG gemacht hat.

Ein solcher Selbstbezug kann einen in Teufels Küche bringen.
Nehmen wir an, in einem Dorf lebt ein Barbier, der genau diejenigen
Leute rasiert, die sich selbst nicht rasieren.
Soll er sich rasieren?
Wenn er sich rasiert, gehört er zu den Leuten, die sich selbst rasieren,
die rasiert er nicht, also darf er sich nicht rasieren.
Wenn er sich
aber nicht rasiert, gehört er zu den Leuten, die sich nicht selbst
rasieren, daher muss er sich rasieren.
Der Rückbezug ist, was den
Barbier in die Zwickmühle bringt.
Und genau so einen ``Barbier''
konstruiert der Beweis des Satzes~\ref{ATM}.

\begin{proof}[Beweis]
Wir führen den Beweis mit Hilfe eines Widerspruchs.
Dazu nehmen wir
an, es gäbe eine Turingmaschine $H$, welche das Akzeptanzproblem entscheidet.
Sie verhält sich wie folgt:
\[
H(\langle M,w\rangle)\begin{cases}
\text{erkennt}&\qquad\text{falls $M$ auf $w$ im Zustand $q_{\text{accept}}$ anhält}
\\
\text{verwirft}&\qquad\text{falls $M$ auf $w$ nicht oder im Zustand $q_{\text{reject}}$ anhält}
\end{cases}
\]
Daraus konstruieren wir jetzt eine neue TM $D$ mit Input $\langle M\rangle$:
\medskip
\begin{compactenum}
\item Lasse $H$ auf dem Input $\langle M,\langle M\rangle\rangle$ laufen
\item Falls $H$ erkennt: $q_{\text{reject}}$
\item Falls $H$ verwirft: $q_{\text{accept}}$
\end{compactenum}
\medskip
Die Maschine $D$ verhält sich wie folgt:
\[
D(\langle M\rangle)\begin{cases}
\text{erkennt}&\qquad\Leftrightarrow\qquad \text{$M$ verwirft $\langle M\rangle$}
\\
\text{verwirft}&\qquad\Leftrightarrow\qquad \text{$M$ erkennt $\langle M\rangle$}
\end{cases}
\]
($D$ entspricht dem Barbier, der genau die rasiert, die sich selbst
nicht rasieren.)

Lassen wir jetzt $D$ auf seiner eigenen Beschreibung operieren
(wir fragen uns, ob der Barbier sich selbst rasieren soll), muss
sich folgendes Resultat ergeben
\[
D(\langle D\rangle)\begin{cases}
\text{erkennt}&\qquad\Leftrightarrow\qquad \text{$D$ verwirft $\langle D\rangle$}
\\
\text{verwirft}&\qquad\Leftrightarrow\qquad \text{$D$ erkennt $\langle D\rangle$}
\end{cases}
\]
$D$ muss also genau das Gegenteil dessen tun, was $D$ tut.
$D$ ist der Barbier, der sich nicht entscheiden kann, ob er sich selbst
rasieren will.
Dieser
Widerspruch zeigt, dass die Annahme, es gäbe einen Entscheider $H$
nicht haltbar ist.
Also ist das Problem nicht entscheidbar.
\end{proof}

\index{Goedel@Gödel, Kurt}%
Die Entdeckung, dass gewisse Probleme nicht entscheidbar sind,
geht auf Kurt Gödel zurück.
Gödel hat sie jedoch in leicht anderem Zusammenhang gefunden.
Er untersuchte die Frage, ob ein
Axiomensystem eine Aussage beweisen kann.
Dabei zeigte sich,
dass es Aussagen geben muss, die nicht bewiesen werden können,
obwohl sie wahr sind.

Die Formulierung für Turing-Maschinen geht auf eine Arbeit von
Alan Turing 1936 zurück.
\index{Turing, Alan}%
Aus der Lösung des Halteproblems kann
man eine konkrete solche Aussage ableiten.

Am Input $w$ allein liegt es nicht, dass das Problem nicht
entschieden werden kann.
Nicht einmal für den speziellen Input $\varepsilon$ ist es
entscheidbar, auch die Sprache
\[
\textit{HALT}\varepsilon_{\text{TM}}
=\{
\langle M\rangle \;|\;
\text{$M$ ist eine TM und hält auf Input $\varepsilon$}
\}
\]
\index{HALTepsilonTM@$\textit{HALT}\varepsilon_{\text{TM}}$}%
ist nicht entscheidbar.
Nehmen wir an, es gäbe einen
Entscheider $H$ für $\textit{HALT}\varepsilon_{\text{TM}}$, dann können wir daraus
auch einen Entscheider für $A_{\text{TM}}$ konstruieren.
Dazu gehen wir wie folgt vor.
Auf dem Input $\langle M,w\rangle$ bauen wir folgende Maschine $M'$:
\medskip
\begin{compactenum}
\item Schreibe $w$ auf das Band
\item Lasse $M$ laufen
\item Falls $M$ erkennt: $q_{\text{accept}}$.
\item Falls $M$ verwirft: gehe in eine Endlosschleife.
\end{compactenum}
\medskip
Die Maschine $M'$ hält auf leerem Input genau dann, wenn $M$
das Wort $w$ erkennt.
Der Entscheider $H$, angewendet auf $M'$
entscheidet also, ob $w\in L(M)$, löst also das Akzeptanzproblem
$A_{\text{TM}}$.
Da $A_{\text{TM}}$ nicht entscheidbar ist, darf es keinen solchen
Entscheider geben.
Dieser Widerspruch zeigt, dass die Annahme, es
gäbe einen Entscheider für $\textit{HALT}\varepsilon_{\text{TM}}$,
nicht haltbar ist.
$\textit{HALT}\varepsilon_{\text{TM}}$ heisst
auch das {\em spezielle Halteproblem}.
\index{Halteproblem!spezielles}%

Das Halteproblem beweist, dass es Sprachen gibt, die nicht
entscheidbar sind, obwohl sie von einer Turing-Maschine
erkennbar sind.
Die Turing-erkennbaren Sprachen bilden also eine
echte Obermenge der entscheidbaren Sprachen.
\begin{center}
%\includegraphics[width=0.8\hsize]{images/lang-3}
\includegraphics{images/lang-3}
\end{center}

\section{Reduktion}
\index{Reduktion}%
\rhead{Reduktion}
Im letzten Abschnitt wurde die Entscheidbarkeit der Sprache $L$
auf das Akzeptanzproblem in dem Sinne zurückgeführt, dass
aus einem Entscheider für $L$ ein Entscheider für das Akzeptanzproblem
konstruiert wurde.
Die Nichtentscheidbarkeit des Akzeptanzproblems
hat dann gezeigt, dass auch $L$ nicht entscheidbar sein kann.

Dieses Vorgehen funktioniert in sehr vielen Fällen, wir wollen es
daher etwas abstrakter formulieren und damit die Anwendung in weiteren
Beispielen vereinfachen.
Gleichzeitig ermöglicht es uns, die
Sprachen mindestens teilweise nach ihrer Schwierigkeit, sie zu entscheiden,
anzuordnen.

\subsection{Reduktionsabbildung}
\index{Reduktionsabbildung}%
\begin{figure}
\begin{center}
%\includegraphics[width=\hsize]{images/red-1}
\includegraphics{images/red-1}
\end{center}
\caption{Reduktionsabbildung $f\colon A\to B$.\label{reduktionsabbildung}}
\end{figure}
Wir möchten zwei verschiedene Sprachen 
$A$ und $B$ 
vergleichen.
Dazu bilden wir Wörter von der einen Sprache in
die andere Sprache ab, wir brauchen also eine Abbildung, die
$f\colon\Sigma^*\to \Sigma^*$, die $A$ in $B$ überführt,
d.\,h.
\[
w\in A\Leftrightarrow f(w)\in B.
\]
Diese Abbildung hilft uns allerdings nur dann, Entscheidbarkeitsfragen
von $B$ nach $A$ zu transportieren, wenn sie mit einer Turingmaschine
berechnet werden kann.
Wir definieren daher

\begin{definition}
\index{berechenbar}%
Eine Abbildung $f\colon \Sigma^*\to \Sigma^*$ heisst berechenbar wenn es eine
Turingmaschine gibt, die auf jedem Input $w\in \Sigma^*$ anhält
und ausschliesslich das Wort $f(w)$ auf dem Band zurücklässt.
\end{definition}

\begin{definition}
\index{Reduktion}%
Eine Sprache $A$ ist reduzierbar auf die Sprache $B$, in Zeichen
\[
A\le B
\]
wenn es eine
berechenbare Abbildung $f\colon \Sigma^*\to \Sigma^*$ gibt, mit
\[
w\in A\Leftrightarrow f(w)\in B.
\]
Wir schreiben dafür auch
\[
f\colon A\le B.
\]
\end{definition}
Die Schreibweise $A\le B$ soll ausdrücken, dass die Sprache $A$ ``leichter''
zu entscheiden ist als die Sprache $B$.
Die nachfolgenden zwei Sätze
zeigen, dass diese Leseart berechtigt ist.

\begin{satz}
Seien $A\le B$ Sprachen.
Ist $B$ entscheidbar, dann auch $A$.
\end{satz}

\begin{proof}[Beweis]
Ist $B$ entscheidbar, dann gibt es einen Entscheider für $B$, also
eine Turingmaschine $M$, die auf jedem Inputwort anhält, und
entscheidet, ob es zu $B$ gehört.
Daraus können wir jetzt einen Entscheider für $A$ konstruieren.
Um zu entscheiden, ob $w\in A$
ist, verwenden wir den folgenden Algorithmus $M'$ mit Input
$w$:
\medskip
\begin{compactenum}
\item Berechne $f(w)$, dafür gibt es wegen $A\le B$ eine Turingmaschine.
\item Verwende den Entscheider $M$ um zu entscheiden, ob $f(w)\in B$.
\item Falls $M$ akzeptiert ist $f(w)\in B\Rightarrow w\in A$
\item Falls $M$ nicht akzeptiert ist $f(w)\not\in B\Rightarrow w\not\in A$.
\end{compactenum}
\medskip
$M'$ ist offenbar ein Entscheider für $A$, also ist $A$ entscheidbar.
\end{proof}

Die Kontraposition dieser Aussage ist
\begin{satz}
Ist $A\le B$ und $A$ nicht entscheidbar, dann ist auch $B$ nicht
entscheidbar.
\end{satz}
Damit haben wir eine Methode, um Sprachen als nicht entscheidbar 
zu beweisen.
Wir müssen nur eine Abbildung $f\colon \Sigma^*\to \Sigma^*$
konstruieren, wobei $A$ ein bekanntermassen nicht
entscheidbares Problem ist, und $f$ eine berechenbare Abbildung ist,
die die Reduktion $A\le B$ vermittelt.

\subsection{Weitere nicht entscheidbare Sprachprobleme}
\index{Halteproblem}%
\begin{satz} Das Halteproblem
\[
\text{\it HALT}_{\text{TM}}=\{\langle M,w\rangle\;|\;
\text{$M$ ist eine TM und $M$ hält auf Input $w$}\}
\]
\index{HALTTM@$\textit{HALT}_{\text{TM}}$}%
ist nicht entscheidbar.
\end{satz}

\begin{proof}[Beweis]
Wir konstruieren eine Reduktion
$
A_{\text{TM}}
\le
\text{\it HALT}_{\text{TM}}
$.
Die Reduktion muss aus einer TM $M$ und einem Inputwort $w$ eine
neue TM $S$ und ein Inputwort $w$ machen, so dass $S$ auf
Input $w$ genau dann hält, wenn $M$ das Wort $w$ akzeptiert.
Die Abbildung $f$ ist also
\[
f\colon 
A_{\text{TM}}
\le
\text{\it HALT}_{\text{TM}}
\colon
\langle M,w\rangle\mapsto \langle S,w\rangle.
\]
$S$ konstruieren wir wie folgt:
\medskip
\begin{compactenum}
\item lasse $M$ laufen auf Input $w$
\item falls $M$ das Wort $w$ erkennt: $q_{\text{accept}}$
\item falls $M$ verwirft: beginne eine Endlosschleife
\end{compactenum}
\medskip
Offenbar ist
$\langle S,w\rangle\in \text{\it HALT}_{\text{TM}}$
genau dann, wenn $\langle M,w\rangle\in A_{\text{TM}}$.
Wir haben also
\[
\langle M,w\rangle\in A_{\text{TM}}
\qquad
\Leftrightarrow
\qquad
\langle S,w\rangle=f(\langle M,w\rangle)
\in \text{\it HALT}_{\text{TM}}.
\]
Somit ist 
$A_{\text{TM}}\le\text{\it HALT}_{\text{TM}}$, und da
$A_{\text{TM}}$ nicht entscheidbar ist, ist auch
$\text{\it HALT}_{\text{TM}}$ nicht entscheidbar.
\end{proof}
Statt des Anhaltens könnte man auch viele andere spezielle
Betriebszustände einer Turingmaschine als ``Haltekriterien''
heranziehen.
Dann bedeutet der Satz, dass es keinen allgemeinen
Algorithmus geben kann, der ein Stück Code inspizieren kann und
daraus ableiten kann, ob dieser Code auf einem System Schaden
anrichtet.
Echte Schadcode-Erkennung gibt es also grundsätzlich
nicht (auch wenn das Marketing-Material gewisser Produkte sich
da zu ganz erstaunlichen Aussagen versteigt).
Das beste, was man
tun kann, ist Mustererkennung (endliche Sprachen sind alle regulär).

\begin{satz}
\index{Leerheitsproblem!f\ür Turingmaschinen}%
Das Leerheitsproblem für Turingmaschinen
\[
E_{\text{TM}}
=
\{ \langle M\rangle\;|\; \text{$M$ ist eine TM und $L(M)=\emptyset$}\}
\]
\index{ETM@$E_{\text{TM}}$}%
ist nicht entscheidbar.
\end{satz}

\begin{proof}[Beweis]
Wir möchten eine Reduktion $A_{\text{TM}}\le E_{\text{TM}}$ konstruieren.
Dazu müssen wir aus $\langle M,w\rangle$ eine Maschine $S$ konstruieren,
die genau dann kein einziges Wort akzeptiert, wenn $M$ das Wort $w$
akzeptiert.
Es ist allerdings einfacher, die Negation zu behandeln,
also zu zeigen, dass 
\[
\bar E_{\text{TM}}=
\{ \langle M\rangle\;|\; \text{$M$ ist eine TM und $L(M)\ne\emptyset$}\}
\]
nicht entscheidbar ist.
Wir brauchen dann eine Reduktion
$A_{\text{TM}}\le \bar E_{\text{TM}}$, d.\,h.~eine Maschine $S$,
die genau dann mindestens ein Wort erkennt, wenn $M$ das Wort $w$ 
erkennt.

Die Maschine $S$ muss auf beliebigen Wörtern $u$ als Input ausgeführt
werden können.
Wir könnten folgenden Algorithmus verwenden:
\medskip
\begin{compactenum}
\item Falls $u\ne w$: $q_{\text{reject}}$.
\item Lasse $M$ auf $w$ laufen.
\item Falls $M$ erkennt: $q_{\text{accept}}$.
\end{compactenum}
\medskip
Wenn $M$ das Wort $w$ erkennt, dann erkennt $S$ die Sprache 
$L(S)=\{w\}$.
Wenn $M$ das Wort $w$ nicht erkennt, ist $L(S)=\emptyset$.
Oder
\begin{align*}
M&\operatorname{erkennt}w              &&\Rightarrow&L(S)&=\{w\}\ne \emptyset
\\
M&\operatorname{erkennt}w\text{ nicht} &&\Rightarrow&L(S)&=\emptyset
\end{align*}
oder
\[
\langle M,w\rangle \in A_{\text{TM}}
\qquad
\Leftrightarrow
\qquad
\langle S\rangle = f(\langle M,w\rangle)\in
\bar E_{\text{TM}},
\]
also $A_{\text{TM}}\le \bar E_{\text{TM}}$.
\end{proof}

\begin{satz}
Es ist nicht entscheidbar, ob eine Turingmaschine eine reguläre
Sprache erkennt.
\end{satz}

\begin{proof}[Beweis]
Als Sprache formuliert möchten wir eine Reduktion 
\[
A_{\text{TM}}\le\text{\it REGULAR}_{\text{TM}}=\{
\langle M\rangle\;|\;\text{$M$ ist eine TM und $L(M)$ ist regulär}
\}
\]
\index{REGULARTM@$\textit{REGULAR}_{\text{TM}}$}%
konstruieren.
Auch in diesem Fall ist es einfacher, eine
Reduktion
\[
A_{\text{TM}}\le\overline{\text{\it REGULAR}}_{\text{TM}}
\]
zu konstruieren.

Wir müssen also eine TM $S$ konstruieren, die genau
dann eine nicht reguläre Sprache erkennt, wenn $w$ von $M$ erkannt wird.
Der folgende Algorithmus tut dies.
Auf Input $u$ geht er wie folgt
vor:
\medskip
\begin{compactenum}
\item Falls $u$ nicht von der Form $0^n1^n$ ist: $q_{\text{reject}}$
\item Lasse $M$ auf $w$ laufen.
\item Falls $M$ $w$ erkennt: $q_{\text{accept}}$
\item Falls $M$ $w$ verwirft: $q_{\text{reject}}$
\end{compactenum}
\medskip
Wegen Schritt~1 kann $S$ höchstens die Wörter der Form
$\texttt{0}^n\texttt{1}^n$ erkennen, alle anderen werden bereits in
Schritt $1$ verworfen.
Ob die Wörter 
$\texttt{0}^n\texttt{1}^n$ erkannt werden, hängt davon ab, ob $M$ das Wort
$w$ erkennt:
\begin{align*}
   M&\text{ erkennt $w$}      &
    &\Leftrightarrow&
L(S)&=\{\texttt{0}^n\texttt{1}^n\,|\, n\ge 0\}&
    &\Leftrightarrow&
    &\text{$L(S)$ nicht regulär}\\
   M&\text{ erkennt $w$ nicht}&
    &\Leftrightarrow&
L(S)&=\{\quad\}=\emptyset&
    &\Leftrightarrow&
    &\text{$L(S)$ regulär}
\end{align*}
$S$ erkennt die nicht reguläre Sprache $\{0^n1^n\;|\;n\in\mathbb N\}$
genau dann, wenn $M$ das Wort $w$ erkennt.
Die Reduktion $\langle M,w\rangle\to \langle S\rangle$ übersetzt
``erkennen'' in ``nicht reguläre Sprache erkennen'', also also $A_{\text{TM}}$
in $\overline{\textsl{REGULAR}}_{\text{TM}}$.
\end{proof}

\index{Gleichheitsproblem!f\ür Turingmaschinen}%
\begin{satz}
Ob zwei Turingmaschinen die gleiche Sprache
\[
\text{\it EQ}_{\text{TM}}=\{
\langle M_1,M_2\rangle\;|\;\text{$M_i$ sind Turingmaschinen und $L(M_1)=L(M_2)$}
\}
\]
\index{EQTM@$\textit{EQ}_{\text{TM}}$}%
erkennen, ist nicht entscheidbar.
\end{satz}

\begin{proof}[Beweis]
Wir konstruieren eine Reduktion
$
E_{\text{TM}}\le \text{\it EQ}_{\text{TM}}
$
.
Dazu sei $M_0$ eine Turingmaschine, die jeden Input verwirft,
also $L(M_0)=\emptyset$.
Für die Abbildung $f$ verwenden wir
\[
f
\colon
E_{\text{TM}}\le \text{\it EQ}_{\text{TM}}
\colon
\langle M\rangle\mapsto \langle M,M_0\rangle
\]
Es ist $\langle M,M_0\rangle\in \text{\it EQ}_{\text{TM}}$
genau dann, wenn
$L(M)=L(M_0)=\emptyset$, also genau dann, wenn
$\langle M\rangle\in E_{\text{TM}}$.
\end{proof}

Dieser Satz hat zum Beispiel zur Folge, dass es keine zuverlässige
maschinelle
Möglichkeit gibt, herauszufinden, ob zwei Programme das gleiche tun.
Damit ist die Entscheidung, ob ein beliebiges Programm ein beliebiges
Softwarepatent verletzt, nur durch den subjektiven Prozess in Form
eines Gerichtes möglich, kann also grundsätzlich niemals objektiv sein.

\section{Nicht entscheidbare Probleme für CFGs}
\rhead{Nicht entscheidbare Probleme für CFGs}
Der letzte Abschnitt könnte den Eindruck hinterlassen haben,
dass nicht entscheidbare Probleme erst bei Turingmaschinen
auftreten, und dass alle naheliegenden Probleme bei regulären
und kontextfreien Sprachen entscheidbar sind.
Dem ist jedoch nicht so.
Es ist zum Beispiel nicht entscheidbar, ob zwei
Grammatiken die gleiche Sprache erzeugen.

\begin{satz} Ob eine kontextfreie Grammatik
\[
\text{\it ALL}_{\text{CFG}}=\{\langle G\rangle\;|\; \text{$G$
ist eine kontextfreie Grammatik und $L(G)=\Sigma^*$}\}
\]
\index{ALLCFG@$\textit{ALL}_{\text{CFG}}$}%
alle Wörter erzeugt, ist nicht entscheidbar.
\end{satz}

\begin{proof}[Beweis]
Wir konstruieren eine Reduktion 
$A_{\text{TM}}\le \text{\it ALL}_{\text{CFG}}$.
Wir müssen also aus einem Paar $\langle M,w\rangle$ einen
Grammatik erzeugen, die genau dann alle Wörter erzeugt, wenn
$M$ das Wort $w$ erkennt.
Dabei müssen wir diese Strings in
Zusammenhang bringen mit einer Berechnung der Turingmaschine, die
auf $w$ im Zustand $q_{\text{accept}}$ endet.

Dazu können wir
die Berechnungsgeschichte verwenden.
Jede Konfiguration der
Turingmaschine ist ja ein String der Form $C=w_1qw_2$.
Die gesamte
Berechnung besteht aus einer Folge $C_i$ von solchen Konfigurationen.
Wir können sie protokollieren, indem wir alle $C_i$
mit Hilfe eines zusätzlichen Zeichens verketten,
welches nicht in $\Gamma$ vorkommt.
Wir bezeichnen diese Zeichen
mit {\tt\#}.

Die Grammatik soll jetzt also alle Strings ausser der Berechnungsgeschichte
erzeugen, die zum Erkennen des Wortes $w$ führen.
Diese Grammatik
erzeugt also genau dann alle Strings, wenn $M$ das Wort $w$ erkennt.

Um alle Strings ausser der erkennden Berechnung zu bekommen,
überlegen wir uns, wie ein String
$h={\tt\#}C_1{\tt\#}C_2{\tt\#}\dots{\tt\#}C_n{\tt\#}$
keine korrekte Berechungsgeschichte sein könnte:
\medskip
\begin{compactenum}
\item $C_1$ könnte nicht die Startkonfiguration sein, also
$C_1\ne q_0w$.
\item $C_n$ könnte nicht eine akzeptierende Konfiguration 
sein, also etwas von der Form $xq_{\text{accept}}y$.
\item Ein Zwischenschritt könnte nicht den Regeln der Turingmaschine
$M$ entsprechen.
\end{compactenum}
\medskip
Die gesuchte Grammatik muss alle Strings erzeugen, die einen dieser ``Fehler''
machen.
Falls es eine erkennende Berechnung für $w$ gibt, wird diese
die einzige sein, die nicht erzeugt wird.

Statt die Grammatik zu konstruieren, konstruieren wir einen Stackautomaten,
den wir später in eine Grammatik umwandeln können.
Der Stackautomat
muss alle Strings akzeptieren, die einen der Defekte 1-3 haben, wir
können also damit beginnen, nichtdeterministisch zwischen diesen drei
Fällen zu unterscheiden:
\[
\entrymodifiers={++[o][F]}
\xymatrix @+3mm{
*+\txt{}
	&q_1
\\
q_0	\ar[ur]^{\varepsilon,\varepsilon\to\varepsilon}
	\ar[r]^{\varepsilon,\varepsilon\to\varepsilon}
	\ar[dr]_{\varepsilon,\varepsilon\to\varepsilon}
	&q_2
\\
*+\txt{}
	&q_3
}
\]
Im Zustand $q_1$ müssen wir einen Stackautomaten anhängen, der
überprüft, ob im Input der String $q_0w$ steht.
Dies kann man 
bereits mit einem DEA machen, man braucht den Stack dafür gar nicht.
Ist $w=a_1a_2\dots a_k$, dann ist der Automat dafür geeignet
\[
\xymatrix @+3mm{
*++[o][F]{q_1}\ar[r]^{q_0,\varepsilon\to\varepsilon}
	&*++[o][F]{}\ar[r]^{a_1,\varepsilon\to\varepsilon}
		&*++[o][F]{}\ar[r]^{a_2,\varepsilon\to\varepsilon}
			&\dots\ar[r]^{a_{k-1},\varepsilon\to\varepsilon}
				&*++[o][F]{}\ar[r]^{a_k,\varepsilon\to\varepsilon}
					&*++[o][F]{}
}
\]
Akzeptieren darf er aber nur, wenn diese Regeln nicht eingehalten
werden, man muss also noch einen Fehlerzustand hinzufügen, der
alles akzeptiert, was in diesem Automaten nicht funktioniert.

Im Zustand $q_2$ müssen wir einen Stackautomaten anhängen, welcher
eine akzeptierende Konfiguration erkennt.
Er muss also zunächst 
alle $C_i$ mit $i<n$ überspringen und dann nichtdeterministisch
die Zeichen von $C_i$ lesen und kontrollieren, ob $q_{\text{accept}}$
darin vorkommt.
Ein Automat dafür ist 
\[
\entrymodifiers={++[o][F]}
\xymatrix @+3mm{
q_2\ar@(ur,ul)
	\ar[r]^{{\tt\#},\varepsilon\to\varepsilon}
	&{}\ar@(ur,ul)
		\ar[rr]^{q_{\text{accept}},\varepsilon\to\varepsilon}
		&*+\txt{}
			&{}\ar@(ur,ul)
				\ar[r]^{{\tt\#},\varepsilon\to\varepsilon}
				&*++[o][F]{}
}
\]
Darin enhalten die Schleifen beim zweiten und dritten Zustand keine
Übergänge mit dem Zeichen {\tt\#}.
Dazu braucht es zusätzliche
Übergänge, damit der Automat alles ausser diesen Strings akzeptiert.

Im Zustand $q_3$ muss ein Automat konstruiert werden, der erkennen kann,
ob die Folge $C_i{\tt\#}C_{i+1}$ einem korrekten Übergang zwischen
zwei Konfigurationen der Turingmaschen $M$ entspricht.
Dazu schreibt
der Stackautomat zunächst alle Zeichen von $C_i$ auf den
Stack, und vergleicht dann den Stackinhalt mit den Zeichen von
$C_{i+1}$.
Bis auf die Zeichen um die Kopfposition dürfen sich
die Strings nicht unterscheiden.
Dieser Automat enthält also
einerseits alle Übergänge, welche das Zeichen auf dem Stack
mit dem nächsten Inputzeichen vergleichen:
\[
\entrymodifiers={++[o][F]}
\xymatrix{
q\ar@(ur,dr)^{a,a\to\varepsilon}
}
\]
Ausserdem Übergänge, die zu gültigen
Turingmaschinen-Konfigurationsübergängen gehören.
Ist zum Beispiel
$px\to yq$ ein solcher Übergang, dann müssen folgende
Stackautomaten-Übergänge hinzugefügt werden:
\[
\entrymodifiers={++[o][F]}
\xymatrix{
{}\ar@/^/[r]^{y,p\to\varepsilon}
	&\ar@/^/[l]^{q,x\to\varepsilon}
}
\]
Leider funktioniert dies noch nicht ganz: die Zeichen von $C_i$
befinden sich auf dem Stack in umgekehrter Reihenfolge, passen
also nicht zu $C_{i+1}$.
Diesen Mangel kann man mit folgendem
Trick beheben: man schreibt in der Berechnungsgeschichte jede
zweite Konfiguration rückwärts, also
\[
{\tt\#}C_1{\tt\#}C_2^t{\tt\#}C_3{\tt\#}C_4^t{\tt\#}\dots
\]
Die Regeln für die korrekten Turingmaschinen-Übergänge müssen
natürlich an diese umgekehrte Ordnung angepasst werden, dabei
muss beachtet werden, dass mit jedem gelesenen {\tt\#}-Zeichen
die ``Leserichtung'' ändert.
Und wiederum muss der Automat alles
akzeptieren, was nicht den Turingmaschinen-Regeln entspricht.

Auf diese Weise kann ein Stackautomat konstruiert werden, der
alle Strings akzeptiert ausser der korrekten Berechnungsgeschichte
(mit alternierenden Schreibrichtungen geschrieben).
Die daraus
erzeugte Grammatik erzeugt alle Strings ausser der
korrekten Berechnungsgeschichte.
Damit ist die Reduktionsabbildung
konstruiert.
\end{proof}

\begin{satz}
\index{Gleichheitsproblem!für kontextfreie Grammatiken}%
Das Gleichheitsproblem für kontextfreie Grammatiken
\[
\text{\it EQ}_{\text{CFG}}=\{
\langle G_1,G_2\rangle\;|\;\text{$G_i$ sind kontextfreie Grammatiken
und $L(G_1)=L(G_2)$}
\}
\]
\index{EQCFG@$\textit{EQ}_{\text{CFG}}$}%
ist nicht entscheidbar.
\end{satz}

\begin{proof}[Beweis]
Wir konstruieren eine Reduktion
\[
\text{\it ALL}_{\text{CFG}}
\le
\text{\it EQ}_{\text{CFG}}.
\]
Da $\text{\it ALL}_{\text{CFG}}$ nicht entscheidbar ist, folgt
auch, dass $\text{\it EQ}_{\text{CFG}}$ nicht entscheidbar
ist.

Sei $G_0$ die Grammatik, die $\Sigma^*$ erzeugt.
Offenbar ist
$\langle G\rangle$ genau dann in $\text{ALL}_{\text{CFG}}$,
wenn $G$ und $G_0$ die gleiche Sprache erzeugen, also
$\langle G,G_0\rangle\in\text{\it EQ}_{\text{CFG}}$.
Die Reduktionsabbildung ist also
\[
\langle G\rangle \mapsto \langle G,G_0\rangle,
\]
dies ist sicher berechenbar.
\end{proof}


\section{Satz von Rice}
\rhead{Satz von Rice}
Eigenschaften von Turing-erkennbaren Sprachen sind im Allgemeinen
nicht entscheidbar, wie die vorangegangenen Beispiele zeigen.
Allerdings müssen sie genü\-gend kompliziert sein.
Dann lässt
sich aber sogar der Beweis vereinheitlichen.
Zum Beispiel wurde
beim Beweis von $\text{\it REGULAR}_{\text{TM}}$ eigentlich
nur verwendet, dass wir eine reguläre Sprache ($\emptyset$) und
eine nicht reguläre ($\{0^n1^n\;|\;n\in\mathbb N\}$) kennen.

\begin{definition}
\index{Eigenschaft!nicht-triviale}%
Sei $P$ eine Eigenschaft einer Sprache $L$, $P(L)$ ist also wahr oder
falsch.
$P$ heisst nicht trivial, falls es eine Sprache $L_1$ gibt,
die die Eigenschaft $P$ hat, und eine Sprache $L_2$, die die
Eigenschaft nicht hat.
\end{definition}

\begin{satz}[Rice]
\index{Rice!Satz von}%
\label{rice-theorem}
Sei $P$ eine nicht triviale Eigenschaft von Turing-erkennbaren Sprachen,
dann ist $P$ nicht entscheidbar.
Als Sprachproblem formuliert:
\[
P_{\text{TM}}=\{ \langle M\rangle\;|\;
\text{$M$ ist eine TM und $L(M)$ hat die Eigenschaft $P$}
\}
\]
ist nicht entscheidbar.
\end{satz}

\begin{proof}[Beweis]
Wir konstruieren eine Reduktion $A_\text{TM}\le P_{\text{TM}}$.
Zu einem Paar
$\langle M,w\rangle$
müssen wir also auf
berechenbare Weise eine Turingmaschine $M'$
\[
f\colon \langle M,w\rangle\mapsto M'
\]
konstruieren, deren Sprache genau dann die Eigenschaft $P$
hat, wenn $M$ das Wort $w$ akzeptiert.

Dazu brauchen wir die beiden Turing-erkennbaren Sprachen $L_1$ und $L_2$,
wobei $L_1$ die Eigenschaft $P$ hat, $L_2$ aber nicht.
Wir können dabei annehmen, dass $L_2=\emptyset$ ist, dass also
$\emptyset$ die Eigenschaft $P$ nicht hat.
Wäre dies nicht so, könnten
wir stattdessen $L_1=\emptyset$  wählen, und die Eigenschaft $\neg P$
untersuchen.
Ausserdem gibt es eine Turingmaschine $M_1$ mit $L(M_1)=L_1$.

Die Turingmaschine $M'$ operiert auf dem Input $u$ wie folgt:
\medskip
\begin{compactenum}
\item Lasse $M$ auf Input $w$ laufen
\item Falls $M$ das Wort $w$ akzeptiert, lasse $M_1$ auf $u$ laufen.
Falls $M_1$ im Zustand $q_{\text{accept}}$ anhält, halte ebenfalls
im Zustand $q_{\text{accept}}$ an.
\item In allen anderen Fällen: $q_{\text{reject}}$
\end{compactenum}
\medskip

Falls $w\in L(M)$ akzeptiert die Turingmaschine $M'$ genau die Wörter
$u\in L_1$.
Falls $w\not\in L(M)$ akzeptiert $M'$ überhaupt keine
Wörter, in diesem Fall ist also $L(M')=\emptyset$:
\begin{align*}
w\in L(M)&\qquad \Rightarrow\qquad L(M')=L_1\\
w\not\in L(M)&\qquad \Rightarrow\qquad L(M')=\emptyset
\end{align*}
Da $f$ berechnenbar ist, haben wir die verlangte Reduktion
gefunden, und es folgt, dass $P_{\text{TM}}$ nicht
entscheidbar ist.
\end{proof}

\subsubsection{Anwendungen}

\begin{beispiel}[\bf $\text{\textsl{ALL}}_{\text{TM}}$ ist nicht entscheidbar]
\index{ALLTM@$\textit{ALL}_{\text{TM}}$}%
$\text{\textsl{ALL}}_{\text{TM}}$ ist die Sprache
\[
\text{\textsl{ALL}}_{\text{TM}}=\{
\langle M\rangle\;|\; \text{$M$ ist eine TM und $L(M)=\Sigma^*$}
\}
\]
Der Satz von Rice verlangt, dass wir das Problem als eine Eigenschaft
der von einer TM erkannten Sprache formulieren.
\[
P=\text{\textsl{ALL}}:\text{Die erkannte Sprache ist $\Sigma^*$}
\]
Offensichtlich ist dies eine Eigenschaft der Sprache, und nicht zum
Beispiel der konkreten Implementation der TM.

Ausserdem verlangt der Satz von Rice, dass die Eigenschaft nicht trivial ist,
dass es also Sprachen gibt, die die Eigenschaft haben, und andere, die
sie nicht haben.
\begin{enumerate}
\item Die Sprache $\Sigma^*$ ist Turing-erkennbar und hat die Eigenschaft
$P=\text{\textsl{ALL}}$.
\item Die Sprache $\emptyset$ ist auch Turing-erkennbar, hat aber die
Eigenschaft
$P=\text{\textsl{ALL}}$ nicht.
\end{enumerate}
Damit ist klar, dass die Eigenschaft nicht trivial ist.

Jetzt kann der Satz von Rice angewendet werden, er besagt, dass die
Eigenschaft
$P=\text{\textsl{ALL}}$ nicht entscheidbar ist.
\end{beispiel}

\begin{beispiel}[\bf Primzahlprüfer] Es ist entscheidbar, ob eine Zahl $n$
prim ist oder nicht, man testet einfach jeden möglichen Teiler $<n$,
wenn immer ein Rest bleibt, ist die Zahl prim.
Dies ist natürlich
kaum der effizienteste Algorithmus, tatsächlich gibt es viele Alternativen,
die wir in eine Menge zusammenfassen können:
\[
\text{\textsl{PRIMALITY-TESTER}}=
\left\{\langle M\rangle\;\left|\;
\begin{minipage}{2.15truein}
\raggedright
$M$ ist eine Turingmaschine und\\
ein korrekter Primzahltester
\end{minipage}
\right.\right\}.
\]
\index{PRIMALITY-TESTER@$\textit{PRIMALITY-TESTER}$}%
\textsl{PRIMALITY-TESTER} enthält also genau diejenigen Programme, welche
als Primzahlprüfer korrekt funktionieren.

Gibt es ein Programm, welches beliebige Primzahlprüfer auf Korrektheit
testen kann? Wenn ja, dann ist \textsl{PRIMALITY-TESTER} entscheidbar.
Leider kann es kein solches Programm geben,
\textsl{PRIMALITY-TESTER} ist nicht entscheidbar, wie man mit dem Satz
von Rice einsehen kann.

Zunächst muss man wieder die Eigenschaft formulieren
\[
P: \text{Die akzeptierte Sprache ist die Menge der Primzahlen}
\]
Weiter muss es zwei Sprachen geben, die eine muss die Eigenschaft
$P$ haben, die andere nicht:
\begin{enumerate}
\item
Der oben beschriebene primitive Primzahlprüfer, der alle möglichen Teiler
durchprobiert, akzeptiert genau die Primzahlen, die akzeptierte Sprache
hat also die Eigenschaft $P$.
\item 
Die leere Sprache $\emptyset$ ist Turing-erkennbar, hat aber ganz bestimmt
die Eigenschaft $P$ nicht.
\end{enumerate}
Die Eigenschaft $P$ ist also nicht trivial, und nach dem Satz von Rice
kann sie daher auch nicht entscheidbar sein.
\end{beispiel}

Die Beispiele illustrieren, dass es im Allgmeinen unmöglich ist, Programme
automatisch darauf zu testen, ob sie in allen Fällen korrekt arbeiten
werden.
Dies ist nur dann möglich, wenn die akzeptierte Sprache einfach,
zum Beispiel regulär, ist, oder nur für eine Teilmenge aller möglichen
Programmen.

%
% komplexitaet.tex -- Kapitel ueber Komplexitaetstheorie
%
% (c) 2011 Prof Dr Andreas Mueller, Hochschule Rapperswil
%
\chapter{Komplexit"atstheorie\label{kapitel-komplexitaet}}
\lhead{Komplexit"atstheorie}
\rhead{}
Im Kapitel \ref{chapter-entscheidbarkeit} wurde untersucht, ob ein
Problem "uberhaupt mit einer Turingmaschine l"osbar ist. Die
Effizienz der gewonnenen Algorithmen spielte keine Rolle.
In diesem Kapitel soll die Laufzeit eines Algorithmus genauer
untersucht werden. Dabei sind nur Unterscheidungen interessant,
die unabh"angig von den speziellen F"ahigkeiten der verwendeten
Maschine sind, nur so lassen sich allgemeing"ultige Aussagen
ableiten, die f"ur jede Art von Computer gelten. Die absolute Laufzeit
ist daher kein brauchbares Kriterium, sie h"angt zu stark
von individuell verschiedenen Parametern wie Taktrate, Wortbreite,
Instruktionssatz etc.~ab. 

Sobald ein geeignetes Kriterium gefunden werden kann, kann man
die Sprachen in ``leicht'' und ``schwierig'' zu entscheidende 
unterteilen. Dabei kristallisiert sich eine Klasse von
Problemen heraus, die von nichtdeterministischen Turingmaschinen
gerade noch gel"ost werden k"onnen, die aber f"ur grosse
Probleme ausserhalb der Reichweite von deterministischen Turingmaschinen
sind. Es scheint, dass diese Kluft nicht "uberbr"uckt werden kann.

F"ur die Praxis bedeutet dies, dass einige Probleme mit effizienten
Algorithmen gel"ost werden k"onnen, w"ahrend f"ur andere Algorithmen
kein effizienten L"osungen m"oglich sind. Der Praktiker wird daher
das Problem einschr"anken m"ussen, denn oft lassen sich solche
Probleme unter zus"atzlichen Bedingungen effizient l"osen.
Voraussetzung ist nat"urlich, dass er Probleme diesen beiden
Klassen zuordnen kann.

\section{Laufzeitkomplexit"at}
\rhead{Laufzeitkomplexit"at}
\index{Laufzeitkomplexit\"at}
\subsection{Inputl"ange}
\index{Inputl\"ange}
Wenn die absolute Laufzeit eines Algorithmus nicht als Mass f"ur
die Komplexit"at eines Problems dienen kann, dann vielleicht
wenigstens die Abh"angigkeit der Laufzeit von der Gr"osse des
Problems. Da wir unter Problemen immer ihre "Ubersetzung in ein
Sprachproblem verstehen, haben wir f"ur die Gr"osse des Problems
ein Mass: die L"ange des Wortes, welches das Problem beschreibt.

Als Beispiel betrachten wir das Problem
\index{PRIME@\textsl{PRIME}}
\[
\text{\textsl{PRIME}}=\{ n\;|\; \text{$n\in\mathbb N$ und $n$ ist prim}\}.
\]
Es ist entscheidbar, man kann zum Beispiel alle Teiler durchtesten,
das ist mit $n$ Tests m"oglich. Ein Entscheider ist also eine Turingmaschine,
der man die Zahl $n$ auf das Band schreibt, und die Turingmaschine
stellt dann fest, ob die Zahl eine Primzahl ist. Die Zahl $n$ muss
als String auf das Band geschrieben werden. Dazu ist eine Codierung
zu w"ahlen, "ublicherweise wird dies eine Zifferndarstellung in
irgend einer Basis $b$ sein. Zur Darstellung der Zahl $n$ in Basis $b$
sind $l=\lfloor \log_bn\rfloor+1$ Zeichen notwendig.

Man k"onnte aber auch die un"are Darstellung verwenden, also so viele
$0$ auf das Band schreiben, wie die Zahl $n$ angibt. In dieser Codierung
ist also der Input wesentlich l"anger, n"amlich $l=n$.
F"ur die Beurteilung der
Komplexit"at kann also die Codierung des Inputs durchaus einen
merklichen Einfluss haben. Nehmen wir an, der Algorithmus hat
eine Laufzeit proportional zu $n$, dann w"achst die Laufzeit im
zweiten Fall proportional zur Inputl"ange $l$, im ersten Fall
aber w"achst die Laufzeit mit $b^l$. Man kann also Algorithmen
nur vergleichen, wenn man die gleiche Inputcodierung verwendet.

\subsection{Laufzeit}
\index{Laufzeit}
Sei jetzt also eine Turingmaschine $M$ gegeben, die ein Entscheider
ist. F"ur jedes Inputwort $w$ wird die Maschine eine gewisse Zeit
laufen und dann im Zustand $q_{\text{accept}}$ oder $q_{\text{reject}}$
anhalten. Die Anzahl der Rechenschritte, die die Maschine auf dem
Inputwort $w$ ben"otigt, bezeichnen wir mit $t(w)$.
Die maximale Laufzeit, die f"ur Inputw"orter der L"ange nicht
gr"osser als $n$ ben"otigt wird, bezeichnen wir mit $t(n)$,
\[
t(n)=\max \{ t(w)\;|\; |w|\le n\}.
\]
Der absolute Wert von $t(n)$ ist nicht interessant, da er sich
selbst unter trivialen Ver"anderungen der Turingmaschine ver"andern
kann. Das Verhalten von $t(n)$ f"ur grosse $n$ hingegen scheint
wesentlich robuster zu sein, es interessiert also nur noch, wie
sich verschiedene Funktionen zueinander verhalten, wenn man $n\to\infty$
streben l"asst. Falls wir die Turingmaschine angeben wollen, deren
Laufzeit bestimmt wird, schreiben wir sie als Index zu $t$: $t_M(n)$.

\begin{definition}
Sind $f$ und $g$ Funktionen $\mathbb N\to\mathbb R^+$, dann ist
$f(n)=O(g(n))$, falls es eine Konstante $c$ und ein $n_0\in\mathbb N$
gibt mit
\[
f(n)\le cg(n)\quad \forall n\in\mathbb N\text{ mit } n > n_0.
\]
Es ist $f(n)=o(g(n))$ wenn 
\begin{equation}
\lim_{n\to\infty}\frac{f(n)}{g(n)}=0.
\label{oquotientlimit}
\end{equation}
\end{definition}
\index{Laufzeit!polynomielle}
Man beachte, dass f"ur $f(n)=O(g(n))$ die Ungleichung $f(n)\le
cg(n)$ nicht f"ur alle $n$ gelten muss, insbesondere spielen kleine
Werte von $n$ keine Rolle, es interessiert uns nur das Verhalten
bei grossen Werten von $n$.
Und auch f"ur $f(n)=o(g(n))$ sind die kleinen Werte von $n$ nicht
von Bedeutung, die Funktionswerte f"ur kleine $n$ haben keinen
Einfluss auf den Grenzwert des Quotienten (\ref{oquotientlimit}).

Die Funktion $g$ muss offenbar nicht sonderlich genau bekannt
sein. Ist $g$ zum Beispiel in Polynom, interessiert nur noch
die h"ochste Potenz. Es gilt ja zum Beispiel
$ n^r\le n^k $ f"ur jeden Exponenten $r<k$, und damit
$n^r=O(n^k)$. Ausserdem ist $an^r=O(n^k)$, und damit
f"ur $f(n)=a_0+a_1n+a_2n^2+\dots+a_{k-1}n^{k-1}+a_nn^k$ auch
$ f(n)=O(n^k) $. Somit gen"ugt es bei Polynomen f"ur $g$ die
h"ochste Potenz anzugeben.

\subsection{Varianten von Turingmaschinen}
Wieviel schneller arbeitet eine Turingmaschine mit mehreren B"andern 
gegen"uber einer Standardturingmaschine? Wieviel schneller ist eine
nichtdeterministische Turingmaschine gegen"uber einer deterministischen?

\begin{satz}
\index{Turing-Maschine!mit mehreren B\"andern}
Eine Turingmaschine mit mehreren B"andern und Laufzeit $t(n)$ kann
in Laufzeit $O(t(n)^2)$ auf einer Standardturingmaschine
simuliert werden.
\end{satz}

\begin{proof}[Beweis]
In Satz \ref{mehrbandturingmaschine} wurde beschrieben, wie man eine
Turingmaschine mit mehreren B"andern auf einer Turingmaschine mit
einem einzigen Band simulieren kann. Dabei ist f"ur jeden der
$t(n)$ Berechnungsschritt ein Durchgang durch das Band n"otig, bei
dem die Turingmaschine die Inhalte der markierten Felder zusammensammelt,
um dann die notwendigen "Anderungen zu bestimmen. Die "Anderungen
auf dem Band einzutragen erfordert dann nochmals einen Durchgang
durch das Band.

Wir nehmen an, dass die $t(n)\ge n$, denn sonst k"onnte die
Turingmaschine nicht einmal ihren Input vollst"andig lesen.
Der tats"achlich benutzte Teil des Bandes kann dann nicht l"anger sein
als $t(n)$. Zusammen mit den $t(n)$ durchl"aufen erhalten wir,
dass die simulierte Turingmaschine die Laufzeit $O(t(n)^2)$ hat.
\end{proof}

Einen wesentlichen Unterschied der Laufzeit erwarten wir aber
bei nichtdeterministischen Turingmaschinen.
Die Berechnung in einer nichtdeterministischen Turingmaschine
kann ja auf ganz verschiedenen Wegen stattfinden, wovon
einer gefunden werden muss, der akzeptiert. Wenn eine simulierende
Turingmaschine alle Wege durchprobiert, spielt die Laufzeit auf
demjenigen Weg, der am l"angsten braucht, eine wesentliche Rolle.

\begin{definition}
\index{Laufzeit!einer nichtdeterministischen Turingmaschine}
Sei $N$ eine nicht deterministische Turingmaschine, die auch ein
Entscheider ist. Dann ist ihre Laufzeit $t(n)$ die maximale Anzahl
von Schritten, die jede m"ogliche Berechnung auf einem Input der
L"ange $\le n$ braucht.
\end{definition}

\begin{satz}
\label{exponentialtime}
Eine nichtdeterministische Turingmaschine mit Laufzeit $t(n)$
kann in Laufzeit $2^{O(t(n))}$
auf einer deterministischen Turingmaschine simuliert werden.
\end{satz}

\begin{proof}[Beweis]
In Satz \ref{nichtdeterministischeturingmaschine} wurde gezeigt,
wie man eine nichtdeterministische Turingmaschine auf einer
Mehrband-Maschine simulieren kann. Da wir im vorliegenden Fall
sogar einen Entscheider haben, k"onnen wir den Algorithmus noch
etwas vereinfachen. Wir verwenden Band~1 wieder f"ur den Input,
Band~2 als Arbeitsband f"ur die simulierte Turingmaschine und
Band~3 als Steuerband f"ur die nichtdeterministischen
Entscheidungen.

Wir m"ussen ausrechnen, wie sich die Simulation auf die 
Gesamtlaufzeit auswirkt. Jeder Berechnungspfad der simulierten
nichtdeterministischen Turingmaschine hat Laufzeit nicht
l"anger als $t(n)$. Die Erzeugung der n"achsten m"oglichen
Berechnungsgeschichte auf Band~3 kann in einem Durchgang
durch Band~3 erfolgen, braucht also Zeit $O(t(n))$.
Die Anzahl der Berechnungsgeschichten ist $|Q\times \Gamma|^{t(n)}$,
also von der Form $c^{t(n)}$. Im schlimmsten Fall ist die
Laufzeit dieses Algorithmus also
\[
c^{t(n)}(t(n) + O(t(n)))
=
2^{t(n) \log_2 c+ \log_2(O(t(n)))}
=
2^{O(t(n))}.
\]
\end{proof}
\subsection{Wie langsam ist exponentielle Laufzeit?}
\index{Laufzeit!exponentielle}
Satz \ref{exponentialtime} zeigt, dass zur Simulation eines
nichtdeterministischen Automaten eine exponentiell l"angere
Zeit n"otig ist, als die nichtdeterministische Maschine
ben"otigt h"atte. Aber wieviel langsamer ist das?
Zun"achst ist f"ur jeden Exponenten $k$
\[
\lim_{n\to\infty}\frac{2^{n}}{n^k}=\infty,
\]
$2^n$ w"achst also schneller als jedes Polynom. 
Setzen wir $t=n^r$, dann ist auch
\[
\lim_{n\to\infty}\frac{2^{n^r}}{n^k}
=
\lim_{t\to\infty}\frac{2^t}{t^{k/r}}
=
\infty,
\]
wie zu erwarten war, w"achst $2^{n^k}$ noch schneller.
Da nur schon f"ur das Lesen des Input $n$ Schritte notwendig sind,
spielt es gar keine Rolle, wie effizient der Algorithmus der
nichtdeterministischen Turingmaschine ist, es ist unm"oglich,
je polynomielle Laufzeit zu erreichen.

Aber wie gross ist denn nun exponentielle Laufzeit?
Nehmen wir an,
ein wir h"atten einen Algorithmus mit Laufzeit $n^k$. Vergr"ossert
man die Input-L"ange um $1$ Zeichen, "andert sich die Laufzeit um
\[
\frac{(n+1)^k}{n^k}
=
1+\frac{k}{n} +\dots,
\]
die relative Zunahme wird also immer kleiner. Bei einer Verdoppelung
der Inputl"ange wird die Laufzeit mit dem Faktor $2^k$ multipliziert.
Ganz anders bei exponentieller Laufzeit.
Erh"oht man die Inputl"ange um $1$, wird die Laufzeit mit einem
konstanten Faktor multipliziert. Und die Verdoppelung von $n$
quadriert die Laufzeit.
Ausgehend von einer angenommenen
Zeit pro Schritt von einer Nanosekunde, ergeben sich
f"ur die Laufzeit eines Algorithmus
mit polynomieller Laufzeit, genauer mit $n^5$ und exponentieller Laufzeit,
genauer mit $2^n$ die Werte in Tabelle~\ref{laufzeittabelle}.
\begin{table}
\begin{center}
\begin{tabular}{|l|rr|}
\hline
$n$&$n^5$&$2^n$\\
\hline
  1&             1ns&             2ns\\
  2&            32ns&             4ns\\
  4&       1.0$\mu$s&            16ns\\
  8&      32.7$\mu$s&           256ns\\
 16&           1.0ms&      65.5$\mu$s\\
 32&          33.5ms&           4.3ms\\
 64&            1.0s&       584 Jahre\\
128&           34.4s& $10^{21}$ Jahre\\
\hline
\end{tabular}
\end{center}
\caption{Polynomielle und exponentielle Laufzeit\label{laufzeittabelle}}
\end{table}
Die Zeit von $10^{21}$ Jahren entspricht dem 78 Milliardenfachen des
Alters des Universums. Grosse Probleme mit exponentieller Laufzeit
sind also schlicht unl"osbar, w"ahrend selbst relativ grosse
polynomielle Laufzeiten durchaus im Bereich des erreichbaren sind.
Der Unterschied zwischen polynomieller und exponentieller Laufzeit
ist also f"ur grosse Inputl"ange der Unterschied zwischen l"osbaren und
unl"osbaren Problemen.

\section{Klassen P und NP}
\index{Klasse P}
\rhead{Klassen P und NP}
\subsection{Klasse P}
Bei der Simulation einer Turingmaschine mit mehreren B"andern
auf einer Standardmaschine wird die Laufzeit im schlimmsten
Fall quadriert. Die Funktion $t_M(n)$ ist also keine Invariante
in der Menge der entscheidbaren Sprachen. F"ur das gleiche Problem kann
je nach eingesetztem Maschinentyp ein v"ollig anderes Laufzeitverhalten
beobachtet werden. 

Erhalten bleibt aber die Eigenschaft der Funktion $t_M(n)$, 
polynomiell zu sein. Simuliert man eine Turingmaschine $M_1$ mit
mehreren B"andern auf einer Standardturingmaschine $M_2$,
dann ist $t_{M_2}=O(t_{M_1}(n)^2)$.
Wenn also $t_{M_1}(n)=O(n^k)$ ist, dann ist $t_{M_2}(n)=O(n^{2k}).$
Beide Maschinen haben eine Laufzeit, die durch ein Polynom beschr"ankt
sind.

\begin{definition}
\index{Klasse P}
Die Klasse $P$ besteht aus den Sprachen, die mit einem Entscheider
mit polynomieller Laufzeit entschieden werden k"onnen.
\end{definition}

\subsection{Beispiele von Sprachen in P}
\index{Sprachen!kontextfreie}
\begin{satz}
Kontextfreie Sprachen sind in P.
\end{satz}

Der Satz besagt, dass es zu jeder kontextfreie Sprache eine
Turingmaschine gibt, die die Zugeh"origkeit eines Wortes zur
Sprache in einer Zeit entscheiden kann, die polynomiell ist in
der L"ange des Wortes.

\begin{proof}[Beweis]
Der CYK-Algorithmus aus Satz \ref{cyk-algorithm}
kann in Zeit $O(n^3)$ durchgef"uhrt werden.
\end{proof}

Sei $G$ ein gerichteter Graph und $s$ und $t$ zwei Vertizes
des Graphen. Das Pfad-Problem fragt, ob es einen von $s$ nach $t$
f"uhrend Pfad in dem Graphen gibt. "Ubersetzt in ein Spracheproblem
liefert es uns den folgenden Satz.

\begin{satz}
\index{PATH@\textsl{PATH}}
Die Sprache
\[
\text{\textsl{PATH}}
=\{
\langle G,s,t\rangle
\;|\;\text{$G$ ist ein gerichteter Graph mit einem Pfad von $s$ nach $t$}
\}
\]
ist in P.
\end{satz}

\begin{proof}[Beweis]
\index{Markierungsalgorithmus}
Ein Markierungsalgorithmus, der beginnend bei $s$ alle erreichbaren
Vertizes markiert, solange sich noch neue Vertizes markieren lassen,
hat polynomielle Laufzeit.
\end{proof}

\begin{satz}
\index{RELPRIM@\textsl{RELPRIME}}
Die Sprache
\[
\text{\textsl{RELPRIME}}
=\{
\langle a,b\rangle \;|\;
\text{$a, b\in \mathbb N$ und $a$ und $b$ sind teilerfremd}
\}
\]
ist in P.
\end{satz}

\begin{proof}[Beweis]
\index{Algorithmus!euklidischer}
Teilerfremdheit kann mit dem euklidischen Algorithmus entscheiden werden.
Dieser funktioniert wie folgt:
\begin{compactenum}
\item Wiederhole Schritte 2 und 3 bis $y=0$:
\item Weise $x$ den Wert $x\mod y$ zu
\item Vertausche $x$ und $y$
\item Gebe Resultat $x$ zur"uck.
\end{compactenum}
Bei diesem Algorithmus wird $x$ in jedem Schritt mindestens halbiert,
die Laufzeit ist also proportional zur gr"osseren der
beiden Zahlen $\log_2 a$ und $\log_2 b$, also ist die Laufzeit
polynomiell in der Inputl"ange $\log_2 a+\log_2 b$.
\end{proof}

\subsection{Verifizierer}
Sprachen, die von einer nichtdeterministischen Turingmaschine
in polynomieller Zeit entschieden werden, brauchen bei der
Simulation auf einer deterministischen Turingmaschine im
schlimmsten Fall exponentiell l"anger. Es ist also im
allgemeinen fast unm"oglich, ein solches Problem zu l"osen.

Gibt man allerdings die L"osung vor, zum Beispiel indem man
die richtigen nichtdeterministischen Entscheidungen vorgibt,
kann man die Probleml"osung in polynomieller Zeit nachvollziehen.
Genau diese Idee formalisiert das Konzept des Verifizierers.

\begin{definition}
\index{Verifizierer}
Ein Verifizierer f"ur die Sprache $A$ ist eine Turingmaschine
$V$ mit
\[
A=\{
w\;|\;\text{$\exists c\in C^*$ so dass $V$ $\langle w,c\rangle$ akzeptiert}
\},
\]
wobei $C$ eine endliche Menge ist.
Ein Verifizierer heisst polynomiell, wenn seine Laufzeit polynomiell
ist in der L"ange des Wortes $w$.
\end{definition}

Man beachte, dass die L"ange von $c$ auf die Laufzeit keinen
Einfluss haben darf. Der Verifizierer muss n"amlich gar nicht
den ganzen String $c$ anschauen, er muss davon nur soviel
nehmen, wie er f"ur die Verifikation braucht. Aber nat"urlich
muss er den ganzen String $w$ lesen, von dessen L"ange muss
die Laufzeit abh"angen.

Ein Verifizierer ist erst n"utzlich, wenn man auch $c\in C^*$
hat. Im Falle des Akzeptanzproblems durch eine nichtdeterministische
Turingmaschine (siehe Einf"uhrungsbeispiel) kann die Folge der
nichtdeterministischen Entscheidungen in der Berechnung diese
Funktion "ubernehmen. Daraus schliessen wir

\begin{satz} Eine Sprache wird genau dann von einer
nichtdeterministischen Turingmaschine in polynomieller Zeit entschieden,
wenn sie einen polynomiellen Verifizierer hat.
\end{satz}

\begin{proof}[Beweis]
Bereits gezeigt haben wir, dass eine nichtdeterministische in
polynomieller Zeit einen polynomiellen Verifizierer hat. Es ist also
nur noch zu kl"aren, dass auch das umgekehrt gilt, dass also eine
Sprache mit einem polynomiellen Verifizierer nichtdeterministisch
in polynomieller Zeit entschieden werden kann.

Dazu verwenden wir den folgenden Entscheidungsalgorithmus:
\begin{compactenum}
\item Erzeuge nicht deterministisch $c\in C^*$.
\item Teste, ob $V$ $\langle w,c\rangle$ akzeptiert.
\item falls $V$ akzeptiert: $q_{\text{accept}}$, falls nicht:
$q_{\text{reject}}$
\end{compactenum}
Die Laufzeit dieses Algorithmus wird bestimmt von der Laufzeit von $V$,
ist also polynomiell in der L"ange von $w$.
\end{proof}

\begin{beispiel}[Faktorisierung]
Die Faktorisierung eines Produktes von zwei grossen Primzahlen
gilt allgemein als schwierig. Die Sicherheit des RSA-Algorithmus,
der in zahlreichen im Internet verbreiteten kryptographischen
Protokollen verwendet wird, basiert wesentlich auf der Tatsache, dass
es sehr einfach ist, ein Produkt von zwei grossen Primzahlen zu
bilden, aber sehr viel schwieriger, aus dem Produkt die beiden
Faktoren wieder zu ermitteln.

Es geht also um die Sprache
\[
L
=
\{
n\;|\,\text{$n=pq$, wobei $p$ und $q$ Primzahlen sind}.
\}
\]
Erfahrungsgem"ass ist es praktisch unm"oglich, die Faktoren
einer Zahl in $n$ zu bestimmen, und damit zu entscheiden, ob
$n\in L$ ist. Kennt man jedoch einen der beiden Faktoren,
zum Beispiel $p$, dann kann man mit dem Divisionsalgorithmus
nachpr"ufen, ob die Division ``aufgeht'', was in einer Anzahl
von Schritten m"oglich ist, die proportional zur L"ange von 
$n$, also zu $\log_b n$ ist. Der Test auf Rest 0 bei Teilung
durch $p$ ist also ein polynomieller Verifizierer mit $c=p$.
\end{beispiel}

\subsection{Klasse NP}
Keine Variante der Turingmaschine macht aus einem Problem,
welches sich in einer Laufzeit $2^{O(n^k)}$ entscheiden
l"asst, ein Problem in $P$. Es ist daher zu vermuten,
dass die Sprachen, die nichtdeterministisch in polynomieller
Zeit entschieden werden k"onnen, eine wesentlich gr"ossere
Klasse bilden als $P$.

\begin{definition}
\index{Klasse NP}
Die Klasse der von einer nichtdeterministischen Turingmaschine
in polynomieller Zeit entscheidbaren Sprachen heisst NP.
\end{definition}

\begin{figure}
\begin{center}
\includegraphics{images/lang-5}
\end{center}
\caption{Jede Sprache in P ist auch in NP\label{psubsetnp}}
\end{figure}%

Insbesondere enth"alt NP alle Sprachen mit einem polynomiellen
Verifizierer. Es ist auch klar, dass die $\text{P}\subset\text{NP}$,
siehe Abbildung \ref{psubsetnp}.
Ein offenes Problem ist jedoch, ob $\text{P}\ne \text{NP}$.
Die Bedeutung dieser Frage wird sp"ater klar werden.

\subsection{Satisfiability: \textsl{SAT}}
\index{SAT}
\index{Satisfiability}
Die Sprache 
\[
\text{\textsl{SAT}}=\{\varphi\;|\;\text{$\varphi$ ist eine erf"ullbare logische Formel}\}
\]
ist entscheidbar. F"ur eine logische Formel $\varphi(x_1,\dots,x_n)$
testet man einfach alle m"oglichen Belegungen der Variablen $x_1,\dots,x_n$
mit Wahrheitswerten.
Daf"ur sind $2^n$ Verifikationen notwendig, dies ist also
bestimmt kein Algorithmus in P.

\textsl{SAT} ist aber in NP,
denn wir k"onnen einen polynomiellen Verifizierer angeben.
Der Verifizierer verlangt als L"osungszertifikate $c$ eine
Belegung der Variablen mit Wahrheitswerten $c=(x_1,\dots,x_n)$,
und "uberpr"uft, ob durch einsetzen der Werte $x_1,\dots,x_n$
die Formel $\varphi(x_1,\dots,x_n)$ wahr wird.

Eine Variante von \textsl{SAT} ist \textsl{3SAT}.
Im Gegensatz zu \textsl{SAT} enth"alt \textsl{3SAT}
nur Formeln in konjunktiver Normalform, und jede Klausel
enth"alt genau drei Terme (3cnf-Formel).
Eine typische Formel in \textsl{3SAT} ist 
\[
\varphi=(x_1\vee x_2\vee x_3)\wedge (\bar x_1\vee x_3\vee \bar x_4)\wedge
	(x_1\vee x_3\vee x_5).
\]
Die Sprache \textsl{3SAT} ist
\[
\text{\textsl{3SAT}} =\{\varphi\;|\; \text{$\varphi$ ist eine erf"ullbare 3cnf-Formel}\}.
\]
Wie \textsl{SAT} ist \textsl{3SAT} in NP.

\subsection{Existenz von Cliquen: $k$-\textsl{CLIQUE}}
\begin{figure}
\begin{center}
\includegraphics[width=0.6\hsize]{images/comp-1}
\end{center}
\caption{Graph mit sechs Knoten, keiner 4-Clique, aber
vier 3-Cliquen\label{6graph}}
\end{figure}%
\begin{figure}
\begin{center}
\begin{tabular}{cccc}
\includegraphics[width=0.45\hsize]{images/comp-2}&%
\includegraphics[width=0.45\hsize]{images/comp-3}\\
\includegraphics[width=0.45\hsize]{images/comp-4}&%
\includegraphics[width=0.45\hsize]{images/comp-5}
\end{tabular}
\end{center}
\caption{3-Cliquen des Graphen von Abbildung~\ref{6graph}}
\end{figure}%
\index{Clique}
\index{Clique@$k$-Clique}
\index{Cliquen-Problem}
\index{CLIQUE@\textsl{CLIQUE}}
Eine $k$-Clique in einem Graph $G$ ist eine Menge von $k$
Ecken des Graphen so, dass in $G$ jede Ecke der Teilmenge mit
jeder anderen Ecke verbunden ist. Im umgangssprachlichen Gebrauch
ist eine Clique eine Gruppe von Leuten, in der jeder jeden kennt.

Das Cliquen-Problem
\[
\text{$k$-\textsl{CLIQUE}} = \{ \langle G\rangle\;|\;
\text{$G$ ist eine Graph mit einer $k$-Clique}\}
\]
ist entscheidbar. Man probiert alle m"oglichen $k$-elementigen
Teilmengen der Ecken des Graphen durch, ob sie eine Clique
bilden. Da Zahl der Teilmengen ist $\binom{n}{k}$, wenn $n$
die Zahl der Ecken ist, und $\binom{n}{k}$ von der Gr"ossenordnung
$O(n^k)$ ist, ist dieser Algorithmus nicht in P.

Das Cliquen-Problem ist aber in NP. Dazu muss wieder ein
Verfizierer angegeben werden. Als L"osungszertifikat verlangt
der Verifizierer die Menge $c=\{v_1,\dots,v_k\}$ der
Vertizes, die angeblich eine $k$-Clique bilden. Dann testet
er jedes Paar von Vertices in $c$ daraufhin, ob sie in $G$
verbunden sind. Dies sind weniger als $k^2$ Tests die nicht
weniger Aufwand als die Gr"osse des Graphen brauchen, die
Komplexit"at dieses Algorithmus ist also $O(n)$, das
Cliquen-Problem ist in NP.

\subsection{F"arbeproblem: $k$-\textsl{VERTEX-COLORING}}
\index{F\"arbeproblem}
\begin{figure}
\begin{center}
\begin{tabular}{ccc}
\includegraphics[width=0.35\hsize]{images/comp-6}&
\qquad&\qquad
\includegraphics[width=0.354\hsize]{images/comp-7}
\end{tabular}
\end{center}
\caption{Zum F"arbeproblem: der Graph links kann mit drei Farben
eingef"arbt werden, der Graph rechts braucht vier verschiedene Farben
\label{vertex-coloring-examples}}
\end{figure}%
\index{VERTEX-COLORING@\textsl{VERTEX-COLORING}}
Man sagt, die Vertizes eines Graphen $G$ k"onnen mit $k$ Farben
eingef"arbt werden, wenn sich f"ur jeden Vertex eine der $k$ Farben
w"ahlen l"asst, so dass nie zwei durch eine Kante verbundene Vertizes
die gleiche Farbe bekommen.

Das F"arbe-Problem 
\[
\text{$k$-\textsl{VERTEX-COLORING}}
=
\{
\langle G\rangle\;|\;
\text{$G$ ist ein mit $k$ Farben einf"arbbarer Graph}\}
\}
\]
ist entscheidbar. Man kann alle $k^n$ m"oglichen F"arbungen
($n$ die Anzahl der Vertizes) durchtesten, ob sie die Bedingung
erf"ullen, dass verbundene Vertizes nicht die gleiche Farbe haben
d"urfen.

$k$-\textsl{VERTEX-COLORING} ist in NP, denn ein polynomieller Verifizierer
wird als L"osungszertifikat die Farbzuordnung $c=(c_1,\dots,c_n)$ der
Ecken $1,\dots,n$ des Graphen verlangen, und kann damit in polynomieller
Zeit pr"ufen, ob die Bedingung verschiedener Farbe an den Enden
jeder Kante erf"ullt ist.

\section{Reduktion}
\rhead{Reduktion}
Die Beispiele des vorangegangenen Abschnittes haben gezeigt,
dass der Nachweis, dass eine Sprache in P ist, mit grossem
Aufwand verbunden sein kann, weil ein geeigneter Algorithmus
gefunden werden muss. Ein "ahnliches Problem konnte in der
Entscheidungstheorie durch die Verwendung einer Reduktionsabbildung
$f\colon A\le B$
gel"ost werden. Eine solche musste berechenbar sein, und erlaubte
die Entscheidung, ob ein Wort in $A$ ist, in die Sprache
$B$ zu transportieren, und dort zu entscheiden.

Im Zusammenhang der Komplexit"at m"ochte man das ebenfalls
tun, doch muss die Reduktionsabbildung jetzt auch noch 
polynomielles Laufzeitverhalten haben, damit sie n"utzlich
bleibt.

\begin{definition}
\index{Reduktion}
Eine berechenbare Abbildung $f\colon \Sigma^*\to\Sigma^*$
mit den Eigenschaften
\begin{compactenum}
\item $w\in A\quad\Leftrightarrow\quad f(w)\in B$
\item Es gibt eine Turingmaschine mit polynomieller Laufzeit, die
$f(w)$ berechnet.
\end{compactenum}
heisst eine polynomielle Reduktion $A\le_P B$.
\end{definition}

\begin{satz}
\label{polynomiellreduction}
Ist $A\le_P B$ und $B$ in P, dann ist auch $A$ in P.
\end{satz}

\begin{proof}[Beweis]
Da $B\in\text{P}$ gibt es eine Turingmaschine $M$ mit polynomieller Laufzeit,
die $w\in B$ entscheiden kann. Der folgende Algorithmus entscheidet
jetzt auch $w\in A$:
\begin{compactenum}
\item Berechnen $f(w)$
\item Wende $M$ auf $f(w)$ an.
\end{compactenum}
Wir m"ussen uns nur noch versichern, dass dieser Algorithmus 
polynomielle Laufzeit hat. Die Laufzeit der Turingmaschine ist
$t_M(|f(w)|)$. Aber $|f(w)|$ kann nicht l"anger als die Laufzeit 
der Berechnung von $f(w)$ sein, man kann ja in jedem Schritt h"ochstens
ein Zeichen schreiben. Schreiben wir $t_f(n)$ f"ur die Laufzeit
der Berechnung von $f$, ist die Gesamtlaufzeit des Algorithmus
$t_M(t_f(|w|))$. Sowohl $t_M$ als auch $t_f$ sind Polynome, 
also ist auch die Zusammensetzung ein Polynom, der Algorithmus
hat somit polynomielle Laufzeit.
\end{proof}

\begin{satz}
Ist $A\le_P B$ und hat $B$ einen polynomiellen Verifizierer, dann
hat auch $A$ einen polynomiellen Verifizierer. Falls $A\le_P B$ und 
$B\in\text{NP}$, dann ist auch $A\in\text{NP}$.
\end{satz}

\begin{proof}[Beweis]
Aus einem polynomiellen Verifizierer $V$ f"ur $B$ kann man durch
Zusammensetzen mit der Reduktionsabbildung $f\colon A\le_P B$
einen polynomiellen Verifizierer f"ur $A$ konstruieren.
\end{proof}

Die letzten zwei S"atze zeigen, dass sich Sprachen in $P$ und $NP$
mit Hilfe der polynomiellen in ``leichte'' und ``schwierige'',
in ``schnell'' bzw.~``langsam'' zu l"osende unterteilen lassen
(Abbildung \ref{pnporder}).
\begin{figure}
\begin{center}
\includegraphics{images/lang-6}
\end{center}
\caption{Sprachen werden durch die polynomielle Reduktion $\le_P$
nach ``Schwierigkeit'' geordnet.\label{pnporder}}
\end{figure}%

\begin{beispiel}[\bf Stundenplan und das F"arbeproblem]
\index{Stundenplanproblem}
Das Studenplanproblem $S$ besteht darin
einen Stundenplan so zu erstellen, dass die Studenten alle
Vorlesungen besuchen k"onnen, f"ur die sie sich angemeldet haben.
Die Lektionen m"ussen dabei auf Zeitfenster verteilt werden,
so dass es keine Kollisionen gibt.

Um dieses Problem zu l"osen, muss man es auf ein bekanntes
Problem reduzieren. In diesem Fall bietet sich das Eckenf"arbeproblem
\textsl{VERTEX-COLORING}
f"ur Graphen an. Es wird verlangt, die Ecken eines Graphen mit 
verschiedenen Farben so einzuf"arben, dass keine zwei benachbarten
Ecken die gleiche Farbe haben.

Den verf"ugbaren Zeitfenstern f"ur Lektionen m"ussen die F"acher zugeordnet werden
so, dass kein Student zwei F"acher im gleich Zeitfenster
besuchen will.
Die Lektionen bilden als Ecken eines Graphen, zwischen zwei
Ecken gibt es eine Kante, wenn ein Student sich f"ur beide 
Lektionen angemeldet hat. Den Lektionen m"ussen jetzt Zeitfenster
zugeordnet werden, so dass es keine Kollisionen gibt, die
Zeitfenster sind also die Farben, mit denen die Ecken  des
Graphen eingef"arbt werden sollen. Die Einf"arbung ist genau
dann m"oglich, wenn es eine L"osung f"ur das Stundenplanproblem gibt.
Damit haben wir eine Abbildung konstruiert
\[
\begin{tabular}{>{$}r<{$}>{$}c<{$}>{$}l<{$}}
S&\to&\text{$k$-\textsl{VERTEX-COLORING}}\\
\text{Fach}&\mapsto&\text{Vertex}\\
\text{Zeitfenster}&\mapsto&\text{Farbe}\\
\text{Anzahl Zeitfenster}&\mapsto&k\\
\text{Student/Anmeldung}&\mapsto&\text{Kante}
\end{tabular}
\]
Diese Abbildung ist offenbar eine polynomielle Reduktion
\[
S\le_P \text{$k$-\textsl{VERTEX-COLORING}}.
\]
Die Reduktion kann aber auch in umgekehrter Richtung erfolgen,
es gibt also auch eine Reduktion
\[
\text{$k$-\textsl{VERTEX-COLORING}}\le_P S.
\]
\end{beispiel}

\begin{beispiel}[\bf\textsl{3SAT} und \textsl{CLIQUE}]
Eine $k$-Clique in einem Graphen ist ein vollst"andiger
Untergraph mit $k$ Ecken in einem gegebenen Graphen. Im Cliquen-Problem m"ussen
in einem gegebenen Graphen $k$ Ecken gefunden werden, so dass
jede m"ogliche Verbindung zwischen den Ecken auch im Graphen $G$ 
besteht. Als Sprache formuliert, ist es
\[
\text{\textsl{CLIQUE}}
=\{
\langle G,k\rangle\;|\;\text{$G$ ist ein Graph mit einer $k$-Clique}
\}
\]
Wir konstruieren jetzt eine Reduktion 
\[
\text{\textsl{3SAT}}\le_P
\text{\textsl{CLIQUE}}
\]
Man muss also aus jeder Formel in konjunktiver Normalform mit
drei Klauseln einen Graphen konstruieren, der genau dann eine
$k$-Clique besitzt, wenn die Formel erf"ullbar ist. Ausserdem
muss die Konstruktion in polynomieller Zeit durchf"uhrbar sein.

Sei also die Formel von der Form
\[
\varphi
=
\varphi_1\wedge\varphi_2\wedge\dots\wedge\varphi_k,
\]
wobei jede Teilformel $\varphi_i$ nur eine Diskjunktion (Oder-Verkn"upfung)
von Variablen oder negierten Variablen ist. Damit $\varphi$
erf"ullbar ist, muss es eine Zuordnung von Wahrheitswerten zu
den Variable geben, so dass jede der Teilformeln $\varphi_i$ wahr
wird. In jeder Teilformel gibt es also eine Variable oder 
negierte Variable, die wahr ist. Es w"urde also gen"ugen,
die Terme in den $\varphi_i$ zu finden, die alle gleichzeitig
wahr sein k"onnen.

Wir konstruieren jetzt den Graphen wie folgt. Die Ecken des Graphen
sind die Variablen der Teilformeln $\varphi_i$. Eine Kante wird
eingezeichnet f"ur jedes Paar von Variablen in verschiedenen
Teilformeln, die gleichzeitig wahr sein k"onnen. Die Formel $\varphi$
ist genau dann erf"ullbar, wenn der Graph eine $k$-Clique enth"alt.

\begin{figure}
\[
\entrymodifiers={++[o][F]}
\xymatrix{
*+\txt{}
	&{\bar x_1}
		\ar@{-}[dl] \ar@{-}[ddl] \ar@{-}[dddl]
		\ar@{-}[dddd] \ar@{-}[ddddr] 
		\ar@{-}[drrr] \ar@{-}[ddrrr] \ar@{-}[dddrrr]
		&{\bar x_2}
			\ar@{-}[dddll]
			\ar@{-}[ddddl] \ar@{-}[dddd] \ar@{-}[ddddr]
			\ar@{-}[drr] \ar@{-}[dddrr]
			&{x_2}
				\ar@{-}[dddlll] \ar@{-}[dddlll]
				\ar@{-}[dddd]
				\ar@{-}[dr] \ar@{-}[ddr]
\\
{\bar x_2}
	\ar@{-}[rrrr] \ar@{-}[rrrrdd]
	\ar@{-}[dddr] \ar@{-}[dddrr] \ar@{-}[dddrrr]
	&*+\txt{}
		&*+\txt{}
			&*+\txt{}
				&{\bar x_1}
					\ar@{-}[dllll] \ar@{-}[ddllll]
					\ar@{-}[dddlll] \ar@{-}[dddll]
\\
{\bar x_2}
	\ar@{-}[rrrrd]
	\ar@{-}[ddr] \ar@{-}[ddrr] \ar@{-}[ddrrr]
	&*+\txt{}
		&*+\txt{}
			&*+\txt{}
				&{x_2}
					\ar@{-}[dllll]
					\ar@{-}[ddl]
\\
{\bar x_1}
	\ar@{-}[rrrr]
	\ar@{-}[dr] \ar@{-}[drr]
	&*+\txt{}
		&*+\txt{}
			&*+\txt{}
				&{\bar x_2}
					\ar@{-}[dlll] \ar@{-}[dll] \ar@{-}[dl]
\\
*+\txt{}
	&{\bar x_2}
		&{\bar x_2}
			&{x_1}
}
\]
\caption{Graph zur Formel $\varphi$\label{phiformel}}
\end{figure}%
Als Beispiel f"ur die Konstruktion des Graphen nehmen wir die Formel
\[
\varphi
=
(\bar x_1\vee \bar x_2\vee x_2)
\wedge
(\bar x_1\vee x_2\vee \bar x_2)
\wedge
(x_1\vee \bar x_2\vee \bar x_2)
\wedge
(\bar x_1\vee \bar x_2\vee \bar x_2)
\]
Der nach obigen Regeln konstruierte Graph ist in Abbildung~\ref{phiformel}
dargestellt.
Nat"urlich enth"alt dieser Graph gen"ugend Kanten, so dass es relativ
leicht ist, eine $4$-Clique zu finden, zum Beispiel die in Abbildung~\ref{phiclique} dargestellte. Man kann daraus ablesen, dass die Formel $\varphi$
wahr wird, wenn $x_1$ und $x_2$ falsch sind.
\begin{figure}
\[
\entrymodifiers={++[o][F]}
\xymatrix{
*+\txt{}
	&{\bar x_1}
		&{\bar x_2}
			\ar@{-}[dddll]
			\ar@{-}[dddd]
			\ar@{-}[dddrr]
			&{x_2}
\\
{\bar x_2}
	&*+\txt{}
		&*+\txt{}
			&*+\txt{}
				&{\bar x_1}
\\
{\bar x_2}
	&*+\txt{}
		&*+\txt{}
			&*+\txt{}
				&{x_2}
\\
{\bar x_1}
	\ar@{-}[rrrr]
	\ar@{-}[drr]
	&*+\txt{}
		&*+\txt{}
			&*+\txt{}
				&{\bar x_2}
					\ar@{-}[dll]
\\
*+\txt{}
	&{\bar x_2}
		&{\bar x_2}
			&{x_1}
}
\]
\caption{$4$-Clique im Graph zur Formel $\varphi$\label{phiclique}}
\end{figure}%
\end{beispiel}

\section{NP-vollst"andige Probleme}
\rhead{NP-vollst"andige Probleme}
\begin{figure}
\begin{center}
\includegraphics{images/lang-4}
\end{center}
\caption{Beziehung zwischen P, NP und NP-vollst"andigen Problemen.
\label{pnpnpcomplete}}
\end{figure}%
Die polynomielle Reduktion ordnet die Sprachen nach ``Schwierigkeitsgrad''
(Abbildung~\ref{pnpnpcomplete}).
Je ``gr"osser'' eine Sprache $B$ ist, desto mehr Sprachen $A$ gibt es
mit $A\le_P B$. Findet man einen polynomiellen Algorithmus f"ur $B$
l"ost man damit automatisch auch $A$ f"ur alle diese $A$. 

M"ochte man $\text{P} = \text{NP}$ beweisen, dann muss man nach
m"oglichst ``schwierigen'' Problemen suchen, also nach solchen,
deren polynomielle L"osung die polynomielle L"osung vieler anderer
Problem nach sich ziehen w"urde. 

\begin{definition}
\index{NP-vollst\"andig}
Eine Sprache $B$ heisst NP-vollst"andig, wenn 
\begin{compactenum}
\item $B\in\text{NP}$
\item $A\le_P B$ f"ur alle $A\in\text{NP}$
\end{compactenum}
\end{definition}

Wenn man die polynomielle L"osung eines NP-vollst"andigen Problemes
findet, sind auch alle anderen Problem in NP in polynomieller Zeit
l"osbar:

\begin{satz}
Falls $B$ NP-vollst"andig ist und $B\in\text{P}$, dann ist
$\text{P}=\text{NP}$.
\end{satz}

\begin{proof}[Beweis]
Ist $A\in\text{NP}$, dann ist $A\le_P B$, weil $B$ NP-vollst"andig ist.
Nach Satz \ref{polynomiellreduction} ist dann aber auch
$A\in\text{P}$, alle Sprachen
in NP sind also auch in P, oder $\text{P}=\text{NP}$.
\end{proof}

\section{SAT}
\rhead{SAT}
\index{Cook, Steven}
\index{Levin, Leonid}
Bislang ist noch nicht klar, dass es "uberhaupt NP-vollst"andige
Problem gibt. Dies "anderte sich mit dem Satz von Cook und Levin,
der beweist, dass \textsl{SAT} NP-vollst"andig ist.

\begin{satz}[Cook-Levin]
\label{cooklevin}
\textsl{SAT} ist NP-vollst"andig.
\end{satz}

\begin{proof}[Beweis]
NP-vollst"andig heisst, dass jedes beliebige Problem in polynomieller
Zeit auf ein \textsl{SAT}-Problem reduziert werden kann.
Wir m"ussen
also einen Algorithmus angeben, mit dem aus einer Turingmachine
$M$ und einem Wort $w$
eine logische Formel $\varphi$ konstruiert werden kann, die genau
dann erf"ullbar ist, wenn die Turingmaschine $M$ das Wort $w$
akzeptieren kann.

Die nicht deterministische Turingmaschine $M$ hat polynomielle Laufzeit,
jede Berechnung auf Inputw"orter der L"ange $n$ ist einer Zeit $n^k$
abgeschlossen. In dieser Zeit kann die Maschine h"ochstens $n^k$ Felder
des Bandes beschreiben, es wird also h"ochstens ein Abschnitt der
L"ange $n^k$ des Bandes benutzt. Dabei werden maximal $n^k$
Konfigurationen durchlaufen. Schreibt man diese untereinander,
haben alle Konfigurationen in einem Quadrat $n^k\times n^k$
Platz.

Wir m"ochten jetzt eine Formel aufstellen, die genau dann erf"ullbar
ist, wenn sich in das Quadrat $n^k\times n^k$ die Bandsymbole und
Zust"ande so hineinschreiben lassen, dass die Konfigurationen eine
Abfolge beschreiben, die einer g"ultigen Berechnung entsprechen.

Die einzelnen Zellen $c_{ij}$ der Tabelle sind mit einer Zeilennummer $i$
und einer Spaltennummer $j$ indiziert, in jede Zelle kann
genau ein Zeichen geschrieben werden.
Zeichen k"onnen entweder Bandalphabetzeichen oder Zust"ande sein.
Der gr"osseren Einheitlichkeit wegen markieren wir die Enden des verwendeten
Bandabschnittes mit einem weiteren, bisher unbenutzten Zeichen {\tt\#}.
In einer Zelle finden wir also immer ein Zeichen aus $C=Q\cup \Gamma\cup\{\text{\tt\#}\}$.
Wir verwenden die logischen
Variablen $x_{ijs}$ mit $1\le i,j\le n^k$ und $s\in C$, die
genau dann wahr ist, wenn die Zelle $c_{ij}$ das Zeichen $s$ enth"alt.

In jede Zelle muss genau ein Zeichen geschrieben werden. Damit die Zelle
$c_{ij}$ ein Zeichen enth"alt, muss mindestens eine der Variablen $x_{ijs}$
wahr sein. Es d"urfen aber keine zwei Zeichen einer Zelle zugeordnet sein,
$x_{ijs}\wedge x_{ijt}$ f"ur zwei verschiedenen Zeichen $s$ und $t$ darf
also nicht wahr sein. Also muss jede Formel
$\neg(x_{ijs}\wedge x_{ijt})=\overline{x_{ijs}}\vee\overline{x_{ijt}}$
f"ur zwei verschiedene Zeichen wahr sein, zusammen mit der
Bedingung, dass ein Zeichen in der Zelle steht, erhalten wir
f"ur die Zellen $c_{ij}$ die Formel
\[
\varphi_{c_{ij}}=
\biggl(\bigvee_{s\in C} x_{ijs}\biggr)\wedge
\bigwedge_{s,t\in C\atop s\ne t} (\overline{x_{ijs}}\vee\overline{x_{ijt}}).
\]
Dies muss f"ur jede Indexkombination gelten, also
\[
\varphi_c
=
\bigwedge_{1\le i,j\le n^k}
\varphi_{c_{ij}}
=
\bigwedge_{1\le i,j\le n^k}\biggl(
\biggl(\bigvee_{s\in C} x_{ijs}\biggr)\wedge
\bigwedge_{s,t\in C\atop s\ne t} (\overline{x_{ijs}}\vee\overline{x_{ijt}})
\biggr).
\]
Da unsere Redukton polynomiell sein soll, m"ussen wir auch bestimmen,
wie lang die Formel $\varphi_c$ ist. Alle Formeln $\varphi_{c_{ij}}$
sind gleich gross, abh"angig nur von $|C|$, also einer Konstanten.
Damit ist die L"ange von $\varphi_c$ bestimmt durch die Gr"osse der
Tabelle, also $O(n^2k)$. Die zum Aufbau von $\varphi_c$ notwendig Zeit
ist ebenfalls $O(n^2k)$.

Als n"achstes dr"ucken wir in einer Formel $\varphi_i$ aus,
dass die Turingmaschine mit dem
Wort $w$ initialisiert worden ist. Dazu muss in irgend einer
Zelle der ersten Zeile das Zeichen $q_0$ stehen, und rechts
anschliessen die Zeichen des Wortes $w=a_1a_2\dots a_{|w|}$.
Steht $q_0$ in der
Zelle $j$, wird die Formel
\[
\varphi_{\text{start},j}
=
x_{11\#}\wedge
x_{12\blank}\wedge \dots \wedge
x_{1,j-1,\blank}\wedge
x_{1,j,q_0}\wedge
x_{1,j+1,a_1}\wedge\dots\wedge
x_{1,j+|w|,a_{|w|}}\wedge
x_{1,j+|w|+1,\blank}\wedge\dots\wedge
x_{1,n^k,\#}
\]
wahr.
Soll der Initialisierungsstring irgendwo in der ersten Zeile
stehen, wird eine der Formeln war, also
\[
\varphi_{\text{start}} = \bigvee_{1\le j\le n^k} \varphi_{\text{start}_j}.
\]
Auch diese Formel ist nicht gr"osser als $O(n^{2k})$ und kann in
polynomieller Zeit erzeugt werden.

Irgendwo in der Tabelle muss der Akzeptierzustand stehen, also
\[
\varphi_{\text{accept}} 
=
\bigvee_{1\le i,j\le n^k} x_{i,j,q_{\text{accept}}}.
\]

Die Konfigurationen in der Tabelle m"ussen alle durch
Turingmaschinen"uberg"ange auseinander hervorgehen. Ob dies
der Fall ist, kann durch die Betrachtung von zwei Zeilen
hohen und drei Zellen breiten Ausschnitten aus der Tabelle
"uberpr"uft werden. Solche Ausschnitte zeigen fast immer zwei
gleiche Zeilen, ausser an der aktuellen Kopfposition, wo
sich etwas "andern kann. Ein "Ubergang
\[
\entrymodifiers={++[o][F]}
\xymatrix{
p\ar[r]^{a\to b,R}
	&q
}
\]
wird zum Beispiel in den Ausschnitten
\[
\begin{tabular}{|c|c|c|}
\hline
$x$&$y$&$p$\\
\hline
$x$&$y$&$b$\\
\hline
\end{tabular}
\quad
\begin{tabular}{|c|c|c|}
\hline
$y$&$p$&$a$\\
\hline
$y$&$b$&$q$\\
\hline
\end{tabular}
\quad
\begin{tabular}{|c|c|c|}
\hline
$p$&$a$&$x$\\
\hline
$b$&$q$&$x$\\
\hline
\end{tabular}
\quad
\begin{tabular}{|c|c|c|}
\hline
$a$&$x$&$y$\\
\hline
$q$&$x$&$y$\\
\hline
\end{tabular}
\]
sichtbar. Ein "Ubergang mit einer Kopfbewegung nach links
\[
\entrymodifiers={++[o][F]}
\xymatrix{
p\ar[r]^{a\to b,L}
	&q
}
\]
dagegen in den Ausschnitten
\[
\begin{tabular}{|c|c|c|}
\hline
$z$&$x$&$y$\\
\hline
$z$&$x$&$q$\\
\hline
\end{tabular}
\quad
\begin{tabular}{|c|c|c|}
\hline
$x$&$y$&$p$\\
\hline
$x$&$q$&$y$\\
\hline
\end{tabular}
\quad
\begin{tabular}{|c|c|c|}
\hline
$y$&$p$&$a$\\
\hline
$q$&$y$&$b$\\
\hline
\end{tabular}
\quad
\begin{tabular}{|c|c|c|}
\hline
$p$&$a$&$x$\\
\hline
$y$&$b$&$x$\\
\hline
\end{tabular}
\quad
\begin{tabular}{|c|c|c|}
\hline
$a$&$x$&$y$\\
\hline
$b$&$x$&$y$\\
\hline
\end{tabular}
\]
Es gibt eine endliche Anzahl korrekter Belegungen solcher
$2\times 3$-Ausschnitte mit den Zeichen aus $C$, die Anzahl
h"angt nur von $|C|$ und den "Uberg"angen der Turingmaschine ab.
Die Belegung
\[
\begin{tabular}{|c|c|c|}
\hline
$a_1$&$a_2$&$a_3$\\
\hline
$a_4$&$a_5$&$a_6$\\
\hline
\end{tabular}
\]
eines Ausschnitts mit der Zelle $c_{ij}$ in der
linken oberen Ecke entspricht der Formel
\[
x_{i,j,a_1}\wedge
x_{i+1,j,a_2}\wedge
x_{i+2,j,a_3}\wedge
x_{i,j+1,a_4}\wedge
x_{i+1,j+1,a_5}\wedge
x_{i+2,j+1,a_6}.
\]
Davon muss eine wahr sein, also
\[
\varphi_{\text{ausschnitt}_{ij}}
=
\bigvee_{\text{$a_1,\dots,a_6$ korrekt}}
x_{i,j,a_1}\wedge
x_{i+1,j,a_2}\wedge
x_{i+2,j,a_3}\wedge
x_{i,j+1,a_4}\wedge
x_{i+1,j+1,a_5}\wedge
x_{i+2,j+1,a_6}.
\]
Ausserdem muss jeder Ausschnitt g"ultig belegt sein, also
muss
\[
\varphi_{\text{ausschnitt}} =\bigwedge_{1\le i,j\le n^k}\varphi_{\text{ausschnit}_{ij}}
\]
wahr sein. Auch diese Formel l"asst sich in polynomieller Zeit konstruieren
und sie hat polynomielle L"ange.

Damit die Tabelle eine akzeptierende Berechnung beschreibt, m"uss
alle Teile wahr sein, also ist
\[
\varphi =
\varphi_{c}\wedge
\varphi_{\text{start}}\wedge
\varphi_{\text{accept}}\wedge
\varphi_{\text{ausschnitt}}
\]
die gesuchte Formel, die genau dann erf"ullbar ist, wenn $w$ von der
Turingmaschine $M$ akzeptiert werden kann.
\end{proof}

\section{Beispiele NP-vollst"andiger Probleme}
\rhead{NP-vollst"andige Probleme}
Viele praktisch wichtige Probleme sind NP-vollst"andig. Der Nachweis
der NP-Vollst"andigkeit ist jedoch nicht immer einfach. Im Allgemeinen
wird er dadurch gef"uhrt, dass eine Reduktion von einem bereits als
NP-vollst"andig bekannten Problem konstruiert wird. Es lohnt sich
daher, einen m"oglichst grossen Katalog von NP-vollst"andigen
Problemen kennen zu lernen, einerseits um "Ubung im Konstruieren
von Reduktionen zu erhalten, andererseits um eine gr"ossere Auswahl 
von Problemen zu bekommen, von denen ausgehend eine Reduktion 
konstruiert werden k"onnte. In diesem Kapitel werden einige
solche Probleme vollst"andig untersucht.
\subsection{3SAT}
\index{3SAT@\textsl{3SAT}}
\begin{satz}
$\text{\textsl{3SAT}}$ ist NP-vollst"andig.
\end{satz}

\begin{proof}[Beweis]
Zum Beweis muss eine polynonmielle Reduktion
$\text{\textsl{SAT}}\le_P\text{\textsl{3SAT}}$ konstruiert werden.
Eine Reduktionsabbilung
$\text{\textsl{SAT}}\to\text{\textsl{3SAT}}$ muss eine allgemeine Formel
$\varphi$ in eine gleichwertige Formel in konjunktiver Normalform umwandeln, 
die ausserdem nur noch maximal drei Variablen in jeder Klausel enth"alt.

Ein erster Versuch verwendet die Distributivgesetze f"ur die 
logischen Operationen, also
\begin{align*}
a\wedge(b\vee c)&=(a\wedge b)\vee(a\wedge c)\\
a\vee(b\wedge c)&=(a\vee b)\wedge(a\vee c),
\end{align*}
um die $\wedge$ nach ``aussen'' und die $\vee$ nach ``innen'' zu
bringen. Aus der Formel 
\begin{equation}
(x_1\wedge y_1)
\vee
(x_2\wedge y_2)
\vee
\dots
\vee
(x_n\wedge y_n)
\label{exponentialformula}
\end{equation}
wird, wie schon in (\ref{bigcnf}) bemerkt, eine Konjunktion von $2^n$
Termen der Form
\[
(u_1\vee u_2\vee \dots u_n)
\]
wobei $u_i=x_i$ oder $u_i=y_i$ sein kann. 
Die entstehende Konjunktion hat also exponentiell viele Terme, insbesondere
ist es nicht m"oglich, auf diesem Weg eine Reduktion
$f\colon\text{\textsl{3SAT}}\to\text{\textsl{SAT}}$ zu konstruieren.

Die gesuchte ``gleichwertige'' Formel muss nur im Sinne des
\textsl{SAT}-Problems gleichwertig sein, sie muss nicht eine
logisch "aquivalente Formel sein. Es reicht, wenn die transformierte
Formel genau dann erf"ullbar ist, wenn auch die urspr"ungliche Formel
erf"ullbar ist. Das bedeutet auch, dass wir neue Variablen einf"uhren
d"urfen.

Das Problem in Formel (\ref{exponentialformula}) entstand dadurch, dass
beim ``ausmultiplizieren'' sehr viele Kombinationen entstehen. H"atte
man eine Abk"urzung $z_i=x_i\wedge y_i$, w"urde die Formel viel
k"urzer, n"amlich nur
\[
z_1\vee z_2\vee\dots\vee z_n.
\]
Diese Formel w"are sogar in konjunktiver Normalform, da man sie
als Formel mit nur einer Klausel betrachten kann.
Die Bedingung, dass man mit $z_i$ eigentlich $x_i\wedge y_i$ 
meint, kann man auch als 
\[
z_i\vee \neg(x_i\wedge y_i)
=
z_i\vee \bar x_i\vee\bar y_i
\]
formulieren. Die "Ubersetzung der Formel (\ref{exponentialformula})
in konjunktive Normalform ist dann
\begin{equation}
(z_1\vee z_2\vee\dots\vee z_n)
\wedge
\bigwedge_{i=1}^n (z_i\vee \bar x_i\vee\bar y_i).
\label{equivalentcnfformula}
\end{equation}
In (\ref{equivalentcnfformula}) kommen $4n$ Variablen vor,
in der urspr"unglichen Formel
nur $2n$, die Formel wird also um den Faktor $2$ l"anger, die
Vergr"osserung ist jetzt linear, nicht mehr exponentiell.

Es gen"ugt also Regeln anzugeben, mit denen man Konjunktionen
``nach aussen'' bringen kann, so dass die Erf"ullbarkeit der
Formel erhalten bleibt, und so dass die L"ange der Formel 
h"ochstens polynomiell zunimmt. F"ur die Formel
\[
\varphi=
(
\varphi_{11}
\wedge
\varphi_{12}
\wedge
\dots
\wedge
\varphi_{1n_1}
)
\vee
(
\varphi_{21}
\wedge
\varphi_{22}
\wedge
\dots
\wedge
\varphi_{2n_2}
)
\vee\dots\vee
(
\varphi_{k1}
\wedge
\varphi_{k2}
\wedge
\dots
\wedge
\varphi_{kn_k}
)
\]
verwendet man wieder Variablen $z_i$, welche daf"ur die einzelnen
Klammerausdr"ucke stehen. Um auszudr"ucken, dass diese wahr sein
sollen, wenn die Klammerausdr"ucke wahr sind, f"ugt man
die Bedingungen
\[
z_i\vee \neg
(
\varphi_{i1}
\wedge
\varphi_{i2}
\wedge
\dots
\wedge
\varphi_{in_i}
)
=
z_i\vee
\neg\varphi_{i1}
\vee
\neg\varphi_{i2}
\vee
\dots
\vee
\neg\varphi_{in_i}
\]
hinzu. So erh"alt man die Formel
\[
(z_1\vee z_2\vee \dots\vee z_k)
\wedge
\bigwedge_{i=1}^k \biggl(z_i\vee \bigvee_{j=1}^{n_i} \neg\varphi_{ij}\biggr).
\]
Ihre L"ange ist $O(k+\sum_{j=1}^k(1+n_j))=O(|\varphi|)$.
Durch wiederholte Anwendung dieser Methode kann die
Formel also in polynomieller Zeit in konjunktive Normalform gebracht
werden.

Die inzwischen erreichte konjunktive Normalform der Formel kann durchaus
noch mehrere Variablen pro Klausel enthalten. Sei 
\[
(x_1\vee x_2\vee \dots\vee x_n)
\]
eine solche Klausel. Wir m"ussen sie ersetze durch eine
Formel in konjunktiver Normalform mit drei Variablen pro Klausel.
Dazu kann man den folgenden Trick verwenden.
Die Formel $(\varphi\vee\psi)$ wird wahr, wenn eine der Unterformeln
$\varphi$ oder $\psi$ war ist. Sie wird nicht wahr sein, wenn 
beide Unterformeln nicht wahr sind. Sei $z$ eine neue Variable,
wir m"ochten mit ihr ausdr"ucken, dass $\psi$ wahr ist. Wir
verlangen $\bar z \vee \psi$, wenn $\psi$ nicht wahr ist, darf auch $z$
nicht wahr sein. Dann ist die Formel
\[
(\varphi \vee z)\wedge(\bar z\vee \psi)
\]
genau dann erf"ullbar, wenn $\varphi\vee \psi$ erf"ullbar ist.
Ist $\varphi$ nicht erf"ullbar, dann muss $z$ wahr sein, und
damit auch $\varphi$, d.\,h.~auch $\varphi\vee\psi$ ist erf"ullbar.
Eine zu lange Disjunktion kann also immer in k"urzere Disjunktionen 
aufgespaltet werden. Rekursive Anwendung liefert jetzt eine Formel
mit mit maximal drei Literalen pro Klausel.
\end{proof}

\begin{satz}
$\text{\textsl{CLIQUE}}$ ist NP-vollst"andig.
\end{satz}

\begin{proof}[Beweis]
Da wir bereits fr"uher gezeigt haben, dass
$\text{\textsl{3SAT}}\le_P\text{\textsl{CLIQUE}}$ ist, folgt jetzt auch,
dass $\text{\textsl{CLIQUE}}$ NP-vollst"andig ist.
\end{proof}

\subsection{Hamiltonscher Pfad}
Ein Hamiltonscher Pfad in einem gerichteten Graphen ist ein Pfad, der jeden
Vertex des Graphen genau einmal besucht. Als Sprachproblem formuliert 
\[
\text{\textsl{HAMPATH}}
=\{\langle G\rangle\,|\,\text{$G$ ist ein Graph mit einem hamiltonschen Pfad}\}
\]
Es ist leicht, zu verifizieren, ob
ein Pfad ein Hamiltonscher Pfad ist:
\begin{satz} $\text{\textsl{HAMPATH}}\in\text{NP}$.
\end{satz}

\begin{proof}[Beweis]
Als Zertifikat $c$ verwendet man den Pfad. Seine L"ange ist geringer
als die L"ange der Beschreibung des Graphen selbst.
Die Punkte des Pfades m"ussen benachbart sein, dies kann man in 
$O(n)$ testen.
Dann muss man "uberpr"ufen, ob jeder Vertex des Graphen vorkommt,
dies ist in Zeit $O(n^2)$ m"oglich. Auf dem Pfad darf kein Vertex
zweimal vorkommen, auch dies kann man in $O(n^2)$ verifizieren.
Insgesamt kann man also in polynomieller Zeit nachpr"ufen, ob ein
Pfad tats"achlich ein hamiltonscher Pfad ist.
\end{proof}

\begin{satz} \textsl{HAMPATH} ist NP-vollst"andig.
\end{satz}

\begin{proof}[Beweis]
Da bereits bekannt ist dass \textsl{HAMPATH} in NP ist, muss nur
noch eine Reduktion von einem bekanntermassen NP-vollst"andigen
Problem auf \textsl{HAMPATH} konstruiert werden. Die hier vorgestellte
ingeni"ose Konstruktion reduziert \textsl{3SAT} auf \textsl{HAMPATH}.
Dazu muss zu jeder \textsl{3CNF}-Formel ein gerichteter Graph angegeben
werden, der genau dann einen hamiltonschen Pfad besitzt, wenn die
Formel erf"ullbar ist.

Sei also $\varphi$ eine \textsl{3CNF}-Formel mit $k$ Klauseln in den Variablen
$x_1,\dots,x_n$. Der zu konstruierende Graph muss zu jeder Variablen
$x_i$ einen Teilgraphen enthalten, mit dem ausserdem ausgedr"uckt
werden kann, ob die Variable wahr oder falsch ist. Der Graph
\[
\entrymodifiers={++[o][F]}
\xymatrix{
*+\txt{}
	&*+\txt{}
		&*+\txt{}
			&*+\txt{}
				&{} \ar@/_10pt/[ddllll]\ar@/^10pt/[ddrrrr]
					&*+\txt{}
						&*+\txt{}
							&*+\txt{}
								&*+\txt{}
\\
*+\txt{}
\\
{}\ar@/^/[r] \ar@/_10pt/[ddrrrr]
	&{}\ar@/^/[r] \ar@/^/[l]
		&{}\ar@/^/[r] \ar@/^/[l]
			&{}\ar@/^/[r] \ar@/^/[l]
				&*+\txt{\dots}\ar@/^/[r] \ar@/^/[l]
					&{}\ar@/^/[r] \ar@/^/[l]
						&{}\ar@/^/[r] \ar@/^/[l]
							&{}\ar@/^/[r] \ar@/^/[l]
								&{} \ar@/^/[l]\ar@/^10pt/[ddllll]
\\	
*+\txt{}
\\
*+\txt{}
	&*+\txt{}
		&*+\txt{}
			&*+\txt{}
				&{}
					&*+\txt{}
						&*+\txt{}
							&*+\txt{}
								&*+\txt{}
}
\]
hat genau zwei hamiltonsche Pfade, die die ``Perlenkette'' in der
Mitte in entgegengesetzter Richtung durchlaufen.
Wir konstruieren einen Graphen, der
genau $n$ solche Teilgraphen untereinander enth"alt. Der $i$-te solche
Teilgraph steht f"ur die Variable $x_i$. Der Durchlaufsinn der ``Perlenkette''
von links nach rechts bedeutet, dass diese Variable wahr sein soll,
der Durchlaufsinn von rechts nach links bedeutet, dass sie falsch ist.

Die Formel $\varphi$ besteht aus $k$ Klauseln, hat also die
Form $c_1\wedge c_2\wedge\dots\wedge c_k$. Wir wollen die Tatsache,
dass Klausel $c_j$ wahr ist, dadurch ausdr"ucken, dass der Pfad einen
Vertex $c_j$ besucht. $c_j$ kann von der Variablen $x_i$ wahr gemacht
werden wird dadurch ausgedr"uckt, dass der Pfad beim Durchlaufen
der ``Perlenkette'' von Variable $x_i$ einen ``Abstecher'' zu $c_j$
machen kann. Jedem Paar von Vertizes in der ``Perlenkette'' ordnen
wir daher eine Klausel zu. Falls $x_i$ in der Klausel $c_j$
vorkommte, f"ugen wir zwischen den beiden der Klausel $c_j$
zugeordneten Vertizes zus"atzliche Kanten nach $c_j$ und wieder
zur"uck hinzu.
\[
\entrymodifiers={++[o][F]}
\xymatrix{
*+\txt{}
	&*+\txt{}
		&*+\txt{}
			&*+\txt{}
				&*+\txt{}
					&*+\txt{}
						&*+\txt{}
							&*+\txt{}
								&{c_j} \ar@/_15pt/[ddllll]
\\
*+\txt{}
	&*+\txt{}\ar[dl]
		&*+\txt{}
			&*+\txt{}
				&*+\txt{}
					&*+\txt{}
						&*+\txt{}\ar[dr]
							&*+\txt{} 
\\
{}\ar@/^/[r] \ar[dr]
	&{}\ar@/^/[r] \ar@/^/[l]
		&*+\txt{\dots}\ar@/^/[r] \ar@/^/[l]
			&{}\ar@/^/[r] \ar@/^/[l] \ar@/^15pt/[uurrrrr]
				&{}\ar@/^/[r] \ar@/^/[l]
					&*+\txt{\dots}\ar@/^/[r] \ar@/^/[l]
						&{}\ar@/^/[r] \ar@/^/[l]
							&{} \ar@/^/[l]\ar[dl]
\\
*+\txt{}
	&*+\txt{}
		&*+\txt{}
			&*+\txt{}
				\&*+\txt{}
					&*+\txt{}
						&*+\txt{}
							&*+\txt{}
}
\]
Enth"alt die Klausel $c_j$ die Variable $\bar x_i$ wird der
``Abstecher'' zu $c_j$ in die Durchlaufrichtung von rechts nach
links eingebaut.

Die Auswahl von Wahrheitswerten f"ur die Variablen $x_i$ entspricht
der Entscheidung, in welcher Richtung die ``Perlenketten'' durchlaufen
werden. Die Formel ist offenbar genau dann erf"ullbar, wenn ein
Abstecher zu jeder Klausel $c_j$ m"oglich ist, oder wenn es einen
hamiltoschen Pfad gibt.

Die Konstruktion erzeugt einen Graphen mit $O(nk)$ Vertizes,
ist also sicher in polynomialer Zeit durchf"uhrbar. Also ist
$\text{\textsl{3SAT}}\le_P\text{\textsl{HAMPATH}}$.
\end{proof}

\subsection{SUBSET-SUM}
\index{SUBSET-SUM@\textsl{SUBSET-SUM}}
Das Problem \textsl{SUBSET-SUM} ist auch bekannt als das Rucksack-Problem.
Gegeben ist eine Menge $S$ von ganzen Zahlen, kann man darin eine
Teilmenge finden, die als Summe einen bestimmte Wert $t$ hat?
Als Sprache formuliert:
\[
\text{\textsl{SUBSET-SUM}}
=\left\{
\langle S,t\rangle\,\left|\,
\begin{minipage}{3truein}
$S$ eine Menge von ganzen Zahlen, in der es eine Teilmenge
$T\subset S$ gibt mit
\[
\sum_{x\in T}x=t.
\]
\end{minipage}\right.
\right\}
\]
Der Name ``Rucksack''-Problem r"uhrt daher, dass man sich 
die Zahlen aus $S$ als ``Gr"osse'' von Gegenst"anden vorstellt, und
wissen m"ochte, ob man einen Rucksack der Gr"osse $t$ exakt f"ullen
kann mit einer Teilmenge von Gegenst"anden aus $S$.

\begin{satz}\textsl{SUBSET-SUM} ist NP-vollst"andig
\end{satz}

\begin{proof}[Beweis]
Es ist ziemlich klar, dass $\text{\textsl{SUBSET-SUM}}\in\operatorname{NP}$,
es ist also nur noch eine Reduktion zu konstruieren, wir w"ahlen
$\text{\textsl{3SAT}}\le_P\text{\textsl{SUBSET-SUM}}$.

Zu einer \textsl{3CNF}-Formel $\varphi$ mit Variablen $x_1,\dots,x_l$
muss eine Menge von Zahlen $S$ und eine Zahl $t$ gefunden werden,
die genau dann eine Teilmenge mit Summe $t$ hat, wenn $\varphi$
erf"ullbar ist.

\begin{table}
\begin{center}
\begin{tabular}{|c|ccccc|}
\hline
Zahl&1&2&3&\dots&l\\
\hline
$y_1$&1&0&0&\dots&0\\
$z_1$&1&0&0&\dots&0\\
$y_2$& &1&0&\dots&0\\
$z_2$& &1&0&\dots&0\\
$y_3$& & &1&\dots&0\\
$z_3$& & &1&\dots&0\\
\vdots&& & &     & \\
$y_l$& & & &\dots&1\\
$z_l$& & & &\dots&1\\
\hline
$t$&1&1&1&\dots&1\\
\hline
\end{tabular}
\end{center}
\caption{Abbildung der Variablen $x_i$ und $\bar x_i$ in Zahlen
eines \textsl{SUBSET-SUM}-Problems\label{subsetsumnumbers}}
\end{table}

Wir konstruieren Zahlen $y_i$ und $z_i$, die jeweils ausdr"ucken
ob $x_i$ wahr oder falsch ist. Wir bauen die Zahlen in einer
Stellendarstellung zu einer sp"ater zu bestimmenden gen"ugend
grossen Basis auf. Sofern die Zahlen nur wenige Stellen haben, die 
von $0$ verschieden sind, gibt es bei der Addition keinen "Ubertrag.
Die Zahlen $y_i$ und $z_i$ m"ussen so sein, dass nur jeweils eine
in der Summe auftreten kann. Dies erreichen wir, wenn wir
sie gem"ass Tabelle \ref{subsetsumnumbers} w"ahlen.
Die Auswahl einer Teilmenge mit Summe $t$ ist gleichbedeutend
mit der Wahl einer Belegung der Variablen $x_i$ mit Wahrheitswerten.
\begin{table}
\begin{center}
\begin{tabular}{|c|ccccc|cccc|}
\hline
Zahl&1&2&3&\dots&l&$c_1$&$c_2$&\dots&$c_k$\\
\hline
$y_1$&1&0&0&\dots&0&1&0&\dots&0\\
$z_1$&1&0&0&\dots&0&0&0&\dots&0\\
$y_2$& &1&0&\dots&0&0&1&\dots&0\\
$z_2$& &1&0&\dots&0&1&0&\dots&0\\
$y_3$& & &1&\dots&0&1&1&\dots&0\\
$z_3$& & &1&\dots&0&0&0&\dots&0\\
\vdots&& & &     & & & &     & \\
$y_l$& & & &     &1&0&1&\dots&1\\
$z_l$& & & &     &1&0&0&\dots&0\\
\hline
$g_1$& & & &     & &1&0&\dots&0\\
$h_1$& & & &     & &1&0&\dots&0\\
$g_2$& & & &     & & &1&\dots&0\\
$h_2$& & & &     & & &1&\dots&0\\
\vdots&& & &     & & & &     & \\
$g_k$& & & &     & & & &     &1\\
$h_k$& & & &     & & & &     &1\\
\hline
$t$  &1&1&1&\dots&1&3&3&\dots&3\\
\hline
\end{tabular}
\end{center}
\caption{Reduktion von \textsl{3SAT} auf
\textsl{SUBSET-SUM}\label{subsetsumtable}}
\end{table}

Jetzt muss noch codiert werden, in welchen Klauseln von $\varphi$ 
die Variablen $x_i$ auftreten. Dazu f"ugen wir der Tabelle f"ur jede
Klausel $c_i$ eine weiter Spalte hinzu, und tragen eine $1$ ein
in der Zeile von $y_i$ wenn $x_i$ in der Klausel $c_j$ vorkommt.
Falls $\bar x_i$ in $c_j$ vorkommt tragen wir eine $1$ in der Zeile
von $z_i$ ein.

Die Formel ist erf"ullbar, wenn es eine Belegung mit Wahrheitswerten
gibt, die jede Klausel wahr macht. Dabei k"onnen auch mehrere Literale
in einer Klausel war werden, die Summe der Spalte einer wahren Klausel kann
also Werte zwischen 1 und 3 annehmen. Da die Summe $t$ exakt erreicht
werden muss, f"uhren wir f"ur jede Klausel zus"atzliche Zahlen $g_j$ und
$h_j$ hinzu, die in der Spalte der Klausel $c_j$ eine $1$ haben und sonst
nur $0$. Indem man zur Teilmenge falls n"otig $g_j$ und $h_j$ hinzunimmt,
kann man erreichen, dass die Summe in der Spalte von Klausel $c_j$ immer
$3$ ergibt.
Die Tabelle \ref{subsetsumtable} zeigt die konstruierten Zahlen
f"ur die Formel
\[
(x_1\vee \bar x_2\vee x_3)\wedge(x_2\vee x_3\vee x_l)\wedge \dots\wedge
(\dots\vee \bar x_3).
\]
Da die gr"osste vorkommende Ziffer eine $3$ ist, kann man die
Zahlen aus $S$  und $t$ im $4$-er-System aus der Tabelle ablesen.

Der Zeitaufwand f"ur die Erstellung der Tabelle \ref{subsetsumtable}
ist $O((k+l)^2)$, also sicher polynomiell in der L"ange der
Formel $\varphi$.

Damit ist 
$\text{\textsl{3SAT}}\le_P\text{\textsl{SUBSET-SUM}}$ gezeigt.
\end{proof}

%\subsection{Vertex Coloring}
%Das Eckenf"arbeproblem wurde bereits verwendet und seine "Aquivalenz
%mit dem Stundenplanproblem gezeigt. Es ist auch einfach zu verifizieren,
%dass \textsl{VERTEX-COLORING} in NP ist. Daher ist es nicht "uberraschend,
%dass es auch NP-vollst"andig ist.
%
%\begin{satz}
%$\text{\textsl{3SAT}}\le_P\text{\textsl{VERTEX-COLORING}}$,
%insbesondere ist \textsl{VERTEX-COLORING} NP-vollst"andig.
%\end{satz}
%
%\begin{proof}[Beweis]
%Eine Reduktion
%$\text{\textsl{3SAT}}\le_P\text{\textsl{VERTEX-COLORING}}$
%ordnet einer \textsl{3CNF}-Formel $\varphi$ einen Graphen $G$
%und eine Zahl $k$ zu, der genau dann mit $k$ Farben eingef"arbt
%werden kann, wenn $\varphi$ erf"ullbar ist.
%
%Wir verwenden $k=2$, die Bedeutung der zwei Farben soll wahr
%oder falsch sein. Die Formel $\varphi$ besteht aus $l$ verschiedenen
%Klauseln, wir schreiben 
%\[
%\varphi = c_1\wedge c_1\wedge \dots\wedge c_l.
%\]
%Dem Graphen $G$ f"ugen wir einen Knoten f"ur jede Klausel hinzu,
%sowie einen Knoten $o$. Jede Klausel wird mit dem Knoten $o$ verbunden.
%Eine Einf"arbung bedeutet, dass alle Klauseln mit der gleichen Farbe
%eingef"arbt werden m"ussen, dies interpretieren wir als Ausdruck
%der Tatsache, dass in einer erf"ullbaren Formel die Wahrheitswerte
%so gew"ahlt werden k"onnen, dass alle Klauseln wahr werden.
%\[
%\xymatrix{
%*+\txt{}
%	&*+\txt{}
%		&*+[o][F-]{o}\ar@{-}[dll] \ar@{-}[dl] \ar@{-}[d] \ar@{-}[drr]
%\\
%*+[o][F-]{c_1}
%	&*+[o][F-]{c_2}
%		&*+[o][F-]{c_3}
%			&\dots
%				&*+[o][F-]{c_l}
%}
%\]
%
%Die Klauseln sind alle von der Form
%\[
%c_i=(x_1\vee x_2\vee x_3)
%=\neg(\bar x_1\wedge \bar x_2 \wedge \bar x_3).
%\]
%Damit sie wahr wird, muss $\bar x_1\wedeg \bar x_2\wedge \bar x_3$
%falsch sein.
%\end{proof}

\section{Karps Katalog NP-vollst"andiger Probleme}
\rhead{Karp's Liste}
\index{Karp, Richard}
\index{Karp's Liste}
Schon kurz nach der Ver"offentlichung des Beweises des Satzes
\ref{cooklevin} hat Richard Karp bei einen Baum von Reduktionen
ver"offentlicht.
Selbstverst"andlich l"asst
sich jedes dieser Probleme auf jedes andere reduzieren. Zum Beispiel
hatten wir \textsl{3SAT} auf \textsl{CLIQUE} reduziert, w"ahrend
Karp direkt von \textsl{SAT} ausgeht. Karps Reduktionsbaum
beginnt wie folgt.
\[
\xymatrix{
{}
	&\text{\textsl{SAT}} \ar[dl] \ar[d] \ar[dr]
\\
\text{\textsl{CLIQUE}} \ar[d] \ar[dr]
	&\text{\textsl{BIP}}
		&\text{\textsl{3SAT}} \ar[d]
\\
\text{\textsl{VERTEX-COVER}}
	&\text{\textsl{SET-PACKING}}
		&\text{\textsl{VERTEX-COLORING}}
}
\]
Weiter oben haben wir \textsl{CLIQUE} von \textsl{3SAT} aus
reduziert. \textsl{VERTEX-COLORING} haben wir im Zusammenhang
mit dem Stundenplanproblem getroffen.

\index{BIP@\textsl{BIP}}
\textsl{BIP} ist ``binary integer programming'', zu einer ganzzahligen
Matrix $C$ und einem ganzzahligen Vektor $d$ ist ein bin"arer
Vektor $x$ zu finden mit $Cx=d$.

\begin{satz}
\textsl{BIP} ist NP-vollst"andig.
\end{satz}

\begin{proof}[Beweis]
Es ist klar, dass \textsl{BIP} in NP ist. Es gen"ugt daher, eine
Reduktion
\[
\text{\textsl{SUBSET-SUM}}\le_P\text{\textsl{BIP}}
\]
zu konstruieren.

Dazu muss zu einer Menge $S$ von ganzen Zahlen und einer Summe $t$
eine Matrix und ein Vektor konstruiert werden. Als Matrix nehmen wir
ein Matrix mit einer Zeile, in der die Elemente von $S$ stehen. Als
Vektor $d$ nehmen wir den Vektor mit der einen Komponenten $t$.
Einen bin"aren Vektor $x$ finden mit $Cx=d$ ist jetzt gleichbedeutend
damit, eine Auswahl von Zahlen in $S$ zu finden, deren Summe $t$ ist.
Also ist
$\text{\textsl{SUBSET-SUM}}\le_P\text{\textsl{BIP}}$.
\end{proof}

Im folgenden wollen die von Karp als NP-vollst"andig erkannten Probleme
ohne Beweis im Sinne einer Referenz zusammenstellen. Hat man ein
Problem als NP-vollst"andig nachzuweisen, kann man von jedem dieser
Probleme aus eine Reduktion versuchen.

F"ur die Abh"angigkeiten unterhalb von \textsl{VERTEX-COVER} gibt 
Karp folgenden Baum
\[
\xymatrix{
	&\text{\textsl{VERTEX-COVER}} \ar[dl] \ar[d]\ar[dr]\ar[drr]
\\
\begin{minipage}{1.0truein}
\textsl{FEEDBACK-NODE-SET}
\end{minipage}
	&\begin{minipage}{1.0truein}\textsl{FEEDBACK-ARC-SET}\end{minipage}
		&\text{\textsl{HAMPATH}}\ar[d]
			&\text{\textsl{SET-COVERING}}
\\
	&
		&\text{\textsl{UHAMPATH}}
}
\]
\textsl{UHAMPATH} ist das Problem in einem ungerichteten Graphen
einen hamiltonschen Pfad zu finden. Die anderen Probleme sind wie
folgt definiert:
\begin{description}
\index{VERTEX-COVER@\textsl{VERTEX-COVER}}
\item[\textsl{VERTEX-COVER}:] Gegeben ein Graph $G$ und eine Zahl
$k$, gibt es eine Teilmenge von $k$ Vertizes so, dass jede
Kante des Graphen ein Ende in dieser Teilmenge hat?
\index{FEEDBACK-NODE-SET@\textsl{FEEDBACK-NODE-SET}}
\item[\textsl{FEEDBACK-NODE-SET}:] Gegeben ein gerichteter Graph $G$
und eine Zahl $k$, gibt es eine endliche Teilmenge von $k$ Vertizes
von $G$ so, dass jeder Zyklus in $G$ einen Vertex in der Teilmenge 
enth"alt?
\index{FEEDBACK-ARC-SET@\textsl{FEEDBACK-ARC-SET}}
\item[\textsl{FEEDBACK-ARC-SET}:] Gegeben ein gerichteter Graph $G$
und eine Zahl $k$, gibt es eine Teilmenge von $k$ Kanten so, dass
jeder Zyklus in $G$ eine Kante aus der Teilmenge enth"alt?
\index{SET-COVERING@\textsl{SET-COVERING}}
\item[\textsl{SET-COVERING}:] Gegeben eine endliche Familie endlicher
Mengen $(S_j)_{1\le j\le n}$ und eine Zahl $k$, gibt es eine Unterfamilie
bestehend aus $k$ Mengen, die die gleiche Vereinigung hat?
\end{description}
Die Abh"angigkeiten unter \textsl{VERTEX-COLORING} sind etwas vielf"altiger:
\[
\xymatrix{
	&\begin{minipage}{1truein}
	\textsl{VERTEX-COLORING}
	\end{minipage} \ar[d]\ar[dr]
\\
	&\begin{minipage}{0.6truein}
	\textsl{EXACT-COVER}
	\end{minipage} \ar[dl] \ar[d] \ar[dr] \ar[drr]
		&\begin{minipage}{0.8truein}
		\textsl{CLIQUE-COVER}
		\end{minipage} 
\\
\text{\textsl{3D-MATCHING}}
	&\text{\textsl{SUBSET-SUM}} \ar[dr] \ar[d]
		&\begin{minipage}{0.8truein}
		\textsl{HITTING-SET}
		\end{minipage}
			&\begin{minipage}{0.8truein}
			\textsl{STEINER-TREE}
			\end{minipage}
\\
	&\text{\textsl{SEQUENCING}}
		&\text{\textsl{PARTITION}} \ar[d]
\\
	&
		&\text{\textsl{MAX-CUT}}
}
\]
\begin{description}
\item[\textsl{EXACT-COVER}] Gegeben eine Familie $(S_j)_{1\le j\le n}$
von Teilmengen
einer Menge $U$ gibt es eine Unterfamilie von Mengen, die disjunkt sind,
aber die gleiche Vereinigung haben?
Die Unterfamilie $(S_{j_i})_{1\le i\le m}$ muss also
$S_{j_i}\cap S_{j_k}=\emptyset$ und 
\[
\bigcup_{j=1}^n S_j=\bigcup_{i=1}^mS_{j_i}
\]
erf"ullen.
\index{CLIQUE-COVER@\textsl{CLIQUE-COVER}}
\item[\textsl{CLIQUE-COVER}:] Gegeben ein Graph $G$ und eine positive Zahl
$k$, gibt es $k$ Cliquen so, dass jede Ecke in einer der Cliquen
vorhanden ist?
\index{3D-MATCHING@\textsl{3D-MATCHING}}
\item[\textsl{3D-MATCHING}:] Sei $T$ eine endliche Menge und $U$ eine 
Menge von Tripeln aus $T$: $U\subset T\times T\times T$. Gibt es eine
Teilmenge $W\subset U$ so, dass $|W|=|T|$ und keine zwei Elemente
von $W$ stimmen in irgend einer Koordinate "uberein?
\index{HITTING-SET@\textsl{HITTING-SET}}
\item[\textsl{HITTING-SET}:] Gegeben eine Menge von Teilmengen $S_i\subset S$,
gibt es eine Menge $H$, die jede Menge in genau einem Punkt trifft, also
$|H\cap S_i|=1\forall i$?
\index{STEINER-TREE@\textsl{STEINER-TREE}}
\item[\textsl{STEINER-TREE}:]
Gegeben ein Graph $G$, eine Teilmenge $R$ von Vertizes, und eine
Gewichtsfunktion $w\colon E\to\mathbb Z$ und eine postive Zahl $k$, gibt es
einen Baum mit Gewicht $\le k$, dessen Knoten in $R$ enthalten sind?
Das Gewicht des Baumes ist die Summe der Gewichte 
$w(\{u,v\})$ "uber alle Kanten $\{u,v\}$ im Baum.
\index{SEQUENCING@\textsl{SEQUENCING}}
\item[\textsl{SEQUENCING}:] Gegeben sei ein Vektor
$(t_1,\dots,t_p)\in\mathbb Z^p$
von Laufzeiten von $p$ Jobs, ein Vektor von sp"atesten Ausf"uhrungszeiten 
$(d_1,\dots,d_p)\in\mathbb Z^p$, einem Strafenvektor 
$(s_1,\dots,s_p)\in\mathbb Z^p$ und eine positive ganze Zahl $k$.
Gibt es eine Permutation $\pi$ der Zahlen $1,\dots,p$, so dass
die Gesamtstrafe f"ur versp"atete Ausf"uhrung bei der Ausf"uhrung der Jobs
in der Reihenfolge $\pi(1),\dots,\pi(p)$ nicht gr"osser ist als $k$? 
Formal lautet die Bedingung
\[
\sum_{j=1}^p
\vartheta(t_{\pi(1)} +\dots +t_{\pi(j)} - d_{\pi(j)}) s_{\pi(j)} \le k,
\]
darin ist $\vartheta$ die Stufenfunktion definiert durch
\[
\vartheta(x)=\begin{cases}
1&x\ge 0\\
0&x<0.
\end{cases}
\]
\index{PARTITION@\textsl{PARTITION}}
\item[\textsl{PARTITION}:] Gegeben ein Folge von $s$ ganzen Zahlen
$c_1,c_2,\dots,c_s$, kann man die Indizes $1,2,\dots,s$ in zwei
Teilmengen $I$ und $\bar I$ teilen, so dass die Summe der zugeh"origen
$c_i$ identisch ist:
\[
\sum_{i\in I}c_i=\sum_{i\not\in I}c_i
\]
\index{MAX-CUT@\textsl{MAX-CUT}}
\item[\textsl{MAX-CUT}:] Gegeben ein Graph $G$ mit einer Gewichtsfunktion
$w\colon E\to\mathbb Z$ und eine ganze Zahl $W$. Gibt es eine Teilmenge
$S$ der Vertizes, so dass das Gesamtgewicht der Kanten, die $S$ mit
seinem Komplement verbinden, mindestens so gross ist wie $W$:
\[
\sum_{\{u,v\}\in E\wedge u\in S\wedge v\not\in S} w(\{u,v\})\ge W
\]
\end{description}

\section{Schaltungen, Minesweeper und \textsl{SAT}}
\rhead{Minesweeper}
\index{Minesweeper}
\subsection{Schaltungen}
\begin{figure}
\begin{center}
\includegraphics{images/mine-1}
\end{center}
\caption{Grundgatter AND, OR und NOT\label{gatter}}
\end{figure}%
\begin{figure}
\begin{center}
\includegraphics{images/mine-2}
\end{center}
\caption{Umsetzung der Formel
$(\bar x\wedge y)\vee(x\wedge \bar z)$ mit Gattern aus der
Abbildung~\ref{gatter}\label{gatterformel}}
\end{figure}%
\index{Gatter}
\index{AND}
\index{OR}
\index{NOT}
In der Computertechnik lernt man, dass sich moderne Computer mit
Hilfe von logischen Grundschaltungen, den Gattern, realisieren lassen.
Abbildung~\ref{gatter} zeigt die Schaltsymbole der Grundgatter AND, OR
und NOT. Wir gehen hier von idealen Gattern aus, deren Ausg"ange beliebig
viele Eing"ange anderer Gatter treiben k"onnen. Mit solche Gattern
lassen sich beliebig komplexe logische Formeln umsetzen. Zum Beispiel
zeigt Abbildung~\ref{gatterformel}, wie die Formel
$(\bar x\wedge y)\vee(x\wedge \bar z)$ umgesetzt werden kann.

\index{CURCUIT}
Die Tatsache, dass sich jede Formel in eine "aquivalente Schaltung mit Gattern
"ubersetzen l"asst kann man so formulieren: Es gibt ein Sprache
\[
\text{\textsl{CIRCUIT}}=\{
C\;|\; \text{$C$ ist eine Schaltung, deren Ausgang wahr werden kann}\}.
\}
\]
Ausserdem gibt es eine polynomielle Reduktion
$\text{\textsl{SAT}}\le_P\text{\textsl{CIRCUIT}}$. Da wir bereits wissen,
dass \textsl{SAT} NP-vollst"andig ist, schliessen wir, dass es ein schwieriges
Problem ist, einer logischen Schaltung anzusehen, ob ihr Ausgang "uberhaupt
je wahr werden kann.

\subsection{Minesweeper}
\index{Minesweeper}
Beim Spiel Mine-Sweeper, wird dem Spieler von einigen der noch nicht
aufgedeckten Felder die Anzahl benachbarter Fehler angezeigt, unter
denen sich eine Bombe befindet. Der Spieler soll dann nur diejenigen
Felder betreten, unter denen sich keine Bombe versteckt, und alle
Felder markieren, unter denen eine Bombe liegt. Betrachten Sie das
Problem {\it MINE-SWEEPER-CONSISTENCY}, in dem zu einer Belegung der
Felder mit Zahlen (der Anzahl Bomben auf Nachbarfeldern) und eventuell
auch einigen bereits platzierten Bomben entscheiden
werden muss, ob diese Zahlen konsistent sind, ob also Bomben so
auf dem Spielfeld verteilt werden k"onnen, dass die Zahlen stimmen.
Es ist ziemlich klar, dass
\[
\text{\textsl{MINE-SWEEPER-CONSISTENCY}}\in \text{NP}.
\]
Als L"osungszertifikat brauchen wir die Verteilung der Bomben.
Zur Verifikation m"ussen wir dann die Anzahl der Bomben auf den
Nachbarfeldern z"ahlen, was in Laufzeit $O(n^2)$ m"oglich ist,
wenn $n$ die L"ange der l"angeren Seite des Spielfeldes ist.
Also hat
\textsl{MINE-SWEEPER-CONSISTENCY} einen polynomiellen Verifizierer und
ist damit in NP.

\subsection{\"Aquivalenz von \textsl{SAT} und \textsl{MINESWEEPER-CONSISTENCY}}
Da \textsl{SAT} NP-vollst"andig ist, l"asst sich jedes NP-Problem auf
\textsl{SAT} reduzieren. Der Beweis des Satzes von Cook-Levin gibt
auch eine Konstruktion, die aber nicht sehr intuitiv ist.
Daher ist die Frage berechtigt, ob man auf etwas intuitivere Art einsehen
kann, dass
\textsl{SAT}
und
\textsl{MINESWEEPER-CONSISTENCY}
"aquivalent sind.

Wir k"onnten dies zum Beispiel dadurch tun, dass wir eine polynomielle
Reduktion von \textsl{SAT} auf \textsl{MINESWEEPER-CONSISTENCY}
konstruieren.
Zu einer beliebigen Formel $\varphi$ m"ussen wir ein Minesweeper-Spielfeld
finden, welches sich genau dann konsistent mit Bomben f"ullen l"asst,
wenn die Formel erf"ullbar ist.

Wir konstruieren eine solche Reduktion in zwei Schritten. Zun"achst ist
klar, dass wir jede Formel in eine Schaltung umwandeln k"onnen, und
dass dies in polynomieller Zeit geht. Wir haben also auf jeden
Fall eine polynomielle Reduktion $\text{\textsl{SAT}}\le_P\text{\textsl{CIRCUIT}}$.
Insbesondere sind wir fertig, wenn wir eine Schaltung in ein Minesweeper-%
Problem "ubersetzen k"onnen.

Gesucht ist jetzt also eine Reduktion 
\[
\text{\textsl{CIRCUIT}} \le_P \text{\textsl{MINESWEEPER-CONSISTENCY}}.
\]
Aus Abbildung~\ref{gatterformel} k"onnen wir sehen, was dazu alles
"ubersetzt werden k"onnen muss. Leitungen, Verzweigungen, und die Grundgatter
m"ussen alle in geeignete Zahlemuster auf einem Minesweeper-Spielfeld
"ubersetzt werden, so dass sich genau dann konsistente Bomben
finden lassen, wenn der Ausgang der Schaltung wahr werden kann.

Auf das OR-Gatter kann genau genommen verzichtet werden, denn eine
ODER-Verkn"upfung kann man nach den de Morganschen Regeln auch durch
Negationen und UND-Verkn"upfungen ausdr"ucken:
\[
x\vee y = \overline{(\bar x \wedge \bar y)}.
\]
Somit bleibt "ubrig, dass wir jeden Schaltplan aus AND- und NOT-Gattern
in einen Minesweeper-Spielplan "ubersetzen k"onnen m"ussen. Dazu geh"oren
auch Verbindungen zwischen verschiedenen Gattern, und "Uberkreuzungen von
solchen Verbindungen. Wir brauchen also einen Spielplan, der die Rolle
eines Drahtes in einem Schaltschema "ubernehmen kann. Eine solchen Spielplan
zeigt Abbildung~\ref{minesweeper-wire}.

\begin{figure}
\begin{center}
\begin{tabular}{cc}
\multicolumn{2}{c}{\includegraphics[width=0.45\hsize]{graphics/wire}}\\
\multicolumn{2}{c}{``Draht'' ohne Signal}\\
&\\
\includegraphics[width=0.45\hsize]{graphics/wire-0}&
\includegraphics[width=0.45\hsize]{graphics/wire-1}\\
Bombenbelegung f"ur logisch ``0''&
Bombenbelegung f"ur logisch ``1''
\end{tabular}
\end{center}
\caption{Minesweeper-``Draht'', Spielplan mit zwei m"oglichen Bombenbelegungen,
die f"ur die zwei m"oglichen Zust"ande stehen k"onnen, die entlang des
Drahtes transportiert werden k"onnen.\label{minesweeper-wire}}
\end{figure}%

Dr"ahte m"ussen sich auch "uberkreuzen, in Abbildung~\ref{minesweeper-cross}
ist ein Spielplan mit zwei sich kreuzenden Dr"ahten gezeigt. Dieses
Kreuz kann zwei verschiedene Inputs vertikal und horizontal haben,
wie in Abbildung~\ref{minesweeper-crosses} dargestellt. Man kann
erkennen, dass sich die Zust"ande bei der Kreuzung gegenseitig nicht
ver"andern. Was links als Zustand 0 eingeht, wird auch rechts als Zustand
0 weitergeleitet.
\begin{figure}
\begin{center}
\includegraphics[width=0.6\hsize]{graphics/cross}
\end{center}
\caption{Kreuzung zweier ``Dr"ahte''\label{minesweeper-cross}}
\end{figure}%
\begin{figure}
\begin{center}
\begin{tabular}{|l|c|c|}
\hline
\raisebox{12ex}{$0$}&
\raisebox{0pt}[0.395\hsize][0pt]{%
\includegraphics[width=0.4\hsize]{graphics/cross-11}}&
\includegraphics[width=0.4\hsize]{graphics/cross-10}\\
\hline
\raisebox{12ex}{$1$}&
\raisebox{0pt}[0.395\hsize][0pt]{%
\includegraphics[width=0.4\hsize]{graphics/cross-01}}&
\includegraphics[width=0.4\hsize]{graphics/cross-00}\\
\hline
&\raisebox{0pt}[15pt][7pt]{$0$}&%
\raisebox{0pt}[15pt][7pt]{$1$}\\
\hline
\end{tabular}
\end{center}
\caption{Vier verschiedene m"ogliche Zustandskombinationen auf
einer Drahtkreuzung\label{minesweeper-crosses}}
\end{figure}%

Das NOT-Gatter muss aus einem Signal das invertierte Signal
machen. Dies schafft der Spielplan in Abbildung~\ref{splitter}.
Der von links eingespeiste Zustand wird rechts invertiert weitergeleitet.
Gleichzeitig wird der eingespeiste Zustand auch unver"andert nach
oben und unten abgeleitet, so dass man mit dieser Struktur auch
eine Aufteilung eines Signals in zwei Signale durchf"uhren kann.
\begin{figure}
\begin{center}
\begin{tabular}{|c|c|c|c|}
\hline
\multirow{2}{0.4\hsize}{%
\raisebox{-5ex}[0.36\hsize][0pt]{%
\includegraphics[width=\hsize]{graphics/splitter}%
}}&
\raisebox{11.5ex}{$0$}&
\raisebox{0pt}[0.355\hsize][0pt]{%
\includegraphics[width=0.4\hsize]{graphics/splitter-1}}&
\raisebox{11ex}{$1$}\\
\cline{2-4}
&
\raisebox{11.5ex}{$1$}&
\raisebox{0pt}[0.355\hsize][0pt]{%
\includegraphics[width=0.4\hsize]{graphics/splitter-0}}&
\raisebox{11ex}{$0$}\\
\hline
\end{tabular}
\end{center}
\caption{NOT-Schaltung und Aufspaltung von Zust"anden\label{splitter}}
\end{figure}%

Damit bleibt nur noch das AND-Gatter. Dieses ist in Abbildung~\ref{andgate}
dargestellt, und in Abbildung~\ref{andstates} kann man die Funktion des
Gatters verfolgen.

Damit ist gezeigt, dass sich jede Schaltung in polynomieller Zeit in 
einen Minesweeper-Plan "ubersetzen l"asst. F"ur eine polynomielle
Reduktion wird aber verlangt, dass eine erf"ullbare Formel auf einen
Plan abgebildet wird, der am Ausgang eine logische 1 haben.
Da in unseren ``Dr"ahten'' eine logische 1 einer Bombe im zweiten
Feld (in Fortpflanzungsrichtung des Signals) entspricht, k"onnen
wir der Forderung nach einer logischen 1 dadurch Nachdruck verleihen,
dass wir diese letzte Bombe bereits platzieren. Als letzten Schritt
in der "Ubersetzung pflanzen wir also im ``letzten'' Feld eine Bombe.
Der so konstruierte Spielplan hat dann genau dann eine konsistente
Bombenbelegung, wenn der urspr"ungliche Schaltplan Output 1 haben kann.

Damit ist jetzt gezeigt, dass
\[
\text{\textsl{SAT}}\le_P
\text{\textsl{CIRCUIT}}\le_P\text{\textsl{MINESWEEPER-CONSISTENCY}},
\]
und weil \textsl{SAT} NP-vollst"andig ist, folgt auch, dass 
\textsl{MINESWEEPER-CONSISTENCY} NP-vollst"andig ist.
\begin{figure}
\begin{center}
\includegraphics[width=0.8\hsize]{graphics/and}
\end{center}
\caption{Minesweeper-Spielplan f"ur das UND-Gatter\label{andgate}}
\end{figure}%
\begin{figure}
\begin{center}
\begin{tabular}{|c|c|c|c|}
\hline
\raisebox{0pt}[13pt][4pt]{oberer Input}&unterer Input&Resultat&Output\\
\hline
\multirow{2}{10pt}{0}&%
\raisebox{11ex}{$0$}&%
\raisebox{0pt}[0.324\hsize][0pt]{%
\includegraphics[width=0.342\hsize]{graphics/and-00}}&%
\raisebox{11ex}{$0$}%
\\
\cline{2-4}
&\raisebox{11ex}{$1$}&%
\raisebox{0pt}[0.324\hsize][0pt]{%
\includegraphics[width=0.342\hsize]{graphics/and-01}}&%
\raisebox{11ex}{$0$}%
\\
\hline
\multirow{2}{10pt}{1}&%
\raisebox{11ex}{$0$}&%
\raisebox{0pt}[0.324\hsize][0pt]{%
\includegraphics[width=0.342\hsize]{graphics/and-10}}&%
\raisebox{11ex}{$0$}%
\\
\cline{2-4}
&\raisebox{11ex}{$1$}&%
\raisebox{0pt}[0.324\hsize][0pt]{%
\includegraphics[width=0.342\hsize]{graphics/and-11}}&%
\raisebox{11ex}{$1$}%
\\
\hline
\end{tabular}
\end{center}
\caption{Nachweis der konsistenten ``Funktion'' des AND-Gatters
\label{andstates}}
\end{figure}%
\subsection{Direkter Beweis}
Unter Verwendung der eben entwickelten Ideen k"onnte es m"oglich sein,
einen direkten Beweis der NP-Vollst"andigkeit von
\textsl{MINESWEEPER-CONSISTENCY} zu geben, ohne die Verwendung
von \textsl{SAT}. Dazu braucht man eine Theorie die besagt, dass
Schaltungen und Turing-Maschinen im wesentlichen "Aquivalent sind.
So eine Theorie existiert, und wird auch ben"otigt, um Turing-Maschinen
mit Quantencomputern vergleichen zu k"onnen. Letztere sind n"amlich
"uber ``Quantenschaltungen'' definiert. Dann sagt die Maschinerie des
vorangegangenen Abschnitts jedoch, dass man jede Turing-Maschine in
ein Minesweeper-Problem "ubersetzen kann, was NP-Vollst"andigkeit
beweist.

%
% vollstaendig.tex -- Turing-Vollständigkeit
%
% (c) 2011 Prof Dr Andreas Mueller, Hochschule Rapperswil
%

\section{Turing-vollständige Programmiersprachen}
\rhead{Turing-vollständige Programmiersprachen}
Die Turing-Maschine liefert einen wohldefinierten Begriff der
Berechenbarkeit, der auch robust gegenüber milden Änderungen
der Definition einer Turing-Maschine ist.
Der Aufbau aus einem endlichen Automaten mit zusätzlichem
Speicher und der einfache Kalkül mit Konfigurationen hat
sie ausserdem Beweisen vieler wichtiger Eigenschaften zugänglich
gemacht. Die vorangegangenen Kapitel über Entscheidbarkeit und
Komplexität legen davon eindrücklich Zeugnis ab. Am direkten
Nutzen dieser Theorie kann jedoch immer noch ein gewisser Zweifel
bestehen, da ein moderner Entwickler seine Programme ja nicht
direkt für eine Turing-Maschine schreibt, sondern nur mittelbar,
da er eine Programmiersprache verwendet, deren Code anschliessend
von einem Compiler oder Interpreter übersetzt und von einer realen
Maschine ausgeführt wird.

Der Aufbau der realen Maschine ist sehr
nahe an einer Turing-Maschine, ein Prozessor liest und schreibt
jeweils einzelne
Speicherzellen eines mindestens für praktische Zwecke unendlich
grossen Speichers und ändert bei Verarbeitung der gelesenen
Inhalte seinen eigenen Zustand. Natürlich ist die Menge der
Zustände eines modernen Prozessors sehr gross, nur schon die $n$
Register der Länge $l$ tragen $2^{nl}$ verschiedene Zustände bei,
und jedes andere Zustandsbit verdoppelt die Zustandsmenge nochmals.
Trotzdem ist die Zustandsmenge endlich, und es braucht nicht viel
Fantasie, sich den Prozessor mit seinem Hauptspeicher als Turingmaschine
vorzustellen. Es gibt also kaum Zweifel, dass die Computer-Hardware
zu all dem fähig ist, was ihr in den letzten zwei Kapiteln an
Fähigkeiten zugesprochen wurde.

Die einzige Einschränkung der Fähigkeiten realer Computer gegenüber
Turing-Maschinen ist
die Tatsache, dass reale Computer nur über einen endlichen Speicher
verfügen, während eine Turing-Maschine ein undendlich langes Band
als Speicher verwenden kann. Da jedoch eine endliche Berechnung auch
nur endlich viel Speicher verwenden kann, sind alle auf einer Turing-Maschine
durchführbaren Berechnungen, die man auch tatsächlich durchführen
will, auch von einem realen Computer durchführbar. Für praktische
Zwecke darf man also annehmen, dass die realen Computer echte Turing-Maschinen
sind.

Trotzdem ist nicht sicher, ob die Programmierung in einer übersetzten
oder interpretierten Sprache alle diese Fähigkeiten auch einem
Anwendungsprogrammierer zugänglich macht.
Letztlich äussert sich dies auch darin, dass Computernutzer
für verschiedene Problemstellung auch verschiedene Werkzeuge
verwenden. Wer tabellarische Daten summieren will, wird gerne
zu einer Spreadsheet-Software greifen, aber nicht erwarten, dass er
damit auch einen Näherungsalgorithmus für das Cliquen-Problem wird
programmieren können. Die Tabellenkalkulation definiert ein eingeschränktes,
an das Problem angepasstes Berechnungsmodell, welches aber
höchstwahrscheinlich weniger leistungsfähig ist als die Hardware, auf
der es läuft. Es ist also durchaus möglich und je nach Anwendung auch
zweckmässig, dass ein Anwender nicht die volle Leistung einer
Turing-Maschine zur Verfügung hat.

Damit stellt sich jetzt die Frage, wie man einem Berechnungsmodell und
das heisst letztlich der Sprache, in der der Berechnungsauftrag
formuliert wird, ansehen kann, ob sie gleich mächtig ist wie eine
Turing-Maschine.

\subsection{Programmiersprachen}
Eine Programmiersprache ist zwar eine Sprache im Sinne dieses Skriptes,
für den Programmierer wesentlich ist jedoch die Semantik, die bisher
nicht Bestandteil der Diskussion war. Für ihn ist die Tatsache wichtig,
dass die Semantik der Sprache Berechnungen beschreibt,
wie sie mit einer Turing-Maschine ausgeführt werden können.

\begin{definition}
\index{Programmiersprache}%
Eine Sprache $A$ heisst eine {\em Programmiersprache}, wenn es eine Abbildung
\[
c\colon A\to \Sigma^*\colon w\mapsto c(w)
\]
gibt, die einem Wort der Sprache die Beschreibung einer Turing-Maschine
zuordnet. Die Abbildung $c$ heisst {\em Compiler} für die Sprache $A$.
\end{definition}
Die Forderung, dass $c(w)$ die Beschreibung einer Turing-Maschine
sein muss, ist nach obiger Diskussion nicht wesentlich.

\subsection{Interaktion}
Man beachte, dass in dieser Definition einer Programmiersprache kein Platz ist
für Input oder Output während des Programmlaufes.
Das Band der Turing-Maschine, bzw.~sein Inhalt bildet den Input, der Output
kann nach Ende der Berechnung vom Band gelesen werden.
Man könnte dies als Mangel dieses Modells ansehen, in der Tat ist aber
keine Erweiterung nötig, um Interaktion abzubilden.
Interaktionen mit einem Benutzer bestehen immer aus einem Strom von
Ereignissen, die dem Benutzer zufliessen (Änderungen des Bildschirminhaltes,
Signaltöne) oder die der Benutzer veranlasst (Bewegungen des Maus-Zeigers,
Maus-Klicks, Tastatureingaben). Alle diese Ereignisse kann man sich codiert
auf ein Band geschrieben denken, welches die Turing-Maschine bei Bedarf
liest.

Der Inhalt des Bandes einer Standard-Turing-Maschine kann während des
Programmlaufes nur von der Turing-Maschine selbst verändert werden.
Da sich die Turing-Maschine aber nicht daran erinnern kann, was beim
letzten Besuch eines Feldes dort stand, ist es für eine Turing-Maschine
auch durchaus akzeptabel, wenn der Inhalt eines Feldes von aussen
geändert wird. Natürlich werden damit das Laufzeitverhalten der
Turing-Maschine verändert. Doch in Anbetracht der
Tatsache, dass von einer Turing-Maschine im Allgemeinen nicht einmal
entschieden werden kann, ob sie anhalten wird, ist wohl nicht mehr
viel zu verlieren.

Die Ausgaben eines Programmes sind deterministisch, und was der Benutzer
erreichen will, sowie die Ereignisse, die er einspeisen wird, sind es ebenfalls.
Man kann also im Prinzip im Voraus wissen, was ausgegeben werden wird
und welche Ereignisse ein Benutzer auslösen wird. Schreibt man diese
vorgängig auf das Band, so wie man es auch beim automatisierten Testen
eines Userinterfaces tut, entsteht aus dem interaktiven Programm eines,
welches ohne Zutun des Benutzers zur Laufzeit funktionieren kann.

\subsection{Die universelle Turing-Maschine}
\index{Turing, Alan}%
\index{Turing-Maschine!universelle}%
In seinem Paper von 1936 hat Alan Turing gezeigt, dass man eine
Turing-Maschine definieren kann,
der man die Beschreibung
$\langle M,w\rangle$
einer Turing-Maschine $M$ und eines Wortes $w$
und die $M$ auf dem Input-Wort $w$ simuliert.
Diese spezielle Turing-Maschine ist also leistungsfähig genug, jede
beliebige andere Turing-Maschine zu simulieren. Sie heisst die {\em universelle
Turing-Maschine}.

Die universelle Turing-Maschine kann die Entscheidung vereinfachen,
ob eine Funktion Turing-berechenbar ist. Statt eine Turing-Maschine
zu beschreiben, die die Funktion berechnet, reicht es, ein Programm
in der Programmiersprache $A$ zu beschreiben, das Programm mit dem
Compiler $c$ zu übersetzen, und die Beschreibung mit der universellen
Turing-Maschine auszuführen.

\index{Church-Turing-Hypothese}%
Die Church-Turing-Hypothese besagt, dass sich alles, was man berechnen
kann, auch mit einer Turing-Maschine berechnen lässt. Die universelle
Turing-Maschine zeigt, dass jede berechenbare Funktion von der
universellen Turing-Maschine berechnet werden kann.
Etwas leistungsfähigeres als eine Turing-Maschine gibt es nicht.

\subsection{Turing-Vollständigkeit}
Jede Funktion, die in der Programmiersprache $A$ implementiert werden
kann, ist Turing-berechenbar.
Der Compiler kann
aber durchaus Fähigkeiten unzugänglich machen, die Programmiersprache
$A$ kann dann gewisse Berechnungen, die mit einer Turing-Maschine
möglich wären, nicht formulieren. Besonderes interessant sind daher
die Sprachen, bei denen ein solcher Verlust nicht eintritt.

\begin{definition}
\index{Turing-vollständig}%
Eine Programmiersprache heisst Turing-vollständig, wenn sich jede
berechenbare Abbildung in dieser Sprache formulieren lässt.
Zu jeder berechenbaren Abbildung $f\colon\Sigma^*\to \Sigma^*$ gibt
es also ein Programm $w$ so, dass $c(w)$ die Funktion $f$ berechnet.
\end{definition}

Zu einer berechenbaren Abbildung gibt es eine Turing-Maschine, die
sie berechnet, es würde also genügen, wenn man diese Turing-Maschine
von einem in der Sprache $A$ geschriebenen Turing-Maschinen-Simulator
ausführen lassen könnte. Dieser Begriff muss noch etwas klarer gefasst
werden:

\begin{definition}
\index{Turing-Maschinen-Simulator}%
Ein Turing-Maschinen-Simulator ist eine Turing-Maschine $S$, die als Input
die Beschreibung $\langle M,w\rangle$ einer Turing-Maschine $M$ und eines
Input-Wortes für $M$ erhält, und die Berechnung durchführt, die $M$ auf $w$
ausführen würde.
Ein Turing-Maschinen-Simulator in der Programmiersprache $A$ ist
ein Wort $s\in A$ so, dass $c(s)$ ein Turing-Maschinen-Simulator ist.
\end{definition}

Damit erhalten wir ein Kriterium für Turing-Vollständigkeit:

\begin{satz}
\label{turingvollstaendigkeitskriterium}
Eine Programmiersprache $A$ ist Turing-vollständig, genau dann
wenn es einen Turing-Maschinen-Simulator in $A$ gibt.
\end{satz}


\subsection{Beispiele}
Die üblichen Programmiersprachen sind alle Turing-vollständig, denn es
ist eine einfache Programmierübung, eines Turing-Maschinen-Simulator
in einer dieser Sprachen zu schreiben. In einigen Programmiersprachen
ist dies jedoch schwieriger als in anderen.

\subsubsection{Javascript}
\index{Javascript}%
Fabrice Bellard hat 2011 einen PC-Emulator in Javascript geschrieben, der
leistungsfähig genug ist, Linux zu booten. Auf seiner Website
\url{http://bellard.org/jslinux/} kann man den Emulator im eigenen Browser
starten. Das gebootete Linux enthält auch einen C-Compiler. Da C
Turing-vollständig ist, gibt es einen Turing-Maschinen-Simulator in
C, den man auch auf dieses Linux bringen und mit dem C-Compiler
kompilieren kann. Somit gibt es einen Turing-Maschinen-Simulator in
Javascript, Javascript ist Turing-vollständig.

\subsubsection{XSLT}
\index{XSLT}%
XSLT ist eine XML-basierte Sprache, die Transformationen von XML-Dokumenten
zu beschreiben erlaubt. XSLT ist jedoch leistungsfähig, eine Turing-Maschine
zu simulieren. Bob Lyons hat auf seiner Website
\url{http://www.unidex.com/turing/utm.htm} ein XSL-Stylesheet publiziert,
welches einen Simulator implementiert. Als Input verlangt es
ein
XML-Dokument, welches die Beschreibung der Turing-Maschine in einem
zu diesem Zweck definierten XML-Format namens Turing Machine Markup
Language (TMML) enthält. TMML definiert XML-Elemente, die das Alphabet
(\verb+<symbols>+),
die Zustandsmenge $Q$ (\verb+<state>...</state>+)
und die Übergangsfunktion $\delta$ in \verb+<mapping>+-Elementen
der Form
\begin{verbatim}
<mapping>
    <from current-state="moveRight1" current-symbol=" " />
    <to next-state="check1" next-symbol=" " movement="left" />
</mapping>
\end{verbatim}
beschreiben. Der initiale Bandinhalt wird als Parameter \verb+tape+
auf der Kommandozeile übergeben.
Das Stylesheet wandelt das TMML Dokument in eine ausführliche
Berechnungsgeschichte um, aus der auch der Bandinhalt am Ende der Berechnung
abzulesen ist. Es beweist somit, dass XSLT einen Turing-Maschinen-Simulator
hat, also Turing-vollständdig ist.

\subsubsection{\LaTeX}
\index{LaTeX@\LaTeX}%
\index{Knuth, Don}%
Don Knuth, der Autor von \TeX, hat sich lange davor gedrückt, seiner
Schriftsatz-Sprache auch eine Turing-vollständige Programmiersprache
zu spendieren. Schliesslich kam er nicht mehr darum herum, und wurde
von Guy Steeles richtigegehend dazu gedrängt, wie er in
\url{http://maps.aanhet.net/maps/pdf/16\_15.pdf}
gesteht.

Dass \TeX Turing-vollständig ist beweist ein Satz von \LaTeX-Macros, den
man auf
\url{https://www.informatik.uni-augsburg.de/en/chairs/swt/ti/staff/mark/projects/turingtex/}
finden kann.
Um ihn zu verwenden, formuliert man die Beschreibung
von Turing-Maschine und initialem Bandinhalt als eine Menge von
\LaTeX-Makros. Ebenso ruft man den Makro \verb+\RunTuringMachine+ auf,
der die Turing-Machine simuliert und die Berechnungsgeschichte im
\TeX-üblichen perfekten Schriftsatz ausgibt.




\appendix
\chapter{Algorithmen-Übersicht\label{skript:algorithmen}}
\lhead{Algorithmen-Übersicht}
\rhead{}
In den vergangenen Kapiteln wurde eine grosse Zahl von Algorithmen
für die verschiedensten Probleme formuliert. Hier werden die wichtigsten
im Sinne einer Übersicht zusammengestellt, mit Verweisen auf die 
detaillierte Beschreibung weiter vorne im Text.
\section{Endliche Automaten und reguläre Sprachen}
\subsection{Minimalautomat}
\newtheorem*{Minimalautomat}{Minimalautomat}
\begin{Minimalautomat}
Zu jedem deterministischen endlichen Automaten $A$ finde den minimalen
Automaten $A'$.
\end{Minimalautomat}
Der Satz~\ref{satz_minimalautomat} beschreibt die Eigenschaften des
Minimalautomaten, anschliessend im Text wird der ``Kreuzchen''-Algorithmus
beschrieben, mit dem der Minimalautomat gefunden werden kann.

\newtheorem*{Automatenvergleich}{Automatenvergleich}
\begin{Automatenvergleich}
Gegeben zwei endliche Automaten $A$ und $B$ finde heraus, ob die
beiden Automaten die gleiche Sprache akzeptieren, also $L(A)=L(B)$.
\end{Automatenvergleich}
Das Problem, ob zwei Automaten die gleiche Sprache akzeptieren,
ist mit Hilfe des Minimalautomaten (Seite \pageref{algorithmus:minimalautomat})
entscheidbar. 
Der Algorithmus wurde auch im Satz~\ref{satz:eqdea} verwendet, wo die
Entscheidbarkeit von $\textsl{EQ}_\textsl{DEA}$ gezeigt wurde.


\subsection{NEA}
\newtheorem*{NEA}{Umwandlung NEA $\to$ DEA}
\begin{NEA}
Ein nicht deterministischer endlicher Automat $A$ kann in einen 
deterministischen endlichen Automaten $B$ umgewandelt werden, der die
gleiche Sprache akzeptiert, $L(A)=L(B)$.
\end{NEA}

Der Umwandlungsalgorithmus NEA $\to$ DEA wird in
Abschnitt~\ref{regulaer:nea-dea}
auf
Seite~\pageref{regulaer:nea-dea}
beschrieben.

\subsection{Mengenoperationen}
Die Menge der reguläre Sprachen ist abgeschlossen bezüglich der
Mengenoperationen, es muss also Algorithmen geben, die die Mengenoperationen
auf Automatenebene implementieren.

\newtheorem*{RegVereinigung}{Vereinigung regulärer Sprachen}
\begin{RegVereinigung}
Zu zwei deterministischen endlichen Automaten $A$ und $B$ berechne einen
Automaten $C$ derart, dass $L(C)=L(A)\cup L(B)$.
\end{RegVereinigung}

Die Vereinigung wird in Satz~\ref{satz_union} beschrieben, sie tritt auch als
Alternative bei den regulären Operationen auf.

\newtheorem*{RegSchnitt}{Schnittemenge regulärer Sprachen}
\begin{RegSchnitt}
Zu zwei deterministischen endlichen Automaten $A$ und $B$ berechne einen
Automaten $C$ derart, dass $L(C)=L(A)\cap L(B)$.
\end{RegSchnitt}

Satz~\ref{satz_intersection} beschreibt eine Konstruktion, mit der
ein deterministischer endlicher Automat für die Schnittmenge gefunden
werden kann. Sie verwendet die Produktautomaten-Konstruktion von Seite
\pageref{reg_produktautomat}.

\newtheorem*{RegNegation}{Komplement einer regulären Sprache}
\begin{RegNegation}
Zu einem deterministischen endlichen Automaten $A$ berechne den Automaten
$B$ mit der Eigenschaft $L(B)=\overline{L(A)}$.
\end{RegNegation}

Der Algorithmus tauscht Akzeptier- und Nichtakzeptierzustände aus,
und funktioniert in dieser Form nur für deterministische endliche
Automaten (Satz~\ref{satz_regcomplement}).

\newtheorem*{RegDifferenz}{Differenz zweier regulärer Sprachen}
\begin{RegDifferenz}
Zu zwei gegebenen deterministischen endlichen Automaten $A$ und $B$ 
berechne einen deterministischen endlichen Automaten $C$ mit
$L(C)=L(A)\setminus L(B)$.
\end{RegDifferenz}

Die Mengendifferenz ist die Schnittmenge mit dem Komplement:
$L(A)\setminus L(B)=L(A)\cap\overline{L(B)}$, der Algorithmus wird im Beweis
von Satz~\ref{satz_regcomplement} dargestellt.

\subsection{Reguläre Operationen}
Reguläre Operationen können in Operationen mit nicht deterministischen
endlichen Automaten übersetzt werden.
Die Verkettung wird in Satz~\ref{satz_concat} beschrieben, die *-Operation
in Satz~\ref{satz_star}.

\newtheorem*{Alternative}{Alternative}
\begin{Alternative}Zu zwei endlichen Automaten $A$ und $B$ berechne einen
Automaten $C$, der die Alternative der beiden Sprachen akzeptiert, also
$L(C)=L(A)|L(B)=L(A)\cup L(B)$.
\end{Alternative}

\newtheorem*{Verkettung}{Verkettung}
\begin{Verkettung}
Zu zwei endlichen Automaten $A$ und $B$ berechnen einen Automaten $C$,
der die Verkettung der Sprachen von $A$ und $B$ akzeptiert: $L(C)=L(A)L(B)$.
\end{Verkettung}

\newtheorem*{Sternoperation}{Stern-Operation}
\begin{Sternoperation}
Zu einem endlichen Automaten $A$ berechne einen Automaten $B$, der die
Stern-Operation der Sprache von $A$ akzeptiert: $L(B)=L(A)^*$.
\end{Sternoperation}



\subsection{Reguläre Ausdrücke}
\newtheorem*{RegexDea}{Regulären Ausdruck in DEA umwandeln}
\begin{RegexDea} Berechne aus einem regulären Ausdruck $r$ einen 
deterministischen endlichen Automaten $A$, der die gleiche Sprache
akzeptiert, also $L(A)=L(r)$.
\end{RegexDea}

Die Umwandlung eines regulären Ausdrucks in einen endlichen Automaten
wird in Abschnitt~\ref{regulaer:regulaere-ausdruecke} beschrieben.

\newtheorem*{DeaRegex}{DEA in regulären Ausdruck umwandeln}
\begin{DeaRegex}
Zu einem deterministischen endlichen Automaten $A$ berechne einen
regulären Ausdruck $r$, der die gleiche Sprache akzeptiert,
also $L(A)=L(r)$.
\end{DeaRegex}

Die Umwandlung eines endlichen Automaten in einen äquivalenten
regulären Ausdrucks 
wird in Abschnitt~\ref{regulaer:dea-re} beschrieben.

\section{Stackautomaten und kontextfreie Grammatiken}

\subsection{Stackautomaten}
Zum Beweis der Äquivalenz von Stackautomaten und kontextfreien Grammatiken
wurden im Abschnitt~\ref{sect:aequivalenz-cfg} zwei Algorithmen beschrieben,
wie man eine kontextfreie Grammatik in einen äquivalenten Stackautomanten
umwandenl kann und umgekehrt.

\newtheorem*{CfgPDA}{Stackautomat einer Grammatik}
\begin{CfgPDA}
Zu einer kontextfreien Grammatik $G$ finde einen Stackautomaten $P$, 
der genau die von $G$ produzierte Sprache akzeptiert: $L(G)=L(P)$.
\end{CfgPDA}

\newtheorem*{PdaCfg}{Grammatik eines Stackautomaten}
\begin{PdaCfg}
Zu einem Stackautomaten $P$ findet eine kontextfreie Grammatik $G$, die
die gleiche Sprache produziert: $L(G)=L(P)$.
\end{PdaCfg}

\subsection{Chomsky-Normalform}
Die Chomsky-Normalform wird in Definition~\ref{definition:cnf} definiert.
Im Beweis von Satz~\ref{satz:cnf} wird die Umwandlung in eine äquivalente
Grammatik in Chomsky-Normalform dargestellt.

\newtheorem*{CNF}{Chomsky-Normalform einer Grammatik}
\begin{CNF}
Zu einer kontextfreien Grammatik $G$ finde eine kontextfreie Grammatik $G'$
in Chomsky-Normalform, die die gleiche Sprache akzeptiert.
\end{CNF}

\subsection{Mengenoperationen}
Nur für die Vereinigung zweier kontextfreier Sprachen haben wir einen
Algorithmus.

\newtheorem*{CfgUnion}{Grammatik einer Vereinigung}
\begin{CfgUnion}
Aus zwei kontextfreien Grammatiken $G_1$ und $G_2$ berechne
eine kontextfreie Grammatik $G$, die die
Vereinigung der von $G_1$ und $G_2$ produzierten Sprachen prodziert:
$L(G)=L(G_1)\cup L(G_2)$.
\end{CfgUnion}

\subsection{Reguläre Operationen}
Die regulären Operationen auf Grammatiken wurden in
Abschnitt~\ref{sect:cfg-regulaer} beschrieben.
Die Alternative ist in Satz~\ref{satz:cfg-union} erklärt,
die Verkettung in Satz~\ref{satz:cfg-verkettung} und die *-Operation
in Satz~\ref{satz:cfg-star}.

\newtheorem*{CfgAlternative}{Grammatik für eine Alterative}
\begin{CfgAlternative}
Zu zwei kontextfreien Grammatiken $G_1$ und $G_2$ finde eine
kontextfreie Grammatik $G$, die die
Alternative der von $G_1$ und $G_2$ produzierten Sprachen produziert:
$L(G)=L(G_1)|L(G_2)=L(G_1)\cup L(G_2)$.
\end{CfgAlternative}

\newtheorem*{CfgConcatenation}{Grammatik einer Verkettung}
\begin{CfgConcatenation}
Zu zwei kontextfreien Grammatiken $G_1$ und $G_2$ berechne eine 
kontextfreie Grammatik $G$, die die Verkettung der von $G_1$ und $G_2$
produzierten Sprachen produziert: $L(G)=L(G_1)L(G_2)$.
\end{CfgConcatenation}

\newtheorem*{CfgStar}{Grammatik für die Sternoperation}
\begin{CfgStar}
Zu einer kontextfreien Grammatik $G$ berechne eine kontextfreie Grammatik
$G'$, die die *-Operation der von $G$ produziert Sprache produziert:
$L(G')=L(G)^*$.
\end{CfgStar}

\subsection{Parser}
Stackautomaten sind nicht deterministisch und eignen sich daher nicht
dazu, ein Wort $w$ daraufhinzu zu prüfen, ob es von einer kontextfreien
Grammatik $G$ produziert werden kann: $w\in L(G)$. 
Es gibt aber einen deterministischen Algorithmus, der in
Satz~\ref{cyk-algorithm} beschrieben wird.
Dieser Algorithmus wird auch zur Entscheidung des Problems $A_\textsl{CFG}$
in Satz~\ref{satz:acfg-entscheidbar}
benötigt.

\newtheorem*{CYK}{Cocke-Younger-Kasami Algorithmus}
\begin{CYK}
Zu einer kontextfreien Grammatik $G$ in Chomsky-Nor\-mal\-form und einem
Wort $w$ berechne den Ableitungsbaum (Parse-Tree) von $w$.
\end{CYK}

%\section{Graphen}

%\section{Turingmaschinen}


\vfill
\pagebreak
\ifodd\value{page}\else\null\clearpage\fi
\lhead{}
\rhead{}
\printbibliography[heading=subbibliography]
\end{refsection}

\input skript.ind
\end{document}
