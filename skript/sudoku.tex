\subsection{SUDOKU und SAT\label{subsection:sudoku-und-sat}}
\index{Sudoku}
Als Vorstufe der angestrebten Erkenntnis, dass sich jedes Problem auf \textsl{SAT}
reduzieren l"asst, wollen wir zeigen, dass sich die Frage nach der
L"osbarkeit eines Sudoku-R"atsels auf \textsl{SAT} reduzieren l"asst.

Ein $n$-Sudoku ist ein $n^2\times n^2$-Feld, welches mit $n^2$ verschiedenen
Zeichen aus einer Menge $\Sigma$ so gef"ullt werden muss,
dass in jeder Zeile, jeder Spalte
und in jedem der $n\times n$-Unterquadrate jedes Zeichen genau einmal
vorkommt.
Die aus R"atselspalten von Zeitungen bekannten $3$-Sudokus verwenden
die Ziffern $1$ bis $9$ f"ur $\Sigma$.
Einige der Zeichen sind bereits vorgegeben, die leeren Felder m"ussen
so gef"ullt werden, dass die Regeln nicht verletzt werden.
Abbildung~\ref{sudoku} zeigt ein l"osbares $3$-Sudoku-R"atsel.
\begin{figure}
\begin{center}
\includegraphics[width=0.4\hsize]{images/sudoku-1.pdf}
\end{center}
\caption{$3$-Sudoku, die hellen Ziffern zeigen, dass die Vorgabe (dunkle
Ziffern) erf"ullbar ist, das $3$-Sudoko ist l"osbar.\label{sudoku}}
\end{figure}
Wir schreiben
\begin{align*}
\textsl{SUDOKU}
=
\{\langle S\rangle\;|\; \text{$S$ ist ein l"osbares $n$-Sudoku}\}.
\end{align*}

Ob ein $n$-Sudoku l"osbar ist, ist sicher entscheidbar: man kann alle
$|\Sigma|^{n^2\cdot n^2}=n^{2n^4}$ m"oglichen Belegungen des Feldes mit
Zeichen durchprobieren, was aber nat"urlich exponentielle Zeit
beansprucht.

\textsl{SUDOKU} ist auch in NP, denn man kann als L"osungszertifikat
die Zeichen im vollst"andig ausgef"ullten Feld verwenden.
Zur Verifikation muss man dann nur "uberpr"ufen, ob die
vorgegebenen Zeichen richtig in die L"osung "ubernommen wurden und
ob die Sudoku-Regeln eingehalten wurden. Dies ist in Laufzeit $O(n^4)$
m"oglich, es gibt also einen polynomiellen Verifizierer.

Der Satz~\ref{cooklevin} behauptet, \textsl{SAT} sei NP-vollst"andig,
es m"usste sich also auch \textsl{SUDOKU} polynomiell auf \textsl{SAT}
reduzieren lassen:

\begin{satz} $\textsl{SUDOKU}\le_P\textsl{SAT}$.
\label{skript:satz:sudoku-sat}
\end{satz}

\begin{proof}[Beweis]
Wir m"ussen eine Reduktion konstrukieren, die jedem Sudoku-R"atsel
eine logische Formel zuordnet, welche genau dann erf"ullbar ist, wenn
das Sudoku-R"atsel l"osbar ist. Wir gehen in zwei Schritten vor. 
Zun"achst konstrieren wir eine Formel, die genau dann erf"ullbar ist,
wenn das Sudoku l"osbar ist, die aber noch keine logische Formel ist.
Im zweiten Schritt wandeln wir die Formel dann in eine "aquivalente
logische Formel um.

1.~Schritt: Sei also $S$ ein gegebenes $n$-Sudoku. Wenn das R"atsel l"osbar
ist, dann gibt es f"ur jedes Feld ein Zeichen, f"ur welches alle Sudoku-Regeln
eingehalten werden. Wir bezeichnen das Zeichen in dem Feld in Zeile $i$
und Spalte $j$ mit $z_{ij}$. F"ur einige Felder gibt es Vorgaben, die wir mit
$v_{ij}$ bezeichnen. Jetzt m"ussen die Bedingungen f"ur eine korrekte
L"osung des Sudoku formuliert werden:
\begin{compactenum}
\item {\em Vorgaben sind korrekt:} Die Vorgabe im Feld $(i,j)$ ist erf"ullt,
wenn gilt $z_{ij}=v_{ij}$. Alle Vorgaben sind erf"ullt, wenn der Ausdruck
\[
\varphi_{\text{Vorgaben}}=\bigwedge_{\text{Feld $(i,j)$ ist Vorgabefeld}}(z_{ij}=v_{ij})
\]
wahr ist.
\item {\em Zeilenbedingung:} Der Wert $z_{ij}$ in Zeile $i$ ist erlaubt,
wenn jedes andere $z_{ij}$ in der Zeile $i$ verschieden davon ist:
\[
\varphi_{\text{Zeilenbedingung $(i,j)$}}
=
\bigwedge_{k\ne j}(z_{ij}\ne z_{ik}).
\]
\item {\em Spaltenbedingung:} Der Wert $z_{ij}$ in Spalte $j$ ist erlaubt,
wenn jedes andere $z_{kj}$ in der Spalte $j$ verschieden ist:
\[
\varphi_{\text{Spaltenbedingung $(i,j)$}}
=
\bigwedge_{k\ne i}(z_{ij}\ne z_{kj}).
\]
\item {\em Unterfeld-Bedingung:} Der Wert $z_{ij}$ im $n\times n$-Unterfeld,
welches $(i,j)$ enth"alt, ist zul"assig, wenn jedes andere Zeichen im gleichen
Unterfeld davon verschieden ist:
\[
\varphi_
{\text{Unterfeldbedingung $(i,j)$}}
=
\bigwedge_{\myatop{\text{$(k,l)$ im Unterfeld $(i,j)$}}{i\ne k\wedge j\ne l}}(z_{ij}\ne z_{kl}).
\]
\end{compactenum}
F"ur jedes Feld $(i,j)$ gibt es also drei Bedingungen:
\[
\varphi_{ij}=
\varphi_{\text{Zeilenbedingung $(i,j)$}}
\wedge
\varphi_{\text{Spaltenbedingung $(i,j)$}}
\wedge
\varphi_{\text{Unterfeldbedingung $(i,j)$}}.
\]
Eine L"osung f"ur das Sudoku kann genau dann gefunden werden, wenn man
$z_{ij}$ so w"ahlen kann, dass alle diese Feld-Bedingungen und die
Vorgabe-Bedingungen erf"ullt sind, wenn also die Formel
\begin{equation}
\varphi=
\varphi_{\text{Vorgaben}}\wedge
\bigwedge_{1\le i,j\le n^2}
\varphi_{ij}
\label{sudoku-formel}
\end{equation}
erf"ullbar ist.

2.~Schritt: Die Formel (\ref{sudoku-formel}) ist nahe am Gesuchten,
sie ist genau dann erf"ullbar, wenn das Sudoku l"osbar ist. Allerdings
ist sie keine logische Formel, die Variablen $z_{ij}$ sind Variablen mit
Zeichenwerten. Wir k"onnen Sie aber durch $n^2$ neue logische Variablen
$x_{ijc}$ ersetzen, wobei $c\in \Sigma$ ist, mit der Bedeutung, dass
$x_{ijc}$ genau dann wahr ist, wenn $z_{ij}=c$. 

Jede Teilformel von (\ref{sudoku-formel}) setzt sich zusammen
aus Ausdr"ucken der Form oder $z_{ij}=c$
oder $z_{ij}\ne z_{kl}$, wobei $c\in\Sigma$.
Es reicht also, wenn man jede Teilformel nur durch die Variablen $x_{ijc}$ 
ausdr"ucken kann.
\begin{compactenum}
\item {\em Nur ein Wert:} F"ur jedes Paar $(i,j)$ darf nur eine der
Variablen $x_{ijc}$ wahr sein, es kann ja nur ein einziges Zeichen $c$
die Bedingung $z_{ij}=c$ erf"ullen. Die Formel
\[
x_{ijc}\wedge \bigwedge_{d\in\Sigma\setminus\{c\}}\bar x_{ijd}
\]
muss also f"ur mindestens ein $c$ wahr sein:
\[
\varphi_{\text{Eintrag $(i,j)$}}
=
\bigvee_{c\in \Sigma}
\biggl(
x_{ijc}\wedge \bigwedge_{d\in\Sigma\setminus\{c\}}\bar x_{ijd}
\biggr).
\]
\item {\em Vorgabe:} Um die Vorgabe $z_{ij}=c$ auszudr"ucken, muss man
nur verlangen, dass $x_{ijc}$ wahr ist.
\item {\em Ungleichheit:} Um auszudr"ucken, dass $z_{ij} \ne z_{kl}$ muss
man nur verlangen, dass $x_{ijc}$ und $x_{klc}$ nicht gleichzeitig wahr
sein k"onnen:
\[
\varphi_{\text{Ungleichheit $z_{ij}\ne z_{kl}$}}
=
\bigwedge_{c\in\Sigma}
\overline{x_{ijc}\wedge x_{klc}}.
\]
\end{compactenum}
Durch Anwendung aller dieser Umformungen auf alle Terme in (\ref{sudoku-formel})
erh"alt man eine neue Formel $\varphi$, die nur noch die logischen
Variablen $x_{ijc}$ enth"alt. Die neue Formel ist genau dann erf"ullbar,
wenn das Sudoku-R"atsel eine L"osung hat.
\end{proof}

\subsection{Vergleichsformeln}
Aus dem letzten Teil des Beweises von Satz~\ref{skript:satz:sudoku-sat}
können wir ausserdem ablesen, dass die Übersetzung von den Variablen $z_{ij}$
in eine logische Formel mit den logischen Variablen $x_{ijc}$ ebenfalls
algorithmisch möglich ist.
Voraussetzung dafür ist, dass es die Regeln als eine logische Formel
in den Ausdrücken $z_{ij}=z_{kl}$ formuliert werden können.
Wir nennen eine solche Formel eine {\em Vergleichsformel}.

\begin{definition}
Eine {\em Vergleichsformel} in den Variablen $z_{ij}$ ist eine
logische Formel in den Teilformeln $f_{ijkl}=(z_{ij}=z_{kl})$.
\end{definition}

Durch Verwendung der Variablen $x_{ijc}$ mit $c\in\Sigma$ kann man
jeden Ausdruck $f_{ijkl}$ in polynomieller Zeit in eine logische
Formel in den Variablen $x_{ijc}$ umwandeln.
Dies beweist den folgenden Satz.

\begin{satz}
\label{skript:satz:vergleichsformel}
Eine Vergleichsformel in den Variablen $z_{ij}$, deren Werte in $\Sigma$
liegen, kann in polynomieller Zeit in eine logische Formel in Variablne
$x_{ijc}$, $c\in\Sigma$ umgewandelt werden, die genau dann erfüllbar ist,
wenn die Vergleichsformel erfüllbar ist.
\end{satz}





