%
% nichtentscheidbar.tex
%
% (c) 2019 Prof Dr Andreas Müller, Hochschule Rapperswil
%
\section{Nicht entscheidbare Probleme für CFGs}
\rhead{Nicht entscheidbare Probleme für CFGs}
Der letzte Abschnitt könnte den Eindruck hinterlassen haben,
dass nicht entscheidbare Probleme erst bei Turingmaschinen
auftreten, und dass alle naheliegenden Probleme bei regulären
und kontextfreien Sprachen entscheidbar sind.
Dem ist jedoch nicht so.
Es ist zum Beispiel nicht entscheidbar, ob zwei
Grammatiken die gleiche Sprache erzeugen.

\begin{satz} Ob eine kontextfreie Grammatik
\[
\text{\it ALL}_{\text{CFG}}=\{\langle G\rangle\;|\; \text{$G$
ist eine kontextfreie Grammatik und $L(G)=\Sigma^*$}\}
\]
\index{ALLCFG@$\textit{ALL}_{\text{CFG}}$}%
alle Wörter erzeugt, ist nicht entscheidbar.
\end{satz}

\begin{proof}[Beweis]
Wir konstruieren eine Reduktion 
$A_{\text{TM}}\le \text{\it ALL}_{\text{CFG}}$.
Wir müssen also aus einem Paar $\langle M,w\rangle$ eine
Grammatik erzeugen, die genau dann alle Wörter erzeugt, wenn
$M$ das Wort $w$ erkennt.
Dabei müssen wir diese Strings in
Zusammenhang bringen mit einer Berechnung der Turingmaschine, die
auf $w$ im Zustand $q_{\text{accept}}$ endet.

Dazu können wir
die Berechnungsgeschichte verwenden.
Jede Konfiguration der
Turingmaschine ist ja ein String der Form $C=w_1qw_2$.
Die gesamte
Berechnung besteht aus einer Folge $C_i$ von solchen Konfigurationen.
Wir können sie protokollieren, indem wir alle $C_i$
mit Hilfe eines zusätzlichen Zeichens verketten,
welches nicht in $\Gamma$ vorkommt.
Wir bezeichnen diese Zeichen
mit {\tt\#}.

Die Grammatik soll jetzt also alle Strings ausser der Berechnungsgeschichte
erzeugen, die zum Erkennen des Wortes $w$ führen.
Diese Grammatik
erzeugt also genau dann alle Strings, wenn $M$ das Wort $w$ erkennt.

Um alle Strings ausser der erkennden Berechnung zu bekommen,
überlegen wir uns, wie ein String
$h={\tt\#}C_1{\tt\#}C_2{\tt\#}\dots{\tt\#}C_n{\tt\#}$
keine korrekte Berechungsgeschichte sein könnte:
\medskip
\begin{compactenum}
\item $C_1$ könnte nicht die Startkonfiguration sein, also
$C_1\ne q_0w$.
\item $C_n$ könnte nicht eine akzeptierende Konfiguration 
sein, also etwas von der Form $xq_{\text{accept}}y$.
\item Ein Zwischenschritt könnte nicht den Regeln der Turingmaschine
$M$ entsprechen.
\end{compactenum}
\medskip
Die gesuchte Grammatik muss alle Strings erzeugen, die einen dieser ``Fehler''
machen.
Falls es eine erkennende Berechnung für $w$ gibt, wird diese
die einzige sein, die nicht erzeugt wird.

Statt die Grammatik zu konstruieren, konstruieren wir einen Stackautomaten,
den wir später in eine Grammatik umwandeln können.
Der Stackautomat
muss alle Strings akzeptieren, die einen der Defekte 1-3 haben, wir
können also damit beginnen, nichtdeterministisch zwischen diesen drei
Fällen zu unterscheiden:
\[
\entrymodifiers={++[o][F]}
\xymatrix @+3mm{
*+\txt{}
	&q_1
\\
q_0	\ar[ur]^{\varepsilon,\varepsilon\to\varepsilon}
	\ar[r]^{\varepsilon,\varepsilon\to\varepsilon}
	\ar[dr]_{\varepsilon,\varepsilon\to\varepsilon}
	&q_2
\\
*+\txt{}
	&q_3
}
\]
Im Zustand $q_1$ müssen wir einen Stackautomaten anhängen, der
überprüft, ob im Input der String $q_0w$ steht.
Dies kann man 
bereits mit einem DEA machen, man braucht den Stack dafür gar nicht.
Ist $w=a_1a_2\dots a_k$, dann ist der Automat dafür geeignet
\[
\xymatrix @+3mm{
*++[o][F]{q_1}\ar[r]^{q_0,\varepsilon\to\varepsilon}
	&*++[o][F]{}\ar[r]^{a_1,\varepsilon\to\varepsilon}
		&*++[o][F]{}\ar[r]^{a_2,\varepsilon\to\varepsilon}
			&\dots\ar[r]^{a_{k-1},\varepsilon\to\varepsilon}
				&*++[o][F]{}\ar[r]^{a_k,\varepsilon\to\varepsilon}
					&*++[o][F]{}
}
\]
Akzeptieren darf er aber nur, wenn diese Regeln nicht eingehalten
werden, man muss also noch einen Fehlerzustand hinzufügen, der
alles akzeptiert, was in diesem Automaten nicht funktioniert.

Im Zustand $q_2$ müssen wir einen Stackautomaten anhängen, welcher
eine akzeptierende Konfiguration erkennt.
Er muss also zunächst 
alle $C_i$ mit $i<n$ überspringen und dann nichtdeterministisch
die Zeichen von $C_i$ lesen und kontrollieren, ob $q_{\text{accept}}$
darin vorkommt.
Ein Automat dafür ist 
\[
\entrymodifiers={++[o][F]}
\xymatrix @+3mm{
q_2\ar@(ur,ul)
	\ar[r]^{{\tt\#},\varepsilon\to\varepsilon}
	&{}\ar@(ur,ul)
		\ar[rr]^{q_{\text{accept}},\varepsilon\to\varepsilon}
		&*+\txt{}
			&{}\ar@(ur,ul)
				\ar[r]^{{\tt\#},\varepsilon\to\varepsilon}
				&*++[o][F]{}
}
\]
Darin enhalten die Schleifen beim zweiten und dritten Zustand keine
Übergänge mit dem Zeichen {\tt\#}.
Dazu braucht es zusätzliche
Übergänge, damit der Automat alles ausser diesen Strings akzeptiert.

Im Zustand $q_3$ muss ein Automat konstruiert werden, der erkennen kann,
ob die Folge $C_i{\tt\#}C_{i+1}$ einem korrekten Übergang zwischen
zwei Konfigurationen der Turingmaschen $M$ entspricht.
Dazu schreibt
der Stackautomat zunächst alle Zeichen von $C_i$ auf den
Stack, und vergleicht dann den Stackinhalt mit den Zeichen von
$C_{i+1}$.
Bis auf die Zeichen um die Kopfposition dürfen sich
die Strings nicht unterscheiden.
Dieser Automat enthält also
einerseits alle Übergänge, welche das Zeichen auf dem Stack
mit dem nächsten Inputzeichen vergleichen:
\[
\entrymodifiers={++[o][F]}
\xymatrix{
q\ar@(ur,dr)^{a,a\to\varepsilon}
}
\]
Ausserdem Übergänge, die zu gültigen
Turingmaschinen-Konfigurationsübergängen gehören.
Ist zum Beispiel
$px\to yq$ ein solcher Übergang, dann müssen folgende
Stackautomaten-Übergänge hinzugefügt werden:
\[
\entrymodifiers={++[o][F]}
\xymatrix{
{}\ar@/^/[r]^{y,p\to\varepsilon}
	&\ar@/^/[l]^{q,x\to\varepsilon}
}
\]
Leider funktioniert dies noch nicht ganz: die Zeichen von $C_i$
befinden sich auf dem Stack in umgekehrter Reihenfolge, passen
also nicht zu $C_{i+1}$.
Diesen Mangel kann man mit folgendem
Trick beheben: man schreibt in der Berechnungsgeschichte jede
zweite Konfiguration rückwärts, also
\[
{\tt\#}C_1{\tt\#}C_2^t{\tt\#}C_3{\tt\#}C_4^t{\tt\#}\dots
\]
Die Regeln für die korrekten Turingmaschinen-Übergänge müssen
natürlich an diese umgekehrte Ordnung angepasst werden, dabei
muss beachtet werden, dass mit jedem gelesenen {\tt\#}-Zeichen
die ``Leserichtung'' ändert.
Und wiederum muss der Automat alles
akzeptieren, was nicht den Turingmaschinen-Regeln entspricht.

Auf diese Weise kann ein Stackautomat konstruiert werden, der
alle Strings akzeptiert ausser der korrekten Berechnungsgeschichte
(mit alternierenden Schreibrichtungen geschrieben).
Die daraus
erzeugte Grammatik erzeugt alle Strings ausser der
korrekten Berechnungsgeschichte.
Damit ist die Reduktionsabbildung
konstruiert.
\end{proof}

\begin{satz}
\index{Gleichheitsproblem!für kontextfreie Grammatiken}%
Das Gleichheitsproblem für kontextfreie Grammatiken
\[
\text{\it EQ}_{\text{CFG}}=\{
\langle G_1,G_2\rangle\;|\;\text{$G_i$ sind kontextfreie Grammatiken
und $L(G_1)=L(G_2)$}
\}
\]
\index{EQCFG@$\textit{EQ}_{\text{CFG}}$}%
ist nicht entscheidbar.
\end{satz}

\begin{proof}[Beweis]
Wir konstruieren eine Reduktion
\[
\text{\it ALL}_{\text{CFG}}
\le
\text{\it EQ}_{\text{CFG}}.
\]
Da $\text{\it ALL}_{\text{CFG}}$ nicht entscheidbar ist, folgt
auch, dass $\text{\it EQ}_{\text{CFG}}$ nicht entscheidbar
ist.

Sei $G_0$ die Grammatik, die $\Sigma^*$ erzeugt.
Offenbar ist
$\langle G\rangle$ genau dann in $\text{ALL}_{\text{CFG}}$,
wenn $G$ und $G_0$ die gleiche Sprache erzeugen, also
$\langle G,G_0\rangle\in\text{\it EQ}_{\text{CFG}}$.
Die Reduktionsabbildung ist also
\[
\langle G\rangle \mapsto \langle G,G_0\rangle,
\]
dies ist sicher berechenbar.
\end{proof}


