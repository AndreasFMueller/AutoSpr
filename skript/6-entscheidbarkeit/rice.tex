%
% rice.tex -- Satz von Rice
%
% (c) 2019 Prof Dr Andreas Müller, Hochschule Rapperswil
%

\section{Satz von Rice}
\rhead{Satz von Rice}
Eigenschaften von Turing-erkennbaren Sprachen sind im Allgemeinen
nicht entscheidbar, wie die vorangegangenen Beispiel zeigen.
Allerdings müssen sie genü\-gend kompliziert sein.
Dann lässt
sich aber sogar der Beweis vereinheitlichen.
Zum Beispiel wurde
beim Beweis von $\text{\it REGULAR}_{\text{TM}}$ eigentlich
nur verwendet, dass wir eine reguläre Sprache ($\emptyset$) und
eine nicht reguläre ($\{0^n1^n\;|\;n\in\mathbb N\}$) kennen.

\begin{definition}
\index{Eigenschaft!nicht-triviale}%
Sei $P$ eine Eigenschaft einer Sprache $L$, $P(L)$ ist also wahr oder
falsch.
$P$ heisst nicht trivial, falls es eine Sprache $L_1$ gibt,
die die Eigenschaft $P$ hat, und eine Sprache $L_2$, die die
Eigenschaft nicht hat.
\end{definition}

\begin{satz}[Rice]
\index{Rice!Satz von}%
\label{rice-theorem}
Sei $P$ eine nicht triviale Eigenschaft von Turing-erkennbaren Sprachen,
dann ist $P$ nicht entscheidbar.
Als Sprachproblem formuliert:
\[
P_{\text{TM}}=\{ \langle M\rangle\;|\;
\text{$M$ ist eine TM und $L(M)$ hat die Eigenschaft $P$}
\}
\]
ist nicht entscheidbar.
\end{satz}

\begin{proof}[Beweis]
Wir konstruieren eine Reduktion $A_\text{TM}\le P_{\text{TM}}$.
Zu einem Paar
$\langle M,w\rangle$
müssen wir also auf
berechenbare Weise eine Turingmaschine $M'$
\[
f\colon \langle M,w\rangle\mapsto M'
\]
konstruieren, deren Sprache genau dann die Eigenschaft $P$
hat, wenn $M$ das Wort $w$ akzeptiert.

Dazu brauchen wir die beiden Turing-erkennbaren Sprachen $L_1$ und $L_2$,
wobei $L_1$ die Eigenschaft $P$ hat, $L_2$ aber nicht.
Wir können dabei annehmen, dass $L_2=\emptyset$ ist, dass also
$\emptyset$ die Eigenschaft $P$ nicht hat.
Wäre dies nicht so, könnten
wir stattdessen $L_1=\emptyset$  wählen, und die Eigenschaft $\neg P$
untersuchen.
Aussserdem gibt es eine Turingmaschine $M_1$ mit $L(M_1)=L_1$.

Die Turingmaschine $M'$ operiert auf dem Input $u$ wie folgt:
\medskip
\begin{compactenum}
\item Lasse $M$ auf Input $w$ laufen
\item Falls $M$ das Wort $w$ akzeptiert, lasse $M_1$ auf $u$ laufen.
Falls $M_1$ im Zustand $q_{\text{accept}}$ anhält, halte ebenfalls
im Zustand $q_{\text{accept}}$ an.
\item In allen anderen Fällen: $q_{\text{reject}}$
\end{compactenum}
\medskip

Falls $w\in L(M)$ akzeptiert die Turingmaschine $M'$ genau die Wörter
$u\in L_1$.
Falls $w\not\in L(M)$ akzeptiert $M'$ überhaupt keine
Wörter, in diesem Fall ist also $L(M')=\emptyset$:
\begin{align*}
w\in L(M)&\qquad \Rightarrow\qquad L(M')=L_1\\
w\not\in L(M)&\qquad \Rightarrow\qquad L(M')=\emptyset
\end{align*}
Da $f$ berechnenbar ist, haben wir die verlangte Reduktion
gefunden, und es folgt, dass $P_{\text{TM}}$ nicht
entscheidbar ist.
\end{proof}

\subsubsection{Anwendungen}

\begin{beispiel}[\bf $\text{\textsl{ALL}}_{\text{TM}}$ ist nicht entscheidbar]
\index{ALLTM@$\textit{ALL}_{\text{TM}}$}%
$\text{\textsl{ALL}}_{\text{TM}}$ ist die Sprache
\[
\text{\textsl{ALL}}_{\text{TM}}=\{
\langle M\rangle\;|\; \text{$M$ ist eine TM und $L(M)=\Sigma^*$}
\}
\]
Der Satz von Rice verlangt, dass wir das Problem als eine Eigenschaft
der von einer TM erkannten Sprache formulieren.
\[
P=\text{\textsl{ALL}}:\text{Die erkannte Sprache ist $\Sigma^*$}
\]
Offensichtlich ist dies eine Eigenschaft der Sprache, und nicht zum
Beispiel der konkreten Implementation der TM.

Ausserdem verlangt der Satz von Rice, dass die Eigenschaft nicht trivial ist,
dass es also Sprachen gibt, die die Eigenschaft haben, und andere, die
sie nicht haben.
\begin{enumerate}
\item Die Sprache $\Sigma^*$ ist Turing-erkennbar und hat die Eigenschaft
$P=\text{\textsl{ALL}}$.
\item Die Sprache $\emptyset$ ist auch Turing-erkennbar, hat aber die
Eigenschaft
$P=\text{\textsl{ALL}}$ nicht.
\end{enumerate}
Damit ist klar, dass die Eigenschaft nicht trivial ist.

Jetzt kann der Satz von Rice angewandt werden, er besagt, dass die
Eigenschaft
$P=\text{\textsl{ALL}}$ nicht entscheidbar ist.
\end{beispiel}

\begin{beispiel}[\bf Primzahlprüfer] Es ist entscheidbar, ob eine Zahl $n$
prim ist oder nicht, man testet einfach jeden möglichen Teiler $<n$,
wenn immer ein Rest bleibt, ist die Zahl prim.
Dies ist natürlich
kaum der effizienteste Algorithmus, tatsächlich gibt es viele Alternativen,
die wir in eine Menge zusammenfassen können:
\[
\text{\textsl{PRIMALITY-TESTER}}=
\left\{\langle M\rangle\;\left|\;
\begin{minipage}{2.15truein}
\raggedright
$M$ ist eine Turingmaschine und\\
ein korrekter Primzahltester
\end{minipage}
\right.\right\}.
\]
\index{PRIMALITY-TESTER@$\textit{PRIMALITY-TESTER}$}%
\textsl{PRIMALITY-TESTER} enthält also genau diejenigen Programme, welche
als Primzahlprüfer korrekt funktionieren.

Gibt es ein Programm, welches beliebige Primzahlprüfer auf Korrektheit
testen kann? Wenn ja, dann ist \textsl{PRIMALITY-TESTER} entscheidbar.
Leider kann es kein solches Programm geben,
\textsl{PRIMALITY-TESTER} ist nicht entscheidbar, wie man mit dem Satz
von Rice einsehen kann.

Zunächst muss man wieder die Eigenschaft formulieren
\[
P: \text{Die akzeptierte Sprache ist die Menge der Primzahlen}
\]
Weiter muss es zwei Sprachen geben, die eine muss die Eigenschaft
$P$ haben, die anderen nicht:
\begin{enumerate}
\item
Der oben beschriebene primitive Primzahlprüfer, der alle möglichen Teiler
durchprobiert, akzeptiert genau die Primzahlen, die akzeptierte Sprache
hat also die Eigenschaft $P$.
\item 
Die leere Sprache $\emptyset$ ist Turing-erkennbar, hat aber ganz bestimmt
die Eigenschaft $P$ nicht.
\end{enumerate}
Die Eigenschaft $P$ ist also nicht trivial, und nach dem Satz von Rice
kann sie daher auch nicht entscheidbar sein.
\end{beispiel}

Die Beispiele illustrieren, dass es im Allgmeinen unmöglich ist, Programme
automatisch darauf zu testen, ob sie in allen Fällen korrekt arbeiten
werden.
Dies ist nur dann möglich, wenn die akzeptierte Sprache einfach,
zum Beispiel regulär, ist, oder nur für eine Teilmenge aller möglichen
Programmen.
