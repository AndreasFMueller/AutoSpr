%
%  entscheidbar.tex -- Entscheidbarkeit
%
% (c) 2019 Prof Dr Andreas Müller, Hochschule Rapperswil
%

\section{Entscheidbare Sprachen}
\rhead{Entscheidbare Sprachen}
\index{Sprache!entscheidbare}%
Wenn ein Problem von einem Computer gelöst werden soll, dann
muss der Computer die Problembeschreibung und eine mögliche Lösung
lesen können, und entscheiden können, ob die beiden zusammen passen.
Er muss also korrekte Problem- und Lösungs-Beschreibungen von
inkorrekten unterscheiden können.
Die Menge der korrekten
Beschreibungen bilden eine Sprache, das Problem ist also darauf
zurückgeführt, eine Sprache zu entscheiden.

\subsection{Entscheidbare Sprachen}
Eine Sprache heisst entscheidbar, wenn es einen Entscheider gibt, der
dazu verwendet werden kann, herauszufinden, ob ein Wort $w$ in der Sprache
liegt.

\begin{beispiel}
Sei $A$ ein DEA.
Dann kann man folgenden Algorithmus $M_A$ mit Input $w$ bauen:
\begin{compactenum}
\item Simuliere den DEA $A$ auf dem Input $w$
\item Falls $A$ in einem Akzeptierzustand steht: $q_{\text{accept}}$
\item Andernfalls: $q_{\text{reject}}$
\end{compactenum}
Dieser Algorithmus ist sicher ein Entscheider, er terminiert immer.
Ob er ein Wort $w$ akzeptiert, hängt davon ab, ob $A$ das Wort akzeptiert:
\begin{align*}
  A&\text{ akzeptiert $w$}&
   &\Leftrightarrow&
M_A&\text{ erkennt $w$}
\\
  A&\text{ akzeptiert $w$ nicht}&
   &\Leftrightarrow&
M_A&\text{ erkennt $w$ nicht}
\end{align*}
Das bedeutet aber, dass $L(A)=L(M_A)$, oder $M_A$ ist ein Entscheider für
die Sprache $L(A)$.
\end{beispiel}

Zu jeder regulären Sprache mit DEA $A$ gibt es also einen
Algorithmus $M_A$, der die gleiche Sprache erkennt: $L(A)=L(M_A)$.
Kürzer:

\begin{satz}
\label{regulaer-entscheidbar}
Reguläre Sprachen sind entscheidbar.
\end{satz}

\begin{beispiel}
Sei $G$ eine kontextfreie Grammatik.
Dann kann man den folgenden Algorithmus $M_G$ mit Input $w$ bauen:
\begin{compactenum}
\item Wandle $G$ in Chomsky-Normalform $G'$ um
\item Wende den CYK-Algorithmus (Satz \ref{cyk-algorithm}) auf $G'$ und das
Wort $w$ an
\item Wenn der CYK-Algorithmus das Wort $w$ akzeptiert: $q_{\text{accept}}$
\item Andernfalls: $q_{\text{reject}}$.
\end{compactenum}
Weil der CYK-Algorithmus (Satz \ref{cyk-algorithm}) deterministisch 
feststellen kann, ob ein Wort von einer Grammatik erzeugt wird, ist
$M_G$ ein Entscheider.
$M_G$ erkennt ein Wort genau dann, wenn des von
der Grammatik produziert wird, also $L(M_G)=L(G)$.
\end{beispiel}

Zu jeder kontextfreien Sprache mit Grammatik $G$ gibt es also einen 
Algorithmus $M_G$, der die gleiche Sprache erkennt: $L(G)=L(M_G)$.
Kürzer:

\begin{satz}
Kontextfreie Sprachen sind entscheidbar.
\end{satz}
Dieser Satz zeigt, dass die entscheidbaren Sprachen eine 
Obermenge der kontextfreien Sprachen sind, wie in folgender
Grafik dargestellt.
\begin{center}
%\includegraphics[width=0.7\hsize]{images/lang-2}
\includegraphics{images/lang-2}
\end{center}

\subsection{Entscheidbare Probleme für reguläre Sprachen}
Wir wollen jetzt eine Ebene höher geben.
Unser Fokus sind jetzt nicht
mehr einzelne Sprachen, sondern eine ganze Menge von Sprachen, und eventuell
weitere Daten.
Wir suchen also nicht mehr Algorithmen, die eine Frage
für eine einzelne Sprache beantworten können, sondern Algorithmen, die
die Frage gleich für mehrere Sprachen beantworten.

In Satz~\ref{regulaer-entscheidbar} ging es darum, einen Algorithmus
zu finden, der entscheidet, ob ein Wort $w$ von {\em einem} ganz bestimmten,
vorgegebenen endlichen Automaten $A$ akzeptiert wird.
Jetzt fragen wir, ob es einen Algorithmus gibt, der für {\em jede beliebige}
Kombination von DEA und Wort $w$ diese Entscheidung treffen kann.
Dies ist offenbar ein viel allgemeineres Problem, ob $A$ das Wort
akzeptiert ist ein Spezialfall.
Dieses Problem heisst das
Akzeptanzproblem für endliche Automaten.

\subsubsection{Akzeptanzproblem}
\index{Akzeptanzproblem!für reguläre Sprachen}%
Wir betrachten das folgende Problem: Gegeben sei ein endlicher Automat $B$
und ein Wort $w$,
kann man entscheiden, ob der Automat das Wort akzeptiert?
Um dies als Sprachproblem zu formulieren, braucht  man eine Codierung
$\langle B,w\rangle$ von Automat und Wort.
Die Sprache, die zu
entscheiden ist, lautet dann:
\[
A_{\text{DEA}} =\{
\langle B,w\rangle\;|\;\text{$B$ ist ein DEA und $B$ akzeptiert $w$}.
\}
\]
\index{ADEA@$A_{\text{DEA}}$}%
\begin{satz}
\label{adea_decidable}
$A_{\text{DEA}}$ ist entscheidbar.
\end{satz}
\index{Akzeptanzproblem!für DEAs}%

\begin{proof}[Beweis]
Der folgende Algorithmus entscheidet
$A_{\text{DEA}}$:
\medskip
\begin{compactenum}
\item Simuliere $B$ auf einer TM mit Input $w$.
\item Falls $B$ akzeptiert, gehe in den Zustand $q_{\text{accept}}$.
\item Falls $B$ nicht akzeptiert, gehe in den Zustand $q_{\text{reject}}$.
\end{compactenum}
\medskip
\end{proof}

Das gleiche Problem kann man auch für nicht deterministische Automaten
formulieren:

\begin{satz}
\index{Akzeptanzproblem!für NEAs}%
Das Akzeptanzproblem für nicht deterministische endlich Automaten
\[
A_{\text{NEA}} =\{
\langle B,w\rangle\;|\;\text{$B$ ist ein NEA und $B$ akzeptiert $w$}.
\}
\]
\index{ANEA@$A_{\text{NEA}}$}%
ist entscheidbar.
\end{satz}

\begin{proof}[Beweis]
Der folgende Algorithmus entscheidet 
$A_{\text{NEA}}$:
\medskip
\begin{compactenum}
\item Wandle $B$ in einen DEA $B'$ um.
\item Verwende die TM von Satz \ref{adea_decidable}, um zu
entscheiden, ob $B'$ das Wort $w$ akzeptiert.
\end{compactenum}
\medskip
\end{proof}

\begin{satz} Das Akzeptanzproblem für reguläre Ausdrücke
\index{Akzeptanzproblem!für regul\äre Ausdr\ücke}%
\[
A_{\text{REX}}=\{
\langle R,w\rangle\;|\;\text{$R$ ist ein regulärer Ausdruck und $R$ akzeptiert $w$}
\}
\]
\index{AREX@$A_{\text{REX}}$}%
ist entscheidbar.
\end{satz}

\subsubsection{Leerheitsproblem}
\index{Leerheitsproblem!für DEAs}%
Wir wollen einem endlichen Automaten ansehen können, ob er überhaupt
irgend ein Wort akzeptieren wird, und untersuchen daher
\[
E_{\text{DEA}}
=\{
\langle A\rangle \;|\;\text{$A$  ist ein DEA und $L(A)=\emptyset$}
\}.
\]
\begin{satz}
$E_{\text{DEA}}$
\index{EDEA@$E_{\text{DEA}}$}%
ist entscheidbar.
\end{satz}

\begin{proof}[Beweis]
Man muss herausfinden, ob überhaupt je ein Akzeptierzustand erreicht
werden kann.
Der folgende Algorithmus schafft dies:
\medskip
\begin{compactenum}
\item Markiere den Startzustand
\item Solange sich noch neue Zustände markieren liessen:
 Markiere alle Zustände, die sich von den bereits markierten aus
erreichen lassen.
\item Falls ein Akzeptierzustand markiert wurde: $q_{\text{reject}}$,
andernfalls
$q_{\text{accept}}$
\end{compactenum}
\medskip
\end{proof}

\subsubsection{Gleichheitsproblem}
\index{Gleichheitsproblem!für DEAs}%
Akzeptieren zwei DEAs die gleiche Sprache? Im Kapitel \ref{chapter-regular}
haben wir einen Algorithmus skizziert, mit dem dies entschieden werden kann.

\begin{satz}
\label{satz:eqdea}
Die Sprache
\[
\text{\it EQ}_{\text{DEA}}=\{
\langle A,B\rangle\;|\;\text{$A$ und $B$ sind DEAs und $L(A)=L(B)$}
\}
\]
\index{EQDEA@$\textit{EQ}_{\text{DEA}}$}%
ist entscheidbar.
\end{satz}

\begin{proof}[Beweis 1]
In Anlehnung an Kapitel \ref{chapter-regular} können wir den folgenden
Algorithmus verwenden:
\medskip
\begin{compactenum}
\item Erzeuge den minimalen Automaten $A'$ zu $A$.
\item Erzeuge den minimalen Automaten $B'$ zu $B$.
\item Falls $A'=B'$: $q_{\text{accept}}$, andernfalls $q_{\text{reject}}$.
\end{compactenum}
\medskip
Dieser Algorithmus entscheidet offenbar 
$\text{\it EQ}_{\text{DEA}}$.
\end{proof}

\begin{proof}[Beweis 2]
Noch etwas eleganter ist der folgende Beweis.
Ob $L(A)=L(B)$ ist
offenbar gleichbedeutend damit, dass die symmetrische Differenz
(Abbildung~\ref{symdiff})
\[
L(A){\;\Delta\;} L(B)=
(L(A)\setminus L(B)) \cup (L(B)\setminus L(A))
\]
leer ist.
Da sich die Mengenoperationen aber durch Operationen mit
endlichen Automaten abbilden lassen, gibt es einen Algorithmus,
der  zu jedem Paar $(A,B)$
von endlichen Automaten  einen endlichen Automaten $C$ konstruiert mit
$L(C)=L(A){\;\Delta\;} L(B)$.
Dann muss nur noch mit Hilfe eines Entscheiders
für $E_{\text{DEA}}$ entschieden werden, ob $\langle C\rangle\in
E_{\text{DEA}}$.
\end{proof}

Das Prinzip dieses zweiten Beweises wird uns später weiter beschäftigen.
Entscheidend war, dass es einen Algorithmus gibt, mit dem das
Entscheidungsproblem 
$\text{\it EQ}_{\text{DEA}}$
auf das Leerheitsproblem
$E_{\text{DEA}}$ zurückgeführt werden konnte.
Die Entscheidung erfolgte dann mit dem Entscheider für
$E_{\text{DEA}}$.

\subsection{Entscheibarkeitsproblem für kontextfreie Sprachen}
\subsubsection{Akzeptanzproblem}
\index{Akzeptanzproblem!für kontextfreie Grammatiken}%
\begin{satz}
\label{satz:acfg-entscheidbar}
Die Sprache
\[
A_{\text{CFG}}=\{
\langle G,w\rangle\;|\; \text{die kontextfreie Grammatik $G$ erzeugt $w$}
\}
\]
ist entscheidbar.
\index{ACFG@$A_{\text{CFG}}$}%
\end{satz}

\begin{proof}[Beweis 1]
Der CYK-Algorithmus von Satz \ref{cyk-algorithm} entscheidet
$A_{\text{CFG}}$.
\end{proof}

\begin{proof}[Beweis 2]
Der folgende Algorithmus entscheidet
$A_{\text{CFG}}$:
\medskip
\begin{compactenum}
\item Wandle $G$ in Chomsky Normalform um.
\item Erzeuge alle Wörter, die sich durch Anwendung
von $|w|-1$ Regeln der Form $A\to BC$ und $|w|$ Regeln
der Form $A\to a$ ableiten lassen.
\item Falls $w$ darunter vorkommt: $q_{\text{accept}}$, 
andernfalls $q_{\text{reject}}$.
\end{compactenum}
\medskip
\end{proof}

\subsubsection{Leerheitsproblem}
\index{Leerheitsproblem!für kontextfreie Grammatiken}%
\begin{satz}
Die Sprache
\[
E_{\text{CFG}}=\{
\langle G\rangle\;|\; \text{$G$ ist eine kontextfreie Grammatik und $L(G)=\emptyset$}
\}
\]
ist entscheidbar.
\index{ECFG@$E_{\text{CFG}}$}%
\end{satz}

\begin{proof}[Beweis]
Eine Grammatik erzeugt Wörter, wenn alle Variablen, die im Laufe
der Ableitung auftreten, am Ende mit Hilfe von Regeln der Form $A\to a$
in Terminalsymbole umgewandelt werden können.
Man könnte also einen
Markierungsalgorithmus entwickeln, der alle Terminalsymbole markiert,
und alle Symbole, die mit geeigneten Regeln ebenfalls markierte Symbole
erzeugen.
Falls die Startvariable nicht markiert wird, ist die Sprache leer.
Der Algorithmus 
\medskip
\begin{compactenum}
\item Wandle $G$ in Chomsky Normalform um.
\item Markiere alle Terminalsymbole.
\item Solange sich neue Variablen markieren lassen: markiere alle
Variablen $A$, zu denen es eine Regel gibt, so dass alle Symbole auf
der rechten Seite der Regel bereits markiert sind.
\item Falls $S$ markiert wurde: $q_{\text{reject}}$, andernfalls
$q_{\text{accept}}$.
\end{compactenum}
\medskip
entscheidet also
$E_{\text{CFG}}$.
\end{proof}

\subsubsection{Gleichheit}
\index{Gleichheitsproblem!für kontextfreie Grammatiken}%
Es ist verlockend zu vermuten, dass das Gleichheitsproblem
\[
\text{\it EQ}_{\text{CFG}}=\{
\langle G,H\rangle\;|\; \text{$G$  und $H$ sind kontextfreie Grammatiken und $L(G)=L(H)$}
\}
\]
\index{EQCFG@$\textit{EQ}_{\text{CFG}}$}%
mit der gleichen Methode mit Hilfe der symmetrischen Differenz
wie bei regulären Sprachen entschieden werden könnte.
Dem ist
allerdings nicht so, denn das Komplement einer kontextfreien Sprache
ist im Allgemeinen nicht kontextfrei, also auch die symmetrische
Differenz nicht.
In der Tat ist 
$\text{\it EQ}_{\text{CFG}}$ nicht entscheidbar, die Methoden zum Beweis
werden allerdings erst später bereitgestellt.

