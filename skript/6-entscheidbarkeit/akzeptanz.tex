%
% akzeptanz.tex -- Akzeptanzproblem für Turing-Maschinen
%
% (c) 2019 Prof Dr Andreas Müller, Hochschule Rapperswil
%
\section{Das Akzeptanzproblem für Turingmaschinen}
\index{Akzeptanzproblem!für Turingmaschinen}%
\rhead{Akzeptanzproblem für Turingmaschinen}
Das Akzeptanzproblem für Turing-erkennbare Sprachen fragt, ob 
eine Turingmaschine $M$ ein Inputwort $w$ erkennen wird.
Das Problem ist entscheidbar, wenn man aus der Beschreibung
der Maschine $M$ und dem Inputwort mit einem Algorithmus, der
für alle Turingmaschinen und alle Wörter immer anhält, ableiten
kann, ob $M$ das Wort $w$ akzeptieren wird.
Man kann also immer
entscheiden, ob $M$ im Zustand $q_{\text{accept}}$ anhalten wird.

Dabei können wir nicht einfach $M$ laufen lassen, weil $M$
ja auch in eine Endlosschleife fallen könnte, womit sich die
Frage nicht mehr entscheiden liesse.

Als Sprachproblem formuliert, suchen wir jetzt also einen
Entscheider für die Sprache
\[
A_{\text{TM}}=\{
\langle M,w\rangle\;|\; \text{$M$ ist eine TM und $M$ erkennt $w$}
\}
\]
\index{ATM@$A_{\text{TM}}$}%

\begin{satz}
\label{ATM}
$A_{\text{TM}}$ ist nicht entscheidbar.
\end{satz}

Auf den ersten Blick ist dieses Resultat sehr überraschend, warum
soll für Turing-Maschinen alles grundsätzlich anders sein, als
für endliche Automaten und Stackautomaten? Der wesentliche Unterschied
verbirgt sich in dem Wort ``entscheidbar'' selbst.
Ein Entscheider ist eine
TM, welche in diesem Fall Aussagen über andere TMs machen muss, also
zum Beispiel auch über sich selbst.
Bei DEAs und CFGs war die Sache
insofern einfacher, als eine TM Aussagen über einen DEA oder eine
CFG gemacht hat.

Ein solcher Selbstbezug kann einen in Teufels Küche bringen.
Nehmen wir an, in einem Dorf lebt ein Barbier, der genau diejenigen
Leute rasiert, die sich selbst nicht rasieren.
Soll er sich rasieren?
Wenn er sich rasiert, gehört er zu den Leuten, die sich selbst rasieren,
die rasiert er nicht, also darf er sich nicht rasieren.
Wenn er sich
aber nicht rasiert, gehört er zu den Leuten, die sich nicht selbst
rasieren, daher muss er sich rasieren.
Der Rückbezug ist, was den
Barbier in die Zwickmühle bringt.
Und genau so einen ``Barbier''
konstruiert der Beweis des Satzes~\ref{ATM}.

\begin{proof}[Beweis]
Wir führen den Beweis mit Hilfe eines Widerspruchs.
Dazu nehmen wir
an, es gäbe eine Turingmaschine $H$, welche das Akzeptanzproblem entscheidet.
Sie verhält sich wie folgt:
\[
H(\langle M,w\rangle)\begin{cases}
\text{erkennt}&\qquad\text{falls $M$ auf $w$ im Zustand $q_{\text{accept}}$ anhält}
\\
\text{verwirft}&\qquad\text{falls $M$ auf $w$ nicht oder im Zustand $q_{\text{reject}}$ anhält}
\end{cases}
\]
Daraus konstruieren wir jetzt eine neue TM $D$ mit Input $\langle M\rangle$:
\medskip
\begin{compactenum}
\item Lasse $H$ auf dem Input $\langle M,\langle M\rangle\rangle$ laufen
\item Falls $H$ erkennt: $q_{\text{reject}}$
\item Falls $H$ verwirft: $q_{\text{accept}}$
\end{compactenum}
\medskip
Die Maschine $D$ verhält sich wie folgt:
\[
D(\langle M\rangle)\begin{cases}
\text{erkennt}&\qquad\Leftrightarrow\qquad \text{$M$ verwirft $\langle M\rangle$}
\\
\text{verwirft}&\qquad\Leftrightarrow\qquad \text{$M$ erkennt $\langle M\rangle$}
\end{cases}
\]
($D$ entspricht dem Barbier, der genau die rasiert, die sich selbst
nicht rasieren.)

Lassen wir jetzt $D$ auf seiner eigenen Beschreibung operieren
(wir fragen uns, ob der Barbier sich selbst rasieren soll), muss
sich folgendes Resultat ergeben
\[
D(\langle D\rangle)\begin{cases}
\text{erkennt}&\qquad\Leftrightarrow\qquad \text{$D$ verwirft $\langle D\rangle$}
\\
\text{verwirft}&\qquad\Leftrightarrow\qquad \text{$D$ erkennt $\langle D\rangle$}
\end{cases}
\]
$D$ muss also genau das Gegenteil dessen tun, was $D$ tut.
$D$ ist der Barbier, der sich nicht entscheiden kann, ob er sich selbst
rasieren will.
Dieser
Widerspruch zeigt, dass die Annahme, es gäbe einen Entscheider $H$
nicht haltbar ist.
Also ist das Problem nicht entscheidbar.
\end{proof}

\index{Goedel@Gödel, Kurt}%
Die Entdeckung, dass gewisse Probleme nicht entscheidbar sind,
geht auf Kurt Gödel zurück.
Gödel hat sie jedoch in leicht anderem Zusammenhang gefunden.
Er untersuchte die Frage, ob ein
Axiomensystem eine Aussage beweisen kann.
Dabei zeigte sich,
dass es Aussagen geben muss, die nicht bewiesen werden können,
obwohl sie wahr sind.

Die Formulierung für Turing-Maschinen geht auf eine Arbeit von
Alan Turing 1936 zurück.
\index{Turing, Alan}%
Aus der Lösung des Halteproblems kann
man eine konkrete solche Aussage ableiten.

Am Input $w$ allein liegt es nicht, dass das Problem nicht
entschieden werden kann.
Nicht einmal für den speziellen Input $\varepsilon$ ist es
entscheidbar, auch die Sprache
\[
\textit{HALT}\varepsilon_{\text{TM}}
=\{
\langle M\rangle \;|\;
\text{$M$ ist eine TM und hält auf Input $\varepsilon$}
\}
\]
\index{HALTepsilonTM@$\textit{HALT}\varepsilon_{\text{TM}}$}%
ist nicht entscheidbar.
Nehmen wir an, es gäbe einen
Entscheider $H$ für $\textit{HALT}\varepsilon_{\text{TM}}$, dann können wir daraus
auch einen Entscheider für $A_{\text{TM}}$ konstruieren.
Dazu gehen wir wie folgt vor.
Auf dem Input $\langle M,w\rangle$ bauen wir folgende Maschine $M'$:
\medskip
\begin{compactenum}
\item Schreibe $w$ auf das Band
\item Lasse $M$ laufen
\item Falls $M$ erkennt: $q_{\text{accept}}$.
\item Falls $M$ verwirft: gehe in eine Endlosschleife.
\end{compactenum}
\medskip
Die Maschine $M'$ hält auf leerem Input genau dann, wenn $M$
das Wort $w$ erkennt.
Der Entscheider $H$, angewendet auf $M'$
entscheidet also, ob $w\in L(M)$, löst also das Akzeptanzproblem
$A_{\text{TM}}$.
Da $A_{\text{TM}}$ nicht entscheidbar ist, darf es keinen solchen
Entscheider geben.
Dieser Widerspruch zeigt, dass die Annahme, es
gäbe einen Entscheider für $\textit{HALT}\varepsilon_{\text{TM}}$ gibt,
nicht haltbar ist.
$\textit{HALT}\varepsilon_{\text{TM}}$ heisst
auch das {\em spezielle Halteproblem}.
\index{Halteproblem!spezielles}%

Das Halteproblem beweist, dass es Sprachen gibt, die nicht
entscheidbar sind, obwohl sie von einer Turing-Maschine
erkennbar sind.
Die Turing-erkennbaren Sprachen bilden also eine
echte Obermenge der entscheidbaren Sprachen.
\begin{center}
%\includegraphics[width=0.8\hsize]{images/lang-3}
\includegraphics{images/lang-3}
\end{center}

