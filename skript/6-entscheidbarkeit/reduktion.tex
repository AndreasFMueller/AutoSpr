%
% reduktion.tex -- Reduktion
%
% (c) 2019 Prof Dr Andreas Müller, Hochschule Rapperswil
%
\section{Reduktion}
\index{Reduktion}%
\rhead{Reduktion}
Im letzten Abschnitt wurde die Entscheidbarkeit der Sprache $L$
auf das Akzeptanzproblem in dem Sinne zurückgeführt, dass
aus einem Entscheider für $L$ ein Entscheider für das Akzeptanzproblem
konstruiert wurde.
Die Nichtentscheidbarkeit des Akzeptanzproblems
hat dann gezeigt, dass auch $L$ nicht entscheidbar sein kann.

Dieses Vorgehen funktioniert in sehr vielen Fällen, wir wollen es
daher etwas abstrakter formulieren und damit die Anwendung in weiteren
Beispielen vereinfachen.
Gleichzeitig ermöglicht es uns, die
Sprachen mindestens teilweise nach ihrer Schwierigkeit, sie zu entscheiden,
anzuordnen.

\subsection{Reduktionsabbildung}
\index{Reduktionsabbildung}%
\begin{figure}
\begin{center}
%\includegraphics[width=\hsize]{images/red-1}
\includegraphics{images/red-1}
\end{center}
\caption{Reduktionsabbildung $f\colon A\to B$.\label{reduktionsabbildung}}
\end{figure}
Wir möchten zwei verschiedene Sprachen 
$A$ und $B$ 
vergleichen.
Dazu bilden wir Wörter von der einen Sprache in
die andere Sprache ab, wir brauchen also eine Abbildung, die
$f\colon\Sigma^*\to \Sigma^*$, die $A$ in $B$ überführt,
d.\,h.
\[
w\in A\Leftrightarrow f(w)\in B.
\]
Diese Abbildung hilft uns allerdings nur dann, Entscheidbarkeitsfragen
von $B$ nach $A$ zu transportieren, wenn sie mit einer Turingmaschine
berechnet werden kann.
Wir definieren daher

\begin{definition}
\index{berechenbar}%
Eine Abbildung $f\colon \Sigma^*\to \Sigma^*$ heisst berechenbar wenn es eine
Turingmaschine gibt, die auf jedem Input $w\in \Sigma^*$ anhält
und ausschliesslich das Wort $f(w)$ auf dem Band zurücklässt.
\end{definition}

\begin{definition}
\index{Reduktion}%
Eine Sprache $A$ ist reduzierbar auf die Sprache $B$, in Zeichen
\[
A\le B
\]
wenn es eine
berechenbare Abbildung $f\colon \Sigma^*\to \Sigma^*$ gibt, mit
\[
w\in A\Leftrightarrow f(w)\in B.
\]
Wir schreiben dafür auch
\[
f\colon A\le B.
\]
\end{definition}
Die Schreibweise $A\le B$ soll ausdrücken, dass die Sprache $A$ ``leichter''
zu entscheiden ist als die Sprache $B$.
Die nachfolgenden zwei Sätze
zeigen, dass diese Leseart berechtigt ist.

\begin{satz}
Seien $A\le B$ Sprachen.
Ist $B$ entscheidbar, dann auch $A$.
\end{satz}

\begin{proof}[Beweis]
Ist $B$ entscheidbar, dann gibt es einen Entscheider für $B$, also
eine Turingmaschine $M$, die auf jedem Inputwort anhält, und
entscheidet, ob es zu $B$ gehört.
Daraus können wir jetzt einen Entscheider für $A$ konstruieren.
Um zu entscheiden, ob $w\in A$
ist, verwenden wir den folgenden Algorithmus $M'$ mit Input
$w$:
\medskip
\begin{compactenum}
\item Berechne $f(w)$, dafür gibt es wegen $A\le B$ eine Turingmaschine.
\item Verwende den Entscheider $M$ um zu entscheiden, ob $f(w)\in B$.
\item Falls $M$ akzeptiert ist $f(w)\in B\Rightarrow w\in A$
\item Falls $M$ nicht akzeptiert ist $f(w)\not\in B\Rightarrow w\not\in A$.
\end{compactenum}
\medskip
$M'$ ist offenbar ein Entscheider für $A$, also ist $A$ entscheidbar.
\end{proof}

Die Kontraposition dieser Aussage ist
\begin{satz}
Ist $A\le B$ und $A$ nicht entscheidbar, dann ist auch $B$ nicht
entscheidbar.
\end{satz}
Damit haben wir eine Methode, um Sprachen als nicht entscheidbar 
zu beweisen.
Wir müssen nur eine Abbildung $f\colon \Sigma^*\to \Sigma^*$
konstruieren, wobei $A$ ein bekanntermassen nicht
entscheidbares Problem ist, und $f$ berechenbare Abbildung ist,
die die Reduktion $A\le B$ vermittelt.

\subsection{Weitere nicht entscheidbare Sprachprobleme}
\index{Halteproblem}%
\begin{satz} Das Halteproblem
\[
\text{\it HALT}_{\text{TM}}=\{\langle M,w\rangle\;|\;
\text{$M$ ist eine TM und $M$ hält auf Input $w$}\}
\]
\index{HALTTM@$\textit{HALT}_{\text{TM}}$}%
ist nicht entscheidbar.
\end{satz}

\begin{proof}[Beweis]
Wir konstruieren eine Reduktion
$
A_{\text{TM}}
\le
\text{\it HALT}_{\text{TM}}
$.
Die Reduktion muss aus einer TM $M$ und einem Inputwort $w$ eine
neue TM $S$ und ein Inputwort $w$ machen, so dass $S$ auf
Input $w$ genau dann hält, wenn $M$ das Wort $w$ akzeptiert.
Die Abbildung $f$ ist also
\[
f\colon 
A_{\text{TM}}
\le
\text{\it HALT}_{\text{TM}}
\colon
\langle M,w\rangle\mapsto \langle S,w\rangle.
\]
$S$ konstruieren wir wie folgt:
\medskip
\begin{compactenum}
\item lasse $M$ laufen auf Input $w$
\item falls $M$ das Wort $w$ erkennt: $q_{\text{accept}}$
\item falls $M$ verwirft: beginne eine Endlosschleife
\end{compactenum}
\medskip
Offenbar ist
$\langle S,w\rangle\in \text{\it HALT}_{\text{TM}}$
genau dann, wenn $\langle M,w\rangle\in A_{\text{TM}}$.
Wir haben also
\[
\langle M,w\rangle\in A_{\text{TM}}
\qquad
\Leftrightarrow
\qquad
\langle S,w\rangle=f(\langle M,w\rangle)
\in \text{\it HALT}_{\text{TM}}.
\]
Somit ist 
$A_{\text{TM}}\le\text{\it HALT}_{\text{TM}}$, und da
$A_{\text{TM}}$ nicht entscheidbar ist, ist auch
$\text{\it HALT}_{\text{TM}}$ nicht entscheidbar.
\end{proof}
Statt des Anhaltens könnte man auch viele andere spezielle
Betriebszustände einer Turingmaschine als ``Haltekriterien''
heranziehen.
Dann bedeutet der Satz, dass es keinen allgemeinen
Algorithmus geben kann, der ein Stück Code inspizieren kann und
daraus ableiten kann, ob dieser Code auf einem System Schaden
anrichtet.
Echte Schadcode-Erkennung gibt es also grundsätzlich
nicht (auch wenn das Marketing-Material gewisser Produkte sich
da zu ganz erstaunlichen Aussagen versteigt).
Das beste, was man
tun kann, ist Mustererkennung (endliche Sprachen sind alle regulär).

\begin{satz}
\index{Leerheitsproblem!f\ür Turingmaschinen}%
Das Leerheitsproblem für Turingmaschinen
\[
E_{\text{TM}}
=
\{ \langle M\rangle\;|\; \text{$M$ ist eine TM und $L(M)=\emptyset$}\}
\]
\index{ETM@$E_{\text{TM}}$}%
ist nicht entscheidbar.
\end{satz}

\begin{proof}[Beweis]
Wir möchten eine Reduktion $A_{\text{TM}}\le E_{\text{TM}}$ konstruieren.
Dazu müssen wir aus $\langle M,w\rangle$ eine Maschine $S$ konstruieren,
die genau dann kein einziges Wort akzeptiert, wenn $M$ das Wort $w$
akzeptiert.
Es ist allerdings einfacher, die Negation zu behandeln,
also zu zeigen, dass 
\[
\bar E_{\text{TM}}=
\{ \langle M\rangle\;|\; \text{$M$ ist eine TM und $L(M)\ne\emptyset$}\}
\]
nicht entscheidbar ist.
Wir brauchen dann eine Reduktion
$A_{\text{TM}}\le \bar E_{\text{TM}}$, d.\,h.~eine Maschine $S$,
die genau dann mindestens ein Wort erkennt, wenn $M$ das Wort $w$ 
erkennt.

Die Maschine $S$ muss auf beliebigen Wörtern $u$ als Input ausgeführt
werden können.
Wir könnten folgenden Algorithmus verwenden:
\medskip
\begin{compactenum}
\item Falls $u\ne w$: $q_{\text{reject}}$.
\item Lasse $M$ auf $w$ laufen.
\item Falls $M$ erkennt: $q_{\text{accept}}$.
\end{compactenum}
\medskip
Wenn $M$ das Wort $w$ erkennt, dann erkennt $S$ die Sprache 
$L(S)=\{w\}$.
Wenn $M$ das Wort $w$ nicht erkennt, ist $L(S)=\emptyset$.
Oder
\begin{align*}
M&\operatorname{erkennt}w              &&\Rightarrow&L(S)&=\{w\}\ne \emptyset
\\
M&\operatorname{erkennt}w\text{ nicht} &&\Rightarrow&L(S)&=\emptyset
\end{align*}
oder
\[
\langle M,w\rangle \in A_{\text{TM}}
\qquad
\Leftrightarrow
\qquad
\langle S\rangle = f(\langle M,w\rangle)\in
\bar E_{\text{TM}},
\]
also $A_{\text{TM}}\le \bar E_{\text{TM}}$.
\end{proof}

\begin{satz}
Es ist nicht entscheidbar, ob eine Turingmaschine eine reguläre
Sprache erkennt.
\end{satz}

\begin{proof}[Beweis]
Als Sprache formuliert möchten wir eine Reduktion 
\[
A_{\text{TM}}\le\text{\it REGULAR}_{\text{TM}}=\{
\langle M\rangle\;|\;\text{$M$ ist eine TM und $L(M)$ ist regulär}
\}
\]
\index{REGULARTM@$\textit{REGULAR}_{\text{TM}}$}%
konstruieren.
Auch in diesem Fall ist es einfacher, eine
Reduktion
\[
A_{\text{TM}}\le\overline{\text{\it REGULAR}}_{\text{TM}}
\]
zu konstruieren.

Wir müssen also eine TM $S$ konstruieren, die genau
dann eine nicht reguläre Sprache erkennt, wenn $w$ von $M$ erkannt wird.
Der folgende Algorithmus tut dies.
Auf Input $u$ geht er wie folgt
vor:
\medskip
\begin{compactenum}
\item Falls $u$ nicht von der Form $0^n1^n$ ist: $q_{\text{reject}}$
\item Lasse $M$ auf $w$ laufen.
\item Falls $M$ $w$ erkennt: $q_{\text{accept}}$
\item Falls $M$ $w$ verwirft: $q_{\text{reject}}$
\end{compactenum}
\medskip
Wegen Schritt~1 kann $S$ höchstens die Wörter der Form
$\texttt{0}^n\texttt{1}^n$ erkennen, alle anderen werden bereits in
Schritt $1$ verworfen.
Ob die Wörter 
$\texttt{0}^n\texttt{1}^n$ erkannt werden, hängt davon ab, ob $M$ das Wort
$w$ erkennt:
\begin{align*}
   M&\text{ erkennt $w$}      &
    &\Leftrightarrow&
L(S)&=\{\texttt{0}^n\texttt{1}^n\,|\, n\ge 0\}&
    &\Leftrightarrow&
    &\text{$L(S)$ nicht regulär}\\
   M&\text{ erkennt $w$ nicht}&
    &\Leftrightarrow&
L(S)&=\{\quad\}=\emptyset&
    &\Leftrightarrow&
    &\text{$L(S)$ regulär}
\end{align*}
$S$ erkennt die nicht reguläre Sprache $\{0^n1^n\;|\;n\in\mathbb N\}$
genau dann, wenn $M$ das Wort $w$ erkennt.
Die Reduktion $\langle M,w\rangle\to \langle S\rangle$ übersetzt
``erkennen'' in ``nicht reguläre Sprache erkennen'', also also $A_{\text{TM}}$
in $\overline{\textsl{REGULAR}}_{\text{TM}}$.
\end{proof}

\index{Gleichheitsproblem!f\ür Turingmaschinen}%
\begin{satz}
Ob zwei Turingmaschinen die gleiche Sprache
\[
\text{\it EQ}_{\text{TM}}=\{
\langle M_1,M_2\rangle\;|\;\text{$M_i$ sind Turingmaschinen und $L(M_1)=L(M_2)$}
\}
\]
\index{EQTM@$\textit{EQ}_{\text{TM}}$}%
erkennen, ist nicht entscheidbar.
\end{satz}

\begin{proof}[Beweis]
Wir konstruieren eine Reduktion
$
E_{\text{TM}}\le \text{\it EQ}_{\text{TM}}
$
.
Dazu sei $M_0$ eine Turingmaschine, die jeden Input verwirft,
also $L(M_0)=\emptyset$.
Für die Abbildung $f$ verwenden wir
\[
f
\colon
E_{\text{TM}}\le \text{\it EQ}_{\text{TM}}
\colon
\langle M\rangle\mapsto \langle M,M_0\rangle
\]
Es ist $\langle M,M_0\rangle\in \text{\it EQ}_{\text{TM}}$
genau dann, wenn
$L(M)=L(M_0)=\emptyset$, also genau dann, wenn
$\langle M\rangle\in E_{\text{TM}}$.
\end{proof}

Dieser Satz hat zum Beispiel zur Folge, dass es keine zuverlässige
maschinelle
Möglichkeit gibt, herauszufinden, ob zwei Programme das gleiche tun.
Damit ist die Entscheidung, ob ein beliebiges Programm ein beliebiges
Softwarepatent verletzt, nur durch den subjektiven Prozess in Form
eines Gerichtes möglich, kann also grundsätzlich niemals objektiv sein.

