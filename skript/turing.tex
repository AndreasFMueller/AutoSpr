%
% Turing Maschinen und rekursiv aufzaehlbare Sprachen
%
% (c) 2009
%
\lhead{Algorithmen}
\chapter{Turing Maschinen\label{chapter-turing}}
Die bisher entwickelten Modelle von Rechenmaschinen haben zwar
interessante Sprachklassen hervorgebracht, sind aber offensichlich nicht
geeignet, die Möglichkeiten eines Computers abzubilden. Ein
Computer verfügt über einen Speicher, auf den er, im Gegensatz zum
Stack eines Stackautomen, beliebigen Zugriff hat. Das Ziel dieses
Abschnittes ist daher, einen endlichen Zustandsautomaten so zu
erweitern, dass er mit einem solchen Speicher arbeiten kann.
Weiterhin soll das Modell so formuliert sein, dass sich damit
Sprachen akzeptieren lassen, und wir hoffen natürlich, dass dadurch
eine nochmals grössere Klasse von Sprachen entsteht, die diesmal
aber alles umfasst, was wir mit modernen Computern abbilden können.

\section{Turing Maschinen}
\rhead{Turingmaschinen}
\subsection{Zwei Stacks: Random Access}
Ein Stack hat die Möglichkeiten eines endlichen Automaten dramatisch
erweitert, ein Stackautomat kann die bedeutend grössere Klasse der
kontextfreien Sprachen analysieren. Wir haben aber auch schon gesehen,
dass es nicht kontextfreie Sprachen gibt. Mit welcher möglichst kleinen
Erweiterung können wir eine Sprache wie $\{a^nb^nc^n\,|\, n\ge 0\}$
analysieren?

Eine erste Idee könnte sein, einen zweiten Stack hinzuzufügen.
Damit könnte man mit Sicherheit die Sprache $\{a^nb^nc^n\,|\,n\ge 0\}$
behandeln. Doch man fügt damit wesentlich mehr hinzu. Ein Buch ist
sicher ein Random-Access Speichermedium. Durch Blättern im Buch
kann man jede beliebige Seite erreichen. Wenn man aber ein Buch
aufschlägt, besteht es eigentlich nur aus zwei Stapeln von Blättern.
Blättern bedeutet nichts anderes, als dass man ein Blatt des Buches vom
einen Stapel wegnimmt und auf den anderen Stapel legt. Zwei Stacks
bilden also bereits ein Random-Access Medium.

Antike Schriftrollen sind auch Bücher, die aber nicht als Stapel
von Seiten organisiert sind, sondern als langes Band. Es ist aber klar,
dass man durch zerschneiden und binden jede Schriftrolle in ein
Buch verwandeln kann, und ein Buch durch zusammenkleben der einzelnen
Seiten zu einem Band in eine Schriftrolle.
Die heutigen Bücher (Kodices) haben sich gegenüber den Schriftrollen
etwa im ersten Jahrhundert n.~Chr. durchgesetzt.
Für unsere Anwendung ist die Vorstellung eines Random-Access Mediums
als endloses Band von Speicherzellen praktischer. Sie passt besser zu
unserem Bild vom Hauptspeicher eines Computers oder von Massenspeichermedien.
\subsection{Definition}
\index{Band}%
\index{Bandalphabet}%
\index{Schreib-/Lese-Kopf}%
Wir wollen einem endlichen Automaten statt eines Stacks einen unendlich
grossen Speicher zur Verfügung stellen. Wir stellen uns diesen
Speicher als ein in beide Richtungen unendlich langes Band vor,
welches in einzelne Speicherzellen aufgeteilt ist, in die jeweils
genau ein Zeichen geschrieben werden kann (Abbildung~\ref{turingfig}).
Wie beim Stackautomaten gehen
wir davon aus, dass dieser Speicher auch Dinge speichern kann, die nicht
im Input vorkommen, wir verwenden daher ein Bandalphabet $\Gamma$, welches
das Input-Alphabet umfasst: $\Gamma\supset \Sigma$.

\begin{figure}
\begin{center}
%\includegraphics[width=0.8\hsize]{images/turing-1}
\includegraphics{images/turing-1}
\end{center}
\caption{Schematische Darstellung einer Turing-Maschine\label{turingfig}}
\end{figure}

Der Zugriff auf den Speicher erfolgt jeweils zellenweise, wir stellen
uns einen Schreib-/Lese-Kopf vor, der auf dem Band positioniert werden
kann, und genau ein Zeichen lesen kann, oder das auf dem Band vorhandene
Zeichen überschreiben kann. Um zu Beginn einen wohldefinierten Zustand
zu haben, brauchen wir ein zusätzliches ``Blank''-Zeichen $\blank$,
welches in $\Gamma$ aber nicht in $\Sigma$ enthalten ist\footnote{Wäre
das ``Blank''-Zeichen $\blank$ auch in $\Sigma$, könnte man das
leere Band nicht vom Input unterscheiden.}.

Da wir jetzt diesen unendlich grossen Speicher haben, gibt es keinen
Grund mehr, sich vorzustellen, dass der Input Zeichen für Zeichen
in den endlichen Automaten gefüttert wird. Stattdessen gehen wir
davon aus, dass der Input auf das Band geschrieben wird, der
Schreib-/Lesekopf auf das erste Feld des Input positioniert wird,
und man dann die Maschine arbeiten lässt.

Die Maschine muss jetzt Zeichen vom Band lesen, und zusammen mit
ihrem aktuellen Zustand entscheiden, in welchem neuen Zustand
sie nach der Verarbeitung dieses Zeichens sein soll, welches Zeichen
an dieser Stelle auf dem Band stehen soll, und ob der Schreib-/Lesekopf
bewegt werden soll. Wir erlauben nur zwei Bewegungen: L und R, insbesondere
ist es nicht zulässig, dass der Kopf nach der Verarbeitung stehen
bleibt.

Im Gegensatz zu DEA oder Stackautomat ist die Berechnung in diesem
Modell nach dem Ende des Input-Strings noch nicht zu Ende. Die Maschine
kann mehrmals über die auf dem Band gespeicherten Daten fahren und
sie immer wieder modifizieren. Sie könnte also im Prinzip auch
gar nie aufhören zu arbeiten. Daher brauchen wir Zustände, die
die Maschine anhalten. Wenn die Maschine angehalten hat, müssen wir
zudem wissen, ob das auf das Band geschriebene Input-Wort für die
Sprache akzeptabel war oder nicht.

All dies führt uns auf folgende formale Definition einer Turing-Maschine
\begin{definition}
\index{Turing-Maschine}%
Eine Turing-Maschine ist ein $7$-tupel
$M=(Q,\Sigma,\Gamma,\delta,q_0,q_{\text{accept}},q_{\text{reject}})$,
wobei $Q$, $\Sigma$ und $\Gamma$ endliche Mengen sind mit folgenden
zusätzlichen Eigenschaften
\begin{compactenum}
\index{Zustand}%
\item $Q$ heisst die Menge der Zustände
\index{Inputalphabet}%
\index{Eingabe-Alphabet}%
\index{blank@Blank}%
\index{$\blank$|see{Blank}}%
\item $\Sigma$ heisst das Inputalphabet, es enthält das spezielle
``Blank''-Zeichen $\blank$ nicht.
\index{Bandalphabet}%
\item $\Gamma$ ist das Bandalphabet, es gilt $\blank\in\Gamma$ und
$\Sigma\subset\Gamma$.
\index{Uebergangsfunktion@Übergangsfunktion}%
\item $\delta\colon Q\times \Gamma\to Q\times\Gamma\times\{\text{L},\text{R}\}$
ist die Übergangsfunktion
\index{Startzustand}%
\item $q_0\in Q$ ist der Startzustand
\index{Akzeptierzustand}%
\item $q_{\text{accept}}\in Q$ ist der Akzeptierzustand
\index{Ablehnungszustand}%
\item $q_{\text{reject}}\in Q$ ist der Ablehnungszustand
\end{compactenum}
\end{definition}
Die Übergangsfunktion liefert also zu jedem Zustand und zum
Inhalt des Feldes unter dem Schreib-/Lesekopf einen neuen
Zustand, einen neuen Bandinhalt und die Information, ob
der Kopf nach links (L) oder rechts (R) bewegt werden muss.

\subsection{Sprache}
Sei eine Turingmaschine
$M=(Q,\Sigma,\Gamma,\delta,q_0,q_{\text{accept}},q_{\text{reject}})$
gegeben.
Zu $M$ kann man auf folgende Weise eine Sprache konstruieren:
Um zu entscheiden, ob ein Wort $w\in\Sigma^*$
zu der Sprache gehört,
schreibt man es auf das Band, platziert den Schreib-/Lesekopf auf
das erste Zeichen und bringt die Maschine in den Zustand $q_0$.
\index{Konfiguration}%
Man beschreibt diese Konfiguration der Maschine mit der Zeichenkette
$q_0w$, die Maschine ist im Zustand $q_0$ und der Schreib-/Lesekopf
steht über dem ersten Zeichen von $w$.

Mit Hilfe der Übergangsfunktion wird dann der nächste Zustand,
der neue Inhalt des Feldes unter dem Schreib-/Lesekopf und die
Kopfbewegung ermittelt. Auch diesen neuen Zustand kann man wieder
als Konfiguration hinschreiben. Der Schreib-/Lesekopf steht
jetzt womöglich mitten im Wort drin, und die Maschine befindet
sich in irgend einem Zustand $q$: $w_1qw_2$.

Nehmen wir an, die Maschine befindet sich zu Beginn eines Zyklus im
Zustand $q_1$ und hat den Kopf auf das Zeichen $a_k$ positioniert.
Die Funktion $\delta$ berechnet daraus ein Tripel $(q_1,b_k,R)$
oder $(q_2,b_k,L)$. Die Konfiguration ändert sich dabei wie
folgt:
\begin{align*}
&\text{Ausgangskonfiguration:}&a_1\dots a_{k-1}&\;q_1\;a_k\dots a_n\\
&\text{Übergang mit } \delta(q_1,a_k)=(q_2,b_k,L)&a_1\dots a_{k-2}&\;q_2\;a_{k-1}b_k\dots a_n\\
&\text{Übergang mit } \delta(q_1,a_k)=(q_2,b_k,R)&a_1\dots a_{k-1}b_k&\;q_2\;a_{k+1}\dots a_n
\end{align*}

Die Maschine arbeitet so immer weiter, bis einer der Zustände $q_{\text{accept}}$
oder $q_{\text{reject}}$ erreicht wird, dann hält sie an.
Das Inputwort gilt als akzeptiert, wenn der Zustand $q_{\text{accept}}$
erreicht wurde, es gilt als verworfen, wenn $q_{\text{reject}}$ erreicht
wurde.

\index{Berechnungsgeschichte}%
Die Folge von Konfigurationen, die die Turing-Maschine
während der Berechnung durchläuft, nennt man auch
{\em Berechnungsgeschichte}.

\begin{definition}
\index{Sprache!von einer Turing-Maschine erkannte}%
Ist $M$ eine Turingmaschine, dann heisst
\[
L(M)=\{w\in\Sigma^*\;|\;\text{$M$ akzeptiert $w$}\}.
\]
die von $M$ erkannte Sprache.
\end{definition}

\begin{definition}
\index{Turing-erkennbar}%
Eine Sprache $L$ heisst Turing-erkennbar, wenn es eine Turing-Maschine
$M$ gibt mit $L=L(M)$.
\end{definition}

Die Definition der erkannten Sprache erlaubt, dass die zur Erkennung
verwendete Turing-Maschine auf einigen Wörtern $w\in\Sigma^*$ nicht
anhält. Diese Wörter werden nie akzeptiert, und gehören
damit nicht zur Sprache. Testet man sie mit $M$, wird die Berechnung
aber nie anhalten, man weiss also eigentlich nie, ob das Wort
nicht zur Sprache gehört, oder ob man einfach noch etwas mehr
Geduld braucht.

Gewissheit hat man erst, wenn man sicher sein kann, dass $M$ auf
jedem Input anhält.

\begin{definition}
\index{Entscheider}%
Ein Entscheider ist eine Turingmaschine, die auf jedem Input $w\in\Sigma^*$
anhält. Eine Sprache heisst entscheidbar, wenn ein Entscheider sie
erkennt.
\index{entscheidbar}%
\end{definition}
Es ist klar, dass eine Turing-entscheidbare Sprache auch Turing-erkennbar
ist.
Die Definition der Eigenschaft ``Turing-entscheidbar'' unterscheidet sich
von der Definition der Eigenschaft ``Turing-erkennbar'' nur dadurch,
dass bei einer entscheidbaren Sprache die Turing-Maschine auf jedem
beliebigen Input anhalten muss, während bei einer nur erkennbaren
Sprache einzelne Input-Wörter auch dazu führen können, dass die
Turing-Maschine endlos weiterrechnet.

\begin{beispiel}
Die kleinsten Mersenne-Primzahlen sind $3=2^2-1$, $7=2^3-1$, $31=2^5-1$,
andererseits ist $2^{11}-1=2047=23\cdot 89$ keine Primzahl.
Das finden von Mersenneschen Primzahlen ist mit GIMPS (Great Internet
Mersenne Prime Search \cite{skript:gimps}) zu einem Volkssport geworden.
Es ist nicht bekannt, ob es unendlich viele Mersennesche Primzahlen gibt.

Wir bilden jetzt die folgende Sprache über dem Alphabet
$\Sigma=\{\texttt{0},\texttt{1}\}$:
\[
L=\{w\in\Sigma^*\,|\,\text{die Anzahl Mersennescher Primzahlen ist grösser als $w$}\}.
\]
Ein Turing-Maschinen-Programm, welches prüfen muss, ob $w\in L$ ist,
muss herausfinden, ob es mindestens $w$ (interpretiert als Zahl)
Mersennsche Primzahlen gibt.
Dazu müssen der Reihe nach alle Zweierpotenzen $2^k$ daraufhin getestet werden,
ob $2^k-1$ eine Primzahl ist.
Wenn mindestens $n$ Mersennsche Primzahlen gefunden wurden, wird $n$
akzeptiert, sonst nicht. Falls $n\in L$ ist, wird der Algorithmus
die Zahl $n$ akzeptieren, die Sprache $L$ ist also Turing-erkennbar.

Falls es nur endlich viele Mersennesche Primzahlen gibt, zum Beispiel
nur $N$, dann wird dieser Algorithmus für jede Zahl $n>N$ nicht
aufhören zu laufen.
Der Algorithmus kann ja nicht abschätzen, ob noch eine Mersennesche
Primzahl kommen wird, ob es also einfach noch etwas mehr Geduld
braucht.
Diese Sprache $L$ ist also nicht Turing-entscheidbar.
\end{beispiel}

\subsection{Zustandsdiagramm}
\index{Graph!gerichteter!beschrifteter!einer Turing-Maschine}%
Auch eine Turingmaschine kann man als gerichteten beschrifteten Graphen
darstellen. Übergänge zwischen zwei Zuständen sind immer begleitet von
einer Änderung des Inhaltes des Feldes unter dem Schreib-/Lesekopf
und von einer Kopfbewegung. Beide müssen mit der Beschriftung
der Pfeile codiert werden. In Anlehnung an die bei den Stackautomaten
verwendete Notation schreiben wir für den Übergang
$\delta(q_1,a)=(q_2,b,\text{R})$
die Notation
\[
\entrymodifiers={++[o][F]}
\xymatrix @+5mm{
{q_1}\ar[r]^{a\to b,\text{R}}
	&{q_2}
}
\]
und für den Übergang
$\delta(q_1,a)=(q_2,b,\text{L})$
\[
\entrymodifiers={++[o][F]}
\xymatrix @+5mm{
{q_1}\ar[r]^{a\to b,\text{L}}
	&{q_2}
}
\]

\subsection{Beispiel}
Wir konstruieren eine Turingmaschine, welche die bereits als weder regulär
noch kontextfrei erkannte Sprache $L=\{\texttt{0}^{2^n}\;|\; n\in\mathbb N\}$
erkennt.

Zweierpotenzen kann man daran erkennen, dass man die Zahl ohne
Rest durch zwei teilen kann, bis nur noch 1 übrig bleibt. Genau
diesen Prozess kann man mit einer Turingmaschine auf dem Band
nachbilden. Die Maschine muss also folgendes tun:

Auf dem Input $w$:
\begin{compactenum}
\item Fahre von links nach rechts über das Band und streiche jedes zweite $\texttt{0}$.
\item Falls im Schritt 1 das Band genau eine $\texttt{0}$ enthielt, {\it akzeptiere}.
\item Falls im Schritt 1 das Band eine ungerade Zahl und mehr als ein $\texttt{0}$
enthielt, {\it verwerfe}.
\item Fahre mit dem Kopf zurück zum linken Ende des Bandes
\item Weiter bei Schritt 1
\end{compactenum}
In jedem Durchgang wird die Anzahl der Nullen halbiert, es sei denn,
sie war nicht gerade. In diesem Fall bleibt nach dem Durchgang eine
ungerade Anzahl $\texttt{0}$ auf dem Band stehen, was beim darauffolgenden Durchgang
in Schritt~3 erkannt wird. Geht die Division immer auf, bleibt am Schluss
nur genau eine $\texttt{0}$ stehen, was im Schritt~2 erkannt wird. Daher funktioniert
dieser Algorithmus.

Um den Algorithmus mit einer Turing-Maschine zu realisieren, brauchen wir
ein neues Zeichen {\tt x}, mit dem wir gestrichene Zeichen markieren
können. Es ist also $\Gamma = \{\texttt{0},{\tt x},\blank\}$.
Ausserdem brauchen wir
Zustände, mit denen man die Parität der Anzahl $\texttt{0}$ bestimmen kann, und
mit denen man die Aktivitäten in Schritt 4 von den anderen unterscheiden
kann. Schritt $4$ wird zum Beispiel implementiert durch das
Zustandsdiagramm
\[
\entrymodifiers={++[o][F]}
\xymatrix @+5mm{
{}\ar@(ul,ur)^{\genfrac{}{}{0pt}{1}{\texttt{0}\to \texttt{0},\text{L}}{{\tt x}\to{\tt x},\text{L}}}
\ar[r]^{\blank\to\blank,\text{R}}
	&
}
\]
Jede zweite Null streichen wird implementiert durch
\[
\entrymodifiers={++[o][F]}
\xymatrix @+5mm{
*+\txt{}\ar[d]
\\
\ar@/^/[r]^{\texttt{0}\to{\tt x},\text{R}}
\ar@(ul,dl)_{\tt x\to\tt x,\text{R}}
	&\ar@/^/[l]^{\texttt{0}\to \texttt{0},\text{R}}
         \ar@(ur,dr)^{\tt x\to\tt x,\text{R}}
	 \ar[d]^{\blank\to\blank,?}
\\
*+\txt{}
	&{}
}
\]
Dass akzeptiert werden soll, wenn nach dem Ersetzen einer $\texttt{0}$ durch
ein {\tt x} keine weiteren $\texttt{0}$ mehr gefunden werden können, kann durch
\[
\entrymodifiers={++[o][F]}
\xymatrix @+10mm{
\ar[r]^{\texttt{0}\to{\tt x},\text{R}}
	&{} \ar@(dl,dr)_{{\tt x}\to{\tt x},\text{R}}
	    \ar[r]^{\blank\to\blank,\text{R}}
		&*++[o][F=]{q_{\text{accept}}}
}
\]
ausgedrückt werden. Die Schleife beim mittleren Zustand
bildet ab, dass {\tt x}-Zeichen übersprungen werden sollen.
Mit solchen Elementen können wir die Turingmaschine jetzt schrittweise
aufbauen.

Zunächst stellen wir sicher, dass der Input tatsächlich mit \texttt{0}
beginnt:
\[
\entrymodifiers={++[o][F]}
\xymatrix{
*+\txt{}\ar[d]
\\
{q_1}\ar[d]_{\genfrac{}{}{0pt}{1}{\blank\to\blank,\text{R}}{{\tt x}\to{\tt x},\text{R}}}
\\
*++[o][F=]{q_\text{reject}}
}
\]
Jetzt hängen wir das Segment an, welches prüft, ob es noch genau ein $\texttt{0}$
auf dem Band hat:
\[
\entrymodifiers={++[o][F]}
\xymatrix{
*+\txt{}\ar[d]
\\
{q_1}\ar[d]_{\genfrac{}{}{0pt}{1}{\blank\to\blank,\text{R}}{{\tt x}\to{\tt x},\text{R}}}
	\ar[r]^{\texttt{0}\to{\tt x},\text{R}}
	&{q_2}\ar[d]^{\blank\to\blank,\text{R}}
              \ar@(u,ul)_{{\tt x}\to {\tt x},\text{R}}
\\
*++[o][F=]{q_{\text{reject}}}
	&*++[o][F=]{q_\text{accept}}
}
\]
Falls dies nicht zutrifft, müssen weitere Nullen weggestrichen werden,
wobei über die Parität der noch vorhandenen Nullen Buch geführt werden
muss. Dafür braucht es die beiden Zustände $q_3$ und $q_4$:
\[
\entrymodifiers={++[o][F]}
\xymatrix{
*+\txt{}\ar[d]
\\
{q_1}\ar[d]_{\genfrac{}{}{0pt}{1}{\blank\to\blank,\text{R}}{{\tt x}\to{\tt x},\text{R}}}
	\ar[r]^{\texttt{0}\to\blank,\text{R}}
	&{q_2}\ar[d]_{\blank\to\blank,\text{R}}
	      \ar[r]^{\texttt{0}\to{\tt x},\text{R}}
              \ar@(u,ul)_{{\tt x}\to {\tt x},\text{R}}
		&{q_3}\ar@(u,r)^{{\tt x}\to{\tt x},\text{R}}
		      \ar@/^/[d]^{\texttt{0}\to \texttt{0},\text{R}}
\\
*++[o][F=]{q_\text{reject}}
	&*++[o][F=]{q_\text{accept}}
		&{q_4}\ar@/^/[u]^{\texttt{0}\to {\tt x},\text{R}}
		      \ar@(d,r)_{{\tt x}\to{\tt x},\text{R}}
}
\]
Wenn die Maschine nach dem Abarbeiten aller \texttt{0} und {\tt x} auf einen
neuen Blank \textvisiblespace\ trifft, ist sie entweder im Zustand
$q_3$ oder in $q_4$. Im letzten Fall hat sie eine ungerade Anzahl
von \texttt{0} gefunden, gemäss Schritt 3 des Algorithmus muss also verworfen
werden:
\[
\entrymodifiers={++[o][F]}
\xymatrix{
*+\txt{}\ar[d]
\\
{q_1}\ar[d]_{\genfrac{}{}{0pt}{1}{\blank\to\blank,\text{R}}{{\tt x}\to{\tt x},\text{R}}}
	\ar[r]^{\texttt{0}\to\blank,\text{R}}
	&{q_2}\ar[d]_{\blank\to\blank,\text{R}}
	      \ar[r]^{\texttt{0}\to{\tt x},\text{R}}
              \ar@(u,ul)_{{\tt x}\to {\tt x},\text{R}}
		&{q_3}\ar@(u,r)^{{\tt x}\to{\tt x},\text{R}}
		      \ar@/^/[d]^{\texttt{0}\to \texttt{0},\text{R}}
\\
*++[o][F=]{q_\text{reject}}
	&*++[o][F=]{q_\text{accept}}
		&{q_4}\ar@/^/[u]^{\texttt{0}\to {\tt x},\text{R}}
		      \ar@(d,r)_{{\tt x}\to{\tt x},\text{R}}
		      \ar@/^20pt/[ll]^{\blank\to\blank,\text{R}}
}
\]
Wenn sie dagegen im Zustand $q_3$ ist, muss sie nach links fahren,
bis zum ersten Blank, und dann im Zustand $q_2$ ankommen. Dies erreicht
man mit Hilfe eines weitere Zustands $q_5$.
\[
\entrymodifiers={++[o][F]}
\xymatrix{
*+\txt{}\ar[d]
	&*+\txt{}
		&{q_5}\ar@(ul,ur)^{\genfrac{}{}{0pt}{1}{\texttt{0}\to \texttt{0},\text{L}}{{\tt x}\to {\tt x},\text{L}}}
		      \ar[dl]_{\blank\to\blank,\text{R}}
\\
{q_1}\ar[d]_{\genfrac{}{}{0pt}{1}{\blank\to\blank,\text{R}}{{\tt x}\to{\tt x},\text{R}}}
	\ar[r]^{\texttt{0}\to\blank,\text{R}}
	&{q_2}\ar[d]_{\blank\to\blank,\text{R}}
	      \ar[rr]^{\texttt{0}\to{\tt x},\text{R}}
              \ar@(u,ul)_{{\tt x}\to {\tt x},\text{R}}
		&*+\txt{}
			&{q_3}\ar@(u,r)^{{\tt x}\to{\tt x},\text{R}}
			      \ar@/^/[d]^{\texttt{0}\to \texttt{0},\text{R}}
			      \ar[ul]_{\blank\to\blank,\text{L}}
\\
*++[o][F=]{q_\text{reject}}
	&*++[o][F=]{q_\text{accept}}
		&*+\txt{}
			&{q_4}\ar@/^/[u]^{\texttt{0}\to {\tt x},\text{R}}
			      \ar@(d,r)_{{\tt x}\to{\tt x},\text{R}}
			      \ar@/^20pt/[lll]^{\blank\to\blank,\text{R}}
}
\]
Mit diesem Beispiel haben wir gezeigt, dass die Menge der
Sprachen, die von einer Turingmaschine erkennbar sind, Sprachen
enthält, die mit den bisherigen Mitteln nicht erkennbar waren.

\begin{definition}
Die von einer Turingmaschine $M$ erkannten Wörter bilden die
Sprache $L(M)$.
\end{definition}

\section{Varianten von Turing Maschinen}
\rhead{Varianten von Turingmaschinen}
Turingmaschinen sollen als universelles Modell für Computer gelten,
dies ist jedoch nur möglich, wenn sich die Menge der erkannten Sprachen
nicht ändert, wenn an dem Modell kleine Veränderungen vorgenommen
werden.

Das Band einer Turingmaschine können wir als den RAM-Speicher
eines modernen Computers interpretieren. Moderne Computer verwenden
jedoch verschiedene Wortlänge bei der Arbeit mit ihrem Speicher,
was auch das Bandalphabet $\Gamma$ verändert. Ein Computer mit
Speicherwortlänge $l$ hat als Bandalphabet die Menge $\Gamma_l=[2^l]$.

\index{von Neumann-Architektur}%
\index{Harvard-Architektur}%
Computer mit von Neumann-Architektur verwenden nur einen einzigen RAM-Speicher,
der Programm und Daten enthält. Computer mit Harvard-Architektur
verwenden dagegen zwei verschiedene Speicher: einen Programm-Speicher
und einen Datenspeicher. Die beiden Speicher können sogar verschiedene
Wortlänge haben. AVR-Microcontroller verwenden zum Beispiel
den Flashspeicher, den sie als 16 Bit breiten Speicher adressieren,
während sie das RAM als 8 Bit breiten Speicher adressieren.
ARM-Microcontroller dagegen betrachten RAM und Flash einfach als
verschiedene Bereich in einem einzigen grossen Adressraum.
AVR-Microcontroller haben also ``zwei Bänder'' mit unterschiedlichen
Bandalphabeten, ARM-Microcontroller verwenden dagegen ein einziges Band
mit immer dem gleichen Alphabet.

\subsection{Mehrspurige Turingmaschine}
\index{Turing-Maschine!mehrspurige}%
\begin{satz}\label{mehrspurigeturingmaschine}
Jede Sprache, die von einer mehrspurigen Turingmaschine
erkannt werden kann, kann auch von einer einspurigen Turingmaschine
erkannten werden.
\end{satz}

\begin{proof}[Beweis]
Sei $L$ eine Sprache, die von der Turingmaschine $M$ mit $n$ Spuren
erkannten wird. Das Bandalphabet ist $\Gamma$. Wir konstruieren aus
$M$ eine neue Turingmaschine $M'$, welche nur noch eine Spur hat,
jedoch als Bandalphabet die Menge $\Gamma^n$. Dadurch wird die ``Wortbreite''
des Bandes erhöht, es wird nur noch eine, breite Spur verwendet, um die
gleiche Information unterzubringen.
\end{proof}

\subsection{Turingmaschine mit mehreren Bändern}
\index{Turing-Maschine!mit mehreren Bändern}%
\begin{figure}
\begin{center}
%\includegraphics[width=0.9\hsize]{images/turing-2}
\includegraphics{images/turing-2}
\end{center}
\caption{Turingmaschine mit mehreren Bändern\label{multitapetm}}
\end{figure}
\begin{satz}
\label{mehrbandturingmaschine}
Jede Sprache, die von einer Turingmaschine mit mehreren Bändern
(Abbildung~\ref{multitapetm})
erkannt
wird, kann auch von einer Turing-Maschine mit nur einem Band erkannt
werden.
\end{satz}

\begin{proof}[Beweis]
Die Konfiguration einer Turingmaschine $M$ mit $n$ Bändern beinhaltet
ausser dem Inhalt der zusätzlichen Bänder auch noch die Position
des Schreib-/Lesekopfes jedes einzelnen Bandes. Um diese Information
zu codieren, konstruieren wir eine neue Turingmaschine $M'$ mit
$2n$ Spuren. Spur $2i$ enthält dabei die Daten von Band $i$.
Auf Spur $2i-1$ verwenden wir eine spezielle Marke $\uparrow$, um die Position
des Schreib-/Lesekopfes auf Band $i$ zu markieren. Somit lässt sich
eine Turingmaschine mit $n$ Bändern auf einer Turingmaschine mit
$2n$ Spuren und Bandalphabet $\Gamma\cup\{\uparrow\}$ codieren.
Und nach Satz \ref{mehrspurigeturingmaschine} lässt sich eine solche wiederum
auf einer Standardturingmaschine simulieren.
\end{proof}
Die geraden Spuren werden also dazu verwendet, die Position des
Schreib-/Lesekopfes zu speichern, während die ungeraden Spuren
die Daten enthalten:
\begin{center}
\begin{tabular}{c|c|c|c|c|c|c|c|c}
\hline
&a&b&c&d&e&f&c&\\
\hline
& & &$\uparrow$& & & & &\\
\hline
&1&2&3&4&5&6&7&\\
\hline
& & & & & &$\uparrow$& &\\
\hline
\end{tabular}
\end{center}

\subsection{Bandalphabet}
Auch das Bandalphabet hat keinen Einfluss auf die Möglichkeiten einer
Turingmaschine. Die Festplatte eines modernen Computers ist eigentlich
ein ``Band'' mit Bandalphabet $\{\texttt{0},1,\blank\}$, erst zusätzliche
Logik im Controller macht daraus einzelne Bytes, also ein grössers
Bandalphabet.

\begin{satz}
Jede Turingmaschine mit Bandalphabet $\Gamma$ kann auf einer Turingmaschine
mit Bandalphabet $\{\texttt{0},1,\blank\}$ simuliert werden.
\end{satz}

\begin{proof}[Beweis]
Sei also $M$ eine Turingmaschine mit Bandalphabet $\Gamma$ gegeben. Wir
wollen daraus eine Turingmaschine $M'$ mit dem Alphabet
$\Gamma_0= \{\texttt{0},\texttt{1},\blank\}$ konstruieren, die dieselbe
Sprache erkennt. Dazu müssen wir die Zeichen aus $\Gamma$ als
Bitfolgen codieren, eine beliebige injektive Abbildung
\[
e\colon\Gamma\to \Gamma_0^l
\]
löst dieses Problem, eine solche existiert, wenn $2^l>|\Gamma|$.
Zusätzlich können
wir verlangen, dass $e(x)\in \{\texttt{0},\texttt{1}\}^l$, die Zeichen $\ne\blank$
also ausschliesslich mit $\texttt{0}$ und $\texttt{1}$ codiert werden, und
$e(\blank)=(\blank,\dots,\blank)$.

Mit der Codierung $e$ wird der Bandinhalt jetzt umcodiert. Der Inhalt
$x$
eines Feldes des Bandes von $M$ wird auf $l$ aufeinanderfolgende Felder
des Bandes von $M'$ verteilt, die mit den einzelnen Komponenten
von $e(x)$ gefüllt werden.

In den folgenden Darstellungen verwenden wir, der grösseren
Übersichtlichkeit halber, $\Gamma=\{\texttt{0},\texttt{1},2,3\}$, wobei eines der
Zeichen die Rolle von $\blank$ übernimmt. Ausserdem ist $l=2$.

Jetzt müssen auch die Übergänge in $M$
durch entsprechend erweiterte Konstruktionen in $M'$ ersetzt werden.
In einem Zyklus müssen zunächst die nächsten $l$ Zeichen vom Band
von $M'$ gelesen werden. Welches Zeichen dieser Bitfolge entspricht,
muss in Zuständen von $M'$ festgehalten werden. Ausgehend vom Zustand
$q$ von $M$ kann zur Decodierung
des Bandes folgender Automat verwendet werden, der $l$ Bits liest
und den Kopf wieder auf den Anfang des Codewortes zurückfährt:
\[
\entrymodifiers={++[o][F]}
\xymatrix{
*+\txt{}
	&*+\txt{}
		&*+\txt{}
			&q\ar[dll]_{\texttt{0}\to \texttt{0},\text{R}} \ar[drr]^{\texttt{1}\to \texttt{1},\text{R}}
\\
*+\txt{}
	&\ar[dl]_{\texttt{0}\to \texttt{0},\text{R}} \ar[dr]^{\texttt{1}\to \texttt{1},\text{R}}
		&*+\txt{}
			&*+\txt{}
				&*+\txt{}
					&\ar[dl]_{\texttt{0}\to \texttt{0},\text{R}} \ar[dr]^{\texttt{1}\to \texttt{1},\text{R}}
						&*+\txt{}
\\
\ar[d]^{.\to .,\text{L}}
	&*+\txt{}
		&\ar[d]^{.\to .,\text{L}}
			&*+\txt{}
				&\ar[d]^{.\to .,\text{L}}
					&*+\txt{}
						&\ar[d]^{.\to .,\text{L}}
\\
\ar[d]^{.\to .,\text{L}}
	&*+\txt{}
		&\ar[d]^{.\to .,\text{L}}
			&*+\txt{}
				&\ar[d]^{.\to .,\text{L}}
					&*+\txt{}
						&\ar[d]^{.\to .,\text{L}}
\\
q_0
	&*+\txt{}
		&q_1
			&*+\txt{}
				&q_2
					&*+\txt{}
						&q_3
}
\]
Die Zustände $q_i$ codieren jetzt zusätzlich, welches
Zeichen $i\in\Gamma$ sich unter dem Schreib-/Lesekopf befindet.

Ein Übergang
\[
\entrymodifiers={++[o][F]}
\xymatrix@C+1pc{
q\ar[r]^{a\to b,\text{R}}
	&p
}
\]
muss jetzt übersetzt werden in einen Übergang von $q_\text{accept}$ in einen
Zwischenzustand, von dem aus das Codewort von $b$ geschrieben.
wird. Wird $b$ als $b_1\dots b_l$ codiert, wird daraus also
\[
\entrymodifiers={++[o][F]}
\xymatrix@C+1pc{
q_a\ar[r]^{.\to b_1,\text{R}}
	&\ar[r]^{.\to b_2,\text{R}}
		&p
}
\]
Falls dem Übergang jedoch eine Kopfbewegung nach links folgt, wie in
\[
\entrymodifiers={++[o][F]}
\xymatrix@C+1pc{
q\ar[r]^{a\to b,\text{L}}
	&p
}
\]
dann muss diese ebenfalls noch angehängt werden:
\[
\entrymodifiers={++[o][F]}
\xymatrix@C+1pc{
q_a\ar[r]^{.\to b_1,\text{R}}
	&\ar[r]^{.\to b_2,\text{R}}
		&\ar[r]^{.\to .,\text{L}}
			&\ar[r]^{.\to .,\text{L}}
				&\ar[r]^{.\to .,\text{L}}
					&\ar[r]^{.\to .,\text{L}}
						&p
}
\]
Diese Konstruktion zeigt, dass sich die Turingmaschine $M$ auf einer
Turingmaschine $M'$ mit Bandalphabet $\{\texttt{0},\texttt{1},\blank\}$ simulieren
lässt.
\end{proof}

\subsection{Aufzähler}
\begin{figure}
\begin{center}
%\includegraphics[width=\hsize]{images/turing-3}
\includegraphics{images/turing-3}
\end{center}
\caption{Schematische Darstellung eines Aufzählers.\label{turing-aufzaehler}}
\end{figure}
Unser bisheriges Konzept einer Turingmaschine hat keine Möglichkeit,
Output zu produzieren. Solange die Turingmaschine arbeitet, wissen wir
nicht, welcher Teil des Bandinhaltes möglicherweise schon ``fertig''
berechnet ist. Erst wenn sie angehalten hat, weil sie $q_{\text{accept}}$
oder $q_{\text{reject}}$ erreicht hat, ist die Berechnung fertig.

Für die Praxis wünschen wir uns jedoch auch eine Möglichkeit,
die Maschine ohne Ende weiterlaufen zu lassen, wobei sie immer neue
Resultate produziert. Zum Beispiel könnte so eine Maschine der Reihe
nach alle Wörter einer Sprache aufzählen wollen. Zu diesem Zweck
fügen wir der Turingmaschine einen Drucker hinzu
(Abbildung~\ref{turing-aufzaehler}). Die Turingmaschine
kann jederzeit ein Wort auf den Drucker schreiben und dann
weiterarbeiten.

\begin{definition}
\index{Aufzahler@Aufzähler}%
Eine Turingmaschine mit einem Drucker, die auf einem leeren Band beginnt,
heisst ein Aufzähler. Die von einem Aufzähler auf dem Drucker ausgegebenen
Wörter bilden eine Sprache, die vom Aufzähler aufgezählte Sprache.
\end{definition}

Wieder haben wir das Berechnungsmodell erweitert, und es stellt sich
die Frage, ob sich dadurch die Menge der Sprachen erweitert.

\begin{satz}
\index{Turing-erkennbar}%
Eine Sprache ist genau dann Turing-erkennbar, wenn sie
von einem Aufzähler aufgezählt wird.\end{satz}

\begin{proof}[Beweis]
Da die Äquivalenz der beiden Modelle zu zeigen ist, sind zwei Implikationen
zu beweisen. Einerseits muss gezeigt werden, dass die von einem
Aufzähler aufgezählte Sprache auch von einer Turingmaschine
erkannt werden kann, andererseits muss zu einer Turingmaschine, die
die Sprache erkennt, ein Aufzähler konstruiert werden, der die Sprache
aufzählt.

Sei also $A$ ein Aufzähler. Wir konstruieren eine Turingmaschine $M$, die
zum Inputwort $w$ folgenden Algorithmus implementiert.
\begin{compactenum}
\item Lasse $A$ laufen. Jedesmal, wenn $A$ ein Wort auf den Drucker schreibt,
vergleiche das Wort mit $w$.
\item Falls das Wort mit $w$ übereinstimmt, gehe in den Zustand
$q_{\text{accept}}$.
\end{compactenum}
Falls das Wort zur Sprache gehört, wird der Aufzähler es früher
oder später aufzählen, und der Algorithmus wird es akzeptieren.

Für die umgekehrte Richtung müssen wir zur Turingmaschine $M$
einen Aufzähler produzieren.
Ein erster Versuch besteht darin, der Reihe nach alle Wörter aus
$\Sigma^*$ zu produzieren, und jedes mit $M$ zu testen.
Ein Algorithmus, der die Wörter produziert, ist einfach herzustellen,
man produziert zuerst alle Wörter der Länge 1, dann alle der Länge 2,
immer lexikographisch geordnet.
Sei also $s_1,s_2,s_3,\dots$ eine Liste aller Wörter von $\Sigma^*$.

Beim Testen der Wörter mit $M$
haben wir die Schwierigkeit, dass $M$ auf einem Inputwort
möglicherweise nicht anhält. Wir müssen also mit einem Trick
simulieren, dass unser Algorithmus nicht in einer ``Endlosschleife''
in $M$ stecken bleibt. Wir lassen $M$ daher nur jeweils für einige
Schritte laufen.
Genauer:
\begin{compactenum}
\item Für $n=1,2,3,\dots$ führe die folgenden zwei Schritte aus:
\item Lasse $M$ auf jedem Wort $s_i, i \le n$ während $n$ Schritten
laufen.
\item Falls $M$ das Wort $s_i$ akzeptiert, schreibe es auf den Drucker.
\end{compactenum}
Dieser Algorithmus verhindert, dass $M$ in eine ``Endlosschleife''
gerät, und druckt alle Wörter aus, die von der Turingmaschine erkannt
werden. Dies sogar unendlich oft.
\end{proof}

\subsection{Nicht deterministische Turingmaschinen}
\index{Turing-Maschine!nicht deterministische}%
Sowohl bei endlichen Automaten wie auch bei Stackautomaten war
Nichtdeterminismus ein Konzept, welches die Formulierung eines
Automaten wesentlich vereinfachen konnte, ohne die Möglichkeiten
zu verändern. Die einzige Änderung in der Definition ist die
Definition von $\delta$, welche nicht mehr Werte in
$Q\times \Gamma\times\{\text{L},\text{R}\}$ annimmt, sondern
in der Potenzmenge:
\[
\delta\colon Q\times\Gamma\to
P(Q\times \Gamma\times\{\text{L},\text{R}\}).
\]
Die Berechnung muss in jedem Schritt eine der Möglichkeiten
aus $\delta(q,a)$ auswählen.
Eine deterministische Turingmaschine ist offenbar auch eine
nichtdeterministische Turingmaschine, die den nichtdeterminismus
gar nicht ausnützt, also $|\delta(q,a)|=1$.

\begin{satz}
\label{nichtdeterministischeturingmaschine}
Jede nichtdeterministische Turingmaschine ist äquivalent zu einer
deterministischen Turingmaschine.
\end{satz}

\begin{proof}[Beweis]
Die Behauptung ist bewiesen, wenn wir zu einer nicht deterministischen
Turingmaschine $M$ eine deterministische Turingmaschine konstruiert haben,
die die gleichen Wörter erkennt.

Die nicht deterministische Maschine kann viele verschiedene Berechnungswege
verwenden, um ein Wort zu akzeptieren.
Es ist aber auch möglich, dass ein solcher Weg in eine Endlosschleife führt.
Daher können wir nicht einfach die nichtdeterministische Turingmaschine
auf einem Berechnungsweg laufen lassen, wir würden nie mit einem
anderen Weg beginnen, der möglicherweise das Wort akzeptiert.
Wir dürfen also die Maschine immer nur einige Schritte laufen lassen,
und müssen dann die anderen Wege durchprobieren.

Wir konstruieren jetzt eine Turingmaschine mit drei Bändern, die die
nichtdeterministische Turingmaschine simuliert. Das erste Band
enthält das Input-Wort $w$.
Das zweite Band dient als Arbeitsband für die Maschine $M$. Das
dritte Band hat Bandalphabet $S=Q\times \Gamma\times\{\text{L},\text{M}\}$
und speichert die nicht deterministischen Auswahlen aus $\delta(q,a)$,
die  in jedem Schritt nötig sein können. Eine Folge von Auswahlen
ist ein String aus $S^*$. Wir wissen bereits, dass die Strings aus $S^*$
aufgezählt werden können.

Auf dieser Maschine führen
wir jetzt folgenden Algorithmus aus:

\begin{compactenum}
\item für $n=1,2,3,\dots$ führe die Schritte 2 bis 5 aus:
\item Schreibe den String $s_i$  auf Band 3,
und führe damit die Schritte 3 bis 5 aus.
\item Kopiere $w$ von Band 1 auf Band 2
\item Lasse $M$ für $n$ Schritte auf Band 2 laufen, verwende für die
nichtdeterministischen Entscheidungen in jedem Schritt das entsprechende
Feld auf Band 3.
\item Falls $M$ akzeptiert, akzeptiere.
\end{compactenum}
Dieser Algorithmus probiert nacheinander alle möglichen Berechnungswege
durch, lässt $M$ aber immer nur eine beschränkte Anzahl Schritte lang
rechnen. Falls $w$ akzeptiert werden kann, wir der dazu nötige Berechnungsweg
irgendwann auf Band 3 auftauchen, und in entsprechend vielen Schritten wird
das Wort akzeptiert werden.
\end{proof}

\begin{satz}
Eine Sprache ist Turing-erkennbar, wenn sie von einer nicht deterministischen
Turingmaschine erkannt werden kann.
\end{satz}

Auch eine nicht deterministische Turingmaschine kann ein Entscheider sein.
Da ein Entscheider aber niemals in eine ``Endlosschleife'' geraten darf,
müssen wir verlangen, dass alle möglichen Berechnungsabläufe
irgenwann terminieren.

\begin{definition}
\index{Entscheider}%
\index{Berechnungsgeschichte}%
Eine nicht deterministische Turingmaschine ist ein Entscheider, wenn
jede mögliche Berechnungsgeschichte terminiert.
\end{definition}

\begin{satz}
Eine Sprache ist entscheidbar, wenn sie von einer nicht deterministischen
Turingmaschine entschieden wird.
\end{satz}

Nichtdeterminismus ändert also die erkennbaren oder entscheidbaren Sprachen
nicht. Doch der Algorithmus lässt bereits vermuten, dass die Simulation
einer nichtdeterministischen Turingmaschine auf einer deterministischen
Maschine sehr zeitaufwendig ist. Wenn man also eine Performance-Vorgabe
macht, kann die Menge der von einer deterministischen Turingmaschine
erkennbaren Sprachen deutlich kleiner sein. Dies wird uns im Kapitel
\ref{chapter-komplexitaet} beschäftigen.

\section{Berechenbarkeit}
\rhead{Berechenbarkeit}
In diesem Abschnitt wollen wir klären, welche Dinge überhaupt
berechnet werden können. Dabei sehen wir die Tatsache, dass für
die Darstellung einer reellen Zahl eine unendliche Folge von Ziffern
nötig sein kann, nicht als Hindernis an. Wir nennen eine Zahl
berechenbar, wenn wir ein Programm schreiben können, welches
möglicherweise unendlich lange läuft und dabei eine Stelle der
Zahl nach der anderen liefern kann. Die Turingmaschine
funktioniert also als Aufzähler, der die Stellen der zu berechnenden
Zahl aufzählen soll.  In diesem Sinne ist die
Zahl $1/3=0.3333\dots$ berechenbar, aber auch die Zahlen $\sqrt{2}$, $\pi$,
$e$ und weitere. Wir werden allerdings sehen, dass die meisten
Zahlen gar nicht berechenbar sind.

\subsection{Abzählbar und überabzählbar}
Unendlich ist nicht gleich unendlich, die Mengen $\mathbb N$ und
$\mathbb R$ haben zwar beide unendlich viele Elemente, dennoch
gibt es keine bijektive Abbildung $\mathbb N\to \mathbb R$.

\begin{definition}
\index{Machtigkeit@Mächtigkeit}%
Zwei Mengen $A$ und $B$ heissen gleich mächtig, wenn es eine bijektive
Abbildung $A\to B$ gibt.
\end{definition}

Es ist bekannt, dass $\mathbb Q$ und $\mathbb N$ gleich mächtig sind,
aber die Menge $\mathbb R$ scheint wesentlich grösser
zu sein als $\mathbb N$.
Diese Unterschiede zwischen verschiedenen
unendlichen Mengen wird illustriert von der nachfolgenden Geschichte,
die David Hilbert zu erzählen pflegte.

\subsubsection{Hotel ``Unendlich''}
\index{Hotel ``Unendlich''}%
\index{Hilbert!-Hotel}%
In einem fernen Land gibt es ein Hotel, welches Hotel ``Unendlich''
genannt wird. Unsere Geschichte beginnt kurz nachdem der Nachtportier
die Nachtschicht übernommen hat. Ein verspäteter Gast meldet sich
am Eingang, während der Nachtportier damit beschäftigt ist,
die herumlungernden Besoffenen
aus den naheliegenden Pubs zu vertreiben, die auch gerade geschlossen haben.
Der Gast möchte im Hotel übernachten, doch der Nachtportier zeigt sich
unnachgiebig: ``Du kommst hier nicht rein!''

Zum Glück hat der Hotelmanager den Wortwechsel mitbekommen und greift
ein. Selbstverständlich haben wir noch ein Zimmer für jeden noch so
späten Gast. Der Nachportier protestiert, es seien doch alle Zimmer
belegt. ``Kein Problem'' sagt der Manager, und greift zum Mikrofon
der Sprechanlage. Er bittet alle Gäste, aus ihrem aktuellen Zimmer
ins nächste Zimmer umzuziehen. Dieses Ansinnen zu später Stunden
verursacht zwar bei einigen Gästen etwas Unmut, aber jeder ist damit
einverstanden, denn hätten sie sich selbst verspätet, wären sie ja
auch auf das Entgegenkommen der anderen Gäste angewiesen. Auf diese
Weise wird das Zimmer mit der Nummer $0$ frei für den neu angekommenen
Gast.

Kurze Zeit später kommt eine verspätete Reisegruppe aus dreissig
Touristen an, die ebenfalls
Einlass verlangt. Der Nachtportier will nicht nochmals einen schlechten
Eindruck hinterlassen und ruft den Manager. Dieser greift
wieder zum Mikrofon, und bittet wieder alle Gäste, diesmal aus
ihrem Zimmer mit der Nummer $n$ ins Zimmer mit der Nummer $n+30$
umzuziehen. So werden die Zimmer mit den Nummern $0$ bis $29$ frei,
Platz genug für die neu angekommene Reisegruppe.

Diese Nacht ist ziemlich viel los im Hotel ``Unendlich'', denn
kaum waren die dreissig Gäste untergebracht, fährt ein Hilbert-Bus
vor: ein Bus mit unendlich vielen Sitzplätzen, numeriert mit den
natürlichen Zahlen. Der Nachtportier ist schockiert: Unendlich
viele Gäste, ohne Reservation. Dass man für endlich viele Gäste
immer noch Platz schaffen kann, hat er inzwischen verstanden, aber
für unendlich viele Gäste sei das unmöglich, stellt er fest, und
will die Gesellschaft wegschicken. Doch der Hotelmanager möchte
sich das Geschäft nicht entgehen lassen. Er greift erneut zum
Mikrofon und bittet die Gäste, vom Zimmer $n$ ins Zimmer $2n$ umzuziehen.
So werden alle Zimmer mit ungeraden Nummern frei, Platz genug für alle
Passagiere des Hilbert-Buses. Nur die Stimmung unter den Stammgästen
hat sich bereits deutlich verschlechtert.

Schliesslich fahren gleich dreissig Hilbertbusse vor. Ein Hilbert-Jumbo
hatte sich verspätet, und so dass die Feriengäste erst sehr spät
mit ihren Hilbertbussen vom Flughafen zum Hotel abfahren konnten.
Der Nachtportier hat schon einen Verdacht, dass der Manager auch
hierfür eine Lösung bereit hat. Tatsächlich: per Mikrofondurchsage
gibt er wieder alle ungerade Zimmer frei. Dann bringt er die Neuankömmlinge
darin unter, und zwar schön nacheinander immer je einen aus jedem Bus,
also in die ersten dreissig freien Zimmer die jeweils ersten Gäste
aus jedem Bus, dann die zweiten aus jedem Bus in die zweiten dreissig
freien Zimmer und so weiter.

Man ahnt es schon, in dieser Nacht gibt es für die Gäste im Hotel
``Unendlich'' keine Ruhe. Als nächstes findet sich ein Hilbert-Konvoi
vor dem Hotel ein. Also einer Folge von Hilbert-Bussen, die mit natürlichen
Zahlen numeriert waren. Der Nachtportier hat gar nicht erst gewagt,
den Neuankömmlingen abschlägigen Bescheid zu geben, sondern gleich
den Manager gerufen. Der muss zwar auch einen Moment nachdenken, findet
dann aber eine Lösung. Zunächst wendet er nochmals den Trick an,
mit dem er jetzt schon mehrmals die ungeraden Zimmer frei bekommen hat.
Dabei kommt es auch zu wüsten Szenen mit Gästen, die jetzt schon vier mal
geweckt worden sind, und sich ihre Nachtruhe nicht von der Geldgier des
Managers ruinieren lassen wollen.

Die Gäste aus dem Hilbert-Konvoi müssen sich in Einerkolonnen auf
dem Vorplatz aufstellen, der Manager rief sie daraufhin in der
eingezeichneten Reihenfolge ab. Auch dies trägt nicht  unbedingt
zur Zufriedenheit der Gäste bei, hinter vorgehaltener Hand wird
über den ``Kasernenton'' und ``Zustände wie in einem Konzentrationslager''
geschimpft.
Aber jeder kommt früher oder später dran,
am Ende hat jeder Gast ein Zimmer, was die Gemüter wieder
etwas beruhigt.

\[
\xymatrix{
\ar[r]
\cdot	     &\cdot\ar[dl] &\cdot\ar[r] &\cdot\ar[dl]  &\cdot\ar[r] &\cdot\ar[dl] &\cdot
\\
\cdot\ar[d]  &\cdot\ar[ur] &\cdot\ar[dl] &\cdot\ar[ur] &\cdot\ar[dl]&\cdot        &\cdot
\\
\cdot\ar[ur] &\cdot\ar[dl] &\cdot\ar[ur] &\cdot\ar[dl] &\cdot       &\cdot        &\cdot
\\
\cdot\ar[d]  &\cdot\ar[ur] &\cdot\ar[dl] &\cdot        &\cdot       &\cdot        &\cdot
\\
\cdot\ar[ur] &\cdot        &\cdot        &\cdot        &\cdot       &\cdot        &\cdot
}
\]

Ein Ereignis steht allerdings noch bevor, und hier sollte sich die
Geldgier des Managers rächen. Gegen morgen nämlich fährt
ein voller Cantor-Bus vor. Dieses hochmoderne Verkehrsmittel hat Sitze,
\index{Cantor!-Bus}%
die mit den reellen Zahlen angeschrieben sind. Natürlich wollen auch
diese Gäste im Hotel ``Unendlich'' untergebracht werden. Aber so sehr
sich der Manager auch anstrengt, in seinem Hotel kann er diese
Gäste nicht unterbringen. Daraufhin wird er zwar vom Verwaltungsrat
entlassen, der noch weniger Mathematik versteht, was ihm dank üppiger
Abgangsentschädigung jedoch nicht weiter Sorgen macht. Unbestätigten
Gerüchten zufolge soll er jetzt bei einer Hotelkette ``Cantor Hotels''
arbeiten, die mit mit dem Spruch ``Present in uncountable locations
throughout the Universe'' für sich wirbt.

\subsubsection{Lehren aus der Geschichte}
Eine unendliche Menge ist offenbar so gross, dass man darin immer noch
Platz genug für eine Kopie der ganzen Menge finden kann. Oder anders
herum: endliche Mengen sind solche, in denen man niemals Platz finden
könnte:

\begin{satz}
Eine Menge ist endlich, wenn jede injektive Abbildung auch surjektiv ist.
\end{satz}

In der Geschichte kam eine ganze Reihe von unendlichen Mengen auf, die
alle in $\mathbb N$ untergebracht werden konnten, die also nicht
grösser waren als $\mathbb N$ selbst, und das obwohl sie aus natürlichen
Zahlen konstruiert worden waren. Mit $\mathbb N$ vergleichbare Mengen
bilden also eine robuste Klasse von Mengen.

\begin{definition}
\index{abzahlbar unendlich@abzählbar unendlich}%
\index{uberabzahlbar unendlich@überabzählbar unendlich}%
Eine unendliche Menge $A$ heisst abzählbar unendlich, wenn sie
gleichmächtig ist wie die natürlichen Zahlen. $A$ heisst
überabzählbar unendlich, wenn es keine Bijektion zwischen
$\mathbb N$ und $A$ gibt.
\end{definition}
Bildlich gesprochen ist eine abzählbare Menge eine solche,
bei der sich die Elemente in eine mit den natürlichen Zahlen numerierte
Liste einordnen lassen.

Auch die Menge der Paare $(k,l)\in \mathbb N^2$ ist abzählbar.
Ein Gast aus dem Hilbert-Konvoi wird durch seine Busnummer $k$ und
seine Platz-Nummer $l$ identifiziert, der Konvoi hat also gleich
viele Gäste wie $\mathbb N^2$ Paare enthält. Und alle diese Paare
passen in die Menge $\mathbb N$, also die Zimmer des Hotels ``Unendlich''
hinein. Somit sind $\mathbb N^2$ und $\mathbb N$ gleich mächtig.

Die Geschichte hat exemplarisch den folgenden Satz gezeigt:

\begin{satz}Die Vereinigung von endlich vielen abzählbaren
Mengen ist abzählbar. Das kartesische Produkt zweier abzählbarer
Mengen ist abzählbar.
\end{satz}

\begin{proof}[Beweis]
Sei $A_1,\dots,A_n$ eine endliche Familie abzählbarer Mengen,
je mit einer Funktion $f_i\colon \mathbb N\to A_i$, dann
können wir eine Abzählfunktion für die Vereinigung konstruieren:
\[
f\colon \mathbb N\to\bigcup_{i=1}^n A_i\colon nk+l \mapsto f_l(k)
\]
wobei wir verwenden, dass wir jede Zahl eindeutig als $nk+l$
schreiben können mit geeigneten Zahlen $k$ und $l$.

Für das kartesische Produkt müssen wir zeigen, dass sich die
Menge der Paare $\mathbb N^2$ aufzählen lässt, was mit dem
Diagonaltrick des Hotelmanagers geschehen kann.
\end{proof}

Daraus folgen jetzt weitere Sätze über die Kardinalität
der bekannten Zahlmengen.

\begin{satz}
Die Mengen $\mathbb Z$ und $\mathbb Q$ sind abzählbar unendlich.
\end{satz}

\begin{proof}[Beweis]
Die Menge $\mathbb Z$ ist abzählbar, da sie die Vereinigung
von zwei Mengen ist, die offensichtlich abzählbar sind:
\[
\mathbb Z = \mathbb N\cup \{-n\;|\;n\in\mathbb N\}.
\]
Die rationalen Zahlen können durch Brüche dargestellt werden,
also durch Paare $(p,q)$ von ganzen Zahlen:
\[
\mathbb Q=\left\{\left.\frac{p}{q}\;\right|\;p\in \mathbb Z,q\in\mathbb N\setminus\{0\}\right\}
\]
Da sich einige Brüche noch kürzen lassen, ist $\mathbb Q$ noch
kleiner als $\mathbb N^2$, aber insbesondere abzählbar.
\end{proof}

\begin{satz}
Die Menge $\mathbb R$ ist überabzählbar unendlich.
\end{satz}

\begin{proof}[Beweis]
Wir führen den Beweis mit Hilfe eines Widerspruchs. Nehmen wir
an, $\mathbb R$ wäre abzählbar unendlich. Dann müssten auch
die reellen Zahlen zwischen $0$ und $1$ abzählbar sein,  es gäbe also eine
Liste all dieser Zahlen, wir schreiben die Zahlen
dieser Liste in Dezimaldarstellung
\begin{align*}
0.&r_{11}r_{12}r_{13}r_{14}\dots\\
0.&r_{21}r_{22}r_{23}r_{24}\dots\\
0.&r_{31}r_{32}r_{33}r_{34}\dots\\
0.&r_{41}r_{42}r_{43}r_{44}\dots\\
&\vdots
\end{align*}
Jetzt konstruieren wir eine Zahl $x$, die nicht in dieser Liste vorkommen
kann. Um die $k$-te Stelle $x_k$ von $x$ zu konstruieren, betrachten
wir die $k$-te Stelle der $k$-ten Zahl $r_{kk}$, und setzen
\[
x_k=\begin{cases}
r_{kk}-1&\qquad r_{kk}>0\\
5&\qquad r_{kk}=0
\end{cases}
\]
$x_k$ ist also verschieden von $r_{kk}$, und zwar für jedes $k$.
Die Ziffer $9$ kommt in $x$ nicht vor, es kann also keine Zahl
mit lauter $9$ am Ende sein, die gleichbedeutend mit einer anderen
Zahl ist, die möglicherweise bereits in der Liste ist.
Also unterscheidet sich $x$ von jeder der Zahlen aus der Liste, $x$
kommt also in der Liste nicht vor. Wir hatten angenommen, dass die
Liste alle reellen Zahlen zwischen $0$ und $1$ umfasst, dieser
Widerspruch zeigt, dass es keine solche Liste geben kann, die
reellen Zahlen sind also überabzählbar.
\end{proof}

\index{Cantor!Georg}%
Georg Cantor (1845-1918) leistete wichtige Beiträge zur Mengenlehre,
unter anderem entdeckte er den Unterschied der Mächtigkeit von
natürlichen und reellen Zahlen. Daher haben wir in der Geschichte
den Bus, dessen Sitzplätze mit reellen Zahlen angeschrieben waren,
als Cantor-Bus bezeichnet.

Verwandt damit und mit einem ähnlichen Beweis kann man auch einsehen,
dass die Potenzmenge einer abzählbaren Menge überabzählbar ist.

\begin{satz}\label{powersetuncountable}
Ist $A$ eine abzählbar unendliche Menge, dann ist $P(A)$
überabzählbar.
\end{satz}

\begin{proof}[Beweis]
Da $A$ abzählbar ist gibt es eine bijektive Abbildung
$i\mapsto a_i\in A$. Nehmen wir an, es gäbe eine bijektive
Abbildung von $\mathbb N$ in $P(A)$, also $i\mapsto A_i\subset A$.
Dann kann man wie folgt eine Menge bilden, die unter den Mengen
$A_i$ nicht vorkommt.
\[
a_i\in B\quad\Leftrightarrow\quad a_i\not\in A_i
\]
Die Menge $B$ unterscheidet sich von jeder Menge $A_i$. Falls
$A_i$ das Element $a_i$ enthält, enthält $B$ es nicht und
umgekehrt. Somit ist $B\ne A_i\forall i$, im Widerspruch zur Annahme
dass $i\mapsto A_i$ eine Bijektion ist.
\end{proof}

Jetzt können wir auch einige Mengen aus der bisher betrachteten
Theorie der Sprachen auf ihre Mächtigkeit untersuchen.

\begin{satz}
$\Sigma^*$ ist abzählbar unendlich. Die Menge aller Sprachen über dem
Alphabet $\Sigma$ ist überabzählbar unendlich.
\end{satz}

\begin{proof}[Beweis]
Eine Aufzählung von $\Sigma^*$ kann man konstruieren, indem man
die Wörter von $\Sigma^*$ der Länge nach sortiert, und innerhalb
der Wörter gleicher Länge die lexikographische Ordnung verwendet.
Dazu braucht man natürlich eine Anordnung der Zeichen des Alphabets $\Sigma$,
da dieses aber endlich ist, kann man immer eine solche Anordnung finden.

Die Menge aller Sprache ist die Potenzmenge der Menge $\Sigma^*$, welche
abzählbar unendlich ist. Nach Satz \ref{powersetuncountable} ist
$P(\Sigma^*)$ überabzählbar unendlich.
\end{proof}

\begin{satz}\label{countablefinite}
Eine abzählbare Vereinigung $\bigcup_{i\in\mathbb N}A_i$ von endlichen
Mengen $(|A_i|<\infty)$ ist abzählbar.
\end{satz}

\begin{proof}[Beweis]
Endliche Mengen kann man immer abzählen, also kann man eine Abzählung
von $\bigcup_{i\in\mathbb N}A_i$ einfach dadurch konstruieren, dass
man zuerst die Elemente von $A_0$ durchzählt, dann die von $A_1$, und
so weiter. Weil keine der Mengen $A_i$ unendlich ist, kommt jede
Menge irgendwann dran.
\end{proof}

\begin{satz} Sei $\Sigma$ ein festes Alphabet. Dann sind die folgenden
Mengen alle abzählbar unendlich:
\begin{enumerate}
\item Die Menge aller deterministischen endlichen Automaten.
\item Die Menge aller nichtdeterministischen endlichen Automaten.
\item Die Menge der regulären Sprachen.
\item Die Menge aller kontextfreien Grammatiken.
\item Die Menge aller kontextfreien Sprachen.
\item Die Menge aller Stackautomaten.
\item Die Menge aller Turingmaschinen.
\end{enumerate}
\end{satz}

\begin{proof}[Beweis]
\begin{enumerate}
\item Es genügt zu zeigen, dass die Menge der deterministischen
endlichen Automaten mit $k$ Zuständen endlich ist, dann ist nach
Satz \ref{countablefinite} auch die Vereinigung abzählbar. Ein
deterministischer endlicher Automat ist aber durch die Tabellendarstellung
gegeben. In der Tabelle sind $|Q|\cdot|\Sigma|$ Felder mit Zuständen
zu besetzen, dafür gibt es $|Q|^{|Q|\cdot|\Sigma|}$ Möglichkeiten.
Für jede solche Möglichkeit ist ausserdem festzulegen, welche Zustände
Akzeptierzustände sind, das sind $|P(Q)|=2^{|Q|}$ Möglichkeiten. Es gibt
also
\[
|Q|^{|Q|\cdot|\Sigma|}\cdot 2^{|Q|}
\]
DEAs mit $|Q|$ Zuständen.
\item Obwohl in der Definition der nichtdeterministischen endlichen
Automaten auch die Potenzmenge vorkommt, wird dadurch die Menge
noch nicht über\-abzählbar. Der Funktionswert der Übergangsfunktion
$\delta$ ist ja immer eine Teilmenge einer endlichen Menge, die Potenzmenge
einer endlichen Menge ist aber auch endlich. Es ändert sich am Argument
nur, dass in die Tabelle eines Automaten mit $|Q|$ Zuständen und
$|\Sigma|$ Zeichen im Alphabet an $|Q|\cdot|\Sigma|$ Stellen
nicht $|Q|$ verschieden Objekte einfüllen lassen, sondern $2^{|Q|}$.
An der Abzählbarkeit ändert dies nichts.
\item Die regulären Sprachen werden von deterministischen endlichen
Automaten akzeptiert, es gibt also eine surjektive Abbildung von
den deterministischen endlichen Automaten auf die regulären Sprachen,
eine Aufzählung der deterministischen endlichen Automaten liefert also
auch automatisch eine Aufzählung der regulären Sprachen.
\item Es gibt nur endliche viele Grammatiken mit $n$ verschiedenen
Variablen und höchstens $n$ Zeichen langen rechten Seiten der Regeln,
einfach weil es nur endlich viele rechte Seiten gibt. Also ist die
Menge aller kontextfreien Grammatiken abzählbar.
\item Kontextfreie Sprachen werden von einer kontextfreien Grammatik
erzeugt. Die Abbildung $G\mapsto L(G)$ macht aus einer Aufzählung
der kontextfreien Grammatiken eine Aufzählung der kontextfreien Sprachen.
\item Stackautomaten sind nichtdeterministische endliche Automaten mit
zusätzlicher Beschriftung der Übergangspfeile. Trotzdem bleibt die
Anzahl der Stackautomaten mit $k$ Zuständen endlich, also ist die
Menge der Stackautomaten abzählbar unendlich.
\item Turingmaschinen sind im wesentlichen deterministische endliche Automaten
mit zusätzlichen Beschriftungen der Pfeile. Wie in 1.~gibt es nur endlich
viele Turingmaschinen mit $k$ Zuständen, also ist die Menge der Turingmaschinen
abzählbar unendlich.
\end{enumerate}
\end{proof}

\subsection{Nicht berechenbare Zahlen}

\begin{satz}
Die rationalen Zahlen sind berechenbar.
\end{satz}

\begin{proof}[Beweis]
Der Algorithmus der schriftlichen Division erlaubt alle Stellen
zu finden, man kann sich gut vorstellen, dass er sich auf einer
Turingmaschine als Aufzähler implementieren lässt. Zähler und
Nenner werden zu Beginn auf das Band geschrieben, der Algorithmus
berechnet dann alle Stellen des Quotienten. Also sind die rationalen
Zahlen berechenbar.
\end{proof}

\begin{satz}Die algebraischen Zahlen, also die Nullstellen von
Polynomen mit rationalen Koeffizienten sind berechenbar.
\end{satz}

\begin{proof}[Beweis]
Zunächst können wir das Polynom mit dem gemeinsamen Nenner der
Koeffizienten multiplizieren und erhalten ein Polynom mit ganzzahligen
Koeffizienten.
Die Nullstellen eines Polynoms können mit dem Newton-Algorithmus
mit beliebiger Genauigkeit bestimmt werden. Dazu muss nur eine
Schätzung $\hat x_0$ für die Nullstelle bekannt sein, dann können
mit der Iteration
\[
x_{n+1}=x_n-\frac{f(x_n)}{f'(x_n)},\quad x_0 = \hat x_0
\]
immer genauere Approximationen berechnet werden. Alle Operationen
in der Iterationsformel sind Operationen in rationalen Zahlen, sind
also berechenbar.
\end{proof}

\begin{satz}
Die Menge der nicht berechenbaren Zahlen ist überabzählbar.
\end{satz}

\begin{proof}[Beweis]
Die berechenbaren Zahlen sind abzählbar. Wir können nämlich
eine Liste aller Turingmaschinen erstellen, zunächst schreiben
wir die Turingmaschinen mit nur einem Zustand hin, dann all jene
mit genau zwei Zuständen usw. Da jede Turingmaschine genau eine
Zahl berechnen kann, ist die Menge der berechnenbaren Zahlen abzählbar,
die Menge der reellen Zahlen ist also überabzählbar, daher müssen die
nicht berechenbaren Zahlen auch überabzählbar sein.
\end{proof}

\subsection{Das 10.~Hilbertsche Problem}
\index{Hilbert!David}%
\index{Hilbertsche Probleme}%
Im Jahre 1900 hielt der deutsche Mathematiker David Hilbert am
\index{Internationaler Kongress}%
internationalen Mathematikerkongress in Paris einen berühmten Vortrag,
in dem er eine Reihe von Problemen zusammenstellte, deren Lösung
nach seiner Meinung die Mathematik im zwanzigsten Jahrhundert
entscheidend voranbringen würden. Einige dieser Probleme wurden
inzwischen gelöst, andere, darunter die Riemannsche Vermutung,
sind immer noch offen.

Von besonderem Interesse für unser Thema war das zehnte Problem:
Gibt es ein Verfahren, das für eine beliebige diophantische Gleichung
entscheidet, ob sie lösbar ist?

Schon die Problemstellung wirft einige Fragen auf.
Zunächst zu den Begriffen:
\index{Gleichung!diophantische}%
eine diophantische Gleichung ist eine
Polynomgleichung mit mehreren Variablen aber ausschliesslich
ganzzahligen Koeffizienten. Die Gleichung
\[
x^2+y^2-z^2=0
\]
ist eine diophantische Gleichung, und es ist auch bekannt, dass sie
Lösungen hat, zum Beispiel $x=3$, $y=4$ und $z=5$. Seit wenigen
Jahren ist auch bekannt, dass
\[
x^n+y^n-z^n=0,
\]
ebenfalls eine diophantische Gleichung, nur ganz wenige Lösungen hat.
Dies ist die berühmte Fermatsche Vermutung, die Andrew Wiles 1995
\index{Fermatsche Vermutung}%
\index{Wiles, Andrew}%
vollständig bewiesen hat.

Die schwierigerere Frage aber ist: Was für eine Art von Verfahren
ist gemeint? Hilbert hat keine Definition gegeben. Heute würden
wir wohl fragen, ob ein Algorithmus angegeben werden kann, mit dem
die Frage entschieden werden kann. Doch das führt uns nur wieder
auf die Frage nach einer  mathematisch strengen Definition was
ein Algorithmus ist.

\index{Turing, Alan}%
Im Jahre 1936 gab Alan Turing eine Antwort: ein Algorithmus ist eine
Rechenvorschrift, die sich mit einer Turingmaschine implementieren
lässt. Probleme, für die es einen Lösungsalgorithmus gibt, sind
also solche, deren Inputdaten man als Wort auf das Band einer
Turingmaschine schreiben kann, die Turingmaschine verarbeitet die
Problembeschreibung, schreibt die Lösung auf das Band und hält im
Zustand $q_{\text{accept}}$ an. Falls das Problem keine Lösung hat,
hält die Maschine im Zustand $q_{\text{reject}}$ an. Solche
Probleme definieren also auch eine Sprache: $L(M)$ ist die
Menge der Inputwörter, für die eine Lösung existiert.

Ein Beispiel soll dies illustrieren. Wir möchten die Frage
von einem Algorithmus beantworten lassen, ob es in einem Graphen $G$
einen Weg gibt, der alle Knoten trifft.  Dazu müssen wir eine
Beschreibung des Graphen erzeugen, wir nennen sie $\langle G\rangle$,
diese auf das Band einer geeignet programmierten Turingmaschine $M$ schreiben,
diese laufen lassen, und warten, bis sie die Frage beantwortet. $L(M)$
besteht aus Beschreibungen $\langle G\rangle$ von Graphen $G$, die
einen solchen Weg haben.

Auf dieser Basis gelang es 1970 Yuri Matijasevi\v c übrigens,
\index{Matijasevi\v c, Yuri}%
das zehnte Hilbertsche Problem etwas überraschend zu lösen: es gibt
keinen Algorithmus, mit dem entschieden werden kann, ob eine
diophantische Gleichung eine Lösung hat. Es ist also durchaus
nicht selbstverständlich, wenn ein Problem eine algorithmische
Lösung hat. Wir werden im nächsten Kapitel sehen, dass es viele
Probleme gibt, die keine algorithmische Lösung haben. Im übernächsten
Kapitel werden wir dann Probleme kennenlernen, die zwar von einem Computer
gelöst werden können, deren Lösung aber länger dauert als man
jeden realistischen Computer laufen lassen kann.
Die im folgenden entwickelte Theorie zeigt also, wo die Grenzen
der Berechenbarkeit mit Computern liegen.
