%
% teilbarkeit.tex -- template for standalon tikz images
%
% (c) 2021 Prof Dr Andreas Müller, Hochschule Rapperswil
%
\documentclass[tikz]{standalone}
\usepackage{amsmath}
\usepackage{times}
\usepackage{txfonts}
\usepackage{pgfplots}
\usepackage{csvsimple}
\usetikzlibrary{arrows,intersections,math,calc}
\begin{document}
\def\skala{1}
\begin{tikzpicture}[>=latex,thick,scale=\skala]

\def\ds{1.5}

\coordinate (start) at ({-\ds},0);
\coordinate (q0) at (0,0);
\coordinate (q1) at ({\ds},{\ds});
\coordinate (q2) at ({2*\ds},{\ds});
\coordinate (q3) at ({\ds},{-\ds});
\coordinate (q4) at ($(q3)+({\ds*cos(30)},{0.5*\ds})$);
\coordinate (q5) at ($(q3)+({\ds*cos(30)},{-0.5*\ds})$);

\draw (q0) circle[radius=0.3];
\draw (q1) circle[radius=0.25];
\draw (q1) circle[radius=0.3];
\draw (q2) circle[radius=0.3];
\draw (q3) circle[radius=0.25];
\draw (q3) circle[radius=0.3];
\draw (q4) circle[radius=0.3];
\draw (q5) circle[radius=0.3];

\node at (q0) {$q_0$};
\node at (q1) {$q_1$};
\node at (q2) {$q_2$};
\node at (q3) {$q_3$};
\node at (q4) {$q_4$};
\node at (q5) {$q_5$};

\draw[->,shorten >= 0.3cm,shorten <= 0.3cm] (q0) to[out=90,in=180] (q1);
\draw[->,shorten >= 0.3cm,shorten <= 0.3cm] (q0) to[out=-90,in=180] (q3);

\draw[->,shorten >= 0.3cm,shorten <= 0.3cm] (q1) to[out=20,in=160] (q2);
\draw[<-,shorten >= 0.3cm,shorten <= 0.3cm] (q1) to[out=-20,in=-160] (q2);

\draw[->,shorten >= 0.3cm,shorten <= 0.3cm] (q3) -- (q4);
\draw[->,shorten >= 0.3cm,shorten <= 0.3cm] (q4) -- (q5);
\draw[->,shorten >= 0.3cm,shorten <= 0.3cm] (q5) -- (q3);

\draw[-,shorten >= 0.3cm,shorten <= 0.3cm] (start) -- (q0);

\node at ($0.5*(q1)+0.5*(q2)+(0,0.2)$) [above] {\texttt{0}};
\node at ($0.5*(q1)+0.5*(q2)+(0,-0.2)$) [below] {\texttt{0}};

\node at ($0.5*(q3)+0.5*(q4)$) [above left] {\texttt{0}};
\node at ($0.5*(q4)+0.5*(q5)$) [right] {\texttt{0}};
\node at ($0.5*(q5)+0.5*(q3)$) [below left] {\texttt{0}};

\node at ({0.15*\ds},{(1-0.15)*\ds}) {$\varepsilon$};
\node at ({0.15*\ds},{-(1-0.15)*\ds}) {$\varepsilon$};

\end{tikzpicture}
\end{document}

