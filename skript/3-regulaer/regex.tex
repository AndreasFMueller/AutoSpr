\rhead{Reguläre Ausdrücke}
\section{Reguläre Ausdrücke\label{regulaer:re}}
\index{reguläre Ausdrücke}%
Endliche Automaten beschreiben reguläre Sprachen, sind aber für die
Anwendung eher unhandlich.
In der Praxis haben sich reguläre Ausdrücke
durchgesetzt, die mit einer einfachen Syntax ebenfalls Mengen von
Wörtern zu spezifizieren erlauben.
Ziel dieses Abschnittes ist
zu zeigen, dass reguläre Ausdrücke genau gleich ausdrucksstark sind
wie DEAs.

\subsection{Reguläre Operationen\label{regulaer:regulaere-operationen}}
\index{reguläre Operationen}%
Wir wissen bereits, dass wir die Mengenoperationen auf reguläre
Sprachen anwenden dürfen, ohne die Klasse der regulären Sprachen
zu verlassen.
Es sind jedoch noch zwei weitere Operationen möglich, die ebenfalls
nicht aus der Klasse herausführen:

\begin{definition}
\index{Verkettung}%
Seien $L_1$ und $L_2$ Sprachen, dann ist die
Verkettung von $L_1$ und $L_2$ die Sprache
\[
L_1L_2=\{w_1w_2\;|\;w_1\in L_1,w_2\in L_2\}.
\]
Die mehrfache Verkettung von $L$ wird mit $L^n$ bezeichnet:
\begin{align*}
L^0&=\{\varepsilon\}\\
L^n&=L^{n-1}L
\end{align*}
\end{definition}

\begin{definition}
\index{*-Operation@$*$-Operation}%
Sei $L$ eine Sprache, dann ist die Stern-Operation
von $L$ die Sprache
\[
L^*=\bigcup_{k\in\mathbb N} L^k.
\]
\end{definition}

Wir möchten jetzt zeigen, dass diese Operationen nicht aus den
regulären Sprachen herausführen.
Dazu müssen wir zu gegebenen Automaten $A_1$ und $A_2$ für zwei Sprachen
$L_1$ und $L_2$ neue Automaten konstruieren, welche
die Sprachen $L_1L_2$ oder $L_1^*$ akzeptieren.
Die Automaten $A_1$ und $A_2$ müssen wir dabei als ``Black Box'' betrachten,
wir dürfen daran nichts ändern.

\subsubsection{Automat zu einer Verkettung $L_1L_2$}
\index{Verkettung}%
Für die Verkettung $L_1L_2$ ist es zum Beispiel
nicht zulässig, die Akzeptierzustände von
$A_1$ mit dem Startzustand von $A_2$ zusammenzulegen.
Dadurch
würde es nämlich möglich, auch Wege wieder zurück in
den ersten Automaten zu verwenden.
Im folgenden Beispiel akzeptiert der erste Automat Strings mit
einer ungeraden Anzahl Nullen, der zweite Strings mit einer
ungeraden Anzahl {\tt a}.
\[
\entrymodifiers={++[o][F]}
\xymatrix{
*+\txt{}\ar[r]
	&{} \ar@/^/[r]^{\tt 0}
		&*++[o][F=]{} \ar@/^/[l]^{\tt 0}
			&*\txt{}\ar[r]
				&{} \ar@/^/[r]^{\tt a}
					&*++[o][F=]{} \ar@/^/[l]^{\tt a }
}
\]
Verkettet man die Automaten, indem man den Akzeptierzustand des
ersten mit dem Startzustand des zweiten Automaten zusammenlegt,
erhält man
\[
\entrymodifiers={++[o][F]}
\xymatrix{
*+\txt{}\ar[r]
	&{} \ar@/^/[r]^{\tt 0}
		&{} \ar@/^/[l]^{\tt 0}
				 \ar@/^/[r]^{\tt a}
					&*++[o][F=]{} \ar@/^/[l]^{\tt a }
}
\]
Dieser Automat akzeptiert aber auch das Wort {\tt 0aa00a}, welches
gar nicht in $L_1L_2$ ist! Die Verkettung muss also mit Hilfe
einer ``Einbahnstrasse'' erfolgen, ein $\varepsilon$-Übergang
ist ideal dafür geeignet.
Die folgende Verkettung funktioniert:
\[
\entrymodifiers={++[o][F]}
\xymatrix{
*+\txt{}\ar[r]
	&{} \ar@/^/[r]^{\tt 0}
		&{} \ar@/^/[l]^{\tt 0} \ar[r]^{\varepsilon}
			&{} \ar@/^/[r]^{\tt a}
				&*++[o][F=]{} \ar@/^/[l]^{\tt a }
}
\]

Für die allgemeine Konstruktion gehen wir aus
von zwei Automaten $A_1$ und $A_2$:
\begin{center}
\includegraphics{images/nea-1}
\end{center}
Ein Automat für die Verkettung entsteht, indem man
die Akzeptierzustände im ersten Automaten über
$\varepsilon$-Übergänge mit dem
Startzustand des zweiten verbindet:
\begin{center}
\includegraphics{images/nea-2}
\end{center}
Die $\varepsilon$-Übergänge
fungieren als Einbahnstrassen und stellen sicher, auch innerhalb der
einzelnen Automaten keine neuen akzeptierten Wörter entstehen können.

Etwas formeller kann man das Resultat im folgenden Satz zusammenfassen.

\begin{satz}
\index{Verkettung}%
\label{satz_concat}
Sind $L_1$ und $L_2$ reguläre Sprachen, dann ist auch $L_1L_2$
regulär.
\end{satz}

\begin{proof}[Beweis]
Ein Wort $w\in L_1L_2$ besteht aus zwei Teilen, die von $L_1$
bzw.~$L_2$ akzeptiert werden.
Ein Automat $A$ für $L=L_1L_2$ kann aus Automaten $A_1$ und $A_2$
konstruiert werden, indem man
die Akzeptierzustände von $A_1$ über $\varepsilon$-Übergänge
mit dem Startzustand von $A_2$ verbindet.
Also verwenden wir $Q=Q_1\cup Q_2$.
Startzustand ist der Startzustand $q_{01}$ von $L_1$, Akzeptierzustände
sind die Akzeptierzustände $F_2$ von $A_2$.
Also
\begin{align*}
Q&=Q_1\cup Q_2\\
F&=F_2\\
q_0&=q_{01}\\
\delta(q,a)&=\begin{cases}
\delta_1(\varepsilon,a)\cup\{q_{02}\}&\qquad q\in F_1 \wedge a=\varepsilon\\
\delta_1(q,a)                        &\qquad q\in F_1 \wedge a\ne\varepsilon
\quad\text{oder}\quad q\in Q_1\setminus F_1\\
%\\
%\delta_1(q,a)                        &\qquad q\in Q_1\setminus F_1\\
\delta_2(q,a)                        &\qquad q\in Q_2
\end{cases}
\end{align*}
\end{proof}

\subsubsection{Automat zur *-Operation $L^*$}
\index{*-Operation@$*$-Operation}%
Die Stern-Operation verlangt, dass man einen Automaten mehrfach
durchlaufen können muss.
Dies könnte man mit einem $\varepsilon$-Übergang
von den Akzeptierzuständen direkt zum Startzustand realisieren.
Dadurch ändert man aber den Automaten für die Sprache $L$,
der Automat ist nicht mehr eine wiederverwendbare Einheit.

Wir streben an, die regulären Operationen ohne interne Änderungen
des Automaten zu konstruieren.
Insbesondere dürfen dem Automaten keine neuen Übergänge hinzugefügt
werden.
Es ist hingegen zulässig, einen Akzeptierzustand zu einem
Nichtakzeptierzustand zu degradieren.
Ob nämlich ein Zustand als Akzeptierzustand interpretiert wird,
hängt davon ab, was ein zusammengesetzter Automat mit der Information
macht, dass ein Teilautomat sich in einem Akzeptierzustand befindet.

Zur Verdeutlichung stelle man sich eine Automaten-Objekt $A$ vor mit
folgendem Interface
\begin{verbatim}
interface Automat {
        // Zeichen verarbeiten
        void    process(char zeichen);
        // testen, ob sich der Automat in einem Akzeptierzustand befindet
        boolean accept();
        // in den Startzustand zurueckversetzen
        void    reset();
}
\end{verbatim}
Offensichtlich enthält dieses Interface keine Möglichkeit, den Automaten
zu modifizieren, oder auch nur den inneren Zustand auf eine andere Art
zu ändern als mit einem normalen Übergang.
Das Interface genügt aber, um die *-Operation auszuführen.
Um ein Wort in $L^*$ zu akzeptieren, muss man nur jedes Zeichen
des Wortes in den Automaten $A$ einspeisen und fragen, ob er sich
in einem Akzeptierzustand befindet.
Falls ja, darf man das Teilwort
akzeptieren und den Automaten in den Startzustand zurückversetzen.

Für die *-Operation darf also der Automat
\begin{center}
\includegraphics{images/nea-3}
\end{center}
nicht modifiziert werden, insbesondere darf der naheligende
$\varepsilon$-Übergang von einem Akzeptierzustand zum Startzustand
nicht hinzugefügt werden.
Um das leere Wort muss daher ein zusätzlicher Zustand hinzugefügt werden.
Und für die Wiederholung darf kann man diesen neuen Zustand ebenfalls
verwenden, indem man von den Akzeptierzuständen von $A$ zu diesem
neuen Zustand zurückkehrt.
So erhält man den Automaten
\begin{center}
\includegraphics{images/nea-4}
\end{center}

Formal kann man das Resultat wie folgt zusammenfassen:
\begin{satz}
\index{*-Operation@$*$-Operation}%
\label{satz_star}
Ist $L$ eine reguläre Sprache, dann ist auch $L^*$ regulär.
\end{satz}

Zum Beweis reicht es nicht, den Satz \ref{satz_concat} zu verwenden
um zu zeigen, dass $L^n$ regulär ist, und dann aus wiederholter
Anwendung Satz \ref{satz_union} zu schliessen zu versuchen,
dass die Vereinigung aller $L^n$ auch regulär sei.
Da dies eine unendliche
Vereinigung ist, würde nach der Konstruktion im Beweis von Satz
\ref{satz_union} ein Automat mit unendlich vielen
Zuständen entstehen, also keinen NEA.

\begin{proof}[Beweis]
Aus dem Automaten
\[
A=(Q,\Sigma, \delta,q_0,F)
\]
mit $L(A)=L$ konstruieren wir einen neuen
Automaten
\[
A'=(Q',\Sigma,\delta',q',F')
\]
mit einem neuen Anfangszustand $q'$, der auch
ein Akzeptierzustand ist.
Damit wird sichergestellt, dass der neue Automat das in $L^0$
enthaltene leere Wort akzeptiert.
Zusätzlich wird der neue Anfangszustand mit einem $\varepsilon$-Übergang
mit $q'$ verbunden.
Ebenso werden alle Akzeptierzustände über $\varepsilon$-Übergänge
mit dem neuen Startzustand verbunden:
\begin{align*}
Q'&=Q\cup \{q'\}\\
F'&=\{q'\}\\
\delta'(q',a)&=\emptyset\\
\delta'(q',\varepsilon)&= \{q_0\}\\
\delta'(q,a)&= \delta(q,a)\\
\delta'(q,\varepsilon)&=\begin{cases}
\delta(q,\varepsilon)          &\qquad q\not\in F\\
\delta(q,\varepsilon)\cup\{q'\}&\qquad q\in F
\end{cases}
\end{align*}
\end{proof}
\subsubsection{Variante: mindestens eine Wiederholung}
Die Sprache $LL^*$ besteht aus Verkettungen von Wörtern aus $L$,
im Unterschied zu $L^*$ kommt aber mindestens ein Wort vor.
Einen Automaten dafür kann man auf verschiedene Arten gewinnen,
sie laufen aber alle im Wesentlichen auf die Lösung
\begin{center}
\includegraphics{images/nea-7}
\end{center}
hinaus.
Wesentlich ist, dass man den Automaten durchlaufen muss, bevor
etwas akzeptiert werden kann, und dass man von den Akzeptierzuständen
auch wieder zum Startzustand zurückkehren kann.
Dabei darf allerdings
kein ``Kurzschluss'' entstehen, man muss eine Einbahnstrasse aus
$\varepsilon$-Übergängen verwenden.

\subsubsection{Automat zur Alternative $L_1\cup L_2$}
\index{Alternative}%
\index{Vereinigung}%
Für die Alternative $L_1\cup L_2$ hatten wir bei der Diskussion
des Produkt-Automaten schon einen Automaten gefunden, auch
hierfür lässt sich ein etwas einfacherer Automat im gleichen
Stil wie für Verkettung und *-Operation finden.
Ein neuer Startzustand
wird dazu mit $\varepsilon$-Übergängen mit den Startzuständen
der Automaten $A_1$ und $A_2$ verbunden, wie in Abbildung~\ref{neaalternative}.
\begin{figure}
\begin{center}
\includegraphics{images/nea-5}
\qquad
\qquad
\includegraphics{images/nea-6}
\end{center}
\caption{Automatenkonstruktion für die Alternative, rechts der Automat
für die Sprache $L(A_1)\cup L(A_2)$\label{neaalternative}}
\end{figure}

\subsubsection{Reguläre Operationen}
Damit haben wir alle drei regulären Operationen kennengelernt:

\begin{definition}
Die drei Operationen
\begin{enumerate}
\item Vereinigung $(L_1,L_2)\mapsto L_1\cup L_2$
\item Verkettung $(L_1,L_2)\mapsto L_1L_2$ und
\item Stern-Operation $L\mapsto L^*$
\end{enumerate}
heissen reguläre Operationen.
\end{definition}

Diese Operationen genügen, um alle regulären Sprachen aus
einelementigen Teilmengen von $\Sigma$ aufzubauen.
Dazu werden wir im nächsten Abschnitt eine prägnantere Notation,
die regulären Ausdrücke definieren.
Danach werden wir zeigen,
dass sich jeder DEA in einen regulären Ausdruck umwandeln lässt.

\subsection{Reguläre Ausdrücke\label{regulaer:regulaere-ausdruecke}}
\index{reguläre Ausdrücke}%
\index{Ausdrücke!reguläre}%
\begin{table}
\begin{center}
\begin{tabular}{|l|l|}
\hline
Ausdruck $r$&Bedeutung\\
\hline
{\tt a}&steht für das Zeichen ${\tt a}\in \Sigma$\\
{\tt .}&steht für ein beliebiges Zeichen aus $\Sigma$\\
{\tt [aeiou]}&steht für ein Zeichen aus $\{{\tt a},{\tt e},{\tt i},{\tt o},{\tt u}\}\subset \Sigma$\\
{\tt [1-9]}&steht für die positiven Ziffern\\
$\varepsilon$&steht für das leere Wort\\
$\emptyset$&steht für die leere Sprache\\
\hline
\end{tabular}
\end{center}
\index{reguläre Ausdrücke!primitive}%
\index{Ausdrücke!reguläre!primitive}%
\caption{Primitive reguläre Ausdrücke.\label{regtab1}}
\end{table}
Reguläre Ausdrücke sind Zeichenketten, die (reguläre) Sprachen
beschreiben.
Die einfachsten regulären Sprachen sind die Sprachen mit Wörtern,
die nicht länger als 1 Zeichen sind.
Reguläre Ausdrücke für
diese primitiven Sprachen werden in Tabelle~\ref{regtab1} zusammengestellt.

Reguläre Ausdrücke bauen reguläre Sprachen aus einzelnen
Zeichen und regulären Operationen aus.
Ist $r$ ein regulärer Ausdruck, dann schreiben wir  $L(r)$ für die
Sprache, die vom regulären Ausdruck $r$ beschrieben wird.
Für die regulären Operationen verwenden wir die Notation
gemäss Tabelle~\ref{regtab2}
\begin{table}
\begin{center}
\begin{tabular}{|l|c|l|}
\hline
&Ausdruck&reguläre Operation\\
\hline
\index{Verkettung}%
Verkettung&$r_1r_2$&$L(r_1)L(r_2)$\\
\index{Alternative}%
Alternative&$r_1{\tt |}r_2$&$L(r_1)\cup L(r_2)$\\
\index{*-Operation@$*$-Operation}%
Stern-Operation&$r{\mathstrut^{\tt *}}$&$L(r)^*$\\
\hline
\end{tabular}
\end{center}
\caption{Notation für reguläre Operationen\label{regtab2}}
\end{table}
Falls nötig können Klammern verwendet werden, um anzuzeigen,
welche Gruppen verknüpft werden.
Da mit regulären Operationen
verknüpfte Sprachen wieder regulär sind, können alle mit
regulären Ausdrücken beschriebenen Sprachen von einem DEA
akzeptiert werden.

\subsubsection{Automaten für die primitiven regulären Ausdrücke}
Die primitiven regulären Ausdrücke aus Tabelle~\ref{regtab1} können
von folgenden Automaten akzeptiert werden:
\begin{enumerate}
\item Ein einzelnes Zeichen {\tt a}:
\[
\entrymodifiers={++[o][F]}
\xymatrix{
*+\txt{}\ar[r]
	&{}\ar[r]^{\tt a}
		&*++[o][F=]{}
}
\]
\item Sprache $L=\Sigma$, regulärer Ausdruck $r={\tt .}$:
\[
\entrymodifiers={++[o][F]}
\xymatrix{
*+\txt{}\ar[r]
	&{}\ar[r]^{\Sigma}
		&*++[o][F=]{}
}
\]
\item Das leere Wort:
\[
\entrymodifiers={++[o][F]}
\xymatrix{
*+\txt{}\ar[r]
	&*++[o][F=]{}
}
\]
\item Die leere Sprache:
\[
\entrymodifiers={++[o][F]}
\xymatrix{
*+\txt{}\ar[r]
	&{}
}
\]
\end{enumerate}

\subsubsection{Automat eines regulären Ausdrucks}
Mit Hilfe der Automaten für die primitven regulären Ausdrücke
und den regulären Operationen kann man jetzt zu jedem regulären
Ausdruck einen NEA aufbauen.

\begin{beispiel}[\bf Beispiel 1] NEA des regulären Ausdrucks
${\tt ab}|{\tt cd}$.

Zunächst brauchen wir einen NEA für die Verkettung ${\tt ab}$.
Diesen können wir mit der Verkettungskonstruktion aus primitiven
NEAs für {\tt a} und {\tt b} bilden:
\[
\entrymodifiers={++[o][F]}
\xymatrix{
*+\txt{}\ar[r]
	&{}\ar[r]^{\tt a}
		&{}\ar[r]^{\varepsilon}
			&{}\ar[r]^{\tt b}
				&*++[o][F=]{}
}
\]
Auf die gleiche Art kann auch ein NEA für ${\tt cd}$ gebildet werden.
Diese müssen jetzt mit Hilfe der Konstruktion für die Alternative
zu einem NEA für den Gesamtausdruck zusammengesetzt werden:
\[
\entrymodifiers={++[o][F]}
\xymatrix{
*+\txt{}
	&*+\txt{}
		&{}\ar[r]^{\tt a}
			&{}\ar[r]^{\varepsilon}
				&{}\ar[r]^{\tt b}
					&*++[o][F=]{}
\\
*+\txt{} \ar[r]
	&{}\ar[ur]^{\varepsilon} \ar[dr]^{\varepsilon}
\\
*+\txt{}
	&*+\txt{}
		&{}\ar[r]^{\tt c}
			&{}\ar[r]^{\varepsilon}
				&{}\ar[r]^{\tt d}
					&*++[o][F=]{}
}
\]
\end{beispiel}

\begin{beispiel}[\bf Beispiel 2] NEA für den regulären Ausdruck
$({\tt ab})^*{\tt c}$.

Die einzelnen Teile sind uns schon bekannt, wir müssen nur noch
die *-Konstruktion auf den verketten Automaten anwenden:
\[
\entrymodifiers={++[o][F]}
\xymatrix{
*+\txt{}\ar[r]
	&{}\ar[r]^{\varepsilon}
		&{}\ar[r]^{\varepsilon}
			\ar[d]^{\varepsilon}
			&\ar[r]_{\tt a}
				&{}\ar[r]^{\varepsilon}
					&{}\ar[r]_{\tt b}
						&{}\ar@/_20pt/[llll]_{\varepsilon}
\\
*+\txt{}
	&*+\txt{}
		&{}\ar[r]^{\tt c}
			&*++[o][F=]{}
}
\]
\end{beispiel}

\begin{beispiel}[\bf Beispiel 3:] NEA für den regulären Ausdruck
$({\tt ab})+$.

Das Zeichen ${\tt+}$ bedeutet bei regulären Ausdrücken ``mindestens ein''.
Die einfachste Modifikation der *-Konstruktion, die dies leistet, besteht
darin, den Akzeptierzustand ans Ende des Automaten zu verschieben:
\[
\entrymodifiers={++[o][F]}
\xymatrix{
*+\txt{}\ar[r]
	&{}\ar[r]^{\varepsilon}
		&{}\ar[r]^{\tt a}
			&{}\ar[r]^{\varepsilon}
				&{}\ar[r]^{\tt b}
					&*++[o][F=]{}\ar@/^20pt/[llll]^{\varepsilon}
}
\]
Alternativ könnte man auch einfach den NEA für die Verkettung
nehmen und keinen neuen Startzustand hinzufügen.
Bei der
*-Konstruktion wurde das ja gemacht, um auch das leere Wort
akzeptieren zu können, was bei der {\tt +}-Operation nicht
nötig ist:
\[
\entrymodifiers={++[o][F]}
\xymatrix{
*+\txt{}\ar[r]
	&{}\ar[r]^{\tt a}
		&{}\ar[r]^{\varepsilon}
			&{}\ar[r]^{\tt b}
				&*++[o][F=]{}\ar@/^20pt/[lll]^{\varepsilon}
}
\]
\end{beispiel}
Diese Beispiele illustrieren, dass sich aus jedem regulären Ausdruck
ein NEA bauen lässt.
Die Konstruktion hat die Form eines Algorithmus,
lässt sich also auch mit einem Computer implementieren, sie geht
auf Ken Thompson zurück.
\index{Thompson, Ken}%

\subsection{Regulärer Ausdruck eines DEA\label{regulaer:dea-re}}
Zu jedem DEA gibt es einen regulären Ausdruck, der angibt,
welche Wörter er akzeptiert.
In diesem Abschnitt geben
wir einen Algorithmus an, der einen DEA in einen äquivalenten
regulären Ausdruck umwandelt.

\index{NEA!verallgemeinerter}%
\index{VNEA}%
Wir verwenden dazu den Begriff eines verallgemeinerten NEA (VNEA),
dessen Pfeile nicht mehr nur mit Zeichen des Alphabets oder mit
$\varepsilon$ angeschrieben sei können, sondern mit regulären
Ausdrücken.
Dann wandelt man einen DEA zunächst um in einen
VNEA und reduziert ihn dann auf einen VNEA mit nur zwei Zuständen,
einem Startzustand und einem Akzeptierzustand:
\[
\entrymodifiers={++[o][F]}
\xymatrix{
*+\txt{}\ar[r]
	&{q_0}\ar[r]^{r}
		&*++[o][F=]{q_1}
}
\]
Offenbar werden genau diejenigen Wörter von diesem Automaten
akzeptiert, welche auf den regulären Ausdruck $r$ passen.

Sei jetzt also ein NEA $A$ gegeben.
Da er möglicherweise viele
Endzustände haben kann, wir aber am Schluss nur noch einen
Endzustand haben wollen, fügen wir einen neuen Endzustand
$q_{\text{accept}}$ hinzu.
Da der Automat auch Pfeile haben kann, die im Startzustand enden,
der endgültige VNEA aber keine solche Pfeile hat, fügen wir
auch einen zusätzlichen Startzustand $q_{\text{start}}$ hinzu.

%Natürlich müssen wir jetzt auch noch Pfeile ergänzen. Zusätzlich
%zu den ursprünglichen Pfeilen des NEA ergänzen wir darin
%alle möglichen Pfeile, die es bis jetzt noch nicht gab, und
%beschriften Sie mit dem regulären Ausdruck $\emptyset$, was bedeutet,
%dass diese Pfeile gar nie ``genommen'' werden.

Ausserdem fügen wir vom neuen Startzustand $q_{\text{start}}$
aus einen Pfeil zum ursprünglichen Startzustand des NEAs mit Beschriftung $\varepsilon$ hinzu.

Alle Akzeptierzustände des ursprünglichen NEAs werden jetzt noch mit
$q_{\text{accept}}$ verbunden, beschriftet mit $\varepsilon$.

Damit haben wir jetzt einen VNEA.
Diesen müssen wir jetzt auf einen VNEA reduzieren, der nur noch
die beiden Zustände $q_{\text{start}}$ und $q_{\text{accept}}$
enthält.
Dazu reissen wir nacheinander alle Zustände des ursprünglichen
NEA heraus.
Die entstehenden Löcher müssen wir natürlich reparieren, indem wir die
verbleibenden Pfeile mit erweiterten regulären Ausdrücken
anschreiben, welche die verlorengegangenen Teile ersetzen.

Nehmen wir also an, wir möchten den Zustand $q_ {\text{rip}}$
herausreissen.
Seien weiter $q_i$ und $q_j$ zwei Zustände, die
Übergänge haben, die über $q_{\text{rip}}$ führen:
\[
\entrymodifiers={++[o][F]}
\xymatrix{
{q_i}\ar[rr]^{r_4} \ar[dr]_{r_1}
	&*+\txt{}
		&{q_j}
\\
*+\txt{}
	&{q_{\text{rip}}}\ar[ur]_{r_3} \ar@(dl,dr)_{r_2}
}
\]
Der Weg über $q_{\text{rip}}$ entspricht einem Übergang
von $q_i$ nach $q_j$ mit regulärem Ausdruck $r_1r_2^*r_3$,
der zusätzlich zum Ausdruck $r_4$, möglich ist.
Wenn $q_{\text{rip}}$ entfernt wird, muss also $r_4$ durch 
\[
r_4{\tt |}r_1r_2^{\tt *}r_3
\]
ersetzt werden:
\[
\entrymodifiers={++[o][F]}
\xymatrix{
{q_i}\ar[rr]^{r_4|r_1r_2^*r_3}
	&*+\txt{}
		&{q_j}
}
\]
Die vom VNEA akzeptierte Sprache ändert sich dabei nicht.
Diese Operation wird wiederholt, bis die Zustände des ursprünglichen
NEA vollständig entfernt sind.

Damit  haben wir jetzt folgenden Satz bewiesen.
\begin{satz}
Ist $A$ ein NEA, dann gibt es einen regulären Ausdruck $r$ mit
$L(A)=L(r)$.
\end{satz}

\subsubsection{Beispiel}
Wir wollen den DEA
\[
\entrymodifiers={++[o][F]}
\xymatrix{
*+\txt{}\ar[r]
	&{1}\ar@/^/[r]^{\tt a} \ar@/^/[d]^{\tt b}
		&*++[o][F=]{2} \ar@(ul,ur)^{\tt b} \ar@/^/[l]^{\tt a}
\\
*+\txt{}
	&*++[o][F=]{3} \ar[ur]_{\tt a} \ar@/^/[u]^{\tt b}
}
\]
in einen äquivalenten regulären Ausdruck umwandeln.

Im ersten Schritt fügen wir die neuen Start- und Akzeptierzustände 
hinzu, die Zustände $2$ und $3$ sind damit nicht mehr Akzeptierzustände:
\[
\entrymodifiers={++[o][F]}
\xymatrix{
*+\txt{}\ar[r]
	&S\ar[r]^{\varepsilon}
		&{1}\ar@/^/[r]^{\tt a} \ar@/^/[d]^{\tt b}
			&{2} \ar@(ul,ur)^{\tt b} \ar@/^/[l]^{\tt a}\ar[r]^{\varepsilon}
				&*++[o][F=]{A}
\\
*+\txt{}
	&*+\txt{}
		&{3} \ar[ur]_{\tt a} \ar@/^/[u]^{\tt b} \ar[urr]^{\varepsilon}
}
\]
Jetzt entfernen wir nacheinander alle ``inneren'' Zustände.
Wir beginnen mit dem Zustand $1$, dabei sind die Pfade
$S$-$1$-$2$,
$S$-$1$-$3$,
$2$-$1$-$3$,
$3$-$1$-$2$
und
$2$-$1$-$2$
zu berücksichtigen
\[
\entrymodifiers={++[o][F]}
\xymatrix{
*+\txt{}\ar[r]
	&S\ar[rr]^{\tt a} \ar[dr]_{\tt b}
		&*+\txt{}
			&{2} \ar@(ul,ur)^{\tt aa|b} \ar@/_/[dl]_{\tt ab} \ar[r]^{\varepsilon}
				&*++[o][F=]{A}
\\
*+\txt{}
	&*+\txt{}
		&{3} \ar@/_/[ur]_{\tt a|ba} \ar@/_20pt/[urr]_{\varepsilon} \ar@(dl,dr)_{\tt bb}
}
\]
Jetzt wird der Zustand $2$ entfernt, dabei sind die Pfade 
$S$-$2$-$A$,
$S$-$2$-$3$,
$3$-$2$-$A$,
und
$3$-$2$-$3$
zu berücksichtigen:
\[
\entrymodifiers={++[o][F]}
\xymatrix{
*+\txt{}\ar[r]
	&S\ar[rrr]^{\tt a(aa|b)^*} \ar[dr]_{\tt b|a(aa|b)^*ab}
		&*+\txt{}
			&*+\txt{}
				&*++[o][F=]{A}
\\
*+\txt{}
	&*+\txt{}
		&{3} \ar[urr]_{{\tt (a|ba)(aa|b)^*}|\varepsilon} \ar@(dl,dr)_{{\tt bb|(a|ba)(aa|b)^*ab}}
}
\]
Jetzt muss nur noch der Zustand $3$ entfernt werden:
\[
\entrymodifiers={++[o][F]}
\xymatrix{
*+\txt{}\ar[r]
	&S\ar[rrrrrrrr]^{\tt (a(aa|b)^*)|(b|a(aa|b)^*ab)(bb|(a|ba)(aa|b)^*ab)^*(((a|ba)(aa|b)^*)|\varepsilon)}
		&*+\txt{}
			&*+\txt{}
			&*+\txt{}
			&*+\txt{}
			&*+\txt{}
			&*+\txt{}
			&*+\txt{}
				&*++[o][F=]{A}
}
\]
Damit ist der reguläre Ausdruck gefunden, der die gleiche Sprache
akzeptiert wie der ursprüngliche DEA.

