\section{Anhang: Reguläre Ausdrücke in der Praxis\label{regulaer:praxis}}
\rhead{Regex in der Praxis}
\subsection{Flex}
\index{Flex}%
\index{Scanner}%
\index{Scannergenerator}%
Flex ist ein Scannergenerator.
Er verarbeitet eine Spezifikation einer
Sprache als Menge von regulären Ausdrücken in einen deterministischen 
endlichen Automaten.
Als Beispiel soll verfolgt werden, wie die
Flex-Spezifikation
\verbatiminput{3-regulaer/lex/example4.l}
in einen endlichen Automaten umgesetzt wird.
Mit der Option {\tt -T}
kann man sich die erzeugten Tabellen des NEA anzeigen lassen:
\verbatiminput{3-regulaer/lex/dumpnfa}
Die Tabelle ist wie folgt zu lesen.
In der mittleren Spalte steht das zu verarbeitende Zeichen, also
die Beschriftung eines Pfeils.
Für jedes Zeichen sind zwei Zielzustände
möglich, die in den letzten beiden Spalten eingetragen werden.
Einen
Zustand {\tt 0} gibt es nicht, Übergänge nach ${\tt 0}$ müssen als
nicht verwendete Pfeile interpretiert werden.
$\varepsilon$-Übergänge
werden durch mit dem Symbol {\tt 257} bezeichnet.
Sind von einem Punkt
aus mehr also zwei Alternativen möglich, wie im vorliegenden Fall
bei der Alternative {\tt h|s|r}, dann muss dies mit Hilfe von zusätzlichen
$\varepsilon$-Übergangen aus Zweier-Alternativen zusammengesetzt werden.
Zeichnet man diesen NEA, ergibt sich das folgende Zustandsdiagramm:
\[
\entrymodifiers={++[o][F]}
\xymatrix{
*+\txt{}\ar[r]
	&{15} \ar[r]^{\varepsilon} \ar[d]^{\varepsilon}
		&{13}\ar[r]^{{\tt EOF}}
			&*++[o][F=]{14}
\\
*+\txt{}
	&{12}\ar[d]^{\varepsilon} \ar[r]^{\varepsilon}
		&{1}
\\
*+\txt{}
	&{8}\ar[d]^{\varepsilon} \ar[r]^{\varepsilon}
		&{7}\ar[dr]^{\tt r}
\\
*+\txt{}
	&{5}\ar[dr]^{\varepsilon} \ar[r]^{\varepsilon}
		&{3}\ar[r]^{\tt h}
			&{6}\ar[r]^{\varepsilon}
				&{9}\ar[r]^{\tt 1}
					&{10}\ar[r]^{\tt 0}
						&*++[o][F=]{11}
\\
*+\txt{}
	&*+\txt{}
		&{4}\ar[ur]^{\tt s}
\\
*+\txt{}
	&{2}
}
\]
Man kan im unteren Teil sehr schön erkennen, wie der Automat, der
die Dreier-Alternative {\tt h|s|r} akzeptiert, über einen
$\varepsilon$-Übergang mit dem Automaten verkettet wird, der 
den String {\tt 10} akzeptiert.
Zustand {\tt 14} wurde von Flex
hinzugefügt, er erlaubt das Fileende zu erkennen, so dass das
erzeugte Programm am Fileende auf jeden Fall terminieren kann.

Der NEA kann natürlich auch in einen DEA umgewandelt werden.
Auch dies protokolliert Flex: 
\verbatiminput{3-regulaer/lex/dumpdfa}
Zu jedem Zustand wird jetzt aufgeführt, welche Zeichen zu welchen
Übergängen führen.
Flex hat die Zeichen in Klassen zusammengefasst,
die Ziffern bedeuten:
\begin{center}
\begin{tabular}{|c|c|}
\hline
Zeichenklasse&Inhalt\\
\hline
{\tt 1}&andere\\
{\tt 2}&{\tt 0}\\
{\tt 3}&{\tt 1}\\
{\tt 4}&{\tt h}\\
{\tt 5}&{\tt s}\\
{\tt 6}&{\tt r}\\
\hline
\end{tabular}
\end{center}
Auch den DEA kann man in ein Zustandsdiagramm zurückübersetzen:
\[
\entrymodifiers={++[o][F]}
\xymatrix{
*+\txt{}
	&*+\txt{}\ar[d]
\\
*+\txt{}
	&{1}\ar[dr]^{{\tt h},{\tt s},{\tt r}} \ar[dl]_{{\tt 0}, {\tt 1}, \text{andere}}
\\
*++[o][F=]{4}
	&*+\txt{}
		&*++[o][F=]{5}\ar[r]^{\tt 1}
			&{6}\ar[r]^{\tt 0}
				&*++[o][F=]{7}
\\
*+\txt{}
	&{2}\ar[ur]_{{\tt h},{\tt s},{\tt r}} \ar[ul]^{{\tt 0}, {\tt 1}, \text{andere}}
\\
*+\txt{}
	&*++[o][F=]{3}
}
\]
Man kann sehen, dass Flex nicht alles wirklich aufräumt, es bleiben
unerreichbare Zustände {\tt 2} und {\tt 3} stehen, einer davon sogar ein
Akzeptierzustand.
Die Zustände 4 und 5 sind als
Akzeptierzustände markiert, aber nur 7 ist als Akzeptierzustand zu
verstehen, der ein auf den regulären Ausdruck passendes Wort
akzeptiert.

\subsection{Performance}
\index{Laufzeit!eines DEA}%
\index{Laufzeit!eines NEA}%
Ein DEA hat immer Laufzeit $O(n)$, d.\,h.~ein Regex-Matcher auf der
Basis eines DEA wird innerhalb einer Zeit proportional zur
Länge des Inputstrings erkennen, ob ein Inputstring passt.
Die einfachste Implementation eines NEA hingegen wird alle Möglichkeiten
von nicht eindeutigen Übergängen durchprobieren müssen, mit
möglicherweise exponentieller Laufzeit.
Normalerweise sind in der
Praxis eingesetzte reguläre Ausdrücke klein und die untersuchten
Strings sind ebenfalls nicht allzu gross.
Man kann allerdings auch
Ausdrücke konstruieren, die den Unterschied zwischen DEA und simplistischer
NEA-Implementation offensichtlich werden lassen.
Ein solcher Ausdruck ist
\begin{center}
\tt a?a?a?a?a?aaaaa
\end{center}
Also $n$ fakultative {\tt a} gefolgt von $n$ obligatorischen {\tt a}.
Um ein Wort bestehend aus $n$ {\tt a} zu akzeptieren, gibt es genau eine
Möglichkeit, die man erhält, wenn man in der Alternativen
${\tt a?} = \varepsilon|{\tt a}$
jeweils das leere Wort wählt.
Falls das immer die zweite ausprobierte 
Möglichkeit ist, ist die Laufzeit dieses Algorithmus proportional zu
$2^n$, wird also bei langen Strings schnell sehr gross.

Es ist einfach, einen geeigneten DEA für den regulären
Ausdruck {\tt a?a?a?aaa} hinzuschreiben:
\[
\entrymodifiers={++[o][F]}
\xymatrix{
*+\txt{}\ar[r]
	&\ar[r]^{\tt a}
		&\ar[r]^{\tt a}
			&\ar[r]^{\tt a}
				&*++[o][F=]{}\ar[r]^{\tt a}
					&*++[o][F=]{}\ar[r]^{\tt a}
						&*++[o][F=]{}\ar[r]^{\tt a}
							&*++[o][F=]{}
}
\]
Und auch einen NEA findet man aus der Konstruktion eines Automaten
aus einem regulären Ausdruck ohne Probleme:
\[
\entrymodifiers={++[o][F]}
\xymatrix{
*+\txt{}\ar[r]
	&\ar@/^/[r]^{\tt a} \ar@/_/[r]_{\varepsilon}
		&\ar@/^/[r]^{\tt a} \ar@/_/[r]_{\varepsilon}
			&\ar@/^/[r]^{\tt a} \ar@/_/[r]_{\varepsilon}
				&\ar[r]^{\tt a}
					&\ar[r]^{\tt a}
						&\ar[r]^{\tt a}
							&*++[o][F=]{}
}
\]
Auch die Vereinfachung der $\varepsilon$-Übergänge ist noch sehr
übersichtlich:
\[
\entrymodifiers={++[o][F]}
\xymatrix{
*+\txt{}\ar[r]
	&\ar[r]^{\tt a}
	\ar@/^20pt/[rr]^{\tt a}
	\ar@/^20pt/[rrr]^{\tt a}
	\ar@/^20pt/[rrrr]^{\tt a}
		&\ar[r]^{\tt a}
		\ar@/_20pt/[rr]_{\tt a}
		\ar@/_20pt/[rrr]_{\tt a}
			&\ar[r]^{\tt a} \ar@/_20pt/[rr]_{\tt a}
				&\ar[r]^{\tt a}
					&\ar[r]^{\tt a}
						&\ar[r]^{\tt a}
							&*++[o][F=]{}
}
\]
Auch in diesem Beispiel kann man sich von Flex helfen lassen,
einen NEA zu erstellen.
Unverkennbar sind auch hier wieder die
Spuren der Konstruktionsschritte, mit denen man aus regulären
Ausdrücken Teilautomaten baut und diese dann zum ganzen Automaten
zusammenbaut.
\[
\entrymodifiers={++[o][F]}
\xymatrix{
*+\txt{}\ar[r]
	&{19}\ar[d]^{\varepsilon}\ar[r]^{\varepsilon}
		&{17}\ar[r]^{\tt EOF}
			&*++[o][F=]{18}
\\
*+\txt{}
	&{16}\ar[d]^{\varepsilon} \ar[r]^{\varepsilon}
		&{1}
\\
*+\txt{}
	&{5} \ar[r]^{\varepsilon}
	     \ar[dr]^{\varepsilon}
		&{3} \ar[d]^{\tt a}
			&{8} \ar[r]^{\varepsilon}
			     \ar[dr]^{\varepsilon}
				&{6} \ar[d]^{\tt a}
					&{11} \ar[r]^{\varepsilon}
					     \ar[dr]^{\varepsilon}
						&{9} \ar[d]^{\tt a}
\\
*+\txt{}
	&*+\txt{}
		&{4} \ar[ur]^{\varepsilon}
			&*+\txt{}
				&{7} \ar[ur]^{\varepsilon}
					&*+\txt{}
						&{10} \ar@/_10pt/[dlll]_{\varepsilon}
\\
*+\txt{}
	&*+\txt{}
		&*+\txt{}
			&{12} \ar[r]^{\tt a}
				&{13} \ar[r]^{\tt a}
					&{14} \ar[r]^{\tt a}
						&*++[o][F=]{15}
}
\]
Daraus macht Flex dann den folgenden DEA:
\[
\entrymodifiers={++[o][F]}
\xymatrix{
*+\txt{}\ar[d]
\\
{1}\ar[r]^{\tt a} \ar[d]^{\text{andere}} \
	&{5}\ar[r]^{\tt a}
		&{6}\ar[r]^{\tt a}
			&*++[o][F=]{7} \ar[r]^{\tt a}
				&*++[o][F=]{8} \ar[r]^{\tt a}
					&*++[o][F=]{9} \ar[r]^{\tt a}
						&*++[o][F=]{10}
\\
{4}
\\
{3}
}
\]
$3$ und $4$ sind wieder Zustände die benötigt werden, um andere Inputs
wie zum Beispiel EOF zu erkennen.
Der resultierend DEA entspricht genau
dem oben vorgeschlagenen DEA.

\begin{sloppypar} % To prevent "java.lang.string" from overflowing
Nicht alle Implementationen verwenden jedoch diese Theorie.
Die Regex-Funktionen in der C-Library verwenden dies und sind entsprechend
schnell.
Alle Unix-Tools, die auf dieser Bibliothek basieren, sind
entsprechend schnell.
Perl hat seine eigene erweiterte Regex-Library,
verwendet keinen DEA und ist langsam.
Ebenfalls langsam ist die
Implementation der {\tt matches} Methode in {\tt java.lang.String}.
Man würde hoffen, dass wenigstens das Package {\tt java.util.regex} einen
besseren Regex-Matcher beinhalten würde.
Leider wurde da nur am Interface herumgebastelt, nicht an der Substanz.
Vielleicht sollten
die Implementatoren von {\tt java.util.regex} mal die Vorlesung AutoSpr
besuchen.
\end{sloppypar}
